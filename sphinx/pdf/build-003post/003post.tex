%% Generated by Sphinx.
\def\sphinxdocclass{report}
\documentclass[letterpaper,12pt,english]{sphinxmanual}
\ifdefined\pdfpxdimen
   \let\sphinxpxdimen\pdfpxdimen\else\newdimen\sphinxpxdimen
\fi \sphinxpxdimen=.75bp\relax
%% turn off hyperref patch of \index as sphinx.xdy xindy module takes care of
%% suitable \hyperpage mark-up, working around hyperref-xindy incompatibility
\PassOptionsToPackage{hyperindex=false}{hyperref}

\PassOptionsToPackage{warn}{textcomp}

\catcode`^^^^00a0\active\protected\def^^^^00a0{\leavevmode\nobreak\ }
\usepackage{cmap}
\usepackage{xeCJK}
\usepackage{amsmath,amssymb,amstext}
\usepackage{babel}



\setCJKmainfont{Adobe Song Std}
%中文字体fontsize放大,kl+
%\defaultCJKfontfeatures{Scale=1.5}


\usepackage[Sonny]{fncychap}
\ChNameVar{\Large\normalfont\sffamily}
\ChTitleVar{\Large\normalfont\sffamily}
\usepackage{sphinx}

\fvset{fontsize=\small}
\usepackage{geometry}


% Include hyperref last.
\usepackage{hyperref}
% Fix anchor placement for figures with captions.
\usepackage{hypcap}% it must be loaded after hyperref.
% Set up styles of URL: it should be placed after hyperref.
\urlstyle{same}

\addto\captionsenglish{\renewcommand{\contentsname}{目录}}

\usepackage{sphinxmessages}
\setcounter{tocdepth}{0}


\usepackage{enumitem}
\setlistdepth{99}


\title{003post}
\date{2020 年 07 月 12 日}
\release{}
\author{kevinluo}
\newcommand{\sphinxlogo}{\vbox{}}
\renewcommand{\releasename}{}
\makeindex
\begin{document}

\ifdefined\shorthandoff
  \ifnum\catcode`\=\string=\active\shorthandoff{=}\fi
  \ifnum\catcode`\"=\active\shorthandoff{"}\fi
\fi

\pagestyle{empty}
\sphinxmaketitle
\pagestyle{plain}
\sphinxtableofcontents
\pagestyle{normal}
\phantomsection\label{\detokenize{index::doc}}



\chapter{1   Hi,p00其它}
\label{\detokenize{p00_u5176_u5b83/Hello_uff0cp00_u5176_u5b83:hi-p00}}\label{\detokenize{p00_u5176_u5b83/Hello_uff0cp00_u5176_u5b83::doc}}
\begin{sphinxShadowBox}
\sphinxstyletopictitle{目录}
\begin{itemize}
\item {} 
\phantomsection\label{\detokenize{p00_u5176_u5b83/Hello_uff0cp00_u5176_u5b83:id2}}{\hyperref[\detokenize{p00_u5176_u5b83/Hello_uff0cp00_u5176_u5b83:hi-p00}]{\sphinxcrossref{1   Hi,p00其它}}}
\begin{itemize}
\item {} 
\phantomsection\label{\detokenize{p00_u5176_u5b83/Hello_uff0cp00_u5176_u5b83:id3}}{\hyperref[\detokenize{p00_u5176_u5b83/Hello_uff0cp00_u5176_u5b83:post}]{\sphinxcrossref{1.1   post}}}

\end{itemize}

\end{itemize}
\end{sphinxShadowBox}


\section{1.1   post}
\label{\detokenize{p00_u5176_u5b83/Hello_uff0cp00_u5176_u5b83:post}}

\chapter{1   Hi,p01散文}
\label{\detokenize{p01_u6563_u6587/Hello_uff0cp01_u6563_u6587:hi-p01}}\label{\detokenize{p01_u6563_u6587/Hello_uff0cp01_u6563_u6587::doc}}
\begin{sphinxShadowBox}
\sphinxstyletopictitle{目录}
\begin{itemize}
\item {} 
\phantomsection\label{\detokenize{p01_u6563_u6587/Hello_uff0cp01_u6563_u6587:id2}}{\hyperref[\detokenize{p01_u6563_u6587/Hello_uff0cp01_u6563_u6587:hi-p01}]{\sphinxcrossref{1   Hi,p01散文}}}
\begin{itemize}
\item {} 
\phantomsection\label{\detokenize{p01_u6563_u6587/Hello_uff0cp01_u6563_u6587:id3}}{\hyperref[\detokenize{p01_u6563_u6587/Hello_uff0cp01_u6563_u6587:post}]{\sphinxcrossref{1.1   post}}}

\end{itemize}

\end{itemize}
\end{sphinxShadowBox}


\section{1.1   post}
\label{\detokenize{p01_u6563_u6587/Hello_uff0cp01_u6563_u6587:post}}

\chapter{1   峻青\sphinxhyphen{}海滨仲夏夜}
\label{\detokenize{p01_u6563_u6587/_u5cfb_u9752-_u6d77_u6ee8_u4ef2_u590f_u591c:id1}}\label{\detokenize{p01_u6563_u6587/_u5cfb_u9752-_u6d77_u6ee8_u4ef2_u590f_u591c::doc}}
\begin{sphinxShadowBox}
\sphinxstyletopictitle{目录}
\begin{itemize}
\item {} 
\phantomsection\label{\detokenize{p01_u6563_u6587/_u5cfb_u9752-_u6d77_u6ee8_u4ef2_u590f_u591c:id5}}{\hyperref[\detokenize{p01_u6563_u6587/_u5cfb_u9752-_u6d77_u6ee8_u4ef2_u590f_u591c:id1}]{\sphinxcrossref{1   峻青\sphinxhyphen{}海滨仲夏夜}}}
\begin{itemize}
\item {} 
\phantomsection\label{\detokenize{p01_u6563_u6587/_u5cfb_u9752-_u6d77_u6ee8_u4ef2_u590f_u591c:id6}}{\hyperref[\detokenize{p01_u6563_u6587/_u5cfb_u9752-_u6d77_u6ee8_u4ef2_u590f_u591c:id3}]{\sphinxcrossref{1.1   作品原文}}}

\item {} 
\phantomsection\label{\detokenize{p01_u6563_u6587/_u5cfb_u9752-_u6d77_u6ee8_u4ef2_u590f_u591c:id7}}{\hyperref[\detokenize{p01_u6563_u6587/_u5cfb_u9752-_u6d77_u6ee8_u4ef2_u590f_u591c:id4}]{\sphinxcrossref{1.2   注释}}}

\end{itemize}

\end{itemize}
\end{sphinxShadowBox}


\section{1.1   作品原文}
\label{\detokenize{p01_u6563_u6587/_u5cfb_u9752-_u6d77_u6ee8_u4ef2_u590f_u591c:id3}}
夕阳落山不久,西方的天空,还燃烧着一片橘红色的晚霞。大海,也被这霞光染成了红色,而且比天空的景色更要壮观。因为它是活动的,每当一排排波浪涌起的时候,那映照在浪峰上的霞光,又红又亮,就像一片片霍霍①燃烧的火焰,闪烁着,消失了。而后面的一排,又闪烁着,滚动着,涌了过来。

天空的霞光渐渐地淡下去了,深红的颜色变成了绯红②,绯红又变为浅红。最后,当这一切红光都消失了的时候,那突然显得高而远了的天空,呈现出一片肃穆,最早出现的启明星③,在这深蓝色的天幕上闪烁起来了。它是那么大,那么亮,整个广漠④的天幕上只有它在那里放射着令人注目的光辉,活像一盏悬挂在高空的明灯。

夜色加浓,苍空中的“明灯”越来越多了。而城市各处的真的灯火也次第⑤亮了起来,尤其是围绕在海港周围山坡上的那一片灯光,从半空倒映在乌蓝的海面上,随着波浪,晃动着,闪烁着,像一串流动着的珍珠,和那一片片密布在苍穹⑥里的星斗互相辉映,煞⑦是好看。

在这幽美的夜色中,我踏着软绵绵的沙滩,沿着海边,慢慢地向前走去。海水,轻轻地抚摸着细软的沙滩,发出温柔的刷刷声。晚来的海风,清新而又凉爽。我的心里,有着说不出的兴奋和愉快。

夜风轻飘飘地吹拂着,空气中飘荡着一种大海和田禾相混合的香味,柔软的沙滩上还残留着白天太阳炙晒的余温。那些在各个工作岗位上劳动了一天的人们,三三两两地来到了这软绵绵的沙滩上,他们浴着凉爽的海风,望着那缀满了星星的夜空,尽情地说笑,尽情地休憩。愉快的笑声,不时地从这儿那儿飞扬开来,像平静的海面上不断地从这儿那儿涌起的波浪。

我漫步沙滩,徘徊在我的乡亲朋友们中间。

我看到,在那边,在一只底儿朝上反扣在沙滩上的木船旁边,是一群刚从田里收割麦子归来的人们,他们在谈论着今年的收成。今春,雨水足,麦苗长得旺,收成比去年好。眼下,又下了一场透雨,秋后的丰收局面,也大体可以确定下来了。人们为这大好年景所鼓舞着,谈话中也充满了愉快欢乐的笑声。

月亮上来了。

是一轮灿烂的满月。它像一面光辉四射的银盘似的,从那平静的大海里涌了出来。大海里,闪烁着一片鱼鳞似的银波。沙滩上,也突然明亮了起来,一片片坐着、卧着、走着的人影,看得清清楚楚了。啊!海滩上,居然有这么多的人在乘凉。说话声、欢笑声、唱歌声、嬉闹声,响遍了整个的海滩。

月亮升得很高了。它是那么皎洁⑧,那么明亮。

夜已经深了。

沙滩上的人,有的躺在那软绵绵的沙滩上睡着了,有的还在谈笑。凉爽的风轻轻地吹拂着,皎洁的月光照耀着。让这些英雄的人们,在这自由的天幕下,干净的沙滩上,海阔天空地尽情谈笑吧,酣畅地休憩吧。{[}1{]}


\section{1.2   注释}
\label{\detokenize{p01_u6563_u6587/_u5cfb_u9752-_u6d77_u6ee8_u4ef2_u590f_u591c:id4}}
①霍霍:这里是闪动的样子。

②绯红:鲜红。绯,红色。

③启明星:早晨出现于天空东方的金星。

④广漠:广大空旷。

⑤次第:一个挨一个。

⑥苍穹:天空。

⑦煞(shà):这里是“很”的意思。

⑧皎洁:(月亮)明亮洁白。


\chapter{1   朱自清\sphinxhyphen{}春}
\label{\detokenize{p01_u6563_u6587/_u6731_u81ea_u6e05-_u6625:id1}}\label{\detokenize{p01_u6563_u6587/_u6731_u81ea_u6e05-_u6625::doc}}
\begin{sphinxShadowBox}
\sphinxstyletopictitle{目录}
\begin{itemize}
\item {} 
\phantomsection\label{\detokenize{p01_u6563_u6587/_u6731_u81ea_u6e05-_u6625:id4}}{\hyperref[\detokenize{p01_u6563_u6587/_u6731_u81ea_u6e05-_u6625:id1}]{\sphinxcrossref{1   朱自清\sphinxhyphen{}春}}}
\begin{itemize}
\item {} 
\phantomsection\label{\detokenize{p01_u6563_u6587/_u6731_u81ea_u6e05-_u6625:id5}}{\hyperref[\detokenize{p01_u6563_u6587/_u6731_u81ea_u6e05-_u6625:id3}]{\sphinxcrossref{1.1   作品原文}}}

\end{itemize}

\end{itemize}
\end{sphinxShadowBox}


\section{1.1   作品原文}
\label{\detokenize{p01_u6563_u6587/_u6731_u81ea_u6e05-_u6625:id3}}
盼望着,盼望着,东风来了,春天的脚步近了。

一切都像刚睡醒的样子,欣欣然张开了眼。山朗润起来了,水涨起来了,太阳的脸红起来了。

小草偷偷地从土里钻出来,嫩嫩的,绿绿的。园子里,田野里,瞧去,一大片一大片满是的。坐着,躺着,打两个滚,踢几脚球,赛几趟跑,捉几回迷藏。风轻悄悄的,草软绵绵的。

桃树、杏树、梨树,你不让我,我不让你,都开满了花赶趟儿。红的像火,粉的像霞,白的像雪。花里带着甜味儿;闭了眼,树上仿佛已经满是桃儿、杏儿、梨儿。花下成千成百的蜜蜂嗡嗡地闹着,大小的蝴蝶飞来飞去。野花遍地是:杂样儿,有名字的,没名字的,散在草丛里,像眼睛,像星星,还眨呀眨的。

“吹面不寒杨柳风”,不错的,像母亲的手抚摸着你。风里带来些新翻的泥土的气息,混着青草味儿,还有各种花的香,都在微微润湿的空气里酝酿。鸟儿将窠巢安在繁花嫩叶当中,高兴起来了,呼朋引伴地卖弄清脆的喉咙,唱出宛转的曲子,与轻风流水应和着。牛背上牧童的短笛,这时候也成天嘹亮地响着。

雨是最寻常的,一下就是三两天。可别恼。看,像牛毛,像花针,像细丝,密密地斜织着,人家屋顶上全笼着一层薄烟。树叶儿却绿得发亮,小草儿也青得逼你的眼。傍晚时候,上灯了,一点点黄晕的光,烘托出一片安静而和平的夜。在乡下,小路上,石桥边,有撑起伞慢慢走着的人,地里还有工作的农民,披着蓑戴着笠。他们的房屋,稀稀疏疏的在雨里静默着。

天上风筝渐渐多了,地上孩子也多了。城里乡下,家家户户,老老小小,也赶趟儿似的,一个个都出来了。舒活舒活筋骨,抖擞抖擞精神,各做各的一份事去。“一年之计在于春”,刚起头儿,有的是工夫,有的是希望。

春天像刚落地的娃娃,从头到脚都是新的,它生长着。

春天像小姑娘,花枝招展的,笑着,走着。

春天像健壮的青年,有铁一般的胳膊和腰脚,领着我们上前去。


\chapter{1   朱自清\sphinxhyphen{}梅雨潭的绿}
\label{\detokenize{p01_u6563_u6587/_u6731_u81ea_u6e05-_u6885_u96e8_u6f6d_u7684_u7eff:id1}}\label{\detokenize{p01_u6563_u6587/_u6731_u81ea_u6e05-_u6885_u96e8_u6f6d_u7684_u7eff::doc}}
\begin{sphinxShadowBox}
\sphinxstyletopictitle{目录}
\begin{itemize}
\item {} 
\phantomsection\label{\detokenize{p01_u6563_u6587/_u6731_u81ea_u6e05-_u6885_u96e8_u6f6d_u7684_u7eff:id9}}{\hyperref[\detokenize{p01_u6563_u6587/_u6731_u81ea_u6e05-_u6885_u96e8_u6f6d_u7684_u7eff:id1}]{\sphinxcrossref{1   朱自清\sphinxhyphen{}梅雨潭的绿}}}
\begin{itemize}
\item {} 
\phantomsection\label{\detokenize{p01_u6563_u6587/_u6731_u81ea_u6e05-_u6885_u96e8_u6f6d_u7684_u7eff:id10}}{\hyperref[\detokenize{p01_u6563_u6587/_u6731_u81ea_u6e05-_u6885_u96e8_u6f6d_u7684_u7eff:id3}]{\sphinxcrossref{1.1   作品原文}}}

\item {} 
\phantomsection\label{\detokenize{p01_u6563_u6587/_u6731_u81ea_u6e05-_u6885_u96e8_u6f6d_u7684_u7eff:id11}}{\hyperref[\detokenize{p01_u6563_u6587/_u6731_u81ea_u6e05-_u6885_u96e8_u6f6d_u7684_u7eff:id4}]{\sphinxcrossref{1.2   梅雨潭位置}}}

\item {} 
\phantomsection\label{\detokenize{p01_u6563_u6587/_u6731_u81ea_u6e05-_u6885_u96e8_u6f6d_u7684_u7eff:id12}}{\hyperref[\detokenize{p01_u6563_u6587/_u6731_u81ea_u6e05-_u6885_u96e8_u6f6d_u7684_u7eff:id5}]{\sphinxcrossref{1.3   梅雨潭主要景点}}}
\begin{itemize}
\item {} 
\phantomsection\label{\detokenize{p01_u6563_u6587/_u6731_u81ea_u6e05-_u6885_u96e8_u6f6d_u7684_u7eff:id13}}{\hyperref[\detokenize{p01_u6563_u6587/_u6731_u81ea_u6e05-_u6885_u96e8_u6f6d_u7684_u7eff:id6}]{\sphinxcrossref{1.3.1   通元洞}}}

\item {} 
\phantomsection\label{\detokenize{p01_u6563_u6587/_u6731_u81ea_u6e05-_u6885_u96e8_u6f6d_u7684_u7eff:id14}}{\hyperref[\detokenize{p01_u6563_u6587/_u6731_u81ea_u6e05-_u6885_u96e8_u6f6d_u7684_u7eff:id7}]{\sphinxcrossref{1.3.2   飞瀑}}}

\item {} 
\phantomsection\label{\detokenize{p01_u6563_u6587/_u6731_u81ea_u6e05-_u6885_u96e8_u6f6d_u7684_u7eff:id15}}{\hyperref[\detokenize{p01_u6563_u6587/_u6731_u81ea_u6e05-_u6885_u96e8_u6f6d_u7684_u7eff:id8}]{\sphinxcrossref{1.3.3   潭水}}}

\end{itemize}

\end{itemize}

\end{itemize}
\end{sphinxShadowBox}


\section{1.1   作品原文}
\label{\detokenize{p01_u6563_u6587/_u6731_u81ea_u6e05-_u6885_u96e8_u6f6d_u7684_u7eff:id3}}
我第二次到仙岩的时候,我惊诧于梅雨潭的绿了。

梅雨潭是一个瀑布潭。仙岩有三个瀑布,梅雨瀑最低。走到山边,便听见哗哗哗哗的声音;抬起头,镶在两条湿湿的黑边儿里的,一带白而发亮的水便呈现于眼前了。我们先到梅雨亭。梅雨亭正对着那条瀑布;坐在亭边,不必仰头,便可见它的全体了。亭下深深的便是梅雨潭。这个亭踞在突出的一角的岩石上,上下都空空儿的;仿佛一只苍鹰展着翼翅浮在天宇中一般。三面都是山,像半个环儿拥着;人如在井底了。这是一个秋季的薄阴的天气。微微的云在我们顶上流着;岩面与草丛都从润湿中透出几分油油的绿意。而瀑布也似乎分外的响了。那瀑布从上面冲下,仿佛已被扯成大小的几绺;不复是一幅整齐而平滑的布。岩上有许多棱角;瀑流经过时,作急剧的撞击,便飞花碎玉般乱溅着了。那溅着的水花,晶莹而多芒;远望去,像一朵朵小小的白梅,微雨似的纷纷落着。据说,这说是梅雨潭之所以得名了。但我觉得像杨花,格外确切些。轻风起来时,点点随风飘散,那更是杨花了。这时偶然有几点送入我们温暖的怀里,便倏的钻了进去,再也寻它不着。

梅雨潭闪闪的绿色招引着我们;我们开始追捉她那离合的神光了。揪着草,攀着乱石,小心探身下去,又鞠躬过了一个石穹门,便到了汪汪一碧的潭边了。瀑布在襟袖之间;但我的心中已没有瀑布了。我的心随潭水的绿而摇荡。那醉人的绿呀,仿佛一张极大极大的荷叶铺着,满是奇异的绿呀。我想张开两臂抱住她;但这是怎样一个妄想呀。—站在水边,望到那面,居然觉着有些远呢!这平铺着,厚积着的绿,着实可爱。她松松的皱缬着,像少妇拖着的裙幅;她轻轻的摆弄着,像跳动的初恋的处女的心;她滑滑的明亮着,像涂了“明油”一般,有鸡蛋清那样软,那样嫩,令人想着所曾触过的最嫩的皮肤;她又不杂些儿法滓,宛然一块温润的碧玉,只清清的一色—但你却看不透她!我曾见过北京什刹海指地的绿杨,脱不了鹅黄的底子,似乎太淡了。我又曾见过杭州虎跑寺旁高峻而深密的“绿壁”,重叠着无穷国的碧草与绿叶的,那又似乎太浓了。其余呢,西湖的波太明了,秦淮河的又太暗了。可爱的,我将什么来比拟你呢?我怎么比拟得出呢?大约潭是很深的、故能蕴蓄着这样奇异的绿;仿佛蔚蓝的天融了一块在里面似的,这才这般的鲜润呀。—那醉人的绿呀!我若能裁你以为带,我将赠给那轻盈的舞女;她必能临风飘举了。我若能挹你以为眼,我将赠给那善歌的盲妹;她必明眸善睐了。我舍不得你;我怎舍得你呢?我用手拍着你,抚摩着你,如同一个十二三岁的小姑娘。我又掬你入口,便是吻着她了。我送你一个名字,我从此叫你“女儿绿”,好么?

我第二次到仙岩的时候,我不禁惊诧于梅雨潭的绿了。


\section{1.2   梅雨潭位置}
\label{\detokenize{p01_u6563_u6587/_u6731_u81ea_u6e05-_u6885_u96e8_u6f6d_u7684_u7eff:id4}}
梅雨潭,是位于浙江温州市瓯海区仙岩街道的一处名胜,是国家AAA级景区。它东临东海,南北各距瑞安和温州三十多里。

温州一带的山,都属于连绵不断的雁荡山脉。然而仙岩所属的大罗山却远离群山,巍然坐落在温瑞平原上。其山平地拔起,峻崖陡壁,水源充沛,虽方圆不过数十里,却多瀑布潭,而尤集中在西麓瑞安境内的仙岩附近。瀑布潭比较著名的有三个:梅雨潭、雷响潭和龙须潭。其中以梅雨潭最有特色。


\section{1.3   梅雨潭主要景点}
\label{\detokenize{p01_u6563_u6587/_u6731_u81ea_u6e05-_u6885_u96e8_u6f6d_u7684_u7eff:id5}}
远远望去,梅雨潭的瀑布狂奔直下;梅雨亭坐落在瀑布前一块突出的
巨石之上,非常显眼,乍一看去,正如《绿》中写的,“仿佛一只苍鹰展着翼翅浮在天宇中一般”。此亭正对瀑布,原为明代少师张孚敬所建,初名泽润亭,因为安坐其中可观赏瀑布的全貌,作为建筑物又恰到好处地与梅雨潭的自然景色融为一体,故后人改称为“梅雨亭”。


\subsection{1.3.1   通元洞}
\label{\detokenize{p01_u6563_u6587/_u6731_u81ea_u6e05-_u6885_u96e8_u6f6d_u7684_u7eff:id6}}
亭下有洞通潭边,叫做“通元洞”,有个石穹门,旁边刻有“四时梅雨”四个丰满有力的大字。


\subsection{1.3.2   飞瀑}
\label{\detokenize{p01_u6563_u6587/_u6731_u81ea_u6e05-_u6885_u96e8_u6f6d_u7684_u7eff:id7}}
梅雨潭的两侧,双崖对耸,绝不可攀,崖壁上附满绿苔及草木,呈自然的暗绿色,飞瀑自崖合掌处喷吐而出,轰轰作响。

悬崖上岩石颇多棱角,瀑布跌撞而下,似散珠一般注入潭中,轻风吹来,水珠飘飘洒洒,犹如朵朵白梅。


\subsection{1.3.3   潭水}
\label{\detokenize{p01_u6563_u6587/_u6731_u81ea_u6e05-_u6885_u96e8_u6f6d_u7684_u7eff:id8}}
潭水很深,经石穹门下到潭边,水珠、雾气、绿水、悬崖,构成一幅奇妙壮观的图画。清代潘耒在《游仙岩记》中云:“常若梅天细雨,故名梅雨潭。”这个奇观使得在温州执教不到一年的朱自清,竟先后两次来此“追捉她那离合的神光”,与梅雨潭结下了不解之缘。

现在有人在梅雨潭的石穹门旁刻了一个斗大的“绿”字,以此纪念这位著名散文家朱自清的不朽名作《绿》。

那溅着的水花,晶莹而多芒,远望去,像一朵朵小小的白梅,微雨似的纷纷落着.这就是梅雨潭的由来

“踞”字表现出梅雨亭的雄伟 而“浮”字又突出了亭的轻盈
像这样用得生动传神的动词还有“镶” 。“镶”表现出了梅雨亭的—————优美


\chapter{1   朱自清\sphinxhyphen{}背影}
\label{\detokenize{p01_u6563_u6587/_u6731_u81ea_u6e05-_u80cc_u5f71:id1}}\label{\detokenize{p01_u6563_u6587/_u6731_u81ea_u6e05-_u80cc_u5f71::doc}}
\begin{sphinxShadowBox}
\sphinxstyletopictitle{目录}
\begin{itemize}
\item {} 
\phantomsection\label{\detokenize{p01_u6563_u6587/_u6731_u81ea_u6e05-_u80cc_u5f71:id14}}{\hyperref[\detokenize{p01_u6563_u6587/_u6731_u81ea_u6e05-_u80cc_u5f71:id1}]{\sphinxcrossref{1   朱自清\sphinxhyphen{}背影}}}
\begin{itemize}
\item {} 
\phantomsection\label{\detokenize{p01_u6563_u6587/_u6731_u81ea_u6e05-_u80cc_u5f71:id15}}{\hyperref[\detokenize{p01_u6563_u6587/_u6731_u81ea_u6e05-_u80cc_u5f71:id3}]{\sphinxcrossref{1.1   作品原文}}}

\item {} 
\phantomsection\label{\detokenize{p01_u6563_u6587/_u6731_u81ea_u6e05-_u80cc_u5f71:id16}}{\hyperref[\detokenize{p01_u6563_u6587/_u6731_u81ea_u6e05-_u80cc_u5f71:id4}]{\sphinxcrossref{1.2   词语注释编辑}}}

\item {} 
\phantomsection\label{\detokenize{p01_u6563_u6587/_u6731_u81ea_u6e05-_u80cc_u5f71:id17}}{\hyperref[\detokenize{p01_u6563_u6587/_u6731_u81ea_u6e05-_u80cc_u5f71:id5}]{\sphinxcrossref{1.3   创作背景}}}

\item {} 
\phantomsection\label{\detokenize{p01_u6563_u6587/_u6731_u81ea_u6e05-_u80cc_u5f71:id18}}{\hyperref[\detokenize{p01_u6563_u6587/_u6731_u81ea_u6e05-_u80cc_u5f71:id6}]{\sphinxcrossref{1.4   内容赏析}}}
\begin{itemize}
\item {} 
\phantomsection\label{\detokenize{p01_u6563_u6587/_u6731_u81ea_u6e05-_u80cc_u5f71:id19}}{\hyperref[\detokenize{p01_u6563_u6587/_u6731_u81ea_u6e05-_u80cc_u5f71:id7}]{\sphinxcrossref{1.4.1   第一部分(第一至第三段)}}}

\item {} 
\phantomsection\label{\detokenize{p01_u6563_u6587/_u6731_u81ea_u6e05-_u80cc_u5f71:id20}}{\hyperref[\detokenize{p01_u6563_u6587/_u6731_u81ea_u6e05-_u80cc_u5f71:id8}]{\sphinxcrossref{1.4.2   第二部分(第四至第六段)}}}

\item {} 
\phantomsection\label{\detokenize{p01_u6563_u6587/_u6731_u81ea_u6e05-_u80cc_u5f71:id21}}{\hyperref[\detokenize{p01_u6563_u6587/_u6731_u81ea_u6e05-_u80cc_u5f71:id9}]{\sphinxcrossref{1.4.3   第三部分(最后一段)}}}

\end{itemize}

\item {} 
\phantomsection\label{\detokenize{p01_u6563_u6587/_u6731_u81ea_u6e05-_u80cc_u5f71:id22}}{\hyperref[\detokenize{p01_u6563_u6587/_u6731_u81ea_u6e05-_u80cc_u5f71:id10}]{\sphinxcrossref{1.5   语言特色}}}

\item {} 
\phantomsection\label{\detokenize{p01_u6563_u6587/_u6731_u81ea_u6e05-_u80cc_u5f71:id23}}{\hyperref[\detokenize{p01_u6563_u6587/_u6731_u81ea_u6e05-_u80cc_u5f71:id11}]{\sphinxcrossref{1.6   写作特色}}}

\item {} 
\phantomsection\label{\detokenize{p01_u6563_u6587/_u6731_u81ea_u6e05-_u80cc_u5f71:id24}}{\hyperref[\detokenize{p01_u6563_u6587/_u6731_u81ea_u6e05-_u80cc_u5f71:id12}]{\sphinxcrossref{1.7   行文立意}}}

\item {} 
\phantomsection\label{\detokenize{p01_u6563_u6587/_u6731_u81ea_u6e05-_u80cc_u5f71:id25}}{\hyperref[\detokenize{p01_u6563_u6587/_u6731_u81ea_u6e05-_u80cc_u5f71:id13}]{\sphinxcrossref{1.8   名家点评}}}

\end{itemize}

\end{itemize}
\end{sphinxShadowBox}


\section{1.1   作品原文}
\label{\detokenize{p01_u6563_u6587/_u6731_u81ea_u6e05-_u80cc_u5f71:id3}}
我与父亲不相见已二年余了,我最不能忘记的是他的背影。

那年冬天,祖母死了,父亲的差使1也交卸了,正是祸不单行的日子。我从北京到徐州,打算跟着父亲奔丧2回家。到徐州见着父亲,看见满院狼藉3的东西,又想起祖母,不禁簌簌地流下眼泪。父亲说:“事已如此,不必难过,好在天无绝人之路!”

回家变卖典质4,父亲还了亏空;又借钱办了丧事。这些日子,家中光景5很是惨澹,一半为了丧事,一半为了父亲赋闲6。丧事完毕,父亲要到南京谋事,我也要回北京念书,我们便同行。

到南京时,有朋友约去游逛,勾留7了一日;第二日上午便须渡江到浦口,下午上车北去。父亲因为事忙,本已说定不送我,叫旅馆里一个熟识的茶房8陪我同去。他再三嘱咐茶房,甚是仔细。但他终于不放心,怕茶房不妥帖9;颇踌躇10了一会。其实我那年已二十岁,北京已来往过两三次,是没有什么要紧的了。他踌躇了一会,终于决定还是自己送我去。我再三劝他不必去;他只说:“不要紧,他们去不好!”

我们过了江,进了车站。我买票,他忙着照看行李。行李太多,得向脚夫11行些小费才可过去。他便又忙着和他们讲价钱。我那时真是聪明过分,总觉他说话不大漂亮,非自己插嘴不可,但他终于讲定了价钱;就送我上车。他给我拣定了靠车门的一张椅子;我将他给我做的紫毛大衣铺好座位。他嘱我路上小心,夜里要警醒些,不要受凉。又嘱托茶房好好照应我。我心里暗笑他的迂;他们只认得钱,托他们只是白托!而且我这样大年纪的人,难道还不能料理自己么?我现在想想,我那时真是太聪明了。

我说道:“爸爸,你走吧。”他望车外看了看,说:“我买几个橘子去。你就在此地,不要走动。”我看那边月台的栅栏外有几个卖东西的等着顾客。走到那边月台,须穿过铁道,须跳下去又爬上去。父亲是一个胖子,走过去自然要费事些。我本来要去的,他不肯,只好让他去。我看见他戴着黑布小帽,穿着黑布大马褂12,深青布棉袍,蹒跚13地走到铁道边,慢慢探身下去,尚不大难。可是他穿过铁道,要爬上那边月台,就不容易了。他用两手攀着上面,两脚再向上缩;他肥胖的身子向左微倾,显出努力的样子。这时我看见他的背影,我的泪很快地流下来了。我赶紧拭干了泪。怕他看见,也怕别人看见。我再向外看时,他已抱了朱红的橘子往回走了。过铁道时,他先将橘子散放在地上,自己慢慢爬下,再抱起橘子走。到这边时,我赶紧去搀他。他和我走到车上,将橘子一股脑儿放在我的皮大衣上。于是扑扑衣上的泥土,心里很轻松似的。过一会儿说:“我走了,到那边来信!”我望着他走出去。他走了几步,回过头看见我,说:“进去吧,里边没人。”等他的背影混入来来往往的人里,再找不着了,我便进来坐下,我的眼泪又来了。

近几年来,父亲和我都是东奔西走,家中光景是一日不如一日。他少年出外谋生,独力支持,做了许多大事。哪知老境却如此颓唐!他触目伤怀,自然情不能自已。情郁于中,自然要发之于外;家庭琐屑便往往触他之怒。他待我渐渐不同往日。但最近两年不见,他终于忘却我的不好,只是惦记着我,惦记着他的儿子。我北来后,他写了一信给我,信中说道:“我身体平安,惟膀子疼痛厉害,举箸14提笔,诸多不便,大约大去之期15不远矣。”我读到此处,在晶莹的泪光中,又看见那肥胖的、青布棉袍黑布马褂的背影。唉!我不知何时再能与他相见!{[}2{]}


\section{1.2   词语注释编辑}
\label{\detokenize{p01_u6563_u6587/_u6731_u81ea_u6e05-_u80cc_u5f71:id4}}
1.差(chāi)使:旧时官场中称临时委任的职务,后来泛指职务或官职。

2.奔丧:在外闻亲人去世而归。

3.狼藉(jí):散乱不整齐的样子。亦作“狼籍”。

4.典质:典当,抵押。

5.光景:境况。

6.赋闲:没有职业在家闲居。

7.勾留:逗留。

8.茶房:旧时称在旅馆、茶馆、轮船、火车、剧场等地方从事供应茶水等杂务工作的人。

9.妥帖:恰当,十分合适。

10.踌躇(chóuchú):犹豫。

11.脚夫:旧称搬运工人。

12.马褂:旧时男子穿在长袍外面的对襟短褂。

13.蹒跚(pánshān):走路缓慢、摇摆的样子。也作“盘跚”。

14.箸(zhù):筷子。

15.大去之期:辞世的日子。


\section{1.3   创作背景}
\label{\detokenize{p01_u6563_u6587/_u6731_u81ea_u6e05-_u80cc_u5f71:id5}}
1917年,作者的祖母去世,父亲任徐州烟酒公卖局局长的差事也交卸了。办完丧事,父子同到南京,父亲送作者上火车北去,那年作者20岁。在那特定的场合下,做为父亲对儿子的关怀、体贴、爱护,使儿子极为感动,这印象经久不忘,并且几年之后,想起那背影,父亲的影子出现在“晶莹的泪光中”,使人不能忘怀。1925年,作者有感于世事,便写了此文。{[}4{]}


\section{1.4   内容赏析}
\label{\detokenize{p01_u6563_u6587/_u6731_u81ea_u6e05-_u80cc_u5f71:id6}}
全文可分为三部分。


\subsection{1.4.1   第一部分(第一至第三段)}
\label{\detokenize{p01_u6563_u6587/_u6731_u81ea_u6e05-_u80cc_u5f71:id7}}
交代人物,叙述跟父亲奔丧回家的有关情节,为描写父亲的背影作好铺垫。文章开头一句,落笔点题。“二年余”表明“我”清楚地记得和父亲分离的日子。副词“已”体现出“二年余”在作者的心目中已相当漫长,想望之情,不言而喻。两年多的分离,“我”对父亲的思念是多方面的。其中“最不能忘记的是他的背影”,点出题目。接着,转入对“那年冬天”往事的追述。“祖母死了,父亲的差使也交卸了”,短短两句呈现出人事错迁、谋生艰难之感。“我”从北京到了父亲的住地以后,“看见满院狼藉的东西”,其潦倒之状,又使“我不禁簌簌地流下眼泪”。因为“祸不单行”,所以回家之后,靠“变卖典质”,才还了“亏空”,又“借钱办了丧事”。这里所用的“祸不单行”、“亏空”,“借钱”、“丧事”等词语,一方面是当时情况的真实写照,同时也使后面“家中光景很是惨澹”的形容更有着落。这些叙述和描写,生动地反映了当时世态的灰暗。毛泽东主席在《中国社会各阶级的分析》一文中,曾对当时小资产阶级左翼的情况做过分析,说:“这种人因为他们过去过着好日子,后来逐年下降,负债渐多,渐次过着凄凉的日子,瞻念前途,不寒而栗”。这篇散文所叙述的情节,所抒发的感情,具有一定的典型意义的,也是此文为之感动共鸣的重要原因。


\subsection{1.4.2   第二部分(第四至第六段)}
\label{\detokenize{p01_u6563_u6587/_u6731_u81ea_u6e05-_u80cc_u5f71:id8}}
写父亲为“我”送行的情景,重点描写父亲的背影,表现父子间的真挚感情。丧事完毕,因为父亲要到南京谋事,“我”也要回北京念书,所以父子便一路同行到了南京。到南京之后,因为父亲要谋事,须接交各种关系,忙是可以想见的。所以说定要一个熟识的茶房为“我”送行。“他再三嘱咐茶房,甚是仔细。”这既表现了父亲对“我”的关怀,同时也说明了他对茶房的不放心。父亲当时异地谋生,正须多方奔走,又难以抽身,因此,他“颇踌躇了一会”。“踌躇”,反映了在父亲心中谋事与送子的矛盾。而“终于决定还是自己送我去”,则又表现了父亲毅然将生计暂时搁置,执意为“我”送行的真切感情。“终予”二字,把父亲对“我”无限关切、过分忧虑的心理,表现得淋漓尽致。接下去写的便是车站送行的场面。进了车站以后,父亲“忙着照看行李”,“忙着向脚夫讲价钱”,“送我上车”,“给我拣定靠车门的一张椅子”,“嘱我路上小心”。父亲操劳忙碌的形象展现在面前。可“我”那时由于太年轻,对父亲尚不能完全理解,以至于还在“心里暗笑他的迂”。作者行文至此,一种近乎忏悔的感情不觉流注笔端——“唉,我现在想想,那时真是太聪明了!”自我责备之中,包含着深切的内疚与怀念。在车上坐定之后,父亲又要为“我”去买橘子。但买橘子,“须穿过铁道,须跳下去又爬上去”。父亲又胖,吃力之状可以想见。因此,父亲当时去买橘子的情景,给“我”留下了极为深刻的记忆。当父亲“蹒跚地走到铁道边”时,“我”心中的酸楚是自不待言的。“蹒跚”一词,说明父亲年事已高,步履不稳,过铁路需人扶持。而今,为了“我”却在铁道间蹒跚前往。因而当看见父亲“用两手攀着……努力的样子”的背影时,“我的眼泪”便“很快地流下来了”。这“背影”集中地体现了父亲待“我”的全部感情,这“背影”使“我”念之心酸,感愧交并!望着父亲那吃力的背影,“我”禁不住热泪涌流,但为了“怕他看见”,“我”又“赶紧拭干了泪”,互相体谅的父子真情,表现得维妙维肖。父亲终于买来了橘子。当他走到这边时,“我赶紧去搀他”。这赶紧去搀的动作,表现了“我”又疼,又愧,又欣然若释的复杂心理。疼的是父亲为“我”受累,愧的是父亲为“我”买橘,欣然若释的是父亲终于安全归来。父亲回来之后,“我”虽然没讲一句话,但一腔深情都流露在这“赶紧去搀扶”的动作之中。回到车上,父亲“将橘子一股脑儿放在我的皮大衣上”。“一股脑儿”一词,表现了父亲当时高兴的心情。但父亲高兴的仅仅是为“我”买到了橘子,他的心头是并不轻松的。他谋生无着,而“我”又即将离他远去,兴从何来,所以文章说“心里很轻松似的”,“似的”二字说明父亲并不真正轻松,之所以做出仿佛轻松的样子,是为了宽慰那正心中眷眷的儿子,橘子已经买来,行李也早就安放停当,嘱咐的话也已经说过,看来没什么事了。但父亲并没有马上离去,而是“过一会”才说出告别的话。这“一会”之间,有拳拳的依恋,有惜别的惆怅。父亲终于说,“我走了;到那边来信!”临别的嘱咐,又一次表现了父亲对“我”的牵挂与系念。一直到他走了几步之后,还回过头来说“进去吧,里边没人”,仍关心着“我”的安全。但“我”并没有马上进去,而是“等他的背影……我便进来坐下”。这里的“等”、“再’、“便”三个字,用得极有层次,它们真实地表现了“我”站在车门口,追寻注视着父亲的背影,直到再也看不见时,才进去坐下的那种怅然若失的心情。“我”坐下之后,也许又看到了刚才父亲买来的橘子,一股热辣辣的感情又从心底兜起,“我的眼泪又来了”。


\subsection{1.4.3   第三部分(最后一段)}
\label{\detokenize{p01_u6563_u6587/_u6731_u81ea_u6e05-_u80cc_u5f71:id9}}
写对父亲的想念。作者在描写了父亲的背影之后,予深沉的怀念之中,又想起了父亲的一生。“他少年出外谋生,独力支持,做了许多大事。”父亲是坚强而能干的。虽然如此,家庭生活仍然每况愈下,“光景是一日不如一日”。父亲“触目伤怀”,脾气也变得易于暴怒了。因而,“他待我渐渐不同往日”,但这并非父亲本来的感情,父亲仍旧是父亲。两年不见,又使他在“举箸提笔,诸多不便”的情况下,写了信来,仍旧“惦记着我,惦记着我的儿子”。并在信中写道,“大约大去之期不远矣”,哀矜之中流露出孤寂、颓唐的况昧。它使“我”震悚,使“我”苦痛,使“我”想起父亲待“我”的种种好处,使“我”透过晶莹的泪光,又看见了父亲那凄楚的背影。父亲现在究竟怎样了,“唉!我不知何时再能与他相见。”盼望之中蕴蓄着热切的思念。


\section{1.5   语言特色}
\label{\detokenize{p01_u6563_u6587/_u6731_u81ea_u6e05-_u80cc_u5f71:id10}}
这篇散文的语言非常忠实朴素,又非常典雅文质。这种高度民族化的语言,和文章所表现的民族的精神气质,和文章的完美结构,恰成和谐的统一。没有《背影》语言的简洁明丽、古朴质实,就没有《背影》的一切风采。《背影》的语言还有文白夹杂的特点。例如不说“失业”,而说“赋闲”,最后一节因父亲来信是文言,引用原句,更见真实,也表达了家庭、父亲的困境和苍凉的心情与复杂的感受,同时,文白夹杂的语句,也笼上了一层时代赋予小资产阶级知识分子的特殊语言色彩。


\section{1.6   写作特色}
\label{\detokenize{p01_u6563_u6587/_u6731_u81ea_u6e05-_u80cc_u5f71:id11}}
这篇散文写作上的主要特点是白描。全文集中描写的,是父亲在特定场合下使作者极为感动的那一个背影。作者写了当时父亲的体态、穿着打扮,更主要地写了买橘子时穿过铁路的情形。并不借助于什么修饰、陪衬之类,只把当时的情景再现于眼前。这种白描的文字,读起来清淡质朴,却情真昧浓,蕴藏着一段深情。所谓于平淡中见神奇。其次,作品还运用了侧面烘托的手法。如写儿子“看见他的背影”,“泪很快地流下来了”。又写父亲买桔子回来时,儿子“赶紧去搀他”。这些侧面烘托手法的运用,更加反衬出父亲爱子的动人力量。


\section{1.7   行文立意}
\label{\detokenize{p01_u6563_u6587/_u6731_u81ea_u6e05-_u80cc_u5f71:id12}}
这篇散文的特点是抓住人物形象的特征“背影”命题立意,在叙事中抒发父子深情。“背影”在文章中出现了四次,每次的情况有所不同,而思想感情却是一脉相承。第一次开篇点题“背影”,有一种浓厚的感情气氛笼罩全文。第二次车站送别,作者对父亲的“背影”做了具体的描绘。第三次是父亲和儿子告别后,儿子眼望着父亲的“背影”在人群中消逝,离情别绪,催人泪下。第四次在文章的结尾,儿子读着父亲的来信,在泪光中再次浮现了父亲的“背影”,思念之情不能自已,与文章开头呼应,把父子之间的真挚感情表现得淋漓尽致。


\section{1.8   名家点评}
\label{\detokenize{p01_u6563_u6587/_u6731_u81ea_u6e05-_u80cc_u5f71:id13}}
李广田《最完整的人格》:《背影》论行数不满五十行,论字数不过千五百言,它之所以能够历久传诵而有感人至深的力量者,当然并不是凭藉了甚么宏伟的结构和华瞻的文字,而只是凭了它的老实,凭了其中所表达的真情。这种表面上看起来简单朴素,而实际上却能发生极大的感动力的文章,最可以作为朱先生的代表作品,因为这样的作品,也正好代表了作者之为人。

叶圣陶《文章例话》:“这篇文章通体干净,没有多余的话,没有多余的字眼,即使一个“的”字,一个“了”字,也是必须用才用”。

吴晗《他们走到了它的反面——朱自清颂》:“《背影》虽然只有一千五百字,却历久传诵,有感人至深的力量,这篇短文被选为中学国文教材,在中学生心目中,‘朱自清’三个字已经和《背影》成为不可分割的一体了”。


\chapter{1   朱自清\sphinxhyphen{}荷塘月色}
\label{\detokenize{p01_u6563_u6587/_u6731_u81ea_u6e05-_u8377_u5858_u6708_u8272:id1}}\label{\detokenize{p01_u6563_u6587/_u6731_u81ea_u6e05-_u8377_u5858_u6708_u8272::doc}}
\begin{sphinxShadowBox}
\sphinxstyletopictitle{目录}
\begin{itemize}
\item {} 
\phantomsection\label{\detokenize{p01_u6563_u6587/_u6731_u81ea_u6e05-_u8377_u5858_u6708_u8272:id5}}{\hyperref[\detokenize{p01_u6563_u6587/_u6731_u81ea_u6e05-_u8377_u5858_u6708_u8272:id1}]{\sphinxcrossref{1   朱自清\sphinxhyphen{}荷塘月色}}}
\begin{itemize}
\item {} 
\phantomsection\label{\detokenize{p01_u6563_u6587/_u6731_u81ea_u6e05-_u8377_u5858_u6708_u8272:id6}}{\hyperref[\detokenize{p01_u6563_u6587/_u6731_u81ea_u6e05-_u8377_u5858_u6708_u8272:id3}]{\sphinxcrossref{1.1   作品原文}}}

\item {} 
\phantomsection\label{\detokenize{p01_u6563_u6587/_u6731_u81ea_u6e05-_u8377_u5858_u6708_u8272:id7}}{\hyperref[\detokenize{p01_u6563_u6587/_u6731_u81ea_u6e05-_u8377_u5858_u6708_u8272:id4}]{\sphinxcrossref{1.2   词语注释}}}

\end{itemize}

\end{itemize}
\end{sphinxShadowBox}


\section{1.1   作品原文}
\label{\detokenize{p01_u6563_u6587/_u6731_u81ea_u6e05-_u8377_u5858_u6708_u8272:id3}}
这几天心里颇不宁静。今晚在院子里坐着乘凉,忽然想起日日走过的荷塘,在这满月的光里,总该另有一番样子吧。月亮渐渐地升高了,墙外马路上孩子们的欢笑,已经听不见了;妻在屋里拍着闰儿⑴,迷迷糊糊地哼着眠歌。我悄悄地披了大衫,带上门出去。

沿着荷塘,是一条曲折的小煤屑路。这是一条幽僻的路;白天也少人走,夜晚更加寂寞。荷塘四面,长着许多树,蓊蓊郁郁⑵的。路的一旁,是些杨柳,和一些不知道名字的树。没有月光的晚上,这路上阴森森的,有些怕人。今晚却很好,虽然月光也还是淡淡的。

路上只我一个人,背着手踱⑶着。这一片天地好像是我的;我也像超出了平常的自己,到了另一个世界里。我爱热闹,也爱冷静;爱群居,也爱独处。像今晚上,一个人在这苍茫的月下,什么都可以想,什么都可以不想,便觉是个自由的人。白天里一定要做的事,一定要说的话,现在都可不理。这是独处的妙处,我且受用这无边的荷香月色好了。

曲曲折折的荷塘上面,弥望⑷的是田田⑸的叶子。叶子出水很高,像亭亭的舞女的裙。层层的叶子中间,零星地点缀着些白花,有袅娜⑹地开着的,有羞涩地打着朵儿的;正如一粒粒的明珠,又如碧天里的星星,又如刚出浴的美人。微风过处,送来缕缕清香,仿佛远处高楼上渺茫的歌声似的。这时候叶子与花也有一丝的颤动,像闪电般,霎时传过荷塘的那边去了。叶子本是肩并肩密密地挨着,这便宛然有了一道凝碧的波痕。叶子底下是脉脉⑺的流水,遮住了,不能见一些颜色;而叶子却更见风致⑻了。

月光如流水一般,静静地泻在这一片叶子和花上。薄薄的青雾浮起在荷塘里。叶子和花仿佛在牛乳中洗过一样;又像笼着轻纱的梦。虽然是满月,天上却有一层淡淡的云,所以不能朗照;但我以为这恰是到了好处——酣眠固不可少,小睡也别有风味的。月光是隔了树照过来的,高处丛生的灌木,落下参差的斑驳的黑影,峭楞楞如鬼一般;弯弯的杨柳的稀疏的倩影,却又像是画在荷叶上。塘中的月色并不均匀;但光与影有着和谐的旋律,如梵婀玲⑼上奏着的名曲。

荷塘的四面,远远近近,高高低低都是树,而杨柳最多。这些树将一片荷塘重重围住;只在小路一旁,漏着几段空隙,像是特为月光留下的。树色一例是阴阴的,乍看像一团烟雾;但杨柳的丰姿⑽,便在烟雾里也辨得出。树梢上隐隐约约的是一带远山,只有些大意罢了。树缝里也漏着一两点路灯光,没精打采的,是渴睡⑾人的眼。这时候最热闹的,要数树上的蝉声与水里的蛙声;但热闹是它们的,我什么也没有。

忽然想起采莲的事情来了。采莲是江南的旧俗,似乎很早就有,而六朝时为盛;从诗歌里可以约略知道。采莲的是少年的女子,她们是荡着小船,唱着艳歌去的。采莲人不用说很多,还有看采莲的人。那是一个热闹的季节,也是一个风流的季节。梁元帝《采莲赋》里说得好:

于是妖童媛女⑿,荡舟心许;鷁首⒀徐回,兼传羽杯⒁;棹⒂将移而藻挂,船欲动而萍开。尔其纤腰束素⒃,迁延顾步⒄;夏始春余,叶嫩花初,恐沾裳而浅笑,畏倾船而敛裾⒅。

可见当时嬉游的光景了。这真是有趣的事,可惜我们现在早已无福消受了。

于是又记起,《西洲曲》里的句子:

采莲南塘秋,莲花过人头;低头弄莲子,莲子清如水。

今晚若有采莲人,这儿的莲花也算得“过人头”了;只不见一些流水的影子,是不行的。这令我到底惦着江南了。——这样想着,猛一抬头,不觉已是自己的门前;轻轻地推门进去,什么声息也没有,妻已睡熟好久了。

一九二七年七月,北京清华园。


\section{1.2   词语注释}
\label{\detokenize{p01_u6563_u6587/_u6731_u81ea_u6e05-_u8377_u5858_u6708_u8272:id4}}
1、闰儿:指朱闰生,朱自清第二子。

2、蓊蓊(wěng)郁郁:树木茂盛的样子。

3、踱(duó):慢慢地走

4、弥望:满眼。弥,满。

5、田田:形容荷叶相连的样子。古乐府《江南曲》中有“莲叶何田田”之句。

6、袅娜(niǎonuó):柔美的样子。

7、脉脉(mò):这里形容水没有声音,好像饱含深情的样子。

8、风致:美的姿态。

9、梵婀玲:violin,小提琴的音译。

10、丰姿:风度,仪态,一般指美好的姿态。也写作“风姿”

11、渴睡:也写作“瞌睡”。

12、妖童媛女:俊俏的少年和美丽的少女。妖,艳丽。媛,女子。

13、鷁首(yìshǒu):船头。古代画鷁鸟于船头。

14、羽杯:古代饮酒用的耳杯。又称羽觞、耳杯。

15、棹(zhào):船桨。

16、纤腰束素:腰如束素,齿如含贝(宋玉《登徒子好色赋》),形容女子腰肢细柔

17、迁延顾步:形容走走退退不住回视自己动作的样子,有顾影自怜之意。

18、敛裾(jū):这里是提着衣襟的意思。裾,衣襟。


\chapter{1   毛泽东\sphinxhyphen{}七律·长征}
\label{\detokenize{p01_u6563_u6587/_u6bdb_u6cfd_u4e1c-_u4e03_u5f8b_xb7_u957f_u5f81:id1}}\label{\detokenize{p01_u6563_u6587/_u6bdb_u6cfd_u4e1c-_u4e03_u5f8b_xb7_u957f_u5f81::doc}}
\begin{sphinxShadowBox}
\sphinxstyletopictitle{目录}
\begin{itemize}
\item {} 
\phantomsection\label{\detokenize{p01_u6563_u6587/_u6bdb_u6cfd_u4e1c-_u4e03_u5f8b_xb7_u957f_u5f81:id7}}{\hyperref[\detokenize{p01_u6563_u6587/_u6bdb_u6cfd_u4e1c-_u4e03_u5f8b_xb7_u957f_u5f81:id1}]{\sphinxcrossref{1   毛泽东\sphinxhyphen{}七律·长征}}}
\begin{itemize}
\item {} 
\phantomsection\label{\detokenize{p01_u6563_u6587/_u6bdb_u6cfd_u4e1c-_u4e03_u5f8b_xb7_u957f_u5f81:id8}}{\hyperref[\detokenize{p01_u6563_u6587/_u6bdb_u6cfd_u4e1c-_u4e03_u5f8b_xb7_u957f_u5f81:id3}]{\sphinxcrossref{1.1   作品原文}}}

\item {} 
\phantomsection\label{\detokenize{p01_u6563_u6587/_u6bdb_u6cfd_u4e1c-_u4e03_u5f8b_xb7_u957f_u5f81:id9}}{\hyperref[\detokenize{p01_u6563_u6587/_u6bdb_u6cfd_u4e1c-_u4e03_u5f8b_xb7_u957f_u5f81:id4}]{\sphinxcrossref{1.2   词句注释}}}

\item {} 
\phantomsection\label{\detokenize{p01_u6563_u6587/_u6bdb_u6cfd_u4e1c-_u4e03_u5f8b_xb7_u957f_u5f81:id10}}{\hyperref[\detokenize{p01_u6563_u6587/_u6bdb_u6cfd_u4e1c-_u4e03_u5f8b_xb7_u957f_u5f81:id5}]{\sphinxcrossref{1.3   白话译文}}}

\item {} 
\phantomsection\label{\detokenize{p01_u6563_u6587/_u6bdb_u6cfd_u4e1c-_u4e03_u5f8b_xb7_u957f_u5f81:id11}}{\hyperref[\detokenize{p01_u6563_u6587/_u6bdb_u6cfd_u4e1c-_u4e03_u5f8b_xb7_u957f_u5f81:id6}]{\sphinxcrossref{1.4   创作背景}}}

\end{itemize}

\end{itemize}
\end{sphinxShadowBox}

《七律·长征》是一首七言律诗,选自《毛泽东诗词集》,这首诗写于1935年10月,当时毛泽东率领中央红军越过岷山,长征即将结束。回顾长征一年来所战胜的无数艰难险阻,他满怀喜悦的战斗豪情。


\section{1.1   作品原文}
\label{\detokenize{p01_u6563_u6587/_u6bdb_u6cfd_u4e1c-_u4e03_u5f8b_xb7_u957f_u5f81:id3}}
七律·长征

七律⑴·长征⑵

红军不怕远征难⑶,万水千山只等闲⑷。

五岭⑸逶迤⑹腾细浪⑺,乌蒙⑻磅礴走泥丸⑼。

金沙⑽水拍云崖暖⑾,大渡桥⑿横铁索⒀寒⒁。

更喜岷山⒂千里雪,三军⒃过后尽开颜⒄。{[}1{]}


\section{1.2   词句注释}
\label{\detokenize{p01_u6563_u6587/_u6bdb_u6cfd_u4e1c-_u4e03_u5f8b_xb7_u957f_u5f81:id4}}
⑴七律:七律是律诗的一种,每篇一般为八句,每句七个字,分四联:首联、颔联、颈联和尾联;偶句末一字押平声韵,首句末字可押可不押,必须一韵到底;句内和句间要讲平仄,中间四句按常规要用对仗。

⑵长征:1934年10月间,中央红军主力从中央革命根据地出发作战略大转移,经过福建、江西、广东、湖南、广西、贵州、四川、云南、西藏、甘肃、陕西等十一省,击溃了敌人多次的围追和堵截,战胜了军事上、政治上和自然界的无数艰险,行军二万五千里,终于在1935年10月到达陕北革命根据地。

⑶难:艰难险阻。

⑷等闲:不怕困难,不可阻止。

⑸五岭:大庾岭,骑田岭,都庞岭,萌渚岭,越城岭,横亘在江西、湖南、两广之间。

⑹逶迤:形容道路、山脉、河流等弯弯曲曲,连绵不断的样子。

⑺细浪:作者自释:“把山比作‘细浪’、‘泥丸’,是‘等闲’之意。”

⑻乌蒙:山名。乌蒙山,在贵州西部与云南东北部的交界处,北临金沙江,山势陡峭。1935年4月,红军长征经过此地。

⑼泥丸:小泥球,整句意思说险峻的乌蒙山在红军战士的脚下,就像是一个小泥球一样。

⑽金沙:金沙江,指长江上游自青海省玉树县至四川省宜宾市的一段,云南等地也有支流。1935年5月,红军曾强渡云南省禄劝县皎平渡渡口。

⑾云崖暖:是指浪花拍打悬崖峭壁,溅起阵阵雾水,在红军的眼中像是冒出的蒸汽一样。(云崖:高耸入云的山崖。暖:被一些学者指为红军巧渡金沙江后的欢快心情,也有学者说意思为直译后的温暖。)

⑿大渡桥:指四川省西部泸定县大渡河上的泸定桥。

⒀铁索:大渡河上泸定桥,它是用十三根铁索组成的桥。

⒁寒:影射敌人的冷酷与形势的严峻。

⒂岷(mín)山:中国西部大山。位于甘肃省西南、四川省北部。西北\sphinxhyphen{}东南走向。西北接西倾山,南与邛崃山相连。包括甘肃南部的迭山,甘肃、四川边境的摩天岭。

⒃三军:作者自注:“红军一方面军,二方面军,四方面军。”

⒄尽开颜:红军的长征到达目的地了,他们取得了胜利,所以个个都笑逐颜开。


\section{1.3   白话译文}
\label{\detokenize{p01_u6563_u6587/_u6bdb_u6cfd_u4e1c-_u4e03_u5f8b_xb7_u957f_u5f81:id5}}
红军不怕万里长征路上的一切艰难困苦,把千山万水都看得极为平常。绵延不断的五岭,在红军看来只不过是微波细浪在起伏,而气势雄伟的乌蒙山,在红军眼里也不过是一颗泥丸。

金沙江浊浪滔天,拍击着高耸入云的峭壁悬崖,热气腾腾。大渡河险桥横架,晃动着凌空高悬的根根铁索,寒意阵阵。

更加令人喜悦的是踏上千里积雪的岷山,红军翻越过去以后个个笑逐颜开。


\section{1.4   创作背景}
\label{\detokenize{p01_u6563_u6587/_u6bdb_u6cfd_u4e1c-_u4e03_u5f8b_xb7_u957f_u5f81:id6}}
1934年10月,中国工农红军为粉碎国民政府的围剿,保存自己的实力,也为了北上抗日,挽救民族危亡,从江西瑞金出发,开始了举世闻名的长征。

这首七律是作于红军战士越过岷山后,长征即将胜利结束前不久的途中。作为红军的领导人,毛泽东在经受了无数次考验后,如今,曙光在前,胜利在望,他心潮澎湃,满怀豪情地写下了这首壮丽的诗篇。《七律·长征》写于1935年9月下旬,10月定稿。


\chapter{1   毛泽东\sphinxhyphen{}沁园春·雪}
\label{\detokenize{p01_u6563_u6587/_u6bdb_u6cfd_u4e1c-_u6c81_u56ed_u6625_xb7_u96ea:id1}}\label{\detokenize{p01_u6563_u6587/_u6bdb_u6cfd_u4e1c-_u6c81_u56ed_u6625_xb7_u96ea::doc}}
\begin{sphinxShadowBox}
\sphinxstyletopictitle{目录}
\begin{itemize}
\item {} 
\phantomsection\label{\detokenize{p01_u6563_u6587/_u6bdb_u6cfd_u4e1c-_u6c81_u56ed_u6625_xb7_u96ea:id8}}{\hyperref[\detokenize{p01_u6563_u6587/_u6bdb_u6cfd_u4e1c-_u6c81_u56ed_u6625_xb7_u96ea:id1}]{\sphinxcrossref{1   毛泽东\sphinxhyphen{}沁园春·雪}}}
\begin{itemize}
\item {} 
\phantomsection\label{\detokenize{p01_u6563_u6587/_u6bdb_u6cfd_u4e1c-_u6c81_u56ed_u6625_xb7_u96ea:id9}}{\hyperref[\detokenize{p01_u6563_u6587/_u6bdb_u6cfd_u4e1c-_u6c81_u56ed_u6625_xb7_u96ea:id3}]{\sphinxcrossref{1.1   作品原文}}}

\item {} 
\phantomsection\label{\detokenize{p01_u6563_u6587/_u6bdb_u6cfd_u4e1c-_u6c81_u56ed_u6625_xb7_u96ea:id10}}{\hyperref[\detokenize{p01_u6563_u6587/_u6bdb_u6cfd_u4e1c-_u6c81_u56ed_u6625_xb7_u96ea:id4}]{\sphinxcrossref{1.2   词句注释}}}

\item {} 
\phantomsection\label{\detokenize{p01_u6563_u6587/_u6bdb_u6cfd_u4e1c-_u6c81_u56ed_u6625_xb7_u96ea:id11}}{\hyperref[\detokenize{p01_u6563_u6587/_u6bdb_u6cfd_u4e1c-_u6c81_u56ed_u6625_xb7_u96ea:id5}]{\sphinxcrossref{1.3   白话译文}}}

\item {} 
\phantomsection\label{\detokenize{p01_u6563_u6587/_u6bdb_u6cfd_u4e1c-_u6c81_u56ed_u6625_xb7_u96ea:id12}}{\hyperref[\detokenize{p01_u6563_u6587/_u6bdb_u6cfd_u4e1c-_u6c81_u56ed_u6625_xb7_u96ea:id6}]{\sphinxcrossref{1.4   创作背景}}}

\item {} 
\phantomsection\label{\detokenize{p01_u6563_u6587/_u6bdb_u6cfd_u4e1c-_u6c81_u56ed_u6625_xb7_u96ea:id13}}{\hyperref[\detokenize{p01_u6563_u6587/_u6bdb_u6cfd_u4e1c-_u6c81_u56ed_u6625_xb7_u96ea:id7}]{\sphinxcrossref{1.5   名家点评}}}

\end{itemize}

\end{itemize}
\end{sphinxShadowBox}

《沁园春·雪》是无产阶级革命家毛泽东创作的一首词。该词上片描写北国壮丽的雪景,纵横千万里,展示了大气磅礴、旷达豪迈的意境,抒发了词人对祖国壮丽河山的热爱。下片议论抒情,重点评论历史人物,歌颂当代英雄,抒发无产阶级要做世界的真正主人的豪情壮志。全词熔写景、议论和抒情于一炉,意境壮美,气势恢宏,感情奔放,胸襟豪迈,颇能代表毛泽东诗词的豪放风格。


\section{1.1   作品原文}
\label{\detokenize{p01_u6563_u6587/_u6bdb_u6cfd_u4e1c-_u6c81_u56ed_u6625_xb7_u96ea:id3}}
沁园春·雪1

北国风光,千里冰封,万里雪飘。望长城内外,惟余莽莽2;大河上下,顿失滔滔3。山舞银蛇,原驰蜡象4,欲与天公试比高。须晴日5,看红装素裹,分外妖娆6。

江山如此多娇,引无数英雄竞折腰7。惜秦皇汉武,略输文采8;唐宗宋祖9,稍逊风骚。一代天骄,成吉思汗,只识弯弓射大雕。俱往矣,数风流人物,还看今朝。


\section{1.2   词句注释}
\label{\detokenize{p01_u6563_u6587/_u6bdb_u6cfd_u4e1c-_u6c81_u56ed_u6625_xb7_u96ea:id4}}
1.沁园春:词牌名,又名“东仙”“寿星明”“洞庭春色”等。双调,一百十四字。前段十三句,四平韵;后段十二句,五平韵。

2.惟余:只剩下。余:有版本作“馀”。莽莽:即茫茫,白茫茫一片。形容空旷无际。

3.顿失:立刻失去。顿:顿时,立刻。滔滔:滚滚的波涛。

4.原驰蜡象:作者原注“原指高原,即秦晋高原”。驰:有版本作“驱”。蜡象:白色的象。

5.须:待、等到。

6.“看红装”二句:红日和白雪互相映照,看去好像装饰艳丽的美女裹着白色的外衣,格外娇媚。红装:身着艳丽服饰的美女。一作银装。妖娆(ráo):娇艳妩媚。

7.竞折腰:争着为江山奔走效劳。折腰:倾倒,躬着腰侍候。

8.“秦皇汉武”二句:是说秦皇汉武,功业甚盛,相比之下,文治方面的成就略有逊色。秦皇:秦始皇赢政,秦朝的创业皇帝。汉武:汉武帝刘彻,西汉第七位皇帝。略输:稍差。文采:本指辞藻、才华。这里引申为文治。

9.唐宗:唐太宗李世民,唐朝建立统一大业的皇帝。宋祖:宋太祖赵匡胤,宋朝的创业皇帝。

10.稍逊风骚:意近“略输文采”。逊:差。风骚:本指《诗经》里的《国风》和《楚辞》里的《离骚》,后来泛指文章辞藻。

11.天骄:汉时匈奴自称为“天之骄子”,以后泛称强盛的边地民族。

12.成吉思汗:元太祖铁木真统一蒙古后的尊称,意思是“强者之汗”。

13.“只识”句:是说只以武功见长。识:知道,懂得。雕:一种鹰类的大型猛禽,善飞难射,古代因用“射雕手”比喻高强的射手。{[}2{]}


\section{1.3   白话译文}
\label{\detokenize{p01_u6563_u6587/_u6bdb_u6cfd_u4e1c-_u6c81_u56ed_u6625_xb7_u96ea:id5}}
北方的风光,千里冰封冻,万里雪花飘。望长城内外,只剩下无边无际白茫茫一片;宽广的黄河上下,顿时失去了滔滔水势。山岭好像银白色的蟒蛇在飞舞,高原上的丘陵好像许多白象在奔跑,它们都想与老天爷比比高。要等到晴天的时候,看红艳艳的阳光和白皑皑的冰雪交相辉映,分外美好。

江山如此媚娇,引得无数英雄竞相倾倒。只可惜秦始皇、汉武帝,略差文学才华;唐太宗、宋太祖,稍逊文治功劳。称雄一世的人物成吉思汗,只知道拉弓射大雕。这些人物全都过去了,称得上能建功立业的英雄人物,还要看今天的人们。{[}3{]}


\section{1.4   创作背景}
\label{\detokenize{p01_u6563_u6587/_u6bdb_u6cfd_u4e1c-_u6c81_u56ed_u6625_xb7_u96ea:id6}}
1936年,红军组织东征部队,准备东渡黄河对日军作战。红军从子长县出发,挺进到清涧县高杰村的袁家沟一带时,部队在这里休整了16天。2月5日至20日,毛泽东在这里居住期间,曾下过一场大雪,长城内外白雪皑皑,隆起的秦晋高原,冰封雪盖。天气严寒,连平日奔腾咆哮的黄河都结了一层厚厚的冰,失去了往日的波涛。毛泽东当时住在农民白治民家中,深夜。见此情景,颇有感触,填写了这首词。《沁园春·雪》最早发表于1945年11月14日重庆《新民报晚刊》,后正式发表于《诗刊》1957年1月号。{[}4\sphinxhyphen{}5{]}


\section{1.5   名家点评}
\label{\detokenize{p01_u6563_u6587/_u6bdb_u6cfd_u4e1c-_u6c81_u56ed_u6625_xb7_u96ea:id7}}
近代诗人柳亚子《沁园春·雪》跋:毛润之沁园春一阕,余推为千古绝唱,虽东坡、幼安,犹瞠乎其后,更无论南唐小令、南宋慢词矣。


\chapter{1   老舍\sphinxhyphen{}济南的冬天}
\label{\detokenize{p01_u6563_u6587/_u8001_u820d-_u6d4e_u5357_u7684_u51ac_u5929:id1}}\label{\detokenize{p01_u6563_u6587/_u8001_u820d-_u6d4e_u5357_u7684_u51ac_u5929::doc}}
\begin{sphinxShadowBox}
\sphinxstyletopictitle{目录}
\begin{itemize}
\item {} 
\phantomsection\label{\detokenize{p01_u6563_u6587/_u8001_u820d-_u6d4e_u5357_u7684_u51ac_u5929:id11}}{\hyperref[\detokenize{p01_u6563_u6587/_u8001_u820d-_u6d4e_u5357_u7684_u51ac_u5929:id1}]{\sphinxcrossref{1   老舍\sphinxhyphen{}济南的冬天}}}
\begin{itemize}
\item {} 
\phantomsection\label{\detokenize{p01_u6563_u6587/_u8001_u820d-_u6d4e_u5357_u7684_u51ac_u5929:id12}}{\hyperref[\detokenize{p01_u6563_u6587/_u8001_u820d-_u6d4e_u5357_u7684_u51ac_u5929:id3}]{\sphinxcrossref{1.1   作品原文}}}

\item {} 
\phantomsection\label{\detokenize{p01_u6563_u6587/_u8001_u820d-_u6d4e_u5357_u7684_u51ac_u5929:id13}}{\hyperref[\detokenize{p01_u6563_u6587/_u8001_u820d-_u6d4e_u5357_u7684_u51ac_u5929:id4}]{\sphinxcrossref{1.2   写景手法}}}
\begin{itemize}
\item {} 
\phantomsection\label{\detokenize{p01_u6563_u6587/_u8001_u820d-_u6d4e_u5357_u7684_u51ac_u5929:id14}}{\hyperref[\detokenize{p01_u6563_u6587/_u8001_u820d-_u6d4e_u5357_u7684_u51ac_u5929:id5}]{\sphinxcrossref{1.2.1   1.基调统一,色彩和谐}}}

\item {} 
\phantomsection\label{\detokenize{p01_u6563_u6587/_u8001_u820d-_u6d4e_u5357_u7684_u51ac_u5929:id15}}{\hyperref[\detokenize{p01_u6563_u6587/_u8001_u820d-_u6d4e_u5357_u7684_u51ac_u5929:id6}]{\sphinxcrossref{1.2.2   2.景物层次,安排得当}}}

\item {} 
\phantomsection\label{\detokenize{p01_u6563_u6587/_u8001_u820d-_u6d4e_u5357_u7684_u51ac_u5929:id16}}{\hyperref[\detokenize{p01_u6563_u6587/_u8001_u820d-_u6d4e_u5357_u7684_u51ac_u5929:id7}]{\sphinxcrossref{1.2.3   3.远近大细,各得其宜}}}

\item {} 
\phantomsection\label{\detokenize{p01_u6563_u6587/_u8001_u820d-_u6d4e_u5357_u7684_u51ac_u5929:id17}}{\hyperref[\detokenize{p01_u6563_u6587/_u8001_u820d-_u6d4e_u5357_u7684_u51ac_u5929:id8}]{\sphinxcrossref{1.2.4   4.虚实手法,同时并用}}}

\item {} 
\phantomsection\label{\detokenize{p01_u6563_u6587/_u8001_u820d-_u6d4e_u5357_u7684_u51ac_u5929:id18}}{\hyperref[\detokenize{p01_u6563_u6587/_u8001_u820d-_u6d4e_u5357_u7684_u51ac_u5929:id9}]{\sphinxcrossref{1.2.5   5.适当点题,意义深远}}}

\item {} 
\phantomsection\label{\detokenize{p01_u6563_u6587/_u8001_u820d-_u6d4e_u5357_u7684_u51ac_u5929:id19}}{\hyperref[\detokenize{p01_u6563_u6587/_u8001_u820d-_u6d4e_u5357_u7684_u51ac_u5929:id10}]{\sphinxcrossref{1.2.6   6.山水画法,以大观小}}}

\end{itemize}

\end{itemize}

\end{itemize}
\end{sphinxShadowBox}


\section{1.1   作品原文}
\label{\detokenize{p01_u6563_u6587/_u8001_u820d-_u6d4e_u5357_u7684_u51ac_u5929:id3}}
对于一个在北平住惯的人,像我,冬天要是不刮风,便觉得是奇迹;济南的冬天是没有风声的。对于一个刚由伦敦回来的人,像我,冬天要能看得见日光,便觉得是怪事;济南的冬天是响晴的。自然,在热带的地方,日光是永远那么毒,响亮的天气,反有点叫人害怕。可是,在北中国的冬天,而能有温晴的天气,济南真得算个宝地。

设若单单是有阳光,那也算不了出奇。请闭上眼睛想:一个老城,有山有水,全在天底下晒着阳光,暖和安适地睡着,只等春风来把它们唤醒,这是不是个理想的境界?

小山整把济南围了个圈儿,只有北边缺着点口儿。这一圈小山在冬天特别可爱,好像是把济南放在一个小摇篮里,它们安静不动地低声地说:“你们放心吧,这儿准保暖和。”真的,济南的人们在冬天是面上含笑的。他们一看那些小山,心中便觉得有了着落,有了依靠。他们由天上看到山上,便不知不觉地想起:“明天也许就是春天了吧?这样的温暖,今天夜里山草也许就绿起来了吧?”就是这点幻想不能一时实现,他们也并不着急,因为有这样慈善的冬天,干啥还希望别的呢!

最妙的是下点小雪呀。看吧,山上的矮松越发的青黑,树尖上顶着一髻儿白花,好像日本看护妇。山尖全白了,给蓝天镶上一道银边。山坡上,有的地方雪厚点,有的地方草色还露着,这样,一道儿白,一道儿暗黄,给山们穿上一件带水纹的花衣;看着看着,这件花衣好像被风儿吹动,叫你希望看见一点更美的山的肌肤。等到快日落的时候,微黄的阳光斜射在山腰上,那点薄雪好像忽然害了羞,微微露出点粉色。就是下小雪吧,济南是受不住大雪的,那些小山太秀气!

古老的济南,城里那么狭窄,城外又那么宽敞,山坡上卧着些小村庄,小村庄的房顶上卧着点雪,对,这是张小水墨画,也许是唐代的名手画的吧。

那水呢,不但不结冰,倒反在绿萍上冒着点热气,水藻真绿,把终年贮蓄的绿色全拿出来了。天儿越晴,水藻越绿,就凭这些绿的精神,水也不忍得冻上,况且那些长枝的垂柳还要在水里照个影儿呢!看吧,由澄清的河水慢慢往上看吧,空中,半空中,天上,自上而下全是那么清亮,那么蓝汪汪的,整个的是块空灵的蓝水晶。这块水晶里,包着红屋顶,黄草山,像地毯上的小团花的灰色树影。这就是冬天的济南。{[}1{]}


\section{1.2   写景手法}
\label{\detokenize{p01_u6563_u6587/_u8001_u820d-_u6d4e_u5357_u7684_u51ac_u5929:id4}}

\subsection{1.2.1   1.基调统一,色彩和谐}
\label{\detokenize{p01_u6563_u6587/_u8001_u820d-_u6d4e_u5357_u7684_u51ac_u5929:id5}}
济南虽然地处北中国,但是冬天无大风而多日照,它在冬天最显著的气候特点是“温晴”(温暖晴朗)。文章紧紧抓住这一点,使笔下的种种景物跟这“温晴”天气紧密联系在一起,构成一幅温暖晴朗的济南冬天图景。文章写山,写水,写城,写人,都无不涂上一层温暖晴朗的色彩,就是写雪景,也仍然跟温暖有联系──因为暖和,所以“最妙的是下点小雪”;而同晴朗分不开──因为晴朗,所以有“等到快日落的时候,微黄的阳光斜射在山腰上,那点薄雪好像忽然害了羞,微微露出点粉色”的景致。

在文中,第二段主要写的是济南全景,第三、四段主要写的是济南的山色,第五段主要写的是济南的水上景色,那么,全文就是由这几幅互相联系而又相对独立的画图组成的长轴。而这幅长轴,也就靠这“温晴”的基调统一起来,给人以和谐一致的美感。


\subsection{1.2.2   2.景物层次,安排得当}
\label{\detokenize{p01_u6563_u6587/_u8001_u820d-_u6d4e_u5357_u7684_u51ac_u5929:id6}}
古老的济南,景色秀丽,素有“家家泉水,户户插柳”、“一城山色半城湖”的美誉。文章依照写景的先后层次,更好地把这些美好的景色展现于出来。文章首先鸟瞰全城,得其全貌(第二段),然后给人以那一城山色,雪后斜阳(第三、四段),最后才写那垂柳岸边,那“水不但不结冰,倒反在绿萍上冒着点热气”,而水藻越晴越绿的水上景色(第五段)。由大到小地写来,从山到水地写去,层次分明,脉络清晰。自然这是就各大层次来说的,各大层次的内部,又同中有异,如第二段的由写景而兼及写人,第三段的由写雪而兼及写晴,第五段的由写水面而兼及写天空。写来笔法活脱,不失参差错落之致。


\subsection{1.2.3   3.远近大细,各得其宜}
\label{\detokenize{p01_u6563_u6587/_u8001_u820d-_u6d4e_u5357_u7684_u51ac_u5929:id7}}
偌大的一个济南,在作者笔下,竟然可以放在一个由四面群山环抱而成的小小摇篮里,而水天一碧的宏伟景色,只不过是一块“空灵的蓝水晶”。这是景物的远者大者。再看,“树尖上顶着一髻儿白花,好像日本看护妇”,“水藻真绿,把终年贮蓄的绿色全拿出来了”。这是景物的近者细者。远景大景,使人视野开阔,顿感心旷神怡;近景小景,叫人近看谛听,更觉景象真切。而且远景大景,还可以冲破“不识庐山真面目,只缘身在此山中”的局限,而近景小景,又能够避免“只见树木不见森林”的弊病。古诗云:“远观山有色,近听水无声。”这是说的非远观不能看到高山居然有色,非近听无以觉出流水竟然无声。这说明,写景手法,远近大细,不可偏废。运用得宜,就可以兼收其效。

该文写景时,不但远近并用,大细兼行,而且往往是由近而远、由细而大,或由远而近、由大而细,写来衔接紧密,推进自然。比如第五段的写景,就是由近而远,由细而大的:先写水冒着点热气,再写水藻,再写垂柳,再写水面的上空以至于半空中、天空上。而第四段的写景,则是由远而近、由大而细的:先写城外,再写城外的山坡,再写山坡上的小村庄,再写小村庄的房顶上的雪。这种写法,既符合叙述的逻辑顺序,又适应读者的视觉需要。


\subsection{1.2.4   4.虚实手法,同时并用}
\label{\detokenize{p01_u6563_u6587/_u8001_u820d-_u6d4e_u5357_u7684_u51ac_u5929:id8}}
实写景物的形象,对景物描写来说,无疑是十分必要的,诸如文章中的“树尖上顶着一髻儿白花,好像日本看护妇”之类。但是,要不止于摹状,还要传神,就得更多地仰仗虚写的手法。因此,在作者笔下,冬天阳光照耀下的济南,就出现了“暖和安适地睡着,只等春风来把它们唤醒”的神情;一圈围城的小山,也就说出“你们放心吧,这儿准保暖和”的细语;薄雪会有“微微露出点粉色”的羞容;水藻会有“把终年贮蓄的绿色全拿出来了”的“精神”;而那水呢,对那水藻也就可以有一副“不忍得冻上”的和善心肠了。至于小雪覆盖不匀的山坡,要“给山们穿上一件带水纹的花衣”,“那些长枝的垂柳还要在水里照个影儿”,自然也是文章中虚写传神的佳句。


\subsection{1.2.5   5.适当点题,意义深远}
\label{\detokenize{p01_u6563_u6587/_u8001_u820d-_u6d4e_u5357_u7684_u51ac_u5929:id9}}
画之所以有题跋,原因之一是题跋可以使画本身蕴含的意义更为显豁。应该说,题跋是一幅画的一个有机的组成部分,虽然它并不是所画的景物的本身。同样,对所写的景物,作者出面直接点题,也是容许的,这些点明题旨的话,不是可有可无的。该文点题得法,寥寥数语,便收到画龙点睛的效果。比如说,文章在描写了小山雪景之后,突然掉转笔锋,作者以评论者的身份,说起点题话来:“就是下小雪吧,济南是受不住大雪的,那些小山太秀气!”这话,既可以说是在所描绘的画面之外,又可以说是在所描绘的画面之中,因为它是画面所本有而又有点不甚明了的。一经点出,济南下点小雪(不能是大雪)的妙处,也就跃然纸上了。

题不可不点,也不可滥点,本文点题恰到好处。最后一句“这就是冬天的济南”,令人读起来有意犹未尽、话犹未了之感,引发读者更深远的思考,这也许正是作者使文章戛然而止的原因吧。


\subsection{1.2.6   6.山水画法,以大观小}
\label{\detokenize{p01_u6563_u6587/_u8001_u820d-_u6d4e_u5357_u7684_u51ac_u5929:id10}}
描绘济南的大地,老舍先生所用的是“以大观小”的中国山水画的构图取景方法。作者展开想像的翅膀飞上济南的云天俯瞰大地,然后对济南大地作了简笔的写意描绘。画城,不画它的东西南北,“一个老城,有山有水,全在天底下晒着阳光,暖和安适地睡着,只等春风来把它们唤醒”(注:此句中的山是济南城中的山)。一些琐碎的细部都被略去了,画的只是冬天济南城秀美的睡态,留下充分的余地让读者去联想、想像,进行艺术的再创造。画山,不画它的上下左右,“小山整把济南围了个圈儿,只有北边缺着点口儿”。一起笔就抓住了景物的主要特征,紧接着就引导读者展开艺术的联想和想像:“这一圈小山在冬天特别可爱,好像是把济南放在一个小摇篮里,它们安静不动地低声地说:‘你们放心吧,这儿准保暖和。’”借这种联想、想像,使画面活灵飞动起来。画人,不画人的男女老少,不但如国画一样略去耳鼻眉目,连形体也完全略去,而只画了济南冬天人物情态的最主要的特征:“济南的人们在冬天是面上含笑的。”和城与山,浑然构成一幅完美的图画。


\chapter{1   艾青\sphinxhyphen{}大堰河——我的保姆}
\label{\detokenize{p01_u6563_u6587/_u827e_u9752-_u5927_u5830_u6cb3_u2014_u2014_u6211_u7684_u4fdd_u59c6:id1}}\label{\detokenize{p01_u6563_u6587/_u827e_u9752-_u5927_u5830_u6cb3_u2014_u2014_u6211_u7684_u4fdd_u59c6::doc}}
\begin{sphinxShadowBox}
\sphinxstyletopictitle{目录}
\begin{itemize}
\item {} 
\phantomsection\label{\detokenize{p01_u6563_u6587/_u827e_u9752-_u5927_u5830_u6cb3_u2014_u2014_u6211_u7684_u4fdd_u59c6:id7}}{\hyperref[\detokenize{p01_u6563_u6587/_u827e_u9752-_u5927_u5830_u6cb3_u2014_u2014_u6211_u7684_u4fdd_u59c6:id1}]{\sphinxcrossref{1   艾青\sphinxhyphen{}大堰河——我的保姆}}}
\begin{itemize}
\item {} 
\phantomsection\label{\detokenize{p01_u6563_u6587/_u827e_u9752-_u5927_u5830_u6cb3_u2014_u2014_u6211_u7684_u4fdd_u59c6:id8}}{\hyperref[\detokenize{p01_u6563_u6587/_u827e_u9752-_u5927_u5830_u6cb3_u2014_u2014_u6211_u7684_u4fdd_u59c6:id3}]{\sphinxcrossref{1.1   作品原文}}}

\item {} 
\phantomsection\label{\detokenize{p01_u6563_u6587/_u827e_u9752-_u5927_u5830_u6cb3_u2014_u2014_u6211_u7684_u4fdd_u59c6:id9}}{\hyperref[\detokenize{p01_u6563_u6587/_u827e_u9752-_u5927_u5830_u6cb3_u2014_u2014_u6211_u7684_u4fdd_u59c6:id4}]{\sphinxcrossref{1.2   创作背景}}}

\item {} 
\phantomsection\label{\detokenize{p01_u6563_u6587/_u827e_u9752-_u5927_u5830_u6cb3_u2014_u2014_u6211_u7684_u4fdd_u59c6:id10}}{\hyperref[\detokenize{p01_u6563_u6587/_u827e_u9752-_u5927_u5830_u6cb3_u2014_u2014_u6211_u7684_u4fdd_u59c6:id5}]{\sphinxcrossref{1.3   作品鉴赏}}}

\item {} 
\phantomsection\label{\detokenize{p01_u6563_u6587/_u827e_u9752-_u5927_u5830_u6cb3_u2014_u2014_u6211_u7684_u4fdd_u59c6:id11}}{\hyperref[\detokenize{p01_u6563_u6587/_u827e_u9752-_u5927_u5830_u6cb3_u2014_u2014_u6211_u7684_u4fdd_u59c6:id6}]{\sphinxcrossref{1.4   名家点评}}}

\end{itemize}

\end{itemize}
\end{sphinxShadowBox}


\section{1.1   作品原文}
\label{\detokenize{p01_u6563_u6587/_u827e_u9752-_u5927_u5830_u6cb3_u2014_u2014_u6211_u7684_u4fdd_u59c6:id3}}
大堰河——我的保姆

大堰河,是我的保姆。

她的名字就是生她的村庄的名字,

她是童养媳,

大堰河,是我的保姆。//

我是地主的儿子;

也是吃了大堰河的奶而长大了的

大堰河的儿子。

大堰河以养育我而养育她的家,

而我,是吃了你的奶而被养育了的,

大堰河啊,我的保姆。//

大堰河,今天我看到雪使我想起了你:

你的被雪压着的草盖的坟墓,

你的关闭了的故居檐头的枯死的瓦菲,

你的被典押了的一丈平方的园地,

你的门前的长了青苔的石椅,

大堰河,今天我看到雪使我想起了你。//

你用你厚大的手掌把我抱在怀里,抚摸我;

在你搭好了灶火之后,

在你拍去了围裙上的炭灰之后,

在你尝到饭已煮熟了之后,

在你把乌黑的酱碗放到乌黑的桌子上之后,

在你补好了儿子们的为山腰的荆棘扯破的衣服之后,

在你把小儿被柴刀砍伤了的手包好之后,

在你把夫儿们的衬衣上的虱子一颗颗地掐死之后,

在你拿起了今天的第一颗鸡蛋之后,

你用你厚大的手掌把我抱在怀里,抚摸我。//

我是地主的儿子,

在我吃光了你大堰河的奶之后,

我被生我的父母领回到自己的家里。

啊,大堰河,你为什么要哭?//

我做了生我的父母家里的新客了!

我摸着红漆雕花的家具,

我摸着父母的睡床上金色的花纹,

我呆呆地看着檐头的我不认得的“天伦叙乐”的匾,

我摸着新换上的衣服的丝的和贝壳的纽扣,

我看着母亲怀里的不熟识的妹妹,

我坐着油漆过的安了火钵的炕凳,

我吃着碾了三番的白米的饭,

但,我是这般忸怩不安!因为我

我做了生我的父母家里的新客了。//

大堰河,为了生活,

在她流尽了她的乳汁之后,

她就开始用抱过我的两臂劳动了;

她含着笑,洗着我们的衣服,

她含着笑,提着菜篮到村边的结冰的池塘去,

她含着笑,切着冰屑悉索的萝卜,

她含着笑,用手掏着猪吃的麦糟,

她含着笑,扇着炖肉的炉子的火,

她含着笑,背了团箕到广场上去,

晒好那些大豆和小麦,

大堰河,为了生活,

在她流尽了她的乳液之后,

她就用抱过我的两臂,劳动了。//

大堰河,深爱着她的乳儿;

在年节里,为了他,忙着切那冬米的糖,

为了他,常悄悄地走到村边的她的家里去,

为了他,走到她的身边叫一声“妈”,

大堰河,把他画的大红大绿的关云长

贴在灶边的墙上,

大堰河,会对她的邻居夸口赞美她的乳儿;

大堰河曾做了一个不能对人说的梦:

在梦里,她吃着她的乳儿的婚酒,

坐在辉煌的结彩的堂上,

而她的娇美的媳妇亲切的叫她“婆婆”

……//

大堰河,深爱着她的乳儿!

大堰河,在她的梦没有做醒的时候已死了。

她死时,乳儿不在她的旁侧,

她死时,平时打骂她的丈夫也为她流泪,

五个儿子,个个哭得很悲,

她死时,轻轻地呼着她的乳儿的名字,

大堰河,已死了,

她死时,乳儿不在她的旁侧。//

大堰河,含泪的去了!

同着四十几年的人世生活的凌侮,

同着数不尽的奴隶的凄苦,

同着四块钱的棺材和几束稻草,

同着几尺长方的埋棺材的土地,

同着一手把的纸钱的灰,

大堰河,她含泪的去了。//

这是大堰河所不知道的:

她的醉酒的丈夫已死去,

大儿做了土匪,

第二个死在炮火的烟里,

第三,第四,第五

在师傅和地主的叱骂声里过着日子。

而我,我是在写着给予这不公道的世界的咒语。

当我经了长长的漂泊回到故土时,

在山腰里,田野上,

兄弟们碰见时,是比六七年前更要亲密!

这,这是为你,静静地睡着的大堰河

所不知道的啊!//

大堰河,今天,你的乳儿是在狱里,

写着一首呈给你的赞美诗,

呈给你黄土下紫色的灵魂,

呈给你拥抱过我的直伸着的手,

呈给你吻过我的唇,

呈给你泥黑的温柔的脸颜,

呈给你养育了我的乳房,

呈给你的儿子们,我的兄弟们,

呈给大地上一切的,

我的大堰河般的保姆和她们的儿子,

呈给爱我如爱她自己的儿子般的大堰河。//

大堰河,

我是吃了你的奶而长大了的

你的儿子,

我敬你

爱你!

一九三三年一月十四日,雪朝


\section{1.2   创作背景}
\label{\detokenize{p01_u6563_u6587/_u827e_u9752-_u5927_u5830_u6cb3_u2014_u2014_u6211_u7684_u4fdd_u59c6:id4}}
1932年,诗人因加入左翼美术家联盟被捕,以“宣传与三民主义不相容主义”罪被判入狱6年。在狱中他写下了这首《大堰河——我的保姆》。{[}2{]}


\section{1.3   作品鉴赏}
\label{\detokenize{p01_u6563_u6587/_u827e_u9752-_u5927_u5830_u6cb3_u2014_u2014_u6211_u7684_u4fdd_u59c6:id5}}
《大堰河,我的保姆》是艾青的成名之作。这是一个地主阶级叛逆的儿子献给他的真正母亲——中国大地善良而不幸的普通农妇的颂歌。

这首诗感情真挚深切。诗中反复陈述:“大堰河,是我的保姆”,诗人是地主的儿子,长在“大堰河”的怀中,吮吸着她的乳汁,这不仅养育了诗人和身体,也养育了诗人的感情。诗人深深领受了她的爱,及至到了上学的年龄离开养母回到亲生父母身边的时候,他感到父母的陌生,更感到养母的对他的重要。养母正直、善良、朴素的品格影响了诗人的一生。这首诗从头到尾,始终围绕“我”与“她”的关系来写,他对大堰河深厚的感情,都表现在娓娓动情的陈述之中,他在监狱里,看见了雪就想到大堰河“被雪压着的草盖的坟墓”,想起她的故居园地,想起她对他的关怀和爱……于是他用他的深情的诗,表现了大堰河的具体劳作情景,也写了她心灵深处的感情波纹,就连她美丽的梦境,也同对乳儿的“幸福命运”的祝愿融合在一起。有了这样的真情,这样的心灵,才使这位劳动妇女形象更加崇高、完美,所以诗人要把热烈的颂扬,“呈给大地上一切的/我的大堰河般的保姆和他们的儿子/呈给爱我如爱她自己的儿子般的大堰河”。这样就使“大堰河”以某种象征意义,升华为永远与山河、村庄同在的人民的化身,或者说是中国农民的化身。

艾青在《大堰河,我的保姆》开始表现他诗作的艺术特色,他首先是从“感觉”出发,像印象派画家那么重视感觉和感受,而且注意主观情感对感觉的渗入与融合。并在二者的融合中产生出多层次的联想,创造出既是清晰的,又具有广阔象征意义的视觉形象。诗总是具体的、有着鲜明形象的,如这首诗写大堰河的劳作,写大堰河的笑,写大堰河的爱和死。都呈现可视可感的立体的意象符号附加形容。最后叠句排比旬的运用,如“呈给你黄土下紫色的灵魂/呈给你拥抱过我的直伸着的手/呈给你吻过我的唇。/呈给你泥黑的温柔的脸颜/呈给你蒜育我的乳房……”具体的描写,保证语言的形象性,这也是艾青诗的艺术魅力的奥秘所在,他后来的诗作,更自觉地将它发扬光大了。

这是一首献给保姆大堰河的诗篇。诗人叙述了这位普通中国妇女平凡而坎坷、不幸的一生,表达了对这位伟大母亲由衷的感恩之情。大堰河,也是千千万万中国母亲的代表,正是这片如同慈母一样宽阔的土地和这个伟大的祖国,尽管她受尽欺辱,满身疮痍,历尽沧桑,然而却永远不失母性和母爱伟大的光辉诗歌饱含深情,反复咏唱,如泣如诉。


\section{1.4   名家点评}
\label{\detokenize{p01_u6563_u6587/_u827e_u9752-_u5927_u5830_u6cb3_u2014_u2014_u6211_u7684_u4fdd_u59c6:id6}}
现代文学家茅盾:“用沉郁的笔调细写了乳娘兼女佣(《大堰河》)的生活痛苦”。(《中国现代文学管窥》)

中国作家协会会员张同吾:它像一颗光华熠熠的新星,出现在30年代的中国诗坛上;它以深沉隽永的情思,在广大读者的心田里镌刻着久远而常新的记忆。(《张同吾文集》){[}6{]}

现代文艺理论家、诗人胡风:“至于《大堰河——我的保姆》,在这里有了一个用乳汁用母爱喂养别人的孩子,用劳力用忠诚服侍别人的农妇的形象,乳儿的作者用着朴素的真实的言语对这形象呈诉了切切的爱心。在这里他提出了对于‘这不公道的世界’的诅咒,告白了他和被侮辱的兄弟们比以前‘更要亲密’。虽然全篇流着私情地温暖,但他和我们之间已没有了难越的界限了。”(《通三统:一种文学史实验》)


\chapter{1   茅盾\sphinxhyphen{}白杨礼赞}
\label{\detokenize{p01_u6563_u6587/_u8305_u76fe-_u767d_u6768_u793c_u8d5e:id1}}\label{\detokenize{p01_u6563_u6587/_u8305_u76fe-_u767d_u6768_u793c_u8d5e::doc}}
\begin{sphinxShadowBox}
\sphinxstyletopictitle{目录}
\begin{itemize}
\item {} 
\phantomsection\label{\detokenize{p01_u6563_u6587/_u8305_u76fe-_u767d_u6768_u793c_u8d5e:id5}}{\hyperref[\detokenize{p01_u6563_u6587/_u8305_u76fe-_u767d_u6768_u793c_u8d5e:id1}]{\sphinxcrossref{1   茅盾\sphinxhyphen{}白杨礼赞}}}
\begin{itemize}
\item {} 
\phantomsection\label{\detokenize{p01_u6563_u6587/_u8305_u76fe-_u767d_u6768_u793c_u8d5e:id6}}{\hyperref[\detokenize{p01_u6563_u6587/_u8305_u76fe-_u767d_u6768_u793c_u8d5e:id3}]{\sphinxcrossref{1.1   作品原文}}}

\item {} 
\phantomsection\label{\detokenize{p01_u6563_u6587/_u8305_u76fe-_u767d_u6768_u793c_u8d5e:id7}}{\hyperref[\detokenize{p01_u6563_u6587/_u8305_u76fe-_u767d_u6768_u793c_u8d5e:id4}]{\sphinxcrossref{1.2   词语注释}}}

\end{itemize}

\end{itemize}
\end{sphinxShadowBox}

《白杨礼赞》是现代作家茅盾于1941年所写的一篇散文。作者以西北黄土高原上“参天耸立,不折不挠,对抗着西北风”的白杨树,来象征坚韧、勤劳的北方农民,歌颂他们在民族解放斗争中的朴实、坚强和力求上进的精神,同时对于那些“贱视民众,顽固的倒退的人们”也投出了辛辣的嘲讽。文章立意高远,形象鲜明,结构严谨,语言简练。


\section{1.1   作品原文}
\label{\detokenize{p01_u6563_u6587/_u8305_u76fe-_u767d_u6768_u793c_u8d5e:id3}}
白杨树实在不是平凡的,我赞美白杨树!

汽车在望不到边际的高原上奔驰,扑入你的视野2的,是黄绿错综的一条大毡子。黄的是土,未开垦的处女土,几十万年前由伟大的自然力堆积成功的黄土高原的外壳;绿的呢,是人类劳力战胜自然的成果,是麦田。和风吹送,翻起了一轮一轮的绿波——这时你会真心佩服昔人所造的两个字“麦浪”,若不是妙手偶得,便确是经过锤炼的语言的精华。黄与绿主宰着,无边无垠,坦荡如砥3,这时如果不是宛若4并肩的远山的连峰提醒了你(这些山峰凭你的肉眼来判断,就知道是在你脚底下的),你会忘记了汽车是在高原上行驶。这时你涌起来的感想也许是“雄壮”,也许是“伟大”,诸如此类的形容词;然而同时你的眼睛也许觉得有点倦怠,你对当前的“雄壮”或“伟大”闭了眼,而另一种的味儿在你心头潜滋暗长5了——“单调”。可不是?单调,有一点儿吧?

然而刹那间,要是你猛抬眼看见了前面远远有一排——不,或者甚至只是三五株,一株,傲然地耸立,像哨兵似的树木的话,那你的恹恹6欲睡的情绪又将如何?我那时是惊奇地叫了一声的。

那就是白杨树,西北极普通的一种树,然而实在不是平凡的一种树。

那是力争上游的一种树,笔直的干,笔直的枝。它的干呢,通常是丈把高,像是加以人工似的,一丈以内绝无旁枝。它所有的丫枝呢,一律向上,而且紧紧靠拢,也像是加以人工似的,成为一束,绝无横斜逸出7。它的宽大的叶子也是片片向上,几乎没有斜生的,更不用说倒垂了;它的皮,光滑而有银色的晕圈8,微微泛出淡青色。这是虽在北方的风雪的压迫下却保持着倔强挺立的一种树。哪怕只有碗来粗细罢,它却努力向上发展,高到丈许,二丈,参天耸立,不折不挠,对抗着西北风。

这就是白杨树,西北极普通的一种树,然而决不是平凡的树!

它没有婆娑9的姿态,没有屈曲盘旋的虬枝10,也许你要说它不美丽,──如果美是专指“婆娑”或“横斜逸出”之类而言,那么白杨树算不得树中的好女子;但是它却是伟岸11,正直,朴质,严肃,也不缺乏温和,更不用提它的坚强不屈与挺拔,它是树中的伟丈夫!当你在积雪初融的高原上走过,看见平坦的大地上傲然挺立这么一株或一排白杨树,难道你觉得树只是树,难道你就不想到它的朴质,严肃,坚强不屈,至少也象征了北方的农民;难道你竟一点也不联想到,在敌后的广大土地上,到处有坚强不屈,就像这白杨树一样傲然挺立的守卫他们家乡的哨兵!难道你又不更远一点想到这样枝枝叶叶靠紧团结,力求上进的白杨树,宛然象征了今天在华北平原纵横决荡12用血写出新中国历史的那种精神和意志。

白杨不是平凡的树。它在西北极普遍,不被人重视,就跟北方农民相似;它有极强的生命力,磨折不了,压迫不倒,也跟北方的农民相似。我赞美白杨树,就因为它不但象征了北方的农民,尤其象征了今天我们民族解放斗争中所不可缺的朴质,坚强,以及力求上进的精神。

让那些看不起民众,贱视民众,顽固的倒退的人们去赞美那贵族化的楠木13(那也是直干秀颀14的),去鄙视这极常见,极易生长的白杨罢,但是我要高声赞美白杨树!

(原载《文艺阵地》月刊第6卷第3期,1941年3月10日出版)


\section{1.2   词语注释}
\label{\detokenize{p01_u6563_u6587/_u8305_u76fe-_u767d_u6768_u793c_u8d5e:id4}}
1.礼赞:崇敬和赞美。

2.视野:视力所及的范围。

3.坦荡如砥(dǐ):平坦得像磨刀石一样。

4.宛若:很像,简直就是。

5.潜滋暗长:暗暗地不知不觉地生长。滋,生长。

6.恹恹(yānyān):困倦的样子。

7.横斜逸出:意思是,(树枝)从树干的旁边斜伸出来。

8.晕(yùn)圈:模模糊糊的圈。

9.婆娑(suō):树木的枝叶随风飘荡,像舞蹈一样的姿态。

10.虬(qiú)枝:像龙一样盘旋的枝条。虬,传说中的一种龙。

11.伟岸:魁梧,高大。

12.纵横决荡:纵横驰骋,冲杀突击。

13.楠(nán)木:常绿乔木,木质坚实,是贵重的木材。

14.秀颀(qí):美而高。颀,高大的意思。


\chapter{1   郁达夫\sphinxhyphen{}古都的秋}
\label{\detokenize{p01_u6563_u6587/_u90c1_u8fbe_u592b-_u53e4_u90fd_u7684_u79cb:id1}}\label{\detokenize{p01_u6563_u6587/_u90c1_u8fbe_u592b-_u53e4_u90fd_u7684_u79cb::doc}}
\begin{sphinxShadowBox}
\sphinxstyletopictitle{目录}
\begin{itemize}
\item {} 
\phantomsection\label{\detokenize{p01_u6563_u6587/_u90c1_u8fbe_u592b-_u53e4_u90fd_u7684_u79cb:id5}}{\hyperref[\detokenize{p01_u6563_u6587/_u90c1_u8fbe_u592b-_u53e4_u90fd_u7684_u79cb:id1}]{\sphinxcrossref{1   郁达夫\sphinxhyphen{}古都的秋}}}
\begin{itemize}
\item {} 
\phantomsection\label{\detokenize{p01_u6563_u6587/_u90c1_u8fbe_u592b-_u53e4_u90fd_u7684_u79cb:id6}}{\hyperref[\detokenize{p01_u6563_u6587/_u90c1_u8fbe_u592b-_u53e4_u90fd_u7684_u79cb:id3}]{\sphinxcrossref{1.1   作品原文}}}

\item {} 
\phantomsection\label{\detokenize{p01_u6563_u6587/_u90c1_u8fbe_u592b-_u53e4_u90fd_u7684_u79cb:id7}}{\hyperref[\detokenize{p01_u6563_u6587/_u90c1_u8fbe_u592b-_u53e4_u90fd_u7684_u79cb:id4}]{\sphinxcrossref{1.2   词语注释}}}

\end{itemize}

\end{itemize}
\end{sphinxShadowBox}


\section{1.1   作品原文}
\label{\detokenize{p01_u6563_u6587/_u90c1_u8fbe_u592b-_u53e4_u90fd_u7684_u79cb:id3}}
秋天,无论在什么地方的秋天,总是好的;可是啊,北国的秋,却特别地来得清,来得静,来得悲凉。我的不远千里,要从杭州赶上青岛,更要从青岛赶上北平来的理由,也不过想饱尝一尝这“秋”,这故都的秋味。

江南,秋当然也是有的,但草木凋得慢,空气来得润,天的颜色显得淡,并且又时常多雨而少风;一个人夹在苏州上海杭州,或厦门香港广州的市民中间,混混沌沌地过去,只能感到一点点清凉,秋的味,秋的色,秋的意境与姿态,总看不饱,尝不透,赏玩不到十足。秋并不是名花,也并不是美酒,那一种半开、半醉的状态,在领略秋的过程上,是不合适的。

不逢北国之秋,已将近十余年了。在南方每年到了秋天,总要想起陶然亭(1)的芦花,钓鱼台(2)的柳影,西山(3)的虫唱,玉泉(4)的夜月,潭柘寺(5)的钟声。在北平即使不出门去吧,就是在皇城人海之中,租人家一椽(6)破屋来住着,早晨起来,泡一碗浓茶,向院子一坐,你也能看得到很高很高的碧绿的天色,听得到青天下驯鸽的飞声。从槐树叶底,朝东细数着一丝一丝漏下来的日光,或在破壁腰中,静对着像喇叭似的牵牛花(朝荣)的蓝朵,自然而然地也能够感觉到十分的秋意。说到了牵牛花,我以为以蓝色或白色者为佳,紫黑色次之,淡红色最下。最好,还要在牵牛花底,叫长着几根疏疏落落的尖细且长的秋草,使作陪衬。

北国的槐树,也是一种能使人联想起秋来的点缀。像花而又不是花的那一种落蕊,早晨起来,会铺得满地。脚踏上去,声音也没有,气味也没有,只能感出一点点极微细极柔软的触觉。扫街的在树影下一阵扫后,灰土上留下来的一条条扫帚的丝纹,看起来既觉得细腻,又觉得清闲,潜意识下并且还觉得有点儿落寞(7),古人所说的梧桐一叶而天下知秋(8)的遥想,大约也就在这些深沉的地方。

秋蝉的衰弱的残声,更是北国的特产,因为北平处处全长着树,屋子又低,所以无论在什么地方,都听得见它们的啼唱。在南方是非要上郊外或山上去才听得到的。这秋蝉的嘶叫,在北方可和蟋蟀耗子一样,简直像是家家户户都养在家里的家虫。

还有秋雨哩,北方的秋雨,也似乎比南方的下得奇,下得有味,下得更像样。

在灰沉沉的天底下,忽而来一阵凉风,便息列索落地下起雨来了。一层雨过,云渐渐地卷向了西去,天又晴了,太阳又露出脸来了,着(9)着很厚的青布单衣或夹袄的都市闲人,咬着烟管,在雨后的斜桥影里,上桥头树底下去一立,遇见熟人,便会用了缓慢悠闲的声调,微叹着互答着地说:

“唉,天可真凉了——”(这了字念得很高,拖得很长。)

“可不是吗?一层秋雨一层凉了!”

北方人念阵字,总老像是层字,平平仄仄起来(10),这念错的歧韵,倒来得正好。

北方的果树,到秋天,也是一种奇景。第一是枣子树,屋角,墙头,茅房边上,灶房门口,它都会一株株地长大起来。像橄榄又像鸽蛋似的这枣子颗儿,在小椭圆形的细叶中间,显出淡绿微黄的颜色的时候,正是秋的全盛时期,等枣树叶落,枣子红完,西北风就要起来了,北方便是沙尘灰土的世界,只有这枣子、柿子、葡萄,成熟到八九分的七八月之交,是北国的清秋的佳日,是一年之中最好也没有的GoldenDays(11)。

有些批评家说,中国的文人学士,尤其是诗人,都带着很浓厚的颓废的色彩,所以中国的诗文里,赞颂秋的文字的特别的多。但外国的诗人,又何尝不然?我虽则外国诗文念的不多,也不想开出帐来,做一篇秋的诗歌散文钞(12),但你若去一翻英德法意等诗人的集子,或各国的诗文的Anthology来(13),总能够看到许多并于秋的歌颂和悲啼。各著名的大诗人的长篇田园诗或四季诗里,也总以关于秋的部分,写得最出色而最有味。足见有感觉的动物,有情趣的人类,对于秋,总是一样地特别能引起深沉,幽远、严厉、萧索的感触来的。不单是诗人,就是被关闭在牢狱里的囚犯,到了秋天,我想也一定能感到一种不能自已的深情,秋之于人,何尝有国别,更何尝有人种阶级的区别呢?不过在中国,文字里有一个“秋士”(14)的成语,读本里又有着很普遍的欧阳子的《秋声》(15)与苏东坡的《赤壁赋》等,就觉得中国的文人,与秋和关系特别深了,可是这秋的深味,尤其是中国的秋的深味,非要在北方,才感受得到底。

南国之秋,当然也是有它的特异的地方的,比如廿四桥的明月,钱塘江的秋潮,普陀山的凉雾,荔枝湾(16)的残荷等等,可是色彩不浓,回味不永。比起北国的秋来,正像是黄酒之与白干,稀饭之与馍馍,鲈鱼之与大蟹,黄犬之与骆驼。

秋天,这北国的秋天,若留得住的话,我愿把寿命的三分之二折去,换得一个三分之一的零头。

一九三四年八月在北平


\section{1.2   词语注释}
\label{\detokenize{p01_u6563_u6587/_u90c1_u8fbe_u592b-_u53e4_u90fd_u7684_u79cb:id4}}
⑴陶然亭:位于北京城南,亭名出自白居易诗句“更待菊黄家酿熟,共君一醉一陶然”。

⑵钓鱼台:在北京阜成门外三里河,玉渊潭公园北面。

⑶西山:北京西郊群山的总称,是京郊名胜。

⑷玉泉:指玉泉山,是西山东麓支脉。

⑸潭柘寺:在北京西山,相传“寺址本在青龙潭上,有古柘千章,寺以此得名。”

⑹一椽:一间屋。椽,放在房檩上架着木板或瓦的木条。

⑺落寞:冷落,寂寞。

⑻梧桐一叶而天下知秋:《淮南子,说山》:“以小明大,见叶落而知岁之将暮。”《太平御览》卷二十四引用“一叶落而知天下秋”。

⑼着:穿(衣)。

⑽平平仄仄起来:意即推敲起字的韵律来。

⑾GoldenDays:英语中指”黄金般的日子”。

⑿钞:同“抄”。

⒀Anthology:英语中指”选集”。

⒁秋士:古时指到了暮年仍不得志的知识分子。

⒂欧阳子的《秋声》:指欧阳修的《秋声赋》。

⒃荔枝湾:位于广州城西。


\chapter{1   Hi,p02读书}
\label{\detokenize{p02_u8bfb_u4e66/Hello_uff0cp02_u8bfb_u4e66:hi-p02}}\label{\detokenize{p02_u8bfb_u4e66/Hello_uff0cp02_u8bfb_u4e66::doc}}
\begin{sphinxShadowBox}
\sphinxstyletopictitle{目录}
\begin{itemize}
\item {} 
\phantomsection\label{\detokenize{p02_u8bfb_u4e66/Hello_uff0cp02_u8bfb_u4e66:id2}}{\hyperref[\detokenize{p02_u8bfb_u4e66/Hello_uff0cp02_u8bfb_u4e66:hi-p02}]{\sphinxcrossref{1   Hi,p02读书}}}
\begin{itemize}
\item {} 
\phantomsection\label{\detokenize{p02_u8bfb_u4e66/Hello_uff0cp02_u8bfb_u4e66:id3}}{\hyperref[\detokenize{p02_u8bfb_u4e66/Hello_uff0cp02_u8bfb_u4e66:post}]{\sphinxcrossref{1.1   post}}}

\end{itemize}

\end{itemize}
\end{sphinxShadowBox}


\section{1.1   post}
\label{\detokenize{p02_u8bfb_u4e66/Hello_uff0cp02_u8bfb_u4e66:post}}

\chapter{1   水浒\sphinxhyphen{}宋江之绰号}
\label{\detokenize{p02_u8bfb_u4e66/_u6c34_u6d52-_u5b8b_u6c5f_u4e4b_u7ef0_u53f7:id1}}\label{\detokenize{p02_u8bfb_u4e66/_u6c34_u6d52-_u5b8b_u6c5f_u4e4b_u7ef0_u53f7::doc}}
\begin{sphinxShadowBox}
\sphinxstyletopictitle{目录}
\begin{itemize}
\item {} 
\phantomsection\label{\detokenize{p02_u8bfb_u4e66/_u6c34_u6d52-_u5b8b_u6c5f_u4e4b_u7ef0_u53f7:id3}}{\hyperref[\detokenize{p02_u8bfb_u4e66/_u6c34_u6d52-_u5b8b_u6c5f_u4e4b_u7ef0_u53f7:id1}]{\sphinxcrossref{1   水浒\sphinxhyphen{}宋江之绰号}}}

\end{itemize}
\end{sphinxShadowBox}

宋江是《水浒传》里边名号最多的一个,共有四个。

第一个是黑宋江。
因为他长得面黑,身材比较矮,这是就他的形体来讲的,其貌不扬。

第二个是孝义黑三郎。
讲的是他对待父母,讲究孝道,他的孝道贯穿到了他的思想当中,成为他思想的一个部分,并且是他的思想的一个很重要的支撑点。

第三个是及时雨。
讲的是他仗义疏财,扶危济困,这在后面他陆续和弟兄们交往中能够看得出来。

第四个是呼保义。
这个词,一直到现在,大家都无法把它解释清楚。有一种解释说,保义是南宋时候武官的一个称呼,叫保义郎。“保义”本是宋代最低的武官名,逐渐成了人人可用的自谦之词。“呼保义”这个词是动宾结构,宋江以“自呼保义”来表示谦虚,意思是说,自己是最低等的人。另外一种解释,说“保”,就是保持的保;“义”就是忠义的义,“保义”即保持忠义,呼的意思,就是大家都那样叫他。大体上说,呼保义这个词实际上讲的是宋江对待国家的态度,对待朝廷的态度,对待皇帝的态度。水浒传里有云:“呼群保义”。


\chapter{1   Hi,p03旅游}
\label{\detokenize{p03_u65c5_u6e38/Hello_uff0cp03_u65c5_u6e38:hi-p03}}\label{\detokenize{p03_u65c5_u6e38/Hello_uff0cp03_u65c5_u6e38::doc}}
\begin{sphinxShadowBox}
\sphinxstyletopictitle{目录}
\begin{itemize}
\item {} 
\phantomsection\label{\detokenize{p03_u65c5_u6e38/Hello_uff0cp03_u65c5_u6e38:id2}}{\hyperref[\detokenize{p03_u65c5_u6e38/Hello_uff0cp03_u65c5_u6e38:hi-p03}]{\sphinxcrossref{1   Hi,p03旅游}}}
\begin{itemize}
\item {} 
\phantomsection\label{\detokenize{p03_u65c5_u6e38/Hello_uff0cp03_u65c5_u6e38:id3}}{\hyperref[\detokenize{p03_u65c5_u6e38/Hello_uff0cp03_u65c5_u6e38:post}]{\sphinxcrossref{1.1   post}}}

\end{itemize}

\end{itemize}
\end{sphinxShadowBox}


\section{1.1   post}
\label{\detokenize{p03_u65c5_u6e38/Hello_uff0cp03_u65c5_u6e38:post}}

\chapter{1   五泄瀑布}
\label{\detokenize{p03_u65c5_u6e38/_u4e94_u6cc4_u7011_u5e03:id1}}\label{\detokenize{p03_u65c5_u6e38/_u4e94_u6cc4_u7011_u5e03::doc}}
\begin{sphinxShadowBox}
\sphinxstyletopictitle{目录}
\begin{itemize}
\item {} 
\phantomsection\label{\detokenize{p03_u65c5_u6e38/_u4e94_u6cc4_u7011_u5e03:id3}}{\hyperref[\detokenize{p03_u65c5_u6e38/_u4e94_u6cc4_u7011_u5e03:id1}]{\sphinxcrossref{1   五泄瀑布}}}

\end{itemize}
\end{sphinxShadowBox}

五泄构成了天然的山水画卷,素有“小雁荡”之称。

当地人称瀑布为洩,一水折为五级,叫“五洩”,正称“五泄”。

月笼轻纱第一泄,
双龙争壑第二泄,
珠帘风动第三泄,
烈马奔腾第四泄,
蛟龙出海第五泄。

五泄从青口进入,古人云:“五泄名山青口锁,到此看山山便可”。沿公路前行,路旁曲溪青流,远处便是叠石岩。壁立数十丈,层层叠叠如彩屏。

再前便是“五泄湖”的水库,弯弯曲曲,长2公里许,犹似一条绿色的绸带飘浮在群山之中,颇有富春山水的风采。

在游船中还可以观赏许多奇特的山石景观,夹岩洞为其中一景。当年夹岩洞下有夹岩寺,香火较旺,水库建成后,寺庙成为水底龙宫。夹岩洞恰好位于湖面之上,洞高16米,深20米,内曾供奉千手观音,外观幽暗莫测,颇具神秘色彩。

沿湖还可以观赏元宝峰、鹫鹰峰、仙桃峰、老僧峰等。

在天一碧码头登岩后,沿五泄溪北上,过遇龙桥,就是五泄禅寺。

继续沿溪往前,不多远,过竹林,便是奔腾而下的第五泄。沿山势而上,依次四泄,三泄,二泄,一泄,逐次趋缓。风景各具。

脚劲不错,还可以继续向上,登上山顶观峡谷。

沿着深谷清溪,可以转回五泄禅寺。


\chapter{1   Hi,p04财经}
\label{\detokenize{p04_u8d22_u7ecf/Hello_uff0cp04_u8d22_u7ecf:hi-p04}}\label{\detokenize{p04_u8d22_u7ecf/Hello_uff0cp04_u8d22_u7ecf::doc}}
\begin{sphinxShadowBox}
\sphinxstyletopictitle{目录}
\begin{itemize}
\item {} 
\phantomsection\label{\detokenize{p04_u8d22_u7ecf/Hello_uff0cp04_u8d22_u7ecf:id2}}{\hyperref[\detokenize{p04_u8d22_u7ecf/Hello_uff0cp04_u8d22_u7ecf:hi-p04}]{\sphinxcrossref{1   Hi,p04财经}}}
\begin{itemize}
\item {} 
\phantomsection\label{\detokenize{p04_u8d22_u7ecf/Hello_uff0cp04_u8d22_u7ecf:id3}}{\hyperref[\detokenize{p04_u8d22_u7ecf/Hello_uff0cp04_u8d22_u7ecf:post}]{\sphinxcrossref{1.1   post}}}

\end{itemize}

\end{itemize}
\end{sphinxShadowBox}


\section{1.1   post}
\label{\detokenize{p04_u8d22_u7ecf/Hello_uff0cp04_u8d22_u7ecf:post}}

\chapter{1   Hi,p05技术}
\label{\detokenize{p05_u6280_u672f/Hello_uff0cp05_u6280_u672f:hi-p05}}\label{\detokenize{p05_u6280_u672f/Hello_uff0cp05_u6280_u672f::doc}}
\begin{sphinxShadowBox}
\sphinxstyletopictitle{目录}
\begin{itemize}
\item {} 
\phantomsection\label{\detokenize{p05_u6280_u672f/Hello_uff0cp05_u6280_u672f:id2}}{\hyperref[\detokenize{p05_u6280_u672f/Hello_uff0cp05_u6280_u672f:hi-p05}]{\sphinxcrossref{1   Hi,p05技术}}}
\begin{itemize}
\item {} 
\phantomsection\label{\detokenize{p05_u6280_u672f/Hello_uff0cp05_u6280_u672f:id3}}{\hyperref[\detokenize{p05_u6280_u672f/Hello_uff0cp05_u6280_u672f:post}]{\sphinxcrossref{1.1   post}}}

\end{itemize}

\end{itemize}
\end{sphinxShadowBox}


\section{1.1   post}
\label{\detokenize{p05_u6280_u672f/Hello_uff0cp05_u6280_u672f:post}}

\chapter{1   Hi,p06历史}
\label{\detokenize{p06_u5386_u53f2/Hello_uff0cp06_u5386_u53f2:hi-p06}}\label{\detokenize{p06_u5386_u53f2/Hello_uff0cp06_u5386_u53f2::doc}}
\begin{sphinxShadowBox}
\sphinxstyletopictitle{目录}
\begin{itemize}
\item {} 
\phantomsection\label{\detokenize{p06_u5386_u53f2/Hello_uff0cp06_u5386_u53f2:id2}}{\hyperref[\detokenize{p06_u5386_u53f2/Hello_uff0cp06_u5386_u53f2:hi-p06}]{\sphinxcrossref{1   Hi,p06历史}}}
\begin{itemize}
\item {} 
\phantomsection\label{\detokenize{p06_u5386_u53f2/Hello_uff0cp06_u5386_u53f2:id3}}{\hyperref[\detokenize{p06_u5386_u53f2/Hello_uff0cp06_u5386_u53f2:post}]{\sphinxcrossref{1.1   post}}}

\end{itemize}

\end{itemize}
\end{sphinxShadowBox}


\section{1.1   post}
\label{\detokenize{p06_u5386_u53f2/Hello_uff0cp06_u5386_u53f2:post}}

\chapter{1   瓦岗寨之李密}
\label{\detokenize{p06_u5386_u53f2/_u74e6_u5c97_u5be8_u4e4b_u674e_u5bc6:id1}}\label{\detokenize{p06_u5386_u53f2/_u74e6_u5c97_u5be8_u4e4b_u674e_u5bc6::doc}}
\begin{sphinxShadowBox}
\sphinxstyletopictitle{目录}
\begin{itemize}
\item {} 
\phantomsection\label{\detokenize{p06_u5386_u53f2/_u74e6_u5c97_u5be8_u4e4b_u674e_u5bc6:id3}}{\hyperref[\detokenize{p06_u5386_u53f2/_u74e6_u5c97_u5be8_u4e4b_u674e_u5bc6:id1}]{\sphinxcrossref{1   瓦岗寨之李密}}}

\end{itemize}
\end{sphinxShadowBox}


\chapter{1   Hi,p07创投}
\label{\detokenize{p07_u521b_u6295/Hello_uff0cp07_u521b_u6295:hi-p07}}\label{\detokenize{p07_u521b_u6295/Hello_uff0cp07_u521b_u6295::doc}}
\begin{sphinxShadowBox}
\sphinxstyletopictitle{目录}
\begin{itemize}
\item {} 
\phantomsection\label{\detokenize{p07_u521b_u6295/Hello_uff0cp07_u521b_u6295:id2}}{\hyperref[\detokenize{p07_u521b_u6295/Hello_uff0cp07_u521b_u6295:hi-p07}]{\sphinxcrossref{1   Hi,p07创投}}}
\begin{itemize}
\item {} 
\phantomsection\label{\detokenize{p07_u521b_u6295/Hello_uff0cp07_u521b_u6295:id3}}{\hyperref[\detokenize{p07_u521b_u6295/Hello_uff0cp07_u521b_u6295:post}]{\sphinxcrossref{1.1   post}}}

\end{itemize}

\end{itemize}
\end{sphinxShadowBox}


\section{1.1   post}
\label{\detokenize{p07_u521b_u6295/Hello_uff0cp07_u521b_u6295:post}}

\chapter{1   风投的前生}
\label{\detokenize{p07_u521b_u6295/_u98ce_u6295_u7684_u524d_u751f:id1}}\label{\detokenize{p07_u521b_u6295/_u98ce_u6295_u7684_u524d_u751f::doc}}
\begin{sphinxShadowBox}
\sphinxstyletopictitle{目录}
\begin{itemize}
\item {} 
\phantomsection\label{\detokenize{p07_u521b_u6295/_u98ce_u6295_u7684_u524d_u751f:id3}}{\hyperref[\detokenize{p07_u521b_u6295/_u98ce_u6295_u7684_u524d_u751f:id1}]{\sphinxcrossref{1   风投的前生}}}

\end{itemize}
\end{sphinxShadowBox}


\chapter{1   Hi,p08写作}
\label{\detokenize{p08_u5199_u4f5c/Hello_uff0cp08_u5199_u4f5c:hi-p08}}\label{\detokenize{p08_u5199_u4f5c/Hello_uff0cp08_u5199_u4f5c::doc}}
\begin{sphinxShadowBox}
\sphinxstyletopictitle{目录}
\begin{itemize}
\item {} 
\phantomsection\label{\detokenize{p08_u5199_u4f5c/Hello_uff0cp08_u5199_u4f5c:id2}}{\hyperref[\detokenize{p08_u5199_u4f5c/Hello_uff0cp08_u5199_u4f5c:hi-p08}]{\sphinxcrossref{1   Hi,p08写作}}}
\begin{itemize}
\item {} 
\phantomsection\label{\detokenize{p08_u5199_u4f5c/Hello_uff0cp08_u5199_u4f5c:id3}}{\hyperref[\detokenize{p08_u5199_u4f5c/Hello_uff0cp08_u5199_u4f5c:post}]{\sphinxcrossref{1.1   post}}}

\end{itemize}

\end{itemize}
\end{sphinxShadowBox}


\section{1.1   post}
\label{\detokenize{p08_u5199_u4f5c/Hello_uff0cp08_u5199_u4f5c:post}}

\chapter{1   Hi,p09work}
\label{\detokenize{p09work/Hello_uff0cp09work:hi-p09work}}\label{\detokenize{p09work/Hello_uff0cp09work::doc}}
\begin{sphinxShadowBox}
\sphinxstyletopictitle{目录}
\begin{itemize}
\item {} 
\phantomsection\label{\detokenize{p09work/Hello_uff0cp09work:id2}}{\hyperref[\detokenize{p09work/Hello_uff0cp09work:hi-p09work}]{\sphinxcrossref{1   Hi,p09work}}}
\begin{itemize}
\item {} 
\phantomsection\label{\detokenize{p09work/Hello_uff0cp09work:id3}}{\hyperref[\detokenize{p09work/Hello_uff0cp09work:post}]{\sphinxcrossref{1.1   post}}}

\end{itemize}

\end{itemize}
\end{sphinxShadowBox}


\section{1.1   post}
\label{\detokenize{p09work/Hello_uff0cp09work:post}}

\chapter{Indices and tables}
\label{\detokenize{index:indices-and-tables}}\begin{itemize}
\item {} 
\DUrole{xref,std,std-ref}{search}

\end{itemize}



\renewcommand{\indexname}{索引}
\printindex
\end{document}