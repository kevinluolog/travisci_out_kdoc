%% Generated by Sphinx.
\def\sphinxdocclass{report}
\documentclass[letterpaper,12pt,english]{sphinxmanual}
\ifdefined\pdfpxdimen
   \let\sphinxpxdimen\pdfpxdimen\else\newdimen\sphinxpxdimen
\fi \sphinxpxdimen=.75bp\relax
%% turn off hyperref patch of \index as sphinx.xdy xindy module takes care of
%% suitable \hyperpage mark-up, working around hyperref-xindy incompatibility
\PassOptionsToPackage{hyperindex=false}{hyperref}

\PassOptionsToPackage{warn}{textcomp}

\catcode`^^^^00a0\active\protected\def^^^^00a0{\leavevmode\nobreak\ }
\usepackage{cmap}
\usepackage{xeCJK}
\usepackage{amsmath,amssymb,amstext}
\usepackage{polyglossia}
\setmainlanguage{english}



\setCJKmainfont{FZWeiBei-S03}


\usepackage[Sonny]{fncychap}
\ChNameVar{\Large\normalfont\sffamily}
\ChTitleVar{\Large\normalfont\sffamily}
\usepackage{sphinx}

\fvset{fontsize=\small}
\usepackage{geometry}

% Include hyperref last.
\usepackage{hyperref}
% Fix anchor placement for figures with captions.
\usepackage{hypcap}% it must be loaded after hyperref.
% Set up styles of URL: it should be placed after hyperref.
\urlstyle{same}
\addto\captionsenglish{\renewcommand{\contentsname}{目录}}

\usepackage{sphinxmessages}
\setcounter{tocdepth}{0}


%中文字体fontsize放大,kl+
\defaultCJKfontfeatures{Scale=2}
\usepackage{enumitem}
\setlistdepth{99}


\title{Hi post}
\date{2019 年 11 月 06 日}
\release{}
\author{kevinluo}
\newcommand{\sphinxlogo}{\vbox{}}
\renewcommand{\releasename}{}
\makeindex
\begin{document}

\pagestyle{empty}
\sphinxmaketitle
\pagestyle{plain}
\sphinxtableofcontents
\pagestyle{normal}
\phantomsection\label{\detokenize{index::doc}}



\chapter{1   Hi,p00其它}
\label{\detokenize{p00_u5176_u5b83/Hello_uff0cp00_u5176_u5b83:hi-p00}}\label{\detokenize{p00_u5176_u5b83/Hello_uff0cp00_u5176_u5b83::doc}}
\begin{sphinxShadowBox}
\sphinxstyletopictitle{目录}
\begin{itemize}
\item {} 
\phantomsection\label{\detokenize{p00_u5176_u5b83/Hello_uff0cp00_u5176_u5b83:id2}}{\hyperref[\detokenize{p00_u5176_u5b83/Hello_uff0cp00_u5176_u5b83:hi-p00}]{\sphinxcrossref{1   Hi,p00其它}}}
\begin{itemize}
\item {} 
\phantomsection\label{\detokenize{p00_u5176_u5b83/Hello_uff0cp00_u5176_u5b83:id3}}{\hyperref[\detokenize{p00_u5176_u5b83/Hello_uff0cp00_u5176_u5b83:post}]{\sphinxcrossref{1.1   post}}}

\end{itemize}

\end{itemize}
\end{sphinxShadowBox}


\section{1.1   post}
\label{\detokenize{p00_u5176_u5b83/Hello_uff0cp00_u5176_u5b83:post}}

\chapter{1   《张英-聪训斋语》《张廷玉-澄怀园语》合辑}
\label{\detokenize{p00_u5176_u5b83/_u300a_u5f20_u82f1-_u806a_u8bad_u658b_u8bed_u300b_u300a_u5f20_u5ef7_u7389-_u6f84_u6000_u56ed_u8bed_u300b_u5408_u8f91:id1}}\label{\detokenize{p00_u5176_u5b83/_u300a_u5f20_u82f1-_u806a_u8bad_u658b_u8bed_u300b_u300a_u5f20_u5ef7_u7389-_u6f84_u6000_u56ed_u8bed_u300b_u5408_u8f91::doc}}
\begin{sphinxShadowBox}
\sphinxstyletopictitle{目录}
\begin{itemize}
\item {} 
\phantomsection\label{\detokenize{p00_u5176_u5b83/_u300a_u5f20_u82f1-_u806a_u8bad_u658b_u8bed_u300b_u300a_u5f20_u5ef7_u7389-_u6f84_u6000_u56ed_u8bed_u300b_u5408_u8f91:id15}}{\hyperref[\detokenize{p00_u5176_u5b83/_u300a_u5f20_u82f1-_u806a_u8bad_u658b_u8bed_u300b_u300a_u5f20_u5ef7_u7389-_u6f84_u6000_u56ed_u8bed_u300b_u5408_u8f91:id1}]{\sphinxcrossref{1   《张英-聪训斋语》《张廷玉-澄怀园语》合辑}}}
\begin{itemize}
\item {} 
\phantomsection\label{\detokenize{p00_u5176_u5b83/_u300a_u5f20_u82f1-_u806a_u8bad_u658b_u8bed_u300b_u300a_u5f20_u5ef7_u7389-_u6f84_u6000_u56ed_u8bed_u300b_u5408_u8f91:id16}}{\hyperref[\detokenize{p00_u5176_u5b83/_u300a_u5f20_u82f1-_u806a_u8bad_u658b_u8bed_u300b_u300a_u5f20_u5ef7_u7389-_u6f84_u6000_u56ed_u8bed_u300b_u5408_u8f91:id3}]{\sphinxcrossref{1.1   《张英-聪训斋语》}}}
\begin{itemize}
\item {} 
\phantomsection\label{\detokenize{p00_u5176_u5b83/_u300a_u5f20_u82f1-_u806a_u8bad_u658b_u8bed_u300b_u300a_u5f20_u5ef7_u7389-_u6f84_u6000_u56ed_u8bed_u300b_u5408_u8f91:id17}}{\hyperref[\detokenize{p00_u5176_u5b83/_u300a_u5f20_u82f1-_u806a_u8bad_u658b_u8bed_u300b_u300a_u5f20_u5ef7_u7389-_u6f84_u6000_u56ed_u8bed_u300b_u5408_u8f91:id4}]{\sphinxcrossref{1.1.1   张英简介}}}

\item {} 
\phantomsection\label{\detokenize{p00_u5176_u5b83/_u300a_u5f20_u82f1-_u806a_u8bad_u658b_u8bed_u300b_u300a_u5f20_u5ef7_u7389-_u6f84_u6000_u56ed_u8bed_u300b_u5408_u8f91:id18}}{\hyperref[\detokenize{p00_u5176_u5b83/_u300a_u5f20_u82f1-_u806a_u8bad_u658b_u8bed_u300b_u300a_u5f20_u5ef7_u7389-_u6f84_u6000_u56ed_u8bed_u300b_u5408_u8f91:id5}]{\sphinxcrossref{1.1.2   有之四纲十二目如下:}}}

\item {} 
\phantomsection\label{\detokenize{p00_u5176_u5b83/_u300a_u5f20_u82f1-_u806a_u8bad_u658b_u8bed_u300b_u300a_u5f20_u5ef7_u7389-_u6f84_u6000_u56ed_u8bed_u300b_u5408_u8f91:id19}}{\hyperref[\detokenize{p00_u5176_u5b83/_u300a_u5f20_u82f1-_u806a_u8bad_u658b_u8bed_u300b_u300a_u5f20_u5ef7_u7389-_u6f84_u6000_u56ed_u8bed_u300b_u5408_u8f91:id6}]{\sphinxcrossref{1.1.3   【正文】张英-聪训斋语:}}}

\end{itemize}

\item {} 
\phantomsection\label{\detokenize{p00_u5176_u5b83/_u300a_u5f20_u82f1-_u806a_u8bad_u658b_u8bed_u300b_u300a_u5f20_u5ef7_u7389-_u6f84_u6000_u56ed_u8bed_u300b_u5408_u8f91:id20}}{\hyperref[\detokenize{p00_u5176_u5b83/_u300a_u5f20_u82f1-_u806a_u8bad_u658b_u8bed_u300b_u300a_u5f20_u5ef7_u7389-_u6f84_u6000_u56ed_u8bed_u300b_u5408_u8f91:id7}]{\sphinxcrossref{1.2   《张廷玉-澄怀园语》}}}
\begin{itemize}
\item {} 
\phantomsection\label{\detokenize{p00_u5176_u5b83/_u300a_u5f20_u82f1-_u806a_u8bad_u658b_u8bed_u300b_u300a_u5f20_u5ef7_u7389-_u6f84_u6000_u56ed_u8bed_u300b_u5408_u8f91:id21}}{\hyperref[\detokenize{p00_u5176_u5b83/_u300a_u5f20_u82f1-_u806a_u8bad_u658b_u8bed_u300b_u300a_u5f20_u5ef7_u7389-_u6f84_u6000_u56ed_u8bed_u300b_u5408_u8f91:id8}]{\sphinxcrossref{1.2.1   张廷玉简介}}}

\item {} 
\phantomsection\label{\detokenize{p00_u5176_u5b83/_u300a_u5f20_u82f1-_u806a_u8bad_u658b_u8bed_u300b_u300a_u5f20_u5ef7_u7389-_u6f84_u6000_u56ed_u8bed_u300b_u5408_u8f91:id22}}{\hyperref[\detokenize{p00_u5176_u5b83/_u300a_u5f20_u82f1-_u806a_u8bad_u658b_u8bed_u300b_u300a_u5f20_u5ef7_u7389-_u6f84_u6000_u56ed_u8bed_u300b_u5408_u8f91:id9}]{\sphinxcrossref{1.2.2   【卷一】}}}

\item {} 
\phantomsection\label{\detokenize{p00_u5176_u5b83/_u300a_u5f20_u82f1-_u806a_u8bad_u658b_u8bed_u300b_u300a_u5f20_u5ef7_u7389-_u6f84_u6000_u56ed_u8bed_u300b_u5408_u8f91:id23}}{\hyperref[\detokenize{p00_u5176_u5b83/_u300a_u5f20_u82f1-_u806a_u8bad_u658b_u8bed_u300b_u300a_u5f20_u5ef7_u7389-_u6f84_u6000_u56ed_u8bed_u300b_u5408_u8f91:id10}]{\sphinxcrossref{1.2.3   【卷二】}}}

\item {} 
\phantomsection\label{\detokenize{p00_u5176_u5b83/_u300a_u5f20_u82f1-_u806a_u8bad_u658b_u8bed_u300b_u300a_u5f20_u5ef7_u7389-_u6f84_u6000_u56ed_u8bed_u300b_u5408_u8f91:id24}}{\hyperref[\detokenize{p00_u5176_u5b83/_u300a_u5f20_u82f1-_u806a_u8bad_u658b_u8bed_u300b_u300a_u5f20_u5ef7_u7389-_u6f84_u6000_u56ed_u8bed_u300b_u5408_u8f91:id11}]{\sphinxcrossref{1.2.4   【卷三】}}}

\item {} 
\phantomsection\label{\detokenize{p00_u5176_u5b83/_u300a_u5f20_u82f1-_u806a_u8bad_u658b_u8bed_u300b_u300a_u5f20_u5ef7_u7389-_u6f84_u6000_u56ed_u8bed_u300b_u5408_u8f91:id25}}{\hyperref[\detokenize{p00_u5176_u5b83/_u300a_u5f20_u82f1-_u806a_u8bad_u658b_u8bed_u300b_u300a_u5f20_u5ef7_u7389-_u6f84_u6000_u56ed_u8bed_u300b_u5408_u8f91:id12}]{\sphinxcrossref{1.2.5   【卷四】}}}

\item {} 
\phantomsection\label{\detokenize{p00_u5176_u5b83/_u300a_u5f20_u82f1-_u806a_u8bad_u658b_u8bed_u300b_u300a_u5f20_u5ef7_u7389-_u6f84_u6000_u56ed_u8bed_u300b_u5408_u8f91:id26}}{\hyperref[\detokenize{p00_u5176_u5b83/_u300a_u5f20_u82f1-_u806a_u8bad_u658b_u8bed_u300b_u300a_u5f20_u5ef7_u7389-_u6f84_u6000_u56ed_u8bed_u300b_u5408_u8f91:id13}]{\sphinxcrossref{1.2.6   【附录】:}}}
\begin{itemize}
\item {} 
\phantomsection\label{\detokenize{p00_u5176_u5b83/_u300a_u5f20_u82f1-_u806a_u8bad_u658b_u8bed_u300b_u300a_u5f20_u5ef7_u7389-_u6f84_u6000_u56ed_u8bed_u300b_u5408_u8f91:id27}}{\hyperref[\detokenize{p00_u5176_u5b83/_u300a_u5f20_u82f1-_u806a_u8bad_u658b_u8bed_u300b_u300a_u5f20_u5ef7_u7389-_u6f84_u6000_u56ed_u8bed_u300b_u5408_u8f91:id14}]{\sphinxcrossref{1.2.6.1   【清朝名相张廷玉的祖上,平价粜米,周济穷苦的故事】}}}

\end{itemize}

\end{itemize}

\end{itemize}

\end{itemize}
\end{sphinxShadowBox}


\section{1.1   《张英-聪训斋语》}
\label{\detokenize{p00_u5176_u5b83/_u300a_u5f20_u82f1-_u806a_u8bad_u658b_u8bed_u300b_u300a_u5f20_u5ef7_u7389-_u6f84_u6000_u56ed_u8bed_u300b_u5408_u8f91:id3}}

\subsection{1.1.1   张英简介}
\label{\detokenize{p00_u5176_u5b83/_u300a_u5f20_u82f1-_u806a_u8bad_u658b_u8bed_u300b_u300a_u5f20_u5ef7_u7389-_u6f84_u6000_u56ed_u8bed_u300b_u5408_u8f91:id4}}
张英(1637—1708),字敦复,一字梦敦,号乐圃,又号倦圃翁,安徽桐城人,清朝人物,清代著名大臣张廷玉之父。

据《桐城县志》记载,康熙时期文华殿大学士兼礼部尚书张英的老家人与邻居吴家在宅基地问题上发生了争执,家人飞书京城,让张英打招呼“摆平”吴家。而张英回馈给老家人的是一首诗“一纸书来只为墙,让他三尺又何妨。长城万里今犹在,不见当年秦始皇。”家人见书,主动在争执线上退让了三尺,下垒建墙,而邻居吴氏也深受感动,退地三尺,建宅置院,六尺之巷因此而成。

在安徽安庆,流传着这样说法:“父子宰相府”、“五里三进士”、“隔河两状元”,指的是张英家庭。张英的儿子是大名鼎鼎的张廷玉,热播的影视剧《康熙大帝》、《康熙王朝》和《雍正王朝》中都有他的重要身影。张廷玉(1672-1755)为康熙时进士,官至保和殿大学士、军机大臣,乾隆时加太保,为官康、雍、乾三代,历半个世纪宝刀不老,为二千年封建官场之罕见。他有这样的官场作为,应该说是他得益于父辈、祖辈淡泊致远、克己清廉的家风。六尺巷在父辈那里宽了六尺,而在他的心胸中又宽了万丈,“心底无私天地宽”,无私的心胸因此坦荡而无垠!

张英、张廷玉父子是安徽省著名历史人物,二人在清初康、雍、乾盛世居官数十年,参与了平藩、收台湾、征漠北、摊丁入亩、改土归流、编棚入户等一系列大政方针的制订和实行。对稳定当时政局,统一国家,消弭满汉矛盾,强盛国计民生都起到了积极而重要的作用。二人为官清廉,人品端方,均官至一品大学士,是历史上著名的贤臣良相。同时二人还是史家公认的学者大儒。

在张家官运的背后是康雍乾三世,他们是清代有作为的皇帝,在有作为的皇帝身边溜须拍马,邀宠作奸是没有市场的,特别是雍正皇帝,为政不长,却厉行政改,一生勤于国政,“崇俭而不奢”,“毫无土木声色之娱”,张廷玉记录雍正:“上进膳,承命侍食,见一于饭颗并屑,未尝废置纤毫!!”饭粒落于桌上也不舍弃!在张家高官的背后,是威严自律的帝国皇帝。

当时的清王朝尽管帝王自律而有作为,但对汉人仍提防有加,防汉人颠覆政权,大兴文字狱,高官厚爵们也伴君如侍虎,如履薄冰。张家人低调屈身,也成自然,据载,张廷玉之子张若霭殿试得一甲第三名(探花),张廷玉跪求雍正换人,以留得名额给天下平民英才,因为张家已太多出人头地的机会了。雍正深为感动,将其子降级任用,可见张家谦卑公允之心,昭昭可鉴日月。

张英,张廷玉父子,均为清代名臣,位居宰相,安徽桐城人,张家在当时举业不断,名宦迭出,在京城、乡里誉称四起,如:“父子双宰相”、“三世得谥”、“六代翰林”、“自祖至玄十二人先后列侍从,跻鼎贵。玉堂谱里,世系蝉联,门阀之清华,殆可空前绝后而已”、“一门之内,祖父子孙先后相继入南书房,自康熙至乾隆,经数十年之久,此他氏所未有也”。影响之大,震惊朝野。张氏之所以如此兴盛,重要原因之一就是有良好的家训家风的教诲和熏陶。研读他们的家训,无疑对治家教子有重要的借鉴作用。

【另注】:《聪训斋语》的作者张英,清安徽桐城人,1637年出生,1708年去世。字敦复,号乐圃。康熙六年考上进士,授编修官,历升至文华殿大学士兼礼部尚书。居官勤俭谨慎,对民生疾苦、四方水旱知无不言,深获皇上倚重。曾受命总裁《清一统志》、《国史方略》、《渊鉴类函》、《政治典训》等书。其他许多典诰之文,亦尝出其手。生平酷好看山种树,以老病辞官,卒諡文端。著有《恒产琐言》、《聪训斋语》,谆谆以务本力田、随分知足告诫子弟,另有《易书衷论》、《笃素堂文集》等书。


\subsection{1.1.2   有之四纲十二目如下:}
\label{\detokenize{p00_u5176_u5b83/_u300a_u5f20_u82f1-_u806a_u8bad_u658b_u8bed_u300b_u300a_u5f20_u5ef7_u7389-_u6f84_u6000_u56ed_u8bed_u300b_u5408_u8f91:id5}}
一立品纲——戒嬉戏、慎威仪、谨言语。

二读书纲——温经书、精举业、学楷字。

三养身纲——谨起居、慎寒暑、节用度。

四择友纲——谢酬应、省宴集、寡交游。


\subsection{1.1.3   【正文】张英-聪训斋语:}
\label{\detokenize{p00_u5176_u5b83/_u300a_u5f20_u82f1-_u806a_u8bad_u658b_u8bed_u300b_u300a_u5f20_u5ef7_u7389-_u6f84_u6000_u56ed_u8bed_u300b_u5408_u8f91:id6}}
人心至灵至动,不可过劳,亦不可过逸,惟读书可以养之。书卷乃养心第一妙物。闲适无事之人,镇日不观书,则起居出入,身心无所栖泊,耳目无所安顿,势必心意颠倒,妄想生嗔。处逆境不乐,处顺境亦不乐。每见人栖栖皇皇,觉举动无不碍者,此必不读书之人也。

富贵贫贱,总难称意,知足即为称意;山水花竹,无恒主人,得闲便是主人。大约富贵人役于名利,贫贱人役于饥寒,总无闲情及此,惟付之浩叹耳。

古人以“眠、食”二者为养生之要务。脏腑肠胃,常令宽舒有余地,则真气得以流行而疾病少。“予从饱食,病安得入?”燔炙熬煎香甘肥腻之物,最悦口而不宜于肠胃。彼肥腻易于粘滞,积久则腹痛气塞,寒暑偶侵,则疾作矣。食忌多品,一席之间,遍食水陆,浓淡杂进,自然损脾;安寝,乃人生最乐,古人有言:不觅仙方觅睡方。冬夜以二鼓为度,暑月以一更为度。每笑人长夜酣饮不休,谓之消夜,夫人终日劳劳,夜则宴息,是极有味,何以消遣为?冬夏,皆当以日出而起,于夏尤宜。天地清旭之气,最为爽神,失之,甚为可惜。予山居颇闲,暑月,日出则起,收水草清香之味,莲方敛而未开,竹含露而犹滴,可谓至快!日长漏永,不妨午睡数刻,睡足而起,神清气爽;居家最宜早起,倘日高客至,僮则垢面,婢且蓬头,庭除未扫,灶突犹寒,大非雅事。

人家僮仆,最多不宜多畜,但有得力二三人,训谕有方,使令得宜,未尝不得兼人之用。太多则彼此相诿,恩养必不能周,教训亦不能及,反不得其力;吾辈居家居宦,皆简静守理,不为暗昧之事;山中耕田锄圃之仆,乃可为宝,其人无奢望,无机智,不为主人敛怨,彼纵不遵约束,不过懒惰、愚蠢之小过,不必加意防闲,岂不为清闲之一助哉?

俭于饮食,可以养脾胃;俭于嗜欲,可以聚精神;俭于言语,可以养气息非;俭于交游,可以择友寡过;俭于酬酢,可以养身息劳;俭于夜坐,可以安神舒体;俭于饮酒,可以清心养德;俭于思虑,可以蠲烦去扰;白香山诗云:“我有一言君记取,世间自取苦人多。”;人常和悦,则心气冲而五脏安,昔人所谓养欢喜神。日间办理公事,每晚家居,必寻可喜笑之事,与客纵谈,掀髯大笑,以发舒一日劳顿郁结之气;砚以世计,墨以时计,笔以日计,动静之分也。静之义有二:一则身不过劳,一则心不轻动。

万事做到极精妙处,无有不圆者。人之一身,与天时相应,大约三四十以前,是夏至前,凡事渐长;三四十以后,是夏至后,凡事渐衰,中间无一刻停留。中间盛衰关头,无一定时候,大概在三四十之间,观于须发可见:其衰缓者,其寿多;其衰急者,其寿寡。人身不能不衰,先从上而下者,多寿,故古人以早脱顶为寿征,先从下而上者,多不寿,故须发如故而脚软者难治;凡人家道亦然,决无中立之理,如一树之花,开到极盛,便是摇落之期。(注:家道是否如此,不论,爱后面一句)

予怪世人于古人诗文集不知爱,而宝其片纸只字,为大惑也。余昔在龙眠,苦于无客为伴,日则步于空潭碧涧、长松茂竹之侧,夕则掩关读苏陆诗,以二鼓为度,烧烛焚香,煮茶延两君子于坐,与之相对,如见其容貌须眉然。诗云:“架头苏陆有遗书,特地携来共索居。日与两君同卧起,人间何客得胜渠。”(渠:他)良非解嘲语也。

门无杂宾,大约门下奔走之客,有损无益。

人生适意之事有三:曰贵,曰富,曰多子孙。然是三者,善处之则为富,不善处之则足为累。高位者,责备之地,忌嫉之门,怨尤之府,利害之关,忧患之窟,劳苦之薮,谤讪之的,攻击之场,古之智人往往望而止步;夫人厚积则必经营布置,生息防守,其劳不可胜言:则必有亲戚之请求,贫穷之怨望,僮仆之奸骗,大而盗贼之劫取,小而穿窬之鼠窃,经商之亏折,行路之失脱,田禾之灾伤,攘夺之争讼,子弟之浪费。种种之苦,贫者不知,惟富厚者兼而有之。人能各富之为累,则取之当廉,而不必厚积以招怨;至子孙之累尤多矣,少小则有疾病之虑,稍长则有功名之虑,浮奢不善治家之虑,纳交匪类之虑,一离膝下,则有道路寒暑饥渴之虑,以至由子而孙,展转无穷,更无底止。

予之立训,更无多言,止有四语:读书者不贱,守田者不饥,积德者不倾,择交者不败。虽至寒苦之人,但能读书为文,必使人钦敬,不敢忽视。其人德性,亦必温和,行事决不颠倒,不在功名之得失,遇合之迟速也。

人生必厚重沉静,而后为载福之器。敦厚谦谨,慎言守礼,不可与寒士同一般感慨欷嘘,放言高论,怨天尤人,庶不为造物鬼神所呵责也。

乡里间荷担负贩及佣工小人,切不可取其便宜,此种人所争不过数文,我辈视之甚轻,而彼之含怨甚重。每有愚人见省得一文,以为得计,而不知此种人心忿口碑,所损实大也。待下我一等之人,言语辞气最为要紧,此事甚不费钱,然彼人受之,同于实惠,只在精神照料得来,不可惮烦。

读书固所以取科名,继家声,然亦使人敬重;每见仕宦显赫之家,其老者或退或故,而其家索然者,其后无读书之人也,其家郁然者,其后有读书之人也;父母之爱子,第一望其康宁,第二冀其成名,第三愿其保家。《语》曰:“父母惟其疾之忧。”夫子以此答武伯之问孝,至哉斯言!安其身以安父母之心,孝莫大焉。养身之道,一在谨嗜欲,一在慎饮食,一在慎忿怒,一在慎寒暑,一在慎思索,一在慎烦劳。吾贻子孙,不过瘠田数处耳,且甚荒芜不治,水旱多虞。岁入之数,谨足以免饥寒,畜妻子而已,一件儿戏事做不得,一件高兴事做不得;人生豪侠周密之名至不易副。事事应之,一事不应,遂生嫌怨,人人周之,一人不周,便存形迹,若平素俭啬,见谅于人,省无穷物力,少无穷嫌怨,不亦至便乎?。

人生二十内外,渐远于师保之严,未跻于成人之列,此时知识大开,性情未定,父师之训不能入,即妻子之言亦不听,惟朋友之言,甘如醴而芳若兰,脱有一淫朋匪友,阑入其侧,朝夕浸灌,鲜有不为其所移者;(坏)朋友,则直以不识其颜面,不知其姓名为善。比之毒草哑泉更当远避。

楷书如坐如立,行书如行,草书如奔。

法昭禅师偈云:“同气连枝各自荣,些些言语各伤情。一回相见一回老,能得几时为弟兄?”词意蔼然,足以启人友于之爱。然予尝谓人伦有五,而兄弟相处之日最长。

世人只因不知命,不安命,生出许多劳扰;(君子)修身以俟之(指机遇);注:安命则心安言诚,有一颗平常心,反而事事办得更好。

余家训有云:“保家莫如择友。”盖痛心疾首其言之也!汝辈但于至戚中,观其德性谨厚,好读书者,交友两三人足矣!且势利言之,则有酒食之费、应酬之扰,一遇婚丧有无,则有资给贷之事。甚至有争讼外侮,则又有关说救援之事。平昔既与之契密,临事却之,必生怨毒反唇。故余以为宜慎之于始也;昔人有戒:“饭不嚼便咽,路不看便走,话不想便说,事不思便做。”予益之曰:“友不择便交,气不忍不便动,财不审便取,衣不慎便脱。”

学字当专一。择古人佳帖或时人墨迹与已笔路相近者,专心学之,若朝更夕改,见异思迁,鲜有得成者。若体格不匀净而遽讲流动,失其本矣!学字忌飞动草率,大小不匀,而妄言奇古磊落,终无进步矣。

读文不必多,择其精纯条畅,有气局词华者,多则百篇,少则六十篇。神明与之浑化,始为有益。若贪多务博,过眼辄忘,及至作时,则彼此不相涉,落笔仍是故吾,所以思常窒而不灵,词常窘而不裕,意常枯而不润。

人能处心积虑,一言一动皆思益人,而痛戒损人,则人望之若鸾凤,宝之如参苓。必为天地所佑,鬼神之所服,而享有多福矣!

凡读书,二十岁以前所读之书与二十岁以后所读之书迥异。幼年知识未开,天真纯固,所读者虽久不温习,偶尔提起,尚可数行成诵。若壮年所读,经月则忘,必不能持久。故六经、秦汉之文,词语古奥,必须幼年读。长壮后,虽倍蓰其功,终属影响。

自八岁至二十岁,中间岁月无多,安可荒弃或读不急之书?此时,时文固不可不读,亦须择典雅醇正、理纯辞裕、可历二三十年无弊者读之。若朝华夕落、浅陋无识、诡僻失体、取悦一时者,安可以珠玉难换之岁月而读此无益之文?何如诵得《左》、《国》一两篇及东西汉典贵华腴之文数篇,为终身之用之宝乎?

古人之书,安可尽读?但我所已读者决不轻弃。得尺则尺,得寸则寸。毋贪多,毋贪名,但求读一篇,必可以背诵。然后思通其义蕴,而运用之于手腕之下,如此则才气自然发越。若曾读此书,而全不能举其词,谓之“画饼充饥”。能举其词而不能运用,谓之“食物不化”。

深恼人读时文累千累百而不知理会,于身心毫无裨益。夫能理会,则数十篇百篇已足,焉用如此之多?不能理会,则读数千篇与不读一字等。徒使精神聩乱,临文捉笔,依旧茫然,不过胸中旧套应副,安有名理精论、佳词妙句,奔汇于笔端乎?古人云:“读生文不如玩熟文。必以我之精神,包乎此一篇之外,以我之心思,入乎此一篇之中。幼年当专攻举业,以为立身之本。

世家子弟,其修行立名之难,较寒士百倍。何以故?人之当面待之者,万不能如寒士之古道:小有失检,谁肯面斥其非?微有骄盈,谁肯深规其过?幼而骄惯,为亲戚之所优容;长而习成,为朋友之所谅恕.

我愿汝曹常以席丰履盛为可危、可虑、难处、难全之地,勿以为可喜、可幸、易安、易逸之地;终身让路,不失尺寸,自古祗闻“忍”与“让”,足以消无穷之灾悔,未闻“忍”与“让”,翻以酿后来之祸患也,欲行忍认之道,先须从小事做起。余曾署刑部事五十日,见天下大讼大狱,多从极小事起。君子敬小慎微,凡事只从小处了。余行年五十余,生平未尝多受小人之侮,只有一善策,能转弯早耳。每思天下事,受得小气,则不至于受大气,吃得小亏,则不至于吃大亏,此生平得力之处。凡事最不可想占便宜,便宜者,天下人所共争也,我一人据之,则怨萃于我矣,我失便宜,则众怨消矣。故终身失便宜,乃终身得便宜也。

座右箴:立品、读书、养身、择友。右四纲。戒嬉戏,慎威仪;谨言语,温经书;精举业,学楷字;谨起居,慎寒暑;节用度,谢酬;省宴集,寡交游。右十二目。

子弟自十七八以至廿三四,实为学业成废之关。盖自初入学至十五六,父师以童子视之,稍知训子者,断不忍听其废业。惟自十七八以后,年渐长,气渐骄,渐有朋友,渐有室家,嗜欲渐广。父母见其长成,师傅视为侪辈。德性未坚,转移最易;学业未就,蒙昧非难。幼年所习经书,此时皆束高阁。酬应交游,侈然大雅。博弈高会,自诩名流。转盼廿五六岁,儿女累多,生计迫蹙,蹉跎潦倒,学殖荒落。予见人家子弟半途而废者,多在此五六年中,弃幼学之功,贻终身之累,盖辙相踵也。汝正当此时,离父母之侧,前言诸弊,事事可虑。为龙为蛇,为虎为鼠,分于一念,介在两歧,可不慎哉!可不畏哉!

读书须明窗净几,案头不可多置书;作文以握管之人为大将,以精熟墨卷百篇为练兵,以杂读时艺为散卒。

天子知俭,则天下足,一人知俭,则一家足。且俭非止节啬财用己也。俭于言语,则元气藏而怨尤寡;则于交游,则匪类远,俭于酬酢,则岁月宽而本业修,俭于书札,则后患寡,俭于嬉游,则学业进;人生俭啬之名,可受而不必避,世俗每以为耻,不知此名一噪,则人绝觊觎之想。偶有所用,人即德之;保家莫如择友,多则二人,少则一人,断无目前良友,遂可得十数人之理!平时既简于应酬,有事可以请教。

惟田产房屋二者可恃以久远,以二者较之,房舍又不如田产。

今人家子弟,鲜衣怒马,恒舞酣歌。一裘之费动至数十金,一席之费动至数金。不思吾乡十余年来谷贱,竭十余石谷,不足供一筵,竭百余石谷,不足供一衣。安知农家作苦,终年沾衣涂足,岂易得此百石?(今天依然如此)

古人之意,全在小处节俭,大处之不足,由于小处之不谨,月计之不足,由于每日之用过多也。

子弟有二三千金之产,方能城居。若千金以下之业,则断不可城居矣!

古人有言,扫地焚香,清福已具。其有福者,佐以读书;其无福者,便生他想。旨哉斯言,予所深赏!且从来拂意之事,自不读书者见之,似为我所独遭,极其难堪,不知古人拂意之事有百倍于此者,特不细心体验耳! 即如东坡先生,殁后遭逢高孝,文字始出,而当时之忧谗畏讥,困顿转徙潮惠之间,苏过跣足涉水,居近牛栏,是何如境界?又如白香山之无嗣,陆放翁之忍饥,皆载在书卷,彼独非千载闻人,而所遇皆如此? 诚一平心静观,则人间拂意之事,可以涣然冰释。若不读书,则但见我所遭甚苦,而无穷怨尤嗔忿之心,烧灼不宁,其苦为何如耶?且富盛之事,古人亦有之,炙手可热,转眼皆空。故读书可以增长道心,为颐养第一事也!

【附录】:


\section{1.2   《张廷玉-澄怀园语》}
\label{\detokenize{p00_u5176_u5b83/_u300a_u5f20_u82f1-_u806a_u8bad_u658b_u8bed_u300b_u300a_u5f20_u5ef7_u7389-_u6f84_u6000_u56ed_u8bed_u300b_u5408_u8f91:id7}}

\subsection{1.2.1   张廷玉简介}
\label{\detokenize{p00_u5176_u5b83/_u300a_u5f20_u82f1-_u806a_u8bad_u658b_u8bed_u300b_u300a_u5f20_u5ef7_u7389-_u6f84_u6000_u56ed_u8bed_u300b_u5408_u8f91:id8}}
张廷玉(1672年10月29日—1755年4月30日),字衡臣,号砚斋,安徽桐城人。清朝杰出政治家,大学士张英次子。

康熙三十九年(1700年)进士,改庶吉士,授检讨,入值南书房,进入权力中枢。康熙朝,官至刑部左侍郎,整饬吏治。雍正帝即位后,历任礼部尚书、户部尚书、吏部尚书,拜保和殿大学士(内阁首辅)、首席军机大臣等职,完善了军机处制度。乾隆帝即位后,君臣渐生嫌疑,晚景凄凉,致仕归家。乾隆二十年(1755年),卒于家中,年八十四,谥号“文和”,配享太庙,是整个清朝唯一一个配享太庙的汉臣。

张廷玉先后任《亲征平定朔北方略》纂修官,《省方盛典》、《清圣祖实录》副总裁官,《明史》、《四朝国史》、《大清会典》、《世宗实录》总裁官。


\subsection{1.2.2   【卷一】}
\label{\detokenize{p00_u5176_u5b83/_u300a_u5f20_u82f1-_u806a_u8bad_u658b_u8bed_u300b_u300a_u5f20_u5ef7_u7389-_u6f84_u6000_u56ed_u8bed_u300b_u5408_u8f91:id9}}
一、凡人得一爱重之物,必思置之善地,以保护之。至于心,乃吾心之至宝也,一念善,是即置之安处矣;一念恶,是即置之危地矣。奈何以吾身之至宝使之舍安而就危乎?亦弗思之甚矣。

二、一语而干天地之和,一事而折生平之福,当时时留心体察,不可于细微处忽之。

三、昔我文端公时时以知命之学训子孙,晏闲之时则诵论语曰:不知命,无以为君子也。盖穷通得失,天命既定,人岂能违?彼营营扰扰,趋利避害者,徒劳心力坏品行耳,究何能增减毫末哉!先兄宫詹公,习闻庭训,是以主试山左,即以不知命一节为题,惜乎能觉悟之人少也。

四、周易曰:吉人之辞寡,可见多言之人即为不吉,不吉则凶矣。趋吉避凶之道只在矢口间,朱子云:祸从口出。此言与周易相表里,黄山谷曰:万言万当,不如一默。当终身诵之。

五、一言一动,常思有益于人,惟恐有损于人。不惟积德,亦是福相。

六、文端公对联曰:万类相感,以诚造物,最忌者巧。又曰:保家莫如择友,求名莫如读书。姚端恪公对联曰:常觉胸中生意满,须知世上苦人多。又虚直斋日记曰:我心有不快,而以戾气加人可乎?我事有未暇,而以缓人之急可乎?均当奉为座右铭。

七、天下之道,宽则能容,能容则物安,而己亦适。虽然宽之道亦难言矣,天下岂无有用宽而养奸贻患者乎?大抵内宽而外严,则庶几矣。

八、凡人病殁之后,其子孙家人往往以为庸医误投方药之所致,甚至有衔恨终身者。余尝笑曰:何其视我命太轻,而视医者之权太重若此耶。庸医用药差误,不过使病体缠绵,多延时日,不能速痊耳。若病至不起前数已定,虽卢扁岂能为功,乃归咎于庸医用药之不善不亦寃哉?

九、世之有心计者,每行一事,必思算无遗策,夫使犹有遗策则多算,何为不过招刻薄之名,致众人怨恨而已。若果算无遗策,则上犯造物之怒,其为不祥莫大焉。

十、凡事当极不好处宜向好处想,当极好处宜向不好处想。

十一、人生荣辱进退皆有一定之数,宜以义命自安。

十二、为善所以端品行也,谓为善必获福,则亦尽有不获福者。譬如文字好,则中式世,亦岂无好文而不中者耶,但不可因好文不中,而遂不作好文耳。

十三、制行愈高,品望愈重,则人之伺之益密,而论之亦愈深。防检稍疏则声名俱损。

十四、凡事贵慎密,而国家之事尤不当轻向人言,观古人温室树可见,总之真神仙必不说上界事,其轻言祸福者,皆师巫邪术,惑世欺人之辈耳。

十五、同居共事则猜忌易生也,至于与我不相干涉之人,闻其有如意之事,而中心怅怅,闻有不如意之事,而喜谈乐道之,此皆忌心为之也。余观天下之人,坐此病者甚多,时时省察防闲,恐蹈此薄福之相,惟我俩先人忠厚仁慈,出于天性,每闻人忧戚患难之事,即愀然不快于心,只此一念,便为人情之所难,而贻子孙之福于无穷矣。

十六、古人以盛满为戒。尚书曰:世禄之家,鲜克由礼。盖席丰履厚,其心易于放逸,而又无端人正士、严师益友为之督责,匡救无怪乎流而不返也。譬如一器贮水,盈满虽置于安稳之地,尚虑有倾溢之患,若置之欹侧之地,又从而摇撼之。不但水至倾覆,即器亦不可保矣。处盛满而不知谨慎者,何以异是。

十七、吾人进德修业,未有不静而能有成者。太极图说曰:圣人定之以中正仁义而主静。大学曰:静而后能安,安而后能虑,且不独学问之道为然也,历观天下享遐龄膺厚福之人,未有不静者,静之时义大矣哉!

十八、人生乐事如宫室之美、妻妾之奉、服饰之鲜、饮馔之丰洁、声技之靡丽,其为适意者,皆在外者也,而心之乐不乐不与也。惟有安分循理,不愧不怍,梦魂恬适,神气安闲,斯为吾心之真乐。彼富贵之人,穷施极欲,而心常戚戚,日夕忧虞者,吾不知其乐果何在也。

十九、凡人耳目听睹大率相同,若能神闲气静,则觉有异人处。

二十、余近蒙圣恩赐以广厦名园,深愧过分,昔文端公官宗伯时,屋止数楹,其后洊(cun)登台辅,数十年不易一椽,不增一瓦,曰:安敢为久远计耶?其谨如此,其俭如此,其刻刻求退如此,我后人岂可不知此意,而犹存见少之思耶?

二十一、大聪明人当困心衡虑之后,自然识见倍增,谨之又谨,慎之又慎,与其于放言高论中求乐境,何如于谨言慎行中求乐境耶?

二十二、人臣奉职惟以公正自守,毁誉在所不计,盖毁誉皆出于私心,我不肯徇人之私,则宁受人毁,不可受人誉也。

二十三、他山石曰:万病之毒皆生于浓,浓于声色生虚怯病,浓于货利生贪饕(tao)病,浓于功业生造作病,浓于名誉生矫激玻吾一味药解之曰:淡,吁斯言,诚药石哉!

二十四、人以不可行之事来求我,我直指其不可而谢绝之。彼必怫然不乐,然早断其妄念,亦一大阴德也。若犹豫含糊,使彼妄生觊觎或更以此得罪,此最造孽。人之精神力量,必使有余于事,而后不为事所苦,如饮酒者,能饮十杯,只饮八杯,则其量宽,然后有余,若饮十五杯则不能胜矣。

二十五、处顺境则退一步想,处逆境则进一步想。

二十六、为官第一要廉,养廉之道莫如能忍。尝记姚和修之言曰:有钱用钱,无钱用命。人能拼命强忍,不受非分之财,则于为官之道思过半矣。

二十七、人之葬坟,所以安先人也。葬后子孙昌盛,可以卜先人坟地之吉祥。若先存发福之心以求吉地,则不可。货悖而入者亦悖而出,平生锱铢必较,用尽心计以求赢余,造物嫉之,必使之用若泥沙以自罄其所有,夫劳苦而积之于平时,欢忻鼓舞而散之于一旦,则贪财果何所为耶?所以古人非道非义一介不龋

二十八、人家子弟承父祖之余荫,不能克家,而每好声伎,好古玩。好声伎者及身必败,好古玩者未有传及两世者,余见此多矣,故深以为戒。

二十九、昔人以论孟二语合成一联云:约失之鲜矣,诚乐莫大焉。余时佩服此十字。

三十、君子可欺以其方,若终身不被人欺,此必无之事。倘自谓人不能欺我,此至愚之见,即受欺之本也。

三十一、天下有学问、有识见、有福泽之人未有不静者。

三十二、天下矜才使气之人,一遇挫折,倍觉人所难堪,细思之,未必非福。

三十三、凡人好为翻案之论、好为翻案之文,是其胸襟褊浅处,即其学问偏僻处。孔子曰:中庸,不可能也。请看一部论语,何曾有一句新奇之语。

三十四、不深知“知人论世”四字之意,不可以读史。


\subsection{1.2.3   【卷二】}
\label{\detokenize{p00_u5176_u5b83/_u300a_u5f20_u82f1-_u806a_u8bad_u658b_u8bed_u300b_u300a_u5f20_u5ef7_u7389-_u6f84_u6000_u56ed_u8bed_u300b_u5408_u8f91:id10}}
一、居官清廉乃分内之事,每见清官多刻且盛气凌人,盖其心以清为异人能,是犹未忘乎货贿之见也,至诚而不动者,未之有也。问如何著力,曰:言忠信,行笃敬。

二、孝昌程封翁汉舒笔记曰:人看得自己重,方能有耻。又曰:人世得意事,我觉得可耻,亦非易事。此有道之言也。

三、读《论语》觉得《孟子》太繁且甚费力,读《孟子》又觉诸子之书费力矣。不可不知。

四、孝昌程封翁汉舒曰:一家之中,老幼男女无一个规矩礼法,虽眼前兴旺,即此便是衰败景象。又曰:小小智巧用惯了,便入于下流。而不觉此二语乃治家训子弟之药石也。

五、凡人看得天下事太容易,由于未曾经历也。待人好为责备之论,由于声在局外人也。恕之一字,圣贤从天性中来,中人以上者则阅历而后得之,资秉庸暗者虽经阅历,而梦梦如初矣。

六、注解古人诗文者,每牵合附会以示淹博,是一大玻古人用事用意,有可以窥测者,有不可窥测者,若必欲强勉著笔,恐差之毫厘失之千里,不可不慎也。

七、欧阳公论诗曰:状难写之景如在目前,含不尽之意见于言外,然后为工。此数语,看来浅近,而义蕴深长,得诗家之三味矣。

八、忧患皆从富贵中来,阅历久而后知之。

九、不虞之誉少,而求全之毁多,此人心厚薄所由分也。

十、孔子曰:如有所誉者,其有所试矣。是圣人之心宁偏于厚,其异于常人者正在此。

十一、开卷有益,此古今不易之理。犹记吾友姚别峰有诗曰:掩书微笑破疑团。尤得开卷有益自然而然之乐境也。余深爱之。

十二、韩魏公遗事曰:公判京兆,日得侄孙书云,田产多为邻近侵占,欲经官陈理,公于书尾题诗一首云,“他人侵我且从伊,子细思量未有时,试上含光殿基看,秋风秋草正离离”,其后子孙繁衍,历华要者不可胜数,以其宽大之德致然也。先文端公日以逊让训子孙,《聪训斋语》往复数千言,剀切缠绵即是此意,从今日观之,从前让人无纤毫污损,而子孙荣显,颇为海内所推,孰非积德累仁之报哉!

十三、欧阳文忠公之子名发,述公事迹,有曰,公奉敕撰唐书,专成纪志表,而列传则宋公祁所撰。朝廷恐其体不一,诏公看详,令删为一体,公虽受命,退而曰,宋公于我为前辈,且各人所见不同,岂可悉如己意?于是一无所易。余览之为之三叹,每见读书人于他人著作,往往恣意吹求以炫己长,至于意见不同则坚执己见,百折不回,此等习气,虽贤者不免,览欧公遗事其亦知古人之忠厚固如是乎!

十四、盖天下之乐,莫乐于闲且静,果能领会此二字,不但有自适之趣,即治事读书必志气清明,精神完足,无障碍亏缺处。若日事笙歌,喧哗杂处,神志渐就昏惰,事务必至废弛,多费又其余事也。至于蓄优人于家,则更不可,此等轻儇(xuan)佻达之辈,日与子弟家人相处,渐染仿效,默夺潜移,日流于匪僻,其害有不可胜言者。余居京师久,见富贵家之蓄优人者或数年或数十年或一再传,而后必至家规荡弃、生计衰微,百不爽一。人情孰不为子孙计而乃图一时之娱乐,则后人无穷之患,不亦重可叹哉!

十五、邵康节尝诵希夷之语曰:得便宜事不可再作,得便宜处不可再去。又曰:落便宜处是得便宜。故康节诗云:珍重至人常有语,落便宜事得便宜。元遗山诗曰:得便宜处落便宜,木石痴儿自不知。此语常人皆能言之,而实能领会其意者,非见道最深之人,不足以语此也,余不敏,愿终身诵之。


\subsection{1.2.4   【卷三】}
\label{\detokenize{p00_u5176_u5b83/_u300a_u5f20_u82f1-_u806a_u8bad_u658b_u8bed_u300b_u300a_u5f20_u5ef7_u7389-_u6f84_u6000_u56ed_u8bed_u300b_u5408_u8f91:id11}}
一、凡人于极得意极失意时,能检点言语无过当之辞,其人之学问气量必有大过人处。

二、乐道人之善,恶称人之恶。皆出于论语,可作书室对联,触目警心也。

三、明儒吕叔简先生坤曰:家人之害,莫大于卑幼各恣其无厌之情,而上之人阿其意,而不之禁;尤莫大于婢子造言,而妇人悦之,妇人附会而丈夫信之。禁此二害,而家不和睦者,鲜矣。又曰:今人骨肉之好不终,只为看得尔我二字太分晓。此二(瑕)语虽浅近,实居家之药石也。

四、董华亭宗伯曰:结千百人之欢,不如释一人之怨。余曰此长厚之言也。

五、山东曹县吕道人不知其年,问之亦不以实告,大约在百龄内外,善养生修炼之术,鹤发童颜,步履矍铄,终日不食亦不饥,顶心出香气,如麝檀硫磺。然此子亲见者以针(zhen)砭为人疗病,辄效赠以财物不受。曰:天下之物哪一件是我的?人曰:聊以表吾心耳。答曰:天下之物哪一件是你的?此二语予最爱之,可以警觉天下之贪取妄求而不知止足者。凡人度量广大、不嫉妒、不猜疑,乃己身享福之相,于人无所损益也。纵生性不能如此,亦当勉强而行之。彼幸灾乐祸之人,不过自成其薄福之相耳,于人又何损乎?不可不发深剩

六、吾乡左忠毅公举乡试,谒本房陈公大绶,陈勉以树立,却红柬不受,谓曰:今日行事节俭,即异日做官清,不就此站定脚跟,后难措手。呜呼,不矜细行,终累大德,前辈之谨小慎微如此,彼后生小子生富贵之家、染纨绔之习,何足以知之?

七、朱子口铭曰:病从口入,祸从口出。此语人人知之,且病与祸人人所恶也,而能致谨于入口出口之际者盖寡,则能忍之难也。书曰:必有忍,其乃有济。武王书铭曰:忍之须臾,乃全汝躯。昔人诗曰:忍过事堪喜。忍之时义大矣哉!

八、余五十年来留心默识彼语言不实之辈,一时可以欺世,而究竟飘荡于终身。凤鉴书所谓到老终无结果也,若怀私挟怨捏造蜚语害人名节身家者,厥后必有恶报,以予所见,屈指而数,未可以为天道渺茫,在可知不可知之间也。

九、武侯戒子书曰:君子之行,静以修身,俭以养德,非淡泊无以明志,非宁静无以致远。夫学须静也,才须学也,非学无以广才,非静无以成学。怠慢则不能研精,险躁则不能理性。予尝以静字训子弟,今再益以静以修身,学须静也二语,其中义蕴精微,非大有识见人不能理会。

十、孟子曰:予岂好辩哉?予不得已也!吾人必深知孟子不得已之苦衷,方可以读孟子,不然则书中可疑可议者,不可胜数也。

十一、邵康节诗曰:静处乾坤大,闲中日月长。“闲中日月长”人所知也,“静处乾坤大”人或未知也。予一生好静,于此中颇有领会,奈此身牵于职守,日在红尘扰攘中,常为设想曰,若能改静处为闹处,则有进步矣,惜乎其不能也。


\subsection{1.2.5   【卷四】}
\label{\detokenize{p00_u5176_u5b83/_u300a_u5f20_u82f1-_u806a_u8bad_u658b_u8bed_u300b_u300a_u5f20_u5ef7_u7389-_u6f84_u6000_u56ed_u8bed_u300b_u5408_u8f91:id12}}
一、隐恶扬善,圣人也;好善恶恶,贤人也;分别善恶无当者,庸人也;颠倒善恶以快其谗,谤者小人也。赴大机者速断,成大功者善藏,同时中庸,而君子小人之别微也哉!

二、予少时夜卧,难以成寐,既寐之后,一闻声息即醒。先兄宫詹公授以引睡之法,背读上论语数页或十数页,使心有所寄。予试之,果然。后推广其意,诵渊明诗“采菊东篱下,悠然见南山”或钱考功诗“曲终人不见,江上数峰青”或陆放翁诗“小楼一夜听春雨,深巷明朝卖杏花”,皆古人潇洒闲适之句,神游其境,往往睡去。盖心不可有著,又不可一无所著也。

三、薛文清曰:多言最使人心志流荡,而气亦损,少言不惟养得德深,又养得气完。

四、陈眉公曰:颐卦,慎言语,节饮食。然口之所入,其祸小;口之所出,其罪多。故鬼谷子云:口可以饮,不可以言。又曰:圣人之言简,贤人之言明,众人之言多,小人之言妄。

五、伊川先生曰:只观发言之平易,躁妄便见,德之厚薄所养之深浅见。人论前辈之短曰:汝辈且取他长处。薛文清公曰:在古人之后,议古人之失,则易处古人之位;为古人之位,为古人之事则难。此处不可不深剩

六、陆士衡豪士赋云:身危由于势过,而不知去势以求安;祸积由于宠盛,而不知辞宠以招福。此富贵人之通病也。

七、李之彦曰:尝玩钱字,旁上著一戈字,下著一戈字,真杀人之物也,然则两戈争贝,岂非贱乎?

八、陈眉公曰:醉人胆大,与酒融洽故也。人能与义命融治,浩然之气自然充塞,何惧之有?

九、象山先生曰:学者不长进,只是好己胜,出一言,做一事,便道全是,岂有此理。古人惟贵知过则改,见善则迁。今各执己是,被人点破便愕然所以,不如古人先生。此言乃天下学者之通病,若能不蹈此病,则其天资识量过人远矣。倘见此而能省察悔悟,将来亦必有所成就。

十、古人云:教子之道有五,静其性,广其志,养其材,鼓其气,攻其病,废一不可。


\subsection{1.2.6   【附录】:}
\label{\detokenize{p00_u5176_u5b83/_u300a_u5f20_u82f1-_u806a_u8bad_u658b_u8bed_u300b_u300a_u5f20_u5ef7_u7389-_u6f84_u6000_u56ed_u8bed_u300b_u5408_u8f91:id13}}

\subsubsection{1.2.6.1   【清朝名相张廷玉的祖上,平价粜米,周济穷苦的故事】}
\label{\detokenize{p00_u5176_u5b83/_u300a_u5f20_u82f1-_u806a_u8bad_u658b_u8bed_u300b_u300a_u5f20_u5ef7_u7389-_u6f84_u6000_u56ed_u8bed_u300b_u5408_u8f91:id14}}
(一)独力办施济,慈德谦光。

财之为物,生不带来,死不带去,传与子孙,无一不败。故智慧之人,当乘有权在手,有钱可施之时,广作利人利物功德,使个个金钱,造成未来胜福。

明末时候,桐城地方有一个张老员外,存心慈善,喜欢施舍。有一年,遇着荒歉,米价腾贵,一般奸利的商人看到这情形,反把米粮囤积着不肯出售,于是平民大起恐慌。官府里请命办账,又是迂回曲折的不能立见施行。员外看了这情形,很是忧急。他家里有存米万石,这时便自动的举行平粜,照市价减半出售,每人每日,限购一升,以防奸人套买图利。平民听到这消息,欢喜若狂。员外又想到一般赤贫的人无钱买米,仍在挨饿,于是又办了一个施粥厂,受施的人隔日领券,统计了人数煮着大量的粥,按券发给,一日三餐,每餐白粥一大碗,咸菜一小碟。许多人枵腹而来,鼓腹而去,大家都称颂员外是个活菩萨。员外很谦虚的说:“荒年米价贵,减半出售,已和平时全价相等,所以我也没有什么损失,至于施粥,也所费有限。总之,只要大家有饭吃,我就很觉安慰了。”

(二)夫妇商典质,同德同心。

世间之财,水能漂没之,火能烧毁之,盗贼能劫掠之,官吏能没收之,不肖儿孙能消败之,故称不坚之财,惟用以利人,可以后福无量。

老员外接连的办理平粜施粥,家里的钱渐渐赔完了,但是荒歉的现象一时不能平复,自己的善事不能半途中止,因此十分心焦。他想,我这时若把救济事业停止了,一般贫民便有饿死的可能,我当初的救济不是和不救济一样了嘛?救人须救到底。现在我还有一部分产业存在,何不变卖了继续办理!想定了主意,便到内室去和夫人商量。他的夫人也是十分贤德的,听了他的话,极端赞成,并且说:“积产业给子孙,没有积德,子孙不肖,就是金山银山也要倒的。若是积德给子孙,虽没有财产,将来子孙好,也会富裕起来的。田地房屋,由相公做主变卖,就是我有许多首饰衣服也卖了吧!”员外听了,额手称赞道:“夫人的话,说得真有理!”于是变卖产业,继续善事,直到荒歉的现象消除了才止。

(三)簪缨绵世泽,善报无差。

祖宗积德,后嗣发福,考之历史,验之现在,百不失一,于此可以证实因果之不虚。读者宜注意,得益无量。

老员外故世后,到了第五代孙子张英,做到宰相之职,张英的儿子张廷玉,也继续着父亲的地位,成为清朝盛世时的著名宰相。以后子孙,累代显荣,时享官禄。

有人说:“老员外死时,有个异人指点着一块好风水的葬地,所以子孙富贵。”这件事的真假,不必细论,总之,员外这样积德,子孙自然应该发达,若是好风水,也要心地好,才能得到呢。


\chapter{1   张英-聪训斋语}
\label{\detokenize{p00_u5176_u5b83/_u5f20_u82f1-_u806a_u8bad_u658b_u8bed:id1}}\label{\detokenize{p00_u5176_u5b83/_u5f20_u82f1-_u806a_u8bad_u658b_u8bed::doc}}
\begin{sphinxShadowBox}
\sphinxstyletopictitle{目录}
\begin{itemize}
\item {} 
\phantomsection\label{\detokenize{p00_u5176_u5b83/_u5f20_u82f1-_u806a_u8bad_u658b_u8bed:id5}}{\hyperref[\detokenize{p00_u5176_u5b83/_u5f20_u82f1-_u806a_u8bad_u658b_u8bed:id1}]{\sphinxcrossref{1   张英-聪训斋语}}}
\begin{itemize}
\item {} 
\phantomsection\label{\detokenize{p00_u5176_u5b83/_u5f20_u82f1-_u806a_u8bad_u658b_u8bed:id6}}{\hyperref[\detokenize{p00_u5176_u5b83/_u5f20_u82f1-_u806a_u8bad_u658b_u8bed:id3}]{\sphinxcrossref{1.1   有之四纲十二目如下:}}}

\item {} 
\phantomsection\label{\detokenize{p00_u5176_u5b83/_u5f20_u82f1-_u806a_u8bad_u658b_u8bed:id7}}{\hyperref[\detokenize{p00_u5176_u5b83/_u5f20_u82f1-_u806a_u8bad_u658b_u8bed:id4}]{\sphinxcrossref{1.2   原文}}}

\end{itemize}

\end{itemize}
\end{sphinxShadowBox}


\section{1.1   有之四纲十二目如下:}
\label{\detokenize{p00_u5176_u5b83/_u5f20_u82f1-_u806a_u8bad_u658b_u8bed:id3}}
一 立品纲——戒嬉戏、慎威仪、谨言语。

二 读书纲——温经书、精举业、学楷字。

三 养身纲——谨起居、慎寒暑、节用度。

四 择友纲——谢酬应、省宴集、寡交游。


\section{1.2   原文}
\label{\detokenize{p00_u5176_u5b83/_u5f20_u82f1-_u806a_u8bad_u658b_u8bed:id4}}
人心至灵至动,不可过劳,亦不可过逸,惟读书可以养之。书卷乃养心第一妙物。闲适无事之人,镇日不观书,则起居出入,身心无所栖泊,耳目无所安顿,势必心意颠倒,妄想生嗔。处逆境不乐,处顺境亦不乐。每见人栖栖皇皇,觉举动无不碍者,此必不读书之人也。

富贵贫贱,总难称意,知足即为称意;山水花竹,无恒主人,得闲便是主人。大约富贵人役于名利,贫贱人役于饥寒,总无闲情及此,惟付之浩叹耳。

古人以“眠、食”二者为养生之要务。脏腑肠胃,常令宽舒有余地,则真气得以流行而疾病少。“予从不饱食,病安得入?”燔炙熬煎香甘肥腻之物,最悦口而不宜于肠胃。彼肥腻易于粘滞,积久则腹痛气塞,寒暑偶侵,则疾作矣。食忌多品,一席之间,遍食水陆,浓淡杂进,自然损脾;安寝,乃人生最乐,古人有言:不觅仙方觅睡方。冬夜以二鼓为度,暑月以一更为度。每笑人长夜酣饮不休,谓之消夜,夫人终日劳劳,夜则宴息,是极有味,何以消遣为?冬夏,皆当以日出而起,于夏尤宜。天地清旭之气,最为爽神,失之,甚为可惜。予山居颇闲,暑月,日出则起,收水草清香之味,莲方敛而未开,竹含露而犹滴,可谓至快!日长漏永,不妨午睡数刻,睡足而起,神清气爽;居家最宜早起,倘日高客至,僮则垢面,婢且蓬头,庭除未扫,灶突犹寒,大非雅事。

人家僮仆,最多不宜多畜,但有得力二三人,训谕有方,使令得宜,未尝不得兼人之用。太多则彼此相诿,恩养必不能周,教训亦不能及,反不得其力;吾辈居家居宦,皆简静守理,不为暗昧之事;山中耕田锄圃之仆,乃可为宝,其人无奢望,无机智,不为主人敛怨,彼纵不遵约束,不过懒惰、愚蠢之小过,不必加意防闲,岂不为清闲之一助哉?

俭于饮食,可以养脾胃;俭于嗜欲,可以聚精神;俭于言语,可以养气息非;俭于交游,可以择友寡过;俭于酬酢,可以养身息劳;俭于夜坐,可以安神舒体;俭于饮酒,可以清心养德;俭于思虑,可以蠲烦去扰;白香山诗云:“我有一言君记取,世间自取苦人多。”;人常和悦,则心气冲而五脏安,昔人所谓养欢喜神,日间办理公事,每晚家居,必寻可喜笑之事,与客纵谈,掀髯大笑,以发舒一日劳顿郁结之气;砚以世计,墨以时计,笔以日计,动静之分也。静之义有二:一则身不过劳,一则心不轻动。

万事做到极精妙处,无有不圆者。人之一身,与天时相应,大约三四十以前,是夏至前,凡事渐长;三四十以后,是夏至后,凡事渐衰,中间无一刻停留。中间盛衰关头,无一定时候,大概在三四十之间,观于须发可见:其衰缓者,其寿多;其衰急者,其寿寡。人身不能不衰,先从上而下者,多寿,故古人以早脱顶为寿征,先从下而上者,多不寿,故须发如故而脚软者难治;凡人家道亦然,决无中立之理,如一树之花,开到极盛,便是摇落之期。
予怪世人于古人诗文集不知爱,而宝其片纸只字,为大惑也。余昔在龙眠,苦于无客为伴,日则步于空潭碧涧、长松茂竹之侧,夕则掩关读苏陆诗,以二鼓为度,烧烛焚香,煮茶延两君子于坐,与之相对,如见其容貌须眉然。诗云:“架头苏陆有遗书,特地携来共索居。日与两君同卧起,人间何客得胜渠。”良非解嘲语也。

门无杂宾,大约门下奔走之客,有损无益。
人生适意之事有三:曰贵,曰富,曰多子孙。然是三者,善处之则为富,不善处之则足为累。高位者,责备之地,忌嫉之门,怨尤之府,利害之关,忧患之窟,劳苦之薮,谤讪之的,攻击之场,古之智人往往望而止步;夫人厚积则必经营布置,生息防守,其劳不可胜言:则必有亲戚之请求,贫穷之怨望,僮仆之奸骗,大而盗贼之劫取,小而穿窬之鼠窃,经商之亏折,行路之失脱,田禾之灾伤,攘夺之争讼,子弟之浪费。种种之苦,贫者不知,惟富厚者兼而有之。人能各富之为累,则取之当廉,而不必厚积以招怨;至子孙之累尤多矣,少小则有疾病之虑,稍长则有功名之虑,浮奢不善治家之虑,纳交匪类之虑,一离膝下,则有道路寒暑饥渴之虑,以至由子而孙,展转无穷,更无底止。

予之立训,更无多言,止有四语:读书者不贱,守田者不饥,积德者不倾,择交者不败。虽至寒苦之人,但能读书为文,必使人钦敬,不敢忽视。其人德性,亦必温和,行事决不颠倒,不在功名之得失,遇合之迟速也。

人生必厚重沉静,而后为载福之器。敦厚谦谨,慎言守礼,不可与寒士同一般感慨欷嘘,放言高论,怨天尤人,庶不为造物鬼神所呵责也。
乡里间荷担负贩及佣工小人,切不可取其便宜,此种人所争不过数文,我辈视之甚轻,而彼之含怨甚重。每有愚人见省得一文,以为得计,而不知此种人心忿口碑,所损实大也。待下我一等之人,言语辞气最为要紧,此事甚不费钱,然彼人受之,同于实惠,只在精神照料得来,不可惮烦;读书固所以取科名,继家声,然亦使人敬重;每见仕宦显赫之家,其老者或退或故,而其家索然者,其后无读书之人也,其家郁然者,其后有读书之人也;父母之爱子,第一望其康宁,第二冀其成名,第三愿其保家。《语》曰:“父母惟其疾之忧。”夫子以此答武伯之问孝,至哉斯言!安其身以安父母之心,孝莫大焉。养身之道,一在谨嗜欲,一在慎饮食,一在慎忿怒,一在慎寒暑,一在慎思索,一在慎烦劳。吾贻子孙,不过瘠田数处耳,且甚荒芜不治,水旱多虞。岁入之数,谨足以免饥寒,畜妻子而已,一件儿戏事做不得,一件高兴事做不得;人生豪侠周密之名至不易副。事事应之,一事不应,遂生嫌怨,人人周之,一人不周,便存形迹,若平素俭啬,见谅于人,省无穷物力,少无穷嫌怨,不亦至便乎?;人生二十内外,渐远于师保之严,未跻于成人之列,此时知识大开,性情未定,父师之训不能入,即妻子之言亦不听,惟朋友之言,甘如醴而芳若兰,脱有一淫朋匪友,阑入其侧,朝夕浸灌,鲜有不为其所移者;(坏)朋友,则直以不识其颜面,不知其姓名为善。比之毒草哑泉更当远避。

楷书如坐如立,行书如行,草书如奔。
法昭禅师偈云:“同气连枝各自荣,些些言语各伤情。一回相见一回老,能得几时为弟兄?”词意蔼然,足以启人友于之爱。然予尝谓人伦有五,而兄弟相处之日最长。

世人只因不知命,不安命,生出许多劳扰;君子修身以俟之。
余家训有云:“保家莫如择友。”盖痛心疾首其言之也!汝辈但于至戚中,观其德性谨厚,好读书者,交友两三人足矣!且势利言之,则有酒食之费、应酬之扰,一遇婚丧有无,则有资给贷之事。甚至有争讼外侮,则又有关说救援之事。平昔既与之契密,临事却之,必生怨毒反唇。故余以为宜慎之于始也;昔人有戒:“饭不嚼便咽,路不看便走,话不想便说,事不思便做。”予益之曰:“友不择便交,气不忍不便动,财不审便取,衣不慎便脱。”

学字当专一。择古人佳帖或时人墨迹与已笔路相近者,专心学之,若朝更夕改,见异思迁,鲜有得成者。若体格不匀净而遽讲流动,失其本矣!学字忌飞动草率,大小不匀,而妄言奇古磊落,终无进步矣。读文不必多,择其精纯条畅,有气局词华者,多则百篇,少则六十篇。神明与之浑化,始为有益。若贪多务博,过眼辄忘,及至作时,则彼此不相涉,落笔仍是故吾,所以思常窒而不灵,词常窘而不裕,意常枯而不润。

人能处心积虑,一言一动皆思益人,而痛戒损人,则人望之若鸾凤,宝之如参苓。必为天地所佑,鬼神之所服,而享有多福矣!

凡读书,二十岁以前所读之书与二十岁以后所读之书迥异。幼年知识未开,天真纯固,所读者虽久不温习,偶尔提起,尚可数行成诵。若壮年所读,经月则忘,必不能持久。故六经、秦汉之文,词语古奥,必须幼年读。长壮后,虽倍蓰其功,终属影响。自八岁至二十岁,中间岁月无多,安可荒弃或读不急之书?此时,时文固不可不读,亦须择典雅醇正、理纯辞裕、可历二三十年无弊者读之。若朝华夕落、浅陋无识、诡僻失体、取悦一时者,安可以珠玉难换之岁月而读此无益之文?何如诵得《左》、《国》一两篇及东西汉典贵华腴之文数篇,为终身之用之宝乎?古人之书,安可尽读?但我所已读者决不轻弃。得尺则尺,得寸则寸。毋贪多,毋贪名,但求读一篇,必可以背诵。然后思通其义蕴,而运用之于手腕之下,如此则才气自然发越。若曾读此书,而全不能举其词,谓之“画饼充饥”。能举其词而不能运用,谓之“食物不化”。

深恼人读时文累千累百而不知理会,于身心毫无裨益。夫能理会,则数十篇百篇已足,焉用如此之多?不能理会,则读数千篇与不读一字等。徒使精神聩乱,临文捉笔,依旧茫然,不过胸中旧套应副,安有名理精论、佳词妙句,奔汇于笔端乎?古人云:“读生文不如玩熟文。必以我之精神,包乎此一篇之外,以我之心思,入乎此一篇之中。幼年当专攻举业,以为立身之本。

世家子弟,其修行立名之难,较寒士百倍。何以故?人之当面待之者,万不能如寒士之古道:小有失检,谁肯面斥其非?微有骄盈,谁肯深规其过?幼而骄惯,为亲戚之所优容;长而习成,为朋友之所谅恕;我愿汝曹常以席丰履盛为可危、可虑、难处、难全之地,勿以为可喜、可幸、易安、易逸之地;终身让路,不失尺寸,自古祗闻“忍”与“让”,足以消无穷之灾悔,未闻“忍”与“让”,翻以酿后来之祸患也,欲行忍认之道,先须从小事做起。余曾署刑部事五十日,见天下大讼大狱,多从极小事起。君子敬小慎微,凡事只从小处了。余行年五十余,生平未尝多受小人之侮,只有一善策,能转弯早耳。每思天下事,受得小气,则不至于受大气,吃得小亏,则不至于吃大亏,此生平得力之处。凡事最不可想占便宜,便宜者,天下人所共争也,我一人据之,则怨萃于我矣,我失便宜,则众怨消矣。故终身失便宜,乃终身得便宜也;座右箴:立品、读书、养身、择友。右四纲。戒嬉戏,慎威仪;谨言语,温经书;精举业,学楷字;谨起居,慎寒暑;节用度,谢酬;省宴集,寡交游。右十二目。

子弟自十七八以至廿三四,实为学业成废之关。盖自初入学至十五六,父师以童子视之,稍知训子者,断不忍听其废业。惟自十七八以后,年渐长,气渐骄,渐有朋友,渐有室家,嗜欲渐广。父母见其长成,师傅视为侪辈。德性未坚,转移最易;学业未就,蒙昧非难。幼年所习经书,此时皆束高阁。酬应交游,侈然大雅。博弈高会,自诩名流。转盼廿五六岁,儿女累多,生计迫蹙,蹉跎潦倒,学殖荒落。予见人家子弟半途而废者,多在此五六年中,弃幼学之功,贻终身之累,盖辙相踵也。汝正当此时,离父母之侧,前言诸弊,事事可虑。为龙为蛇,为虎为鼠,分于一念,介在两歧,可不慎哉!可不畏哉!

读书须明窗净几,案头不可多置书;作文以握管之人为大将,以精熟墨卷百篇为练兵,以杂读时艺为散卒,以题为坚垒。

天子知俭,则天下足,一人知俭,则一家足。且俭非止节啬财用己也。俭于言语,则元气藏而怨尤寡;则于交游,则匪类远,俭于酬酢,则岁月宽而本业修,俭于书札,则后患寡,俭于嬉游,则学业进;人生俭啬之名,可受而不必避,世俗每以为耻,不知此名一噪,则人绝觊觎之想。偶有所用,人即德之;保家莫如择友,多则二人,少则一人,断无目前良友,遂可得十数人之理!平时既简于应酬,有事可以请教。

惟田产房屋二者可恃以久远,以二者较之,房舍又不如田产。
今人家子弟,鲜衣怒马,恒舞酣歌。一裘之费动至数十金,一席之费动至数金。不思吾乡十余年来谷贱,竭十余石谷,不足供一筵,竭百余石谷,不足供一衣。安知农家作苦,终年沾衣涂足,岂易得此百石?
古人之意,全在小处节俭,大处之不足,由于小处之不谨,月计之不足,由于每日之用过多也。
子弟有二三千金之产,方能城居。若千金以下之业,则断不可城居矣!

古人有言,扫地焚香,清福已具。其有福者,佐以读书;其无福者,便生他想。旨哉斯言,予所深赏!且从来拂意之事,自不读书者见之,似为我所独遭,极其难堪,不知古人拂意之事有百倍于此者,特不细心体验耳! 即如东坡先生,殁后遭逢高孝,文字始出,而当时之忧谗畏讥,困顿转徙潮惠之间,苏过跣足涉水,居近牛栏,是何如境界?又如白香山之无嗣,陆放翁之忍饥,皆载在书卷,彼独非千载闻人,而所遇皆如此? 诚一平心静观,则人间拂意之事,可以涣然冰释。若不读书,则但见我所遭甚苦,而无穷怨尤嗔忿之心,烧灼不宁,其苦为何如耶?且富盛之事,古人亦有之,炙手可热,转眼皆空。故读书可以增长道心,为颐养第一事也!


\chapter{1   袁了凡-了凡四训}
\label{\detokenize{p00_u5176_u5b83/_u8881_u4e86_u51e1-_u4e86_u51e1_u56db_u8bad:id1}}\label{\detokenize{p00_u5176_u5b83/_u8881_u4e86_u51e1-_u4e86_u51e1_u56db_u8bad::doc}}
\begin{sphinxShadowBox}
\sphinxstyletopictitle{目录}
\begin{itemize}
\item {} 
\phantomsection\label{\detokenize{p00_u5176_u5b83/_u8881_u4e86_u51e1-_u4e86_u51e1_u56db_u8bad:id9}}{\hyperref[\detokenize{p00_u5176_u5b83/_u8881_u4e86_u51e1-_u4e86_u51e1_u56db_u8bad:id1}]{\sphinxcrossref{1   袁了凡-了凡四训}}}
\begin{itemize}
\item {} 
\phantomsection\label{\detokenize{p00_u5176_u5b83/_u8881_u4e86_u51e1-_u4e86_u51e1_u56db_u8bad:id10}}{\hyperref[\detokenize{p00_u5176_u5b83/_u8881_u4e86_u51e1-_u4e86_u51e1_u56db_u8bad:id3}]{\sphinxcrossref{1.1   第一篇 立命之学}}}

\item {} 
\phantomsection\label{\detokenize{p00_u5176_u5b83/_u8881_u4e86_u51e1-_u4e86_u51e1_u56db_u8bad:id11}}{\hyperref[\detokenize{p00_u5176_u5b83/_u8881_u4e86_u51e1-_u4e86_u51e1_u56db_u8bad:id4}]{\sphinxcrossref{1.2   第二篇 改过之法}}}

\item {} 
\phantomsection\label{\detokenize{p00_u5176_u5b83/_u8881_u4e86_u51e1-_u4e86_u51e1_u56db_u8bad:id12}}{\hyperref[\detokenize{p00_u5176_u5b83/_u8881_u4e86_u51e1-_u4e86_u51e1_u56db_u8bad:id5}]{\sphinxcrossref{1.3   第三篇 积善之方}}}

\item {} 
\phantomsection\label{\detokenize{p00_u5176_u5b83/_u8881_u4e86_u51e1-_u4e86_u51e1_u56db_u8bad:id13}}{\hyperref[\detokenize{p00_u5176_u5b83/_u8881_u4e86_u51e1-_u4e86_u51e1_u56db_u8bad:id6}]{\sphinxcrossref{1.4   第四篇 谦德之效}}}

\item {} 
\phantomsection\label{\detokenize{p00_u5176_u5b83/_u8881_u4e86_u51e1-_u4e86_u51e1_u56db_u8bad:id14}}{\hyperref[\detokenize{p00_u5176_u5b83/_u8881_u4e86_u51e1-_u4e86_u51e1_u56db_u8bad:id7}]{\sphinxcrossref{1.5   【袁了凡居士传】}}}

\item {} 
\phantomsection\label{\detokenize{p00_u5176_u5b83/_u8881_u4e86_u51e1-_u4e86_u51e1_u56db_u8bad:id15}}{\hyperref[\detokenize{p00_u5176_u5b83/_u8881_u4e86_u51e1-_u4e86_u51e1_u56db_u8bad:id8}]{\sphinxcrossref{1.6   【袁了凡居士传】【注】}}}

\end{itemize}

\end{itemize}
\end{sphinxShadowBox}


\section{1.1   第一篇 立命之学}
\label{\detokenize{p00_u5176_u5b83/_u8881_u4e86_u51e1-_u4e86_u51e1_u56db_u8bad:id3}}
余童年丧父,母命弃举业学医,谓可以养生,可以济人,且习一艺以成名,尔父夙心也。后余在慈云寺,遇一老者,修髯伟貌,飘飘若仙,余敬礼之。

语余曰:“子仕路中人也,明年即进学矣,何不读书?”余告以故,并叩老者姓氏里居。

曰:“吾姓孔,云南人也。得邵子皇极正传,数该传汝。”余即引之归,告母。

母曰:“善待之。”试其数,谶悉皆验。余遂启读书之念,谋之表兄沈称, 言:“郁海谷先生,在沈友夫家开馆,我送汝寄学甚便。”余遂礼郁为师。

孔为余起数:县考童生,当十四名;府考七十一名,提学考第九名。明年赴考,三处名数皆合。复为卜终身休咎,言:某年考第几名,某年当补廪,某年当贡,贡后某年,当选四川一大尹,在任三年半,即宜告归。五十三岁八月十四日丑时,当终于正寝,惜无子。余备录而谨记之。

自此以后,凡遇考校,其名数先后,皆不出孔公所悬定者。独算余食廪米九十一石五斗当出贡。及食米七十一石,屠宗师即批准补贡,余窃疑之。后果为署印杨公所驳,直至丁卯年(西元1567年),殷秋溟宗师见余场中备卷,叹曰:“五策,即五篇奏议也,岂可使博洽淹贯之儒,老于窗下乎!”遂依县申文准贡,连前食米计之,实九十一石五斗也。余因此益信进退有命,迟速有时,澹然无求矣。

贡入燕都,留京一年,终日静坐,不阅文字。己巳(西元1569年)归,游南雍,未入监,先访云谷会禅师于栖霞山中,对坐一室,凡三昼夜不瞑目。

云谷问曰:“凡人所以不得作圣者,只为妄念相缠耳。汝坐三日,不见起一妄念,何也?”

余曰:“吾为孔先生算定,荣辱生死,皆有定数,即要妄想,亦无可妄想。”

云谷笑曰:“我待汝是豪杰,原来只是凡夫。”问其故?

曰:“人未能无心,终为阴阳所缚,安得无数?但惟凡人有数。极善之人,数固拘他不定;极恶之人,数亦拘他不定。汝二十年来,被他算定,不曾转动一毫,岂非是凡夫?”

余问曰:“然则数可逃乎?”

曰:“命由我作,福自己求。诗书所称,的为明训。我教典中说:‘求富贵得富贵,求男女得男女,求长寿得长寿。’夫妄语乃释迦大戒,诸佛菩萨,岂诳语欺人?”

余进曰:“孟子言:‘求则得之’,是求在我者也。道德仁义可以力求;功名富贵,如何求得?”

云谷曰:“孟子之言不错,汝自错解耳。汝不见六祖说:‘一切福田,不离方寸;从心而觅,感无不通。’求在我,不独得道德仁义,亦得功名富贵;内外双得,是求有益于得也。若不反躬内省,而徒向外驰求,则求之有道,而得之有命矣,内外双失,故无益。”

因问:“孔公算汝终身若何?”余以实告。

云谷曰:“汝自揣应得科第否?应生子否?”余追省良久,曰:“不应也。科第中人,有福相,余福薄,又不能积功累行,以基厚福;兼不耐烦剧,不能容人;时或以才智盖人,直心直行,轻言妄谈。凡此皆薄福之相也,岂宜科第哉。地之秽者多生物,水之清者常无鱼;余好洁,宜无子者一;和气能育万物,余善怒,宜无子者二;爱为生生之本,忍为不育之根;余矜惜名节,常不能舍己救人,宜无子者三; 多言耗气,宜无子者四;喜饮铄精,宜无子者五; 好彻夜长坐,而不知葆元毓神,宜无子者六。其余过恶尚多,不能悉数。”

云谷曰:“岂惟科第哉。世间享千金之者,定是千金人物;享百金之产者,定是百金人物;应饿死者,定是饿死人物;天不过因材而笃,几曾加纤毫意思。即如生子,有百世之德者,定有百世子孙保之;有十世之德者,定有十世子孙保之;有三世二世之德者,定有三世二世子孙保之;其斩焉无后者,德至薄也。 汝今既知非。将向来不发科第,及不生子之相,尽情改刷;务要积德,务要包荒,务要和爱,务要惜精神。从前种种,譬如昨日死;从后种种,譬如今日生;此义理再生之身。夫血肉之身,尚然有数;义理之身,岂不能格天。太甲曰:‘天作孽,犹可违;自作孽,不可活。’诗云:‘永言配命,自求多福。’孔先生算汝不登科第,不生子者,此天作之孽,犹可得而违;汝今扩充德性,力行善事,多积阴德,此自己所作之福也,安得而不受享乎?易为君子谋,趋吉避凶;若言天命有常,吉何可趋,凶何可避?开章第一义,便说:‘积善之家,必有余庆。’汝信得及否?”

余信其言,拜而受教。因将往日之罪,佛前尽情发露,为疏一通,先求登科;誓行善事三千条,以报天地祖宗之德。

云谷出功过格示余,令所行之事,逐日登记;善则记数,恶则退除,且教持准提咒,以期必验。

语余曰:“符录家有云:‘不会书符,被鬼神笑。’此有秘传,只是不动念也。执笔书符,先把万缘放下,一尘不起。从此念头不动处,下一点,谓之混沌开基。由此而一笔挥成,更无思虑,此符便灵。凡祈天立命,都要从无思无虑处感格。孟子论立命之学,而曰:‘夭寿不贰。’夫夭寿,至贰者也。当其不动念时,孰为夭,孰为寿?细分之,丰歉不贰,然后可立贫富之命;穷通不贰,然后可立贵贱之命;夭寿不贰,然后可立生死之命。人生世间,惟死生为重,曰夭寿,则一切顺逆皆该之矣。至修身以俟之,乃积德祈天之事。曰修,则身有过恶,皆当治而去之;曰俟,则一毫觊觎,一毫将迎,皆当斩绝之矣。到此地位,直造先天之境,即此便是实学。汝未能无心,但能持准提咒,无记无数,不令间断,持得纯熟,于持中不持,于不持中持。到得念头不动,则灵验矣。”

余初号学海,是日改号了凡;盖悟立命之说,而不欲落凡夫窠臼也。从此而后,终日兢兢,便觉与前不同。前日只是悠悠放任,到此自有战兢惕厉景象,在暗室屋漏中,常恐得罪天地鬼神;遇人憎我毁我,自能恬然容受。

到明年(西元1570年)礼部考科举,孔先生算该第三,忽考第一;其言不验,而秋闱中式矣。然行义未纯,检身多误;或见善而行之不勇,或救人而心常自疑;或身勉为善,而口有过言;或醒时操持,而醉后放逸;以过折功,日常虚度。自己巳岁(西元1569年)发愿,直至己卯岁(西元1579年),历十余年,而三千善行始完。

时方从李渐庵入关,未及回向。庚辰(西元1580年)南还。始请性空、慧空诸上人,就东塔禅堂回向。遂起求子愿,亦许行三千善事。辛巳(西元1581年),生男天启。

余行一事,随以笔记;汝母不能书,每行一事,辄用鹅毛管,印一朱圈于历日之上。或施食贫人,或放生命,一日有多至十余者。至癸未(西元1583年)八月,三千之数已满。复请性空辈,就家庭回向。九月十三日,复起求中进士愿,许行善事一万条,丙戌(西元1586年)登第,授宝坻知县。

余置空格一册,名曰治心篇。晨起坐堂,家人携付门役,置案上,所行善恶,纤悉必记。夜则设桌于庭,效赵阅道焚香告帝。

汝母见所行不多,辄颦蹙曰:“我前在家,相助为善,故三千之数得完;今许一万,衙中无事可行,何时得圆满乎?”

夜间偶梦见一神人,余言善事难完之故。神曰:“只减粮一节,万行俱完矣。”盖宝坻之田,每亩二分三厘七毫。余为区处,减至一分四厘六毫,委有此事,心颇惊疑。适幻余禅师自五台来,余以梦告之,且问此事宜信否?

师曰:“善心真切,即一行可当万善,况合县减粮,万民受福乎?”吾即捐俸银,请其就五台山斋僧一万而回向之。

孔公算予五十三岁有厄,余未尝祈寿,是岁竟无恙,今六十九矣。书曰:“天难谌,命靡常。”又云:“惟命不于常”,皆非诳语。

吾于是而知,凡称祸福自己求之者,乃圣贤之言。若谓祸福惟天所命,则世俗之论矣。

汝之命,未知若何?即命当荣显,常作落寞想;即时当顺利,常作拂逆想;即眼前足食,常作贫窭想;即人相爱敬,常作恐惧想;即家世望重,常作卑下想;即学问颇优,常作浅陋想。

远思扬德,近思盖父母之愆;上思报国之恩,下思造家之福;外思济人之急,内思闲己之邪。

务要日日知非,日日改过;一日不知非,即一日安于自是; 一日无过可改,即一日无步可进;天下聪明俊秀不少,所以德不加修,业不加广者,只为因循二字,耽阁一生。

云谷禅师所授立命之说,乃至精至邃,至真至正之理,其熟读而勉行之,毋自旷也。


\section{1.2   第二篇 改过之法}
\label{\detokenize{p00_u5176_u5b83/_u8881_u4e86_u51e1-_u4e86_u51e1_u56db_u8bad:id4}}
春秋诸大夫,见人言动,亿而谈其祸福,靡不验者,左国诸记可观也。大都吉凶之兆,萌乎心而动乎四体,其过於厚者常获福,过於薄者常近祸,俗眼多翳,谓有未定而不可测者。至诚合天,福之将至,观而必先知之矣。祸之将至,观其不善而必先知之矣。今欲获福而远祸,未论行善,先须改过。

但改过者,第一,要发耻心。思古之圣贤,与我同为丈夫,彼何以百世可师?我何以一身瓦裂?耽染尘情,私行不义,谓人不知,傲然无愧,将日沦於禽兽而不自知矣;世之可羞可耻者,莫大乎此。孟子曰:耻之於人大矣。以其得之则圣贤,失之则禽兽耳。此改过之要机也。

第二,要发畏心。天地在上,鬼神难欺,吾虽过在隐微,而天地鬼神,实鉴临之,重则降之百殃,轻则损其现福,吾何可以不惧?不惟此也。闲居之地,指视昭然;吾虽掩之甚密,文之甚巧,而肺肝早露,终难自欺;被人觑破,不值一文矣,乌得不懔懔?不惟是也。一息尚存,弥天之恶,犹可悔改;古人有一生作恶,临死悔悟,发一善念,遂得善终者。谓一念猛厉,足以涤百年之恶也。譬如千年幽谷,一灯才照,则千年之暗俱除;故过不论久近,惟以改为贵。但尘世无常,肉身易殒,一息不属,欲改无由矣。明则千百年担负恶名,虽孝子慈孙,不能洗涤;幽则千百劫沈沦狱报,虽圣贤佛菩萨,不能援引。乌得不畏?

第三,须发勇心。人不改过,多是因循退缩;吾须奋然振作,不用迟疑,不烦等待。小者如芒刺在肉,速与抉剔;大者如毒蛇啮指,速与斩除,无丝毫凝滞,此风雷之所以为益也。

具是三心,则有过斯改,如春冰遇日,何患不消乎?然人之过,有从事上改者,有从理上改者,有从心上改者;工夫不同,效验亦异。

如前日杀生,今戒不杀;前日怒詈,今戒不怒;此就其事而改之者也。强制於外,其难百倍,且病根终在,东灭西生,非究竟廓然之道也。

善改过者,未禁其事,先明其理;如过在杀生,即思曰:上帝好生,物皆恋命,杀彼养己,岂能自安?且彼之杀也,既受屠割,复入鼎镬,种种痛苦,彻入骨髓;己之养也,珍膏罗列,食过即空,疏食菜羹,尽可充腹,何必戕彼之生,损己之福哉?又思血气之属,皆含灵知,既有灵知,皆我一体;纵不能躬修至德,使之尊我亲我,岂可日戕物命,使之仇我憾我於无穷也?一思及此,将有对食痛心,不能下咽者矣。

如前日好怒,必思曰:人有不及,情所宜矜;悖理相干,於我何与?本无可怒者。又思天下无自是之豪杰,亦无尤人之学问;有不得,皆己之德未修,感未至也。吾悉以自反,则谤毁之来,皆磨炼玉成之地;我将欢然受赐,何怒之有?又闻而不怒,虽谗焰薰天,如举火焚空,终将自息;闻谤而怒,虽巧心力辩,如春蚕作茧,自取缠绵;怒不惟无益,且有害也。其馀种种过恶,皆当据理思之。此理既明,过将自止。

何谓从心而改?过有千端,惟心所造;吾心不动,过安从生?学者於好色,好名,好货,好怒,种种诸过,不必逐类寻求;但当一心为善,正念现前,邪念自然污染不上。如太阳当空,魍魉潜消,此精一之真传也。过由心造,亦由心改,如斩毒树,直断其根,奚必枝枝而伐,叶叶而摘哉?

大抵最上治心,当下清净;才动即觉,觉之即无;苟未能然,须明理以遣之;又未能然,须随事以禁之;以上事而兼行下功,未为失策。执下而昧上,则拙矣。

顾发愿改过,明须良朋提醒,幽须鬼神证明;一心忏悔,昼夜不懈,经一七,二七,以至一月,二月,三月,必有效验。

或觉心神恬旷;或觉智慧顿开;或处冗沓而触念皆通;或遇怨仇而回镇作喜;或梦吐黑物;或梦往圣先贤,提携接引;或梦飞步太虚;或梦幢幡宝盖,种种胜事,皆过消灭之象也。然不得执此自高,画而不进。

昔蘧伯玉当二十岁时,已觉前日之非而尽改之矣。至二十一岁,乃知前之所改,未尽也;及二十二岁,回视二十一岁,犹在梦中,岁复一岁,递递改之,行年五十,而犹知四十九年之非,古人改过之学如此。

吾辈身为凡流,过恶猬集,而回思往事,常若不见其有过者,心粗而眼翳也。然人之过恶深重者,亦有效验:或心神昏塞,转头即忘;或无事而常烦恼;或见君子而赧然相沮;或闻正论而不乐;或施惠而人反怨;或夜梦颠倒,甚则妄言失志;皆作孽之相也,苟一类此,即须奋发,舍旧图新,幸勿自误。


\section{1.3   第三篇 积善之方}
\label{\detokenize{p00_u5176_u5b83/_u8881_u4e86_u51e1-_u4e86_u51e1_u56db_u8bad:id5}}
易曰:「积善之家,必有馀庆。」昔颜氏将以女妻叔梁纥,而历叙其祖宗积德之长,逆知其子孙必有兴者。孔子称舜之大孝,曰:「宗庙飨之,子孙保之」,皆至论也。试以往事徵之。

杨少师荣,建宁人。世以济渡为生,久雨溪涨,横流冲毁民居,溺死者顺流而下,他舟皆捞取货物,独少师曾祖及祖,惟救人,而货物一无所取,乡人嗤其愚。逮少师父生,家渐裕,有神人化为道者,语之曰:「汝祖父有阴功,子孙当贵显,宜葬某地。」遂依其所指而窆之,即今白兔坟也。后生少师,弱冠登第,位至三公,加曾祖,祖,父,如其官。子孙贵盛,至今尚多贤者。

鄞人杨自惩,初为县吏,存心仁厚,守法公平。时县宰严肃,偶挞一囚,血流满前,而怒犹未息,杨跪而宽解之。宰曰:「怎奈此人越法悖理,不由人不怒。」自惩叩首曰:「上失其道,民散久矣,如得其情,哀矜勿喜;喜且不可,而况怒乎?」宰为之霁颜。

家甚贫,馈遗一无所取,遇囚人乏粮,常多方以济之。一日,有新囚数人待哺,家又缺米;给囚则家人无食;自顾则囚人堪悯;与其妇商之。

妇曰:「囚从何来?」

曰:「自杭而来。沿路忍饥,菜色可掬。」因撤己之米,煮粥以食囚。后生二子,长曰守陈,次曰守址,为南北吏部侍郎;长孙为刑部侍郎;次孙为四川廉宪,又俱为名臣;今楚亭,德政,亦其裔也。

昔正统间,邓茂七倡乱於福建,士民从贼者甚众;朝廷起鄞县张都宪楷南征,以计擒贼,后委布政司谢都事,搜杀东路贼党;谢求贼中党附册籍,凡不附贼者,密授以白布小旗,约兵至日,插旗门首,戒军兵无妄杀,全活万人;后谢之子迁,中状元,为宰辅;孙丕,复中探花。

莆田林氏,先世有老母好善,常作粉团施人,求取即与之,无倦色;一仙化为道人,每旦索食六七团。母日日与之,终三年如一日,乃知其诚也。因谓之曰:「吾食汝三年粉团,何以报汝?府后有一地,葬之,子孙官爵,有一升麻子之数。」其子依所点葬之,初世即有九人登第,累代簪缨甚盛,福建有无林不开榜之谣。

冯琢庵太史之父,为邑庠生。隆冬早起赴学,路遇一人,倒卧雪中,扪之,半僵矣。遂解己绵裘衣之,且扶归救苏。梦神告之曰:「汝救人一命,出至诚心,吾遣韩琦为汝子。」及生琢庵,遂名琦。

台州应尚书,壮年习业於山中。夜鬼啸集,往往惊人,公不惧也;一夕闻鬼云:「某妇以夫久客不归,翁姑逼其嫁人。明夜当缢死於此,吾得代矣。」公潜卖田,得银四两。即伪作其夫之书,寄银还家;其父母见书,以手迹不类,疑之。既而曰:「书可假,银不可假,想儿无恙。」妇遂不嫁。其子后归,夫妇相保如初。

公又闻鬼语曰:「我当得代,奈此秀才坏吾事。」

旁一鬼曰:「尔何不祸之?」

曰:「上帝以此人心好,命作阴德尚书矣,吾何得而祸之?」应公因此益自努励,善日加修,德日加厚;遇岁饥,辄捐谷以赈之;遇亲戚有急,辄委曲维持;遇有横逆,辄反躬自责,怡然顺受;子孙登科第者,今累累也。

常熟徐凤竹〔木式〕,其父素富,偶遇年荒,先捐租以为同邑之倡,又分谷以赈贫乏,夜闻鬼唱於门曰:「千不诓,万不诓;徐家秀才,做到了举人郎。」相续而呼,连夜不断。是岁,凤竹果举於乡,其父因而益积德,孳孳不怠,修桥修路,斋僧接众,凡有利益,无不尽心。后又闻鬼唱於门曰:「千不诓,万不诓;徐家举人,直做到都堂。」凤竹官终两浙巡抚。

喜兴屠康僖公,初为刑部主事,宿狱中,细询诸囚情状,得无辜者若干人,公不自以为功,密疏其事,以白堂官。后朝审,堂官摘其语,以讯诸囚,无不服者,释冤抑十馀人。一时辇下咸颂尚书之明。

公复禀曰:「辇毂之下,尚多冤民,四海之广,兆民之众,岂无枉者?宜五年差一减刑官,核实而平反之。」尚书为奏,允其议。时公亦差减刑之列,梦一神告之曰:「汝命无子,今减刑之议,深合天心,上帝赐汝三子,皆衣紫腰金。」是夕夫人有娠,后生应埙,应坤,应【俊】,皆显官。

嘉兴包凭,字信之,其父为池阳太守,生七子,凭最少,赘平湖袁氏,与吾父往来甚厚,博学高才,累举不第,留心二氏之学。一日东游泖湖,偶至一村寺中,见观音像,淋漓露立,即解橐中十金,授主僧,令修屋宇,僧告以功大银少,不能竣事;复取松布四疋,检箧中衣七件与之,内〔纟宁〕褶,系新置,其仆请已之。

凭曰:「但得圣像无恙,吾虽裸裎何伤?」

僧垂泪曰:「舍银及衣布,犹非难事。只此一点心,如何易得。」后功完,拉老父同游,宿寺中。公梦伽蓝来曰:「汝子当享世禄矣。」后子汴,孙柽芳,皆登第,作显官。

嘉善支立之父,为刑房吏,有囚无辜陷重辟,意哀之,欲求其生。囚语其妻曰:「支公嘉意,愧无以报,明日延之下乡,汝以身事之,彼或肯用意,则我可生也。」其妻泣而听命。及至,妻自出劝酒,具告以夫意。支不听,卒为尽力平反之。囚出狱,夫妻登门叩谢曰:「公如此厚德,晚世所稀,今无子,吾有弱女,送为箕帚妾,此则礼之可通者。」支为备礼而纳之,生立,弱冠中魁,官至翰林孔目,立生高,高生禄,皆贡为学博。禄生大纶,登第。

凡此十条,所行不同,同归於善而已。若复精而言之,则善有真,有假;有端,有曲;有阴,有阳;有是,有非;有偏,有正;有半,有满;有大,有小;有难,有易;皆当深辨。为善而不穷理,则自谓行持,岂知造孽,枉费苦心,无益也。

何谓真假?昔有儒生数辈,谒中峰和尚,

问曰:「佛氏论善恶报应,如影随形。今某人善,而子孙不兴;某人恶,而家门隆盛;佛说无稽矣。」

中峰云:「凡情未涤,正眼未开,认善为恶,指恶为善,往往有之。不憾己之是非颠倒,而反怨天之报应有差乎?」

众曰:「善恶何致相反?」中峰令试言。

一人谓「詈人殴人是恶;敬人礼人是善。」

中峰云:「未必然也。」

一人谓「贪财妄取是恶,廉洁有守是善。」

中峰云:「未必然也。」众人历言其状,中峰皆谓不然。因请问。

中峰告之曰:「有益於人,是善;有益於己,是恶。有益於人,则殴人,詈人皆善也;有益於己,则敬人,礼人皆恶也。是故人之行善,利人者公,公则为真;利己者私,私则为假。又根心者真,袭迹者假;又无为而为者真,有为而为者假;皆当自考。」

何谓端曲?今人见谨愿之士,类称为善而取之;圣人则宁取狂狷。至於谨愿之士,虽一乡皆好,而必以为德之贼;是世人之善恶,分明与圣人相反。推此一端,种种取舍,无有不谬;天地鬼神之福善祸淫,皆与圣人同是非,而不与世俗同取舍。凡欲积善,决不可徇耳目,惟从心源隐微处,默默洗涤,纯是济世之心,则为端;苟有一毫媚世之心,即为曲;纯是爱人之心,则为端;有一毫愤世之心,即为曲;纯是敬人之心,则为端;有一毫玩世之心,即为曲;皆当细辨。 何谓阴阳?凡为善而人知之,则为阳善;为善而人不知,则为阴德。阴德,天报之;阳善,享世名。名,亦福也。名者,造物所忌;世之享盛名而实不副者,多有奇祸;人之无过咎而横被恶名者,子孙往往骤发,阴阳之际微矣哉。

何谓是非?鲁国之法,鲁人有赎人臣妾於诸侯,皆受金於府,子贡赎人而不受金。孔子闻而恶之曰:「赐失之矣。夫圣人举事,可以移风易俗,而教道可施於百姓,非独适己之行也。今鲁国富者寡而贫者众,受金则为不廉,何以相赎乎?自今以后,不复赎人於诸侯矣。」

子路拯人於溺,其人谢之以牛,子路受之。孔子喜曰:「自今鲁国多拯人於溺矣。」自俗眼观之,子贡不受金为优,子路之受牛为劣;孔子则取由而黜赐焉。乃知人之为善,不论现行而论流弊;不论一时而论久远;不论一身而论天下。现行虽善,其流足以害人;则似善而实非也;现行虽不善,而其流足以济人,则非善而实是也。然此就一节论之耳。他如非义之义,非礼之礼,非信之信,非慈之慈,皆当抉择。

何谓偏正?昔吕文懿公,初辞相位,归故里,海内仰之,如泰山北斗。有一乡人,醉而詈之,吕公不动,谓其仆曰:「醉者勿与较也。」闭门谢之。逾年,其人犯死刑入狱。吕公始悔之曰:「使当时稍与计较,送公家责治,可以小惩而大戒;吾当时只欲存心於厚,不谓养成其恶,以至於此。」此以善心而行恶事者也。

又有以恶心而行善事者。如某家大富,值岁荒,穷民白昼抢粟於市;告之县,县不理,穷民愈肆,遂私执而困辱之,众始定;不然,几乱矣。故善者为正,恶者为偏,人皆知之;其以善心行恶事者,正中偏也;以恶心而行善事者,偏中正也;不可不知也。

何谓半满?易曰:「善不积,不足以成名;恶不积,不足以灭身。」书曰:「商罪贯盈,如贮物於器。」勤而积之,则满;懈而不积,则不满。此一说也。

昔有某氏女入寺,欲施而无财,止有钱二文,捐而与之,主席者亲为忏悔;及后入宫富贵,携数千金入寺舍之,主僧惟令其徒回向而已。

因问曰:「吾前施钱二文,师亲为忏悔,今施数千金,而师不回向,何也?」

曰:「前者物虽薄,而施心甚真,非老僧亲忏,不足报德;今物虽厚,而施心不若前日之切,令人代忏足矣。」此千金为半,而二文为满也。

锺离授丹於吕祖,点铁为金,可以济世。

吕问曰:「终变否?」

曰:「五百年后,当复本质。」

吕曰:「如此则害五百年后人矣,吾不愿为也。」

曰:「修仙要积三千功行,汝此一言,三千功行已满矣。」此又一说也。

又为善而心不著善,则随所成就,皆得圆满。心著於善,虽终身勤励,止於半善而已。譬如以财济人,内不见己,外不见人,中不见所施之物,是谓三轮体空,是谓一心清净,则斗粟可以种无涯之福,一文可以消千劫之罪,倘此心未忘,虽黄金万镒,福不满也。此又一说也。 何谓大小?昔卫仲达为馆职,被摄至冥司,主者命吏呈善恶二录,比至,则恶录盈庭,其善录一轴,仅如筋而已。索秤称之,则盈庭者反轻,而如筋者反重。

仲达曰:「某年未四十,安得过恶如是多乎?」

曰:「一念不正即是,不待犯也。」因问轴中所书何事?

曰:「朝廷尝兴大工,修三山石桥,君上疏谏之,此疏稿也。」

仲达曰:「某虽言,朝廷不从,於事无补,而能有如是之力。」

曰:「朝廷虽不从,君之一念,已在万民;向使听从,善力更大矣。」故志在天下国家,则善虽少而大;苟在一身,虽多亦小。

何谓难易?先儒谓克己须从难克处克将去。夫子论为仁,亦曰先难。必如江西舒翁,舍二年仅得之束修,代偿官银,而全人夫妇;与邯郸张翁,舍十年所积之钱,代完赎银,而活人妻子,皆所谓难舍处能舍也。如镇江靳翁,虽年老无子,不忍以幼女为妾,而还之邻,此难忍处能忍也;故天降之福亦厚。凡有财有势者,其立德皆易,易而不为,是为自暴。贫贱作福皆难,难而能为,斯可贵耳。

随缘济众,其类至繁,约言其纲,大约有十:第一,与人为善;第二,爱敬存心;第三,成人之美;第四,劝人为善;第五,救人危急;第六,兴建大利;第七,舍财作福;第八,护持正法;第九,敬重尊长;第十,爱惜物命。

何谓与人为善?昔舜在雷泽,见渔者皆取深潭厚泽,而老弱则渔於急流浅滩之中,恻然哀之,往而渔焉;见争者皆匿其过而不谈,见有让者,则揄扬而取法之。期年,皆以深潭厚泽相让矣。夫以舜之明哲,岂不能出一言教众人哉?乃不以言教而以身转之,此良工苦心也。

吾辈处未世,勿以己之长而盖人;勿以己之善而形人;勿以己之多能而困人。收敛才智,若无若虚;见人过失,且涵容而掩覆之。一则令其可改,一则令其有所顾忌而不敢纵,见人有微长可取,小善可录,翻然舍己而从之;且为艳称而广述之。凡日用间,发一言,行一事,全不为自己起念,全是为物立则;此大人天下为公之度也。

何谓爱敬存心?君子与小人,就形迹观,常易相混,惟一点存心处,则善恶悬绝,判然如黑白之相反。故曰:君子所以异於人者,以其存心也。君子所存之心,只是爱人敬人之心。盖人有亲疏贵贱,有智愚贤不肖;万品不齐,皆吾同胞,皆吾一体,孰非当敬爱者?爱敬众人,即是爱敬圣贤;能通众人之志,即是通圣贤之志。何者?圣贤志,本欲斯世斯人,各得其所。吾合爱合敬,而安一世之人,即是为圣贤而安之也。

何谓成人之美?玉之在石,抵掷则瓦砾,追琢则圭璋;故凡见人行一善事,或其人志可取而资可进,皆须诱掖而成就之。或为之奖借,或为之维持;或为白其诬而分其谤;务使成立而后已。

大抵人各恶其非类,乡人之善者少,不善者多。善人在俗,亦难自立。且豪杰铮铮,不甚修形迹,多易指摘;故善事常易败,而善人常得谤;惟仁人长者,匡直而辅翼之,其功德最宏。

何谓劝人为善?生为人类,孰无良心?世路役役,最易没溺。凡与人相处,当方便提撕,开其迷惑。譬犹长夜大梦,而令之一觉;譬犹久陷烦恼,而拔之清凉,为惠最溥。韩愈云:「一时劝人以口,百世劝人以书。」较之与人为善,虽有形迹,然对证发药,时有奇效,不可废也;失言失人,当反吾智。

何谓救人危急?患难颠沛,人所时有。偶一遇之,当如恫【环】在身,速为解救。或以一言伸其屈抑;或以多方济其颠连。崔子曰:「惠不在大,赴人之急可也。」盖仁人之言哉。

何谓兴建大利?小而一乡之内,大而一邑之中,凡有利益,最宜兴建;或开渠导水,或筑堤防患;或修桥梁,以便行旅;或施茶饭,以济饥渴;随缘劝导,协力兴修,勿避嫌疑,勿辞劳怨。

何谓舍财作福?释门万行,以布施为先。所谓布施者,只是舍之一字耳。达者内舍六根,外舍六尘,一切所有,无不舍者。苟非能然,先从财上布施。世人以衣食为命,故财为最重。吾从而舍之,内以破吾之悭,外以济人之急;始而勉强,终则泰然,最可以荡涤私情,〔衤去〕除执吝。

何谓护持正法?法者,万世生灵之眼目也。不有正法,何以参赞天地?何以裁成万物?何以脱尘离缚?何以经世出世?故凡见圣贤庙貌,经书典籍,皆当敬重而修饬之。至於举扬正法,上报佛恩,尤当勉励。

何谓敬重尊长?家之父兄,国之君长,与凡年高,德高,位高,识高者,皆当加意奉事。在家而奉侍父母,使深爱婉容,柔声下气,习以成性,便是和气格天之本。出而事君,行一事,毋谓君不知而自恣也。刑一人,毋谓君不知而作威也。事君如天,古人格论,此等处最关阴德。试看忠孝之家,子孙未有不绵远而昌盛者,切须慎之。

何谓爱惜物命?凡人之所以为人者,惟此恻隐之心而已;求仁者求此,积德者积此。周礼,「孟春之月,牺牲毋用牝。」孟子谓君子远庖厨,所以全吾恻隐之心也。故前辈有四不食之戒,谓闻杀不食,见杀不食,自养者不食,专为我杀者不食。学者未能断肉,且当从此戒之。

渐渐增进,慈心愈长,不特杀生当戒,蠢动含灵,皆为物命。求丝煮茧,锄地杀虫,念衣食之由来,皆杀彼以自活。故暴殄之孽,当与杀生等。至於手所误伤,足所误践者,不知其几,皆当委曲防之。古诗云:「爱鼠常留饭,怜蛾不点灯。」何其仁也!

善行无穷,不能殚述;由此十事而推广之,则万德可备矣。


\section{1.4   第四篇 谦德之效}
\label{\detokenize{p00_u5176_u5b83/_u8881_u4e86_u51e1-_u4e86_u51e1_u56db_u8bad:id6}}
易曰:「天道亏盈而益谦;地道变盈而流谦;鬼神害盈而福谦;人道恶盈而好谦。」是故谦之一卦,六爻皆吉。书曰:「满招损,谦受益。」予屡同诸公应试,每见寒士将达,必有一段谦光可掬。

辛未(西元1571年)计偕,我嘉善同袍凡十人,惟丁敬宇宾,年最少,极其谦虚。

予告费锦坡曰:「此兄今年必第。」

费曰:「何以见之?」

予曰:「惟谦受福。兄看十人中,有恂恂款款,不敢先人,如敬宇者乎?有恭敬顺承,小心谦畏,如敬宇者乎?有受侮不答,闻谤不辩,如敬宇者乎?人能如此,即天地鬼神,犹将佑之,岂有不发者?」及开榜,丁果中式。

丁丑(西元1577年)在京,与冯开之同处,见其虚己敛容,大变其幼年之习。李霁岩直谅益友,时面攻其非,但见其平怀顺受,未尝有一言相报。予告之曰:「福有福始,祸有祸先,此心果谦,天必相之,兄今年决第矣。」已而果然。

赵裕峰,光远,山东冠县人,童年举於乡,久不第。其父为嘉善三尹,随之任。慕钱明吾,而执文见之,明吾悉抹其文,赵不惟不怒,且心服而速改焉。明年,遂登第。

壬辰岁(西元1592年),予入觐,晤夏建所,见其人气虚意下,谦光逼人,归而告友人曰:「凡天将发斯人也,未发其福,先发其慧;此慧一发,则浮者自实,肆者自敛;建所温良若此,天启之矣。」及开榜,果中式。

江阴张畏岩,积学工文,有声艺林。甲午(西元1594年),南京乡试,寓一寺中,揭晓无名,大骂试官,以为眯目。时有一道者,在傍微笑,张遽移怒道者。

道者曰:「相公文必不佳。」

张怒曰:「汝不见我文,乌知不佳?」

道者曰:「闻作文,贵心气和平,今听公骂詈,不平甚矣,文安得工?」张不觉屈服,因就而请教焉。

道者曰:「中全要命;命不该中,文虽工,无益也。须自己做个转变。」

张曰:「既是命,如何转变?」

道者曰:「造命者天,立命者我;力行善事,广积阴德,何福不可求哉?」

张曰:「我贫士,何能为?」

道者曰:「善事阴功,皆由心造,常存此心,功德无量,且如谦虚一节,并不费钱,你如何不自反而骂试官乎?」

张由此折节自持,善日加修,德日加厚。丁酉(西元1597年),梦至一高房,得试录一册,中多缺行。问旁人,

曰:「此今科试录。」

问:「何多缺名?」

曰:「科第阴间三年一考较,须积德无咎者,方有名。如前所缺,皆系旧该中式,因新有薄行而去之者也。」

后指一行云:「汝三年来,持身颇慎,或当补此,幸自爱。」是科果中一百五名。

由此观之,举头三尺,决有神明;趋吉避凶,断然由我。须使我存心制行,毫不得罪於天地鬼神,而虚心屈己,使天地鬼神,时时怜我,方有受福之基。彼气盈者,必非远器,纵发亦无受用。稍有识见之士,必不忍自狭其量,而自拒其福也,况谦则受教有地,而取善无穷,尤修业者所必不可少者也。

古语云:「有志於功名者,必得功名;有志於富贵者,必得富贵。」人之有志,如树之有根,立定此志,须念念谦虚,尘尘方便,自然感动天地,而造福由我。今之求登科第者,初未尝有真志,不过一时意兴耳;兴到则求,兴阑则止。孟子曰:「王之好乐甚,齐其庶几乎?」予於科名亦然。


\section{1.5   【袁了凡居士传】}
\label{\detokenize{p00_u5176_u5b83/_u8881_u4e86_u51e1-_u4e86_u51e1_u56db_u8bad:id7}}
(原文系文言文,为清朝彭绍升撰)

袁了凡先生,本名袁黄,字坤仪;江苏省吴江县人。年轻时入赘到浙江省嘉善县姓殳的人家;因此,在嘉善县得了公费做县里的公读生。他於明穆宗隆庆四年(西元一五七○年),在乡里中了举人;明神宗万历十四年(西元一五八六年)考上进士,奉命到河北省宝坻县做县长。过了七年升拔为兵部「职方司」的主管人,任中刚好碰到日寇侵犯朝鲜,朝鲜向中国求救兵。当时的「经略」(驻朝鲜军事长官)宋应昌奏准请了凡为「军前赞画」(参谋长)的职务,并兼督导支援朝鲜的军队。提督李如松掌握兵权,假装赐给高官俸禄与日寇谈和,日寇信以为真,没有设防;李如松发动突击,攻破形势险要的平壤,因而打败了日寇。

了凡先生因为这件事当面指责李如松,不应用诡诈的手段对付日寇,这样有损大明朝的国威;而且李如松手下的士兵随便杀害百姓,并以头来记功。了凡向李如松据理力争,李如松发怒;不但不接受劝诫,反而独自带著军队东走,使得了凡所率领的军队孤立无援。日寇因而乘机攻击了凡的军队,幸赖了凡机智应对,将日寇击退。而李如松的军队,最后终於被日寇击败了;他想要脱却自己的罪状,反而以十项罪名弹劾袁了凡;了凡很快地被提出审判,终於在「拾遗」(谏官)的仕内,被迫停职返乡。在家里,了凡非常恳切,认真地行善直到去世,过逝时享年七十四岁。

明熹宗天启年间,了凡的冤案终於真相大白,朝廷追叙了凡征讨日寇的功绩,赠封他为「尚宝司少卿」的官衔。了凡先生从当学生时,就非常喜欢研究学问,书不论古今,事不分轻重,他都认真研究,并且非常通达。例如:星象,法律,水利,理数,兵备,政治,堪舆等。

了凡先生在宝坻县当县长时,非常注重人民的福利,常常想做些有利地方的事情;宝坻县当时常有水灾泛滥,了凡先生於是积极兴办水利,将三汊河疏通,筑堤防以抵挡水患侵袭;并且教导百姓沿著海岸种植柳树,每当海水泛滥,挟带沙土冲上岸时,遇到柳树就积挡下来,久而久之变成一道堤防。於是了凡先生又督导百姓在堤防上建造沟渠,并鼓励百姓耕种;因此,荒废的土地渐渐地开垦,了凡先生又免除百姓种种杂役以便民,使得百姓安居乐业。

了凡先生家里并不富有,可是却非常喜欢布施,家居生活俭朴,每天诵经持咒,参禅打坐,修习止观。不管公私事务再忙,早晚定课从不间断。在这当中,了凡先生写下四篇短文,当时命名为「戒子文」,用来训诫他儿子,就是后来广行於世的「了凡四训」这本书。

了凡先生的夫人非常贤慧,经常帮助他行善布施,并且依照功过格记下所做的功德,因为她没有读过书,不会写字;因此用鹅毛管沾红墨水,每天在历书上做记号。有时了凡先生较忙,当天所做功德较少,她就皱眉头,希望先生能多做些善事。有一次,她为儿子裁制冬天的大袍子,想买棉絮做内里。

了凡先生问:「家里有丝绵又轻又暖和,为什麽还买棉絮呢?」

了凡夫人答:「丝绵较贵,棉絮便宜,我想将家里的丝绵拿去换棉絮,这样可以多裁几件棉袄,赠送给贫寒的人家过冬!」

了凡先生听了非常高兴说:「你这样虔诚的布施,不怕我们孩子没有福报了!」他们的儿子袁俨,后来中了进士,最后以广东省高要县的县长退休。


\section{1.6   【袁了凡居士传】【注】}
\label{\detokenize{p00_u5176_u5b83/_u8881_u4e86_u51e1-_u4e86_u51e1_u56db_u8bad:id8}}\begin{enumerate}
\sphinxsetlistlabels{\arabic}{enumi}{enumii}{}{.}%
\item {} 
代用字:

\end{enumerate}
\begin{itemize}
\item {} 
【俊】:如「俊」字形,「人」旁换成「土」旁

\item {} 
【环】:取「环」字右侧,填入「病」中「丙」字的位置

\end{itemize}
\begin{enumerate}
\sphinxsetlistlabels{\arabic}{enumi}{enumii}{}{.}%
\setcounter{enumi}{1}
\item {} 
本文输入和初校所据如下:

\end{enumerate}

了凡四训白话解释【精简本】

著作:明朝,袁了凡

演述:民初,黄智海

整理:民国,王丽民


\chapter{1   Hi,p01散文}
\label{\detokenize{p01_u6563_u6587/Hello_uff0cp01_u6563_u6587:hi-p01}}\label{\detokenize{p01_u6563_u6587/Hello_uff0cp01_u6563_u6587::doc}}
\begin{sphinxShadowBox}
\sphinxstyletopictitle{目录}
\begin{itemize}
\item {} 
\phantomsection\label{\detokenize{p01_u6563_u6587/Hello_uff0cp01_u6563_u6587:id2}}{\hyperref[\detokenize{p01_u6563_u6587/Hello_uff0cp01_u6563_u6587:hi-p01}]{\sphinxcrossref{1   Hi,p01散文}}}
\begin{itemize}
\item {} 
\phantomsection\label{\detokenize{p01_u6563_u6587/Hello_uff0cp01_u6563_u6587:id3}}{\hyperref[\detokenize{p01_u6563_u6587/Hello_uff0cp01_u6563_u6587:post}]{\sphinxcrossref{1.1   post}}}

\end{itemize}

\end{itemize}
\end{sphinxShadowBox}


\section{1.1   post}
\label{\detokenize{p01_u6563_u6587/Hello_uff0cp01_u6563_u6587:post}}

\chapter{1   宋濂-送东阳马生序}
\label{\detokenize{p01_u6563_u6587/_u5b8b_u6fc2-_u9001_u4e1c_u9633_u9a6c_u751f_u5e8f:id1}}\label{\detokenize{p01_u6563_u6587/_u5b8b_u6fc2-_u9001_u4e1c_u9633_u9a6c_u751f_u5e8f::doc}}
\begin{sphinxShadowBox}
\sphinxstyletopictitle{目录}
\begin{itemize}
\item {} 
\phantomsection\label{\detokenize{p01_u6563_u6587/_u5b8b_u6fc2-_u9001_u4e1c_u9633_u9a6c_u751f_u5e8f:id14}}{\hyperref[\detokenize{p01_u6563_u6587/_u5b8b_u6fc2-_u9001_u4e1c_u9633_u9a6c_u751f_u5e8f:id1}]{\sphinxcrossref{1   宋濂-送东阳马生序}}}
\begin{itemize}
\item {} 
\phantomsection\label{\detokenize{p01_u6563_u6587/_u5b8b_u6fc2-_u9001_u4e1c_u9633_u9a6c_u751f_u5e8f:id15}}{\hyperref[\detokenize{p01_u6563_u6587/_u5b8b_u6fc2-_u9001_u4e1c_u9633_u9a6c_u751f_u5e8f:id3}]{\sphinxcrossref{1.1   作品原文}}}

\item {} 
\phantomsection\label{\detokenize{p01_u6563_u6587/_u5b8b_u6fc2-_u9001_u4e1c_u9633_u9a6c_u751f_u5e8f:id16}}{\hyperref[\detokenize{p01_u6563_u6587/_u5b8b_u6fc2-_u9001_u4e1c_u9633_u9a6c_u751f_u5e8f:id4}]{\sphinxcrossref{1.2   词句注释}}}

\item {} 
\phantomsection\label{\detokenize{p01_u6563_u6587/_u5b8b_u6fc2-_u9001_u4e1c_u9633_u9a6c_u751f_u5e8f:id17}}{\hyperref[\detokenize{p01_u6563_u6587/_u5b8b_u6fc2-_u9001_u4e1c_u9633_u9a6c_u751f_u5e8f:id5}]{\sphinxcrossref{1.3   白话译文}}}

\item {} 
\phantomsection\label{\detokenize{p01_u6563_u6587/_u5b8b_u6fc2-_u9001_u4e1c_u9633_u9a6c_u751f_u5e8f:id18}}{\hyperref[\detokenize{p01_u6563_u6587/_u5b8b_u6fc2-_u9001_u4e1c_u9633_u9a6c_u751f_u5e8f:id6}]{\sphinxcrossref{1.4   创作背景}}}

\item {} 
\phantomsection\label{\detokenize{p01_u6563_u6587/_u5b8b_u6fc2-_u9001_u4e1c_u9633_u9a6c_u751f_u5e8f:id19}}{\hyperref[\detokenize{p01_u6563_u6587/_u5b8b_u6fc2-_u9001_u4e1c_u9633_u9a6c_u751f_u5e8f:id7}]{\sphinxcrossref{1.5   作品鉴赏}}}
\begin{itemize}
\item {} 
\phantomsection\label{\detokenize{p01_u6563_u6587/_u5b8b_u6fc2-_u9001_u4e1c_u9633_u9a6c_u751f_u5e8f:id20}}{\hyperref[\detokenize{p01_u6563_u6587/_u5b8b_u6fc2-_u9001_u4e1c_u9633_u9a6c_u751f_u5e8f:id8}]{\sphinxcrossref{1.5.1   第一段}}}

\item {} 
\phantomsection\label{\detokenize{p01_u6563_u6587/_u5b8b_u6fc2-_u9001_u4e1c_u9633_u9a6c_u751f_u5e8f:id21}}{\hyperref[\detokenize{p01_u6563_u6587/_u5b8b_u6fc2-_u9001_u4e1c_u9633_u9a6c_u751f_u5e8f:id9}]{\sphinxcrossref{1.5.2   第二段}}}

\item {} 
\phantomsection\label{\detokenize{p01_u6563_u6587/_u5b8b_u6fc2-_u9001_u4e1c_u9633_u9a6c_u751f_u5e8f:id22}}{\hyperref[\detokenize{p01_u6563_u6587/_u5b8b_u6fc2-_u9001_u4e1c_u9633_u9a6c_u751f_u5e8f:id10}]{\sphinxcrossref{1.5.3   第三段}}}

\item {} 
\phantomsection\label{\detokenize{p01_u6563_u6587/_u5b8b_u6fc2-_u9001_u4e1c_u9633_u9a6c_u751f_u5e8f:id23}}{\hyperref[\detokenize{p01_u6563_u6587/_u5b8b_u6fc2-_u9001_u4e1c_u9633_u9a6c_u751f_u5e8f:id11}]{\sphinxcrossref{1.5.4   总结}}}

\end{itemize}

\item {} 
\phantomsection\label{\detokenize{p01_u6563_u6587/_u5b8b_u6fc2-_u9001_u4e1c_u9633_u9a6c_u751f_u5e8f:id24}}{\hyperref[\detokenize{p01_u6563_u6587/_u5b8b_u6fc2-_u9001_u4e1c_u9633_u9a6c_u751f_u5e8f:id12}]{\sphinxcrossref{1.6   名家点评}}}

\item {} 
\phantomsection\label{\detokenize{p01_u6563_u6587/_u5b8b_u6fc2-_u9001_u4e1c_u9633_u9a6c_u751f_u5e8f:id25}}{\hyperref[\detokenize{p01_u6563_u6587/_u5b8b_u6fc2-_u9001_u4e1c_u9633_u9a6c_u751f_u5e8f:id13}]{\sphinxcrossref{1.7   作者简介}}}

\end{itemize}

\end{itemize}
\end{sphinxShadowBox}

《送东阳马生序》是明代文学家宋濂创作的一篇赠序。在这篇赠序里,作者叙述个人早年虚心求教和勤苦学习的经历,生动而具体地描述了自己借书求师之难,饥寒奔走之苦,并与太学生优越的条件加以对比,有力地说明学业能否有所成就,主要在于主观努力,不在天资的高下和条件的优劣,以勉励青年人珍惜良好的读书环境,专心治学。全文结构严谨,详略有致,用对比说理,在叙事中穿插细节描绘,读来生动感人。


\section{1.1   作品原文}
\label{\detokenize{p01_u6563_u6587/_u5b8b_u6fc2-_u9001_u4e1c_u9633_u9a6c_u751f_u5e8f:id3}}
送东阳马生序1

余幼时即嗜学2。家贫,无从致书以观3,每假借于藏书之家4,手自笔录,计日以还。天大寒,砚冰坚,手指不可屈伸,弗之怠5。录毕,走送之6,不敢稍逾约7。以是人多以书假余,余因得遍观群书。既加冠8,益慕圣贤之道9,又患无硕师、名人与游10,尝趋百里外11,从乡之先达执经叩问12。先达德隆望尊,门人弟子填其室,未尝稍降辞色13。余立侍左右,援疑质理14,俯身倾耳以请;或遇其叱咄15,色愈恭,礼愈至,不敢出一言以复;俟其欣悦16,则又请焉。故余虽愚,卒获有所闻17。

当余之从师也,负箧曳屣18,行深山巨谷中,穷冬烈风19,大雪深数尺,足肤皲裂而不知20。至舍,四支僵劲不能动21,媵人持汤沃灌22,以衾拥覆23,久而乃和。寓逆旅24,主人日再食25,无鲜肥滋味之享。同舍生皆被绮绣26,戴朱缨宝饰之帽27,腰白玉之环28,左佩刀,右备容臭29,烨然若神人30;余则缊袍敝衣处其间31,略无慕艳意。以中有足乐者,不知口体之奉不若人也。盖余之勤且艰若此。

今虽耄老32,未有所成,犹幸预君子之列33,而承天子之宠光,缀公卿之后34,日侍坐备顾问,四海亦谬称其氏名35,况才之过于余者乎?

今诸生学于太学36,县官日有廪稍之供37,父母岁有裘葛之遗38,无冻馁之患矣;坐大厦之下而诵《诗》《书》,无奔走之劳矣;有司业、博士为之师39,未有问而不告,求而不得者也;凡所宜有之书,皆集于此,不必若余之手录,假诸人而后见也。其业有不精,德有不成者,非天质之卑40,则心不若余之专耳,岂他人之过哉!

东阳马生君则,在太学已二年,流辈甚称其贤41。余朝京师42,生以乡人子谒余43,撰长书以为贽44,辞甚畅达,与之论辩45,言和而色夷46。自谓少时用心于学甚劳,是可谓善学者矣!其将归见其亲也47,余故道为学之难以告之。谓余勉乡人以学者,余之志也48;诋我夸际遇之盛而骄乡人者49,岂知余者哉!{[}2{]}


\section{1.2   词句注释}
\label{\detokenize{p01_u6563_u6587/_u5b8b_u6fc2-_u9001_u4e1c_u9633_u9a6c_u751f_u5e8f:id4}}
1.东阳:今浙江东阳市{[}3{]},当时与潜溪同属金华府。马生:姓马的太学生,即文中的马君则。序:文体名,有书序、赠序二种,本篇为赠序。

2.余:我。嗜(shì)学:爱好读书。

3.致:得到。

4.假借:借。

5.弗之怠:即“弗怠之”,不懈怠,不放松读书。弗,不。之,指代抄书。

6.走:跑,这里意为“赶快”。

7.逾约:超过约定的期限。

8.既:已经,到了。加冠:古代男子到二十岁时,举行加冠(束发戴帽)仪式,表示已成年。

9.圣贤之道:指孔孟儒家的道统。宋濂是一个主张仁义道德的理学家,所以十分推崇它。

10.硕(shuò)师:学问渊博的老师。游:交游。

11.尝:曾。趋:奔赴。

12.乡之先达:当地在道德学问上有名望的前辈。这里指浦江的柳贯、义乌的黄溍等古文家。执经叩问:携带经书去请教。

13.稍降辞色:把言辞放委婉些,把脸色放温和些。辞色,言辞和脸色。

14.援疑质理:提出疑难,询问道理。

15.叱(chì)(咄duō):训斥,呵责。

16.俟(sì):等待。忻(xīn):同“欣”。

17.卒:终于。

18.箧(qiè):箱子。曳屣(yèxǐ):拖着鞋子。

19.穷冬:隆冬。

20.皲(jūn)裂:皮肤因寒冷干燥而开裂。

21.僵劲:僵硬。

22.媵人:陪嫁的女子。这里指女仆。持汤沃灌:指拿热水喝或拿热水浸洗。汤:热水。沃灌:浇水洗。

23.衾(qīn):被子。

24.逆旅:旅店。

25.日再食:每日两餐。

26.被(pī)绮绣:穿着华丽的绸缎衣服。被,同“披”。绮,有花纹的丝织品。

27.朱缨宝饰:红穗子上穿有珠子等装饰品。

28.腰白玉之环:腰间悬着白玉圈。

29.容臭:香袋子。臭(xiù):气味,这里指香气。

30.烨(yè页)然:光采照人的样子。

31.缊(yùn)袍:粗麻絮制作的袍子。敝衣:破衣。

32.耄(mào)老:年老。八九十岁的人称耄。宋濂此时已六十九岁。

33.幸预:有幸参与。君子指有道德学问的读书人,另译指有官位的人{[}4{]}。

34.缀:这里意为“跟随”。

35.谬称:不恰当地赞许。这是作者的谦词。

36.诸生:指太学生。太学:明代中央政府设立的教育士人的学校,称作太学或国子监。

37.县官:这里指朝廷。廪(lǐn)稍:当时政府免费供给的俸粮称“廪”或“稍”。

38.裘(qiú):皮衣。葛:夏布衣服。遗(wèi):赠,这里指接济。

39.司业、博士:分别为太学的次长官和教授。

40.非天质之卑:如果不是由于天资太低下。

41.流辈:同辈。

42.朝:旧时臣下朝见君主。宋濂写此文时,正值他从家乡到京城应天(南京)见朱元璋。

43.以乡人子:以同乡之子的身份。谒(yè):拜见。

44.撰(zhuàn):同“撰”,写。长书:长信。贽(zhì):古时初次拜见时所赠的礼物。

45.辩:同辨。{[}5{]}

46.夷:平易。

47.归见:回家探望。

48.“谓余”二句:认为我是在勉励同乡人努力学习,这是说到了我的本意。

49.诋(dǐ):毁谤。际遇之盛:遭遇的得意,指得到皇帝的赏识重用。骄乡人:对同乡骄傲。


\section{1.3   白话译文}
\label{\detokenize{p01_u6563_u6587/_u5b8b_u6fc2-_u9001_u4e1c_u9633_u9a6c_u751f_u5e8f:id5}}
我年幼时就爱学习。因为家中贫穷,无法得到书来看,常向藏书的人家求借,亲手抄录,约定日期送还。天气酷寒时,砚池中的水冻成了坚冰,手指不能屈伸,我仍不放松抄书。抄写完后,赶快送还人家,不敢稍稍超过约定的期限。因此人们大多肯将书借给我,我因而能够看各种各样的书。已经成年之后,更加仰慕圣贤的学说,又苦于不能与学识渊博的老师和名人交往,曾快步走到百里之外,手拿着经书向同乡前辈求教。前辈德高望重,门人学生挤满了他的房间,他的言辞和态度从未稍有委婉。我站着陪侍在他左右,提出疑难,询问道理,低身侧耳向他请教;有时遭到他的训斥,表情更为恭敬,礼貌更为周到,不敢答复一句话;等到他高兴时,就又向他请教。所以我虽然愚钝,最终还是得到不少教益。

当我寻师时,背着书箱,把鞋后帮踩在脚后跟下,行走在深山大谷之中,严冬寒风凛冽,大雪深达几尺,脚和皮肤受冻裂开都不知道。到学舍后,四肢僵硬不能动弹,仆人给我灌下热水,用被子围盖身上,过了很久才暖和过来。住在旅馆,我每天吃两顿饭,没有新鲜肥嫩的美味享受。同学舍的求学者都穿着锦绣衣服,戴着有红色帽带、饰有珍宝的帽子,腰间挂着白玉环,左边佩戴着刀,右边备有香囊,光彩鲜明,如同神人;我却穿着旧棉袍、破衣服处于他们之间,毫无羡慕的意思。因为心中有足以使自己高兴的事,并不觉得吃穿的享受不如人家。我的勤劳和艰辛大概就是这样。

如今我虽已年老,没有什么成就,但所幸还得以置身于君子的行列中,承受着天子的恩宠荣耀,追随在公卿之后,每天陪侍着皇上,听候询问,天底下也不适当地称颂自己的姓名,更何况才能超过我的人呢?

如今的学生们在太学中学习,朝廷每天供给膳食,父母每年都赠给冬天的皮衣和夏天的葛衣,没有冻饿的忧虑了;坐在大厦之下诵读经书,没有奔走的劳苦了;有司业和博士当他们的老师,没有询问而不告诉,求教而无所收获的了;凡是所应该具备的书籍,都集中在这里,不必再像我这样用手抄录,从别人处借来然后才能看到了。他们中如果学业有所不精通,品德有所未养成的,如果不是天赋、资质低下,就是用心不如我这样专一,难道可以说是别人的过错吗!

东阳马生君则,在太学中已学习二年了,同辈人很称赞他的德行。我到京师朝见皇帝时,马生以同乡晚辈的身份拜见我,写了一封长信作为礼物,文辞很顺畅通达,同他论辩,言语温和而态度谦恭。他自己说少年时对于学习很用心、刻苦,这可以称作善于学习者吧!他将要回家拜见父母双亲,我特地将自己治学的艰难告诉他。如果说我勉励同乡努力学习,则是我的志意;如果诋毁我夸耀自己遭遇之好而在同乡前骄傲,难道是了解我吗?


\section{1.4   创作背景}
\label{\detokenize{p01_u6563_u6587/_u5b8b_u6fc2-_u9001_u4e1c_u9633_u9a6c_u751f_u5e8f:id6}}
明洪武十一年(1378),宋濂告老还乡的第二年,应诏从家乡浦江(浙江省浦江县)到应天(今江苏南京)去朝见,同乡晚辈马君则前来拜访,宋濂写下了此篇赠序,介绍自己的学习经历和学习态度,以勉励他人勤奋。


\section{1.5   作品鉴赏}
\label{\detokenize{p01_u6563_u6587/_u5b8b_u6fc2-_u9001_u4e1c_u9633_u9a6c_u751f_u5e8f:id7}}
此篇赠序是宋濂写给他的同乡晚生马君则的。作者赠他这篇文章,是以勉励他勤奋学习,但意思却不直接说出,而是从自己的亲身经历和体会中引申而出,婉转含蓄,平易亲切,字里行间充满了一个硕德长者对晚生后辈的殷切期望,读来令人感动。

全文分三大段。


\subsection{1.5.1   第一段}
\label{\detokenize{p01_u6563_u6587/_u5b8b_u6fc2-_u9001_u4e1c_u9633_u9a6c_u751f_u5e8f:id8}}
写自己青少年时代求学的情形,着意突出其“勤且艰”的好学精神。内中又分四个层次。第一层从借书之难写自己学习条件的艰苦。因家贫无书,只好借书、抄书,尽管天大寒,砚结冰,手指冻僵,也不敢稍有懈怠。第二层从求师之难,写虚心好学的必要。百里求师,恭谨小心。虽遇叱咄,终有所获。第三层从生活条件之难,写自己安于清贫,不慕富贵,因学有所得,故只觉其乐而不觉其苦,强调只要精神充实,生活条件的艰苦是微不足道的。第四层是这一段的总结。由于自己不怕各种艰难,勤苦学习,所以终于学有所成。虽然作者谦虚地说自己“未有所成”,但一代大儒的事实,是不待自言而人都明白的。最后“况才之过于余者乎”的反诘句承前启后,内容十分丰富。首先作者用反诘的语气强调了天分稍高的人若能像自己这样勤奋,必能取得越自己的卓绝成就。同时言外之意是说自己并不是天才,所以能取得现在的成绩,都是勤奋苦学的结果。推而言之,人若不是天资过分低下,学无所成,就只怪自己刻苦努力不够了。从下文知道,马生是一个勤奋好学的青年,他只要坚持下去,其前途也是不可限量的。所以这一句话虽寥寥数字,但含义深厚,作用大,既照应了上文,又关联了下文,扣紧了赠序的主题,把自己对马生的劝诫、勉励和期望,诚恳而又不失含蓄地从容道出,表现出“雍容浑穆”的大家风度。


\subsection{1.5.2   第二段}
\label{\detokenize{p01_u6563_u6587/_u5b8b_u6fc2-_u9001_u4e1c_u9633_u9a6c_u751f_u5e8f:id9}}
紧承第一段,写当代太学生学习条件的优越,与作者青年时代求学的艰难形成鲜明的对照,从反面强调了勤苦学习的必要性。“日有廪稍之供”云云是与上文生活条件之苦对比,“有司业、博士为之师”云云是与上文求师之难对比,“凡所宜有之书,皆集于此”云云,与上文借书之难对比。通过对比,人们很清楚地看出当今太学生在读书、求师、生活等几个方面,都比作者当年的求学条件优越得多,但却业有未精,德有未成。最后用一个选择句式又加一个反诘句式,强调指出:关键就在于这些太学生既不勤奋又不刻苦。这又是对上段第四层的照应。


\subsection{1.5.3   第三段}
\label{\detokenize{p01_u6563_u6587/_u5b8b_u6fc2-_u9001_u4e1c_u9633_u9a6c_u751f_u5e8f:id10}}
以上两段从正反两个方面强调了勤苦学习的重要性,虽未明言是对马生的劝励,而劝励之意自明。然而文章毕竟是为马生而作的,所以至第三段便明确地写到马生,点明写序的目的,这就是“道为学之难”,“勉乡人以学者”。因为劝励的内容在上两段中已经写足,所以这里便只讲些推奖褒美的话,但是殷切款诚之意,马生是不难心领神会的。


\subsection{1.5.4   总结}
\label{\detokenize{p01_u6563_u6587/_u5b8b_u6fc2-_u9001_u4e1c_u9633_u9a6c_u751f_u5e8f:id11}}
宋濂为人宽厚诚谨,谦恭下人。此文也是一如其人,写得情辞婉转,平易亲切。其实按他的声望、地位,他完全可以摆出长者的架子,正面说理大发议论,把这个青年教训一通的。然而他却不这样做。他绝口不说你们青年应当怎样怎样,而只是说“我”曾经怎样怎样,自己放在与对方平等的地位上,用自己亲身的经历和切身的体会去和人谈心。不仅从道理上,而且从形象上、情感上去启发影响读者,使人感到在文章深处有一种崇高的人格感召力量,在阅读过程中,读者会在不知不觉中缩短了与作者思想上的距离,赞同他的意见,并乐于照着他的意见去做。写文章要能达到这一步,决非只是一个文章技巧问题,这是需要有深厚的思想修养作基础的。

其次,作者在说理上,也不是凭空论道,而是善于让思想、道理从事实的叙述中自然地流露出来。而在事实的叙述中,又善于将概括的述说与典型的细节描绘有机地结合起来,这就使文章具体实在,仅在行文上简练生动,而且还具有很强的说服力和感染力。例如在说到读书之难时,作者在概括地叙述了自己因家贫无书,不得不借书、抄书,计日以还的情形后说:“天大寒,砚冰坚,手指不可屈伸,弗之怠。”通过这样一个典型的细节描写,就使人对作者当初读书的勤奋及学习条件的艰苦,有了一个生动形象的具体感受。理在事中,而事颇感人。这也是此文使人乐于赞同并接受作者意见的又一个内在的原因。

而且,文章浑然天成,内在结构却十分严密而紧凑。本来文章所赠送的对象是一篇之主体。然而文章却偏把主体抛在一边,先从自己谈起,从容道来,由己及人,至最后才谈及赠送的对象。看似漫不经心,实则匠心独运。在文章的深层结构中,主宾之间有一种紧密的内在联系,时时针对着主,处处照应到主,而却避免了一般赠序文章直露生硬的缺点,使文章委婉含蓄,意味深长。在写作中又成功地运用了对比映衬的手法,使左右有对比,前后有照应,文章于宽闲中显示严整,“鱼鱼雅雅,自中节度”。这一点给人的印象也是十分深刻的。


\section{1.6   名家点评}
\label{\detokenize{p01_u6563_u6587/_u5b8b_u6fc2-_u9001_u4e1c_u9633_u9a6c_u751f_u5e8f:id12}}
辽宁省作协主席、辽宁大学中文系教授王充闾《中国好文章·你不能错过的文言文》:“本文通过叙述自己年轻时求学的迫切、境遇的艰难,勉励马生等太学生刻苦向学,期于有成。感情真挚,语重心长,说理透彻,颇具感人力量。”


\section{1.7   作者简介}
\label{\detokenize{p01_u6563_u6587/_u5b8b_u6fc2-_u9001_u4e1c_u9633_u9a6c_u751f_u5e8f:id13}}
宋濂

宋濂(1310—1381),字景濂,号潜溪,别号玄真子、玄真道士、玄真遁叟。谥号文宪。浦江(今浙江浦江)人,汉族。明初文学家,曾被明太祖朱元璋誉为“开国文臣”。因其长孙宋慎牵连胡惟庸党案,全家流放茂州。其散文质朴简洁,或雍容典雅,各有特色。他推崇台阁文学,文风淳厚飘逸,为其后“台阁体”作家的文学创作提供范本。其作品大部分被合刻为《宋学士全集》七十五卷。


\chapter{1   岳飞-满江红·怒发冲冠}
\label{\detokenize{p01_u6563_u6587/_u5cb3_u98de-_u6ee1_u6c5f_u7ea2_xb7_u6012_u53d1_u51b2_u51a0:id1}}\label{\detokenize{p01_u6563_u6587/_u5cb3_u98de-_u6ee1_u6c5f_u7ea2_xb7_u6012_u53d1_u51b2_u51a0::doc}}
\begin{sphinxShadowBox}
\sphinxstyletopictitle{目录}
\begin{itemize}
\item {} 
\phantomsection\label{\detokenize{p01_u6563_u6587/_u5cb3_u98de-_u6ee1_u6c5f_u7ea2_xb7_u6012_u53d1_u51b2_u51a0:id10}}{\hyperref[\detokenize{p01_u6563_u6587/_u5cb3_u98de-_u6ee1_u6c5f_u7ea2_xb7_u6012_u53d1_u51b2_u51a0:id1}]{\sphinxcrossref{1   岳飞-满江红·怒发冲冠}}}
\begin{itemize}
\item {} 
\phantomsection\label{\detokenize{p01_u6563_u6587/_u5cb3_u98de-_u6ee1_u6c5f_u7ea2_xb7_u6012_u53d1_u51b2_u51a0:id11}}{\hyperref[\detokenize{p01_u6563_u6587/_u5cb3_u98de-_u6ee1_u6c5f_u7ea2_xb7_u6012_u53d1_u51b2_u51a0:id3}]{\sphinxcrossref{1.1   作品原文}}}

\item {} 
\phantomsection\label{\detokenize{p01_u6563_u6587/_u5cb3_u98de-_u6ee1_u6c5f_u7ea2_xb7_u6012_u53d1_u51b2_u51a0:id12}}{\hyperref[\detokenize{p01_u6563_u6587/_u5cb3_u98de-_u6ee1_u6c5f_u7ea2_xb7_u6012_u53d1_u51b2_u51a0:id4}]{\sphinxcrossref{1.2   词句注释}}}

\item {} 
\phantomsection\label{\detokenize{p01_u6563_u6587/_u5cb3_u98de-_u6ee1_u6c5f_u7ea2_xb7_u6012_u53d1_u51b2_u51a0:id13}}{\hyperref[\detokenize{p01_u6563_u6587/_u5cb3_u98de-_u6ee1_u6c5f_u7ea2_xb7_u6012_u53d1_u51b2_u51a0:id5}]{\sphinxcrossref{1.3   白话译文}}}

\item {} 
\phantomsection\label{\detokenize{p01_u6563_u6587/_u5cb3_u98de-_u6ee1_u6c5f_u7ea2_xb7_u6012_u53d1_u51b2_u51a0:id14}}{\hyperref[\detokenize{p01_u6563_u6587/_u5cb3_u98de-_u6ee1_u6c5f_u7ea2_xb7_u6012_u53d1_u51b2_u51a0:id6}]{\sphinxcrossref{1.4   创作背景}}}

\item {} 
\phantomsection\label{\detokenize{p01_u6563_u6587/_u5cb3_u98de-_u6ee1_u6c5f_u7ea2_xb7_u6012_u53d1_u51b2_u51a0:id15}}{\hyperref[\detokenize{p01_u6563_u6587/_u5cb3_u98de-_u6ee1_u6c5f_u7ea2_xb7_u6012_u53d1_u51b2_u51a0:id7}]{\sphinxcrossref{1.5   作品鉴赏}}}

\item {} 
\phantomsection\label{\detokenize{p01_u6563_u6587/_u5cb3_u98de-_u6ee1_u6c5f_u7ea2_xb7_u6012_u53d1_u51b2_u51a0:id16}}{\hyperref[\detokenize{p01_u6563_u6587/_u5cb3_u98de-_u6ee1_u6c5f_u7ea2_xb7_u6012_u53d1_u51b2_u51a0:id8}]{\sphinxcrossref{1.6   名家点评}}}

\item {} 
\phantomsection\label{\detokenize{p01_u6563_u6587/_u5cb3_u98de-_u6ee1_u6c5f_u7ea2_xb7_u6012_u53d1_u51b2_u51a0:id17}}{\hyperref[\detokenize{p01_u6563_u6587/_u5cb3_u98de-_u6ee1_u6c5f_u7ea2_xb7_u6012_u53d1_u51b2_u51a0:id9}]{\sphinxcrossref{1.7   作者简介}}}

\end{itemize}

\end{itemize}
\end{sphinxShadowBox}

《满江红·怒发冲冠》,一般认为是宋代抗金将领岳飞的词作。此词上片抒写作者对中原重陷敌手的悲愤,对局势前功尽弃的痛惜,表达了自己继续努力争取壮年立功的心愿;下片抒写作者对民族敌人的深仇大恨,对祖国统一的殷切愿望,对国家朝廷的赤胆忠诚。全词情调激昂,慷慨壮烈,显示出一种浩然正气和英雄气质,表现了作者报国立功的信心和乐观主义精神。


\section{1.1   作品原文}
\label{\detokenize{p01_u6563_u6587/_u5cb3_u98de-_u6ee1_u6c5f_u7ea2_xb7_u6012_u53d1_u51b2_u51a0:id3}}
满江红⑴

怒发冲冠⑵,凭阑处⑶、潇潇雨歇⑷。抬望眼,仰天长啸⑸,壮怀激烈⑹。三十功名尘与土⑺,八千里路云和月⑻。莫等闲⑼、白了少年头,空悲切⑽。

靖康耻⑾,犹未雪。臣子恨,何时灭。驾长车,踏破贺兰山缺⑿。壮志饥餐胡虏肉⒀,笑谈渴饮匈奴血⒁。待从头、收拾旧山河,朝天阙⒂。


\section{1.2   词句注释}
\label{\detokenize{p01_u6563_u6587/_u5cb3_u98de-_u6ee1_u6c5f_u7ea2_xb7_u6012_u53d1_u51b2_u51a0:id4}}
⑴满江红:词牌名,又名“上江虹”“念良游”“伤春曲”等。双调九十三字。

⑵怒发(fà)冲冠:气得头发竖起,以至于将帽子顶起。形容愤怒至极。

⑶凭阑:身倚栏杆。阑,同“栏”。

⑷潇潇:形容雨势急骤。

⑸长啸:大声呼叫。啸,蹙口发出的叫声。

⑹壮怀:奋发图强的志向。

⑺“三十”句:谓自己已经三十岁了,得到的功名,如同尘土一样微不足道。三十,是约数。功名,或指岳飞攻克襄阳六郡以后建节晋升之事。

⑻“八千”句:形容南征北战、路途遥远、披星戴月。八千,是约数,极言沙场征战行程之远。

⑼等闲:轻易,随便。

⑽空悲切:即白白的痛苦。

⑾靖康耻:宋钦宗靖康二年(1127),金兵攻陷汴京,虏走徽、钦二帝。靖康,宋钦宗赵桓的年号。

⑿贺兰山:贺兰山脉,位于宁夏回族自治区与内蒙古自治区交界处,当时被金兵占领。一说是位于邯郸市磁县境内的贺兰山。

⒀胡虏:对女真贵族入侵者的蔑称。

⒁匈奴:古代北方民族之一,这里指金入侵者。

⒂朝天阙:朝见皇帝。天阙,本指宫殿前的楼观,此指皇帝居住的地方。明代王熙书《满江红》词碑作“朝金阙”。


\section{1.3   白话译文}
\label{\detokenize{p01_u6563_u6587/_u5cb3_u98de-_u6ee1_u6c5f_u7ea2_xb7_u6012_u53d1_u51b2_u51a0:id5}}
我怒发冲冠登高倚栏杆,一场潇潇细雨刚刚停歇。抬头放眼四望辽阔一片,仰天长声啸叹。壮怀激烈,三十年勋业如今成尘土,征战千里只有浮云明月。莫虚度年华白了少年头,只有独自悔恨悲悲切切。

靖康年的奇耻尚未洗雪,臣子愤恨何时才能泯灭。我只想驾御着一辆辆战车踏破贺兰山敌人营垒。壮志同仇饿吃敌军的肉,笑谈蔑敌渴饮敌军的血。我要从头彻底地收复旧日河山,再回京阙向皇帝报捷。{[}5{]}


\section{1.4   创作背景}
\label{\detokenize{p01_u6563_u6587/_u5cb3_u98de-_u6ee1_u6c5f_u7ea2_xb7_u6012_u53d1_u51b2_u51a0:id6}}
关于此词的创作背景,有多种说法。有学者认为此词约创作于宋高宗绍兴二年(1132)前后,也有人认为作于绍兴四年(1134)岳飞克复襄阳六郡晋升清远军节度使之后。


\section{1.5   作品鉴赏}
\label{\detokenize{p01_u6563_u6587/_u5cb3_u98de-_u6ee1_u6c5f_u7ea2_xb7_u6012_u53d1_u51b2_u51a0:id7}}
此词上片写作者悲愤中原重陷敌手,痛惜前功尽弃的局面,也表达自己继续努力,争取壮年立功的心愿。

开头五句,起势突兀,破空而来。胸中的怒火在熊熊燃烧,不可阻遏。这时,一阵急雨刚刚停止,作者站在楼台高处,正凭栏远望。他看到那已经收复却又失掉的国土,想到了重陷水火之中的百姓,不由得“怒发冲冠”,“仰天长啸”,“壮怀激烈”。“怒发冲冠”是艺术夸张,是说由于异常愤怒,以致头发竖起,把帽子也顶起来了。作者表现出如此强烈的愤怒的感情并不是偶然的,这是他的理想与现实发生尖锐激烈的矛盾的结果。他面对投降派的不抵抗政策,气愤填膺。岳飞之怒,是金兵侵扰中原,烧杀虏掠的罪行所激起的雷霆之怒;岳飞之啸,是无路请缨,报国无门的忠愤之啸;岳飞之怀,是杀敌为国的宏大理想和豪壮襟怀。这几句一气贯注,生动地描绘了一位忠臣义士和忧国忧民的英雄形象。

接着四句激励自己,不要轻易虚度这壮年光阴,争取早日完成抗金大业。“三十功名尘于土”,是对过去的反省,表现作者渴望建立功名、努力抗战的思想。三十岁左右正当壮年,古人认为这时应当有所作为,可是,岳飞悔恨自己功名还与尘土一样,没有什么成就。宋朝以“三十之节”为殊荣。然而,岳飞梦寐以求的并不是建节封侯,身受殊荣,而是渡过黄河,收复国土,完成抗金救国的神圣事业。正如他自己所说“誓将直节报君仇”,“不问登坛万户侯”,对功名感觉不过像尘土一样,微不足道。“八千里路云和月”,是说不分阴晴,转战南北,在为收复中原而战斗。这是对未来的瞻望。“云和月”是特意写出,说出师北伐是十分艰苦的,任重道远,尚须披星戴月,日夜兼程,才能“北逾沙漠,喋血虏廷”(《五岳祠盟记》),赢得最后抗金的胜利。上一句写视功名为尘土,下一句写杀敌任重道远,个人为轻,国家为重,生动地表现了作者强烈的爱国热忱。“莫等闲”二句与“少壮不努力,老大徒伤悲”的意思相同,反映了作者积极进取的精神。这对当时抗击金兵,收复中原的斗争,显然起到了鼓舞斗志的作用;与主张议和,偏安江南,苟延残喘的投降派,形成了鲜明的对照。这既是岳飞的自勉之辞,也是对抗金将士的鼓励和鞭策。

词的下片运转笔端,抒写词人对于民族敌人的深仇大恨,统一祖国的殷切愿望,忠于朝廷即忠于祖国的赤诚之心。

“靖康耻”四句突出全词中心,由于没有雪“靖康”之耻,岳飞发出了心中的恨何时才能消除的感慨。这也是他要“驾长车踏破贺兰山缺”的原因,又把“驾长车踏破贺兰山缺”具体化了。从“驾长车”到“笑谈渴饮匈奴血”都以夸张的手法表达了对凶残敌人的愤恨之情,同时表现了英勇的信心和大无畏的乐观精神。

“壮志”二句把收复山河的宏愿,把艰苦的征战,以一种乐观主义精神表现出来。“待从头”二句,既表达要胜利的信心,也说了对朝廷和皇帝的忠诚。岳飞在这里不直接说凯旋、胜利等,而用了“收拾旧山河”,显得有诗意又形象。一腔忠愤,碧血丹心,肺腑倾出,以此收拾全篇,神完气足,无复毫发遗憾。

这首词代表了岳飞“精忠报国”的英雄之志,词里句中无不透出雄壮之气,显示了作者忧国报国的壮志胸怀。它作为爱国将领的抒怀之作,情调激昂,慷慨壮烈,充分表现了中华民族不甘屈辱,奋发图强,雪耻若渴的神威,从而成为反侵略战争的名篇。


\section{1.6   名家点评}
\label{\detokenize{p01_u6563_u6587/_u5cb3_u98de-_u6ee1_u6c5f_u7ea2_xb7_u6012_u53d1_u51b2_u51a0:id8}}
明代沈际飞:“胆量、意见、文章悉无今古。”(《草堂诗余正集》)

明末清初潘游龙:“胆量意见,俱超今古。”(《古今诗余醉》)

明末清初刘体仁:“词有与古诗同义者,‘潇潇雨歇’,《易水》之歌也。”(《七颂堂词绎》)

清代沈雄:“《满江红》忠愤可见,其不欲等闲白了少年头,可以明其心事。”(《古今词话·词话》上卷)

清代丁绍仪:“至寓议论于协律宫,犹觉激昂慷慨,读之色舞。”(《听秋声馆词话》卷九)

清代陈廷焯:“何等气概!何等志向!千载下读之,凛凛有生气焉。‘莫等闲’二语,当为千古箴铭。”(《白雨斋词话》)


\section{1.7   作者简介}
\label{\detokenize{p01_u6563_u6587/_u5cb3_u98de-_u6ee1_u6c5f_u7ea2_xb7_u6012_u53d1_u51b2_u51a0:id9}}
岳飞(1103—1142),南宋抗金将领。字鹏举,相州汤阴(今属河南)人。官至枢密副使,封武昌郡开国公。以不附和议,被秦桧所陷,被害于大理寺狱。孝宗时追谥武穆,宁宗时追封鄂王,理宗时改谥忠武。《宋史》有传。《直斋书录解题》著录《岳武穆集》十卷,不传。明徐阶编《岳武穆遗文》一卷。《全宋词》录其词三首。


\chapter{1   峻青-海滨仲夏夜}
\label{\detokenize{p01_u6563_u6587/_u5cfb_u9752-_u6d77_u6ee8_u4ef2_u590f_u591c:id1}}\label{\detokenize{p01_u6563_u6587/_u5cfb_u9752-_u6d77_u6ee8_u4ef2_u590f_u591c::doc}}
\begin{sphinxShadowBox}
\sphinxstyletopictitle{目录}
\begin{itemize}
\item {} 
\phantomsection\label{\detokenize{p01_u6563_u6587/_u5cfb_u9752-_u6d77_u6ee8_u4ef2_u590f_u591c:id5}}{\hyperref[\detokenize{p01_u6563_u6587/_u5cfb_u9752-_u6d77_u6ee8_u4ef2_u590f_u591c:id1}]{\sphinxcrossref{1   峻青-海滨仲夏夜}}}
\begin{itemize}
\item {} 
\phantomsection\label{\detokenize{p01_u6563_u6587/_u5cfb_u9752-_u6d77_u6ee8_u4ef2_u590f_u591c:id6}}{\hyperref[\detokenize{p01_u6563_u6587/_u5cfb_u9752-_u6d77_u6ee8_u4ef2_u590f_u591c:id3}]{\sphinxcrossref{1.1   作品原文}}}

\item {} 
\phantomsection\label{\detokenize{p01_u6563_u6587/_u5cfb_u9752-_u6d77_u6ee8_u4ef2_u590f_u591c:id7}}{\hyperref[\detokenize{p01_u6563_u6587/_u5cfb_u9752-_u6d77_u6ee8_u4ef2_u590f_u591c:id4}]{\sphinxcrossref{1.2   注释}}}

\end{itemize}

\end{itemize}
\end{sphinxShadowBox}


\section{1.1   作品原文}
\label{\detokenize{p01_u6563_u6587/_u5cfb_u9752-_u6d77_u6ee8_u4ef2_u590f_u591c:id3}}
夕阳落山不久,西方的天空,还燃烧着一片橘红色的晚霞。大海,也被这霞光染成了红色,而且比天空的景色更要壮观。因为它是活动的,每当一排排波浪涌起的时候,那映照在浪峰上的霞光,又红又亮,就像一片片霍霍①燃烧的火焰,闪烁着,消失了。而后面的一排,又闪烁着,滚动着,涌了过来。

天空的霞光渐渐地淡下去了,深红的颜色变成了绯红②,绯红又变为浅红。最后,当这一切红光都消失了的时候,那突然显得高而远了的天空,呈现出一片肃穆,最早出现的启明星③,在这深蓝色的天幕上闪烁起来了。它是那么大,那么亮,整个广漠④的天幕上只有它在那里放射着令人注目的光辉,活像一盏悬挂在高空的明灯。

夜色加浓,苍空中的“明灯”越来越多了。而城市各处的真的灯火也次第⑤亮了起来,尤其是围绕在海港周围山坡上的那一片灯光,从半空倒映在乌蓝的海面上,随着波浪,晃动着,闪烁着,像一串流动着的珍珠,和那一片片密布在苍穹⑥里的星斗互相辉映,煞⑦是好看。

在这幽美的夜色中,我踏着软绵绵的沙滩,沿着海边,慢慢地向前走去。海水,轻轻地抚摸着细软的沙滩,发出温柔的刷刷声。晚来的海风,清新而又凉爽。我的心里,有着说不出的兴奋和愉快。

夜风轻飘飘地吹拂着,空气中飘荡着一种大海和田禾相混合的香味,柔软的沙滩上还残留着白天太阳炙晒的余温。那些在各个工作岗位上劳动了一天的人们,三三两两地来到了这软绵绵的沙滩上,他们浴着凉爽的海风,望着那缀满了星星的夜空,尽情地说笑,尽情地休憩。愉快的笑声,不时地从这儿那儿飞扬开来,像平静的海面上不断地从这儿那儿涌起的波浪。

我漫步沙滩,徘徊在我的乡亲朋友们中间。

我看到,在那边,在一只底儿朝上反扣在沙滩上的木船旁边,是一群刚从田里收割麦子归来的人们,他们在谈论着今年的收成。今春,雨水足,麦苗长得旺,收成比去年好。眼下,又下了一场透雨,秋后的丰收局面,也大体可以确定下来了。人们为这大好年景所鼓舞着,谈话中也充满了愉快欢乐的笑声。

月亮上来了。

是一轮灿烂的满月。它像一面光辉四射的银盘似的,从那平静的大海里涌了出来。大海里,闪烁着一片鱼鳞似的银波。沙滩上,也突然明亮了起来,一片片坐着、卧着、走着的人影,看得清清楚楚了。啊!海滩上,居然有这么多的人在乘凉。说话声、欢笑声、唱歌声、嬉闹声,响遍了整个的海滩。

月亮升得很高了。它是那么皎洁⑧,那么明亮。

夜已经深了。

沙滩上的人,有的躺在那软绵绵的沙滩上睡着了,有的还在谈笑。凉爽的风轻轻地吹拂着,皎洁的月光照耀着。让这些英雄的人们,在这自由的天幕下,干净的沙滩上,海阔天空地尽情谈笑吧,酣畅地休憩吧。{[}1{]}


\section{1.2   注释}
\label{\detokenize{p01_u6563_u6587/_u5cfb_u9752-_u6d77_u6ee8_u4ef2_u590f_u591c:id4}}
①霍霍:这里是闪动的样子。

②绯红:鲜红。绯,红色。

③启明星:早晨出现于天空东方的金星。

④广漠:广大空旷。

⑤次第:一个挨一个。

⑥苍穹:天空。

⑦煞(shà):这里是“很”的意思。

⑧皎洁:(月亮)明亮洁白。


\chapter{1   曹操-观沧海}
\label{\detokenize{p01_u6563_u6587/_u66f9_u64cd-_u89c2_u6ca7_u6d77:id1}}\label{\detokenize{p01_u6563_u6587/_u66f9_u64cd-_u89c2_u6ca7_u6d77::doc}}
\begin{sphinxShadowBox}
\sphinxstyletopictitle{目录}
\begin{itemize}
\item {} 
\phantomsection\label{\detokenize{p01_u6563_u6587/_u66f9_u64cd-_u89c2_u6ca7_u6d77:id10}}{\hyperref[\detokenize{p01_u6563_u6587/_u66f9_u64cd-_u89c2_u6ca7_u6d77:id1}]{\sphinxcrossref{1   曹操-观沧海}}}
\begin{itemize}
\item {} 
\phantomsection\label{\detokenize{p01_u6563_u6587/_u66f9_u64cd-_u89c2_u6ca7_u6d77:id11}}{\hyperref[\detokenize{p01_u6563_u6587/_u66f9_u64cd-_u89c2_u6ca7_u6d77:id3}]{\sphinxcrossref{1.1   作品原文}}}

\item {} 
\phantomsection\label{\detokenize{p01_u6563_u6587/_u66f9_u64cd-_u89c2_u6ca7_u6d77:id12}}{\hyperref[\detokenize{p01_u6563_u6587/_u66f9_u64cd-_u89c2_u6ca7_u6d77:id4}]{\sphinxcrossref{1.2   词语注释}}}

\item {} 
\phantomsection\label{\detokenize{p01_u6563_u6587/_u66f9_u64cd-_u89c2_u6ca7_u6d77:id13}}{\hyperref[\detokenize{p01_u6563_u6587/_u66f9_u64cd-_u89c2_u6ca7_u6d77:id5}]{\sphinxcrossref{1.3   白话译文}}}

\item {} 
\phantomsection\label{\detokenize{p01_u6563_u6587/_u66f9_u64cd-_u89c2_u6ca7_u6d77:id14}}{\hyperref[\detokenize{p01_u6563_u6587/_u66f9_u64cd-_u89c2_u6ca7_u6d77:id6}]{\sphinxcrossref{1.4   创作背景}}}

\item {} 
\phantomsection\label{\detokenize{p01_u6563_u6587/_u66f9_u64cd-_u89c2_u6ca7_u6d77:id15}}{\hyperref[\detokenize{p01_u6563_u6587/_u66f9_u64cd-_u89c2_u6ca7_u6d77:id7}]{\sphinxcrossref{1.5   作品鉴赏}}}

\item {} 
\phantomsection\label{\detokenize{p01_u6563_u6587/_u66f9_u64cd-_u89c2_u6ca7_u6d77:id16}}{\hyperref[\detokenize{p01_u6563_u6587/_u66f9_u64cd-_u89c2_u6ca7_u6d77:id8}]{\sphinxcrossref{1.6   名家点评}}}

\item {} 
\phantomsection\label{\detokenize{p01_u6563_u6587/_u66f9_u64cd-_u89c2_u6ca7_u6d77:id17}}{\hyperref[\detokenize{p01_u6563_u6587/_u66f9_u64cd-_u89c2_u6ca7_u6d77:id9}]{\sphinxcrossref{1.7   作者简介}}}

\end{itemize}

\end{itemize}
\end{sphinxShadowBox}

《观沧海》是东汉末年诗人曹操创作的一首四言诗。这首诗是曹操在碣石山登山望海时,用饱蘸浪漫主义激情的大笔,所勾勒出的大海吞吐日月、包蕴万千的壮丽景象;描绘了祖国河山的雄伟壮丽,既刻画了高山大海的壮阔,更表达了诗人以景托志,胸怀天下的进取精神。全诗语言质朴,想象丰富,气势磅礴,苍凉悲壮。


\section{1.1   作品原文}
\label{\detokenize{p01_u6563_u6587/_u66f9_u64cd-_u89c2_u6ca7_u6d77:id3}}
观沧海

东临⑴碣⑵石,以观沧⑶海⑷。

水何⑸澹澹⑹,山岛竦峙⑺。

树木丛生,百草丰茂。

秋风萧瑟⑻,洪波⑼涌起。

日月⑽之行,若⑾出其中。

星汉⑿灿烂,若出其里。

幸⒀甚⒁至⒂哉,歌以咏志⒃。


\section{1.2   词语注释}
\label{\detokenize{p01_u6563_u6587/_u66f9_u64cd-_u89c2_u6ca7_u6d77:id4}}
⑴临:登上,有游览的意思。

⑵碣(jié)石:山名。碣石山,河北昌黎碣石山。公元207年秋天,曹操征乌桓得胜回师时经过此地。

⑶沧:通“苍”,青绿色。

⑷海:渤海。

⑸何:多么。

⑹澹澹(dàn):水波摇动的样子。

⑺竦峙(sǒngzhì):高高地挺立。竦,高起。峙,挺立。

⑻萧瑟:树木被秋风吹的声音。

⑼洪波:汹涌澎湃的波浪。

⑽日月:太阳和月亮。

⑾若:如同,好像是。

⑿星汉:银河,天河。

⒀幸:庆幸。

⒁甚:非常。

⒂至:极点。

⒃幸甚至哉,歌以咏志:乐府歌结束用语,不影响全诗内容与感情。意为太值得庆幸了!就用诗歌来表达心志吧。


\section{1.3   白话译文}
\label{\detokenize{p01_u6563_u6587/_u66f9_u64cd-_u89c2_u6ca7_u6d77:id5}}
东行登上碣石山,来观赏那苍茫的海。

海水多么宽阔浩荡,山岛高高地挺立在海边。

树木和百草丛生,十分繁茂。

秋风吹动树木发出悲凉的声音,海中涌着巨大的海浪。

太阳和月亮的运行,好像是从这浩瀚的海洋中发出的。

银河星光灿烂,好像是从这浩瀚的海洋中产生出来的。

我很高兴,就用这首诗歌来表达自己内心的志向。


\section{1.4   创作背景}
\label{\detokenize{p01_u6563_u6587/_u66f9_u64cd-_u89c2_u6ca7_u6d77:id6}}
乌桓是当时东北方的大患,建安十一年(206年),乌桓攻破幽州,俘虏了汉民十余万户。同年,袁绍的儿子袁尚和袁熙又勾结辽西乌桓首领蹋顿,屡次骚扰边境,以致曹操不得不在建安十二年(207年)毅然决定北上征伐乌桓。后来在田畴的指引下,小用计策。大约在这年八月的一次大战中,曹操终于取得了决定性的胜利。这次胜利巩固了曹操的后方,奠定了次年挥戈南下,以期实现统一中国的宏愿。而《观沧海》正是北征乌桓得胜回师经过碣石山时写的。{[}3{]}


\section{1.5   作品鉴赏}
\label{\detokenize{p01_u6563_u6587/_u66f9_u64cd-_u89c2_u6ca7_u6d77:id7}}
这首诗是曹操北征乌桓胜利班师,途中登临碣石山时所作,诗人借大海的雄伟壮丽景象,表达了自己渴望建功立业,统一中原的雄心伟志和宽广的胸襟。从诗的体裁看,这是一首古体诗;从表达方式看,这是一首四言写景诗。

“东临碣石,以观沧海”这两句话点明“观沧海”的位置:诗人登上碣石山顶,居高临海,视野寥廓,大海的壮阔景象尽收眼底。以下十句描写,概由此拓展而来。“观”字起到统领全篇的作用,体现了这首诗意境开阔,气势雄浑的特点。

“水何澹澹,山岛竦峙。树木丛生,百草丰茂。秋风萧瑟,洪波涌起”是实写眼前的景观,神奇而又壮观。“水何澹澹,山岛竦峙”是望海初得的大致印象,有点像绘画的轮廓。在这水波“澹澹”的海上,最先映入眼帘的是那突兀耸立的山岛,它们点缀在平阔的海面上,使大海显得神奇壮观。这两句写出了大海远景的一般轮廓,下面再层层深入描写。“树木丛生,百草丰茂。秋风萧瑟,洪波涌起。”前二句具体写竦峙的山岛:虽然已到秋风萧瑟,草木摇落的季节,但岛上树木繁茂,百草丰美,给人诗意盎然之感。后二句则是对“水何澹澹”一句的进一层描写:定神细看,在秋风萧瑟中的海面竟是洪波巨澜,汹涌起伏。作者面对萧瑟秋风,老骥伏枥,志在千里”的“壮志”胸怀。虽是秋天的典型环境,却无半点萧瑟凄凉的悲秋意绪。作者面对萧瑟秋风,极写大海的辽阔壮美:在秋风萧瑟中,大海汹涌澎湃,浩淼接天;山岛高耸挺拔,草木繁茂,没有丝毫凋衰感伤的情调。这种新的境界,新的格调,正反映了他“老骥伏枥,志在千里”的“烈士”胸襟。

“日月之行,若出其中;星汉灿烂,若出其里”则是虚写,作者运用想象,写出了自己的壮志情怀。前面的描写,将大海的气势和威力凸显在读者面前;在丰富的联想中表现出作者博大的胸怀、开阔的胸襟、宏大的抱负,暗含一种要像大海容纳万物一样把天下纳入自己掌中的胸襟。“幸甚至哉,歌以咏志。”这是合乐时的套语,与诗的内容无关,也说明这是乐府唱过的。

这首诗全篇写景,其中并无直抒胸臆的感慨之词,但是诵读全诗,仍能令人感到它所深深寄托的诗人的情怀。通过诗人对波涛汹涌、吞吐日月的大海的生动描绘,读者仿佛看到了曹操奋发进取,立志统一国家的伟大抱负和壮阔胸襟,触摸到了作为一个诗人、政治家、军事家的曹操,在一种典型环境中思想感情的流动。写景部分准确生动地描绘出海洋的形象,单纯而又饱满,丰富而不琐细,好像一幅粗线条的炭笔画一样。尤其可贵的是,这首诗不仅仅反映了海洋的形象,同时也赋予它以性格。句句写景,又是句句抒情。既表现了大海,也表现了诗人自己。诗人不满足于对海洋做形似的摹拟,而是通过形象,力求表现海洋那种孕大含深、动荡不安的性格。海,本来是没有生命的,然而在诗人笔下却具有了性格。这样才更真实、更深刻地反映了大海的面貌。

这首诗不但写景,而且借景抒情,把眼前的海上景色和自己的雄心壮志很巧妙地融合在一起。这首诗的高潮放在诗的末尾,它的感情非常奔放,思想却很含蓄。不但做到了情景交融,而且做到了情理结合、寓情于景。因为它含蓄,所以更有启发性,更能激发我们的想像,更耐人寻味。过去人们称赞曹操的诗深沉饱满、雄健有力,“如幽燕老将,气韵沉雄”,从这里可以得到印证。全诗的基调为苍凉慷慨的,这也是建安风骨的代表作。全诗语言质朴,想象丰富,气势磅礴,苍凉悲壮。


\section{1.6   名家点评}
\label{\detokenize{p01_u6563_u6587/_u66f9_u64cd-_u89c2_u6ca7_u6d77:id8}}
唐·吴兢《乐府古题要解》:“东临碣石,见沧海之广,日月出入其中。”{[}6{]}

清·张玉榖《古诗赏析》:“此志在容纳,而以海自比也;‘日月’四句,转就日月星汉,凭空想象其包含度量,写沧海,正自写也。”{[}1{]}

清·沈德潜《古诗源》:“有吞吐宇宙气象。”{[}1{]}


\section{1.7   作者简介}
\label{\detokenize{p01_u6563_u6587/_u66f9_u64cd-_u89c2_u6ca7_u6d77:id9}}
曹操(155~220年),字孟德,谯(今安徽亳县)县人,建安时代杰出的政治家、军事家和文学家。建安元年(196年)迎献帝都许(今河南许昌东),挟天子以令诸侯,先后削平吕布等割据势力。官渡之战大破军阀袁绍后,逐渐统一了中国北部。建安十三年(208年),进位为丞相,率军南下,被孙权和刘备的联军击败于赤壁。后封魏王。子曹丕称帝,追尊为武帝。事迹见《三国志》卷一本纪。有集三十卷,已散佚。明人辑有《魏武帝集》,今又有《曹操集》。


\chapter{1   朱自清-春}
\label{\detokenize{p01_u6563_u6587/_u6731_u81ea_u6e05-_u6625:id1}}\label{\detokenize{p01_u6563_u6587/_u6731_u81ea_u6e05-_u6625::doc}}
\begin{sphinxShadowBox}
\sphinxstyletopictitle{目录}
\begin{itemize}
\item {} 
\phantomsection\label{\detokenize{p01_u6563_u6587/_u6731_u81ea_u6e05-_u6625:id4}}{\hyperref[\detokenize{p01_u6563_u6587/_u6731_u81ea_u6e05-_u6625:id1}]{\sphinxcrossref{1   朱自清-春}}}
\begin{itemize}
\item {} 
\phantomsection\label{\detokenize{p01_u6563_u6587/_u6731_u81ea_u6e05-_u6625:id5}}{\hyperref[\detokenize{p01_u6563_u6587/_u6731_u81ea_u6e05-_u6625:id3}]{\sphinxcrossref{1.1   作品原文}}}

\end{itemize}

\end{itemize}
\end{sphinxShadowBox}


\section{1.1   作品原文}
\label{\detokenize{p01_u6563_u6587/_u6731_u81ea_u6e05-_u6625:id3}}
盼望着,盼望着,东风来了,春天的脚步近了。

一切都像刚睡醒的样子,欣欣然张开了眼。山朗润起来了,水涨起来了,太阳的脸红起来了。

小草偷偷地从土里钻出来,嫩嫩的,绿绿的。园子里,田野里,瞧去,一大片一大片满是的。坐着,躺着,打两个滚,踢几脚球,赛几趟跑,捉几回迷藏。风轻悄悄的,草软绵绵的。

桃树、杏树、梨树,你不让我,我不让你,都开满了花赶趟儿。红的像火,粉的像霞,白的像雪。花里带着甜味儿;闭了眼,树上仿佛已经满是桃儿、杏儿、梨儿。花下成千成百的蜜蜂嗡嗡地闹着,大小的蝴蝶飞来飞去。野花遍地是:杂样儿,有名字的,没名字的,散在草丛里,像眼睛,像星星,还眨呀眨的。

“吹面不寒杨柳风”,不错的,像母亲的手抚摸着你。风里带来些新翻的泥土的气息,混着青草味儿,还有各种花的香,都在微微润湿的空气里酝酿。鸟儿将窠巢安在繁花嫩叶当中,高兴起来了,呼朋引伴地卖弄清脆的喉咙,唱出宛转的曲子,与轻风流水应和着。牛背上牧童的短笛,这时候也成天嘹亮地响着。

雨是最寻常的,一下就是三两天。可别恼。看,像牛毛,像花针,像细丝,密密地斜织着,人家屋顶上全笼着一层薄烟。树叶儿却绿得发亮,小草儿也青得逼你的眼。傍晚时候,上灯了,一点点黄晕的光,烘托出一片安静而和平的夜。在乡下,小路上,石桥边,有撑起伞慢慢走着的人,地里还有工作的农民,披着蓑戴着笠。他们的房屋,稀稀疏疏的在雨里静默着。

天上风筝渐渐多了,地上孩子也多了。城里乡下,家家户户,老老小小,也赶趟儿似的,一个个都出来了。舒活舒活筋骨,抖擞抖擞精神,各做各的一份事去。“一年之计在于春”,刚起头儿,有的是工夫,有的是希望。

春天像刚落地的娃娃,从头到脚都是新的,它生长着。

春天像小姑娘,花枝招展的,笑着,走着。

春天像健壮的青年,有铁一般的胳膊和腰脚,领着我们上前去。


\chapter{1   朱自清-梅雨潭的绿}
\label{\detokenize{p01_u6563_u6587/_u6731_u81ea_u6e05-_u6885_u96e8_u6f6d_u7684_u7eff:id1}}\label{\detokenize{p01_u6563_u6587/_u6731_u81ea_u6e05-_u6885_u96e8_u6f6d_u7684_u7eff::doc}}
\begin{sphinxShadowBox}
\sphinxstyletopictitle{目录}
\begin{itemize}
\item {} 
\phantomsection\label{\detokenize{p01_u6563_u6587/_u6731_u81ea_u6e05-_u6885_u96e8_u6f6d_u7684_u7eff:id9}}{\hyperref[\detokenize{p01_u6563_u6587/_u6731_u81ea_u6e05-_u6885_u96e8_u6f6d_u7684_u7eff:id1}]{\sphinxcrossref{1   朱自清-梅雨潭的绿}}}
\begin{itemize}
\item {} 
\phantomsection\label{\detokenize{p01_u6563_u6587/_u6731_u81ea_u6e05-_u6885_u96e8_u6f6d_u7684_u7eff:id10}}{\hyperref[\detokenize{p01_u6563_u6587/_u6731_u81ea_u6e05-_u6885_u96e8_u6f6d_u7684_u7eff:id3}]{\sphinxcrossref{1.1   作品原文}}}

\item {} 
\phantomsection\label{\detokenize{p01_u6563_u6587/_u6731_u81ea_u6e05-_u6885_u96e8_u6f6d_u7684_u7eff:id11}}{\hyperref[\detokenize{p01_u6563_u6587/_u6731_u81ea_u6e05-_u6885_u96e8_u6f6d_u7684_u7eff:id4}]{\sphinxcrossref{1.2   梅雨潭位置}}}

\item {} 
\phantomsection\label{\detokenize{p01_u6563_u6587/_u6731_u81ea_u6e05-_u6885_u96e8_u6f6d_u7684_u7eff:id12}}{\hyperref[\detokenize{p01_u6563_u6587/_u6731_u81ea_u6e05-_u6885_u96e8_u6f6d_u7684_u7eff:id5}]{\sphinxcrossref{1.3   梅雨潭主要景点}}}
\begin{itemize}
\item {} 
\phantomsection\label{\detokenize{p01_u6563_u6587/_u6731_u81ea_u6e05-_u6885_u96e8_u6f6d_u7684_u7eff:id13}}{\hyperref[\detokenize{p01_u6563_u6587/_u6731_u81ea_u6e05-_u6885_u96e8_u6f6d_u7684_u7eff:id6}]{\sphinxcrossref{1.3.1   通元洞}}}

\item {} 
\phantomsection\label{\detokenize{p01_u6563_u6587/_u6731_u81ea_u6e05-_u6885_u96e8_u6f6d_u7684_u7eff:id14}}{\hyperref[\detokenize{p01_u6563_u6587/_u6731_u81ea_u6e05-_u6885_u96e8_u6f6d_u7684_u7eff:id7}]{\sphinxcrossref{1.3.2   飞瀑}}}

\item {} 
\phantomsection\label{\detokenize{p01_u6563_u6587/_u6731_u81ea_u6e05-_u6885_u96e8_u6f6d_u7684_u7eff:id15}}{\hyperref[\detokenize{p01_u6563_u6587/_u6731_u81ea_u6e05-_u6885_u96e8_u6f6d_u7684_u7eff:id8}]{\sphinxcrossref{1.3.3   潭水}}}

\end{itemize}

\end{itemize}

\end{itemize}
\end{sphinxShadowBox}


\section{1.1   作品原文}
\label{\detokenize{p01_u6563_u6587/_u6731_u81ea_u6e05-_u6885_u96e8_u6f6d_u7684_u7eff:id3}}
我第二次到仙岩的时候,我惊诧于梅雨潭的绿了。

梅雨潭是一个瀑布潭。仙岩有三个瀑布,梅雨瀑最低。走到山边,便听见哗哗哗哗的声音;抬起头,镶在两条湿湿的黑边儿里的,一带白而发亮的水便呈现于眼前了。我们先到梅雨亭。梅雨亭正对着那条瀑布;坐在亭边,不必仰头,便可见它的全体了。亭下深深的便是梅雨潭。这个亭踞在突出的一角的岩石上,上下都空空儿的;仿佛一只苍鹰展着翼翅浮在天宇中一般。三面都是山,像半个环儿拥着;人如在井底了。这是一个秋季的薄阴的天气。微微的云在我们顶上流着;岩面与草丛都从润湿中透出几分油油的绿意。而瀑布也似乎分外的响了。那瀑布从上面冲下,仿佛已被扯成大小的几绺;不复是一幅整齐而平滑的布。岩上有许多棱角;瀑流经过时,作急剧的撞击,便飞花碎玉般乱溅着了。那溅着的水花,晶莹而多芒;远望去,像一朵朵小小的白梅,微雨似的纷纷落着。据说,这说是梅雨潭之所以得名了。但我觉得像杨花,格外确切些。轻风起来时,点点随风飘散,那更是杨花了。这时偶然有几点送入我们温暖的怀里,便倏的钻了进去,再也寻它不着。

梅雨潭闪闪的绿色招引着我们;我们开始追捉她那离合的神光了。揪着草,攀着乱石,小心探身下去,又鞠躬过了一个石穹门,便到了汪汪一碧的潭边了。瀑布在襟袖之间;但我的心中已没有瀑布了。我的心随潭水的绿而摇荡。那醉人的绿呀,仿佛一张极大极大的荷叶铺着,满是奇异的绿呀。我想张开两臂抱住她;但这是怎样一个妄想呀。—站在水边,望到那面,居然觉着有些远呢!这平铺着,厚积着的绿,着实可爱。她松松的皱缬着,像少妇拖着的裙幅;她轻轻的摆弄着,像跳动的初恋的处女的心;她滑滑的明亮着,像涂了“明油”一般,有鸡蛋清那样软,那样嫩,令人想着所曾触过的最嫩的皮肤;她又不杂些儿法滓,宛然一块温润的碧玉,只清清的一色—但你却看不透她!我曾见过北京什刹海指地的绿杨,脱不了鹅黄的底子,似乎太淡了。我又曾见过杭州虎跑寺旁高峻而深密的“绿壁”,重叠着无穷国的碧草与绿叶的,那又似乎太浓了。其余呢,西湖的波太明了,秦淮河的又太暗了。可爱的,我将什么来比拟你呢?我怎么比拟得出呢?大约潭是很深的、故能蕴蓄着这样奇异的绿;仿佛蔚蓝的天融了一块在里面似的,这才这般的鲜润呀。—那醉人的绿呀!我若能裁你以为带,我将赠给那轻盈的舞女;她必能临风飘举了。我若能挹你以为眼,我将赠给那善歌的盲妹;她必明眸善睐了。我舍不得你;我怎舍得你呢?我用手拍着你,抚摩着你,如同一个十二三岁的小姑娘。我又掬你入口,便是吻着她了。我送你一个名字,我从此叫你“女儿绿”,好么?

我第二次到仙岩的时候,我不禁惊诧于梅雨潭的绿了。


\section{1.2   梅雨潭位置}
\label{\detokenize{p01_u6563_u6587/_u6731_u81ea_u6e05-_u6885_u96e8_u6f6d_u7684_u7eff:id4}}
梅雨潭,是位于浙江温州市瓯海区仙岩街道的一处名胜,是国家AAA级景区。它东临东海,南北各距瑞安和温州三十多里。

温州一带的山,都属于连绵不断的雁荡山脉。然而仙岩所属的大罗山却远离群山,巍然坐落在温瑞平原上。其山平地拔起,峻崖陡壁,水源充沛,虽方圆不过数十里,却多瀑布潭,而尤集中在西麓瑞安境内的仙岩附近。瀑布潭比较著名的有三个:梅雨潭、雷响潭和龙须潭。其中以梅雨潭最有特色。


\section{1.3   梅雨潭主要景点}
\label{\detokenize{p01_u6563_u6587/_u6731_u81ea_u6e05-_u6885_u96e8_u6f6d_u7684_u7eff:id5}}
远远望去,梅雨潭的瀑布狂奔直下;梅雨亭坐落在瀑布前一块突出的
巨石之上,非常显眼,乍一看去,正如《绿》中写的,“仿佛一只苍鹰展着翼翅浮在天宇中一般”。此亭正对瀑布,原为明代少师张孚敬所建,初名泽润亭,因为安坐其中可观赏瀑布的全貌,作为建筑物又恰到好处地与梅雨潭的自然景色融为一体,故后人改称为“梅雨亭”。


\subsection{1.3.1   通元洞}
\label{\detokenize{p01_u6563_u6587/_u6731_u81ea_u6e05-_u6885_u96e8_u6f6d_u7684_u7eff:id6}}
亭下有洞通潭边,叫做“通元洞”,有个石穹门,旁边刻有“四时梅雨”四个丰满有力的大字。


\subsection{1.3.2   飞瀑}
\label{\detokenize{p01_u6563_u6587/_u6731_u81ea_u6e05-_u6885_u96e8_u6f6d_u7684_u7eff:id7}}
梅雨潭的两侧,双崖对耸,绝不可攀,崖壁上附满绿苔及草木,呈自然的暗绿色,飞瀑自崖合掌处喷吐而出,轰轰作响。

悬崖上岩石颇多棱角,瀑布跌撞而下,似散珠一般注入潭中,轻风吹来,水珠飘飘洒洒,犹如朵朵白梅。


\subsection{1.3.3   潭水}
\label{\detokenize{p01_u6563_u6587/_u6731_u81ea_u6e05-_u6885_u96e8_u6f6d_u7684_u7eff:id8}}
潭水很深,经石穹门下到潭边,水珠、雾气、绿水、悬崖,构成一幅奇妙壮观的图画。清代潘耒在《游仙岩记》中云:“常若梅天细雨,故名梅雨潭。”这个奇观使得在温州执教不到一年的朱自清,竟先后两次来此“追捉她那离合的神光”,与梅雨潭结下了不解之缘。

现在有人在梅雨潭的石穹门旁刻了一个斗大的“绿”字,以此纪念这位著名散文家朱自清的不朽名作《绿》。

那溅着的水花,晶莹而多芒,远望去,像一朵朵小小的白梅,微雨似的纷纷落着.这就是梅雨潭的由来

“踞”字表现出梅雨亭的雄伟 而“浮”字又突出了亭的轻盈
像这样用得生动传神的动词还有“镶” 。“镶”表现出了梅雨亭的—————优美


\chapter{1   朱自清-背影}
\label{\detokenize{p01_u6563_u6587/_u6731_u81ea_u6e05-_u80cc_u5f71:id1}}\label{\detokenize{p01_u6563_u6587/_u6731_u81ea_u6e05-_u80cc_u5f71::doc}}
\begin{sphinxShadowBox}
\sphinxstyletopictitle{目录}
\begin{itemize}
\item {} 
\phantomsection\label{\detokenize{p01_u6563_u6587/_u6731_u81ea_u6e05-_u80cc_u5f71:id14}}{\hyperref[\detokenize{p01_u6563_u6587/_u6731_u81ea_u6e05-_u80cc_u5f71:id1}]{\sphinxcrossref{1   朱自清-背影}}}
\begin{itemize}
\item {} 
\phantomsection\label{\detokenize{p01_u6563_u6587/_u6731_u81ea_u6e05-_u80cc_u5f71:id15}}{\hyperref[\detokenize{p01_u6563_u6587/_u6731_u81ea_u6e05-_u80cc_u5f71:id3}]{\sphinxcrossref{1.1   作品原文}}}

\item {} 
\phantomsection\label{\detokenize{p01_u6563_u6587/_u6731_u81ea_u6e05-_u80cc_u5f71:id16}}{\hyperref[\detokenize{p01_u6563_u6587/_u6731_u81ea_u6e05-_u80cc_u5f71:id4}]{\sphinxcrossref{1.2   词语注释编辑}}}

\item {} 
\phantomsection\label{\detokenize{p01_u6563_u6587/_u6731_u81ea_u6e05-_u80cc_u5f71:id17}}{\hyperref[\detokenize{p01_u6563_u6587/_u6731_u81ea_u6e05-_u80cc_u5f71:id5}]{\sphinxcrossref{1.3   创作背景}}}

\item {} 
\phantomsection\label{\detokenize{p01_u6563_u6587/_u6731_u81ea_u6e05-_u80cc_u5f71:id18}}{\hyperref[\detokenize{p01_u6563_u6587/_u6731_u81ea_u6e05-_u80cc_u5f71:id6}]{\sphinxcrossref{1.4   内容赏析}}}
\begin{itemize}
\item {} 
\phantomsection\label{\detokenize{p01_u6563_u6587/_u6731_u81ea_u6e05-_u80cc_u5f71:id19}}{\hyperref[\detokenize{p01_u6563_u6587/_u6731_u81ea_u6e05-_u80cc_u5f71:id7}]{\sphinxcrossref{1.4.1   第一部分(第一至第三段)}}}

\item {} 
\phantomsection\label{\detokenize{p01_u6563_u6587/_u6731_u81ea_u6e05-_u80cc_u5f71:id20}}{\hyperref[\detokenize{p01_u6563_u6587/_u6731_u81ea_u6e05-_u80cc_u5f71:id8}]{\sphinxcrossref{1.4.2   第二部分(第四至第六段)}}}

\item {} 
\phantomsection\label{\detokenize{p01_u6563_u6587/_u6731_u81ea_u6e05-_u80cc_u5f71:id21}}{\hyperref[\detokenize{p01_u6563_u6587/_u6731_u81ea_u6e05-_u80cc_u5f71:id9}]{\sphinxcrossref{1.4.3   第三部分(最后一段)}}}

\end{itemize}

\item {} 
\phantomsection\label{\detokenize{p01_u6563_u6587/_u6731_u81ea_u6e05-_u80cc_u5f71:id22}}{\hyperref[\detokenize{p01_u6563_u6587/_u6731_u81ea_u6e05-_u80cc_u5f71:id10}]{\sphinxcrossref{1.5   语言特色}}}

\item {} 
\phantomsection\label{\detokenize{p01_u6563_u6587/_u6731_u81ea_u6e05-_u80cc_u5f71:id23}}{\hyperref[\detokenize{p01_u6563_u6587/_u6731_u81ea_u6e05-_u80cc_u5f71:id11}]{\sphinxcrossref{1.6   写作特色}}}

\item {} 
\phantomsection\label{\detokenize{p01_u6563_u6587/_u6731_u81ea_u6e05-_u80cc_u5f71:id24}}{\hyperref[\detokenize{p01_u6563_u6587/_u6731_u81ea_u6e05-_u80cc_u5f71:id12}]{\sphinxcrossref{1.7   行文立意}}}

\item {} 
\phantomsection\label{\detokenize{p01_u6563_u6587/_u6731_u81ea_u6e05-_u80cc_u5f71:id25}}{\hyperref[\detokenize{p01_u6563_u6587/_u6731_u81ea_u6e05-_u80cc_u5f71:id13}]{\sphinxcrossref{1.8   名家点评}}}

\end{itemize}

\end{itemize}
\end{sphinxShadowBox}


\section{1.1   作品原文}
\label{\detokenize{p01_u6563_u6587/_u6731_u81ea_u6e05-_u80cc_u5f71:id3}}
我与父亲不相见已二年余了,我最不能忘记的是他的背影。

那年冬天,祖母死了,父亲的差使1也交卸了,正是祸不单行的日子。我从北京到徐州,打算跟着父亲奔丧2回家。到徐州见着父亲,看见满院狼藉3的东西,又想起祖母,不禁簌簌地流下眼泪。父亲说:“事已如此,不必难过,好在天无绝人之路!”

回家变卖典质4,父亲还了亏空;又借钱办了丧事。这些日子,家中光景5很是惨澹,一半为了丧事,一半为了父亲赋闲6。丧事完毕,父亲要到南京谋事,我也要回北京念书,我们便同行。

到南京时,有朋友约去游逛,勾留7了一日;第二日上午便须渡江到浦口,下午上车北去。父亲因为事忙,本已说定不送我,叫旅馆里一个熟识的茶房8陪我同去。他再三嘱咐茶房,甚是仔细。但他终于不放心,怕茶房不妥帖9;颇踌躇10了一会。其实我那年已二十岁,北京已来往过两三次,是没有什么要紧的了。他踌躇了一会,终于决定还是自己送我去。我再三劝他不必去;他只说:“不要紧,他们去不好!”

我们过了江,进了车站。我买票,他忙着照看行李。行李太多,得向脚夫11行些小费才可过去。他便又忙着和他们讲价钱。我那时真是聪明过分,总觉他说话不大漂亮,非自己插嘴不可,但他终于讲定了价钱;就送我上车。他给我拣定了靠车门的一张椅子;我将他给我做的紫毛大衣铺好座位。他嘱我路上小心,夜里要警醒些,不要受凉。又嘱托茶房好好照应我。我心里暗笑他的迂;他们只认得钱,托他们只是白托!而且我这样大年纪的人,难道还不能料理自己么?我现在想想,我那时真是太聪明了。

我说道:“爸爸,你走吧。”他望车外看了看,说:“我买几个橘子去。你就在此地,不要走动。”我看那边月台的栅栏外有几个卖东西的等着顾客。走到那边月台,须穿过铁道,须跳下去又爬上去。父亲是一个胖子,走过去自然要费事些。我本来要去的,他不肯,只好让他去。我看见他戴着黑布小帽,穿着黑布大马褂12,深青布棉袍,蹒跚13地走到铁道边,慢慢探身下去,尚不大难。可是他穿过铁道,要爬上那边月台,就不容易了。他用两手攀着上面,两脚再向上缩;他肥胖的身子向左微倾,显出努力的样子。这时我看见他的背影,我的泪很快地流下来了。我赶紧拭干了泪。怕他看见,也怕别人看见。我再向外看时,他已抱了朱红的橘子往回走了。过铁道时,他先将橘子散放在地上,自己慢慢爬下,再抱起橘子走。到这边时,我赶紧去搀他。他和我走到车上,将橘子一股脑儿放在我的皮大衣上。于是扑扑衣上的泥土,心里很轻松似的。过一会儿说:“我走了,到那边来信!”我望着他走出去。他走了几步,回过头看见我,说:“进去吧,里边没人。”等他的背影混入来来往往的人里,再找不着了,我便进来坐下,我的眼泪又来了。

近几年来,父亲和我都是东奔西走,家中光景是一日不如一日。他少年出外谋生,独力支持,做了许多大事。哪知老境却如此颓唐!他触目伤怀,自然情不能自已。情郁于中,自然要发之于外;家庭琐屑便往往触他之怒。他待我渐渐不同往日。但最近两年不见,他终于忘却我的不好,只是惦记着我,惦记着他的儿子。我北来后,他写了一信给我,信中说道:“我身体平安,惟膀子疼痛厉害,举箸14提笔,诸多不便,大约大去之期15不远矣。”我读到此处,在晶莹的泪光中,又看见那肥胖的、青布棉袍黑布马褂的背影。唉!我不知何时再能与他相见!{[}2{]}


\section{1.2   词语注释编辑}
\label{\detokenize{p01_u6563_u6587/_u6731_u81ea_u6e05-_u80cc_u5f71:id4}}
1.差(chāi)使:旧时官场中称临时委任的职务,后来泛指职务或官职。

2.奔丧:在外闻亲人去世而归。

3.狼藉(jí):散乱不整齐的样子。亦作“狼籍”。

4.典质:典当,抵押。

5.光景:境况。

6.赋闲:没有职业在家闲居。

7.勾留:逗留。

8.茶房:旧时称在旅馆、茶馆、轮船、火车、剧场等地方从事供应茶水等杂务工作的人。

9.妥帖:恰当,十分合适。

10.踌躇(chóuchú):犹豫。

11.脚夫:旧称搬运工人。

12.马褂:旧时男子穿在长袍外面的对襟短褂。

13.蹒跚(pánshān):走路缓慢、摇摆的样子。也作“盘跚”。

14.箸(zhù):筷子。

15.大去之期:辞世的日子。


\section{1.3   创作背景}
\label{\detokenize{p01_u6563_u6587/_u6731_u81ea_u6e05-_u80cc_u5f71:id5}}
1917年,作者的祖母去世,父亲任徐州烟酒公卖局局长的差事也交卸了。办完丧事,父子同到南京,父亲送作者上火车北去,那年作者20岁。在那特定的场合下,做为父亲对儿子的关怀、体贴、爱护,使儿子极为感动,这印象经久不忘,并且几年之后,想起那背影,父亲的影子出现在“晶莹的泪光中”,使人不能忘怀。1925年,作者有感于世事,便写了此文。{[}4{]}


\section{1.4   内容赏析}
\label{\detokenize{p01_u6563_u6587/_u6731_u81ea_u6e05-_u80cc_u5f71:id6}}
全文可分为三部分。


\subsection{1.4.1   第一部分(第一至第三段)}
\label{\detokenize{p01_u6563_u6587/_u6731_u81ea_u6e05-_u80cc_u5f71:id7}}
交代人物,叙述跟父亲奔丧回家的有关情节,为描写父亲的背影作好铺垫。文章开头一句,落笔点题。“二年余”表明“我”清楚地记得和父亲分离的日子。副词“已”体现出“二年余”在作者的心目中已相当漫长,想望之情,不言而喻。两年多的分离,“我”对父亲的思念是多方面的。其中“最不能忘记的是他的背影”,点出题目。接着,转入对“那年冬天”往事的追述。“祖母死了,父亲的差使也交卸了”,短短两句呈现出人事错迁、谋生艰难之感。“我”从北京到了父亲的住地以后,“看见满院狼藉的东西”,其潦倒之状,又使“我不禁簌簌地流下眼泪”。因为“祸不单行”,所以回家之后,靠“变卖典质”,才还了“亏空”,又“借钱办了丧事”。这里所用的“祸不单行”、“亏空”,“借钱”、“丧事”等词语,一方面是当时情况的真实写照,同时也使后面“家中光景很是惨澹”的形容更有着落。这些叙述和描写,生动地反映了当时世态的灰暗。毛泽东主席在《中国社会各阶级的分析》一文中,曾对当时小资产阶级左翼的情况做过分析,说:“这种人因为他们过去过着好日子,后来逐年下降,负债渐多,渐次过着凄凉的日子,瞻念前途,不寒而栗”。这篇散文所叙述的情节,所抒发的感情,具有一定的典型意义的,也是此文为之感动共鸣的重要原因。


\subsection{1.4.2   第二部分(第四至第六段)}
\label{\detokenize{p01_u6563_u6587/_u6731_u81ea_u6e05-_u80cc_u5f71:id8}}
写父亲为“我”送行的情景,重点描写父亲的背影,表现父子间的真挚感情。丧事完毕,因为父亲要到南京谋事,“我”也要回北京念书,所以父子便一路同行到了南京。到南京之后,因为父亲要谋事,须接交各种关系,忙是可以想见的。所以说定要一个熟识的茶房为“我”送行。“他再三嘱咐茶房,甚是仔细。”这既表现了父亲对“我”的关怀,同时也说明了他对茶房的不放心。父亲当时异地谋生,正须多方奔走,又难以抽身,因此,他“颇踌躇了一会”。“踌躇”,反映了在父亲心中谋事与送子的矛盾。而“终于决定还是自己送我去”,则又表现了父亲毅然将生计暂时搁置,执意为“我”送行的真切感情。“终予”二字,把父亲对“我”无限关切、过分忧虑的心理,表现得淋漓尽致。接下去写的便是车站送行的场面。进了车站以后,父亲“忙着照看行李”,“忙着向脚夫讲价钱”,“送我上车”,“给我拣定靠车门的一张椅子”,“嘱我路上小心”。父亲操劳忙碌的形象展现在面前。可“我”那时由于太年轻,对父亲尚不能完全理解,以至于还在“心里暗笑他的迂”。作者行文至此,一种近乎忏悔的感情不觉流注笔端——“唉,我现在想想,那时真是太聪明了!”自我责备之中,包含着深切的内疚与怀念。在车上坐定之后,父亲又要为“我”去买橘子。但买橘子,“须穿过铁道,须跳下去又爬上去”。父亲又胖,吃力之状可以想见。因此,父亲当时去买橘子的情景,给“我”留下了极为深刻的记忆。当父亲“蹒跚地走到铁道边”时,“我”心中的酸楚是自不待言的。“蹒跚”一词,说明父亲年事已高,步履不稳,过铁路需人扶持。而今,为了“我”却在铁道间蹒跚前往。因而当看见父亲“用两手攀着……努力的样子”的背影时,“我的眼泪”便“很快地流下来了”。这“背影”集中地体现了父亲待“我”的全部感情,这“背影”使“我”念之心酸,感愧交并!望着父亲那吃力的背影,“我”禁不住热泪涌流,但为了“怕他看见”,“我”又“赶紧拭干了泪”,互相体谅的父子真情,表现得维妙维肖。父亲终于买来了橘子。当他走到这边时,“我赶紧去搀他”。这赶紧去搀的动作,表现了“我”又疼,又愧,又欣然若释的复杂心理。疼的是父亲为“我”受累,愧的是父亲为“我”买橘,欣然若释的是父亲终于安全归来。父亲回来之后,“我”虽然没讲一句话,但一腔深情都流露在这“赶紧去搀扶”的动作之中。回到车上,父亲“将橘子一股脑儿放在我的皮大衣上”。“一股脑儿”一词,表现了父亲当时高兴的心情。但父亲高兴的仅仅是为“我”买到了橘子,他的心头是并不轻松的。他谋生无着,而“我”又即将离他远去,兴从何来,所以文章说“心里很轻松似的”,“似的”二字说明父亲并不真正轻松,之所以做出仿佛轻松的样子,是为了宽慰那正心中眷眷的儿子,橘子已经买来,行李也早就安放停当,嘱咐的话也已经说过,看来没什么事了。但父亲并没有马上离去,而是“过一会”才说出告别的话。这“一会”之间,有拳拳的依恋,有惜别的惆怅。父亲终于说,“我走了;到那边来信!”临别的嘱咐,又一次表现了父亲对“我”的牵挂与系念。一直到他走了几步之后,还回过头来说“进去吧,里边没人”,仍关心着“我”的安全。但“我”并没有马上进去,而是“等他的背影……我便进来坐下”。这里的“等”、“再’、“便”三个字,用得极有层次,它们真实地表现了“我”站在车门口,追寻注视着父亲的背影,直到再也看不见时,才进去坐下的那种怅然若失的心情。“我”坐下之后,也许又看到了刚才父亲买来的橘子,一股热辣辣的感情又从心底兜起,“我的眼泪又来了”。


\subsection{1.4.3   第三部分(最后一段)}
\label{\detokenize{p01_u6563_u6587/_u6731_u81ea_u6e05-_u80cc_u5f71:id9}}
写对父亲的想念。作者在描写了父亲的背影之后,予深沉的怀念之中,又想起了父亲的一生。“他少年出外谋生,独力支持,做了许多大事。”父亲是坚强而能干的。虽然如此,家庭生活仍然每况愈下,“光景是一日不如一日”。父亲“触目伤怀”,脾气也变得易于暴怒了。因而,“他待我渐渐不同往日”,但这并非父亲本来的感情,父亲仍旧是父亲。两年不见,又使他在“举箸提笔,诸多不便”的情况下,写了信来,仍旧“惦记着我,惦记着我的儿子”。并在信中写道,“大约大去之期不远矣”,哀矜之中流露出孤寂、颓唐的况昧。它使“我”震悚,使“我”苦痛,使“我”想起父亲待“我”的种种好处,使“我”透过晶莹的泪光,又看见了父亲那凄楚的背影。父亲现在究竟怎样了,“唉!我不知何时再能与他相见。”盼望之中蕴蓄着热切的思念。


\section{1.5   语言特色}
\label{\detokenize{p01_u6563_u6587/_u6731_u81ea_u6e05-_u80cc_u5f71:id10}}
这篇散文的语言非常忠实朴素,又非常典雅文质。这种高度民族化的语言,和文章所表现的民族的精神气质,和文章的完美结构,恰成和谐的统一。没有《背影》语言的简洁明丽、古朴质实,就没有《背影》的一切风采。《背影》的语言还有文白夹杂的特点。例如不说“失业”,而说“赋闲”,最后一节因父亲来信是文言,引用原句,更见真实,也表达了家庭、父亲的困境和苍凉的心情与复杂的感受,同时,文白夹杂的语句,也笼上了一层时代赋予小资产阶级知识分子的特殊语言色彩。


\section{1.6   写作特色}
\label{\detokenize{p01_u6563_u6587/_u6731_u81ea_u6e05-_u80cc_u5f71:id11}}
这篇散文写作上的主要特点是白描。全文集中描写的,是父亲在特定场合下使作者极为感动的那一个背影。作者写了当时父亲的体态、穿着打扮,更主要地写了买橘子时穿过铁路的情形。并不借助于什么修饰、陪衬之类,只把当时的情景再现于眼前。这种白描的文字,读起来清淡质朴,却情真昧浓,蕴藏着一段深情。所谓于平淡中见神奇。其次,作品还运用了侧面烘托的手法。如写儿子“看见他的背影”,“泪很快地流下来了”。又写父亲买桔子回来时,儿子“赶紧去搀他”。这些侧面烘托手法的运用,更加反衬出父亲爱子的动人力量。


\section{1.7   行文立意}
\label{\detokenize{p01_u6563_u6587/_u6731_u81ea_u6e05-_u80cc_u5f71:id12}}
这篇散文的特点是抓住人物形象的特征“背影”命题立意,在叙事中抒发父子深情。“背影”在文章中出现了四次,每次的情况有所不同,而思想感情却是一脉相承。第一次开篇点题“背影”,有一种浓厚的感情气氛笼罩全文。第二次车站送别,作者对父亲的“背影”做了具体的描绘。第三次是父亲和儿子告别后,儿子眼望着父亲的“背影”在人群中消逝,离情别绪,催人泪下。第四次在文章的结尾,儿子读着父亲的来信,在泪光中再次浮现了父亲的“背影”,思念之情不能自已,与文章开头呼应,把父子之间的真挚感情表现得淋漓尽致。


\section{1.8   名家点评}
\label{\detokenize{p01_u6563_u6587/_u6731_u81ea_u6e05-_u80cc_u5f71:id13}}
李广田《最完整的人格》:《背影》论行数不满五十行,论字数不过千五百言,它之所以能够历久传诵而有感人至深的力量者,当然并不是凭藉了甚么宏伟的结构和华瞻的文字,而只是凭了它的老实,凭了其中所表达的真情。这种表面上看起来简单朴素,而实际上却能发生极大的感动力的文章,最可以作为朱先生的代表作品,因为这样的作品,也正好代表了作者之为人。

叶圣陶《文章例话》:“这篇文章通体干净,没有多余的话,没有多余的字眼,即使一个“的”字,一个“了”字,也是必须用才用”。

吴晗《他们走到了它的反面——朱自清颂》:“《背影》虽然只有一千五百字,却历久传诵,有感人至深的力量,这篇短文被选为中学国文教材,在中学生心目中,‘朱自清’三个字已经和《背影》成为不可分割的一体了”。


\chapter{1   朱自清-荷塘月色}
\label{\detokenize{p01_u6563_u6587/_u6731_u81ea_u6e05-_u8377_u5858_u6708_u8272:id1}}\label{\detokenize{p01_u6563_u6587/_u6731_u81ea_u6e05-_u8377_u5858_u6708_u8272::doc}}
\begin{sphinxShadowBox}
\sphinxstyletopictitle{目录}
\begin{itemize}
\item {} 
\phantomsection\label{\detokenize{p01_u6563_u6587/_u6731_u81ea_u6e05-_u8377_u5858_u6708_u8272:id5}}{\hyperref[\detokenize{p01_u6563_u6587/_u6731_u81ea_u6e05-_u8377_u5858_u6708_u8272:id1}]{\sphinxcrossref{1   朱自清-荷塘月色}}}
\begin{itemize}
\item {} 
\phantomsection\label{\detokenize{p01_u6563_u6587/_u6731_u81ea_u6e05-_u8377_u5858_u6708_u8272:id6}}{\hyperref[\detokenize{p01_u6563_u6587/_u6731_u81ea_u6e05-_u8377_u5858_u6708_u8272:id3}]{\sphinxcrossref{1.1   作品原文}}}

\item {} 
\phantomsection\label{\detokenize{p01_u6563_u6587/_u6731_u81ea_u6e05-_u8377_u5858_u6708_u8272:id7}}{\hyperref[\detokenize{p01_u6563_u6587/_u6731_u81ea_u6e05-_u8377_u5858_u6708_u8272:id4}]{\sphinxcrossref{1.2   词语注释}}}

\end{itemize}

\end{itemize}
\end{sphinxShadowBox}


\section{1.1   作品原文}
\label{\detokenize{p01_u6563_u6587/_u6731_u81ea_u6e05-_u8377_u5858_u6708_u8272:id3}}
这几天心里颇不宁静。今晚在院子里坐着乘凉,忽然想起日日走过的荷塘,在这满月的光里,总该另有一番样子吧。月亮渐渐地升高了,墙外马路上孩子们的欢笑,已经听不见了;妻在屋里拍着闰儿⑴,迷迷糊糊地哼着眠歌。我悄悄地披了大衫,带上门出去。

沿着荷塘,是一条曲折的小煤屑路。这是一条幽僻的路;白天也少人走,夜晚更加寂寞。荷塘四面,长着许多树,蓊蓊郁郁⑵的。路的一旁,是些杨柳,和一些不知道名字的树。没有月光的晚上,这路上阴森森的,有些怕人。今晚却很好,虽然月光也还是淡淡的。

路上只我一个人,背着手踱⑶着。这一片天地好像是我的;我也像超出了平常的自己,到了另一个世界里。我爱热闹,也爱冷静;爱群居,也爱独处。像今晚上,一个人在这苍茫的月下,什么都可以想,什么都可以不想,便觉是个自由的人。白天里一定要做的事,一定要说的话,现在都可不理。这是独处的妙处,我且受用这无边的荷香月色好了。

曲曲折折的荷塘上面,弥望⑷的是田田⑸的叶子。叶子出水很高,像亭亭的舞女的裙。层层的叶子中间,零星地点缀着些白花,有袅娜⑹地开着的,有羞涩地打着朵儿的;正如一粒粒的明珠,又如碧天里的星星,又如刚出浴的美人。微风过处,送来缕缕清香,仿佛远处高楼上渺茫的歌声似的。这时候叶子与花也有一丝的颤动,像闪电般,霎时传过荷塘的那边去了。叶子本是肩并肩密密地挨着,这便宛然有了一道凝碧的波痕。叶子底下是脉脉⑺的流水,遮住了,不能见一些颜色;而叶子却更见风致⑻了。

月光如流水一般,静静地泻在这一片叶子和花上。薄薄的青雾浮起在荷塘里。叶子和花仿佛在牛乳中洗过一样;又像笼着轻纱的梦。虽然是满月,天上却有一层淡淡的云,所以不能朗照;但我以为这恰是到了好处——酣眠固不可少,小睡也别有风味的。月光是隔了树照过来的,高处丛生的灌木,落下参差的斑驳的黑影,峭楞楞如鬼一般;弯弯的杨柳的稀疏的倩影,却又像是画在荷叶上。塘中的月色并不均匀;但光与影有着和谐的旋律,如梵婀玲⑼上奏着的名曲。

荷塘的四面,远远近近,高高低低都是树,而杨柳最多。这些树将一片荷塘重重围住;只在小路一旁,漏着几段空隙,像是特为月光留下的。树色一例是阴阴的,乍看像一团烟雾;但杨柳的丰姿⑽,便在烟雾里也辨得出。树梢上隐隐约约的是一带远山,只有些大意罢了。树缝里也漏着一两点路灯光,没精打采的,是渴睡⑾人的眼。这时候最热闹的,要数树上的蝉声与水里的蛙声;但热闹是它们的,我什么也没有。

忽然想起采莲的事情来了。采莲是江南的旧俗,似乎很早就有,而六朝时为盛;从诗歌里可以约略知道。采莲的是少年的女子,她们是荡着小船,唱着艳歌去的。采莲人不用说很多,还有看采莲的人。那是一个热闹的季节,也是一个风流的季节。梁元帝《采莲赋》里说得好:

于是妖童媛女⑿,荡舟心许;鷁首⒀徐回,兼传羽杯⒁;棹⒂将移而藻挂,船欲动而萍开。尔其纤腰束素⒃,迁延顾步⒄;夏始春余,叶嫩花初,恐沾裳而浅笑,畏倾船而敛裾⒅。

可见当时嬉游的光景了。这真是有趣的事,可惜我们现在早已无福消受了。

于是又记起,《西洲曲》里的句子:

采莲南塘秋,莲花过人头;低头弄莲子,莲子清如水。

今晚若有采莲人,这儿的莲花也算得“过人头”了;只不见一些流水的影子,是不行的。这令我到底惦着江南了。——这样想着,猛一抬头,不觉已是自己的门前;轻轻地推门进去,什么声息也没有,妻已睡熟好久了。

一九二七年七月,北京清华园。


\section{1.2   词语注释}
\label{\detokenize{p01_u6563_u6587/_u6731_u81ea_u6e05-_u8377_u5858_u6708_u8272:id4}}
1、闰儿:指朱闰生,朱自清第二子。

2、蓊蓊(wěng)郁郁:树木茂盛的样子。

3、踱(duó):慢慢地走

4、弥望:满眼。弥,满。

5、田田:形容荷叶相连的样子。古乐府《江南曲》中有“莲叶何田田”之句。

6、袅娜(niǎonuó):柔美的样子。

7、脉脉(mò):这里形容水没有声音,好像饱含深情的样子。

8、风致:美的姿态。

9、梵婀玲:violin,小提琴的音译。

10、丰姿:风度,仪态,一般指美好的姿态。也写作“风姿”

11、渴睡:也写作“瞌睡”。

12、妖童媛女:俊俏的少年和美丽的少女。妖,艳丽。媛,女子。

13、鷁首(yìshǒu):船头。古代画鷁鸟于船头。

14、羽杯:古代饮酒用的耳杯。又称羽觞、耳杯。

15、棹(zhào):船桨。

16、纤腰束素:腰如束素,齿如含贝(宋玉《登徒子好色赋》),形容女子腰肢细柔

17、迁延顾步:形容走走退退不住回视自己动作的样子,有顾影自怜之意。

18、敛裾(jū):这里是提着衣襟的意思。裾,衣襟。


\chapter{1   李白-将进酒}
\label{\detokenize{p01_u6563_u6587/_u674e_u767d-_u5c06_u8fdb_u9152:id1}}\label{\detokenize{p01_u6563_u6587/_u674e_u767d-_u5c06_u8fdb_u9152::doc}}
\begin{sphinxShadowBox}
\sphinxstyletopictitle{目录}
\begin{itemize}
\item {} 
\phantomsection\label{\detokenize{p01_u6563_u6587/_u674e_u767d-_u5c06_u8fdb_u9152:id10}}{\hyperref[\detokenize{p01_u6563_u6587/_u674e_u767d-_u5c06_u8fdb_u9152:id1}]{\sphinxcrossref{1   李白-将进酒}}}
\begin{itemize}
\item {} 
\phantomsection\label{\detokenize{p01_u6563_u6587/_u674e_u767d-_u5c06_u8fdb_u9152:id11}}{\hyperref[\detokenize{p01_u6563_u6587/_u674e_u767d-_u5c06_u8fdb_u9152:id3}]{\sphinxcrossref{1.1   作品原文}}}

\item {} 
\phantomsection\label{\detokenize{p01_u6563_u6587/_u674e_u767d-_u5c06_u8fdb_u9152:id12}}{\hyperref[\detokenize{p01_u6563_u6587/_u674e_u767d-_u5c06_u8fdb_u9152:id4}]{\sphinxcrossref{1.2   词句注释}}}

\item {} 
\phantomsection\label{\detokenize{p01_u6563_u6587/_u674e_u767d-_u5c06_u8fdb_u9152:id13}}{\hyperref[\detokenize{p01_u6563_u6587/_u674e_u767d-_u5c06_u8fdb_u9152:id5}]{\sphinxcrossref{1.3   白话译文}}}

\item {} 
\phantomsection\label{\detokenize{p01_u6563_u6587/_u674e_u767d-_u5c06_u8fdb_u9152:id14}}{\hyperref[\detokenize{p01_u6563_u6587/_u674e_u767d-_u5c06_u8fdb_u9152:id6}]{\sphinxcrossref{1.4   创作背景}}}

\item {} 
\phantomsection\label{\detokenize{p01_u6563_u6587/_u674e_u767d-_u5c06_u8fdb_u9152:id15}}{\hyperref[\detokenize{p01_u6563_u6587/_u674e_u767d-_u5c06_u8fdb_u9152:id7}]{\sphinxcrossref{1.5   作品鉴赏}}}

\item {} 
\phantomsection\label{\detokenize{p01_u6563_u6587/_u674e_u767d-_u5c06_u8fdb_u9152:id16}}{\hyperref[\detokenize{p01_u6563_u6587/_u674e_u767d-_u5c06_u8fdb_u9152:id8}]{\sphinxcrossref{1.6   名家点评}}}

\item {} 
\phantomsection\label{\detokenize{p01_u6563_u6587/_u674e_u767d-_u5c06_u8fdb_u9152:id17}}{\hyperref[\detokenize{p01_u6563_u6587/_u674e_u767d-_u5c06_u8fdb_u9152:id9}]{\sphinxcrossref{1.7   作者简介}}}

\end{itemize}

\end{itemize}
\end{sphinxShadowBox}

《将进酒》是唐代大诗人李白沿用乐府古题创作的一首诗。此诗为李白长安放还以后所作,思想内容非常深沉,艺术表现非常成熟,在同题作品中影响最大。诗人豪饮高歌,借酒消愁,抒发了忧愤深广的人生感慨。诗中交织着失望与自信、悲愤与抗争的情怀,体现出强烈的豪纵狂放的个性。全诗情感饱满,无论喜怒哀乐,其奔涌迸发均如江河流泻,不可遏止,且起伏跌宕,变化剧烈;在手法上多用夸张,且往往以巨额数量词进行修饰,既表现出诗人豪迈洒脱的情怀,又使诗作本身显得笔墨酣畅,抒情有力;在结构上大开大阖,充分体现了李白七言歌行的特色。


\section{1.1   作品原文}
\label{\detokenize{p01_u6563_u6587/_u674e_u767d-_u5c06_u8fdb_u9152:id3}}
将进酒⑴

君不见,黄河之水天上来⑵,奔流到海不复回。

君不见,高堂明镜悲白发,朝如青丝暮成雪⑶。

人生得意须尽欢⑷,莫使金樽空对月。

天生我材必有用,千金散尽还复来。

烹羊宰牛且为乐,会须一饮三百杯⑸。

岑夫子,丹丘生⑹,将进酒,杯莫停⑺。

与君歌一曲⑻,请君为我倾耳听⑼。

钟鼓馔玉不足贵⑽,但愿长醉不复醒⑾。

古来圣贤皆寂寞,惟有饮者留其名。

陈王昔时宴平乐,斗酒十千恣欢谑⑿。

主人何为言少钱⒀,径须沽取对君酌⒁。

五花马⒂,千金裘,呼儿将出换美酒,与尔同销万古愁⒃。


\section{1.2   词句注释}
\label{\detokenize{p01_u6563_u6587/_u674e_u767d-_u5c06_u8fdb_u9152:id4}}
⑴将(qiāng)进酒:请饮酒。乐府古题,原是汉乐府短箫铙歌的曲调。《乐府诗集》卷十六引《古今乐录》曰:“汉鼓吹铙歌十八曲,九曰《将进酒》。”《敦煌诗集残卷》三个手抄本此诗均题作“惜罇空”。《文苑英华》卷三三六题作“惜空罇酒”。将,请。

⑵君不见:乐府诗常用作提醒人语。天上来:黄河发源于青海,因那里地势极高,故称。

⑶高堂:房屋的正室厅堂。一说指父母,不合诗意。一作“床头”。青丝:喻柔软的黑发。一作“青云”。成雪:一作“如雪”。

⑷得意:适意高兴的时候。

⑸会须:正应当。

⑹岑夫子:岑勋。丹丘生:元丹丘。二人均为李白的好友。

⑺杯莫停:一作“君莫停”。

⑻与君:给你们,为你们。君,指岑、元二人。

⑼倾耳听:一作“侧耳听”。

⑽钟鼓:富贵人家宴会中奏乐使用的乐器。馔(zhuàn)玉:形容食物如玉一样精美。

⑾不复醒:也有版本为“不用醒”或“不愿醒”。

⑿陈王:指陈思王曹植。平乐(lè):观名。在洛阳西门外,为汉代富豪显贵的娱乐场所。恣:纵情任意。谑(xuè):戏。

⒀言少钱:一作“言钱少”。

⒁径须:干脆,只管。沽:通“酤”,买。

⒂五花马:指名贵的马。一说毛色作五花纹,一说颈上长毛修剪成五瓣。

⒃尔:你。


\section{1.3   白话译文}
\label{\detokenize{p01_u6563_u6587/_u674e_u767d-_u5c06_u8fdb_u9152:id5}}
你可见黄河水从天上流下来,波涛滚滚直奔向东海不回还。

你可见高堂明镜中苍苍白发,早上满头青丝晚上就如白雪。

人生得意时要尽情享受欢乐,不要让金杯空对皎洁的明月。

天造就了我成材必定会有用,即使散尽黄金也还会再得到,

煮羊宰牛姑且尽情享受欢乐,一气喝他三百杯也不要嫌多。

岑夫子啊、丹丘生啊,快喝酒啊,不要停啊。

我为在坐各位朋友高歌一曲,请你们一定要侧耳细细倾听。

钟乐美食这样的富贵不稀罕,我愿永远沉醉酒中不愿清醒。

圣者仁人自古就寂然悄无声,只有那善饮的人才留下美名。

当年陈王曹植平乐观摆酒宴,一斗美酒值万钱他们开怀饮。

主人你为什么说钱已经不多,你尽管端酒来让我陪朋友喝。

管它名贵五花马还是狐皮裘,快叫侍儿拿去统统来换美酒,

与你同饮来消融这万古常愁。李白


\section{1.4   创作背景}
\label{\detokenize{p01_u6563_u6587/_u674e_u767d-_u5c06_u8fdb_u9152:id6}}
关于这首诗的写作时间,说法不一。郁贤皓《李白集》认为此诗约作于开元二十四年(736)前后。黄锡珪《李太白编年诗集目录》系于天宝十一载(752)。一般认为这是李白天宝年间离京后,漫游梁、宋,与友人岑勋、元丹丘相会时所作。

唐玄宗天宝初年,李白由道士吴筠推荐,由唐玄宗招进京,命李白为供奉翰林。不久,因权贵的谗毁,于天宝三载(744年),李白被排挤出京,唐玄宗赐金放还。此后,李白在江淮一带盘桓,思想极度烦闷,又重新踏上了云游祖国山河的漫漫旅途。李白作此诗时距李白被唐玄宗“赐金放还”已有八年之久。这一时期,李白多次与友人岑勋(岑夫子)应邀到嵩山另一好友元丹丘的颍阳山居为客,三人登高饮宴,借酒放歌。诗人在政治上被排挤,受打击,理想不能实现,常常借饮酒来发泄胸中的郁积。人生快事莫若置酒会友,作者又正值“抱用世之才而不遇合”之际,于是满腔不合时宜借酒兴诗情,以抒发满腔不平之气。


\section{1.5   作品鉴赏}
\label{\detokenize{p01_u6563_u6587/_u674e_u767d-_u5c06_u8fdb_u9152:id7}}
这首诗非常形象地表现了李白桀骜不驯的性格:一方面对自己充满自信,孤高自傲;一方面在政治前途出现波折后,又流露出纵情享乐之情。在这首诗里,李白演绎庄子的乐生哲学,表示对富贵、圣贤的藐视。而在豪饮行乐中,实则深含怀才不遇之情。诗人借题发挥,借酒浇愁,抒发自己的愤激情绪。全诗气势豪迈,感情奔放,语言流畅,具有很强的感染力。

时光流逝,如江河入海一去无回;人生苦短,看朝暮间青丝白雪;生命的渺小似乎是个无法挽救的悲剧,能够解忧的惟有金樽美酒。这便是李白式的悲哀:悲而能壮,哀而不伤,极愤慨而又极豪放。表是在感叹人生易老,里则在感叹怀才不遇。诗篇开头是两组排比长句,如挟天风海雨向读者迎面扑来,气势豪迈。“君不见黄河之水天上来,奔流到海不复回”,李白此时在颍阳山,距离黄河不远,登高纵目,所以借黄河来起兴。黄河源远流长,落差极大,如从天而降,一泻千里,东走大海。景象之壮阔,并不是肉眼可见,所以此情此景是李白幻想的,“自道所得”,言语中带有夸张。上句写大河之来,势不可挡;下句写大河之去,势不可回。一涨一消,形成舒卷往复的咏叹味,是短促的单句(如“黄河落天走东海”)所没有的。

紧接着,“君不见高堂明镜悲白发,朝如青丝暮成雪”,恰似一波未平、一波又起。前二句为空间范畴的夸张,这二句则是时间范畴的夸张。悲叹人生短促;而不直接说出自己感伤生命短暂而人一下就会变老,却说“高堂明镜悲白发”,显现出一种对镜自照手抚两鬓、却无可奈何的情态。将人生由青春至衰老的全过程说成“朝”“暮”之事,把本来短暂的说得更短暂,与前两句把本来壮浪的说得更壮浪,是“反向”的夸张。于是,开篇的这组排比长句既有比意——以河水一去不返喻人生易逝,又有反衬作用——以黄河的伟大永恒形出生命的渺小脆弱。这个开端可谓悲感已极,却不堕纤弱,可说是巨人式的感伤,具有惊心动魄的艺术力量,同时也是由长句排比开篇的气势感造成的。这种开篇的手法作者常用,他如“弃我去者,昨日之日不可留;乱我心者,今日之日多烦忧”(《宣城谢朓楼饯别校书叔云》),沈德潜说:“此种格调,太白从心化出”,可见其颇具创造性。此诗两作“君不见”的呼告(一般乐府诗只于篇首或篇末偶一用之),又使诗句感情色彩大大增强。诗有所谓大开大阖者,此可谓大开。

“夫天地者,万物之逆旅也;光阴者,百代之过客也”(《春夜宴从弟桃李园序》),悲感虽然不免,但悲观却非李白性分之所近。在他看来,只要“人生得意”便无所遗憾,当纵情欢乐。五六两句便是一个逆转,由“悲”而翻作“欢“”乐”。从此直到“杯莫停”,诗情渐趋狂放。“人生达命岂暇愁,且饮美酒登高楼”(《梁园吟》),行乐不可无酒,这就入题。但句中没有直写杯中之物,而用“金樽”、“对月”的形象语言来突出隐喻,更将饮酒诗意化了;未直写应该痛饮狂欢,而以“莫使”、“空”的双重否定句式代替直陈,语气更为强调。“人生得意须尽欢”,这似乎是宣扬及时行乐的思想,然而只不过是现象而已。诗人此时郁郁不得志。“凤凰初下紫泥诏,谒帝称觞登御筵”(《玉壶吟》),奉诏进京、皇帝赐宴的时候似乎得意过,然而那不过是一场幻影。再到“弹剑作歌奏苦声,曳裾王门不称情”(《行路难三首》其二),古时冯谖在孟尝君门下作客,觉得孟尝君对自己不够礼遇,开始时经常弹剑而歌,表示要回去。李白希望“平交王侯”的,而在长安,权贵们并不把他当一回事,李白借冯谖的典故比喻自己的处境。这时又似乎并没有得意,有的是失望与愤慨。但并不就此消沉。诗人于是用乐观好强的口吻肯定人生,肯定自我:“天生我材必有用”,这是一个令人击节赞叹的句子。“有用”而“必”,非常自信,简直像是人的价值宣言,而这个人——“我”——是须大写的。于此,从貌似消极的现象中露出了深藏其内的一种怀才不遇而又渴望入世的积极的本质内容来。正是“长风破浪会有时”,实现自我理想的这一天总会来到的,应为这样的未来痛饮高歌,破费又算得了什么。“千金散尽还复来!”这又是一个高度自信的惊人之句,能驱使金钱而不为金钱所使,真足令一切凡夫俗子们咋舌。诗如其人,想诗人“曩者(过去)游维扬,不逾一年(不到一年),散金三十余万”(《上安州裴长史书》),是何等豪举。故此句深蕴在骨子里的豪情,绝非装腔作势者可得其万一。与此气派相当,作者描绘了一场盛筵,那决不是“菜要一碟乎,两碟乎?酒要一壶乎,两壶乎?”而是整头整头地“烹羊宰牛”,不喝上“三百杯”决不甘休。筵宴中展示的痛快气氛,诗句豪壮。

至此,狂放之情趋于高潮,诗的旋律加快。诗人那眼花耳热的醉态跃然纸上,恍然使人如闻其高声劝酒:“岑夫子,丹丘生,将进酒,杯莫停!”几个短句忽然加入,不但使诗歌节奏富于变化,而且写来逼肖席上声口。既是生逢知己,又是酒逢对手,不但“忘形到尔汝”,诗人甚而忘却是在写诗,笔下之诗似乎还原为生活,他还要“与君歌一曲,请君为我倾耳听”。以下八句就是诗中之歌了。这着想奇之又奇,纯系神来之笔。

“钟鼓馔玉”意即富贵生活(富贵人家吃饭时鸣钟列鼎,食物精美如玉),可诗人以为“不足贵”,并放言“但愿长醉不复醒”。诗情至此,便分明由狂放转而为愤激。这里不仅是酒后吐狂言,而且是酒后吐真言了。以“我”天生有用之才,本当位至卿相,飞黄腾达,然而“大道如青天,我独不得出”(《行路难》)。说富贵“不足贵”,乃出于愤慨。以下“古来圣贤皆寂寞”二句亦属愤语。李白曾喟叹“自言管葛竟谁许”,称自己有管仲之才,诸葛亮之智却没人相信,所以说古人“寂寞”,同时表现出自己“寂寞”。因此才情愿醉生梦死长醉不醒了。这里,诗人已是用古人酒杯,浇自己块垒了。说到“唯有饮者留其名”,便举出“陈王”曹植作代表,并化用其《名都篇》“归来宴平乐,美酒斗十千”之句。古来酒徒历历,而偏举“陈王”,这与李白一向自命不凡分不开,他心目中树为榜样的是谢安之类高级人物,而这类人物中,“陈王”与酒联系较多。这样写便有气派,与前文极度自信的口吻一贯。再者,“陈王”曹植于丕、睿两朝备受猜忌,有志难展,亦激起诗人的同情。一提“古来圣贤”,二提“陈王”曹植,满纸不平之气。此诗开始似只涉人生感慨,而不染政治色彩,其实全篇饱含一种深广的忧愤和对自我的信念。诗情所以悲而不伤,悲而能壮,即根源于此。

刚露一点深衷,又回到说酒了,酒兴更高。以下诗情再入狂放,而且愈来愈狂。“主人何为言少钱”,既照应“千金散尽”句,又故作跌宕,引出最后一番豪言壮语:即便千金散尽,也当不惜将出名贵宝物——“五花马”(毛色作五花纹的良马)、“千金裘”来换取美酒,图个一醉方休。这结尾之妙,不仅在于“呼儿”、“与尔”,口气甚大;而且具有一种作者一时可能觉察不到的将宾作主的任诞情态。须知诗人不过是被友招饮的客人,此刻他却高踞一席,气使颐指,提议典裘当马,几令人不知谁是“主人”。浪漫色彩极浓。快人快语,非不拘形迹的豪迈知交断不能出此。诗情至此狂放至极,令人嗟叹咏歌,直欲“手之舞之,足之蹈之”。情犹未已,诗已告终,突然又迸出一句“与尔同销万古愁”,与开篇之“悲”关合,而“万古愁”的含义更其深沉。这“白云从空,随风变灭”的结尾,显见诗人奔涌跌宕的感情激流。通观全篇,真是大起大落,非如椽巨笔不办。

《将进酒》篇幅不算长,却五音繁会,气象不凡。它笔酣墨饱,情极悲愤而作狂放,语极豪纵而又沉着。诗篇具有震动古今的气势与力量,这诚然与夸张手法不无关系,比如诗中屡用巨额数目字(“千金”、“三百杯”、“斗酒十千”、“千金裘”、“万古愁”等等)表现豪迈诗情,同时,又不给人空洞浮夸感,其根源就在于它那充实深厚的内在感情,那潜在酒话底下如波涛汹涌的郁怒情绪。此外,全篇大起大落,诗情忽翕忽张,由悲转乐、转狂放、转愤激、再转狂放、最后结穴于“万古愁”,回应篇首,如大河奔流,有气势,亦有曲折,纵横捭阖,力能扛鼎。其歌中有歌的包孕写法,又有鬼斧神工、“绝去笔墨畦径”之妙,既不是刻意刻画和雕凿能学到的,也不是草率就可达到的境界。通篇以七言为主,而以三、五十言句“破”之,极参差错综之致;诗句以散行为主,又以短小的对仗语点染(如“岑夫子,丹丘生”“五花马,千金裘”),节奏疾徐尽变,奔放而不流易。{[}6{]}{[}7{]}{[}8{]}


\section{1.6   名家点评}
\label{\detokenize{p01_u6563_u6587/_u674e_u767d-_u5c06_u8fdb_u9152:id8}}
《李太白诗集》:严羽评:一结豪情,使人不能句字赏摘。盖他人作诗用笔想,太白但用胸口一喷即是,此其所长。

《唐诗广选》:转折动荡自然(“岑夫子”二句下)。杨升庵曰:太白狂歌。实中玄理,非故为狂语者。

《唐诗解》卷上:此怀才不遇,托于酒以自放也。

《唐诗选脉会通评林》:周珽曰:首以“黄河”起兴,见人之年貌倏改,有如河流莫返。一篇主意全在“人生得意须尽欢,莫使金樽空对月”两句。

《此木轩论诗汇编》:“惟有饮者留其名”,乱道故妙,一学便俗。

《古唐诗合解》:太白此歌豪放极矣。

《而庵说唐诗》:太白此歌,最为豪放,才气干古无双。

《唐诗选胜直解》:此诗妙在自解又以劝人。“主人”是谁?“对君”是谁?骂尽窃高位、守钱虏辈,妙,妙!

《唐诗合选详解》:王翼云曰:此篇用长短句为章法,篇首两个“君不见”领起,亦一局也。

《唐宋诗举要》:吴曰:驱迈淋漓之气(“人生得意”一句下)。吴曰:豪健(末句下)。

《李太白诗醇》:一起奇想,亦自天外来。


\section{1.7   作者简介}
\label{\detokenize{p01_u6563_u6587/_u674e_u767d-_u5c06_u8fdb_u9152:id9}}
李白(701~762),字太白,号青莲居士。是屈原之后最具个性特色、最伟大的浪漫主义诗人。有“诗仙”之美誉,与杜甫并称“李杜”。其诗以抒情为主,表现出蔑视权贵的傲岸精神,对人民疾苦表示同情,又善于描绘自然景色,表达对祖国山河的热爱。诗风雄奇豪放,想象丰富,语言流转自然,音律和谐多变,善于从民间文艺和神话传说中吸取营养和素材,构成其特有的瑰玮绚烂的色彩,达到盛唐诗歌艺术的巅峰。存世诗文千余篇,有《李太白集》30卷。


\chapter{1   李白-梦游天姥吟留别}
\label{\detokenize{p01_u6563_u6587/_u674e_u767d-_u68a6_u6e38_u5929_u59e5_u541f_u7559_u522b:id1}}\label{\detokenize{p01_u6563_u6587/_u674e_u767d-_u68a6_u6e38_u5929_u59e5_u541f_u7559_u522b::doc}}
\begin{sphinxShadowBox}
\sphinxstyletopictitle{目录}
\begin{itemize}
\item {} 
\phantomsection\label{\detokenize{p01_u6563_u6587/_u674e_u767d-_u68a6_u6e38_u5929_u59e5_u541f_u7559_u522b:id10}}{\hyperref[\detokenize{p01_u6563_u6587/_u674e_u767d-_u68a6_u6e38_u5929_u59e5_u541f_u7559_u522b:id1}]{\sphinxcrossref{1   李白-梦游天姥吟留别}}}
\begin{itemize}
\item {} 
\phantomsection\label{\detokenize{p01_u6563_u6587/_u674e_u767d-_u68a6_u6e38_u5929_u59e5_u541f_u7559_u522b:id11}}{\hyperref[\detokenize{p01_u6563_u6587/_u674e_u767d-_u68a6_u6e38_u5929_u59e5_u541f_u7559_u522b:id3}]{\sphinxcrossref{1.1   作品原文}}}

\item {} 
\phantomsection\label{\detokenize{p01_u6563_u6587/_u674e_u767d-_u68a6_u6e38_u5929_u59e5_u541f_u7559_u522b:id12}}{\hyperref[\detokenize{p01_u6563_u6587/_u674e_u767d-_u68a6_u6e38_u5929_u59e5_u541f_u7559_u522b:id4}]{\sphinxcrossref{1.2   词句注释}}}

\item {} 
\phantomsection\label{\detokenize{p01_u6563_u6587/_u674e_u767d-_u68a6_u6e38_u5929_u59e5_u541f_u7559_u522b:id13}}{\hyperref[\detokenize{p01_u6563_u6587/_u674e_u767d-_u68a6_u6e38_u5929_u59e5_u541f_u7559_u522b:id5}]{\sphinxcrossref{1.3   白话译文}}}

\item {} 
\phantomsection\label{\detokenize{p01_u6563_u6587/_u674e_u767d-_u68a6_u6e38_u5929_u59e5_u541f_u7559_u522b:id14}}{\hyperref[\detokenize{p01_u6563_u6587/_u674e_u767d-_u68a6_u6e38_u5929_u59e5_u541f_u7559_u522b:id6}]{\sphinxcrossref{1.4   创作背景}}}

\item {} 
\phantomsection\label{\detokenize{p01_u6563_u6587/_u674e_u767d-_u68a6_u6e38_u5929_u59e5_u541f_u7559_u522b:id15}}{\hyperref[\detokenize{p01_u6563_u6587/_u674e_u767d-_u68a6_u6e38_u5929_u59e5_u541f_u7559_u522b:id7}]{\sphinxcrossref{1.5   作品鉴赏}}}

\item {} 
\phantomsection\label{\detokenize{p01_u6563_u6587/_u674e_u767d-_u68a6_u6e38_u5929_u59e5_u541f_u7559_u522b:id16}}{\hyperref[\detokenize{p01_u6563_u6587/_u674e_u767d-_u68a6_u6e38_u5929_u59e5_u541f_u7559_u522b:id8}]{\sphinxcrossref{1.6   名家点评}}}

\item {} 
\phantomsection\label{\detokenize{p01_u6563_u6587/_u674e_u767d-_u68a6_u6e38_u5929_u59e5_u541f_u7559_u522b:id17}}{\hyperref[\detokenize{p01_u6563_u6587/_u674e_u767d-_u68a6_u6e38_u5929_u59e5_u541f_u7559_u522b:id9}]{\sphinxcrossref{1.7   作者简介}}}

\end{itemize}

\end{itemize}
\end{sphinxShadowBox}

《梦游天姥吟留别》是唐代大诗人李白的诗作。这是一首记梦诗,也是一首游仙诗。此诗以记梦为由,抒写了对光明、自由的渴求,对黑暗现实的不满,表现了蔑视权贵、不卑不屈的叛逆精神。诗人运用丰富奇特的想象和大胆夸张的手法,组成一幅亦虚亦实、亦幻亦真的梦游图。全诗构思精密,意境雄伟,内容丰富曲折,形象辉煌流丽,感慨深沉激烈,富有浪漫主义色彩。其在形式上杂言相间,兼用骚体,不受律束,笔随兴至,体制解放,堪称绝世名作。


\section{1.1   作品原文}
\label{\detokenize{p01_u6563_u6587/_u674e_u767d-_u68a6_u6e38_u5929_u59e5_u541f_u7559_u522b:id3}}
梦游天姥吟留别1

海客谈瀛洲,烟涛微茫信难求2。

越人语天姥3,云霞明灭或可睹4。

天姥连天向天横5,势拔五岳掩赤城6。

天台四万八千丈7,对此欲倒东南倾8。

我欲因之梦吴越9,一夜飞度镜湖月10。

湖月照我影,送我至剡溪11。

谢公宿处今尚在12,渌水荡漾清猿啼13。

脚著谢公屐14,身登青云梯15。

半壁见海日16,空中闻天鸡17。

千岩万转路不定,迷花倚石忽已暝18。

熊咆龙吟殷岩泉19,栗深林兮惊层巅20。

云青青兮欲雨21,水澹澹兮生烟22。

列缺霹雳23,丘峦崩摧。

洞天石扉24,訇然中开25。

青冥浩荡不见底26,日月照耀金银台27。

霓为衣兮风为马28,云之君兮纷纷而来下29。

虎鼓瑟兮鸾回车30,仙之人兮列如麻。

忽魂悸以魄动31,恍惊起而长嗟32。

惟觉时之枕席33,失向来之烟霞34。

世间行乐亦如此,古来万事东流水35。

别君去兮何时还?

且放白鹿青崖间36,须行即骑访名山37。

安能摧眉折腰事权贵38,使我不得开心颜!


\section{1.2   词句注释}
\label{\detokenize{p01_u6563_u6587/_u674e_u767d-_u68a6_u6e38_u5929_u59e5_u541f_u7559_u522b:id4}}
1.天姥山:在浙江新昌东面。传说登山的人能听到仙人天姥唱歌的声音,山因此得名。

2.瀛洲:古代传说中的东海三座仙山之一(另两座叫蓬莱和方丈)。烟涛:波涛渺茫,远看像烟雾笼罩的样子。微茫:景象模糊不清。信:确实,实在。

3.越人:指浙江一带的人。

4.明灭:忽明忽暗。

5.向天横:直插天空。横,直插。

6.”势拔“句:山势高过五岳,遮掩了赤城。拔,超出。五岳,指东岳泰山、西岳华山、中岳嵩山、北岳恒山、南岳衡山。赤城,山名,在浙江天台西北。

7.天台(tāi):山名,在浙江天台北部。

8.”对此“句:对着天姥这座山,天台山就好像要倒向它的东南一样。意思是天台山和天姥山相比,显得低多了。

9.因:依据。之:指代前边越人的话。

10.度:一作“渡”{[}2{]}。镜湖:又名鉴湖,在浙江绍兴南面。

11.剡(shàn)溪:水名,在浙江嵊州南面。

12.谢公:指南朝诗人谢灵运。谢灵运喜欢游山。游天姥山时,他曾在剡溪这个地方住宿。

13.渌(lù):清。清:这里是凄清的意思。

14.谢公屐(jī):谢灵运穿的那种木屐。《南史·谢灵运传》记载:谢灵运游山,必到幽深高峻的地方;他备有一种特制的木屐,屐底装有活动的齿,上山时去掉前齿,下山时去掉后齿。木屐,以木板作底,上面有带子,形状像拖鞋。

15.青云梯:指直上云霄的山路。

16.半壁见海日:上到半山腰就看到从海上升起的太阳。

17.天鸡:古代传说,东南有桃都山,山上有棵大树叫桃都,树枝绵延三千里,树上栖有天鸡,每当太阳初升,照到这棵树上,天鸡就叫起来,天下的鸡也都跟着它叫。

18.”迷花“句:迷恋着花,依靠着石,不觉天色已经很晚了。暝(míng),日落,天黑。

19.”熊咆“句:熊在怒吼,龙在长鸣,岩中的泉水在震响。殷(yǐn),这里用作动词,震响。

20.”栗深林“句:使深林战栗,使层巅震惊。栗、惊,使动用法。

21.青青:黑沉沉的。

22.澹澹:波浪起伏的样子。

23.列缺:指闪电。

24.洞天:仙人居住的洞府。扉:门扇。一作“扇”。

25.訇(hōng)然:形容声音很大。

26.青冥:指天空。浩荡:广阔远大的样子。

27.金银台:金银铸成的宫阙,指神仙居住的地方。

28.风:一作“凤”。

29.云之君:云里的神仙。

30.鸾回车:鸾鸟驾着车。鸾,传说中的如凤凰一类的神鸟。回,旋转,运转。

31.魂悸:心跳。

32.恍:恍然,猛然。

33.觉时:醒时。

34.失向来之烟霞:刚才梦中所见的烟雾云霞消失了。向来,原来。烟霞,指前面所写的仙境。

35.东流水:像东流的水一样一去不复返。

36.白鹿:传说神仙或隐士多骑白鹿。青崖:青山。

37.须:等待。

38.摧眉折腰:低头弯腰。摧眉,即低眉。


\section{1.3   白话译文}
\label{\detokenize{p01_u6563_u6587/_u674e_u767d-_u68a6_u6e38_u5929_u59e5_u541f_u7559_u522b:id5}}
海外来客们谈起瀛洲,烟波渺茫实在难以寻求。

越中来人说起天姥山,在云雾忽明忽暗间有人可以看见。

天姥山仿佛连接着天遮断了天空,山势高峻超过五岳,遮掩过赤城山。

天台山虽高一万八千丈,面对着它好像要向东南倾斜拜倒一样。

我根据越人说的话梦游到吴越,一天夜晚飞渡过明月映照下的镜湖。

镜湖上的月光照着我的影子,一直伴随我到了剡溪。

谢灵运住的地方如今还在,清澈的湖水荡漾,猿猴清啼。

我脚上穿着谢公当年特制的木鞋,攀登直上云霄的山路。

上到半山腰就看见了从海上升起的太阳,在半空中传来天鸡报晓的叫声。

无数山岩重叠,道路盘旋弯曲,方向不定,迷恋着花,依倚着石头,不觉天色已晚。

熊在怒吼,龙在长鸣,岩中的泉水在震响,使森林战栗,使山峰惊颤。

云层黑沉沉的,像是要下雨,水波动荡生起了烟雾。

电光闪闪,雷声轰鸣,山峰好像要被崩塌似的。

仙府的石门,“訇”的一声从中间打开。

洞中蔚蓝的天空广阔无际,看不到尽头,日月照耀着金银做的宫阙。

用彩虹做衣裳,将风作为马来乘,云中的神仙们纷纷下来。

老虎弹奏着琴瑟,鸾鸟驾着车,仙人们成群结队密密如麻。

忽然魂魄惊动,我猛然惊醒,不禁长声叹息。

醒来时只有身边的枕席,刚才梦中所见的烟雾云霞全都消失了。

人世间的欢乐也是像梦中的幻境这样,自古以来万事都像东流的水一样一去不复返。

告别诸位朋友远去东鲁啊,什么时候才能回来?

暂且把白鹿放牧在青崖间,等到要远行时就骑上它访名山。

岂能卑躬屈膝去侍奉权贵,使我不能有舒心畅意的笑颜!


\section{1.4   创作背景}
\label{\detokenize{p01_u6563_u6587/_u674e_u767d-_u68a6_u6e38_u5929_u59e5_u541f_u7559_u522b:id6}}
此诗作于李白出翰林之后,其作年一说天宝四载(745年),一说天宝五载(746年)。唐玄宗天宝三载(744年),李白在长安受到权贵的排挤,被放出京,返回东鲁(在今山东)家园。之后再度踏上漫游的旅途。这首描绘梦中游历天姥山的诗,大约作于李白即将离开东鲁南游吴越之时。

李白早年就有济世的抱负,但不屑于经由科举登上仕途。因此他漫游全国各地,结交名流,以此广造声誉。唐玄宗天宝元年(742年),李白的朋友道士吴筠向玄宗推荐李白,玄宗于是召他到长安来。李白对这次长安之行抱有很大的希望,在给妻子的留别诗《别内赴征》中写道:“归时倘佩黄金印,莫见苏秦不下机。”李白初到长安,也曾有过短暂的得意,但他一身傲骨,不肯与权贵同流合污,又因得罪了权贵,及翰林院同事进谗言,连玄宗也对他不满。他在长安仅住了一年多,就被唐玄宗赐金放还,他那由布衣而卿相的梦幻从此完全破灭。这是李白政治上的一次大失败。离开长安后,他曾与杜甫、高适游梁、宋、齐、鲁,又在东鲁家中居住过一个时期。这时东鲁的家已颇具规模,尽可在家中怡情养性,以度时光。可是李白没有这么做。他有一个不安定的灵魂,他有更高更远的追求,于是离别东鲁家园,又一次踏上漫游的旅途。这首诗就是他告别东鲁朋友时所作,所以又题作“梦游天姥山别东鲁诸公”。{[}4{]}{[}5{]}{[}6{]}{[}7{]}{[}8{]}


\section{1.5   作品鉴赏}
\label{\detokenize{p01_u6563_u6587/_u674e_u767d-_u68a6_u6e38_u5929_u59e5_u541f_u7559_u522b:id7}}
这是一首记梦诗,也是一首游仙诗。意境雄伟,变化惝恍莫测,缤纷多采的艺术形象,新奇的表现手法,向来为人传诵,被视为李白的代表作之一。

这首诗的思想内容相当复杂。李白从离开长安后,因政治上遭受挫折,精神上的苦闷愤怨郁结于怀。在现实社会中找不到出路,只有向虚幻的神仙世界和远离尘俗的山林去寻求解脱。这种遁世思想看似消沉,却不能一笔抹杀,它在一定程度上表现了李白在精神上摆脱了尘俗的桎梏。而这才导致他在诗的最后发出“安能摧眉折腰事权贵,使我不得开心颜”那样激越的呼声。这种坚决不妥协的精神和强烈的反抗情绪正是这首诗的基调。

李白一生徜徉山水之间,热爱山水,达到梦寐以求的境地。此诗所描写的梦游,也许并非完全虚托,但无论是否虚托,梦游就更适于超脱现实,更便于发挥他的想象和夸张的才能了。

“海客谈瀛洲,烟涛微茫信难求;越人语天姥,云霓明灭或可睹。”诗一开始先说古代传说中的海外仙境──瀛洲,虚无缥缈,不可寻求;而现实中的天姥山在浮云彩霓中时隐时现,真是胜似仙境。以虚衬实,突出了天姥胜景,暗蕴着诗人对天姥山的向往,写得富有神奇色彩,引人入胜。

天姥山临近剡溪,传说登山的人听到过仙人天姥的歌唱,因此得名。天姥山与天台山相对,峰峦峭峙,仰望如在天表,冥茫如堕仙境,容易引起游者想入非非的幻觉。浙东山水是李白青年时代就向往的地方,初出川时曾说“此行不为鲈鱼鲙,自爱名山入剡中”。入翰林前曾不止一次往游,他对这里的山水不但非常热爱,也是非常熟悉的。

天姥山号称奇绝,是越东灵秀之地。但比之其他崇山峻岭如我国的五大名山──五岳,在人们心目中的地位仍有小巫见大巫之别。可是李白却在诗中夸说它“势拔五岳掩赤城”,比五岳还更挺拔。有名的天台山则倾斜着如拜倒在天姥的足下一样。这个天姥山,被写得耸立天外,直插云霄,巍巍然非同凡比。这座梦中的天姥山,应该说是李白平生所经历的奇山峻岭的幻影,它是现实中的天姥山在李白笔下夸大了的影子。

接着展现出的是一幅一幅瑰丽变幻的奇景:天姥山隐于云霓明灭之中,引起了诗人探求的想望。诗人进入了梦幻之中,仿佛在月夜清光的照射下,他飞渡过明镜一样的镜湖。明月把他的影子映照在镜湖之上,又送他降落在谢灵运当年曾经歇宿过的地方。他穿上谢灵运当年特制的木屐,登上谢公当年曾经攀登过的石径──青云梯。只见:“半壁见海日,空中闻天鸡。千岩万转路不定,迷花倚石忽已暝。熊咆龙吟殷岩泉,栗深林兮惊层巅。云青青兮欲雨,水澹澹兮生烟。”继飞渡而写山中所见,石径盘旋,深山中光线幽暗,看到海日升空,天鸡高唱,这本是一片曙色;却又于山花迷人、倚石暂憩之中,忽觉暮色降临,旦暮之变何其倏忽。暮色中熊咆龙吟,震响于山谷之间,深林为之战栗,层巅为之惊动。不止有生命的熊与龙以吟、咆表示情感,就连层巅、深林也能战栗、惊动,烟、水、青云都满含阴郁,与诗人的情感,协成一体,形成统一的氛围。前面是浪漫主义地描写天姥山,既高且奇;这里又是浪漫主义地抒情,既深且远。这奇异的境界,已经使人够惊骇的了,但诗人并未到此止步,而诗境却由奇异而转入荒唐,全诗也更进入高潮。在令人惊悚不已的幽深暮色之中,霎时间“丘峦崩摧”,一个神仙世界“訇然中开”,“青冥浩荡不见底,日月照耀金银台。霓为衣兮风为马,云之君兮纷纷而来下。”洞天福地,于此出现。“云之君”披彩虹为衣,驱长风为马,虎为之鼓瑟,鸾为之驾车,皆受命于诗人之笔,奔赴仙山的盛会来了。这是多么盛大而热烈的场面。“仙之人兮列如麻”!群仙好像列队迎接诗人的到来。金台、银台与日月交相辉映,景色壮丽,异彩缤纷,何等的惊心眩目,光耀夺人!仙山的盛会正是人世间生活的反映。这里除了有他长期漫游经历过的万壑千山的印象、古代传说、屈原诗歌的启发与影响,也有长安三年宫廷生活的迹印,这一切通过浪漫主义的非凡想象凝聚在一起,才有这般辉煌灿烂、气象万千的描绘。

这首诗写梦游奇境,不同于一般游仙诗,它感慨深沉,抗议激烈,并非真正依托于虚幻之中,而是在神仙世界虚无飘渺的描述中,依然着眼于现实。神游天上仙境,而心觉“世间行乐亦如此”。

仙境倏忽消失,梦境旋亦破灭,诗人终于在惊悸中返回现实。梦境破灭后,人,不是随心所欲地轻飘飘地在梦幻中翱翔了,而是沉甸甸地躺在枕席之上。“古来万事东流水”,其中包含着诗人对人生的几多失意和深沉的感慨。此时此刻诗人感到最能抚慰心灵的是“且放白鹿青崖间,须行即骑访名山”。徜徉山水的乐趣,才是最快意的,也就是在《春夜宴从弟桃花园序》中所说:“古人秉烛夜游,良有以也。”本来诗意到此似乎已尽,可是最后却愤愤然加添了两句“安能摧眉折腰事权贵,使我不得开心颜!”一吐长安三年的郁闷之气。天外飞来之笔,点亮了全诗的主题:对于名山仙境的向往,是出之于对权贵的抗争,它唱出封建社会中多少怀才不遇的人的心声。在等级森严的封建社会中,多少人屈身权贵,多少人埋没无闻!唐朝比之其他朝代是比较开明的,较为重视人才,但也只是比较而言。人才在当时仍然摆脱不了“臣妾气态间”的屈辱地位。“折腰”一词出之于东晋的陶渊明,他由于不愿忍辱而赋《归去来兮辞》。李白虽然受帝王优宠,也不过是个词臣,在宫廷中所受到的屈辱,大约可以从这两句诗中得到一些消息。封建君主把自己称“天子”,君临天下,把自己升高到至高无上的地位,却抹煞了一切人的尊严。李白在这里所表示的决绝态度,是向封建统治者所投过去的一瞥蔑视。在封建社会,敢于这样想、敢于这样说的人并不多。李白说了,也做了,这是他异乎常人的伟大之处。

这首诗的内容丰富、曲折、奇谲、多变,它的形象辉煌流丽,缤纷多彩,构成了全诗的浪漫主义华赡情调。它的主观意图本来在于宣扬“古来万事东流水”这样颇有消极意味的思想,可是它的格调却是昂扬振奋的,潇洒出尘的,有一种不卑不屈的气概流贯其间,并无消沉之感。{[}7{]}{[}8{]}


\section{1.6   名家点评}
\label{\detokenize{p01_u6563_u6587/_u674e_u767d-_u68a6_u6e38_u5929_u59e5_u541f_u7559_u522b:id8}}
明代高棅《唐诗品汇》:范云:瀛洲难求而不必求,天姥可睹而实未睹,故欲因梦而睹之耳(“海客”四句下)。甚显(“半壁”二句下)。甚晦(“千岩万转”二句下)。又甚显(“洞天”四句下)。又甚晦(“霓为衣兮”四句下)。范云:“梦吴越”以下,梦之源也;次诸节,梦之波澜。其间显而晦,晦而显,至“失向来之烟霞”极而与人接矣,非太白之胸次、笔力,亦不能发此。“枕席”“烟霞”二句最有力。结语平衍,亦文势之当如此也。

明代桂天祥《批点唐诗正声》:《梦游天姥吟》胸次皆烟霞云石,无分毫尘浊,别是一副言语,故特为难到。

明代郭濬《增订评注唐诗正声》:郭云:恍恍惚惚,奇奇幻幻,非满肚皮烟霞,决挥洒不出。

明代周敬、周珽《唐诗选脉会通评林》:周珽曰:出于千丝铁网之思,运以百色流苏之局,忽而飞步凌顶,忽而烟云自舒。想其拈笔时,神魂毛发尽脱于毫楮而不自知,其神耶!吴山民曰:“天台四万八千丈”,形容语,“白发三千丈”同意,有形容天姥高意。“千岩万转”句,语有概括。下三句,梦中危景。又八句,梦中奇景。又四句,梦中所遇。“唯觉时之枕席”二语,篇中神句,结上启下。“世间行乐”二句,因梦生意。结超。

清代朱之荆《增订唐诗摘钞》:“忽魂”四句,束上生下,笔意最紧。万斛之舟,收于一柁(末二句下)。

清代沈德潜《唐诗别裁》:“飞渡镜湖月”以下,皆言梦中所历。一路离奇灭没,恍恍惚惚,是梦境,是仙境(“列缺霹雳”十二句下)。托言梦游,穷形尽相以极“洞天”之奇幻;至酲后,顿失烟霞矣。知世间行乐,亦同一梦,安能于梦中屈身权贵乎?吾当别去,遍游名山,以终天年也。诗境虽奇,脉理极细。

清高宗敕编《唐宋诗醇》:七古歌行,本出楚骚、乐府。至于太白,然后穷极笔力,优入圣域。昔人谓其“以气为主,以自然为宗,以俊逸高畅为贵,咏之使人飘飘欲仙”,而尤推其《天姥吟》《远别离》等篇,以为虽子美不能道。盖其才横绝一世,故兴会标举,非学可及,正不必执此谓子美不能及也。此篇夭矫离奇,不可方物,然因语而梦,因梦而悟,因悟而别,节次柑生,丝毫不乱;若中间梦境迷离,不过词意伟怪耳。胡应麟以为“无首无尾,窈冥昏默”,是真不可以说梦也特谓非其才力,学之立见踬踣,则诚然耳。

清代翁方纲《赵秋谷所传声调谱》:方纲按:《扶风豪士歌》《梦游天姥吟》二篇,虽句法、音节极其变化,然实皆自然入拍,非任意参错也。秋谷于《豪士》篇但评其神变,于《天姥》篇则第云“观此知转韵元无定格”,正恐难以示后学耳。

清代宋宗元《网师园唐诗笺》:纵横变化,离奇光怪,以奇笔写梦境,吐句皆仙,着纸谷飞(“列缺霹雳”十句下)。砉然收勒,通体宗主攸在,线索都灵(“世间行乐”二句下)。

清代方东树《昭昧詹言》:陪起,令人迷。“我欲”以下正叙梦,愈唱愈高,愈出愈奇“失向”句,收住。“世间”二句,入作意,因梦游推开,见世事皆成虚幻也;不如此,则作诗之旨无归宿。留别意,只末后一点。韩《记梦》之本。

清代延君寿《老生常谈》:《梦游天姥吟留别》诗,奇离惝恍,似无门径可寻。细玩之,起首入梦不突,后幅出梦不竭,极恣肆幻化之中,又极经营惨淡之苦,若只貌其右句字面,则失之远矣。一起淡淡引入,至“我欲因之梦吴越”句,乘势即入,使笔如风,所谓缓则按辔徐行,急则短兵相接也。“湖月照我影”八句,他人捉笔可云已尽能事矣,岂料后边尚有许多奇奇怪怪。“千岩万转”二句,用仄韵一束以下至“仙之人兮”句,转韵不转气,全以笔力驱驾,遂成鞭山倒海之能,读云似未曾转韵者,有真气行乎其间也。此妙可心悟,不可言喻。出梦时,用“忽动悸以魄动”四句,似亦可以收煞得住,试想若不再足“世间行乐”二句,非但叫题不酲,抑亦尚欠圆满。“且放白鹿”二句,一纵一收,用笔灵妙不测。后来慢东坡解此法,他人多昧昧耳。

日本近藤元粹《李太白诗醇》:严云:“半壁”一句,不独境界超绝,语音亦复高朗。严云:有意味在“青青”“澹澹”字作叠(“云青青兮”二句下)。严云:太白写仙人境界皆渺茫寂历,独此一段极真,极雄,反不似梦中语(“霓为衣兮”四句下)。又云:“世间”云云,甚达,甚警策,然自是唐人语,无宋气。{[}9{]}


\section{1.7   作者简介}
\label{\detokenize{p01_u6563_u6587/_u674e_u767d-_u68a6_u6e38_u5929_u59e5_u541f_u7559_u522b:id9}}
李白(701—762),字太白,号青莲居士。是屈原之后最具个性特色、最伟大的浪漫主义诗人。有“诗仙”之美誉,与杜甫并称“李杜”。其诗以抒情为主,表现出蔑视权贵的傲岸精神,对人民疾苦表示同情,又善于描绘自然景色,表达对祖国山河的热爱。诗风雄奇豪放,想象丰富,语言流转自然,音律和谐多变,善于从民间文艺和神话传说中吸取营养和素材,构成其特有的瑰玮绚烂的色彩,达到盛唐诗歌艺术的巅峰。存世诗文千余篇,有《李太白集》30卷。


\chapter{1   李白-蜀道难}
\label{\detokenize{p01_u6563_u6587/_u674e_u767d-_u8700_u9053_u96be:id1}}\label{\detokenize{p01_u6563_u6587/_u674e_u767d-_u8700_u9053_u96be::doc}}
\begin{sphinxShadowBox}
\sphinxstyletopictitle{目录}
\begin{itemize}
\item {} 
\phantomsection\label{\detokenize{p01_u6563_u6587/_u674e_u767d-_u8700_u9053_u96be:id10}}{\hyperref[\detokenize{p01_u6563_u6587/_u674e_u767d-_u8700_u9053_u96be:id1}]{\sphinxcrossref{1   李白-蜀道难}}}
\begin{itemize}
\item {} 
\phantomsection\label{\detokenize{p01_u6563_u6587/_u674e_u767d-_u8700_u9053_u96be:id11}}{\hyperref[\detokenize{p01_u6563_u6587/_u674e_u767d-_u8700_u9053_u96be:id3}]{\sphinxcrossref{1.1   作品原文}}}

\item {} 
\phantomsection\label{\detokenize{p01_u6563_u6587/_u674e_u767d-_u8700_u9053_u96be:id12}}{\hyperref[\detokenize{p01_u6563_u6587/_u674e_u767d-_u8700_u9053_u96be:id4}]{\sphinxcrossref{1.2   词句注释}}}

\item {} 
\phantomsection\label{\detokenize{p01_u6563_u6587/_u674e_u767d-_u8700_u9053_u96be:id13}}{\hyperref[\detokenize{p01_u6563_u6587/_u674e_u767d-_u8700_u9053_u96be:id5}]{\sphinxcrossref{1.3   白话译文}}}

\item {} 
\phantomsection\label{\detokenize{p01_u6563_u6587/_u674e_u767d-_u8700_u9053_u96be:id14}}{\hyperref[\detokenize{p01_u6563_u6587/_u674e_u767d-_u8700_u9053_u96be:id6}]{\sphinxcrossref{1.4   创作背景}}}

\item {} 
\phantomsection\label{\detokenize{p01_u6563_u6587/_u674e_u767d-_u8700_u9053_u96be:id15}}{\hyperref[\detokenize{p01_u6563_u6587/_u674e_u767d-_u8700_u9053_u96be:id7}]{\sphinxcrossref{1.5   整体赏析}}}

\item {} 
\phantomsection\label{\detokenize{p01_u6563_u6587/_u674e_u767d-_u8700_u9053_u96be:id16}}{\hyperref[\detokenize{p01_u6563_u6587/_u674e_u767d-_u8700_u9053_u96be:id8}]{\sphinxcrossref{1.6   历代评论}}}

\item {} 
\phantomsection\label{\detokenize{p01_u6563_u6587/_u674e_u767d-_u8700_u9053_u96be:id17}}{\hyperref[\detokenize{p01_u6563_u6587/_u674e_u767d-_u8700_u9053_u96be:id9}]{\sphinxcrossref{1.7   作者简介}}}

\end{itemize}

\end{itemize}
\end{sphinxShadowBox}

《蜀道难》是中国唐代大诗人李白的代表作品。此诗袭用乐府旧题,以浪漫主义的手法,展开丰富的想象,艺术地再现了蜀道峥嵘、突兀、强悍、崎岖等奇丽惊险和不可凌越的磅礴气势,借以歌咏蜀地山川的壮秀,显示出祖国山河的雄伟壮丽,充分显示了诗人的浪漫气质和热爱自然的感情。全诗二百九十四字,采用律体与散文间杂,文句参差,笔意纵横,豪放洒脱,感情强烈,一唱三叹。诗中诸多的画面此隐彼现,无论是山之高,水之急,河山之改观,林木之荒寂,连峰绝壁之险,皆有逼人之势,气象宏伟,境界阔大,集中体现了李白诗歌的艺术特色和创作个性,深受学者好评,被誉为“奇之又奇”之作。


\section{1.1   作品原文}
\label{\detokenize{p01_u6563_u6587/_u674e_u767d-_u8700_u9053_u96be:id3}}
蜀道难1

噫吁嚱2,危乎高哉!蜀道之难,难于上青天!

蚕丛及鱼凫3,开国何茫然4!尔来四万八千岁5,不与秦塞通人烟6。西当太白有鸟道7,可以横绝峨眉巅8。地崩山摧壮士死9,然后天梯石栈相钩连10。

上有六龙回日之高标11,下有冲波逆折之回川12。黄鹤之飞尚不得过13,猿猱欲度愁攀援14。青泥何盘盘15,百步九折萦岩峦16。扪参历井仰胁息17,以手抚膺坐长叹18。

问君西游何时还19?畏途巉岩不可攀20。但见悲鸟号古木21,雄飞雌从绕林间22。又闻子规啼夜月,愁空山23。蜀道之难,难于上青天,使人听此凋朱颜24。

连峰去天不盈尺25,枯松倒挂倚绝壁。飞湍瀑流争喧豗26,砯崖转石万壑雷27。其险也如此,嗟尔远道之人胡为乎来哉28!

剑阁峥嵘而崔嵬29,一夫当关,万夫莫开30。所守或匪亲31,化为狼与豺。

朝避猛虎32,夕避长蛇;磨牙吮血33,杀人如麻。锦城虽云乐34,不如早还家。蜀道之难,难于上青天,侧身西望长咨嗟35!


\section{1.2   词句注释}
\label{\detokenize{p01_u6563_u6587/_u674e_u767d-_u8700_u9053_u96be:id4}}
1.蜀道难:南朝乐府旧题,属《相和歌·瑟调曲》。

2.噫(yī)吁(xū)嚱(xī):惊叹声,蜀方言,表示惊讶的声音。宋庠《宋景文公笔记》卷上:“蜀人见物惊异,辄曰‘噫吁嚱’。”

3.蚕丛、鱼凫(fú):传说中古蜀国两位国王的名字;难以考证。

4.何茫然:何:多么。茫然:完全不知道的样子。指古史传说悠远难详,不知道。据西汉扬雄《蜀本王纪》记载:“蜀王之先,名蚕丛、柏灌、鱼凫,蒲泽、开明。……从开明上至蚕丛,积三万四千岁。”

5.尔来:从那时以来。四万八千岁:极言时间之漫长,夸张而大约言之。

6.秦塞(sài):秦的关塞,指秦地。秦地四周有山川险阻,故称”四塞之地”。通人烟:人员往来。

7.西当:在西边的。当:在。太白:太白山,又名太乙山,在长安西(今陕西眉县、太白县一带)。鸟道:指连绵高山间的低缺处,只有鸟能飞过,人迹所不能至。

8.横绝:横越。峨眉巅(diān):峨眉顶峰。苏教版语文课本为“峨眉颠”。

9.“地崩”句:《华阳国志·蜀志》:相传秦惠王想征服蜀国,知道蜀王好色,答应送给他五个美女。蜀王派五位壮士去接人。回到梓潼(今四川剑阁之南)的时候,看见一条大蛇进入穴中,一位壮士抓住了它的尾巴,其余四人也来相助,用力往外拽。不多时,山崩地裂,壮士和美女都被压死。山分为五岭,入蜀之路遂通。这便是有名的“五丁开山”的故事。摧,倒塌。

10.天梯:非常陡峭的山路。石栈(zhàn):栈道。

11.六龙回日:《淮南子》注云:“日乘车,驾以六龙。羲和御之。日至此面而薄于虞渊,羲和至此而回六螭。”,意思就是传说中的羲和驾驶着六龙之车(即太阳)到此处便迫近虞渊(传说中的日落处)。高标:指蜀山中可作一方之标识的最高峰。

12.冲波:水流冲击腾起的波浪,这里指激流。逆折:水流回旋。回川:有漩涡的河流。

13.黄鹤:即黄鹄(hú),善飞的大鸟。尚:尚且。得:能。

14.猿猱(náo):蜀山中最善攀援的猴类。

15.青泥:青泥岭,在今甘肃徽县南,陕西略阳县北。《元和郡县志》卷二十二:“青泥岭,在县西北五十三里,接溪山东,即今通路也。悬崖万仞,山多云雨,行者屡逢泥淖,故号青泥岭。”盘盘:曲折回旋的样子。

16.百步九折:百步之内拐九道弯。萦(yíng):盘绕。岩峦:山峰。

17.扪(mén)参(shēn)历井:参、井是二星宿名。古人把天上的星宿分别指配于地上的州国,叫做“分野”,以便通过观察天象来占卜地上所配州国的吉凶。参星为蜀之分野,井星为秦之分野。扪,用手摸。历,经过。胁息:屏气不敢呼吸。

18.膺(yīng):胸。坐:徒,空。

19.君:入蜀的友人。

20.畏途:可怕的路途。巉(chán)岩:险恶陡峭的山壁。

21.但见:只听见。号(háo)古木:在古树木中大声啼鸣。

22.从:跟随。

23.“又闻”二句:一本断为“又闻子规啼,夜月愁空山”。子规,即杜鹃鸟,蜀地最多,鸣声悲哀,若云“不如归去”。《蜀记》曰:“昔有人姓杜名宇,王蜀,号曰望帝。宇死,俗说杜宇化为子规。子规,鸟名也。蜀人闻子规鸣,皆曰望帝也。”

24.凋朱颜:红颜带忧色,如花凋谢。凋,使动用法,使……凋谢,这里指脸色由红润变成铁青。

25.去:距离。盈:满。

26.飞湍(tuān):飞奔而下的急流。喧豗(huī):喧闹声,这里指急流和瀑布发出的巨大响声。

27.砯(pīng)崖:水撞石之声。砯,水冲击石壁发出的响声,这里作动词用,冲击的意思。转:使滚动。壑(hè):山谷。

28.嗟(jiē):感叹声。尔:你。胡为:为什么。来:指入蜀。

29.剑阁:又名剑门关,在四川剑阁县北,是大、小剑山之间的一条栈道,长约三十余里。峥嵘、崔(cuīwéi)嵬:都是形容山势高大雄峻的样子。

30.“一夫”两句:《文选》卷四左思《蜀都赋》:“一人守隘,万夫莫向”。《文选》卷五十六张载《剑阁铭》:“一人荷戟,万夫趦趄。形胜之地,匪亲勿居。”一夫,一人。当关,守关。莫开,不能打开。

31.所守:指把守关口的人。或匪(fěi)亲:倘若不是可信赖的人。匪,同“非”。

32.朝(zhāo):早上。

33.吮(shǔn)血(xuè):吸血。

34.锦城:成都古代以产棉闻名,朝廷曾经设官于此,专收棉织品,故称锦城或锦官城。《元和郡县志》卷三十一剑南道成都府成都县:“锦城在县南十里,故锦官城也。”今四川成都市。

35.咨(zī)嗟:叹息。{[}2{]}{[}3-4{]}


\section{1.3   白话译文}
\label{\detokenize{p01_u6563_u6587/_u674e_u767d-_u8700_u9053_u96be:id5}}
唉呀呀,多么危险多么高峻伟岸!蜀道真太难攀简直难于上青天。

传说中蚕丛和鱼凫建立了蜀国,开国的年代实在久远无法详谈。自从那时至今约有四万八千年,秦蜀被秦岭所阻从不沟通往返。西边太白山有飞鸟能过的小道。从那小路走可横渡峨嵋山顶端。山崩地裂蜀国五壮士被压死了,两地才有天梯栈道开始相通连。

上有挡住太阳神六龙车的山巅,下有激浪排空纡回曲折的大川。善于高飞的黄鹤尚且无法飞过,即使猢狲要想翻过也愁于攀援。青泥岭多么曲折绕着山峦盘旋,百步之内萦绕岩峦转九个弯弯。可以摸到参井星叫人仰首屏息,用手抚胸惊恐不已坐下来长叹。

好朋友呵请问你西游何时回还?可怕的岩山栈道实在难以登攀!只见那悲鸟在古树上哀鸣啼叫,雄雌相随飞翔在原始森林之间。月夜听到的是杜鹃悲惨的啼声,令人愁思绵绵呵这荒荡的空山!蜀道真难走呵简直难于上青天,叫人听到这些怎么不脸色突变?

山峰座座相连离天还不到一尺;枯松老枝倒挂倚贴在绝壁之间。漩涡飞转瀑布飞泻争相喧闹着;水石相击转动象万壑鸣雷一般。那去处恶劣艰险到了这种地步;唉呀呀你这个远方而来的客人,为了什么要来到这个险要地方?

剑阁那地方崇峻巍峨高入云端,只要一人把守千军万马难攻占。驻守的官员若不是皇家的近亲;难免要变为豺狼踞此为非造反。

清晨你要提心吊胆地躲避猛虎,傍晚你要警觉防范长蛇的灾难。豺狼虎豹磨牙吮血真叫人不安,毒蛇猛兽杀人如麻即令你胆寒。锦官城虽然说是个快乐的所在,如此险恶还不如早早地把家还。蜀道太难走呵简直难于上青天,侧身西望令人不免感慨与长叹!{[}2{]}


\section{1.4   创作背景}
\label{\detokenize{p01_u6563_u6587/_u674e_u767d-_u8700_u9053_u96be:id6}}
对《蜀道难》的创作背景,从唐代开始人们就多有猜测,主要有四种说法:甲、此诗系为房琯、杜甫二人担忧,希望他们早日离开四川,免遭剑南节度使严武的毒手;乙、此诗是为躲避安史之乱逃亡至蜀的唐玄宗李隆基而作,劝喻他归返长安,以免受四川地方军阀挟制;丙、此诗旨在讽刺当时蜀地长官章仇兼琼想凭险割据,不听朝廷节制;丁,此诗纯粹歌咏山水风光,并无寓意。

这首诗最早见录于唐人殷璠所编的《河岳英灵集》,该书编成于唐玄宗天宝十二载(753年),由此可知李白这首诗的写作年代最迟也应该在《河岳英灵集》编成之前。而那时,安史之乱尚未发生,唐玄宗安居长安,房(琯)、杜甫也都还未入川,所以,甲、乙两说显然错误。至于讽刺章仇兼琼的说法,从一些史书的有关记载来看,也缺乏依据。章仇兼琼镇蜀时一直理想去长安做官。相对而言,还是最后一种说法比较客观,接近于作品实际。

这可能是一首赠友诗。有学者认为这首诗可能是天宝元年至三年(742至744年)李白在长安时为送友人王炎入蜀而写的,目的是规劝王炎不要羁留蜀地,早日回归长安,避免遭到嫉妒小人不测之手;也有学者认为此诗是开元年间李白初入长安无成而归时,送友人寄意之作。


\section{1.5   整体赏析}
\label{\detokenize{p01_u6563_u6587/_u674e_u767d-_u8700_u9053_u96be:id7}}
《蜀道难》是李白袭用乐府古题,展开丰富的想象,着力描绘了秦蜀道路上奇丽惊险的山川,并从中透露了对社会的某些忧虑与关切。

诗人大体按照由古及今,自秦入蜀的线索,抓住各处山水特点来描写,以展示蜀道之难。

从“噫吁嚱”到“然后天梯石栈相钩连”为一个段落。一开篇就极言蜀道之难,以感情强烈的咏叹点出主题,为全诗奠定了雄放的基调。以下随着感情的起伏和自然场景的变化,“蜀道之难,难于上青天”的咏叹反复出现,像一首乐曲的主旋律一样激荡着读者的心弦。

说蜀道的难行比上天还难,这是因为自古以来秦、蜀之间被高山峻岭阻挡,由秦入蜀,太白峰首当其冲,只有高飞的鸟儿能从低缺处飞过。太白峰在秦都咸阳西南,是关中一带的最高峰。民谚云:“武公太白,去天三百。”诗人以夸张的笔墨写出了历史上不可逾越的险阻,并融汇了五丁开山的神话,点染了神奇色彩,犹如一部乐章的前奏,具有引人入胜的妙用。下面即着力刻画蜀道的高危难行了。

从“上有六龙回日之高标”至“使人听此凋朱颜”为又一段落。这一段极写山势的高危,山高写得愈充分,愈可见路之难行。你看那突兀而立的高山,高标接天,挡住了太阳神的运行;山下则是冲波激浪、曲折回旋的河川。诗人不但把夸张和神话融为一体,直写山高,而且衬以“回川”之险。唯其水险,更见山势的高危。诗人意犹未足,又借黄鹤与猿猱来反衬。山高得连千里翱翔的黄鹤也不得飞度,轻疾敏捷的猿猴也愁于攀援,不言而喻,人行走就难上加难了。以上用虚写手法层层映衬,下面再具体描写青泥岭的难行。

青泥岭,“悬崖万仞,山多云雨”(《元和郡县志》),为唐代入蜀要道。诗人着重就其峰路的萦回和山势的峻危来表现人行其上的艰难情状和畏惧心理,捕捉了在岭上曲折盘桓、手扪星辰、呼吸紧张、抚胸长叹等细节动作加以摹写,寥寥数语,便把行人艰难的步履、惶悚的神情,绘声绘色地刻画出来,困危之状如在目前。

至此蜀道的难行似乎写到了极处。但诗人笔锋一转,借“问君”引出旅愁,以忧切低昂的旋律,把读者带进一个古木荒凉、鸟声悲凄的境界。杜鹃鸟空谷传响,充满哀愁,使人闻声失色,更觉蜀道之难。诗人借景抒情,用“悲鸟号古木”、“子规啼夜月”等感情色彩浓厚的自然景观,渲染了旅愁和蜀道上空寂苍凉的环境气氛,有力地烘托了蜀道之难。

然而,逶迤千里的蜀道,还有更为奇险的风光。自“连峰去天不盈尺”至全篇结束,主要从山川之险来揭示蜀道之难,着力渲染惊险的气氛。如果说“连峰去天不盈尺”是夸饰山峰之高,“枯松倒挂倚绝壁”则是衬托绝壁之险。

诗人先托出山势的高险,然后由静而动,写出水石激荡、山谷轰鸣的惊险场景。好像一串电影镜头:开始是山峦起伏、连峰接天的远景画面;接着平缓地推成枯松倒挂绝壁的特写;而后,跟踪而来的是一组快镜头,飞湍、瀑流、悬崖、转石,配合着万壑雷鸣的音响,飞快地从眼前闪过,惊险万状,目不暇接,从而造成一种势若排山倒海的强烈艺术效果,使蜀道之难的描写,简直达到了登峰造极的地步。如果说上面山势的高危已使人望而生畏,那此处山川的险要更令人惊心动魄了。

风光变幻,险象丛生。在十分惊险的气氛中,最后写到蜀中要塞剑阁,在大剑山和小剑山之间有一条三十里长的栈道,群峰如剑,连山耸立,削壁中断如门,形成天然要塞。因其地势险要,易守难攻,历史上在此割据称王者不乏其人。诗人从剑阁的险要引出对政治形势的描写。他化用西晋张载《剑阁铭》中“形胜之地,匪亲勿居”的语句,劝人引为鉴戒,警惕战乱的发生,并联系当时的社会背景,揭露了蜀中豺狼的“磨牙吮血,杀人如麻”,从而表达了对国事的忧虑与关切。唐天宝初年,太平景象的背后正潜伏着危机,后来发生的安史之乱,证明诗人的忧虑是有现实意义的。

李白以变化莫测的笔法,淋漓尽致地刻画了蜀道之难,艺术地展现了古老蜀道逶迤、峥嵘、高峻、崎岖的面貌,描绘出一幅色彩绚丽的山水画卷。诗中那些动人的景象宛如历历在目。

李白之所以描绘得如此动人,还在于融贯其间的浪漫主义激情。诗人寄情山水,放浪形骸。他对自然景物不是冷漠的观赏,而是热情地赞叹,借以抒发自己的理想感受。那飞流惊湍、奇峰险壑,赋予了诗人的情感气质,因而才呈现出飞动的灵魂和瑰伟的姿态。诗人善于把想象、夸张和神话传说融为一体进行写景抒情。言山之高峻,则曰“上有六龙回日之高标”;状道之险阻,则曰“地崩山摧壮士死,然后天梯石栈相钩连”。诗人“驰走风云,鞭挞海岳”(陆时雍《诗镜总论》评李白七古语),从蚕丛开国说到五丁开山,由六龙回日写到子规夜啼,天马行空般地驰骋想象,创造出博大浩渺的艺术境界,充满了浪漫主义色彩。透过奇丽峭拔的山川景物,仿佛可以看到诗人那“落笔摇五岳、笑傲凌沧洲”的高大形象。

唐以前的《蜀道难》作品,简短单薄。李白对东府古题有所创新和发展,用了大量散文化诗句,字数从三言、四言、五言、七言,直到十一言,参差错落,长短不齐,形成极为奔放的语言风格。诗的用韵,也突破了梁陈时代旧作一韵到底的程式。后面描写蜀中险要环境,一连三换韵脚,极尽变化之能事。所以殷璠编《河岳英灵集》称此诗“奇之又奇,自骚人以还,鲜有此体调”。

关于此篇,前人有种种寓意之说,断定是专为某人某事而作的。明人胡震亨、顾炎武认为,李白“自为蜀咏”,“别无寓意”。今人有谓此诗表面写蜀道艰险,实则写仕途坎坷,反映了诗人在长期漫游中屡逢踬碍的生活经历和怀才不遇的愤懑,迄无定论。


\section{1.6   历代评论}
\label{\detokenize{p01_u6563_u6587/_u674e_u767d-_u8700_u9053_u96be:id8}}
《本事诗》:李太白初自蜀至京师,舍于逆旅,贺监知孕闻其名,首访之。既奇其姿,复请所为文。出《蜀道难》以示之。读未竟,称叹者数四,号为“谪仙”,解金龟换酒,与倾尽醉,期不间日。由足称誉光赫。

《木天禁语》:七言长古篇法……旧题乃篇末一、二句缴上起句,又谓之“顾首”、如《蜀道难》、《古别离》、《洗兵马行》是也。

《唐诗品汇》:刘须溪云:妙在起伏,其才思放肆,语次崛奇,自不在言。

《四溟诗话》:九言体,无名氏拟之曰:“昨夜西风摇落千林梢,渡头小舟卷入寒塘坳。”声调散缓而无气魄。惟太白上篇突出两句,殊不可及,若“上有六龙回日之高标,下有冲波逆折之回川”是也。

《批选唐诗正声》:辞旨深远,雄浑飘逸,杜子美所不可到。欧阳子以《庐山高》方之,殊为哂。

《唐诗援》:太白创体,空前绝后。诸说纷纷不一,然细观此诗,定为明皇幸蜀而作。萧说是。

《批选唐诗》:太白长歌,森秀飞扬,疾于风雨,本其才性独诣,非由人力。人所不及在此,诗教大坏亦在此。后生学步,奋猛亢厉之音作,而温柔敦厚之意尽,露才扬己,长慠负气、辞人所以多轻薄,由来远已。嗟乎,西日东流,又岂人力哉!但可谓之唐体而已矣。

《唐音癸签》:《蜀道难》自是古曲,梁陈作者,止言其险,时不及其他。白则兼采张载《剑阁铭》“一人荷戟,万夬趑趄,形胜之地,匪亲弗居”等语用之,为恃险割据与羁留佐逆者著戒。惟其海说事理,故苞括大,而有合乐府讽世立教本旨。若第取一时一人事实之,反失之细而不足味矣。

《唐诗镜》:《蜀道难》近赋体,魁梧奇谲,知是伟大。

《唐诗选脉会通评林》:周珽曰:……“一夫当关”四句,设意外之忧;“朝避猛虎”四句,指阶见之恐,见变生肘腋,地终不可居。总言蜀道之难也。劈空落想,窍凿幽发,应使笔墨生而混沌死。

《诗源辨体》:屈原《离骚》本千古辞赋之宗,而后人摹仿盗袭,不胜厌饫……至《远别离》、《蜀道难》、《天姥吟》,则变幻恍惚,尽脱蹊径,实与屈子互相照映。

《唐风定》:变幻神奇,仙而不鬼,长吉魔语视之何如?亘古代无能仿象,才涉意即入长吉魔中矣。通篇奇险,不涉旁意,不参平调,其胜《天姥》、《鸣皋》以此。

《王文简古诗平仄论》:(七言古)又有长短句者,唐惟李太白多有之,然不必学。如《蜀道难》……效之而无其才,洵难免沧溟“英雄欺人”之诮。

《增订唐诗摘钞》:倏起倏落,忽虚忽实。真如烟水杳渺,绝世奇文也。

《载酒园诗话又编》:《蜀道难》一篇,真与河岳并垂不朽。即起句“噫吁戏,危沪高哉”七字,如累棋架卵,谁敢并于一处?至其造句之妙:“连峰去天不盈尺,枯松倒扯倚绝壁。飞湍瀑流争喧豗,砅虚转石万壑雷。”每读之。剑阁、阴平、如在目前。又如“一夫当关,万夫莫开。所守或匪亲,化为狼与豺”、不惟刘璋、李势恨事如见,即孟知祥一辈亦逆揭其肺肝。此真诗之有关系者,岂特文词之雄!

《唐音审体》:篇中三言蜀道之难,所谓一唱三叹也。突然以嗟叹起,嗟叹结,创格也。

《放胆诗》:太白《蜀道难》、《远别离》等篇出鬼入冲,惝恍莫测。

《此木轩论诗汇编》:《蜀道难》、旧题也,太白为之,加奇肆耳。此千古绝调也,后人妄意学步,何其不知量也!“噫吁嚱,危乎高哉”,七字五句。“连峰去天不盈尺”无理之极,俗本作“连峰入烟几千尺”,有理之极。无理之妙,妙不可言。有理之不妙,其不妙亦不可胜言。举此一隅,即是学诗家万金良药也。

《而庵说唐诗》:“尔来四万八千岁”,此云总非实据也。人言文人无实语,而不知文章家妙在跌宕;每说到已甚,太白用此,正跌宕法也。“蜀道之难,难于上青天”再一提,此句妙有关锁,上来笔气纵横,逸宕不如此,则散无统束矣。“锦城虽云乐”:上面说到蜀如此可惊、可畏,而忽下一“乐”字,妙极。“不如早还家”:此虽是乐,不可久居,“不如早还家”之句尤乐也。文势至此甚紧,必须一放,方得宽转,所谓“一张一弛,文武之道”也。“蜀道之难,难于上青天”,复提此句为结束,妙。篇中凡三见,与《庄子·逍遥游》叙鲲鹏同。吾尝谓作长篇古诗,须读《庄子》、《史记》。子美歌行纯学《史记》,太白歌行纯学《庄子》。故两先生为歌行之双绝,不诬也。

《唐诗别裁》:笔阵纵撗,如虬飞蠖动,起雷霆于指顾之间。任华,卢仝辈仿之,适得其怪耳,太白所以为仙才也。

《剑溪说诗》:太白诗“蜀道之难,难于上青天”句,凡三叠。管子曰:“使海于有蔽,渠弥于有渚,纲山于有牢。”谷梁氏曰:“梁山崩,壅遏河三日不流。”一篇之中,三番叙述,愈见其妙。所谓“闭户造车,出门合辙”者也。

《网师园唐诗笺》:造语奇险(“地崩山摧”二句下)。玩此,为明皇幸蜀作无疑(“问君西游”句下)。兜来何等力量。(“其险”句下)!高文险语,动魄惊心(“磨牙”二句下)。主意在此(“不如”句下)。

《李太白诗醇》:严云:提“蜀道难”,篇中三致意;用“噫吁戏”三字起,非无谓。后人学袭,便成恶道。{[}8{]}


\section{1.7   作者简介}
\label{\detokenize{p01_u6563_u6587/_u674e_u767d-_u8700_u9053_u96be:id9}}
李白(701—762),字太白,号青莲居士。是屈原之后最具个性特色、最伟大的浪漫主义诗人。有“诗仙”之美誉,与杜甫并称“李杜”。其诗以抒情为主,表现出蔑视权贵的傲岸精神,对人民疾苦表示同情,又善于描绘自然景色,表达对祖国山河的热爱。诗风雄奇豪放,想象丰富,语言流转自然,音律和谐多变,善于从民间文艺和神话传说中吸取营养和素材,构成其特有的瑰玮绚烂的色彩,达到盛唐诗歌艺术的巅峰。存世诗文千余篇,有《李太白集》三十卷。


\chapter{1   毛泽东-七律·长征}
\label{\detokenize{p01_u6563_u6587/_u6bdb_u6cfd_u4e1c-_u4e03_u5f8b_xb7_u957f_u5f81:id1}}\label{\detokenize{p01_u6563_u6587/_u6bdb_u6cfd_u4e1c-_u4e03_u5f8b_xb7_u957f_u5f81::doc}}
\begin{sphinxShadowBox}
\sphinxstyletopictitle{目录}
\begin{itemize}
\item {} 
\phantomsection\label{\detokenize{p01_u6563_u6587/_u6bdb_u6cfd_u4e1c-_u4e03_u5f8b_xb7_u957f_u5f81:id7}}{\hyperref[\detokenize{p01_u6563_u6587/_u6bdb_u6cfd_u4e1c-_u4e03_u5f8b_xb7_u957f_u5f81:id1}]{\sphinxcrossref{1   毛泽东-七律·长征}}}
\begin{itemize}
\item {} 
\phantomsection\label{\detokenize{p01_u6563_u6587/_u6bdb_u6cfd_u4e1c-_u4e03_u5f8b_xb7_u957f_u5f81:id8}}{\hyperref[\detokenize{p01_u6563_u6587/_u6bdb_u6cfd_u4e1c-_u4e03_u5f8b_xb7_u957f_u5f81:id3}]{\sphinxcrossref{1.1   作品原文}}}

\item {} 
\phantomsection\label{\detokenize{p01_u6563_u6587/_u6bdb_u6cfd_u4e1c-_u4e03_u5f8b_xb7_u957f_u5f81:id9}}{\hyperref[\detokenize{p01_u6563_u6587/_u6bdb_u6cfd_u4e1c-_u4e03_u5f8b_xb7_u957f_u5f81:id4}]{\sphinxcrossref{1.2   词句注释}}}

\item {} 
\phantomsection\label{\detokenize{p01_u6563_u6587/_u6bdb_u6cfd_u4e1c-_u4e03_u5f8b_xb7_u957f_u5f81:id10}}{\hyperref[\detokenize{p01_u6563_u6587/_u6bdb_u6cfd_u4e1c-_u4e03_u5f8b_xb7_u957f_u5f81:id5}]{\sphinxcrossref{1.3   白话译文}}}

\item {} 
\phantomsection\label{\detokenize{p01_u6563_u6587/_u6bdb_u6cfd_u4e1c-_u4e03_u5f8b_xb7_u957f_u5f81:id11}}{\hyperref[\detokenize{p01_u6563_u6587/_u6bdb_u6cfd_u4e1c-_u4e03_u5f8b_xb7_u957f_u5f81:id6}]{\sphinxcrossref{1.4   创作背景}}}

\end{itemize}

\end{itemize}
\end{sphinxShadowBox}

《七律·长征》是一首七言律诗,选自《毛泽东诗词集》,这首诗写于1935年10月,当时毛泽东率领中央红军越过岷山,长征即将结束。回顾长征一年来所战胜的无数艰难险阻,他满怀喜悦的战斗豪情。


\section{1.1   作品原文}
\label{\detokenize{p01_u6563_u6587/_u6bdb_u6cfd_u4e1c-_u4e03_u5f8b_xb7_u957f_u5f81:id3}}
七律·长征

七律⑴·长征⑵

红军不怕远征难⑶,万水千山只等闲⑷。

五岭⑸逶迤⑹腾细浪⑺,乌蒙⑻磅礴走泥丸⑼。

金沙⑽水拍云崖暖⑾,大渡桥⑿横铁索⒀寒⒁。

更喜岷山⒂千里雪,三军⒃过后尽开颜⒄。{[}1{]}


\section{1.2   词句注释}
\label{\detokenize{p01_u6563_u6587/_u6bdb_u6cfd_u4e1c-_u4e03_u5f8b_xb7_u957f_u5f81:id4}}
⑴七律:七律是律诗的一种,每篇一般为八句,每句七个字,分四联:首联、颔联、颈联和尾联;偶句末一字押平声韵,首句末字可押可不押,必须一韵到底;句内和句间要讲平仄,中间四句按常规要用对仗。

⑵长征:1934年10月间,中央红军主力从中央革命根据地出发作战略大转移,经过福建、江西、广东、湖南、广西、贵州、四川、云南、西藏、甘肃、陕西等十一省,击溃了敌人多次的围追和堵截,战胜了军事上、政治上和自然界的无数艰险,行军二万五千里,终于在1935年10月到达陕北革命根据地。

⑶难:艰难险阻。

⑷等闲:不怕困难,不可阻止。

⑸五岭:大庾岭,骑田岭,都庞岭,萌渚岭,越城岭,横亘在江西、湖南、两广之间。

⑹逶迤:形容道路、山脉、河流等弯弯曲曲,连绵不断的样子。

⑺细浪:作者自释:“把山比作‘细浪’、‘泥丸’,是‘等闲’之意。”

⑻乌蒙:山名。乌蒙山,在贵州西部与云南东北部的交界处,北临金沙江,山势陡峭。1935年4月,红军长征经过此地。

⑼泥丸:小泥球,整句意思说险峻的乌蒙山在红军战士的脚下,就像是一个小泥球一样。

⑽金沙:金沙江,指长江上游自青海省玉树县至四川省宜宾市的一段,云南等地也有支流。1935年5月,红军曾强渡云南省禄劝县皎平渡渡口。

⑾云崖暖:是指浪花拍打悬崖峭壁,溅起阵阵雾水,在红军的眼中像是冒出的蒸汽一样。(云崖:高耸入云的山崖。暖:被一些学者指为红军巧渡金沙江后的欢快心情,也有学者说意思为直译后的温暖。)

⑿大渡桥:指四川省西部泸定县大渡河上的泸定桥。

⒀铁索:大渡河上泸定桥,它是用十三根铁索组成的桥。

⒁寒:影射敌人的冷酷与形势的严峻。

⒂岷(mín)山:中国西部大山。位于甘肃省西南、四川省北部。西北-东南走向。西北接西倾山,南与邛崃山相连。包括甘肃南部的迭山,甘肃、四川边境的摩天岭。

⒃三军:作者自注:“红军一方面军,二方面军,四方面军。”

⒄尽开颜:红军的长征到达目的地了,他们取得了胜利,所以个个都笑逐颜开。


\section{1.3   白话译文}
\label{\detokenize{p01_u6563_u6587/_u6bdb_u6cfd_u4e1c-_u4e03_u5f8b_xb7_u957f_u5f81:id5}}
红军不怕万里长征路上的一切艰难困苦,把千山万水都看得极为平常。绵延不断的五岭,在红军看来只不过是微波细浪在起伏,而气势雄伟的乌蒙山,在红军眼里也不过是一颗泥丸。

金沙江浊浪滔天,拍击着高耸入云的峭壁悬崖,热气腾腾。大渡河险桥横架,晃动着凌空高悬的根根铁索,寒意阵阵。

更加令人喜悦的是踏上千里积雪的岷山,红军翻越过去以后个个笑逐颜开。


\section{1.4   创作背景}
\label{\detokenize{p01_u6563_u6587/_u6bdb_u6cfd_u4e1c-_u4e03_u5f8b_xb7_u957f_u5f81:id6}}
1934年10月,中国工农红军为粉碎国民政府的围剿,保存自己的实力,也为了北上抗日,挽救民族危亡,从江西瑞金出发,开始了举世闻名的长征。

这首七律是作于红军战士越过岷山后,长征即将胜利结束前不久的途中。作为红军的领导人,毛泽东在经受了无数次考验后,如今,曙光在前,胜利在望,他心潮澎湃,满怀豪情地写下了这首壮丽的诗篇。《七律·长征》写于1935年9月下旬,10月定稿。


\chapter{1   毛泽东-沁园春·雪}
\label{\detokenize{p01_u6563_u6587/_u6bdb_u6cfd_u4e1c-_u6c81_u56ed_u6625_xb7_u96ea:id1}}\label{\detokenize{p01_u6563_u6587/_u6bdb_u6cfd_u4e1c-_u6c81_u56ed_u6625_xb7_u96ea::doc}}
\begin{sphinxShadowBox}
\sphinxstyletopictitle{目录}
\begin{itemize}
\item {} 
\phantomsection\label{\detokenize{p01_u6563_u6587/_u6bdb_u6cfd_u4e1c-_u6c81_u56ed_u6625_xb7_u96ea:id8}}{\hyperref[\detokenize{p01_u6563_u6587/_u6bdb_u6cfd_u4e1c-_u6c81_u56ed_u6625_xb7_u96ea:id1}]{\sphinxcrossref{1   毛泽东-沁园春·雪}}}
\begin{itemize}
\item {} 
\phantomsection\label{\detokenize{p01_u6563_u6587/_u6bdb_u6cfd_u4e1c-_u6c81_u56ed_u6625_xb7_u96ea:id9}}{\hyperref[\detokenize{p01_u6563_u6587/_u6bdb_u6cfd_u4e1c-_u6c81_u56ed_u6625_xb7_u96ea:id3}]{\sphinxcrossref{1.1   作品原文}}}

\item {} 
\phantomsection\label{\detokenize{p01_u6563_u6587/_u6bdb_u6cfd_u4e1c-_u6c81_u56ed_u6625_xb7_u96ea:id10}}{\hyperref[\detokenize{p01_u6563_u6587/_u6bdb_u6cfd_u4e1c-_u6c81_u56ed_u6625_xb7_u96ea:id4}]{\sphinxcrossref{1.2   词句注释}}}

\item {} 
\phantomsection\label{\detokenize{p01_u6563_u6587/_u6bdb_u6cfd_u4e1c-_u6c81_u56ed_u6625_xb7_u96ea:id11}}{\hyperref[\detokenize{p01_u6563_u6587/_u6bdb_u6cfd_u4e1c-_u6c81_u56ed_u6625_xb7_u96ea:id5}]{\sphinxcrossref{1.3   白话译文}}}

\item {} 
\phantomsection\label{\detokenize{p01_u6563_u6587/_u6bdb_u6cfd_u4e1c-_u6c81_u56ed_u6625_xb7_u96ea:id12}}{\hyperref[\detokenize{p01_u6563_u6587/_u6bdb_u6cfd_u4e1c-_u6c81_u56ed_u6625_xb7_u96ea:id6}]{\sphinxcrossref{1.4   创作背景}}}

\item {} 
\phantomsection\label{\detokenize{p01_u6563_u6587/_u6bdb_u6cfd_u4e1c-_u6c81_u56ed_u6625_xb7_u96ea:id13}}{\hyperref[\detokenize{p01_u6563_u6587/_u6bdb_u6cfd_u4e1c-_u6c81_u56ed_u6625_xb7_u96ea:id7}]{\sphinxcrossref{1.5   名家点评}}}

\end{itemize}

\end{itemize}
\end{sphinxShadowBox}

《沁园春·雪》是无产阶级革命家毛泽东创作的一首词。该词上片描写北国壮丽的雪景,纵横千万里,展示了大气磅礴、旷达豪迈的意境,抒发了词人对祖国壮丽河山的热爱。下片议论抒情,重点评论历史人物,歌颂当代英雄,抒发无产阶级要做世界的真正主人的豪情壮志。全词熔写景、议论和抒情于一炉,意境壮美,气势恢宏,感情奔放,胸襟豪迈,颇能代表毛泽东诗词的豪放风格。


\section{1.1   作品原文}
\label{\detokenize{p01_u6563_u6587/_u6bdb_u6cfd_u4e1c-_u6c81_u56ed_u6625_xb7_u96ea:id3}}
沁园春·雪1

北国风光,千里冰封,万里雪飘。望长城内外,惟余莽莽2;大河上下,顿失滔滔3。山舞银蛇,原驰蜡象4,欲与天公试比高。须晴日5,看红装素裹,分外妖娆6。

江山如此多娇,引无数英雄竞折腰7。惜秦皇汉武,略输文采8;唐宗宋祖9,稍逊风骚。一代天骄,成吉思汗,只识弯弓射大雕。俱往矣,数风流人物,还看今朝。


\section{1.2   词句注释}
\label{\detokenize{p01_u6563_u6587/_u6bdb_u6cfd_u4e1c-_u6c81_u56ed_u6625_xb7_u96ea:id4}}
1.沁园春:词牌名,又名“东仙”“寿星明”“洞庭春色”等。双调,一百十四字。前段十三句,四平韵;后段十二句,五平韵。

2.惟余:只剩下。余:有版本作“馀”。莽莽:即茫茫,白茫茫一片。形容空旷无际。

3.顿失:立刻失去。顿:顿时,立刻。滔滔:滚滚的波涛。

4.原驰蜡象:作者原注“原指高原,即秦晋高原”。驰:有版本作“驱”。蜡象:白色的象。

5.须:待、等到。

6.“看红装”二句:红日和白雪互相映照,看去好像装饰艳丽的美女裹着白色的外衣,格外娇媚。红装:身着艳丽服饰的美女。一作银装。妖娆(ráo):娇艳妩媚。

7.竞折腰:争着为江山奔走效劳。折腰:倾倒,躬着腰侍候。

8.“秦皇汉武”二句:是说秦皇汉武,功业甚盛,相比之下,文治方面的成就略有逊色。秦皇:秦始皇赢政,秦朝的创业皇帝。汉武:汉武帝刘彻,西汉第七位皇帝。略输:稍差。文采:本指辞藻、才华。这里引申为文治。

9.唐宗:唐太宗李世民,唐朝建立统一大业的皇帝。宋祖:宋太祖赵匡胤,宋朝的创业皇帝。

10.稍逊风骚:意近“略输文采”。逊:差。风骚:本指《诗经》里的《国风》和《楚辞》里的《离骚》,后来泛指文章辞藻。

11.天骄:汉时匈奴自称为“天之骄子”,以后泛称强盛的边地民族。

12.成吉思汗:元太祖铁木真统一蒙古后的尊称,意思是“强者之汗”。

13.“只识”句:是说只以武功见长。识:知道,懂得。雕:一种鹰类的大型猛禽,善飞难射,古代因用“射雕手”比喻高强的射手。{[}2{]}


\section{1.3   白话译文}
\label{\detokenize{p01_u6563_u6587/_u6bdb_u6cfd_u4e1c-_u6c81_u56ed_u6625_xb7_u96ea:id5}}
北方的风光,千里冰封冻,万里雪花飘。望长城内外,只剩下无边无际白茫茫一片;宽广的黄河上下,顿时失去了滔滔水势。山岭好像银白色的蟒蛇在飞舞,高原上的丘陵好像许多白象在奔跑,它们都想与老天爷比比高。要等到晴天的时候,看红艳艳的阳光和白皑皑的冰雪交相辉映,分外美好。

江山如此媚娇,引得无数英雄竞相倾倒。只可惜秦始皇、汉武帝,略差文学才华;唐太宗、宋太祖,稍逊文治功劳。称雄一世的人物成吉思汗,只知道拉弓射大雕。这些人物全都过去了,称得上能建功立业的英雄人物,还要看今天的人们。{[}3{]}


\section{1.4   创作背景}
\label{\detokenize{p01_u6563_u6587/_u6bdb_u6cfd_u4e1c-_u6c81_u56ed_u6625_xb7_u96ea:id6}}
1936年,红军组织东征部队,准备东渡黄河对日军作战。红军从子长县出发,挺进到清涧县高杰村的袁家沟一带时,部队在这里休整了16天。2月5日至20日,毛泽东在这里居住期间,曾下过一场大雪,长城内外白雪皑皑,隆起的秦晋高原,冰封雪盖。天气严寒,连平日奔腾咆哮的黄河都结了一层厚厚的冰,失去了往日的波涛。毛泽东当时住在农民白治民家中,深夜。见此情景,颇有感触,填写了这首词。《沁园春·雪》最早发表于1945年11月14日重庆《新民报晚刊》,后正式发表于《诗刊》1957年1月号。{[}4-5{]}


\section{1.5   名家点评}
\label{\detokenize{p01_u6563_u6587/_u6bdb_u6cfd_u4e1c-_u6c81_u56ed_u6625_xb7_u96ea:id7}}
近代诗人柳亚子《沁园春·雪》跋:毛润之沁园春一阕,余推为千古绝唱,虽东坡、幼安,犹瞠乎其后,更无论南唐小令、南宋慢词矣。


\chapter{1   老舍-济南的冬天}
\label{\detokenize{p01_u6563_u6587/_u8001_u820d-_u6d4e_u5357_u7684_u51ac_u5929:id1}}\label{\detokenize{p01_u6563_u6587/_u8001_u820d-_u6d4e_u5357_u7684_u51ac_u5929::doc}}
\begin{sphinxShadowBox}
\sphinxstyletopictitle{目录}
\begin{itemize}
\item {} 
\phantomsection\label{\detokenize{p01_u6563_u6587/_u8001_u820d-_u6d4e_u5357_u7684_u51ac_u5929:id11}}{\hyperref[\detokenize{p01_u6563_u6587/_u8001_u820d-_u6d4e_u5357_u7684_u51ac_u5929:id1}]{\sphinxcrossref{1   老舍-济南的冬天}}}
\begin{itemize}
\item {} 
\phantomsection\label{\detokenize{p01_u6563_u6587/_u8001_u820d-_u6d4e_u5357_u7684_u51ac_u5929:id12}}{\hyperref[\detokenize{p01_u6563_u6587/_u8001_u820d-_u6d4e_u5357_u7684_u51ac_u5929:id3}]{\sphinxcrossref{1.1   作品原文}}}

\item {} 
\phantomsection\label{\detokenize{p01_u6563_u6587/_u8001_u820d-_u6d4e_u5357_u7684_u51ac_u5929:id13}}{\hyperref[\detokenize{p01_u6563_u6587/_u8001_u820d-_u6d4e_u5357_u7684_u51ac_u5929:id4}]{\sphinxcrossref{1.2   写景手法}}}
\begin{itemize}
\item {} 
\phantomsection\label{\detokenize{p01_u6563_u6587/_u8001_u820d-_u6d4e_u5357_u7684_u51ac_u5929:id14}}{\hyperref[\detokenize{p01_u6563_u6587/_u8001_u820d-_u6d4e_u5357_u7684_u51ac_u5929:id5}]{\sphinxcrossref{1.2.1   1.基调统一,色彩和谐}}}

\item {} 
\phantomsection\label{\detokenize{p01_u6563_u6587/_u8001_u820d-_u6d4e_u5357_u7684_u51ac_u5929:id15}}{\hyperref[\detokenize{p01_u6563_u6587/_u8001_u820d-_u6d4e_u5357_u7684_u51ac_u5929:id6}]{\sphinxcrossref{1.2.2   2.景物层次,安排得当}}}

\item {} 
\phantomsection\label{\detokenize{p01_u6563_u6587/_u8001_u820d-_u6d4e_u5357_u7684_u51ac_u5929:id16}}{\hyperref[\detokenize{p01_u6563_u6587/_u8001_u820d-_u6d4e_u5357_u7684_u51ac_u5929:id7}]{\sphinxcrossref{1.2.3   3.远近大细,各得其宜}}}

\item {} 
\phantomsection\label{\detokenize{p01_u6563_u6587/_u8001_u820d-_u6d4e_u5357_u7684_u51ac_u5929:id17}}{\hyperref[\detokenize{p01_u6563_u6587/_u8001_u820d-_u6d4e_u5357_u7684_u51ac_u5929:id8}]{\sphinxcrossref{1.2.4   4.虚实手法,同时并用}}}

\item {} 
\phantomsection\label{\detokenize{p01_u6563_u6587/_u8001_u820d-_u6d4e_u5357_u7684_u51ac_u5929:id18}}{\hyperref[\detokenize{p01_u6563_u6587/_u8001_u820d-_u6d4e_u5357_u7684_u51ac_u5929:id9}]{\sphinxcrossref{1.2.5   5.适当点题,意义深远}}}

\item {} 
\phantomsection\label{\detokenize{p01_u6563_u6587/_u8001_u820d-_u6d4e_u5357_u7684_u51ac_u5929:id19}}{\hyperref[\detokenize{p01_u6563_u6587/_u8001_u820d-_u6d4e_u5357_u7684_u51ac_u5929:id10}]{\sphinxcrossref{1.2.6   6.山水画法,以大观小}}}

\end{itemize}

\end{itemize}

\end{itemize}
\end{sphinxShadowBox}


\section{1.1   作品原文}
\label{\detokenize{p01_u6563_u6587/_u8001_u820d-_u6d4e_u5357_u7684_u51ac_u5929:id3}}
对于一个在北平住惯的人,像我,冬天要是不刮风,便觉得是奇迹;济南的冬天是没有风声的。对于一个刚由伦敦回来的人,像我,冬天要能看得见日光,便觉得是怪事;济南的冬天是响晴的。自然,在热带的地方,日光是永远那么毒,响亮的天气,反有点叫人害怕。可是,在北中国的冬天,而能有温晴的天气,济南真得算个宝地。

设若单单是有阳光,那也算不了出奇。请闭上眼睛想:一个老城,有山有水,全在天底下晒着阳光,暖和安适地睡着,只等春风来把它们唤醒,这是不是个理想的境界?

小山整把济南围了个圈儿,只有北边缺着点口儿。这一圈小山在冬天特别可爱,好像是把济南放在一个小摇篮里,它们安静不动地低声地说:“你们放心吧,这儿准保暖和。”真的,济南的人们在冬天是面上含笑的。他们一看那些小山,心中便觉得有了着落,有了依靠。他们由天上看到山上,便不知不觉地想起:“明天也许就是春天了吧?这样的温暖,今天夜里山草也许就绿起来了吧?”就是这点幻想不能一时实现,他们也并不着急,因为有这样慈善的冬天,干啥还希望别的呢!

最妙的是下点小雪呀。看吧,山上的矮松越发的青黑,树尖上顶着一髻儿白花,好像日本看护妇。山尖全白了,给蓝天镶上一道银边。山坡上,有的地方雪厚点,有的地方草色还露着,这样,一道儿白,一道儿暗黄,给山们穿上一件带水纹的花衣;看着看着,这件花衣好像被风儿吹动,叫你希望看见一点更美的山的肌肤。等到快日落的时候,微黄的阳光斜射在山腰上,那点薄雪好像忽然害了羞,微微露出点粉色。就是下小雪吧,济南是受不住大雪的,那些小山太秀气!

古老的济南,城里那么狭窄,城外又那么宽敞,山坡上卧着些小村庄,小村庄的房顶上卧着点雪,对,这是张小水墨画,也许是唐代的名手画的吧。

那水呢,不但不结冰,倒反在绿萍上冒着点热气,水藻真绿,把终年贮蓄的绿色全拿出来了。天儿越晴,水藻越绿,就凭这些绿的精神,水也不忍得冻上,况且那些长枝的垂柳还要在水里照个影儿呢!看吧,由澄清的河水慢慢往上看吧,空中,半空中,天上,自上而下全是那么清亮,那么蓝汪汪的,整个的是块空灵的蓝水晶。这块水晶里,包着红屋顶,黄草山,像地毯上的小团花的灰色树影。这就是冬天的济南。{[}1{]}


\section{1.2   写景手法}
\label{\detokenize{p01_u6563_u6587/_u8001_u820d-_u6d4e_u5357_u7684_u51ac_u5929:id4}}

\subsection{1.2.1   1.基调统一,色彩和谐}
\label{\detokenize{p01_u6563_u6587/_u8001_u820d-_u6d4e_u5357_u7684_u51ac_u5929:id5}}
济南虽然地处北中国,但是冬天无大风而多日照,它在冬天最显著的气候特点是“温晴”(温暖晴朗)。文章紧紧抓住这一点,使笔下的种种景物跟这“温晴”天气紧密联系在一起,构成一幅温暖晴朗的济南冬天图景。文章写山,写水,写城,写人,都无不涂上一层温暖晴朗的色彩,就是写雪景,也仍然跟温暖有联系──因为暖和,所以“最妙的是下点小雪”;而同晴朗分不开──因为晴朗,所以有“等到快日落的时候,微黄的阳光斜射在山腰上,那点薄雪好像忽然害了羞,微微露出点粉色”的景致。

在文中,第二段主要写的是济南全景,第三、四段主要写的是济南的山色,第五段主要写的是济南的水上景色,那么,全文就是由这几幅互相联系而又相对独立的画图组成的长轴。而这幅长轴,也就靠这“温晴”的基调统一起来,给人以和谐一致的美感。


\subsection{1.2.2   2.景物层次,安排得当}
\label{\detokenize{p01_u6563_u6587/_u8001_u820d-_u6d4e_u5357_u7684_u51ac_u5929:id6}}
古老的济南,景色秀丽,素有“家家泉水,户户插柳”、“一城山色半城湖”的美誉。文章依照写景的先后层次,更好地把这些美好的景色展现于出来。文章首先鸟瞰全城,得其全貌(第二段),然后给人以那一城山色,雪后斜阳(第三、四段),最后才写那垂柳岸边,那“水不但不结冰,倒反在绿萍上冒着点热气”,而水藻越晴越绿的水上景色(第五段)。由大到小地写来,从山到水地写去,层次分明,脉络清晰。自然这是就各大层次来说的,各大层次的内部,又同中有异,如第二段的由写景而兼及写人,第三段的由写雪而兼及写晴,第五段的由写水面而兼及写天空。写来笔法活脱,不失参差错落之致。


\subsection{1.2.3   3.远近大细,各得其宜}
\label{\detokenize{p01_u6563_u6587/_u8001_u820d-_u6d4e_u5357_u7684_u51ac_u5929:id7}}
偌大的一个济南,在作者笔下,竟然可以放在一个由四面群山环抱而成的小小摇篮里,而水天一碧的宏伟景色,只不过是一块“空灵的蓝水晶”。这是景物的远者大者。再看,“树尖上顶着一髻儿白花,好像日本看护妇”,“水藻真绿,把终年贮蓄的绿色全拿出来了”。这是景物的近者细者。远景大景,使人视野开阔,顿感心旷神怡;近景小景,叫人近看谛听,更觉景象真切。而且远景大景,还可以冲破“不识庐山真面目,只缘身在此山中”的局限,而近景小景,又能够避免“只见树木不见森林”的弊病。古诗云:“远观山有色,近听水无声。”这是说的非远观不能看到高山居然有色,非近听无以觉出流水竟然无声。这说明,写景手法,远近大细,不可偏废。运用得宜,就可以兼收其效。

该文写景时,不但远近并用,大细兼行,而且往往是由近而远、由细而大,或由远而近、由大而细,写来衔接紧密,推进自然。比如第五段的写景,就是由近而远,由细而大的:先写水冒着点热气,再写水藻,再写垂柳,再写水面的上空以至于半空中、天空上。而第四段的写景,则是由远而近、由大而细的:先写城外,再写城外的山坡,再写山坡上的小村庄,再写小村庄的房顶上的雪。这种写法,既符合叙述的逻辑顺序,又适应读者的视觉需要。


\subsection{1.2.4   4.虚实手法,同时并用}
\label{\detokenize{p01_u6563_u6587/_u8001_u820d-_u6d4e_u5357_u7684_u51ac_u5929:id8}}
实写景物的形象,对景物描写来说,无疑是十分必要的,诸如文章中的“树尖上顶着一髻儿白花,好像日本看护妇”之类。但是,要不止于摹状,还要传神,就得更多地仰仗虚写的手法。因此,在作者笔下,冬天阳光照耀下的济南,就出现了“暖和安适地睡着,只等春风来把它们唤醒”的神情;一圈围城的小山,也就说出“你们放心吧,这儿准保暖和”的细语;薄雪会有“微微露出点粉色”的羞容;水藻会有“把终年贮蓄的绿色全拿出来了”的“精神”;而那水呢,对那水藻也就可以有一副“不忍得冻上”的和善心肠了。至于小雪覆盖不匀的山坡,要“给山们穿上一件带水纹的花衣”,“那些长枝的垂柳还要在水里照个影儿”,自然也是文章中虚写传神的佳句。


\subsection{1.2.5   5.适当点题,意义深远}
\label{\detokenize{p01_u6563_u6587/_u8001_u820d-_u6d4e_u5357_u7684_u51ac_u5929:id9}}
画之所以有题跋,原因之一是题跋可以使画本身蕴含的意义更为显豁。应该说,题跋是一幅画的一个有机的组成部分,虽然它并不是所画的景物的本身。同样,对所写的景物,作者出面直接点题,也是容许的,这些点明题旨的话,不是可有可无的。该文点题得法,寥寥数语,便收到画龙点睛的效果。比如说,文章在描写了小山雪景之后,突然掉转笔锋,作者以评论者的身份,说起点题话来:“就是下小雪吧,济南是受不住大雪的,那些小山太秀气!”这话,既可以说是在所描绘的画面之外,又可以说是在所描绘的画面之中,因为它是画面所本有而又有点不甚明了的。一经点出,济南下点小雪(不能是大雪)的妙处,也就跃然纸上了。

题不可不点,也不可滥点,本文点题恰到好处。最后一句“这就是冬天的济南”,令人读起来有意犹未尽、话犹未了之感,引发读者更深远的思考,这也许正是作者使文章戛然而止的原因吧。


\subsection{1.2.6   6.山水画法,以大观小}
\label{\detokenize{p01_u6563_u6587/_u8001_u820d-_u6d4e_u5357_u7684_u51ac_u5929:id10}}
描绘济南的大地,老舍先生所用的是“以大观小”的中国山水画的构图取景方法。作者展开想像的翅膀飞上济南的云天俯瞰大地,然后对济南大地作了简笔的写意描绘。画城,不画它的东西南北,“一个老城,有山有水,全在天底下晒着阳光,暖和安适地睡着,只等春风来把它们唤醒”(注:此句中的山是济南城中的山)。一些琐碎的细部都被略去了,画的只是冬天济南城秀美的睡态,留下充分的余地让读者去联想、想像,进行艺术的再创造。画山,不画它的上下左右,“小山整把济南围了个圈儿,只有北边缺着点口儿”。一起笔就抓住了景物的主要特征,紧接着就引导读者展开艺术的联想和想像:“这一圈小山在冬天特别可爱,好像是把济南放在一个小摇篮里,它们安静不动地低声地说:‘你们放心吧,这儿准保暖和。’”借这种联想、想像,使画面活灵飞动起来。画人,不画人的男女老少,不但如国画一样略去耳鼻眉目,连形体也完全略去,而只画了济南冬天人物情态的最主要的特征:“济南的人们在冬天是面上含笑的。”和城与山,浑然构成一幅完美的图画。


\chapter{1   艾青-大堰河——我的保姆}
\label{\detokenize{p01_u6563_u6587/_u827e_u9752-_u5927_u5830_u6cb3_u2014_u2014_u6211_u7684_u4fdd_u59c6:id1}}\label{\detokenize{p01_u6563_u6587/_u827e_u9752-_u5927_u5830_u6cb3_u2014_u2014_u6211_u7684_u4fdd_u59c6::doc}}
\begin{sphinxShadowBox}
\sphinxstyletopictitle{目录}
\begin{itemize}
\item {} 
\phantomsection\label{\detokenize{p01_u6563_u6587/_u827e_u9752-_u5927_u5830_u6cb3_u2014_u2014_u6211_u7684_u4fdd_u59c6:id7}}{\hyperref[\detokenize{p01_u6563_u6587/_u827e_u9752-_u5927_u5830_u6cb3_u2014_u2014_u6211_u7684_u4fdd_u59c6:id1}]{\sphinxcrossref{1   艾青-大堰河——我的保姆}}}
\begin{itemize}
\item {} 
\phantomsection\label{\detokenize{p01_u6563_u6587/_u827e_u9752-_u5927_u5830_u6cb3_u2014_u2014_u6211_u7684_u4fdd_u59c6:id8}}{\hyperref[\detokenize{p01_u6563_u6587/_u827e_u9752-_u5927_u5830_u6cb3_u2014_u2014_u6211_u7684_u4fdd_u59c6:id3}]{\sphinxcrossref{1.1   作品原文}}}

\item {} 
\phantomsection\label{\detokenize{p01_u6563_u6587/_u827e_u9752-_u5927_u5830_u6cb3_u2014_u2014_u6211_u7684_u4fdd_u59c6:id9}}{\hyperref[\detokenize{p01_u6563_u6587/_u827e_u9752-_u5927_u5830_u6cb3_u2014_u2014_u6211_u7684_u4fdd_u59c6:id4}]{\sphinxcrossref{1.2   创作背景}}}

\item {} 
\phantomsection\label{\detokenize{p01_u6563_u6587/_u827e_u9752-_u5927_u5830_u6cb3_u2014_u2014_u6211_u7684_u4fdd_u59c6:id10}}{\hyperref[\detokenize{p01_u6563_u6587/_u827e_u9752-_u5927_u5830_u6cb3_u2014_u2014_u6211_u7684_u4fdd_u59c6:id5}]{\sphinxcrossref{1.3   作品鉴赏}}}

\item {} 
\phantomsection\label{\detokenize{p01_u6563_u6587/_u827e_u9752-_u5927_u5830_u6cb3_u2014_u2014_u6211_u7684_u4fdd_u59c6:id11}}{\hyperref[\detokenize{p01_u6563_u6587/_u827e_u9752-_u5927_u5830_u6cb3_u2014_u2014_u6211_u7684_u4fdd_u59c6:id6}]{\sphinxcrossref{1.4   名家点评}}}

\end{itemize}

\end{itemize}
\end{sphinxShadowBox}


\section{1.1   作品原文}
\label{\detokenize{p01_u6563_u6587/_u827e_u9752-_u5927_u5830_u6cb3_u2014_u2014_u6211_u7684_u4fdd_u59c6:id3}}
大堰河——我的保姆

大堰河,是我的保姆。

她的名字就是生她的村庄的名字,

她是童养媳,

大堰河,是我的保姆。//

我是地主的儿子;

也是吃了大堰河的奶而长大了的

大堰河的儿子。

大堰河以养育我而养育她的家,

而我,是吃了你的奶而被养育了的,

大堰河啊,我的保姆。//

大堰河,今天我看到雪使我想起了你:

你的被雪压着的草盖的坟墓,

你的关闭了的故居檐头的枯死的瓦菲,

你的被典押了的一丈平方的园地,

你的门前的长了青苔的石椅,

大堰河,今天我看到雪使我想起了你。//

你用你厚大的手掌把我抱在怀里,抚摸我;

在你搭好了灶火之后,

在你拍去了围裙上的炭灰之后,

在你尝到饭已煮熟了之后,

在你把乌黑的酱碗放到乌黑的桌子上之后,

在你补好了儿子们的为山腰的荆棘扯破的衣服之后,

在你把小儿被柴刀砍伤了的手包好之后,

在你把夫儿们的衬衣上的虱子一颗颗地掐死之后,

在你拿起了今天的第一颗鸡蛋之后,

你用你厚大的手掌把我抱在怀里,抚摸我。//

我是地主的儿子,

在我吃光了你大堰河的奶之后,

我被生我的父母领回到自己的家里。

啊,大堰河,你为什么要哭?//

我做了生我的父母家里的新客了!

我摸着红漆雕花的家具,

我摸着父母的睡床上金色的花纹,

我呆呆地看着檐头的我不认得的“天伦叙乐”的匾,

我摸着新换上的衣服的丝的和贝壳的纽扣,

我看着母亲怀里的不熟识的妹妹,

我坐着油漆过的安了火钵的炕凳,

我吃着碾了三番的白米的饭,

但,我是这般忸怩不安!因为我

我做了生我的父母家里的新客了。//

大堰河,为了生活,

在她流尽了她的乳汁之后,

她就开始用抱过我的两臂劳动了;

她含着笑,洗着我们的衣服,

她含着笑,提着菜篮到村边的结冰的池塘去,

她含着笑,切着冰屑悉索的萝卜,

她含着笑,用手掏着猪吃的麦糟,

她含着笑,扇着炖肉的炉子的火,

她含着笑,背了团箕到广场上去,

晒好那些大豆和小麦,

大堰河,为了生活,

在她流尽了她的乳液之后,

她就用抱过我的两臂,劳动了。//

大堰河,深爱着她的乳儿;

在年节里,为了他,忙着切那冬米的糖,

为了他,常悄悄地走到村边的她的家里去,

为了他,走到她的身边叫一声“妈”,

大堰河,把他画的大红大绿的关云长

贴在灶边的墙上,

大堰河,会对她的邻居夸口赞美她的乳儿;

大堰河曾做了一个不能对人说的梦:

在梦里,她吃着她的乳儿的婚酒,

坐在辉煌的结彩的堂上,

而她的娇美的媳妇亲切的叫她“婆婆”

……//

大堰河,深爱着她的乳儿!

大堰河,在她的梦没有做醒的时候已死了。

她死时,乳儿不在她的旁侧,

她死时,平时打骂她的丈夫也为她流泪,

五个儿子,个个哭得很悲,

她死时,轻轻地呼着她的乳儿的名字,

大堰河,已死了,

她死时,乳儿不在她的旁侧。//

大堰河,含泪的去了!

同着四十几年的人世生活的凌侮,

同着数不尽的奴隶的凄苦,

同着四块钱的棺材和几束稻草,

同着几尺长方的埋棺材的土地,

同着一手把的纸钱的灰,

大堰河,她含泪的去了。//

这是大堰河所不知道的:

她的醉酒的丈夫已死去,

大儿做了土匪,

第二个死在炮火的烟里,

第三,第四,第五

在师傅和地主的叱骂声里过着日子。

而我,我是在写着给予这不公道的世界的咒语。

当我经了长长的漂泊回到故土时,

在山腰里,田野上,

兄弟们碰见时,是比六七年前更要亲密!

这,这是为你,静静地睡着的大堰河

所不知道的啊!//

大堰河,今天,你的乳儿是在狱里,

写着一首呈给你的赞美诗,

呈给你黄土下紫色的灵魂,

呈给你拥抱过我的直伸着的手,

呈给你吻过我的唇,

呈给你泥黑的温柔的脸颜,

呈给你养育了我的乳房,

呈给你的儿子们,我的兄弟们,

呈给大地上一切的,

我的大堰河般的保姆和她们的儿子,

呈给爱我如爱她自己的儿子般的大堰河。//

大堰河,

我是吃了你的奶而长大了的

你的儿子,

我敬你

爱你!

一九三三年一月十四日,雪朝


\section{1.2   创作背景}
\label{\detokenize{p01_u6563_u6587/_u827e_u9752-_u5927_u5830_u6cb3_u2014_u2014_u6211_u7684_u4fdd_u59c6:id4}}
1932年,诗人因加入左翼美术家联盟被捕,以“宣传与三民主义不相容主义”罪被判入狱6年。在狱中他写下了这首《大堰河——我的保姆》。{[}2{]}


\section{1.3   作品鉴赏}
\label{\detokenize{p01_u6563_u6587/_u827e_u9752-_u5927_u5830_u6cb3_u2014_u2014_u6211_u7684_u4fdd_u59c6:id5}}
《大堰河,我的保姆》是艾青的成名之作。这是一个地主阶级叛逆的儿子献给他的真正母亲——中国大地善良而不幸的普通农妇的颂歌。

这首诗感情真挚深切。诗中反复陈述:“大堰河,是我的保姆”,诗人是地主的儿子,长在“大堰河”的怀中,吮吸着她的乳汁,这不仅养育了诗人和身体,也养育了诗人的感情。诗人深深领受了她的爱,及至到了上学的年龄离开养母回到亲生父母身边的时候,他感到父母的陌生,更感到养母的对他的重要。养母正直、善良、朴素的品格影响了诗人的一生。这首诗从头到尾,始终围绕“我”与“她”的关系来写,他对大堰河深厚的感情,都表现在娓娓动情的陈述之中,他在监狱里,看见了雪就想到大堰河“被雪压着的草盖的坟墓”,想起她的故居园地,想起她对他的关怀和爱……于是他用他的深情的诗,表现了大堰河的具体劳作情景,也写了她心灵深处的感情波纹,就连她美丽的梦境,也同对乳儿的“幸福命运”的祝愿融合在一起。有了这样的真情,这样的心灵,才使这位劳动妇女形象更加崇高、完美,所以诗人要把热烈的颂扬,“呈给大地上一切的/我的大堰河般的保姆和他们的儿子/呈给爱我如爱她自己的儿子般的大堰河”。这样就使“大堰河”以某种象征意义,升华为永远与山河、村庄同在的人民的化身,或者说是中国农民的化身。

艾青在《大堰河,我的保姆》开始表现他诗作的艺术特色,他首先是从“感觉”出发,像印象派画家那么重视感觉和感受,而且注意主观情感对感觉的渗入与融合。并在二者的融合中产生出多层次的联想,创造出既是清晰的,又具有广阔象征意义的视觉形象。诗总是具体的、有着鲜明形象的,如这首诗写大堰河的劳作,写大堰河的笑,写大堰河的爱和死。都呈现可视可感的立体的意象符号附加形容。最后叠句排比旬的运用,如“呈给你黄土下紫色的灵魂/呈给你拥抱过我的直伸着的手/呈给你吻过我的唇。/呈给你泥黑的温柔的脸颜/呈给你蒜育我的乳房……”具体的描写,保证语言的形象性,这也是艾青诗的艺术魅力的奥秘所在,他后来的诗作,更自觉地将它发扬光大了。

这是一首献给保姆大堰河的诗篇。诗人叙述了这位普通中国妇女平凡而坎坷、不幸的一生,表达了对这位伟大母亲由衷的感恩之情。大堰河,也是千千万万中国母亲的代表,正是这片如同慈母一样宽阔的土地和这个伟大的祖国,尽管她受尽欺辱,满身疮痍,历尽沧桑,然而却永远不失母性和母爱伟大的光辉诗歌饱含深情,反复咏唱,如泣如诉。


\section{1.4   名家点评}
\label{\detokenize{p01_u6563_u6587/_u827e_u9752-_u5927_u5830_u6cb3_u2014_u2014_u6211_u7684_u4fdd_u59c6:id6}}
现代文学家茅盾:“用沉郁的笔调细写了乳娘兼女佣(《大堰河》)的生活痛苦”。(《中国现代文学管窥》)

中国作家协会会员张同吾:它像一颗光华熠熠的新星,出现在30年代的中国诗坛上;它以深沉隽永的情思,在广大读者的心田里镌刻着久远而常新的记忆。(《张同吾文集》){[}6{]}

现代文艺理论家、诗人胡风:“至于《大堰河——我的保姆》,在这里有了一个用乳汁用母爱喂养别人的孩子,用劳力用忠诚服侍别人的农妇的形象,乳儿的作者用着朴素的真实的言语对这形象呈诉了切切的爱心。在这里他提出了对于‘这不公道的世界’的诅咒,告白了他和被侮辱的兄弟们比以前‘更要亲密’。虽然全篇流着私情地温暖,但他和我们之间已没有了难越的界限了。”(《通三统:一种文学史实验》)


\chapter{1   范仲淹-岳阳楼记}
\label{\detokenize{p01_u6563_u6587/_u8303_u4ef2_u6df9-_u5cb3_u9633_u697c_u8bb0:id1}}\label{\detokenize{p01_u6563_u6587/_u8303_u4ef2_u6df9-_u5cb3_u9633_u697c_u8bb0::doc}}
\begin{sphinxShadowBox}
\sphinxstyletopictitle{目录}
\begin{itemize}
\item {} 
\phantomsection\label{\detokenize{p01_u6563_u6587/_u8303_u4ef2_u6df9-_u5cb3_u9633_u697c_u8bb0:id10}}{\hyperref[\detokenize{p01_u6563_u6587/_u8303_u4ef2_u6df9-_u5cb3_u9633_u697c_u8bb0:id1}]{\sphinxcrossref{1   范仲淹-岳阳楼记}}}
\begin{itemize}
\item {} 
\phantomsection\label{\detokenize{p01_u6563_u6587/_u8303_u4ef2_u6df9-_u5cb3_u9633_u697c_u8bb0:id11}}{\hyperref[\detokenize{p01_u6563_u6587/_u8303_u4ef2_u6df9-_u5cb3_u9633_u697c_u8bb0:id3}]{\sphinxcrossref{1.1   作品原文}}}

\item {} 
\phantomsection\label{\detokenize{p01_u6563_u6587/_u8303_u4ef2_u6df9-_u5cb3_u9633_u697c_u8bb0:id12}}{\hyperref[\detokenize{p01_u6563_u6587/_u8303_u4ef2_u6df9-_u5cb3_u9633_u697c_u8bb0:id4}]{\sphinxcrossref{1.2   词句注释}}}

\item {} 
\phantomsection\label{\detokenize{p01_u6563_u6587/_u8303_u4ef2_u6df9-_u5cb3_u9633_u697c_u8bb0:id13}}{\hyperref[\detokenize{p01_u6563_u6587/_u8303_u4ef2_u6df9-_u5cb3_u9633_u697c_u8bb0:id5}]{\sphinxcrossref{1.3   白话译文}}}

\item {} 
\phantomsection\label{\detokenize{p01_u6563_u6587/_u8303_u4ef2_u6df9-_u5cb3_u9633_u697c_u8bb0:id14}}{\hyperref[\detokenize{p01_u6563_u6587/_u8303_u4ef2_u6df9-_u5cb3_u9633_u697c_u8bb0:id6}]{\sphinxcrossref{1.4   创作背景}}}

\item {} 
\phantomsection\label{\detokenize{p01_u6563_u6587/_u8303_u4ef2_u6df9-_u5cb3_u9633_u697c_u8bb0:id15}}{\hyperref[\detokenize{p01_u6563_u6587/_u8303_u4ef2_u6df9-_u5cb3_u9633_u697c_u8bb0:id7}]{\sphinxcrossref{1.5   文学赏析}}}

\item {} 
\phantomsection\label{\detokenize{p01_u6563_u6587/_u8303_u4ef2_u6df9-_u5cb3_u9633_u697c_u8bb0:id16}}{\hyperref[\detokenize{p01_u6563_u6587/_u8303_u4ef2_u6df9-_u5cb3_u9633_u697c_u8bb0:id8}]{\sphinxcrossref{1.6   名家点评}}}

\item {} 
\phantomsection\label{\detokenize{p01_u6563_u6587/_u8303_u4ef2_u6df9-_u5cb3_u9633_u697c_u8bb0:id17}}{\hyperref[\detokenize{p01_u6563_u6587/_u8303_u4ef2_u6df9-_u5cb3_u9633_u697c_u8bb0:id9}]{\sphinxcrossref{1.7   作者简介}}}

\end{itemize}

\end{itemize}
\end{sphinxShadowBox}

《岳阳楼记》是北宋文学家范仲淹于庆历六年九月十五日(1046年10月17日)应好友巴陵郡太守滕子京之请为重修岳阳楼而创作的一篇散文。这篇文章通过写岳阳楼的景色,以及阴雨和晴朗时带给人的不同感受,揭示了“不以物喜,不以己悲”的古仁人之心,也表达了自己“先天下之忧而忧,后天下之乐而乐”的爱国爱民情怀。文章超越了单纯写山水楼观的狭境,将自然界的晦明变化、风雨阴晴和“迁客骚人”的“览物之情”结合起来写,从而将全文的重心放到了纵议政治理想方面,扩大了文章的境界。全文记叙、写景、抒情、议论融为一体,动静相生,明暗相衬,文词简约,音节和谐,用排偶章法作景物对比,成为杂记中的创新。


\section{1.1   作品原文}
\label{\detokenize{p01_u6563_u6587/_u8303_u4ef2_u6df9-_u5cb3_u9633_u697c_u8bb0:id3}}
岳阳楼记1

庆历四年春2,滕子京谪守巴陵郡3。越明年4,政通人和5,百废具兴6。乃重修岳阳楼7,增其旧制8,刻唐贤今人诗赋于其上9。属予作文以记之10。

予观夫巴陵胜状11,在洞庭一湖。衔远山12,吞长江13,浩浩汤汤14,横无际涯15;朝晖夕阴,气象万千16。此则岳阳楼之大观也17,前人之述备矣18。然则北通巫峡19,南极潇湘20,迁客骚人21,多会于此22,览物之情,得无异乎23?

若夫淫雨霏霏24,连月不开25,阴风怒号26,浊浪排空27;日星隐曜28,山岳潜形29;商旅不行30,樯倾楫摧31;薄暮冥冥32,虎啸猿啼。登斯楼也,则有去国怀乡33,忧谗畏讥34,满目萧然35,感极而悲者矣36。

至若春和景明37,波澜不惊38,上下天光39,一碧万顷;沙鸥翔集,锦鳞游泳40;岸芷汀兰41,郁郁青青42。而或长烟一空43,皓月千里44,浮光跃金45,静影沉璧46,渔歌互答47,此乐何极48!登斯楼也,则有心旷神怡49,宠辱偕忘50,把酒临风51,其喜洋洋者矣52。

嗟夫53!予尝求古仁人之心54,或异二者之为55。何哉?不以物喜,不以己悲56;居庙堂之高则忧其民57;处江湖之远则忧其君58。是进亦忧,退亦忧。然则何时而乐耶?其必曰:“先天下之忧而忧,后天下之乐而乐”乎59。噫!微斯人,吾谁与归60?

时六年九月十五日。{[}1{]}


\section{1.2   词句注释}
\label{\detokenize{p01_u6563_u6587/_u8303_u4ef2_u6df9-_u5cb3_u9633_u697c_u8bb0:id4}}
1.记:一种文体。可以写景、叙事,多为议论。但目的是为了抒发作者的情怀和政治抱负(阐述作者的某些观念)。

2.庆历四年:公元1044年。庆历,宋仁宗赵祯的年号。文章末句中的“时六年”,指庆历六年(1046),点明作文的时间。

3.滕子京谪(zhé)守巴陵郡:滕子京降职任岳州太守。滕子京,名宗谅,子京是他的字,范仲淹的朋友。谪守,把被革职的官吏或犯了罪的人充发到边远的地方。在这里作为动词被贬官,降职解释。谪,封建王朝官吏降职或远调。守,做郡的长官。汉朝“守某郡”,就是做某郡的太守;宋朝废郡称州,应说“知某州”。巴陵郡,即岳州,治所在今湖南岳阳,这里沿用古称。“守巴陵郡”就是“守岳州”。

4.越明年:有三说,其一指庆历五年,为针对庆历四年而言;其二指庆历六年,此“越”为经过、经历;其三指庆历七年,针对作记时间庆历六年而言。

5.政通人和:政事顺利,百姓和乐。政,政事。通,通顺。和,和乐。这是赞美滕子京的话。

6.百废具兴:各种荒废的事业都兴办起来了。百,不是确指,形容其多。废,这里指荒废的事业。具,通“俱”,全,皆。兴,复兴。

7.乃:于是。

8.制:规模。

9.唐贤今人:唐代和当代名人。贤,形容词作名词用。

10.属(zhǔ):通“嘱”,嘱托、嘱咐。予:我。作文:写文章。以:连词,用来。记:记述。

11.夫:那。胜状:胜景,好景色。

12.衔:包含。

13.吞:吞吐。

14.浩浩汤汤(shāng):水波浩荡的样子。汤汤,水流大而急。

15.横无际涯:宽阔无边。横,广远。际涯,边。际专指陆地边界,涯专指水的边界)。

16.朝晖夕阴,气象万千:或早或晚(一天里)阴晴多变化。朝,在早晨,名词做状语。晖,日光。气象,景象。万千,千变万化。

17.此则岳阳楼之大观也:这就是岳阳楼的雄伟景象。此,这。则,就。大观,雄伟景象。

18.前人之述备矣:前人的记述很详尽了。前人之述,指上面说的“唐贤今人诗赋”。备,详尽,完备。矣,语气词“了”。之,助词,的。

19.然则:虽然如此,那么。

20.南极潇湘:南面直到潇水、湘水。潇水是湘水的支流。湘水流入洞庭湖。南,向南。极,尽,最远到达。

21.迁客:谪迁的人,指降职远调的人。骚人:诗人。战国时屈原作《离骚》,因此后人也称诗人为骚人。

22.多:大多。会:聚集。

23.览物之情,得无异乎:看到自然景物而引发的情感,怎能不有所不同呢?览,观看,欣赏。得无……乎,大概……吧。

24.若夫:用在一段话的开头以引起下文。下文的“至若”,同此。“若夫”近似“像那”。“至若”近似“至于”。淫雨,连绵不断的雨。霏霏,雨或雪(繁密)的样子。

25.开:(天气)放晴。

26.阴,阴冷。

27.排空,冲向天空。

28.日星隐曜(yào):太阳和星星隐藏起光辉。曜(不为耀,古文中以此当作日光),光辉,日光。

29.山岳潜形:山岳隐没了形体。岳,高大的山。潜,隐没。形,形迹。

30.行:走,此指前行。

31.樯(qiáng)倾楫(jí)摧:桅杆倒下,船桨折断。樯,桅杆。楫,船桨。倾,倒下。摧,折断。

32.薄暮冥冥:傍晚天色昏暗。薄,迫近。冥冥,昏暗的样子。

33.则,就。有:产生……的(情感)。

34.去国怀乡,忧谗畏讥:离开国都,怀念家乡,担心(人家)说坏话,惧怕(人家)批评指责。去,离开。国,国都,指京城。忧,担忧。谗,谗言。畏,害怕,惧怕。讥,嘲讽。

35.萧然:凄凉冷落的样子。

36.感极,感慨到了极点。而,连词,表顺接。

37.至若春和景明:至于到了春天气候暖和,阳光普照。至若,至于。春和,春风和煦。景,日光。明,明媚。

38.波澜不惊:湖面平静,没有惊涛骇浪。惊,这里有“起”“动”的意思。

39.上下天光,一碧万顷:天色湖面光色交映,一片碧绿,广阔无边。一,一片。万顷,极言其广。

40.沙鸥翔集,锦鳞游泳:沙鸥时而飞翔,时而停歇,美丽的鱼在水中游来游去。沙鸥,沙洲上的鸥鸟。翔集,时而飞翔,时而停歇。集,栖止,鸟停息在树上。锦鳞,指美丽的鱼。鳞,代指鱼。游泳,或浮或沉。游,贴着水面游。泳,潜入水里游。

41.岸芷(zhǐ)汀(tīng)兰:岸上的小草,小洲上的兰花。芷,香草的一种。汀,小洲,水边平地。

42.郁郁:形容草木茂盛。

43.而或长烟一空:有时大片烟雾完全消散。或,有时。长,大片。一,全。空,消散。

44.皓月千里:皎洁的月光照耀千里。

45.浮光跃金:湖水波动时,浮在水面上的月光闪耀起金光。这是描写月光照耀下的水波。有些版本作“浮光耀金”。

46.静影沉璧:湖水平静时,明月映入水中,好似沉下一块玉璧。这里是写无风时水中的月影。璧,圆形正中有孔的玉。沉璧,像沉入水中的璧玉。

47.互答:一唱一和。

48.何极:哪有穷尽。何,怎么。极,穷尽。

49.心旷神怡:心情开朗,精神愉快。旷,开阔。怡,愉快。

50.宠辱偕忘:荣耀和屈辱一并都忘了。宠,荣耀。辱,屈辱。偕,一起,一作“皆”。

51.把酒临风:端酒面对着风,就是在清风吹拂中端起酒来喝。把,持,执。临,面对。

52.洋洋:高兴的样子。

53.嗟(jiē)夫:唉。嗟夫为两个词,皆为语气词。

54.尝:曾经。求:探求。古仁人:古时品德高尚的人。心:思想(感情心思)。

55.或异二者之为:或许不同于(以上)两种心情。或,近于“或许”“也许”的意思,表委婉口气。为,这里指心理活动,即两种心情。二者,这里指前两段的“悲”与“喜”。

56.不以物喜,不以己悲:不因为外物好坏和自己得失而或喜或悲(此句为互文)。以,因为。

57.居庙堂之高则忧其民:在朝中做官就担忧百姓。居庙堂之高:处在高高的庙堂上,意为在朝中做官。庙,宗庙。堂,殿堂。庙堂:指朝廷。下文的“进”,即指“居庙堂之高”。

58.处江湖之远则忧其君:处在僻远的地方做官就为君主担忧。处江湖之远:处在偏远的江湖间,意思是不在朝廷上做官。之:定语后置的标志。是,这样。下文的“退”,即指“处江湖之远”。

59.先天下之忧而忧,后天下之乐而乐:在天下人担忧之前先担忧,在天下人享乐之后才享乐。先,在……之前。后,在……之后。其,指“古仁人”。

60.微斯人,吾谁与归:(如果)没有这种人,那我同谁一道呢?微,(如果)没有。斯人,这种人(指前文的“古仁人”)。谁与归,就是“与谁归”。归,归依。{[}1{]}{[}2-3{]}


\section{1.3   白话译文}
\label{\detokenize{p01_u6563_u6587/_u8303_u4ef2_u6df9-_u5cb3_u9633_u697c_u8bb0:id5}}
庆历四年春天,滕子京降职到岳州做太守。到了第二年,政务顺利,百姓和乐,各种荒废了的事业都兴办起来了。于是重新修建岳阳楼,扩展它原有的规模,把唐代名人家和今人的诗赋刻在上面。嘱咐我写一篇文章来记述这件事。

我看那巴陵郡的美景,全在洞庭湖上。洞庭湖包含远方的山脉,吞吐着长江的流水,浩浩荡荡,宽阔无边,清晨湖面上撒满阳光、傍晚又是一片阴暗,景物的变化无穷无尽。这就是岳阳楼雄伟壮丽的景象。前人对这些景象的记述已经很详尽了,虽然这样,那么这里北面通向巫峡,南面直到潇水、湘江,被降职远调的人员和吟诗作赋的诗人,大多在这里聚会,观赏这里的自然景物而触发的感情,大概会有所不同吧?

像那连绵细雨纷纷而下,整月不放晴的时候,阴冷的风怒吼着,浑浊的波浪冲向天空;日月星辰隐藏起光辉,山岳也隐没了形迹;商人和旅客无法通行,桅杆倒下,船桨折断;傍晚时分天色昏暗,只听到老虎的吼叫和猿猴的悲啼。这时登上这座楼,就会产生被贬官离开京城,怀念家乡,担心人家说坏话,惧怕人家讥讽的心情,再抬眼望去尽是萧条冷落的景象,一定会感慨万千而十分悲伤了。

至于春风和煦、阳光明媚时,湖面波平浪静,天色与湖光相接,一片碧绿,广阔无际;沙洲上的白鸥,时而飞翔时而停歇,美丽的鱼儿或浮或沉;岸上的小草,小洲上的兰花,香气浓郁,颜色青翠。有时湖面上的大片烟雾完全消散,皎洁的月光一泻千里,有时湖面上微波荡漾,浮动的月光闪着金色;有时湖面波澜不起,静静的月影像沉在水中的玉璧。渔夫的歌声响起了,一唱一和,这种乐趣真是无穷无尽!这时登上这座楼,就会感到胸怀开阔,精神愉快,光荣和屈辱一并忘了,在清风吹拂中端起酒杯痛饮,那心情真是快乐高兴极了。

唉!我曾经探求古时品德高尚的人的思想感情,他们或许不同于以上两种心情,这是什么缘故呢?是因为古时品德高尚的人不因外物好坏和自己得失而或喜或悲。在朝廷做官就为百姓忧虑;不在朝廷做官而处在僻远的江湖中间就为国君忧虑。这样他们进入朝廷做官也忧虑,退处江湖也忧虑。虽然这样,那么他们什么时候才快乐呢?那一定要说“在天下人忧愁之前先忧愁,在天下人快乐以后才快乐”吧?唉!如果没有这种人,我同谁一路呢?

写于庆历六年九月十五日。


\section{1.4   创作背景}
\label{\detokenize{p01_u6563_u6587/_u8303_u4ef2_u6df9-_u5cb3_u9633_u697c_u8bb0:id6}}
这篇文章写于庆历六年(1046)。范仲淹生活在北宋王朝内忧外患的年代,对内阶级矛盾日益突出,对外契丹和西夏虎视眈眈。为了巩固政权,改善这一处境,以范仲淹为首的政治集团开始进行改革,后人称之为“庆历新政”。但改革触犯了封建大地主阶级保守派的利益,遭到了他们的强烈反对。而皇帝改革的决心也不坚定,在以太后为首的保守官僚集团的压迫下,改革以失败告终。“庆历新政”失败后,范仲淹又因得罪了宰相吕夷简,范仲淹贬放河南邓州,这篇文章便是写于邓州,而非写于岳阳楼。

按照宋代人的习惯,写“记”以及散文一类的文章,本人并不一定要身在其地,主要是通过这种文章记录事情、写景、记人来抒发作者的感情或见解,借景抒情,托物言志。古时,邀人作记通常要附带一份所记之物的样本,也就是画卷或相关文献之类的资料,以供作记之人参考。滕子京虽然被贬岳州,但他在任期间,做了三件政绩工程,希望能够取得朝廷的谅解。重修岳阳楼便是其中之一,完成于庆历五年(1045)。滕子京为了提高其政绩工程的知名度,赠给范仲淹《洞庭晚秋图》,并向他求作两记,一则就是《岳阳楼记》,另一则是《偃虹堤记》。《岳阳楼记》所述内容有实物可查,然而《偃虹堤记》则无迹可寻。但是在《偃虹堤记》中,范仲淹也同样将偃虹堤描写得具体翔实,相较岳阳楼毫不逊色。因而,便引发了少数学者关于范仲淹写《岳阳楼记》时是否去过岳阳楼的争议。{[}4-6{]}


\section{1.5   文学赏析}
\label{\detokenize{p01_u6563_u6587/_u8303_u4ef2_u6df9-_u5cb3_u9633_u697c_u8bb0:id7}}
《岳阳楼记》全文有三百六十八字,共六段。

文章开头即切入正题,叙述事情的本末缘起。以“庆历四年春”点明时间起笔,格调庄重雅正;说滕子京为“谪守”,已暗喻对仕途沉浮的悲慨,为后文抒情设伏。下面仅用“政通人和,百废具兴”八个字,写出滕子京的政绩,引出重修岳阳楼和作记一事,为全篇文字的导引。

第二段,格调振起,情辞激昂。先总说“巴陵胜状,在洞庭一湖”,设定下文写景范围。以下“衔远山,吞长江”寥寥数语,写尽洞庭湖之大观胜概。一“衔”一“吞”,有气势。“浩浩汤汤,横无际涯”,极言水波壮阔;“朝晖夕阴,气象万千”,概说阴晴变化,简练而又生动。前四句从空间角度,后两句从时间角度,写尽了洞庭湖的壮观景象。“前人之述备矣”一句承前启后,并回应前文“唐贤今人诗赋”一语。这句话既是谦虚,也暗含转机,经“然则”一转,引出新的意境,由单纯写景,到以情景交融的笔法来写“迁客骚人”的“览物之情”,从而构出全文的主体。

三、四两段是两个排比段,并行而下,一悲一喜,一暗一明,像两股不同的情感之流,传达出景与情互相感应的两种截然相反的人生情境。

第三段写览物而悲者。以“若夫”起笔,意味深长。这是一个引发议论的词,又表明了虚拟的情调,而这种虚拟又是对无数实境的浓缩、提炼和升华,颇有典型意义。“若夫”以下描写了一种悲凉的情境,由天气的恶劣写到人心的凄楚。这里用四字短句,层层渲染,渐次铺叙。淫雨、阴风、浊浪构成了主景,不但使日星无光,山岳藏形,也使商旅不前;或又值暮色沉沉、“虎啸猿啼”之际,令过往的“迁客骚人”有“去国怀乡”之慨、“忧谗畏讥”之惧、“感极而悲”之情。

第四段写览物而喜者。以“至若”领起,打开了一个阳光灿烂的画面。“至若”尽管也是列举性的语气,但从音节上已变得高亢嘹亮,格调上已变得明快有力。下面的描写,虽然仍为四字短句,色调却为之一变,绘出春风和畅、景色明丽、水天一碧的良辰美景。更有鸥鸟在自由翱翔,鱼儿在欢快游荡,连无知的水草兰花也充满活力。作者以极为简练的笔墨,描摹出一幅湖光春色图,读之如在眼前。值得注意的是,这一段的句式、节奏与上一段大体相仿,却也另有变奏。“而或”一句就进一步扩展了意境,增强了叠加咏叹的意味,把“喜洋洋”的气氛推向高潮,而“登斯楼也”的心境也变成了“宠辱偕忘”的超脱和“把酒临风”的挥洒自如。

第五段是全篇的重心,以“嗟夫”开启,兼有抒情和议论的意味。作者在列举了悲喜两种情境后,笔调突然激扬,道出了超乎这两者之上的一种更高的理想境界,那就是“不以物喜,不以己悲”。感物而动,因物悲喜虽然是人之常情,但并不是做人的最高境界。古代的仁人,就有坚定的意志,不为外界条件的变化动摇。无论是“居庙堂之高”还是“处江湖之远”,忧国忧民之心不改,“进亦忧,退亦忧”。这似乎有悖于常理,有些不可思议。作者也就此拟出一问一答,假托古圣立言,发出了“先天下之忧而忧,后天下之乐而乐”的誓言,曲终奏雅,点明了全篇的主旨。“噫!微斯人,吾谁与归”一句结语,“如怨如慕,如泣如诉”,悲凉慷慨,一往情深,令人感喟。文章最后标明写作时间,与篇首照应。

这篇文章表现作者虽身居江湖,心忧国事,虽遭迫害,仍不放弃理想的顽强意志,同时,也是对被贬战友的鼓励和安慰。《岳阳楼记》的著名,是因为它的思想境界崇高。和它同时的另一位文学家欧阳修在为他写的碑文中说,他从小就有志于天下,常自诵曰:“士当先天下之忧而忧,后天下之乐而乐也。”可见《岳阳楼记》末尾所说的“先天下之忧而忧,后天下之乐而乐”,是范仲淹一生行为的准则。孟子说:“达则兼善天下,穷则独善其身”。这已成为封建时代许多士大夫的信条。范仲淹写这篇文章的时候正贬官在外,“处江湖之远”,本来可以采取独善其身的态度,落得清闲快乐,但他提出正直的士大夫应立身行一的准则,认为个人的荣辱升迁应置之度外,“不以物喜,不以己悲”要“先天下之忧而忧,后天下之乐而乐”,勉励自己和朋友,这是难能可贵的。这两句话所体现的精神,那种吃苦在前,享乐在后的品质,无疑仍有教育意义。

就艺术而论,《岳阳楼记》也是一篇优秀的文章。

第一,岳阳楼之大观,前人已经说尽了,再重复那些老话没有意思。范仲淹就是采取了换一个新的角度,找一个新的题目,另说他的一套。文章的题目是“岳阳楼记”,却巧妙地避开楼不写,而去写洞庭湖,写登楼的迁客骚人看到洞庭湖的不同景色时产生的不同感情,以衬托最后一段所谓“古仁人之心”。范仲淹的别出心裁,让人佩服。

第二,记事、写景、抒情和议论交融在一篇文章中,记事简明,写景铺张,抒情真切,议论精辟。议论的部分字数不多,但有统帅全文的作用,所以有人说这是一篇独特的议论文。《岳阳楼记》的议论技巧,确实有值得借鉴的地方。

第三,这篇文章的语言很有特色。它虽然是一篇散文,却穿插了许多四言的对偶句,如“日星隐曜,山岳潜形。”“沙鸥翔集,锦鳞游泳。”“长烟一空,皓月千里;浮光跃金,静影沉璧。”这些骈句为文章增添了色彩。作者锤炼字句的功夫也很深,如“衔远山,吞长江”这两句的“衔”字、“吞”字,恰切地表现了洞庭湖浩瀚的气势。“不以物喜,不以己悲”,简洁的八个字,像格言那样富有启示性。“先天下之忧而忧,后天下之乐而乐”,把丰富的意义熔铸到短短的两句话中,字字有力。

全文记叙、写景、抒情、议论融为一体,动静相生,明暗相衬,文词简约,音节和谐,用排偶章法作景物对比,成为杂记中的创新。{[}7-9{]}


\section{1.6   名家点评}
\label{\detokenize{p01_u6563_u6587/_u8303_u4ef2_u6df9-_u5cb3_u9633_u697c_u8bb0:id8}}
北宋陈师道《后山诗话》:范文正公为《岳阳楼记》,用对语说时景,世以为奇。尹师鲁读之曰:传奇体尔。传奇,唐裴铏所著小说也。

明代孙绪《无用闲谈》:范文正公《岳阳楼记》,或谓其用赋体,殆未深考耳。此是学吕温《三堂记》,体制如出一轴。然《岳阳楼记》闳远超越,青出于蓝矣。夫以文正千载人物,而乃肯学吕温,亦见君子不以人废言之盛心也。

清代金圣叹《天下才子必读书》:中间悲喜二大段,只是借来翻出后丈优乐耳,不然便是赋体类。一肚皮圣贤心地,圣贤学问,发而为才子文章。

清代顾兖《文章规范百家坪注》:楼迁斋评:首尾布置与中间状物之妙,不可及矣。然最妙处在临末断遗一转语。乃知此老胸襟度量,直与岳阳洞庭同其广。

清代蔡世远《古文雅正》:前半设局造句,犹是文人手笔。末段直达胸臆,非文正公不足以当之。或问史臣吕本中及朱文公,皆以文正公为宋朝人物第一,何也?曰:魏文会大矣,而本领徽不及;派公诚矣,而规局徽不及。尧舜君民之念,无日不存于中心,事如白日青天;公诚绝伦超群也。

清代林云铭《古文折义》:题是记岳阳楼,任他高手,少不得要说此楼前此如何倾坏,如何狭小,然后叙增修之劳。再写楼外佳景。以为滕公此举大有益于登临已耳。文正却把这些话头点过,便尽情阁起,单就迁客骚人登楼异情处,转入古仁人用心,遂将平日胸中致君泽民、先忧后乐大本领一齐揭出。盖滕公以司谏谪守巴陵,居庙堂之高者忽处江湖之远。其忧谗畏讥之念,宠辱之怀,抚景感触,不能自遣,情所必至。若知念及君民之当忧,自有不暇于为物喜,为己悲者。篇首提出“谪守”二字,本是此意。妙在借他方之迁客骚人,闲闲点缀,不即不离。谓之为子京说法可也,谓之自述其怀抱可也,即谓之遍告天下后世君予俱应如此存心,亦无不可也。嘻,此其所以为文,公正之文欤。

清代吴楚材、吴调候《古文观止》:岳阳楼大观,已被前人写尽,先生更不赘述,止将登楼者览物之情,写出悲喜二意。只是翻出后文忧乐一段正论。以圣贤忧国忧民心地,发而为文幸,非先生其孰能之?

清代过珙《古文评注》:首尾布置与中间状物之妙不可及矣。尤妙在入后忧乐一段,见得惟贤者而后有真忧,亦惟贤者而后有真乐。乐不以忧而废,忧不以乐而忘。此虽文正自负之词,而期望子京,隐然言外。必如是始得斯文本旨。

清代余城《重订古文释义新编》:通体俱在谪守上着笔,确是子京重修击阳楼记,一字不肯苟下。圣贤经济,才子文章,于此可兼得矣。

清代浦起龙《古文眉诠》:先忧后乐两言,先生生平所持诵也。缘情设景,借题引合,想见万物一体胸襟。

清代唐德宜《古文翼》:撇过岳阳之景,专写览物之情,引起忧乐二意,又从忧乐写出绝大本领。从来名公作记,未有若此篇之正大堂皇者,可想见文公一生节概。

清代李扶九原编、黄仁黼重订《古文笔法百篇》:入手即将题点过,而“谪守”二字,已伏一篇之意。盖谪者多悲而少喜,故将景物随写一笔,即便昂开,提出主意,隐对子京。切定洞庭畅发两段,得宽题走窄境法。末段提出仁人之用心,以规勉之,何其正大。不知此即文正公自己写照也。公为秀才时,尝言“士君子当先天下之忧而忧,后天下之乐而乐。”不觉因上悲喜,即便吐露,而忧乐正与悲喜对也。亦岂己所不能而貌为大言乎?楼记发此大议,可谓小中见大之文。看其一结,虚托闪开,作想慕不已之情,冷冷而住,不自任而矜张,不打照子京而寡迹,尤为巧妙绝伦。至中间两对,已早开有明八股之风矣。黼按君子之所以异于人者,以其存心也。心可即境而存,心不可随境而变。其所存于中者大,斯其所遇于外者小矣。文正此记,前半为岳阳写景绘情,经营惨淡,已到十分。而其中或悲或喜,处处隐对子京,即处处从请守著想。故末以忧乐二字,易悲喜二字,归到仁人身上。见得境虽变,心不与之俱变;心所存,道即与之俱存。出忧其民,处忧其君,仁人之心,自有其所以异者存也。通幅不矜才,不使气,使自己胸襟显得磊磊落落,正大而光明。非其存于中者大,而能若是乎?

清代尤焴《可斋杂稿》:文正《岳阳楼记》,精切高古,而欧公犹不以文章许之。然要皆磊磊落落,确实典重,凿凿乎如五谷之疗饥,与世之图章绘句、不根事实者,不可同年而语也。


\section{1.7   作者简介}
\label{\detokenize{p01_u6563_u6587/_u8303_u4ef2_u6df9-_u5cb3_u9633_u697c_u8bb0:id9}}
范仲淹(989-1052),字希文,北宋思想家、政治家、文学家。大中祥符八年(1015),进士及第。庆历三年(1043),参与庆历新政,提出了十项改革主张。庆历五年(1045),新政受挫,范仲淹被贬出京。皇祐四年(1052),溘然长逝,享年六十四岁,谥号文正,世称范文正公。范仲淹文学成就突出,其“先天下之忧而忧,后天下之乐而乐”思想,对后世影响深远。有《范文正公文集》。


\chapter{1   茅盾-白杨礼赞}
\label{\detokenize{p01_u6563_u6587/_u8305_u76fe-_u767d_u6768_u793c_u8d5e:id1}}\label{\detokenize{p01_u6563_u6587/_u8305_u76fe-_u767d_u6768_u793c_u8d5e::doc}}
\begin{sphinxShadowBox}
\sphinxstyletopictitle{目录}
\begin{itemize}
\item {} 
\phantomsection\label{\detokenize{p01_u6563_u6587/_u8305_u76fe-_u767d_u6768_u793c_u8d5e:id5}}{\hyperref[\detokenize{p01_u6563_u6587/_u8305_u76fe-_u767d_u6768_u793c_u8d5e:id1}]{\sphinxcrossref{1   茅盾-白杨礼赞}}}
\begin{itemize}
\item {} 
\phantomsection\label{\detokenize{p01_u6563_u6587/_u8305_u76fe-_u767d_u6768_u793c_u8d5e:id6}}{\hyperref[\detokenize{p01_u6563_u6587/_u8305_u76fe-_u767d_u6768_u793c_u8d5e:id3}]{\sphinxcrossref{1.1   作品原文}}}

\item {} 
\phantomsection\label{\detokenize{p01_u6563_u6587/_u8305_u76fe-_u767d_u6768_u793c_u8d5e:id7}}{\hyperref[\detokenize{p01_u6563_u6587/_u8305_u76fe-_u767d_u6768_u793c_u8d5e:id4}]{\sphinxcrossref{1.2   词语注释}}}

\end{itemize}

\end{itemize}
\end{sphinxShadowBox}

《白杨礼赞》是现代作家茅盾于1941年所写的一篇散文。作者以西北黄土高原上“参天耸立,不折不挠,对抗着西北风”的白杨树,来象征坚韧、勤劳的北方农民,歌颂他们在民族解放斗争中的朴实、坚强和力求上进的精神,同时对于那些“贱视民众,顽固的倒退的人们”也投出了辛辣的嘲讽。文章立意高远,形象鲜明,结构严谨,语言简练。


\section{1.1   作品原文}
\label{\detokenize{p01_u6563_u6587/_u8305_u76fe-_u767d_u6768_u793c_u8d5e:id3}}
白杨树实在不是平凡的,我赞美白杨树!

汽车在望不到边际的高原上奔驰,扑入你的视野2的,是黄绿错综的一条大毡子。黄的是土,未开垦的处女土,几十万年前由伟大的自然力堆积成功的黄土高原的外壳;绿的呢,是人类劳力战胜自然的成果,是麦田。和风吹送,翻起了一轮一轮的绿波——这时你会真心佩服昔人所造的两个字“麦浪”,若不是妙手偶得,便确是经过锤炼的语言的精华。黄与绿主宰着,无边无垠,坦荡如砥3,这时如果不是宛若4并肩的远山的连峰提醒了你(这些山峰凭你的肉眼来判断,就知道是在你脚底下的),你会忘记了汽车是在高原上行驶。这时你涌起来的感想也许是“雄壮”,也许是“伟大”,诸如此类的形容词;然而同时你的眼睛也许觉得有点倦怠,你对当前的“雄壮”或“伟大”闭了眼,而另一种的味儿在你心头潜滋暗长5了——“单调”。可不是?单调,有一点儿吧?

然而刹那间,要是你猛抬眼看见了前面远远有一排——不,或者甚至只是三五株,一株,傲然地耸立,像哨兵似的树木的话,那你的恹恹6欲睡的情绪又将如何?我那时是惊奇地叫了一声的。

那就是白杨树,西北极普通的一种树,然而实在不是平凡的一种树。

那是力争上游的一种树,笔直的干,笔直的枝。它的干呢,通常是丈把高,像是加以人工似的,一丈以内绝无旁枝。它所有的丫枝呢,一律向上,而且紧紧靠拢,也像是加以人工似的,成为一束,绝无横斜逸出7。它的宽大的叶子也是片片向上,几乎没有斜生的,更不用说倒垂了;它的皮,光滑而有银色的晕圈8,微微泛出淡青色。这是虽在北方的风雪的压迫下却保持着倔强挺立的一种树。哪怕只有碗来粗细罢,它却努力向上发展,高到丈许,二丈,参天耸立,不折不挠,对抗着西北风。

这就是白杨树,西北极普通的一种树,然而决不是平凡的树!

它没有婆娑9的姿态,没有屈曲盘旋的虬枝10,也许你要说它不美丽,──如果美是专指“婆娑”或“横斜逸出”之类而言,那么白杨树算不得树中的好女子;但是它却是伟岸11,正直,朴质,严肃,也不缺乏温和,更不用提它的坚强不屈与挺拔,它是树中的伟丈夫!当你在积雪初融的高原上走过,看见平坦的大地上傲然挺立这么一株或一排白杨树,难道你觉得树只是树,难道你就不想到它的朴质,严肃,坚强不屈,至少也象征了北方的农民;难道你竟一点也不联想到,在敌后的广大土地上,到处有坚强不屈,就像这白杨树一样傲然挺立的守卫他们家乡的哨兵!难道你又不更远一点想到这样枝枝叶叶靠紧团结,力求上进的白杨树,宛然象征了今天在华北平原纵横决荡12用血写出新中国历史的那种精神和意志。

白杨不是平凡的树。它在西北极普遍,不被人重视,就跟北方农民相似;它有极强的生命力,磨折不了,压迫不倒,也跟北方的农民相似。我赞美白杨树,就因为它不但象征了北方的农民,尤其象征了今天我们民族解放斗争中所不可缺的朴质,坚强,以及力求上进的精神。

让那些看不起民众,贱视民众,顽固的倒退的人们去赞美那贵族化的楠木13(那也是直干秀颀14的),去鄙视这极常见,极易生长的白杨罢,但是我要高声赞美白杨树!

(原载《文艺阵地》月刊第6卷第3期,1941年3月10日出版)


\section{1.2   词语注释}
\label{\detokenize{p01_u6563_u6587/_u8305_u76fe-_u767d_u6768_u793c_u8d5e:id4}}
1.礼赞:崇敬和赞美。

2.视野:视力所及的范围。

3.坦荡如砥(dǐ):平坦得像磨刀石一样。

4.宛若:很像,简直就是。

5.潜滋暗长:暗暗地不知不觉地生长。滋,生长。

6.恹恹(yānyān):困倦的样子。

7.横斜逸出:意思是,(树枝)从树干的旁边斜伸出来。

8.晕(yùn)圈:模模糊糊的圈。

9.婆娑(suō):树木的枝叶随风飘荡,像舞蹈一样的姿态。

10.虬(qiú)枝:像龙一样盘旋的枝条。虬,传说中的一种龙。

11.伟岸:魁梧,高大。

12.纵横决荡:纵横驰骋,冲杀突击。

13.楠(nán)木:常绿乔木,木质坚实,是贵重的木材。

14.秀颀(qí):美而高。颀,高大的意思。


\chapter{1   荀子-劝学}
\label{\detokenize{p01_u6563_u6587/_u8340_u5b50-_u529d_u5b66:id1}}\label{\detokenize{p01_u6563_u6587/_u8340_u5b50-_u529d_u5b66::doc}}
\begin{sphinxShadowBox}
\sphinxstyletopictitle{目录}
\begin{itemize}
\item {} 
\phantomsection\label{\detokenize{p01_u6563_u6587/_u8340_u5b50-_u529d_u5b66:id5}}{\hyperref[\detokenize{p01_u6563_u6587/_u8340_u5b50-_u529d_u5b66:id1}]{\sphinxcrossref{1   荀子-劝学}}}
\begin{itemize}
\item {} 
\phantomsection\label{\detokenize{p01_u6563_u6587/_u8340_u5b50-_u529d_u5b66:id6}}{\hyperref[\detokenize{p01_u6563_u6587/_u8340_u5b50-_u529d_u5b66:id3}]{\sphinxcrossref{1.1   作品原文}}}

\item {} 
\phantomsection\label{\detokenize{p01_u6563_u6587/_u8340_u5b50-_u529d_u5b66:id7}}{\hyperref[\detokenize{p01_u6563_u6587/_u8340_u5b50-_u529d_u5b66:id4}]{\sphinxcrossref{1.2   词句注释}}}

\end{itemize}

\end{itemize}
\end{sphinxShadowBox}

《劝学》是战国时期思想家、文学家荀子创作的一篇论说文,是《荀子》一书的首篇。文章较系统地论述了学习的理论和方法,分别从学习的重要性、学习的态度以及学习的内容和方法等方面,全面而深刻地论说了有关学习的问题。全文可分四段,第一段阐明学习的重要性,第二段讲正确的学习态度,第三段讲学习的内容,第四段讲学习应当善始善终。全文说理深入,结构严谨,代表了先秦论说文成熟阶段的水平。


\section{1.1   作品原文}
\label{\detokenize{p01_u6563_u6587/_u8340_u5b50-_u529d_u5b66:id3}}
君子曰1:学不可以已2。

青,取之于蓝3,而青于蓝;冰,水为之,而寒于水。木直中绳4,輮以为轮5,其曲中规6。虽有槁暴7,不复挺者8,輮使之然也。故木受绳则直9,金就砺则利10,君子博学而日参省乎己11,则知明而行无过矣12。

故不登高山,不知天之高也;不临深溪,不知地之厚也;不闻先王之遗言13,不知学问之大也。干、越、夷、貉之子14,生而同声,长而异俗,教使之然也。诗曰:“嗟尔君子,无恒安息。靖共尔位,好是正直。神之听之,介尔景福15。”神莫大于化道,福莫长于无祸。

吾尝终日而思矣,不如须臾之所学也16;吾尝跂而望矣17,不如登高之博见也18。登高而招,臂非加长也,而见者远;顺风而呼,声非加疾也19,而闻者彰20。假舆马者21,非利足也22,而致千里;假舟楫者,非能水也23,而绝江河24。君子生非异也25,善假于物也。

南方有鸟焉,名曰蒙鸠26,以羽为巢,而编之以发,系之苇苕27,风至苕折,卵破子死。巢非不完也,所系者然也。西方有木焉,名曰射干28,茎长四寸,生于高山之上,而临百仞之渊,木茎非能长也,所立者然也。蓬生麻中,不扶而直;白沙在涅,与之俱黑29。兰槐之根是为芷30,其渐之滫31,君子不近,庶人不服32。其质非不美也,所渐者然也33。故君子居必择乡,游必就士,所以防邪辟而近中正也34。

物类之起,必有所始。荣辱之来,必象其德。肉腐出虫,鱼枯生蠹35。怠慢忘身,祸灾乃作。强自取柱36,柔自取束37。邪秽在身,怨之所构38。施薪若一,火就燥也,平地若一,水就湿也。草木畴生39,禽兽群焉,物各从其类也。是故质的张40,而弓矢至焉;林木茂,而斧斤至焉41;树成荫,而众鸟息焉。醯酸42,而蜹聚焉43。故言有招祸也,行有招辱也,君子慎其所立乎!

积土成山,风雨兴焉;积水成渊,蛟龙生焉;积善成德,而神明自得,圣心备焉。故不积跬步44,无以至千里;不积小流,无以成江海。骐骥一跃45,不能十步;驽马十驾46,功在不舍47。锲而舍之48,朽木不折;锲而不舍,金石可镂49。蚓无爪牙之利,筋骨之强,上食埃土,下饮黄泉,用心一也。蟹六跪而二螯50,非蛇鳝93之穴无可寄托者,用心躁也。

是故无冥冥之志者51,无昭昭之明52;无惛惛之事者,无赫赫之功。行衢道者不至,事两君者不容。目不能两视而明,耳不能两听而聪。螣蛇无足而飞53,鼫鼠五技而穷54。《诗》曰:“尸鸠在桑,其子七兮。淑人君子,其仪一兮。其仪一兮,心如结兮55!”故君子结于一也56。

昔者瓠巴鼓瑟57,而沈鱼出听58;伯牙鼓琴59,而六马仰秣60。故声无小而不闻,行无隐而不形。玉在山而草木润,渊生珠而崖不枯61。为善不积邪62?安有不闻者乎?

学恶乎始?恶乎终?曰:其数则始乎诵经63,终乎读礼;其义则始乎为士,终乎为圣人,真积力久则入,学至乎没而后止也。故学数有终,若其义则不可须臾舍也。为之,人也;舍之,禽兽也。故书者,政事之纪也;诗者,中声之所止也;礼者,法之大分64,类之纲纪也。故学至乎礼而止矣。夫是之谓道德之极。礼之敬文也,乐之中和也,诗书之博也,春秋之微也,在天地之间者毕矣。

君子之学也,入乎耳,著乎心,布乎四体,形乎动静。端而言,蝡而动65,一可以为法则。小人之学也,入乎耳,出乎口;口耳之间,则四寸耳,曷足以美七尺之躯哉!古之学者为己,今之学者为人。君子之学也,以美其身;小人之学也,以为禽犊。故不问而告谓之傲66,问一而告二谓之囋67。傲、非也,囋、非也;君子如向矣68。

学莫便乎近其人。礼乐法而不说,诗书故而不切,春秋约而不速。方其人之习君子之说69,则尊以遍矣,周于世矣。故曰:学莫便乎近其人。

学之经莫速乎好其人,隆礼次之。上不能好其人,下不能隆礼,安特将学杂识志,顺诗书而已耳70。则末世穷年,不免为陋儒而已。将原先王,本仁义71,则礼正其经纬蹊径也72。若挈裘领73,诎五指而顿之74,顺者不可胜数也。不道礼宪75,以诗书为之,譬之犹以指测河也,以戈舂黍也76,以锥飡壶也77,不可以得之矣。故隆礼,虽未明,法士也;不隆礼,虽察辩,散儒也。

问楛者78,勿告也;告楛者,勿问也;说楛者,勿听也。有争气者79,勿与辩也。故必由其道至,然后接之;非其道则避之。故礼恭,而后可与言道之方;辞顺,而后可与言道之理;色从而后可与言道之致80。故未可与言而言,谓之傲;可与言而不言,谓之隐81;不观气色而言,谓之瞽82。故君子不傲、不隐、不瞽,谨顺其身83。诗曰:“匪交匪舒,天子所予84。”此之谓也。

百发失一,不足谓善射;千里跬步不至,不足谓善御;伦类不通85,仁义不一,不足谓善学。学也者,固学一之也。一出焉,一入焉,涂巷之人也;其善者少,不善者多,桀纣盗跖也86;全之尽之,然后学者也。

君子知夫不全不粹之不足以为美也,故诵数以贯之87,思索以通之,为其人以处之,除其害者以持养之。使目非是无欲见也88,使耳非是无欲闻也,使口非是无欲言也,使心非是无欲虑也。及至其致好之也,目好之五色,耳好之五声89,口好之五味90,心利之有天下。是故权利不能倾也,群众不能移也,天下不能荡也。生乎由是,死乎由是,夫是之谓德操。德操然后能定,能定然后能应91。能定能应,夫是之谓成人92。天见其明,地见其光,君子贵其全也。{[}1{]}


\section{1.2   词句注释}
\label{\detokenize{p01_u6563_u6587/_u8340_u5b50-_u529d_u5b66:id4}}
1.君子:指有学问有修养的人。

2.学不可以已(yǐ):学习不能停止。

3.青取之于蓝:靛青,从蓝草中取得。青,靛青,一种染料。蓝,蓼蓝,一年生草本植物,叶子含蓝汁,可以做蓝色染料。

4.中(zhòng)绳:(木材)合乎拉直的墨线。绳,墨线。

5.輮(róu):通“煣”,古代用火烤使木条弯曲的一种工艺。

6.规:圆规,画圆的工具。

7.虽有(yòu)槁暴(pù):即使又晒干了。有,通“又”。槁,枯。暴,同“曝”,晒干。

8.挺:直。

9.受绳:用墨线量过。

10.金:指金属制的刀剑等。就砺:拿到磨刀石上去磨。砺,磨刀石。就,动词,接近,靠近。

11.日参(cān)省(xǐng)乎己:每天对照反省自己。参,一译检验,检查;二译同“叁”,多次。省,省察。乎,介词,于。博学:广泛地学习。日:每天。

12.知(zhì):通“智”,智慧。明:明达。行无过:行为没有过错。

13.遗言:犹古训。

14.干(hán):同“邗”,古国名,在今江苏扬州东北,春秋时被吴国所灭而成为吴邑,此指代吴国。夷:中国古代居住在东部的民族。貉(mò):通“貊”,中国古代居住在东北部的民族。

15.“嗟尔君子”六句:引诗见《诗经·小雅·小明》。靖,安。共,通“供”。介,给予。景,大。

16.须臾(yú):片刻,一会儿。

17.跂(qǐ):踮起脚后跟。

18.博见:看见的范围广,见得广。

19.疾:声音宏大。

20.彰:明显,清楚。这里指听得更清楚。

21.假:凭借,利用。舆:车厢,这里指车。

22.利足:脚走得快。

23.水:游泳。

24.绝:横渡。

25.生(xìng)非异:本性(同一般人)没有差别。生,通“性”,天赋,资质。

26.蒙鸠:即鹪鹩,俗称黄脰鸟,又称巧妇鸟,全身灰色*,有斑,常取茅苇一毛一毳为巢。 4) (5) (6)滫(xiu朽音):淘米水,此引为脏水、臭水。

27.苕(tiáo):芦苇的花穗。

28.射(yè)干:又名乌扇,一种草本植物,根入药,茎细长,多生于山崖之间,形似树木,所以荀子称它为“木”,其实是一种草。一说“木”为“草”字之误。

29.“蓬生麻中”四句:草长在麻地里,不用扶持也能挺立住,白沙混进了黑土里,就会变得和土一样黑。比喻生活在好的环境里,也能成为好人。蓬,蓬草。麻,麻丛。涅,黑色染料。《集解》无“白沙在涅与之俱黑”八字,据《尚书·洪范》“时人斯其惟皇之极”《正义》引文补。

30.兰槐:香草名,又叫白芷,开白花,味香。古人称其苗为“兰”,称其根为“芷”。

31.渐(jiān):浸。滫(xiǔ):泔水,已酸臭的淘米水。此引为脏水、臭水。

32.服:穿戴。

33.所渐者然也:被熏陶、影响的情况就是这样的。然,这样。

34.邪辟:品行不端的人。中正:正直之士。

35.蠹(dù):蛀蚀器物的虫子。

36.强自取柱:谓物性过硬则反易折断。柱,通“祝”(王引之说),折断。《大戴礼记·劝学》作“折”。

37.柔自取束:柔弱的东西自己导致约束。

38.构:结,造成。

39.畴:通“俦”,类。

40.质:箭靶。的(dì):箭靶的中心。

41.斤:斧子。

42.醯(xī):本意指醋。

43.蜹(ruì):飞虫名,属蚊类。

44.跬(kuǐ):行走时两脚之间的距离,等于现在所说的一步、古人所说的半步。步:古人说一步,指左右脚都向前迈一次的距离,等于现在的两步。

45.骐(qí)骥(jì):骏马,千里马。

46.驽马十驾:劣马拉车连走十天也能到达。驽马,劣马。驾,古代马拉车时,早晨套一上车,晚上卸去。套车叫驾,所以这里用“驾”指代马车一天的行程。十驾就是套十次车,指十天的行程。此指千里的路程。

47.舍:舍弃。指不放弃行路。

48.锲(qiè):用刀雕刻。

49.镂(lòu):原指在金属上雕刻,泛指雕刻。

50.蟹六跪而二螯(áo):螃蟹有六只爪子,两个钳子。六跪,六条腿。蟹实际上是八条腿。跪,蟹脚。一说,海蟹后面的两条腿只能划水,不能用来走路或自卫,所以不能算在“跪”里面。螯,螃蟹等节肢动物身前的大爪,形如钳。

51.冥冥:昏暗不明的样子,形容专心致志、埋头苦干。下文“惛惛”与此同义。

52.昭昭:明白的样子。

53.螣(téng)蛇:古代传说中的一种能飞的神蛇。

54.鼫(shí)鼠:原作“梧鼠”,据《大戴礼记·劝学》改。鼫鼠能飞但不能飞上屋面,能爬树但不能爬到树梢,能游泳但不能渡过山谷,能挖洞但不能藏身,能奔跑但不能追过人,所以说它“五技而穷”。穷:窘困。

55.“尸鸠在桑”六句:引诗见《诗经·曹风·鸤鸠》。仪,通“义”。

56.结:结聚不散开,比喻专心一致,坚定不移。

57.瓠(hù)巴:楚国人,善于弹瑟。

58.沈:同“沉”。《集解》作“流”,据《大戴礼记·劝学》改。

59.伯牙:古代善于弹琴的人。

60.六马:古代天子之车驾用六匹马拉;此指拉车之马。仰秣:《淮南子·说山训》高诱注:“仰秣,仰头吹吐,谓马笑也。”一说“秣”通“末”,头。

61.崖:岸边。

62.邪:同“耶”,疑问语气词。

63.数:术,即方法、途径,引申为“科目”。

64.大分:大的原则、界限。

65.蝡(rú):同“蠕”,微动。

66.傲:浮躁。

67.囋:形容言语繁碎。

68.向:通“响”,回音。

69.方:通“仿”,仿效。

70.顺:通“训”,解释词义。

71.原、本:均作动词,指追溯本源。

72.经纬:直线与横线,这里指道路。另辟蹊径:小路,这里泛指道路。

73.挈:提,拎。裘:皮衣。

74.诎:通“屈”,弯曲。顿:抖动,整理。

75.道:由,遵。礼宪:礼法。

76.舂:把谷类的皮捣掉。黍:黍子,谷类。

77.飡:即“餐”,吃。壶:古代盛食物的器皿,这里指饭。

78.楛:原指器物粗糙恶劣,这里是恶劣的意思,即指不合礼义。

79.争气:指意气用事。

80.致:极致,最高的境界。

81.隐:有意隐瞒。

82.瞽:盲目从事。

83.谨顺其身:指君子谨慎修养自己,做到不傲、不隐、不瞽,待人接物恰到好处。

84.“匪交匪舒”二句:语本《诗经·小雅·采菽》。匪,非,不。交,通“侥”,侥幸急躁。舒,缓,慢。予,通“与”,赞成。

85.伦:与“类”同义,指类别。

86.桀纣:夏朝和商朝的亡国之君。盗跖:古代一个名叫跖的大盗。

87.数:数说,与“诵”意义相近。

88.是:指全而粹合乎礼仪之美。

89.五声:宫、商、角、徵、羽,这里指美妙的音乐。

90.五味:甜、酸、苦、辣、咸,这里指美味。

91.应:指对外界事物的应变能力。

92.成人:全人,完美的人。{[}2{]}

93.蛇鳝:异文“蛇蟮”。{[}3{]}


\chapter{1   郁达夫-古都的秋}
\label{\detokenize{p01_u6563_u6587/_u90c1_u8fbe_u592b-_u53e4_u90fd_u7684_u79cb:id1}}\label{\detokenize{p01_u6563_u6587/_u90c1_u8fbe_u592b-_u53e4_u90fd_u7684_u79cb::doc}}
\begin{sphinxShadowBox}
\sphinxstyletopictitle{目录}
\begin{itemize}
\item {} 
\phantomsection\label{\detokenize{p01_u6563_u6587/_u90c1_u8fbe_u592b-_u53e4_u90fd_u7684_u79cb:id5}}{\hyperref[\detokenize{p01_u6563_u6587/_u90c1_u8fbe_u592b-_u53e4_u90fd_u7684_u79cb:id1}]{\sphinxcrossref{1   郁达夫-古都的秋}}}
\begin{itemize}
\item {} 
\phantomsection\label{\detokenize{p01_u6563_u6587/_u90c1_u8fbe_u592b-_u53e4_u90fd_u7684_u79cb:id6}}{\hyperref[\detokenize{p01_u6563_u6587/_u90c1_u8fbe_u592b-_u53e4_u90fd_u7684_u79cb:id3}]{\sphinxcrossref{1.1   作品原文}}}

\item {} 
\phantomsection\label{\detokenize{p01_u6563_u6587/_u90c1_u8fbe_u592b-_u53e4_u90fd_u7684_u79cb:id7}}{\hyperref[\detokenize{p01_u6563_u6587/_u90c1_u8fbe_u592b-_u53e4_u90fd_u7684_u79cb:id4}]{\sphinxcrossref{1.2   词语注释}}}

\end{itemize}

\end{itemize}
\end{sphinxShadowBox}


\section{1.1   作品原文}
\label{\detokenize{p01_u6563_u6587/_u90c1_u8fbe_u592b-_u53e4_u90fd_u7684_u79cb:id3}}
秋天,无论在什么地方的秋天,总是好的;可是啊,北国的秋,却特别地来得清,来得静,来得悲凉。我的不远千里,要从杭州赶上青岛,更要从青岛赶上北平来的理由,也不过想饱尝一尝这“秋”,这故都的秋味。

江南,秋当然也是有的,但草木凋得慢,空气来得润,天的颜色显得淡,并且又时常多雨而少风;一个人夹在苏州上海杭州,或厦门香港广州的市民中间,混混沌沌地过去,只能感到一点点清凉,秋的味,秋的色,秋的意境与姿态,总看不饱,尝不透,赏玩不到十足。秋并不是名花,也并不是美酒,那一种半开、半醉的状态,在领略秋的过程上,是不合适的。

不逢北国之秋,已将近十余年了。在南方每年到了秋天,总要想起陶然亭(1)的芦花,钓鱼台(2)的柳影,西山(3)的虫唱,玉泉(4)的夜月,潭柘寺(5)的钟声。在北平即使不出门去吧,就是在皇城人海之中,租人家一椽(6)破屋来住着,早晨起来,泡一碗浓茶,向院子一坐,你也能看得到很高很高的碧绿的天色,听得到青天下驯鸽的飞声。从槐树叶底,朝东细数着一丝一丝漏下来的日光,或在破壁腰中,静对着像喇叭似的牵牛花(朝荣)的蓝朵,自然而然地也能够感觉到十分的秋意。说到了牵牛花,我以为以蓝色或白色者为佳,紫黑色次之,淡红色最下。最好,还要在牵牛花底,叫长着几根疏疏落落的尖细且长的秋草,使作陪衬。

北国的槐树,也是一种能使人联想起秋来的点缀。像花而又不是花的那一种落蕊,早晨起来,会铺得满地。脚踏上去,声音也没有,气味也没有,只能感出一点点极微细极柔软的触觉。扫街的在树影下一阵扫后,灰土上留下来的一条条扫帚的丝纹,看起来既觉得细腻,又觉得清闲,潜意识下并且还觉得有点儿落寞(7),古人所说的梧桐一叶而天下知秋(8)的遥想,大约也就在这些深沉的地方。

秋蝉的衰弱的残声,更是北国的特产,因为北平处处全长着树,屋子又低,所以无论在什么地方,都听得见它们的啼唱。在南方是非要上郊外或山上去才听得到的。这秋蝉的嘶叫,在北方可和蟋蟀耗子一样,简直像是家家户户都养在家里的家虫。

还有秋雨哩,北方的秋雨,也似乎比南方的下得奇,下得有味,下得更像样。

在灰沉沉的天底下,忽而来一阵凉风,便息列索落地下起雨来了。一层雨过,云渐渐地卷向了西去,天又晴了,太阳又露出脸来了,着(9)着很厚的青布单衣或夹袄的都市闲人,咬着烟管,在雨后的斜桥影里,上桥头树底下去一立,遇见熟人,便会用了缓慢悠闲的声调,微叹着互答着地说:

“唉,天可真凉了——”(这了字念得很高,拖得很长。)

“可不是吗?一层秋雨一层凉了!”

北方人念阵字,总老像是层字,平平仄仄起来(10),这念错的歧韵,倒来得正好。

北方的果树,到秋天,也是一种奇景。第一是枣子树,屋角,墙头,茅房边上,灶房门口,它都会一株株地长大起来。像橄榄又像鸽蛋似的这枣子颗儿,在小椭圆形的细叶中间,显出淡绿微黄的颜色的时候,正是秋的全盛时期,等枣树叶落,枣子红完,西北风就要起来了,北方便是沙尘灰土的世界,只有这枣子、柿子、葡萄,成熟到八九分的七八月之交,是北国的清秋的佳日,是一年之中最好也没有的GoldenDays(11)。

有些批评家说,中国的文人学士,尤其是诗人,都带着很浓厚的颓废的色彩,所以中国的诗文里,赞颂秋的文字的特别的多。但外国的诗人,又何尝不然?我虽则外国诗文念的不多,也不想开出帐来,做一篇秋的诗歌散文钞(12),但你若去一翻英德法意等诗人的集子,或各国的诗文的Anthology来(13),总能够看到许多并于秋的歌颂和悲啼。各著名的大诗人的长篇田园诗或四季诗里,也总以关于秋的部分,写得最出色而最有味。足见有感觉的动物,有情趣的人类,对于秋,总是一样地特别能引起深沉,幽远、严厉、萧索的感触来的。不单是诗人,就是被关闭在牢狱里的囚犯,到了秋天,我想也一定能感到一种不能自已的深情,秋之于人,何尝有国别,更何尝有人种阶级的区别呢?不过在中国,文字里有一个“秋士”(14)的成语,读本里又有着很普遍的欧阳子的《秋声》(15)与苏东坡的《赤壁赋》等,就觉得中国的文人,与秋和关系特别深了,可是这秋的深味,尤其是中国的秋的深味,非要在北方,才感受得到底。

南国之秋,当然也是有它的特异的地方的,比如廿四桥的明月,钱塘江的秋潮,普陀山的凉雾,荔枝湾(16)的残荷等等,可是色彩不浓,回味不永。比起北国的秋来,正像是黄酒之与白干,稀饭之与馍馍,鲈鱼之与大蟹,黄犬之与骆驼。

秋天,这北国的秋天,若留得住的话,我愿把寿命的三分之二折去,换得一个三分之一的零头。

一九三四年八月在北平


\section{1.2   词语注释}
\label{\detokenize{p01_u6563_u6587/_u90c1_u8fbe_u592b-_u53e4_u90fd_u7684_u79cb:id4}}
⑴陶然亭:位于北京城南,亭名出自白居易诗句“更待菊黄家酿熟,共君一醉一陶然”。

⑵钓鱼台:在北京阜成门外三里河,玉渊潭公园北面。

⑶西山:北京西郊群山的总称,是京郊名胜。

⑷玉泉:指玉泉山,是西山东麓支脉。

⑸潭柘寺:在北京西山,相传“寺址本在青龙潭上,有古柘千章,寺以此得名。”

⑹一椽:一间屋。椽,放在房檩上架着木板或瓦的木条。

⑺落寞:冷落,寂寞。

⑻梧桐一叶而天下知秋:《淮南子,说山》:“以小明大,见叶落而知岁之将暮。”《太平御览》卷二十四引用“一叶落而知天下秋”。

⑼着:穿(衣)。

⑽平平仄仄起来:意即推敲起字的韵律来。

⑾GoldenDays:英语中指”黄金般的日子”。

⑿钞:同“抄”。

⒀Anthology:英语中指”选集”。

⒁秋士:古时指到了暮年仍不得志的知识分子。

⒂欧阳子的《秋声》:指欧阳修的《秋声赋》。

⒃荔枝湾:位于广州城西。


\chapter{1   韩愈-师说}
\label{\detokenize{p01_u6563_u6587/_u97e9_u6108-_u5e08_u8bf4:id1}}\label{\detokenize{p01_u6563_u6587/_u97e9_u6108-_u5e08_u8bf4::doc}}
\begin{sphinxShadowBox}
\sphinxstyletopictitle{目录}
\begin{itemize}
\item {} 
\phantomsection\label{\detokenize{p01_u6563_u6587/_u97e9_u6108-_u5e08_u8bf4:id14}}{\hyperref[\detokenize{p01_u6563_u6587/_u97e9_u6108-_u5e08_u8bf4:id1}]{\sphinxcrossref{1   韩愈-师说}}}
\begin{itemize}
\item {} 
\phantomsection\label{\detokenize{p01_u6563_u6587/_u97e9_u6108-_u5e08_u8bf4:id15}}{\hyperref[\detokenize{p01_u6563_u6587/_u97e9_u6108-_u5e08_u8bf4:id3}]{\sphinxcrossref{1.1   作品原文}}}

\item {} 
\phantomsection\label{\detokenize{p01_u6563_u6587/_u97e9_u6108-_u5e08_u8bf4:id16}}{\hyperref[\detokenize{p01_u6563_u6587/_u97e9_u6108-_u5e08_u8bf4:id4}]{\sphinxcrossref{1.2   词句注释}}}

\item {} 
\phantomsection\label{\detokenize{p01_u6563_u6587/_u97e9_u6108-_u5e08_u8bf4:id17}}{\hyperref[\detokenize{p01_u6563_u6587/_u97e9_u6108-_u5e08_u8bf4:id5}]{\sphinxcrossref{1.3   白话译文}}}

\item {} 
\phantomsection\label{\detokenize{p01_u6563_u6587/_u97e9_u6108-_u5e08_u8bf4:id18}}{\hyperref[\detokenize{p01_u6563_u6587/_u97e9_u6108-_u5e08_u8bf4:id6}]{\sphinxcrossref{1.4   创作背景}}}

\item {} 
\phantomsection\label{\detokenize{p01_u6563_u6587/_u97e9_u6108-_u5e08_u8bf4:id19}}{\hyperref[\detokenize{p01_u6563_u6587/_u97e9_u6108-_u5e08_u8bf4:id7}]{\sphinxcrossref{1.5   文学赏析}}}

\item {} 
\phantomsection\label{\detokenize{p01_u6563_u6587/_u97e9_u6108-_u5e08_u8bf4:id20}}{\hyperref[\detokenize{p01_u6563_u6587/_u97e9_u6108-_u5e08_u8bf4:id8}]{\sphinxcrossref{1.6   名家点评}}}
\begin{itemize}
\item {} 
\phantomsection\label{\detokenize{p01_u6563_u6587/_u97e9_u6108-_u5e08_u8bf4:id21}}{\hyperref[\detokenize{p01_u6563_u6587/_u97e9_u6108-_u5e08_u8bf4:id9}]{\sphinxcrossref{1.6.1   唐代}}}

\item {} 
\phantomsection\label{\detokenize{p01_u6563_u6587/_u97e9_u6108-_u5e08_u8bf4:id22}}{\hyperref[\detokenize{p01_u6563_u6587/_u97e9_u6108-_u5e08_u8bf4:id10}]{\sphinxcrossref{1.6.2   宋代}}}

\item {} 
\phantomsection\label{\detokenize{p01_u6563_u6587/_u97e9_u6108-_u5e08_u8bf4:id23}}{\hyperref[\detokenize{p01_u6563_u6587/_u97e9_u6108-_u5e08_u8bf4:id11}]{\sphinxcrossref{1.6.3   元代}}}

\item {} 
\phantomsection\label{\detokenize{p01_u6563_u6587/_u97e9_u6108-_u5e08_u8bf4:id24}}{\hyperref[\detokenize{p01_u6563_u6587/_u97e9_u6108-_u5e08_u8bf4:id12}]{\sphinxcrossref{1.6.4   明代}}}

\item {} 
\phantomsection\label{\detokenize{p01_u6563_u6587/_u97e9_u6108-_u5e08_u8bf4:id25}}{\hyperref[\detokenize{p01_u6563_u6587/_u97e9_u6108-_u5e08_u8bf4:id13}]{\sphinxcrossref{1.6.5   清代}}}

\end{itemize}

\end{itemize}

\end{itemize}
\end{sphinxShadowBox}

《师说》是唐代文学家韩愈创作的一篇议论文。文章阐说从师求学的道理,讽刺耻于相师的世态,教育了青年,起到转变风气的作用。文中列举正反面的事例层层对比,反复论证,论述了从师表学习的必要性和原则,批判了当时社会上“耻学于师”的陋习,表现出非凡的勇气和斗争精神,也表现出作者不顾世俗独抒己见的精神。全文幅虽不长,但涵义深广,论点鲜明,结构严谨,说理透彻,富有较强的说服力和感染力。


\section{1.1   作品原文}
\label{\detokenize{p01_u6563_u6587/_u97e9_u6108-_u5e08_u8bf4:id3}}
古之学者1必有师。师者,所以传道受业解惑也2。人非生而知之3者,孰能无惑?惑而不从师,其为惑也4,终不解矣。生乎吾前5,其闻6道也固先乎吾,吾从而师之7;生乎吾后,其闻道也亦先乎吾,吾从而师之。吾师道也8,夫庸知其年之先后生于吾乎9?是故10无11贵无贱,无长无少,道之所存,师之所存也。

嗟乎!师道之不传也久矣!欲人之无惑也难矣!古之圣人,其出人也远矣,犹且从师而问焉;今之众人,其下圣人也亦远矣,而耻学于师。是故圣益圣,愚益愚。圣人之所以为圣,愚人之所以为愚,其皆出于此乎?爱其子,择师而教之;于其身也,则耻师焉,惑矣。彼童子之师,授之书而习其句读者,非吾所谓传其道解其惑者也。句读之不知,惑之不解,或师焉,或不焉,小学而大遗,吾未见其明也。巫医乐师百工之人,不耻相师。士大夫之族,曰师曰弟子云者,则群聚而笑之。问之,则曰:“彼与彼年相若也,道相似也,位卑则足羞,官盛则近谀。”呜呼!师道之不复可知矣。巫医乐师百工之人,君子不齿,今其智乃反不能及,其可怪也欤!

圣人无常师。孔子师郯子、苌弘、师襄、老聃。郯子之徒,其贤不及孔子。孔子曰:“三人行,则必有我师”。是故弟子不必不如师,师不必贤于弟子。闻道有先后,术业有专攻,如是而已。

李氏子蟠,年十七,好古文,六艺经传皆通习之,不拘于时,学于余。余嘉其能行古道,作《师说》以贻之。{[}2{]}


\section{1.2   词句注释}
\label{\detokenize{p01_u6563_u6587/_u97e9_u6108-_u5e08_u8bf4:id4}}
1.学者:求学的人。

2.师者,所以传道受业解惑也:老师,是用来传授道理、交给学业、解释疑难问题的人。所以:用来……的。道:指儒家之道。受:通“授”,传授。业:泛指古代经、史、诸子之学及古文写作。惑:疑难问题。

3.人非生而知之者:人不是生下来就懂得道理。之:指知识和道理。《论语·季氏》:“生而知之者,上也;学而知之者,次也;困而学之,又其次之;困而不学,民斯为下矣。”知:懂得。

4.其为惑也:他所存在的疑惑。

5.生乎吾前:即生乎吾前者。乎:相当于“于”,与下文“先乎吾”的“乎”相同。

6.闻:听见,引申为知道,懂得。

7.从而师之:跟从(他),拜他为老师。从师:跟从老师学习。师:意动用法,以……为师。

8.吾师道也:我(是向他)学习道理。

9.夫庸知其年之先后生于吾乎:哪里去考虑他的年龄比我大还是小呢?庸:发语词,难道。知:了解、知道。

10.是故:因此,所以。

11.无:无论、不分。

12.道之所存,师之所存也:意思说哪里有道存在,哪里就有我的老师存在。

13.师道:从师的传统。即“古之学者必有师”。

14.出人:超出于众人之上。

15.犹且:尚且。

16.众人:普通人,一般人。

17..下:不如,名词作动词。

18.耻学于师:以向老师学习为耻。耻:以……为耻。

19.是故圣益圣,愚益愚:因此圣人更加圣明,愚人更加愚昧。益:更加、越发。

20.于其身:对于他自己。身:自身、自己。

21.惑矣:糊涂啊!

22.彼童子之师:那些教小孩子的启蒙老师。

23.授之书而习其句读(dòu):教给他书,帮助他学习其中的文句。之:指童子。习:使……学习。其:指书。句读:也叫句逗,古人指文辞休止和停顿处。文辞意尽处为句,语意未尽而须停顿处为读(逗)。古代书籍上没有标点,老师教学童读书时要进行句读(逗)的教学。

24.句读之不知:不知断句风逗。

25.或师焉,或不(fǒu)焉:有的从师,有的不从师。不:通“否”。

26.小学而大遗:学了小的(指“句读之不知”)却丢了大的(指“惑之不解”)。遗:丢弃,放弃。

27.巫医:古时巫、医不分,指以看病和降神祈祷为职业的人。

28.百工:各种手艺。

29.相师:拜别人为师。

30.族:类。

31.曰师曰弟子云者:说起老师、弟子的时候。

32.年相若:年岁相近。

33.位卑则足羞,官盛则近谀:以地位低的人为师就感到羞耻,以高官为师就近乎谄媚。足:可,够得上。盛:高大。谀:谄媚。

34.复:恢复。

35.君子:即上文的“士大夫之族”。

36.不齿:不屑与之同列,即看不起。或作“鄙之”。

37.乃:竟,竟然。

38.其可怪也欤:难道值得奇怪吗。其:难道,表反问。欤:语气词,表感叹。

39.圣人无常师:圣人没有固定的老师。常:固定的。

40.郯(tán)子:春秋时郯国(今山东省郯城县境)的国君,相传孔子曾向他请教官职。

41.苌(cháng)弘:东周敬王时候的大夫,相传孔子曾向他请教古乐。

42.师襄:春秋时鲁国的乐官,名襄,相传孔子曾向他学琴。

43.老聃(dān):即老子,姓李名耳,春秋时楚国人,思想家,道家学派创始人。相传孔子曾向他学习周礼。聃是老子的字。

44.之徒:这类。

45.三人行,则必有我师:三人同行,其中必定有我的老师。《论语·述而》原话:“子曰:‘三人行,必有我师焉。择其善者而从之,其不善者而改之。’”

46.不必:不一定。

47.术业有专攻:在业务上各有自己的专门研究。攻:学习、研究。

48.李氏子蟠(pán):李家的孩子名蟠。李蟠:韩愈的弟子,唐德宗贞元十九年(803)进士。

49.六艺经传(zhuàn)皆通习之:六艺的经文和传文都普遍的学习了。六艺:指六经,即《诗》《书》《礼》《乐》《易》《春秋》六部儒家经典。《乐》已失传,此为古说。经:两汉及其以前的散文。传,古称解释经文的著作为传。通:普遍。

50.不拘于时:指不受当时以求师为耻的不良风气的束缚。时:时俗,指当时士大夫中耻于从师的不良风气。于:被。

51.余嘉其能行古道:我赞许他能遵行古人从师学习的风尚。嘉:赞许,嘉奖。

52.贻(yí):赠送,赠予。{[}3-4{]}


\section{1.3   白话译文}
\label{\detokenize{p01_u6563_u6587/_u97e9_u6108-_u5e08_u8bf4:id5}}
古代求学的人一定有老师。老师,是可以依靠来传授道理、教授学业、解答疑难问题的。人不是生下来就懂得道理的,谁能没有疑惑?有了疑惑,如果不跟从老师学习,那些成为疑难问题的,就最终不能理解了。生在我前面,他懂得道理本来就早于我,我应该跟从他把他当作老师;生在我后面,如果他懂得的道理也早于我,我也应该跟从他把他当作老师。我是向他学习道理啊,哪管他的生年比我早还是比我晚呢?因此,无论地位高低贵贱,无论年纪大小,道理存在的地方,就是老师存在的地方。

唉,古代从师学习的风尚不流传已经很久了,想要人没有疑惑难啊!古代的圣人,他们超出一般人很远,尚且跟从老师而请教;现在的一般人,他们的才智低于圣人很远,却以向老师学习为耻。因此圣人就更加圣明,愚人就更加愚昧。圣人之所以能成为圣人,愚人之所以能成为愚人,大概都出于这吧?人们爱他们的孩子,就选择老师来教他,但是对于他自己呢,却以跟从老师学习为可耻,真是糊涂啊!那些孩子们的老师,是教他们读书,帮助他们学习断句的,不是我所说的能传授那些道理,解答那些疑难问题的。一方面不通晓句读,另一方面不能解决疑惑,有的句读向老师学习,有的疑惑却不向老师学习;小的方面倒要学习,大的方面反而放弃不学,我没看出那种人是明智的。巫医乐师和各种工匠这些人,不以互相学习为耻。士大夫这类人,听到称“老师”称“弟子”的,就成群聚在一起讥笑人家。问他们为什么讥笑,就说:“他和他年龄差不多,道德学问也差不多,以地位低的人为师,就觉得羞耻,以官职高的人为师,就近乎谄媚了。”唉!古代那种跟从老师学习的风尚不能恢复,从这些话里就可以明白了。巫医乐师和各种工匠这些人,君子们不屑一提,现在他们的见识竟反而赶不上这些人,真是令人奇怪啊!

圣人没有固定的老师。孔子曾以郯子、苌弘、师襄、老聃为师。郯子这些人,他们的贤能都比不上孔子。孔子说:“几个人一起走,其中一定有可以当我的老师的人。”因此学生不一定不如老师,老师不一定比学生贤能,听到的道理有早有晚,学问技艺各有专长,如此罢了。

李家的孩子蟠,年龄十七,喜欢古文,六经的经文和传文都普遍地学习了,不受时俗的拘束,向我学习。我赞许他能够遵行古人从师的途径,写这篇《师说》来赠送他。{[}5{]}


\section{1.4   创作背景}
\label{\detokenize{p01_u6563_u6587/_u97e9_u6108-_u5e08_u8bf4:id6}}
《师说》大约是作者于贞元十七年至十八年(801—802),在京任国子监四门博士时所作。贞元十七年(801),辞退徐州官职,闲居洛阳传道授徒的作者,经过两次赴京调选,方于当年十月授予国子监四门博士之职。此时的作者决心借助国子监这个平台来振兴儒教、改革文坛,以实现其报国之志。但来到国子监上任后,却发现科场黑暗,朝政腐败,吏制弊端重重,致使不少学子对科举入仕失去信心,因而放松学业;当时的上层社会,看不起教书之人。在士大夫阶层中存在着既不愿求师,又“羞于为师”的观念,直接影响到国子监的教学和管理。作者对此痛心疾首,借用回答李蟠的提问撰写这篇文章,以澄清人们在“求师”和“为师”上的模糊认识。{[}6{]}


\section{1.5   文学赏析}
\label{\detokenize{p01_u6563_u6587/_u97e9_u6108-_u5e08_u8bf4:id7}}
文中虽也正面论及师的作用、从师的重要性和以什么人为师等问题,但重点是批判当时流行于士大夫阶层中的耻于从师的不良风气。就文章的写作意图和主要精神看,这是一篇针对性很强的批驳性论文。

文章开头一段,先从正面论述师道:从师的必要性和从师的标准(以谁为师)。劈头提出“古之学者必有师”这个论断,紧接着概括指出师的作用:“传道受业解惑”,作为立论的出发点与依据。从“解惑”(道与业两方面的疑难)出发,推论人非生而知之者,不能无惑,惑则必从师的道理;从“传道”出发,推论从师即是学道,因此无论贵贱长幼都可为师,“道之所存,师之所存也”。这一段,层层顶接,逻辑严密,概括精练,一气呵成,在全文中是一个纲领。这一段的“立”,是为了下文的“破”。一开头郑重揭出“古之学者必有师”,就隐然含有对“今之学者”不从师的批判意味。势如风雨骤至,先声夺人。接着,就分三层从不同的侧面批判当时士大夫中流行的耻于从师的不良风气。先以“古之圣人”与“今之众人”作对比,指出圣与愚的分界就在于是否从师而学;再以士大夫对待自己的孩子跟对待自己在从师而学问题上的相反态度作对比,指出这是“小学而大遗”的糊涂作法;最后以巫医、乐师、百工不耻相师与士大夫耻于相师作对比,指出士大夫之智不及他们所不齿的巫医、乐师、百工。作者分别用“愚”、“惑”、“可怪”来揭示士大夫耻于从师的风气的不正常。由于对比的鲜明突出,作者的这种贬抑之辞便显得恰如其分,具有说服力。

在批判的基础上,文章又转而从正面论述“圣人无常师”,以孔子的言论和实践,说明师弟关系是相对的,凡是在道与业方面胜过自己或有一技之长的人都可以为师。这是对“道之所存,师之所存”这一观点的进一步论证,也是对士大夫之族耻于师事“位卑”者、“年近”者的现象进一步批判。

文章的最后一段,交待作这篇文章的缘由。李蟠“能行古道”,就是指他能继承久已不传的“师道”,乐于从师而学。因此这个结尾不妨说是借表彰“行古道”来进一步批判抛弃师道的今之众人。“古道”与首段“古之学者必有师”正遥相呼应。

在作者的论说文中,《师说》是属于文从字顺、平易畅达一类的,与《原道》一类豪放磅礴、雄奇桀傲的文章显然有别。但在平易畅达中仍贯注着一种气势。这种气势的形成,有多方面的因素。

首先是理论本身的说服力和严密的逻辑所形成的夺人气势。作者对自己的理论主张高度自信,对事理又有透彻的分析,因而在论述中不但步骤严密,一气旋折,而且常常在行文关键处用极概括而准确的语言将思想的精粹鲜明地表达出来,形成一段乃至一篇中的警策,给读者留下强烈深刻的印象。如首段在一路顶接,论述从师学道的基础上,结尾处就势作一总束:“是故无贵无贱,无长无少,道之所存,师之所存也。”大有如截奔马之势。“圣人无常师”一段,于举孔子言行为例之后,随即指出:“是故弟子不必不如师,师不必贤于弟子。闻道有先后,术业有专攻,如是而已。”从“无常师”的现象一下子引出这样透辟深刻的见解,有一种高瞻远瞩的气势。

其次是硬转直接,不作任何过渡,形成一种陡直峭绝的文势。开篇直书“古之学者必有师”,突兀而起,已见出奇;中间批判不良风气三小段,各以“嗟乎”、“爱其子”、“巫医、乐师、百工之人”发端,段与段问,没有任何承转过渡,兀然峭立,直起直落,了不相涉。这种转接发端,最为韩愈所长,读来自觉具有一种雄直峭兀之势。

此外,散体中参入对偶与排比句式,使奇偶骈散结合,也有助于加强文章的气势。{[}7{]}


\section{1.6   名家点评}
\label{\detokenize{p01_u6563_u6587/_u97e9_u6108-_u5e08_u8bf4:id8}}

\subsection{1.6.1   唐代}
\label{\detokenize{p01_u6563_u6587/_u97e9_u6108-_u5e08_u8bf4:id9}}
柳宗元《答韦中立论师道书》:孟子称人之患在好为人师。由魏晋氏以下,人益不事师。今之世不闻有师。有辄哗笑之以为狂人。独韩愈奋不顾流俗,犯笑侮,收召后学,作《师说》,抗颜而为师,世果群怪聚骂,指目牵引,而增与为言词,愈以是得狂名。又《答严厚舆论师道书》:言道讲古穷文辞以为师,则固吾属事。仆才能勇敢不如韩退之,故又不为人师。人之所见有异同,吾子无以韩责我。


\subsection{1.6.2   宋代}
\label{\detokenize{p01_u6563_u6587/_u97e9_u6108-_u5e08_u8bf4:id10}}
朱熹《朱子考异》:余观退之《师说》云:“弟子不必不如师,师不必贤于弟子。”其言非好为人师者也。学者不归子厚,归退之,故子厚有此说耳。

黄震《黄氏日抄》:前起后收,中排三节,皆以轻重相形。初以圣与愚相形,圣且从师,况愚乎?次以子与身相似,子且择师,况身乎?次以巫医、乐师、百工与士大夫相形,巫、乐、百工且从师,况士大夫乎?公之提诲后学,亦可谓深切著明矣。而文法则自然而成者也。


\subsection{1.6.3   元代}
\label{\detokenize{p01_u6563_u6587/_u97e9_u6108-_u5e08_u8bf4:id11}}
程端礼《昌黎文式》:此篇有诗人讽喻法,读之自知师道不可废。


\subsection{1.6.4   明代}
\label{\detokenize{p01_u6563_u6587/_u97e9_u6108-_u5e08_u8bf4:id12}}
茅坤《唐宋八大家文钞》:昌黎当时抗师道,以号召后辈,故为此倡赤帜云。


\subsection{1.6.5   清代}
\label{\detokenize{p01_u6563_u6587/_u97e9_u6108-_u5e08_u8bf4:id13}}
蔡世远《古文雅正》:师道立则善人多。汉世经学详明者,以师弟子相承故也。宋代理学昌明者,以师弟子相信故也。唐时知道者,独有一韩子,而当时又少肯师者,即如张文昌、李习之、皇甫持正,韩子得意弟子也,然诸人集中亦鲜推尊为师者,况其它乎?以此知唐时气习最重,故韩子痛切言之。唐学不及汉宋者,亦以此也。

储欣《唐宋十大家全集录·昌黎先生全集录》:题易迂,就浅处指点,乃无一点迂气。曾、王理学文,似未解此。又云:以眼前事指点化诲,使人易知,颇与《讳辩》一例。

孙琮《山晓阁选唐大家韩昌黎全集》:大意是欲李氏子能自得师,故一起提出师之为道,以下便说师无长幼贵贱,惟人自择。借写时人不肯从师,历引童子、巫医、孔子喻之,总是欲其能自得师。劝勉李氏子蟠,非是訾议世人。

爱新觉罗·玄烨《古文渊鉴》引洪迈:此文如常山蛇势,救首救尾,段段有力,学者宜熟读。

林云铭《韩文起》:其行文错综变化,反复引证,似无段落可寻。一气读之,只觉意味无穷。

吴楚材、吴调侯《古文观止》:通篇只是“吾师道也”一语,言触处皆师,无论长幼贵贱,惟人自择。因借时人不肯从师,历引童子、巫医、孔子喻之,总是欲李氏子能自得师,不必谓公慨然以师道自任,而作此以倡后学也。

张伯行《唐宋八大家文钞》:师者,师其道也,年之先后,位之尊卑,自不必论。彼不知求师者,曾百工之不若,乌有长进哉!《说命》篇曰:“德无常师。”朱子释之,以为天下之德,无一定之师,惟善是从。则凡有善者,皆可师,亦此意也。

方苞《方望溪先生全集·集外文·古文约选》:自“人非生而知之者”至“吾未见其明也”,言解惑。自“巫医乐师百工之人”至“如是而已”,言授业。而皆以传道贯之,盖舍授业无所谓传道也。

浦起龙《古文眉诠》:柳子谓韩子犯笑侮,收召后学,抗颜而为师,作《师说》,故知“师道不传”及“耻”“笑”等字,是著眼处。世不知古必有师,徒以为年不先我,以为不必贤于我,风俗人心,浇可知已。韩子见道于文,起衰八代,思得吾与,借李氏子发所欲言,不敢以告年长而自贤者,而私以告十七岁人,思深哉。

何焯《义门读书记》引李锺伦:“无贵无贱”,见不当挟贵;“无少无长”,见不当挟长;“圣人出人也远矣,犹且从师”,见不当挟贤。后即此三柱而申之。童子之师是年不相若者,引起世俗以年相若相师为耻;巫医、乐师、百工是无名位之人,引起世俗以官位不同相师为耻,而语势错综,不露痕也。


\chapter{1   Hi,p02读书}
\label{\detokenize{p02_u8bfb_u4e66/Hello_uff0cp02_u8bfb_u4e66:hi-p02}}\label{\detokenize{p02_u8bfb_u4e66/Hello_uff0cp02_u8bfb_u4e66::doc}}
\begin{sphinxShadowBox}
\sphinxstyletopictitle{目录}
\begin{itemize}
\item {} 
\phantomsection\label{\detokenize{p02_u8bfb_u4e66/Hello_uff0cp02_u8bfb_u4e66:id2}}{\hyperref[\detokenize{p02_u8bfb_u4e66/Hello_uff0cp02_u8bfb_u4e66:hi-p02}]{\sphinxcrossref{1   Hi,p02读书}}}
\begin{itemize}
\item {} 
\phantomsection\label{\detokenize{p02_u8bfb_u4e66/Hello_uff0cp02_u8bfb_u4e66:id3}}{\hyperref[\detokenize{p02_u8bfb_u4e66/Hello_uff0cp02_u8bfb_u4e66:post}]{\sphinxcrossref{1.1   post}}}

\end{itemize}

\end{itemize}
\end{sphinxShadowBox}


\section{1.1   post}
\label{\detokenize{p02_u8bfb_u4e66/Hello_uff0cp02_u8bfb_u4e66:post}}

\chapter{1   水浒-宋江之绰号}
\label{\detokenize{p02_u8bfb_u4e66/_u6c34_u6d52-_u5b8b_u6c5f_u4e4b_u7ef0_u53f7:id1}}\label{\detokenize{p02_u8bfb_u4e66/_u6c34_u6d52-_u5b8b_u6c5f_u4e4b_u7ef0_u53f7::doc}}
\begin{sphinxShadowBox}
\sphinxstyletopictitle{目录}
\begin{itemize}
\item {} 
\phantomsection\label{\detokenize{p02_u8bfb_u4e66/_u6c34_u6d52-_u5b8b_u6c5f_u4e4b_u7ef0_u53f7:id3}}{\hyperref[\detokenize{p02_u8bfb_u4e66/_u6c34_u6d52-_u5b8b_u6c5f_u4e4b_u7ef0_u53f7:id1}]{\sphinxcrossref{1   水浒-宋江之绰号}}}

\end{itemize}
\end{sphinxShadowBox}

宋江是《水浒传》里边名号最多的一个,共有四个。

第一个是黑宋江。
因为他长得面黑,身材比较矮,这是就他的形体来讲的,其貌不扬。

第二个是孝义黑三郎。
讲的是他对待父母,讲究孝道,他的孝道贯穿到了他的思想当中,成为他思想的一个部分,并且是他的思想的一个很重要的支撑点。

第三个是及时雨。
讲的是他仗义疏财,扶危济困,这在后面他陆续和弟兄们交往中能够看得出来。

第四个是呼保义。
这个词,一直到现在,大家都无法把它解释清楚。有一种解释说,保义是南宋时候武官的一个称呼,叫保义郎。“保义”本是宋代最低的武官名,逐渐成了人人可用的自谦之词。“呼保义”这个词是动宾结构,宋江以“自呼保义”来表示谦虚,意思是说,自己是最低等的人。另外一种解释,说“保”,就是保持的保;“义”就是忠义的义,“保义”即保持忠义,呼的意思,就是大家都那样叫他。大体上说,呼保义这个词实际上讲的是宋江对待国家的态度,对待朝廷的态度,对待皇帝的态度。水浒传里有云:“呼群保义”。


\chapter{1   Hi,p03旅游}
\label{\detokenize{p03_u65c5_u6e38/Hello_uff0cp03_u65c5_u6e38:hi-p03}}\label{\detokenize{p03_u65c5_u6e38/Hello_uff0cp03_u65c5_u6e38::doc}}
\begin{sphinxShadowBox}
\sphinxstyletopictitle{目录}
\begin{itemize}
\item {} 
\phantomsection\label{\detokenize{p03_u65c5_u6e38/Hello_uff0cp03_u65c5_u6e38:id2}}{\hyperref[\detokenize{p03_u65c5_u6e38/Hello_uff0cp03_u65c5_u6e38:hi-p03}]{\sphinxcrossref{1   Hi,p03旅游}}}
\begin{itemize}
\item {} 
\phantomsection\label{\detokenize{p03_u65c5_u6e38/Hello_uff0cp03_u65c5_u6e38:id3}}{\hyperref[\detokenize{p03_u65c5_u6e38/Hello_uff0cp03_u65c5_u6e38:post}]{\sphinxcrossref{1.1   post}}}

\end{itemize}

\end{itemize}
\end{sphinxShadowBox}


\section{1.1   post}
\label{\detokenize{p03_u65c5_u6e38/Hello_uff0cp03_u65c5_u6e38:post}}

\chapter{1   五泄瀑布}
\label{\detokenize{p03_u65c5_u6e38/_u4e94_u6cc4_u7011_u5e03:id1}}\label{\detokenize{p03_u65c5_u6e38/_u4e94_u6cc4_u7011_u5e03::doc}}
\begin{sphinxShadowBox}
\sphinxstyletopictitle{目录}
\begin{itemize}
\item {} 
\phantomsection\label{\detokenize{p03_u65c5_u6e38/_u4e94_u6cc4_u7011_u5e03:id3}}{\hyperref[\detokenize{p03_u65c5_u6e38/_u4e94_u6cc4_u7011_u5e03:id1}]{\sphinxcrossref{1   五泄瀑布}}}

\end{itemize}
\end{sphinxShadowBox}

五泄构成了天然的山水画卷,素有“小雁荡”之称。

当地人称瀑布为洩,一水折为五级,叫“五洩”,正称“五泄”。

月笼轻纱第一泄,
双龙争壑第二泄,
珠帘风动第三泄,
烈马奔腾第四泄,
蛟龙出海第五泄。

五泄从青口进入,古人云:“五泄名山青口锁,到此看山山便可”。沿公路前行,路旁曲溪青流,远处便是叠石岩。壁立数十丈,层层叠叠如彩屏。

再前便是“五泄湖”的水库,弯弯曲曲,长2公里许,犹似一条绿色的绸带飘浮在群山之中,颇有富春山水的风采。

在游船中还可以观赏许多奇特的山石景观,夹岩洞为其中一景。当年夹岩洞下有夹岩寺,香火较旺,水库建成后,寺庙成为水底龙宫。夹岩洞恰好位于湖面之上,洞高16米,深20米,内曾供奉千手观音,外观幽暗莫测,颇具神秘色彩。

沿湖还可以观赏元宝峰、鹫鹰峰、仙桃峰、老僧峰等。

在天一碧码头登岩后,沿五泄溪北上,过遇龙桥,就是五泄禅寺。

继续沿溪往前,不多远,过竹林,便是奔腾而下的第五泄。沿山势而上,依次四泄,三泄,二泄,一泄,逐次趋缓。风景各具。

脚劲不错,还可以继续向上,登上山顶观峡谷。

沿着深谷清溪,可以转回五泄禅寺。


\chapter{1   Hi,p04财经}
\label{\detokenize{p04_u8d22_u7ecf/Hello_uff0cp04_u8d22_u7ecf:hi-p04}}\label{\detokenize{p04_u8d22_u7ecf/Hello_uff0cp04_u8d22_u7ecf::doc}}
\begin{sphinxShadowBox}
\sphinxstyletopictitle{目录}
\begin{itemize}
\item {} 
\phantomsection\label{\detokenize{p04_u8d22_u7ecf/Hello_uff0cp04_u8d22_u7ecf:id2}}{\hyperref[\detokenize{p04_u8d22_u7ecf/Hello_uff0cp04_u8d22_u7ecf:hi-p04}]{\sphinxcrossref{1   Hi,p04财经}}}
\begin{itemize}
\item {} 
\phantomsection\label{\detokenize{p04_u8d22_u7ecf/Hello_uff0cp04_u8d22_u7ecf:id3}}{\hyperref[\detokenize{p04_u8d22_u7ecf/Hello_uff0cp04_u8d22_u7ecf:post}]{\sphinxcrossref{1.1   post}}}

\end{itemize}

\end{itemize}
\end{sphinxShadowBox}


\section{1.1   post}
\label{\detokenize{p04_u8d22_u7ecf/Hello_uff0cp04_u8d22_u7ecf:post}}

\chapter{1   Hi,p05技术}
\label{\detokenize{p05_u6280_u672f/Hello_uff0cp05_u6280_u672f:hi-p05}}\label{\detokenize{p05_u6280_u672f/Hello_uff0cp05_u6280_u672f::doc}}
\begin{sphinxShadowBox}
\sphinxstyletopictitle{目录}
\begin{itemize}
\item {} 
\phantomsection\label{\detokenize{p05_u6280_u672f/Hello_uff0cp05_u6280_u672f:id2}}{\hyperref[\detokenize{p05_u6280_u672f/Hello_uff0cp05_u6280_u672f:hi-p05}]{\sphinxcrossref{1   Hi,p05技术}}}
\begin{itemize}
\item {} 
\phantomsection\label{\detokenize{p05_u6280_u672f/Hello_uff0cp05_u6280_u672f:id3}}{\hyperref[\detokenize{p05_u6280_u672f/Hello_uff0cp05_u6280_u672f:post}]{\sphinxcrossref{1.1   post}}}

\end{itemize}

\end{itemize}
\end{sphinxShadowBox}


\section{1.1   post}
\label{\detokenize{p05_u6280_u672f/Hello_uff0cp05_u6280_u672f:post}}

\chapter{1   Hi,p06历史}
\label{\detokenize{p06_u5386_u53f2/Hello_uff0cp06_u5386_u53f2:hi-p06}}\label{\detokenize{p06_u5386_u53f2/Hello_uff0cp06_u5386_u53f2::doc}}
\begin{sphinxShadowBox}
\sphinxstyletopictitle{目录}
\begin{itemize}
\item {} 
\phantomsection\label{\detokenize{p06_u5386_u53f2/Hello_uff0cp06_u5386_u53f2:id2}}{\hyperref[\detokenize{p06_u5386_u53f2/Hello_uff0cp06_u5386_u53f2:hi-p06}]{\sphinxcrossref{1   Hi,p06历史}}}
\begin{itemize}
\item {} 
\phantomsection\label{\detokenize{p06_u5386_u53f2/Hello_uff0cp06_u5386_u53f2:id3}}{\hyperref[\detokenize{p06_u5386_u53f2/Hello_uff0cp06_u5386_u53f2:post}]{\sphinxcrossref{1.1   post}}}

\end{itemize}

\end{itemize}
\end{sphinxShadowBox}


\section{1.1   post}
\label{\detokenize{p06_u5386_u53f2/Hello_uff0cp06_u5386_u53f2:post}}

\chapter{1   瓦岗寨之李密}
\label{\detokenize{p06_u5386_u53f2/_u74e6_u5c97_u5be8_u4e4b_u674e_u5bc6:id1}}\label{\detokenize{p06_u5386_u53f2/_u74e6_u5c97_u5be8_u4e4b_u674e_u5bc6::doc}}
\begin{sphinxShadowBox}
\sphinxstyletopictitle{目录}
\begin{itemize}
\item {} 
\phantomsection\label{\detokenize{p06_u5386_u53f2/_u74e6_u5c97_u5be8_u4e4b_u674e_u5bc6:id3}}{\hyperref[\detokenize{p06_u5386_u53f2/_u74e6_u5c97_u5be8_u4e4b_u674e_u5bc6:id1}]{\sphinxcrossref{1   瓦岗寨之李密}}}

\end{itemize}
\end{sphinxShadowBox}


\chapter{1   Hi,p07创投}
\label{\detokenize{p07_u521b_u6295/Hello_uff0cp07_u521b_u6295:hi-p07}}\label{\detokenize{p07_u521b_u6295/Hello_uff0cp07_u521b_u6295::doc}}
\begin{sphinxShadowBox}
\sphinxstyletopictitle{目录}
\begin{itemize}
\item {} 
\phantomsection\label{\detokenize{p07_u521b_u6295/Hello_uff0cp07_u521b_u6295:id2}}{\hyperref[\detokenize{p07_u521b_u6295/Hello_uff0cp07_u521b_u6295:hi-p07}]{\sphinxcrossref{1   Hi,p07创投}}}
\begin{itemize}
\item {} 
\phantomsection\label{\detokenize{p07_u521b_u6295/Hello_uff0cp07_u521b_u6295:id3}}{\hyperref[\detokenize{p07_u521b_u6295/Hello_uff0cp07_u521b_u6295:post}]{\sphinxcrossref{1.1   post}}}

\end{itemize}

\end{itemize}
\end{sphinxShadowBox}


\section{1.1   post}
\label{\detokenize{p07_u521b_u6295/Hello_uff0cp07_u521b_u6295:post}}

\chapter{1   风投的前生}
\label{\detokenize{p07_u521b_u6295/_u98ce_u6295_u7684_u524d_u751f:id1}}\label{\detokenize{p07_u521b_u6295/_u98ce_u6295_u7684_u524d_u751f::doc}}
\begin{sphinxShadowBox}
\sphinxstyletopictitle{目录}
\begin{itemize}
\item {} 
\phantomsection\label{\detokenize{p07_u521b_u6295/_u98ce_u6295_u7684_u524d_u751f:id3}}{\hyperref[\detokenize{p07_u521b_u6295/_u98ce_u6295_u7684_u524d_u751f:id1}]{\sphinxcrossref{1   风投的前生}}}

\end{itemize}
\end{sphinxShadowBox}


\chapter{1   Hi,p08写作}
\label{\detokenize{p08_u5199_u4f5c/Hello_uff0cp08_u5199_u4f5c:hi-p08}}\label{\detokenize{p08_u5199_u4f5c/Hello_uff0cp08_u5199_u4f5c::doc}}
\begin{sphinxShadowBox}
\sphinxstyletopictitle{目录}
\begin{itemize}
\item {} 
\phantomsection\label{\detokenize{p08_u5199_u4f5c/Hello_uff0cp08_u5199_u4f5c:id2}}{\hyperref[\detokenize{p08_u5199_u4f5c/Hello_uff0cp08_u5199_u4f5c:hi-p08}]{\sphinxcrossref{1   Hi,p08写作}}}
\begin{itemize}
\item {} 
\phantomsection\label{\detokenize{p08_u5199_u4f5c/Hello_uff0cp08_u5199_u4f5c:id3}}{\hyperref[\detokenize{p08_u5199_u4f5c/Hello_uff0cp08_u5199_u4f5c:post}]{\sphinxcrossref{1.1   post}}}

\end{itemize}

\end{itemize}
\end{sphinxShadowBox}


\section{1.1   post}
\label{\detokenize{p08_u5199_u4f5c/Hello_uff0cp08_u5199_u4f5c:post}}

\chapter{1   Hi,p09work}
\label{\detokenize{p09work/Hello_uff0cp09work:hi-p09work}}\label{\detokenize{p09work/Hello_uff0cp09work::doc}}
\begin{sphinxShadowBox}
\sphinxstyletopictitle{目录}
\begin{itemize}
\item {} 
\phantomsection\label{\detokenize{p09work/Hello_uff0cp09work:id2}}{\hyperref[\detokenize{p09work/Hello_uff0cp09work:hi-p09work}]{\sphinxcrossref{1   Hi,p09work}}}
\begin{itemize}
\item {} 
\phantomsection\label{\detokenize{p09work/Hello_uff0cp09work:id3}}{\hyperref[\detokenize{p09work/Hello_uff0cp09work:post}]{\sphinxcrossref{1.1   post}}}

\end{itemize}

\end{itemize}
\end{sphinxShadowBox}


\section{1.1   post}
\label{\detokenize{p09work/Hello_uff0cp09work:post}}

\chapter{Indices and tables}
\label{\detokenize{index:indices-and-tables}}\begin{itemize}
\item {} 
\DUrole{xref,std,std-ref}{search}

\end{itemize}



\renewcommand{\indexname}{索引}
\printindex
\end{document}