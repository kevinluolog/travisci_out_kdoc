%% Generated by Sphinx.
\def\sphinxdocclass{report}
\documentclass[letterpaper,12pt,english]{sphinxmanual}
\ifdefined\pdfpxdimen
   \let\sphinxpxdimen\pdfpxdimen\else\newdimen\sphinxpxdimen
\fi \sphinxpxdimen=.75bp\relax
%% turn off hyperref patch of \index as sphinx.xdy xindy module takes care of
%% suitable \hyperpage mark-up, working around hyperref-xindy incompatibility
\PassOptionsToPackage{hyperindex=false}{hyperref}

\PassOptionsToPackage{warn}{textcomp}

\catcode`^^^^00a0\active\protected\def^^^^00a0{\leavevmode\nobreak\ }
\usepackage{cmap}
\usepackage{xeCJK}
\usepackage{amsmath,amssymb,amstext}
\usepackage{polyglossia}
\setmainlanguage{english}



\setCJKmainfont{FZWeiBei-S03}


\usepackage[Sonny]{fncychap}
\ChNameVar{\Large\normalfont\sffamily}
\ChTitleVar{\Large\normalfont\sffamily}
\usepackage{sphinx}

\fvset{fontsize=\small}
\usepackage{geometry}

% Include hyperref last.
\usepackage{hyperref}
% Fix anchor placement for figures with captions.
\usepackage{hypcap}% it must be loaded after hyperref.
% Set up styles of URL: it should be placed after hyperref.
\urlstyle{same}
\addto\captionsenglish{\renewcommand{\contentsname}{目录}}

\usepackage{sphinxmessages}
\setcounter{tocdepth}{0}


%中文字体fontsize放大,kl+
\defaultCJKfontfeatures{Scale=1.2}
\usepackage{enumitem}
\setlistdepth{99}


\title{Hi post}
\date{2019 年 11 月 07 日}
\release{}
\author{kevinluo}
\newcommand{\sphinxlogo}{\vbox{}}
\renewcommand{\releasename}{}
\makeindex
\begin{document}

\pagestyle{empty}
\sphinxmaketitle
\pagestyle{plain}
\sphinxtableofcontents
\pagestyle{normal}
\phantomsection\label{\detokenize{index::doc}}



\chapter{1   Hi,p00其它}
\label{\detokenize{p00_u5176_u5b83/Hello_uff0cp00_u5176_u5b83:hi-p00}}\label{\detokenize{p00_u5176_u5b83/Hello_uff0cp00_u5176_u5b83::doc}}
\begin{sphinxShadowBox}
\sphinxstyletopictitle{目录}
\begin{itemize}
\item {} 
\phantomsection\label{\detokenize{p00_u5176_u5b83/Hello_uff0cp00_u5176_u5b83:id2}}{\hyperref[\detokenize{p00_u5176_u5b83/Hello_uff0cp00_u5176_u5b83:hi-p00}]{\sphinxcrossref{1   Hi,p00其它}}}
\begin{itemize}
\item {} 
\phantomsection\label{\detokenize{p00_u5176_u5b83/Hello_uff0cp00_u5176_u5b83:id3}}{\hyperref[\detokenize{p00_u5176_u5b83/Hello_uff0cp00_u5176_u5b83:post}]{\sphinxcrossref{1.1   post}}}

\end{itemize}

\end{itemize}
\end{sphinxShadowBox}


\section{1.1   post}
\label{\detokenize{p00_u5176_u5b83/Hello_uff0cp00_u5176_u5b83:post}}

\chapter{1   《张英-聪训斋语》《张廷玉-澄怀园语》合辑}
\label{\detokenize{p00_u5176_u5b83/_u300a_u5f20_u82f1-_u806a_u8bad_u658b_u8bed_u300b_u300a_u5f20_u5ef7_u7389-_u6f84_u6000_u56ed_u8bed_u300b_u5408_u8f91:id1}}\label{\detokenize{p00_u5176_u5b83/_u300a_u5f20_u82f1-_u806a_u8bad_u658b_u8bed_u300b_u300a_u5f20_u5ef7_u7389-_u6f84_u6000_u56ed_u8bed_u300b_u5408_u8f91::doc}}
\begin{sphinxShadowBox}
\sphinxstyletopictitle{目录}
\begin{itemize}
\item {} 
\phantomsection\label{\detokenize{p00_u5176_u5b83/_u300a_u5f20_u82f1-_u806a_u8bad_u658b_u8bed_u300b_u300a_u5f20_u5ef7_u7389-_u6f84_u6000_u56ed_u8bed_u300b_u5408_u8f91:id15}}{\hyperref[\detokenize{p00_u5176_u5b83/_u300a_u5f20_u82f1-_u806a_u8bad_u658b_u8bed_u300b_u300a_u5f20_u5ef7_u7389-_u6f84_u6000_u56ed_u8bed_u300b_u5408_u8f91:id1}]{\sphinxcrossref{1   《张英-聪训斋语》《张廷玉-澄怀园语》合辑}}}
\begin{itemize}
\item {} 
\phantomsection\label{\detokenize{p00_u5176_u5b83/_u300a_u5f20_u82f1-_u806a_u8bad_u658b_u8bed_u300b_u300a_u5f20_u5ef7_u7389-_u6f84_u6000_u56ed_u8bed_u300b_u5408_u8f91:id16}}{\hyperref[\detokenize{p00_u5176_u5b83/_u300a_u5f20_u82f1-_u806a_u8bad_u658b_u8bed_u300b_u300a_u5f20_u5ef7_u7389-_u6f84_u6000_u56ed_u8bed_u300b_u5408_u8f91:id3}]{\sphinxcrossref{1.1   《张英-聪训斋语》}}}
\begin{itemize}
\item {} 
\phantomsection\label{\detokenize{p00_u5176_u5b83/_u300a_u5f20_u82f1-_u806a_u8bad_u658b_u8bed_u300b_u300a_u5f20_u5ef7_u7389-_u6f84_u6000_u56ed_u8bed_u300b_u5408_u8f91:id17}}{\hyperref[\detokenize{p00_u5176_u5b83/_u300a_u5f20_u82f1-_u806a_u8bad_u658b_u8bed_u300b_u300a_u5f20_u5ef7_u7389-_u6f84_u6000_u56ed_u8bed_u300b_u5408_u8f91:id4}]{\sphinxcrossref{1.1.1   张英简介}}}

\item {} 
\phantomsection\label{\detokenize{p00_u5176_u5b83/_u300a_u5f20_u82f1-_u806a_u8bad_u658b_u8bed_u300b_u300a_u5f20_u5ef7_u7389-_u6f84_u6000_u56ed_u8bed_u300b_u5408_u8f91:id18}}{\hyperref[\detokenize{p00_u5176_u5b83/_u300a_u5f20_u82f1-_u806a_u8bad_u658b_u8bed_u300b_u300a_u5f20_u5ef7_u7389-_u6f84_u6000_u56ed_u8bed_u300b_u5408_u8f91:id5}]{\sphinxcrossref{1.1.2   有之四纲十二目如下:}}}

\item {} 
\phantomsection\label{\detokenize{p00_u5176_u5b83/_u300a_u5f20_u82f1-_u806a_u8bad_u658b_u8bed_u300b_u300a_u5f20_u5ef7_u7389-_u6f84_u6000_u56ed_u8bed_u300b_u5408_u8f91:id19}}{\hyperref[\detokenize{p00_u5176_u5b83/_u300a_u5f20_u82f1-_u806a_u8bad_u658b_u8bed_u300b_u300a_u5f20_u5ef7_u7389-_u6f84_u6000_u56ed_u8bed_u300b_u5408_u8f91:id6}]{\sphinxcrossref{1.1.3   【正文】张英-聪训斋语:}}}

\end{itemize}

\item {} 
\phantomsection\label{\detokenize{p00_u5176_u5b83/_u300a_u5f20_u82f1-_u806a_u8bad_u658b_u8bed_u300b_u300a_u5f20_u5ef7_u7389-_u6f84_u6000_u56ed_u8bed_u300b_u5408_u8f91:id20}}{\hyperref[\detokenize{p00_u5176_u5b83/_u300a_u5f20_u82f1-_u806a_u8bad_u658b_u8bed_u300b_u300a_u5f20_u5ef7_u7389-_u6f84_u6000_u56ed_u8bed_u300b_u5408_u8f91:id7}]{\sphinxcrossref{1.2   《张廷玉-澄怀园语》}}}
\begin{itemize}
\item {} 
\phantomsection\label{\detokenize{p00_u5176_u5b83/_u300a_u5f20_u82f1-_u806a_u8bad_u658b_u8bed_u300b_u300a_u5f20_u5ef7_u7389-_u6f84_u6000_u56ed_u8bed_u300b_u5408_u8f91:id21}}{\hyperref[\detokenize{p00_u5176_u5b83/_u300a_u5f20_u82f1-_u806a_u8bad_u658b_u8bed_u300b_u300a_u5f20_u5ef7_u7389-_u6f84_u6000_u56ed_u8bed_u300b_u5408_u8f91:id8}]{\sphinxcrossref{1.2.1   张廷玉简介}}}

\item {} 
\phantomsection\label{\detokenize{p00_u5176_u5b83/_u300a_u5f20_u82f1-_u806a_u8bad_u658b_u8bed_u300b_u300a_u5f20_u5ef7_u7389-_u6f84_u6000_u56ed_u8bed_u300b_u5408_u8f91:id22}}{\hyperref[\detokenize{p00_u5176_u5b83/_u300a_u5f20_u82f1-_u806a_u8bad_u658b_u8bed_u300b_u300a_u5f20_u5ef7_u7389-_u6f84_u6000_u56ed_u8bed_u300b_u5408_u8f91:id9}]{\sphinxcrossref{1.2.2   【卷一】}}}

\item {} 
\phantomsection\label{\detokenize{p00_u5176_u5b83/_u300a_u5f20_u82f1-_u806a_u8bad_u658b_u8bed_u300b_u300a_u5f20_u5ef7_u7389-_u6f84_u6000_u56ed_u8bed_u300b_u5408_u8f91:id23}}{\hyperref[\detokenize{p00_u5176_u5b83/_u300a_u5f20_u82f1-_u806a_u8bad_u658b_u8bed_u300b_u300a_u5f20_u5ef7_u7389-_u6f84_u6000_u56ed_u8bed_u300b_u5408_u8f91:id10}]{\sphinxcrossref{1.2.3   【卷二】}}}

\item {} 
\phantomsection\label{\detokenize{p00_u5176_u5b83/_u300a_u5f20_u82f1-_u806a_u8bad_u658b_u8bed_u300b_u300a_u5f20_u5ef7_u7389-_u6f84_u6000_u56ed_u8bed_u300b_u5408_u8f91:id24}}{\hyperref[\detokenize{p00_u5176_u5b83/_u300a_u5f20_u82f1-_u806a_u8bad_u658b_u8bed_u300b_u300a_u5f20_u5ef7_u7389-_u6f84_u6000_u56ed_u8bed_u300b_u5408_u8f91:id11}]{\sphinxcrossref{1.2.4   【卷三】}}}

\item {} 
\phantomsection\label{\detokenize{p00_u5176_u5b83/_u300a_u5f20_u82f1-_u806a_u8bad_u658b_u8bed_u300b_u300a_u5f20_u5ef7_u7389-_u6f84_u6000_u56ed_u8bed_u300b_u5408_u8f91:id25}}{\hyperref[\detokenize{p00_u5176_u5b83/_u300a_u5f20_u82f1-_u806a_u8bad_u658b_u8bed_u300b_u300a_u5f20_u5ef7_u7389-_u6f84_u6000_u56ed_u8bed_u300b_u5408_u8f91:id12}]{\sphinxcrossref{1.2.5   【卷四】}}}

\item {} 
\phantomsection\label{\detokenize{p00_u5176_u5b83/_u300a_u5f20_u82f1-_u806a_u8bad_u658b_u8bed_u300b_u300a_u5f20_u5ef7_u7389-_u6f84_u6000_u56ed_u8bed_u300b_u5408_u8f91:id26}}{\hyperref[\detokenize{p00_u5176_u5b83/_u300a_u5f20_u82f1-_u806a_u8bad_u658b_u8bed_u300b_u300a_u5f20_u5ef7_u7389-_u6f84_u6000_u56ed_u8bed_u300b_u5408_u8f91:id13}]{\sphinxcrossref{1.2.6   【附录】:}}}
\begin{itemize}
\item {} 
\phantomsection\label{\detokenize{p00_u5176_u5b83/_u300a_u5f20_u82f1-_u806a_u8bad_u658b_u8bed_u300b_u300a_u5f20_u5ef7_u7389-_u6f84_u6000_u56ed_u8bed_u300b_u5408_u8f91:id27}}{\hyperref[\detokenize{p00_u5176_u5b83/_u300a_u5f20_u82f1-_u806a_u8bad_u658b_u8bed_u300b_u300a_u5f20_u5ef7_u7389-_u6f84_u6000_u56ed_u8bed_u300b_u5408_u8f91:id14}]{\sphinxcrossref{1.2.6.1   【清朝名相张廷玉的祖上,平价粜米,周济穷苦的故事】}}}

\end{itemize}

\end{itemize}

\end{itemize}

\end{itemize}
\end{sphinxShadowBox}


\section{1.1   《张英-聪训斋语》}
\label{\detokenize{p00_u5176_u5b83/_u300a_u5f20_u82f1-_u806a_u8bad_u658b_u8bed_u300b_u300a_u5f20_u5ef7_u7389-_u6f84_u6000_u56ed_u8bed_u300b_u5408_u8f91:id3}}

\subsection{1.1.1   张英简介}
\label{\detokenize{p00_u5176_u5b83/_u300a_u5f20_u82f1-_u806a_u8bad_u658b_u8bed_u300b_u300a_u5f20_u5ef7_u7389-_u6f84_u6000_u56ed_u8bed_u300b_u5408_u8f91:id4}}
张英(1637—1708),字敦复,一字梦敦,号乐圃,又号倦圃翁,安徽桐城人,清朝人物,清代著名大臣张廷玉之父。

据《桐城县志》记载,康熙时期文华殿大学士兼礼部尚书张英的老家人与邻居吴家在宅基地问题上发生了争执,家人飞书京城,让张英打招呼“摆平”吴家。而张英回馈给老家人的是一首诗“一纸书来只为墙,让他三尺又何妨。长城万里今犹在,不见当年秦始皇。”家人见书,主动在争执线上退让了三尺,下垒建墙,而邻居吴氏也深受感动,退地三尺,建宅置院,六尺之巷因此而成。

在安徽安庆,流传着这样说法:“父子宰相府”、“五里三进士”、“隔河两状元”,指的是张英家庭。张英的儿子是大名鼎鼎的张廷玉,热播的影视剧《康熙大帝》、《康熙王朝》和《雍正王朝》中都有他的重要身影。张廷玉(1672-1755)为康熙时进士,官至保和殿大学士、军机大臣,乾隆时加太保,为官康、雍、乾三代,历半个世纪宝刀不老,为二千年封建官场之罕见。他有这样的官场作为,应该说是他得益于父辈、祖辈淡泊致远、克己清廉的家风。六尺巷在父辈那里宽了六尺,而在他的心胸中又宽了万丈,“心底无私天地宽”,无私的心胸因此坦荡而无垠!

张英、张廷玉父子是安徽省著名历史人物,二人在清初康、雍、乾盛世居官数十年,参与了平藩、收台湾、征漠北、摊丁入亩、改土归流、编棚入户等一系列大政方针的制订和实行。对稳定当时政局,统一国家,消弭满汉矛盾,强盛国计民生都起到了积极而重要的作用。二人为官清廉,人品端方,均官至一品大学士,是历史上著名的贤臣良相。同时二人还是史家公认的学者大儒。

在张家官运的背后是康雍乾三世,他们是清代有作为的皇帝,在有作为的皇帝身边溜须拍马,邀宠作奸是没有市场的,特别是雍正皇帝,为政不长,却厉行政改,一生勤于国政,“崇俭而不奢”,“毫无土木声色之娱”,张廷玉记录雍正:“上进膳,承命侍食,见一于饭颗并屑,未尝废置纤毫!!”饭粒落于桌上也不舍弃!在张家高官的背后,是威严自律的帝国皇帝。

当时的清王朝尽管帝王自律而有作为,但对汉人仍提防有加,防汉人颠覆政权,大兴文字狱,高官厚爵们也伴君如侍虎,如履薄冰。张家人低调屈身,也成自然,据载,张廷玉之子张若霭殿试得一甲第三名(探花),张廷玉跪求雍正换人,以留得名额给天下平民英才,因为张家已太多出人头地的机会了。雍正深为感动,将其子降级任用,可见张家谦卑公允之心,昭昭可鉴日月。

张英,张廷玉父子,均为清代名臣,位居宰相,安徽桐城人,张家在当时举业不断,名宦迭出,在京城、乡里誉称四起,如:“父子双宰相”、“三世得谥”、“六代翰林”、“自祖至玄十二人先后列侍从,跻鼎贵。玉堂谱里,世系蝉联,门阀之清华,殆可空前绝后而已”、“一门之内,祖父子孙先后相继入南书房,自康熙至乾隆,经数十年之久,此他氏所未有也”。影响之大,震惊朝野。张氏之所以如此兴盛,重要原因之一就是有良好的家训家风的教诲和熏陶。研读他们的家训,无疑对治家教子有重要的借鉴作用。

【另注】:《聪训斋语》的作者张英,清安徽桐城人,1637年出生,1708年去世。字敦复,号乐圃。康熙六年考上进士,授编修官,历升至文华殿大学士兼礼部尚书。居官勤俭谨慎,对民生疾苦、四方水旱知无不言,深获皇上倚重。曾受命总裁《清一统志》、《国史方略》、《渊鉴类函》、《政治典训》等书。其他许多典诰之文,亦尝出其手。生平酷好看山种树,以老病辞官,卒諡文端。著有《恒产琐言》、《聪训斋语》,谆谆以务本力田、随分知足告诫子弟,另有《易书衷论》、《笃素堂文集》等书。


\subsection{1.1.2   有之四纲十二目如下:}
\label{\detokenize{p00_u5176_u5b83/_u300a_u5f20_u82f1-_u806a_u8bad_u658b_u8bed_u300b_u300a_u5f20_u5ef7_u7389-_u6f84_u6000_u56ed_u8bed_u300b_u5408_u8f91:id5}}
一立品纲——戒嬉戏、慎威仪、谨言语。

二读书纲——温经书、精举业、学楷字。

三养身纲——谨起居、慎寒暑、节用度。

四择友纲——谢酬应、省宴集、寡交游。


\subsection{1.1.3   【正文】张英-聪训斋语:}
\label{\detokenize{p00_u5176_u5b83/_u300a_u5f20_u82f1-_u806a_u8bad_u658b_u8bed_u300b_u300a_u5f20_u5ef7_u7389-_u6f84_u6000_u56ed_u8bed_u300b_u5408_u8f91:id6}}
人心至灵至动,不可过劳,亦不可过逸,惟读书可以养之。书卷乃养心第一妙物。闲适无事之人,镇日不观书,则起居出入,身心无所栖泊,耳目无所安顿,势必心意颠倒,妄想生嗔。处逆境不乐,处顺境亦不乐。每见人栖栖皇皇,觉举动无不碍者,此必不读书之人也。

富贵贫贱,总难称意,知足即为称意;山水花竹,无恒主人,得闲便是主人。大约富贵人役于名利,贫贱人役于饥寒,总无闲情及此,惟付之浩叹耳。

古人以“眠、食”二者为养生之要务。脏腑肠胃,常令宽舒有余地,则真气得以流行而疾病少。“予从饱食,病安得入?”燔炙熬煎香甘肥腻之物,最悦口而不宜于肠胃。彼肥腻易于粘滞,积久则腹痛气塞,寒暑偶侵,则疾作矣。食忌多品,一席之间,遍食水陆,浓淡杂进,自然损脾;安寝,乃人生最乐,古人有言:不觅仙方觅睡方。冬夜以二鼓为度,暑月以一更为度。每笑人长夜酣饮不休,谓之消夜,夫人终日劳劳,夜则宴息,是极有味,何以消遣为?冬夏,皆当以日出而起,于夏尤宜。天地清旭之气,最为爽神,失之,甚为可惜。予山居颇闲,暑月,日出则起,收水草清香之味,莲方敛而未开,竹含露而犹滴,可谓至快!日长漏永,不妨午睡数刻,睡足而起,神清气爽;居家最宜早起,倘日高客至,僮则垢面,婢且蓬头,庭除未扫,灶突犹寒,大非雅事。

人家僮仆,最多不宜多畜,但有得力二三人,训谕有方,使令得宜,未尝不得兼人之用。太多则彼此相诿,恩养必不能周,教训亦不能及,反不得其力;吾辈居家居宦,皆简静守理,不为暗昧之事;山中耕田锄圃之仆,乃可为宝,其人无奢望,无机智,不为主人敛怨,彼纵不遵约束,不过懒惰、愚蠢之小过,不必加意防闲,岂不为清闲之一助哉?

俭于饮食,可以养脾胃;俭于嗜欲,可以聚精神;俭于言语,可以养气息非;俭于交游,可以择友寡过;俭于酬酢,可以养身息劳;俭于夜坐,可以安神舒体;俭于饮酒,可以清心养德;俭于思虑,可以蠲烦去扰;白香山诗云:“我有一言君记取,世间自取苦人多。”;人常和悦,则心气冲而五脏安,昔人所谓养欢喜神。日间办理公事,每晚家居,必寻可喜笑之事,与客纵谈,掀髯大笑,以发舒一日劳顿郁结之气;砚以世计,墨以时计,笔以日计,动静之分也。静之义有二:一则身不过劳,一则心不轻动。

万事做到极精妙处,无有不圆者。人之一身,与天时相应,大约三四十以前,是夏至前,凡事渐长;三四十以后,是夏至后,凡事渐衰,中间无一刻停留。中间盛衰关头,无一定时候,大概在三四十之间,观于须发可见:其衰缓者,其寿多;其衰急者,其寿寡。人身不能不衰,先从上而下者,多寿,故古人以早脱顶为寿征,先从下而上者,多不寿,故须发如故而脚软者难治;凡人家道亦然,决无中立之理,如一树之花,开到极盛,便是摇落之期。(注:家道是否如此,不论,爱后面一句)

予怪世人于古人诗文集不知爱,而宝其片纸只字,为大惑也。余昔在龙眠,苦于无客为伴,日则步于空潭碧涧、长松茂竹之侧,夕则掩关读苏陆诗,以二鼓为度,烧烛焚香,煮茶延两君子于坐,与之相对,如见其容貌须眉然。诗云:“架头苏陆有遗书,特地携来共索居。日与两君同卧起,人间何客得胜渠。”(渠:他)良非解嘲语也。

门无杂宾,大约门下奔走之客,有损无益。

人生适意之事有三:曰贵,曰富,曰多子孙。然是三者,善处之则为富,不善处之则足为累。高位者,责备之地,忌嫉之门,怨尤之府,利害之关,忧患之窟,劳苦之薮,谤讪之的,攻击之场,古之智人往往望而止步;夫人厚积则必经营布置,生息防守,其劳不可胜言:则必有亲戚之请求,贫穷之怨望,僮仆之奸骗,大而盗贼之劫取,小而穿窬之鼠窃,经商之亏折,行路之失脱,田禾之灾伤,攘夺之争讼,子弟之浪费。种种之苦,贫者不知,惟富厚者兼而有之。人能各富之为累,则取之当廉,而不必厚积以招怨;至子孙之累尤多矣,少小则有疾病之虑,稍长则有功名之虑,浮奢不善治家之虑,纳交匪类之虑,一离膝下,则有道路寒暑饥渴之虑,以至由子而孙,展转无穷,更无底止。

予之立训,更无多言,止有四语:读书者不贱,守田者不饥,积德者不倾,择交者不败。虽至寒苦之人,但能读书为文,必使人钦敬,不敢忽视。其人德性,亦必温和,行事决不颠倒,不在功名之得失,遇合之迟速也。

人生必厚重沉静,而后为载福之器。敦厚谦谨,慎言守礼,不可与寒士同一般感慨欷嘘,放言高论,怨天尤人,庶不为造物鬼神所呵责也。

乡里间荷担负贩及佣工小人,切不可取其便宜,此种人所争不过数文,我辈视之甚轻,而彼之含怨甚重。每有愚人见省得一文,以为得计,而不知此种人心忿口碑,所损实大也。待下我一等之人,言语辞气最为要紧,此事甚不费钱,然彼人受之,同于实惠,只在精神照料得来,不可惮烦。

读书固所以取科名,继家声,然亦使人敬重;每见仕宦显赫之家,其老者或退或故,而其家索然者,其后无读书之人也,其家郁然者,其后有读书之人也;父母之爱子,第一望其康宁,第二冀其成名,第三愿其保家。《语》曰:“父母惟其疾之忧。”夫子以此答武伯之问孝,至哉斯言!安其身以安父母之心,孝莫大焉。养身之道,一在谨嗜欲,一在慎饮食,一在慎忿怒,一在慎寒暑,一在慎思索,一在慎烦劳。吾贻子孙,不过瘠田数处耳,且甚荒芜不治,水旱多虞。岁入之数,谨足以免饥寒,畜妻子而已,一件儿戏事做不得,一件高兴事做不得;人生豪侠周密之名至不易副。事事应之,一事不应,遂生嫌怨,人人周之,一人不周,便存形迹,若平素俭啬,见谅于人,省无穷物力,少无穷嫌怨,不亦至便乎?。

人生二十内外,渐远于师保之严,未跻于成人之列,此时知识大开,性情未定,父师之训不能入,即妻子之言亦不听,惟朋友之言,甘如醴而芳若兰,脱有一淫朋匪友,阑入其侧,朝夕浸灌,鲜有不为其所移者;(坏)朋友,则直以不识其颜面,不知其姓名为善。比之毒草哑泉更当远避。

楷书如坐如立,行书如行,草书如奔。

法昭禅师偈云:“同气连枝各自荣,些些言语各伤情。一回相见一回老,能得几时为弟兄?”词意蔼然,足以启人友于之爱。然予尝谓人伦有五,而兄弟相处之日最长。

世人只因不知命,不安命,生出许多劳扰;(君子)修身以俟之(指机遇);注:安命则心安言诚,有一颗平常心,反而事事办得更好。

余家训有云:“保家莫如择友。”盖痛心疾首其言之也!汝辈但于至戚中,观其德性谨厚,好读书者,交友两三人足矣!且势利言之,则有酒食之费、应酬之扰,一遇婚丧有无,则有资给贷之事。甚至有争讼外侮,则又有关说救援之事。平昔既与之契密,临事却之,必生怨毒反唇。故余以为宜慎之于始也;昔人有戒:“饭不嚼便咽,路不看便走,话不想便说,事不思便做。”予益之曰:“友不择便交,气不忍不便动,财不审便取,衣不慎便脱。”

学字当专一。择古人佳帖或时人墨迹与已笔路相近者,专心学之,若朝更夕改,见异思迁,鲜有得成者。若体格不匀净而遽讲流动,失其本矣!学字忌飞动草率,大小不匀,而妄言奇古磊落,终无进步矣。

读文不必多,择其精纯条畅,有气局词华者,多则百篇,少则六十篇。神明与之浑化,始为有益。若贪多务博,过眼辄忘,及至作时,则彼此不相涉,落笔仍是故吾,所以思常窒而不灵,词常窘而不裕,意常枯而不润。

人能处心积虑,一言一动皆思益人,而痛戒损人,则人望之若鸾凤,宝之如参苓。必为天地所佑,鬼神之所服,而享有多福矣!

凡读书,二十岁以前所读之书与二十岁以后所读之书迥异。幼年知识未开,天真纯固,所读者虽久不温习,偶尔提起,尚可数行成诵。若壮年所读,经月则忘,必不能持久。故六经、秦汉之文,词语古奥,必须幼年读。长壮后,虽倍蓰其功,终属影响。

自八岁至二十岁,中间岁月无多,安可荒弃或读不急之书?此时,时文固不可不读,亦须择典雅醇正、理纯辞裕、可历二三十年无弊者读之。若朝华夕落、浅陋无识、诡僻失体、取悦一时者,安可以珠玉难换之岁月而读此无益之文?何如诵得《左》、《国》一两篇及东西汉典贵华腴之文数篇,为终身之用之宝乎?

古人之书,安可尽读?但我所已读者决不轻弃。得尺则尺,得寸则寸。毋贪多,毋贪名,但求读一篇,必可以背诵。然后思通其义蕴,而运用之于手腕之下,如此则才气自然发越。若曾读此书,而全不能举其词,谓之“画饼充饥”。能举其词而不能运用,谓之“食物不化”。

深恼人读时文累千累百而不知理会,于身心毫无裨益。夫能理会,则数十篇百篇已足,焉用如此之多?不能理会,则读数千篇与不读一字等。徒使精神聩乱,临文捉笔,依旧茫然,不过胸中旧套应副,安有名理精论、佳词妙句,奔汇于笔端乎?古人云:“读生文不如玩熟文。必以我之精神,包乎此一篇之外,以我之心思,入乎此一篇之中。幼年当专攻举业,以为立身之本。

世家子弟,其修行立名之难,较寒士百倍。何以故?人之当面待之者,万不能如寒士之古道:小有失检,谁肯面斥其非?微有骄盈,谁肯深规其过?幼而骄惯,为亲戚之所优容;长而习成,为朋友之所谅恕.

我愿汝曹常以席丰履盛为可危、可虑、难处、难全之地,勿以为可喜、可幸、易安、易逸之地;终身让路,不失尺寸,自古祗闻“忍”与“让”,足以消无穷之灾悔,未闻“忍”与“让”,翻以酿后来之祸患也,欲行忍认之道,先须从小事做起。余曾署刑部事五十日,见天下大讼大狱,多从极小事起。君子敬小慎微,凡事只从小处了。余行年五十余,生平未尝多受小人之侮,只有一善策,能转弯早耳。每思天下事,受得小气,则不至于受大气,吃得小亏,则不至于吃大亏,此生平得力之处。凡事最不可想占便宜,便宜者,天下人所共争也,我一人据之,则怨萃于我矣,我失便宜,则众怨消矣。故终身失便宜,乃终身得便宜也。

座右箴:立品、读书、养身、择友。右四纲。戒嬉戏,慎威仪;谨言语,温经书;精举业,学楷字;谨起居,慎寒暑;节用度,谢酬;省宴集,寡交游。右十二目。

子弟自十七八以至廿三四,实为学业成废之关。盖自初入学至十五六,父师以童子视之,稍知训子者,断不忍听其废业。惟自十七八以后,年渐长,气渐骄,渐有朋友,渐有室家,嗜欲渐广。父母见其长成,师傅视为侪辈。德性未坚,转移最易;学业未就,蒙昧非难。幼年所习经书,此时皆束高阁。酬应交游,侈然大雅。博弈高会,自诩名流。转盼廿五六岁,儿女累多,生计迫蹙,蹉跎潦倒,学殖荒落。予见人家子弟半途而废者,多在此五六年中,弃幼学之功,贻终身之累,盖辙相踵也。汝正当此时,离父母之侧,前言诸弊,事事可虑。为龙为蛇,为虎为鼠,分于一念,介在两歧,可不慎哉!可不畏哉!

读书须明窗净几,案头不可多置书;作文以握管之人为大将,以精熟墨卷百篇为练兵,以杂读时艺为散卒。

天子知俭,则天下足,一人知俭,则一家足。且俭非止节啬财用己也。俭于言语,则元气藏而怨尤寡;则于交游,则匪类远,俭于酬酢,则岁月宽而本业修,俭于书札,则后患寡,俭于嬉游,则学业进;人生俭啬之名,可受而不必避,世俗每以为耻,不知此名一噪,则人绝觊觎之想。偶有所用,人即德之;保家莫如择友,多则二人,少则一人,断无目前良友,遂可得十数人之理!平时既简于应酬,有事可以请教。

惟田产房屋二者可恃以久远,以二者较之,房舍又不如田产。

今人家子弟,鲜衣怒马,恒舞酣歌。一裘之费动至数十金,一席之费动至数金。不思吾乡十余年来谷贱,竭十余石谷,不足供一筵,竭百余石谷,不足供一衣。安知农家作苦,终年沾衣涂足,岂易得此百石?(今天依然如此)

古人之意,全在小处节俭,大处之不足,由于小处之不谨,月计之不足,由于每日之用过多也。

子弟有二三千金之产,方能城居。若千金以下之业,则断不可城居矣!

古人有言,扫地焚香,清福已具。其有福者,佐以读书;其无福者,便生他想。旨哉斯言,予所深赏!且从来拂意之事,自不读书者见之,似为我所独遭,极其难堪,不知古人拂意之事有百倍于此者,特不细心体验耳! 即如东坡先生,殁后遭逢高孝,文字始出,而当时之忧谗畏讥,困顿转徙潮惠之间,苏过跣足涉水,居近牛栏,是何如境界?又如白香山之无嗣,陆放翁之忍饥,皆载在书卷,彼独非千载闻人,而所遇皆如此? 诚一平心静观,则人间拂意之事,可以涣然冰释。若不读书,则但见我所遭甚苦,而无穷怨尤嗔忿之心,烧灼不宁,其苦为何如耶?且富盛之事,古人亦有之,炙手可热,转眼皆空。故读书可以增长道心,为颐养第一事也!

【附录】:


\section{1.2   《张廷玉-澄怀园语》}
\label{\detokenize{p00_u5176_u5b83/_u300a_u5f20_u82f1-_u806a_u8bad_u658b_u8bed_u300b_u300a_u5f20_u5ef7_u7389-_u6f84_u6000_u56ed_u8bed_u300b_u5408_u8f91:id7}}

\subsection{1.2.1   张廷玉简介}
\label{\detokenize{p00_u5176_u5b83/_u300a_u5f20_u82f1-_u806a_u8bad_u658b_u8bed_u300b_u300a_u5f20_u5ef7_u7389-_u6f84_u6000_u56ed_u8bed_u300b_u5408_u8f91:id8}}
张廷玉(1672年10月29日—1755年4月30日),字衡臣,号砚斋,安徽桐城人。清朝杰出政治家,大学士张英次子。

康熙三十九年(1700年)进士,改庶吉士,授检讨,入值南书房,进入权力中枢。康熙朝,官至刑部左侍郎,整饬吏治。雍正帝即位后,历任礼部尚书、户部尚书、吏部尚书,拜保和殿大学士(内阁首辅)、首席军机大臣等职,完善了军机处制度。乾隆帝即位后,君臣渐生嫌疑,晚景凄凉,致仕归家。乾隆二十年(1755年),卒于家中,年八十四,谥号“文和”,配享太庙,是整个清朝唯一一个配享太庙的汉臣。

张廷玉先后任《亲征平定朔北方略》纂修官,《省方盛典》、《清圣祖实录》副总裁官,《明史》、《四朝国史》、《大清会典》、《世宗实录》总裁官。


\subsection{1.2.2   【卷一】}
\label{\detokenize{p00_u5176_u5b83/_u300a_u5f20_u82f1-_u806a_u8bad_u658b_u8bed_u300b_u300a_u5f20_u5ef7_u7389-_u6f84_u6000_u56ed_u8bed_u300b_u5408_u8f91:id9}}
一、凡人得一爱重之物,必思置之善地,以保护之。至于心,乃吾心之至宝也,一念善,是即置之安处矣;一念恶,是即置之危地矣。奈何以吾身之至宝使之舍安而就危乎?亦弗思之甚矣。

二、一语而干天地之和,一事而折生平之福,当时时留心体察,不可于细微处忽之。

三、昔我文端公时时以知命之学训子孙,晏闲之时则诵论语曰:不知命,无以为君子也。盖穷通得失,天命既定,人岂能违?彼营营扰扰,趋利避害者,徒劳心力坏品行耳,究何能增减毫末哉!先兄宫詹公,习闻庭训,是以主试山左,即以不知命一节为题,惜乎能觉悟之人少也。

四、周易曰:吉人之辞寡,可见多言之人即为不吉,不吉则凶矣。趋吉避凶之道只在矢口间,朱子云:祸从口出。此言与周易相表里,黄山谷曰:万言万当,不如一默。当终身诵之。

五、一言一动,常思有益于人,惟恐有损于人。不惟积德,亦是福相。

六、文端公对联曰:万类相感,以诚造物,最忌者巧。又曰:保家莫如择友,求名莫如读书。姚端恪公对联曰:常觉胸中生意满,须知世上苦人多。又虚直斋日记曰:我心有不快,而以戾气加人可乎?我事有未暇,而以缓人之急可乎?均当奉为座右铭。

七、天下之道,宽则能容,能容则物安,而己亦适。虽然宽之道亦难言矣,天下岂无有用宽而养奸贻患者乎?大抵内宽而外严,则庶几矣。

八、凡人病殁之后,其子孙家人往往以为庸医误投方药之所致,甚至有衔恨终身者。余尝笑曰:何其视我命太轻,而视医者之权太重若此耶。庸医用药差误,不过使病体缠绵,多延时日,不能速痊耳。若病至不起前数已定,虽卢扁岂能为功,乃归咎于庸医用药之不善不亦寃哉?

九、世之有心计者,每行一事,必思算无遗策,夫使犹有遗策则多算,何为不过招刻薄之名,致众人怨恨而已。若果算无遗策,则上犯造物之怒,其为不祥莫大焉。

十、凡事当极不好处宜向好处想,当极好处宜向不好处想。

十一、人生荣辱进退皆有一定之数,宜以义命自安。

十二、为善所以端品行也,谓为善必获福,则亦尽有不获福者。譬如文字好,则中式世,亦岂无好文而不中者耶,但不可因好文不中,而遂不作好文耳。

十三、制行愈高,品望愈重,则人之伺之益密,而论之亦愈深。防检稍疏则声名俱损。

十四、凡事贵慎密,而国家之事尤不当轻向人言,观古人温室树可见,总之真神仙必不说上界事,其轻言祸福者,皆师巫邪术,惑世欺人之辈耳。

十五、同居共事则猜忌易生也,至于与我不相干涉之人,闻其有如意之事,而中心怅怅,闻有不如意之事,而喜谈乐道之,此皆忌心为之也。余观天下之人,坐此病者甚多,时时省察防闲,恐蹈此薄福之相,惟我俩先人忠厚仁慈,出于天性,每闻人忧戚患难之事,即愀然不快于心,只此一念,便为人情之所难,而贻子孙之福于无穷矣。

十六、古人以盛满为戒。尚书曰:世禄之家,鲜克由礼。盖席丰履厚,其心易于放逸,而又无端人正士、严师益友为之督责,匡救无怪乎流而不返也。譬如一器贮水,盈满虽置于安稳之地,尚虑有倾溢之患,若置之欹侧之地,又从而摇撼之。不但水至倾覆,即器亦不可保矣。处盛满而不知谨慎者,何以异是。

十七、吾人进德修业,未有不静而能有成者。太极图说曰:圣人定之以中正仁义而主静。大学曰:静而后能安,安而后能虑,且不独学问之道为然也,历观天下享遐龄膺厚福之人,未有不静者,静之时义大矣哉!

十八、人生乐事如宫室之美、妻妾之奉、服饰之鲜、饮馔之丰洁、声技之靡丽,其为适意者,皆在外者也,而心之乐不乐不与也。惟有安分循理,不愧不怍,梦魂恬适,神气安闲,斯为吾心之真乐。彼富贵之人,穷施极欲,而心常戚戚,日夕忧虞者,吾不知其乐果何在也。

十九、凡人耳目听睹大率相同,若能神闲气静,则觉有异人处。

二十、余近蒙圣恩赐以广厦名园,深愧过分,昔文端公官宗伯时,屋止数楹,其后洊(cun)登台辅,数十年不易一椽,不增一瓦,曰:安敢为久远计耶?其谨如此,其俭如此,其刻刻求退如此,我后人岂可不知此意,而犹存见少之思耶?

二十一、大聪明人当困心衡虑之后,自然识见倍增,谨之又谨,慎之又慎,与其于放言高论中求乐境,何如于谨言慎行中求乐境耶?

二十二、人臣奉职惟以公正自守,毁誉在所不计,盖毁誉皆出于私心,我不肯徇人之私,则宁受人毁,不可受人誉也。

二十三、他山石曰:万病之毒皆生于浓,浓于声色生虚怯病,浓于货利生贪饕(tao)病,浓于功业生造作病,浓于名誉生矫激玻吾一味药解之曰:淡,吁斯言,诚药石哉!

二十四、人以不可行之事来求我,我直指其不可而谢绝之。彼必怫然不乐,然早断其妄念,亦一大阴德也。若犹豫含糊,使彼妄生觊觎或更以此得罪,此最造孽。人之精神力量,必使有余于事,而后不为事所苦,如饮酒者,能饮十杯,只饮八杯,则其量宽,然后有余,若饮十五杯则不能胜矣。

二十五、处顺境则退一步想,处逆境则进一步想。

二十六、为官第一要廉,养廉之道莫如能忍。尝记姚和修之言曰:有钱用钱,无钱用命。人能拼命强忍,不受非分之财,则于为官之道思过半矣。

二十七、人之葬坟,所以安先人也。葬后子孙昌盛,可以卜先人坟地之吉祥。若先存发福之心以求吉地,则不可。货悖而入者亦悖而出,平生锱铢必较,用尽心计以求赢余,造物嫉之,必使之用若泥沙以自罄其所有,夫劳苦而积之于平时,欢忻鼓舞而散之于一旦,则贪财果何所为耶?所以古人非道非义一介不龋

二十八、人家子弟承父祖之余荫,不能克家,而每好声伎,好古玩。好声伎者及身必败,好古玩者未有传及两世者,余见此多矣,故深以为戒。

二十九、昔人以论孟二语合成一联云:约失之鲜矣,诚乐莫大焉。余时佩服此十字。

三十、君子可欺以其方,若终身不被人欺,此必无之事。倘自谓人不能欺我,此至愚之见,即受欺之本也。

三十一、天下有学问、有识见、有福泽之人未有不静者。

三十二、天下矜才使气之人,一遇挫折,倍觉人所难堪,细思之,未必非福。

三十三、凡人好为翻案之论、好为翻案之文,是其胸襟褊浅处,即其学问偏僻处。孔子曰:中庸,不可能也。请看一部论语,何曾有一句新奇之语。

三十四、不深知“知人论世”四字之意,不可以读史。


\subsection{1.2.3   【卷二】}
\label{\detokenize{p00_u5176_u5b83/_u300a_u5f20_u82f1-_u806a_u8bad_u658b_u8bed_u300b_u300a_u5f20_u5ef7_u7389-_u6f84_u6000_u56ed_u8bed_u300b_u5408_u8f91:id10}}
一、居官清廉乃分内之事,每见清官多刻且盛气凌人,盖其心以清为异人能,是犹未忘乎货贿之见也,至诚而不动者,未之有也。问如何著力,曰:言忠信,行笃敬。

二、孝昌程封翁汉舒笔记曰:人看得自己重,方能有耻。又曰:人世得意事,我觉得可耻,亦非易事。此有道之言也。

三、读《论语》觉得《孟子》太繁且甚费力,读《孟子》又觉诸子之书费力矣。不可不知。

四、孝昌程封翁汉舒曰:一家之中,老幼男女无一个规矩礼法,虽眼前兴旺,即此便是衰败景象。又曰:小小智巧用惯了,便入于下流。而不觉此二语乃治家训子弟之药石也。

五、凡人看得天下事太容易,由于未曾经历也。待人好为责备之论,由于声在局外人也。恕之一字,圣贤从天性中来,中人以上者则阅历而后得之,资秉庸暗者虽经阅历,而梦梦如初矣。

六、注解古人诗文者,每牵合附会以示淹博,是一大玻古人用事用意,有可以窥测者,有不可窥测者,若必欲强勉著笔,恐差之毫厘失之千里,不可不慎也。

七、欧阳公论诗曰:状难写之景如在目前,含不尽之意见于言外,然后为工。此数语,看来浅近,而义蕴深长,得诗家之三味矣。

八、忧患皆从富贵中来,阅历久而后知之。

九、不虞之誉少,而求全之毁多,此人心厚薄所由分也。

十、孔子曰:如有所誉者,其有所试矣。是圣人之心宁偏于厚,其异于常人者正在此。

十一、开卷有益,此古今不易之理。犹记吾友姚别峰有诗曰:掩书微笑破疑团。尤得开卷有益自然而然之乐境也。余深爱之。

十二、韩魏公遗事曰:公判京兆,日得侄孙书云,田产多为邻近侵占,欲经官陈理,公于书尾题诗一首云,“他人侵我且从伊,子细思量未有时,试上含光殿基看,秋风秋草正离离”,其后子孙繁衍,历华要者不可胜数,以其宽大之德致然也。先文端公日以逊让训子孙,《聪训斋语》往复数千言,剀切缠绵即是此意,从今日观之,从前让人无纤毫污损,而子孙荣显,颇为海内所推,孰非积德累仁之报哉!

十三、欧阳文忠公之子名发,述公事迹,有曰,公奉敕撰唐书,专成纪志表,而列传则宋公祁所撰。朝廷恐其体不一,诏公看详,令删为一体,公虽受命,退而曰,宋公于我为前辈,且各人所见不同,岂可悉如己意?于是一无所易。余览之为之三叹,每见读书人于他人著作,往往恣意吹求以炫己长,至于意见不同则坚执己见,百折不回,此等习气,虽贤者不免,览欧公遗事其亦知古人之忠厚固如是乎!

十四、盖天下之乐,莫乐于闲且静,果能领会此二字,不但有自适之趣,即治事读书必志气清明,精神完足,无障碍亏缺处。若日事笙歌,喧哗杂处,神志渐就昏惰,事务必至废弛,多费又其余事也。至于蓄优人于家,则更不可,此等轻儇(xuan)佻达之辈,日与子弟家人相处,渐染仿效,默夺潜移,日流于匪僻,其害有不可胜言者。余居京师久,见富贵家之蓄优人者或数年或数十年或一再传,而后必至家规荡弃、生计衰微,百不爽一。人情孰不为子孙计而乃图一时之娱乐,则后人无穷之患,不亦重可叹哉!

十五、邵康节尝诵希夷之语曰:得便宜事不可再作,得便宜处不可再去。又曰:落便宜处是得便宜。故康节诗云:珍重至人常有语,落便宜事得便宜。元遗山诗曰:得便宜处落便宜,木石痴儿自不知。此语常人皆能言之,而实能领会其意者,非见道最深之人,不足以语此也,余不敏,愿终身诵之。


\subsection{1.2.4   【卷三】}
\label{\detokenize{p00_u5176_u5b83/_u300a_u5f20_u82f1-_u806a_u8bad_u658b_u8bed_u300b_u300a_u5f20_u5ef7_u7389-_u6f84_u6000_u56ed_u8bed_u300b_u5408_u8f91:id11}}
一、凡人于极得意极失意时,能检点言语无过当之辞,其人之学问气量必有大过人处。

二、乐道人之善,恶称人之恶。皆出于论语,可作书室对联,触目警心也。

三、明儒吕叔简先生坤曰:家人之害,莫大于卑幼各恣其无厌之情,而上之人阿其意,而不之禁;尤莫大于婢子造言,而妇人悦之,妇人附会而丈夫信之。禁此二害,而家不和睦者,鲜矣。又曰:今人骨肉之好不终,只为看得尔我二字太分晓。此二(瑕)语虽浅近,实居家之药石也。

四、董华亭宗伯曰:结千百人之欢,不如释一人之怨。余曰此长厚之言也。

五、山东曹县吕道人不知其年,问之亦不以实告,大约在百龄内外,善养生修炼之术,鹤发童颜,步履矍铄,终日不食亦不饥,顶心出香气,如麝檀硫磺。然此子亲见者以针(zhen)砭为人疗病,辄效赠以财物不受。曰:天下之物哪一件是我的?人曰:聊以表吾心耳。答曰:天下之物哪一件是你的?此二语予最爱之,可以警觉天下之贪取妄求而不知止足者。凡人度量广大、不嫉妒、不猜疑,乃己身享福之相,于人无所损益也。纵生性不能如此,亦当勉强而行之。彼幸灾乐祸之人,不过自成其薄福之相耳,于人又何损乎?不可不发深剩

六、吾乡左忠毅公举乡试,谒本房陈公大绶,陈勉以树立,却红柬不受,谓曰:今日行事节俭,即异日做官清,不就此站定脚跟,后难措手。呜呼,不矜细行,终累大德,前辈之谨小慎微如此,彼后生小子生富贵之家、染纨绔之习,何足以知之?

七、朱子口铭曰:病从口入,祸从口出。此语人人知之,且病与祸人人所恶也,而能致谨于入口出口之际者盖寡,则能忍之难也。书曰:必有忍,其乃有济。武王书铭曰:忍之须臾,乃全汝躯。昔人诗曰:忍过事堪喜。忍之时义大矣哉!

八、余五十年来留心默识彼语言不实之辈,一时可以欺世,而究竟飘荡于终身。凤鉴书所谓到老终无结果也,若怀私挟怨捏造蜚语害人名节身家者,厥后必有恶报,以予所见,屈指而数,未可以为天道渺茫,在可知不可知之间也。

九、武侯戒子书曰:君子之行,静以修身,俭以养德,非淡泊无以明志,非宁静无以致远。夫学须静也,才须学也,非学无以广才,非静无以成学。怠慢则不能研精,险躁则不能理性。予尝以静字训子弟,今再益以静以修身,学须静也二语,其中义蕴精微,非大有识见人不能理会。

十、孟子曰:予岂好辩哉?予不得已也!吾人必深知孟子不得已之苦衷,方可以读孟子,不然则书中可疑可议者,不可胜数也。

十一、邵康节诗曰:静处乾坤大,闲中日月长。“闲中日月长”人所知也,“静处乾坤大”人或未知也。予一生好静,于此中颇有领会,奈此身牵于职守,日在红尘扰攘中,常为设想曰,若能改静处为闹处,则有进步矣,惜乎其不能也。


\subsection{1.2.5   【卷四】}
\label{\detokenize{p00_u5176_u5b83/_u300a_u5f20_u82f1-_u806a_u8bad_u658b_u8bed_u300b_u300a_u5f20_u5ef7_u7389-_u6f84_u6000_u56ed_u8bed_u300b_u5408_u8f91:id12}}
一、隐恶扬善,圣人也;好善恶恶,贤人也;分别善恶无当者,庸人也;颠倒善恶以快其谗,谤者小人也。赴大机者速断,成大功者善藏,同时中庸,而君子小人之别微也哉!

二、予少时夜卧,难以成寐,既寐之后,一闻声息即醒。先兄宫詹公授以引睡之法,背读上论语数页或十数页,使心有所寄。予试之,果然。后推广其意,诵渊明诗“采菊东篱下,悠然见南山”或钱考功诗“曲终人不见,江上数峰青”或陆放翁诗“小楼一夜听春雨,深巷明朝卖杏花”,皆古人潇洒闲适之句,神游其境,往往睡去。盖心不可有著,又不可一无所著也。

三、薛文清曰:多言最使人心志流荡,而气亦损,少言不惟养得德深,又养得气完。

四、陈眉公曰:颐卦,慎言语,节饮食。然口之所入,其祸小;口之所出,其罪多。故鬼谷子云:口可以饮,不可以言。又曰:圣人之言简,贤人之言明,众人之言多,小人之言妄。

五、伊川先生曰:只观发言之平易,躁妄便见,德之厚薄所养之深浅见。人论前辈之短曰:汝辈且取他长处。薛文清公曰:在古人之后,议古人之失,则易处古人之位;为古人之位,为古人之事则难。此处不可不深剩

六、陆士衡豪士赋云:身危由于势过,而不知去势以求安;祸积由于宠盛,而不知辞宠以招福。此富贵人之通病也。

七、李之彦曰:尝玩钱字,旁上著一戈字,下著一戈字,真杀人之物也,然则两戈争贝,岂非贱乎?

八、陈眉公曰:醉人胆大,与酒融洽故也。人能与义命融治,浩然之气自然充塞,何惧之有?

九、象山先生曰:学者不长进,只是好己胜,出一言,做一事,便道全是,岂有此理。古人惟贵知过则改,见善则迁。今各执己是,被人点破便愕然所以,不如古人先生。此言乃天下学者之通病,若能不蹈此病,则其天资识量过人远矣。倘见此而能省察悔悟,将来亦必有所成就。

十、古人云:教子之道有五,静其性,广其志,养其材,鼓其气,攻其病,废一不可。


\subsection{1.2.6   【附录】:}
\label{\detokenize{p00_u5176_u5b83/_u300a_u5f20_u82f1-_u806a_u8bad_u658b_u8bed_u300b_u300a_u5f20_u5ef7_u7389-_u6f84_u6000_u56ed_u8bed_u300b_u5408_u8f91:id13}}

\subsubsection{1.2.6.1   【清朝名相张廷玉的祖上,平价粜米,周济穷苦的故事】}
\label{\detokenize{p00_u5176_u5b83/_u300a_u5f20_u82f1-_u806a_u8bad_u658b_u8bed_u300b_u300a_u5f20_u5ef7_u7389-_u6f84_u6000_u56ed_u8bed_u300b_u5408_u8f91:id14}}
(一)独力办施济,慈德谦光。

财之为物,生不带来,死不带去,传与子孙,无一不败。故智慧之人,当乘有权在手,有钱可施之时,广作利人利物功德,使个个金钱,造成未来胜福。

明末时候,桐城地方有一个张老员外,存心慈善,喜欢施舍。有一年,遇着荒歉,米价腾贵,一般奸利的商人看到这情形,反把米粮囤积着不肯出售,于是平民大起恐慌。官府里请命办账,又是迂回曲折的不能立见施行。员外看了这情形,很是忧急。他家里有存米万石,这时便自动的举行平粜,照市价减半出售,每人每日,限购一升,以防奸人套买图利。平民听到这消息,欢喜若狂。员外又想到一般赤贫的人无钱买米,仍在挨饿,于是又办了一个施粥厂,受施的人隔日领券,统计了人数煮着大量的粥,按券发给,一日三餐,每餐白粥一大碗,咸菜一小碟。许多人枵腹而来,鼓腹而去,大家都称颂员外是个活菩萨。员外很谦虚的说:“荒年米价贵,减半出售,已和平时全价相等,所以我也没有什么损失,至于施粥,也所费有限。总之,只要大家有饭吃,我就很觉安慰了。”

(二)夫妇商典质,同德同心。

世间之财,水能漂没之,火能烧毁之,盗贼能劫掠之,官吏能没收之,不肖儿孙能消败之,故称不坚之财,惟用以利人,可以后福无量。

老员外接连的办理平粜施粥,家里的钱渐渐赔完了,但是荒歉的现象一时不能平复,自己的善事不能半途中止,因此十分心焦。他想,我这时若把救济事业停止了,一般贫民便有饿死的可能,我当初的救济不是和不救济一样了嘛?救人须救到底。现在我还有一部分产业存在,何不变卖了继续办理!想定了主意,便到内室去和夫人商量。他的夫人也是十分贤德的,听了他的话,极端赞成,并且说:“积产业给子孙,没有积德,子孙不肖,就是金山银山也要倒的。若是积德给子孙,虽没有财产,将来子孙好,也会富裕起来的。田地房屋,由相公做主变卖,就是我有许多首饰衣服也卖了吧!”员外听了,额手称赞道:“夫人的话,说得真有理!”于是变卖产业,继续善事,直到荒歉的现象消除了才止。

(三)簪缨绵世泽,善报无差。

祖宗积德,后嗣发福,考之历史,验之现在,百不失一,于此可以证实因果之不虚。读者宜注意,得益无量。

老员外故世后,到了第五代孙子张英,做到宰相之职,张英的儿子张廷玉,也继续着父亲的地位,成为清朝盛世时的著名宰相。以后子孙,累代显荣,时享官禄。

有人说:“老员外死时,有个异人指点着一块好风水的葬地,所以子孙富贵。”这件事的真假,不必细论,总之,员外这样积德,子孙自然应该发达,若是好风水,也要心地好,才能得到呢。


\chapter{1   张英-聪训斋语}
\label{\detokenize{p00_u5176_u5b83/_u5f20_u82f1-_u806a_u8bad_u658b_u8bed:id1}}\label{\detokenize{p00_u5176_u5b83/_u5f20_u82f1-_u806a_u8bad_u658b_u8bed::doc}}
\begin{sphinxShadowBox}
\sphinxstyletopictitle{目录}
\begin{itemize}
\item {} 
\phantomsection\label{\detokenize{p00_u5176_u5b83/_u5f20_u82f1-_u806a_u8bad_u658b_u8bed:id5}}{\hyperref[\detokenize{p00_u5176_u5b83/_u5f20_u82f1-_u806a_u8bad_u658b_u8bed:id1}]{\sphinxcrossref{1   张英-聪训斋语}}}
\begin{itemize}
\item {} 
\phantomsection\label{\detokenize{p00_u5176_u5b83/_u5f20_u82f1-_u806a_u8bad_u658b_u8bed:id6}}{\hyperref[\detokenize{p00_u5176_u5b83/_u5f20_u82f1-_u806a_u8bad_u658b_u8bed:id3}]{\sphinxcrossref{1.1   有之四纲十二目如下:}}}

\item {} 
\phantomsection\label{\detokenize{p00_u5176_u5b83/_u5f20_u82f1-_u806a_u8bad_u658b_u8bed:id7}}{\hyperref[\detokenize{p00_u5176_u5b83/_u5f20_u82f1-_u806a_u8bad_u658b_u8bed:id4}]{\sphinxcrossref{1.2   原文}}}

\end{itemize}

\end{itemize}
\end{sphinxShadowBox}


\section{1.1   有之四纲十二目如下:}
\label{\detokenize{p00_u5176_u5b83/_u5f20_u82f1-_u806a_u8bad_u658b_u8bed:id3}}
一 立品纲——戒嬉戏、慎威仪、谨言语。

二 读书纲——温经书、精举业、学楷字。

三 养身纲——谨起居、慎寒暑、节用度。

四 择友纲——谢酬应、省宴集、寡交游。


\section{1.2   原文}
\label{\detokenize{p00_u5176_u5b83/_u5f20_u82f1-_u806a_u8bad_u658b_u8bed:id4}}
人心至灵至动,不可过劳,亦不可过逸,惟读书可以养之。书卷乃养心第一妙物。闲适无事之人,镇日不观书,则起居出入,身心无所栖泊,耳目无所安顿,势必心意颠倒,妄想生嗔。处逆境不乐,处顺境亦不乐。每见人栖栖皇皇,觉举动无不碍者,此必不读书之人也。

富贵贫贱,总难称意,知足即为称意;山水花竹,无恒主人,得闲便是主人。大约富贵人役于名利,贫贱人役于饥寒,总无闲情及此,惟付之浩叹耳。

古人以“眠、食”二者为养生之要务。脏腑肠胃,常令宽舒有余地,则真气得以流行而疾病少。“予从不饱食,病安得入?”燔炙熬煎香甘肥腻之物,最悦口而不宜于肠胃。彼肥腻易于粘滞,积久则腹痛气塞,寒暑偶侵,则疾作矣。食忌多品,一席之间,遍食水陆,浓淡杂进,自然损脾;安寝,乃人生最乐,古人有言:不觅仙方觅睡方。冬夜以二鼓为度,暑月以一更为度。每笑人长夜酣饮不休,谓之消夜,夫人终日劳劳,夜则宴息,是极有味,何以消遣为?冬夏,皆当以日出而起,于夏尤宜。天地清旭之气,最为爽神,失之,甚为可惜。予山居颇闲,暑月,日出则起,收水草清香之味,莲方敛而未开,竹含露而犹滴,可谓至快!日长漏永,不妨午睡数刻,睡足而起,神清气爽;居家最宜早起,倘日高客至,僮则垢面,婢且蓬头,庭除未扫,灶突犹寒,大非雅事。

人家僮仆,最多不宜多畜,但有得力二三人,训谕有方,使令得宜,未尝不得兼人之用。太多则彼此相诿,恩养必不能周,教训亦不能及,反不得其力;吾辈居家居宦,皆简静守理,不为暗昧之事;山中耕田锄圃之仆,乃可为宝,其人无奢望,无机智,不为主人敛怨,彼纵不遵约束,不过懒惰、愚蠢之小过,不必加意防闲,岂不为清闲之一助哉?

俭于饮食,可以养脾胃;俭于嗜欲,可以聚精神;俭于言语,可以养气息非;俭于交游,可以择友寡过;俭于酬酢,可以养身息劳;俭于夜坐,可以安神舒体;俭于饮酒,可以清心养德;俭于思虑,可以蠲烦去扰;白香山诗云:“我有一言君记取,世间自取苦人多。”;人常和悦,则心气冲而五脏安,昔人所谓养欢喜神,日间办理公事,每晚家居,必寻可喜笑之事,与客纵谈,掀髯大笑,以发舒一日劳顿郁结之气;砚以世计,墨以时计,笔以日计,动静之分也。静之义有二:一则身不过劳,一则心不轻动。

万事做到极精妙处,无有不圆者。人之一身,与天时相应,大约三四十以前,是夏至前,凡事渐长;三四十以后,是夏至后,凡事渐衰,中间无一刻停留。中间盛衰关头,无一定时候,大概在三四十之间,观于须发可见:其衰缓者,其寿多;其衰急者,其寿寡。人身不能不衰,先从上而下者,多寿,故古人以早脱顶为寿征,先从下而上者,多不寿,故须发如故而脚软者难治;凡人家道亦然,决无中立之理,如一树之花,开到极盛,便是摇落之期。
予怪世人于古人诗文集不知爱,而宝其片纸只字,为大惑也。余昔在龙眠,苦于无客为伴,日则步于空潭碧涧、长松茂竹之侧,夕则掩关读苏陆诗,以二鼓为度,烧烛焚香,煮茶延两君子于坐,与之相对,如见其容貌须眉然。诗云:“架头苏陆有遗书,特地携来共索居。日与两君同卧起,人间何客得胜渠。”良非解嘲语也。

门无杂宾,大约门下奔走之客,有损无益。
人生适意之事有三:曰贵,曰富,曰多子孙。然是三者,善处之则为富,不善处之则足为累。高位者,责备之地,忌嫉之门,怨尤之府,利害之关,忧患之窟,劳苦之薮,谤讪之的,攻击之场,古之智人往往望而止步;夫人厚积则必经营布置,生息防守,其劳不可胜言:则必有亲戚之请求,贫穷之怨望,僮仆之奸骗,大而盗贼之劫取,小而穿窬之鼠窃,经商之亏折,行路之失脱,田禾之灾伤,攘夺之争讼,子弟之浪费。种种之苦,贫者不知,惟富厚者兼而有之。人能各富之为累,则取之当廉,而不必厚积以招怨;至子孙之累尤多矣,少小则有疾病之虑,稍长则有功名之虑,浮奢不善治家之虑,纳交匪类之虑,一离膝下,则有道路寒暑饥渴之虑,以至由子而孙,展转无穷,更无底止。

予之立训,更无多言,止有四语:读书者不贱,守田者不饥,积德者不倾,择交者不败。虽至寒苦之人,但能读书为文,必使人钦敬,不敢忽视。其人德性,亦必温和,行事决不颠倒,不在功名之得失,遇合之迟速也。

人生必厚重沉静,而后为载福之器。敦厚谦谨,慎言守礼,不可与寒士同一般感慨欷嘘,放言高论,怨天尤人,庶不为造物鬼神所呵责也。
乡里间荷担负贩及佣工小人,切不可取其便宜,此种人所争不过数文,我辈视之甚轻,而彼之含怨甚重。每有愚人见省得一文,以为得计,而不知此种人心忿口碑,所损实大也。待下我一等之人,言语辞气最为要紧,此事甚不费钱,然彼人受之,同于实惠,只在精神照料得来,不可惮烦;读书固所以取科名,继家声,然亦使人敬重;每见仕宦显赫之家,其老者或退或故,而其家索然者,其后无读书之人也,其家郁然者,其后有读书之人也;父母之爱子,第一望其康宁,第二冀其成名,第三愿其保家。《语》曰:“父母惟其疾之忧。”夫子以此答武伯之问孝,至哉斯言!安其身以安父母之心,孝莫大焉。养身之道,一在谨嗜欲,一在慎饮食,一在慎忿怒,一在慎寒暑,一在慎思索,一在慎烦劳。吾贻子孙,不过瘠田数处耳,且甚荒芜不治,水旱多虞。岁入之数,谨足以免饥寒,畜妻子而已,一件儿戏事做不得,一件高兴事做不得;人生豪侠周密之名至不易副。事事应之,一事不应,遂生嫌怨,人人周之,一人不周,便存形迹,若平素俭啬,见谅于人,省无穷物力,少无穷嫌怨,不亦至便乎?;人生二十内外,渐远于师保之严,未跻于成人之列,此时知识大开,性情未定,父师之训不能入,即妻子之言亦不听,惟朋友之言,甘如醴而芳若兰,脱有一淫朋匪友,阑入其侧,朝夕浸灌,鲜有不为其所移者;(坏)朋友,则直以不识其颜面,不知其姓名为善。比之毒草哑泉更当远避。

楷书如坐如立,行书如行,草书如奔。
法昭禅师偈云:“同气连枝各自荣,些些言语各伤情。一回相见一回老,能得几时为弟兄?”词意蔼然,足以启人友于之爱。然予尝谓人伦有五,而兄弟相处之日最长。

世人只因不知命,不安命,生出许多劳扰;君子修身以俟之。
余家训有云:“保家莫如择友。”盖痛心疾首其言之也!汝辈但于至戚中,观其德性谨厚,好读书者,交友两三人足矣!且势利言之,则有酒食之费、应酬之扰,一遇婚丧有无,则有资给贷之事。甚至有争讼外侮,则又有关说救援之事。平昔既与之契密,临事却之,必生怨毒反唇。故余以为宜慎之于始也;昔人有戒:“饭不嚼便咽,路不看便走,话不想便说,事不思便做。”予益之曰:“友不择便交,气不忍不便动,财不审便取,衣不慎便脱。”

学字当专一。择古人佳帖或时人墨迹与已笔路相近者,专心学之,若朝更夕改,见异思迁,鲜有得成者。若体格不匀净而遽讲流动,失其本矣!学字忌飞动草率,大小不匀,而妄言奇古磊落,终无进步矣。读文不必多,择其精纯条畅,有气局词华者,多则百篇,少则六十篇。神明与之浑化,始为有益。若贪多务博,过眼辄忘,及至作时,则彼此不相涉,落笔仍是故吾,所以思常窒而不灵,词常窘而不裕,意常枯而不润。

人能处心积虑,一言一动皆思益人,而痛戒损人,则人望之若鸾凤,宝之如参苓。必为天地所佑,鬼神之所服,而享有多福矣!

凡读书,二十岁以前所读之书与二十岁以后所读之书迥异。幼年知识未开,天真纯固,所读者虽久不温习,偶尔提起,尚可数行成诵。若壮年所读,经月则忘,必不能持久。故六经、秦汉之文,词语古奥,必须幼年读。长壮后,虽倍蓰其功,终属影响。自八岁至二十岁,中间岁月无多,安可荒弃或读不急之书?此时,时文固不可不读,亦须择典雅醇正、理纯辞裕、可历二三十年无弊者读之。若朝华夕落、浅陋无识、诡僻失体、取悦一时者,安可以珠玉难换之岁月而读此无益之文?何如诵得《左》、《国》一两篇及东西汉典贵华腴之文数篇,为终身之用之宝乎?古人之书,安可尽读?但我所已读者决不轻弃。得尺则尺,得寸则寸。毋贪多,毋贪名,但求读一篇,必可以背诵。然后思通其义蕴,而运用之于手腕之下,如此则才气自然发越。若曾读此书,而全不能举其词,谓之“画饼充饥”。能举其词而不能运用,谓之“食物不化”。

深恼人读时文累千累百而不知理会,于身心毫无裨益。夫能理会,则数十篇百篇已足,焉用如此之多?不能理会,则读数千篇与不读一字等。徒使精神聩乱,临文捉笔,依旧茫然,不过胸中旧套应副,安有名理精论、佳词妙句,奔汇于笔端乎?古人云:“读生文不如玩熟文。必以我之精神,包乎此一篇之外,以我之心思,入乎此一篇之中。幼年当专攻举业,以为立身之本。

世家子弟,其修行立名之难,较寒士百倍。何以故?人之当面待之者,万不能如寒士之古道:小有失检,谁肯面斥其非?微有骄盈,谁肯深规其过?幼而骄惯,为亲戚之所优容;长而习成,为朋友之所谅恕;我愿汝曹常以席丰履盛为可危、可虑、难处、难全之地,勿以为可喜、可幸、易安、易逸之地;终身让路,不失尺寸,自古祗闻“忍”与“让”,足以消无穷之灾悔,未闻“忍”与“让”,翻以酿后来之祸患也,欲行忍认之道,先须从小事做起。余曾署刑部事五十日,见天下大讼大狱,多从极小事起。君子敬小慎微,凡事只从小处了。余行年五十余,生平未尝多受小人之侮,只有一善策,能转弯早耳。每思天下事,受得小气,则不至于受大气,吃得小亏,则不至于吃大亏,此生平得力之处。凡事最不可想占便宜,便宜者,天下人所共争也,我一人据之,则怨萃于我矣,我失便宜,则众怨消矣。故终身失便宜,乃终身得便宜也;座右箴:立品、读书、养身、择友。右四纲。戒嬉戏,慎威仪;谨言语,温经书;精举业,学楷字;谨起居,慎寒暑;节用度,谢酬;省宴集,寡交游。右十二目。

子弟自十七八以至廿三四,实为学业成废之关。盖自初入学至十五六,父师以童子视之,稍知训子者,断不忍听其废业。惟自十七八以后,年渐长,气渐骄,渐有朋友,渐有室家,嗜欲渐广。父母见其长成,师傅视为侪辈。德性未坚,转移最易;学业未就,蒙昧非难。幼年所习经书,此时皆束高阁。酬应交游,侈然大雅。博弈高会,自诩名流。转盼廿五六岁,儿女累多,生计迫蹙,蹉跎潦倒,学殖荒落。予见人家子弟半途而废者,多在此五六年中,弃幼学之功,贻终身之累,盖辙相踵也。汝正当此时,离父母之侧,前言诸弊,事事可虑。为龙为蛇,为虎为鼠,分于一念,介在两歧,可不慎哉!可不畏哉!

读书须明窗净几,案头不可多置书;作文以握管之人为大将,以精熟墨卷百篇为练兵,以杂读时艺为散卒,以题为坚垒。

天子知俭,则天下足,一人知俭,则一家足。且俭非止节啬财用己也。俭于言语,则元气藏而怨尤寡;则于交游,则匪类远,俭于酬酢,则岁月宽而本业修,俭于书札,则后患寡,俭于嬉游,则学业进;人生俭啬之名,可受而不必避,世俗每以为耻,不知此名一噪,则人绝觊觎之想。偶有所用,人即德之;保家莫如择友,多则二人,少则一人,断无目前良友,遂可得十数人之理!平时既简于应酬,有事可以请教。

惟田产房屋二者可恃以久远,以二者较之,房舍又不如田产。
今人家子弟,鲜衣怒马,恒舞酣歌。一裘之费动至数十金,一席之费动至数金。不思吾乡十余年来谷贱,竭十余石谷,不足供一筵,竭百余石谷,不足供一衣。安知农家作苦,终年沾衣涂足,岂易得此百石?
古人之意,全在小处节俭,大处之不足,由于小处之不谨,月计之不足,由于每日之用过多也。
子弟有二三千金之产,方能城居。若千金以下之业,则断不可城居矣!

古人有言,扫地焚香,清福已具。其有福者,佐以读书;其无福者,便生他想。旨哉斯言,予所深赏!且从来拂意之事,自不读书者见之,似为我所独遭,极其难堪,不知古人拂意之事有百倍于此者,特不细心体验耳! 即如东坡先生,殁后遭逢高孝,文字始出,而当时之忧谗畏讥,困顿转徙潮惠之间,苏过跣足涉水,居近牛栏,是何如境界?又如白香山之无嗣,陆放翁之忍饥,皆载在书卷,彼独非千载闻人,而所遇皆如此? 诚一平心静观,则人间拂意之事,可以涣然冰释。若不读书,则但见我所遭甚苦,而无穷怨尤嗔忿之心,烧灼不宁,其苦为何如耶?且富盛之事,古人亦有之,炙手可热,转眼皆空。故读书可以增长道心,为颐养第一事也!


\chapter{1   白话聊斋志异}
\label{\detokenize{p00_u5176_u5b83/_u767d_u8bdd_u804a_u658b_u5fd7_u5f02:id1}}\label{\detokenize{p00_u5176_u5b83/_u767d_u8bdd_u804a_u658b_u5fd7_u5f02::doc}}
\begin{sphinxShadowBox}
\sphinxstyletopictitle{contents}
\begin{itemize}
\item {} 
\phantomsection\label{\detokenize{p00_u5176_u5b83/_u767d_u8bdd_u804a_u658b_u5fd7_u5f02:id511}}{\hyperref[\detokenize{p00_u5176_u5b83/_u767d_u8bdd_u804a_u658b_u5fd7_u5f02:id1}]{\sphinxcrossref{1   白话聊斋志异}}}
\begin{itemize}
\item {} 
\phantomsection\label{\detokenize{p00_u5176_u5b83/_u767d_u8bdd_u804a_u658b_u5fd7_u5f02:id512}}{\hyperref[\detokenize{p00_u5176_u5b83/_u767d_u8bdd_u804a_u658b_u5fd7_u5f02:id2}]{\sphinxcrossref{1.1   卷 一}}}
\begin{itemize}
\item {} 
\phantomsection\label{\detokenize{p00_u5176_u5b83/_u767d_u8bdd_u804a_u658b_u5fd7_u5f02:id513}}{\hyperref[\detokenize{p00_u5176_u5b83/_u767d_u8bdd_u804a_u658b_u5fd7_u5f02:id3}]{\sphinxcrossref{1.1.1   考 城 隍}}}

\item {} 
\phantomsection\label{\detokenize{p00_u5176_u5b83/_u767d_u8bdd_u804a_u658b_u5fd7_u5f02:id514}}{\hyperref[\detokenize{p00_u5176_u5b83/_u767d_u8bdd_u804a_u658b_u5fd7_u5f02:id4}]{\sphinxcrossref{1.1.2   耳 中 人}}}

\item {} 
\phantomsection\label{\detokenize{p00_u5176_u5b83/_u767d_u8bdd_u804a_u658b_u5fd7_u5f02:id515}}{\hyperref[\detokenize{p00_u5176_u5b83/_u767d_u8bdd_u804a_u658b_u5fd7_u5f02:id5}]{\sphinxcrossref{1.1.3   尸 变}}}

\item {} 
\phantomsection\label{\detokenize{p00_u5176_u5b83/_u767d_u8bdd_u804a_u658b_u5fd7_u5f02:id516}}{\hyperref[\detokenize{p00_u5176_u5b83/_u767d_u8bdd_u804a_u658b_u5fd7_u5f02:id6}]{\sphinxcrossref{1.1.4   喷 水}}}

\item {} 
\phantomsection\label{\detokenize{p00_u5176_u5b83/_u767d_u8bdd_u804a_u658b_u5fd7_u5f02:id517}}{\hyperref[\detokenize{p00_u5176_u5b83/_u767d_u8bdd_u804a_u658b_u5fd7_u5f02:id7}]{\sphinxcrossref{1.1.5   瞳 人 语}}}

\item {} 
\phantomsection\label{\detokenize{p00_u5176_u5b83/_u767d_u8bdd_u804a_u658b_u5fd7_u5f02:id518}}{\hyperref[\detokenize{p00_u5176_u5b83/_u767d_u8bdd_u804a_u658b_u5fd7_u5f02:id8}]{\sphinxcrossref{1.1.6   画 壁}}}

\item {} 
\phantomsection\label{\detokenize{p00_u5176_u5b83/_u767d_u8bdd_u804a_u658b_u5fd7_u5f02:id519}}{\hyperref[\detokenize{p00_u5176_u5b83/_u767d_u8bdd_u804a_u658b_u5fd7_u5f02:id9}]{\sphinxcrossref{1.1.7   山 魈}}}

\item {} 
\phantomsection\label{\detokenize{p00_u5176_u5b83/_u767d_u8bdd_u804a_u658b_u5fd7_u5f02:id520}}{\hyperref[\detokenize{p00_u5176_u5b83/_u767d_u8bdd_u804a_u658b_u5fd7_u5f02:id10}]{\sphinxcrossref{1.1.8   咬 鬼}}}

\item {} 
\phantomsection\label{\detokenize{p00_u5176_u5b83/_u767d_u8bdd_u804a_u658b_u5fd7_u5f02:id521}}{\hyperref[\detokenize{p00_u5176_u5b83/_u767d_u8bdd_u804a_u658b_u5fd7_u5f02:id11}]{\sphinxcrossref{1.1.9   捉 狐}}}

\item {} 
\phantomsection\label{\detokenize{p00_u5176_u5b83/_u767d_u8bdd_u804a_u658b_u5fd7_u5f02:id522}}{\hyperref[\detokenize{p00_u5176_u5b83/_u767d_u8bdd_u804a_u658b_u5fd7_u5f02:id12}]{\sphinxcrossref{1.1.10   荍 中 怪}}}

\item {} 
\phantomsection\label{\detokenize{p00_u5176_u5b83/_u767d_u8bdd_u804a_u658b_u5fd7_u5f02:id523}}{\hyperref[\detokenize{p00_u5176_u5b83/_u767d_u8bdd_u804a_u658b_u5fd7_u5f02:id13}]{\sphinxcrossref{1.1.11   宅 妖}}}

\item {} 
\phantomsection\label{\detokenize{p00_u5176_u5b83/_u767d_u8bdd_u804a_u658b_u5fd7_u5f02:id524}}{\hyperref[\detokenize{p00_u5176_u5b83/_u767d_u8bdd_u804a_u658b_u5fd7_u5f02:id14}]{\sphinxcrossref{1.1.12   王 六 郎}}}

\item {} 
\phantomsection\label{\detokenize{p00_u5176_u5b83/_u767d_u8bdd_u804a_u658b_u5fd7_u5f02:id525}}{\hyperref[\detokenize{p00_u5176_u5b83/_u767d_u8bdd_u804a_u658b_u5fd7_u5f02:id15}]{\sphinxcrossref{1.1.13   偷 桃}}}

\item {} 
\phantomsection\label{\detokenize{p00_u5176_u5b83/_u767d_u8bdd_u804a_u658b_u5fd7_u5f02:id526}}{\hyperref[\detokenize{p00_u5176_u5b83/_u767d_u8bdd_u804a_u658b_u5fd7_u5f02:id16}]{\sphinxcrossref{1.1.14   种 梨}}}

\item {} 
\phantomsection\label{\detokenize{p00_u5176_u5b83/_u767d_u8bdd_u804a_u658b_u5fd7_u5f02:id527}}{\hyperref[\detokenize{p00_u5176_u5b83/_u767d_u8bdd_u804a_u658b_u5fd7_u5f02:id17}]{\sphinxcrossref{1.1.15   劳 山 道 士}}}

\item {} 
\phantomsection\label{\detokenize{p00_u5176_u5b83/_u767d_u8bdd_u804a_u658b_u5fd7_u5f02:id528}}{\hyperref[\detokenize{p00_u5176_u5b83/_u767d_u8bdd_u804a_u658b_u5fd7_u5f02:id18}]{\sphinxcrossref{1.1.16   长 清 僧}}}

\item {} 
\phantomsection\label{\detokenize{p00_u5176_u5b83/_u767d_u8bdd_u804a_u658b_u5fd7_u5f02:id529}}{\hyperref[\detokenize{p00_u5176_u5b83/_u767d_u8bdd_u804a_u658b_u5fd7_u5f02:id19}]{\sphinxcrossref{1.1.17   蛇 人}}}

\item {} 
\phantomsection\label{\detokenize{p00_u5176_u5b83/_u767d_u8bdd_u804a_u658b_u5fd7_u5f02:id530}}{\hyperref[\detokenize{p00_u5176_u5b83/_u767d_u8bdd_u804a_u658b_u5fd7_u5f02:id20}]{\sphinxcrossref{1.1.18   斫 蟒}}}

\item {} 
\phantomsection\label{\detokenize{p00_u5176_u5b83/_u767d_u8bdd_u804a_u658b_u5fd7_u5f02:id531}}{\hyperref[\detokenize{p00_u5176_u5b83/_u767d_u8bdd_u804a_u658b_u5fd7_u5f02:id21}]{\sphinxcrossref{1.1.19   犬 奸}}}

\item {} 
\phantomsection\label{\detokenize{p00_u5176_u5b83/_u767d_u8bdd_u804a_u658b_u5fd7_u5f02:id532}}{\hyperref[\detokenize{p00_u5176_u5b83/_u767d_u8bdd_u804a_u658b_u5fd7_u5f02:id22}]{\sphinxcrossref{1.1.20   雹 神}}}

\item {} 
\phantomsection\label{\detokenize{p00_u5176_u5b83/_u767d_u8bdd_u804a_u658b_u5fd7_u5f02:id533}}{\hyperref[\detokenize{p00_u5176_u5b83/_u767d_u8bdd_u804a_u658b_u5fd7_u5f02:id23}]{\sphinxcrossref{1.1.21   狐 嫁 女}}}

\item {} 
\phantomsection\label{\detokenize{p00_u5176_u5b83/_u767d_u8bdd_u804a_u658b_u5fd7_u5f02:id534}}{\hyperref[\detokenize{p00_u5176_u5b83/_u767d_u8bdd_u804a_u658b_u5fd7_u5f02:id24}]{\sphinxcrossref{1.1.22   娇 娜}}}

\item {} 
\phantomsection\label{\detokenize{p00_u5176_u5b83/_u767d_u8bdd_u804a_u658b_u5fd7_u5f02:id535}}{\hyperref[\detokenize{p00_u5176_u5b83/_u767d_u8bdd_u804a_u658b_u5fd7_u5f02:id25}]{\sphinxcrossref{1.1.23   僧 孽}}}

\item {} 
\phantomsection\label{\detokenize{p00_u5176_u5b83/_u767d_u8bdd_u804a_u658b_u5fd7_u5f02:id536}}{\hyperref[\detokenize{p00_u5176_u5b83/_u767d_u8bdd_u804a_u658b_u5fd7_u5f02:id26}]{\sphinxcrossref{1.1.24   妖 术}}}

\item {} 
\phantomsection\label{\detokenize{p00_u5176_u5b83/_u767d_u8bdd_u804a_u658b_u5fd7_u5f02:id537}}{\hyperref[\detokenize{p00_u5176_u5b83/_u767d_u8bdd_u804a_u658b_u5fd7_u5f02:id27}]{\sphinxcrossref{1.1.25   野 狗}}}

\item {} 
\phantomsection\label{\detokenize{p00_u5176_u5b83/_u767d_u8bdd_u804a_u658b_u5fd7_u5f02:id538}}{\hyperref[\detokenize{p00_u5176_u5b83/_u767d_u8bdd_u804a_u658b_u5fd7_u5f02:id28}]{\sphinxcrossref{1.1.26   三 生}}}

\item {} 
\phantomsection\label{\detokenize{p00_u5176_u5b83/_u767d_u8bdd_u804a_u658b_u5fd7_u5f02:id539}}{\hyperref[\detokenize{p00_u5176_u5b83/_u767d_u8bdd_u804a_u658b_u5fd7_u5f02:id29}]{\sphinxcrossref{1.1.27   狐 入 瓶}}}

\item {} 
\phantomsection\label{\detokenize{p00_u5176_u5b83/_u767d_u8bdd_u804a_u658b_u5fd7_u5f02:id540}}{\hyperref[\detokenize{p00_u5176_u5b83/_u767d_u8bdd_u804a_u658b_u5fd7_u5f02:id30}]{\sphinxcrossref{1.1.28   鬼 哭}}}

\item {} 
\phantomsection\label{\detokenize{p00_u5176_u5b83/_u767d_u8bdd_u804a_u658b_u5fd7_u5f02:id541}}{\hyperref[\detokenize{p00_u5176_u5b83/_u767d_u8bdd_u804a_u658b_u5fd7_u5f02:id31}]{\sphinxcrossref{1.1.29   真 定 女}}}

\item {} 
\phantomsection\label{\detokenize{p00_u5176_u5b83/_u767d_u8bdd_u804a_u658b_u5fd7_u5f02:id542}}{\hyperref[\detokenize{p00_u5176_u5b83/_u767d_u8bdd_u804a_u658b_u5fd7_u5f02:id32}]{\sphinxcrossref{1.1.30   焦 螟}}}

\item {} 
\phantomsection\label{\detokenize{p00_u5176_u5b83/_u767d_u8bdd_u804a_u658b_u5fd7_u5f02:id543}}{\hyperref[\detokenize{p00_u5176_u5b83/_u767d_u8bdd_u804a_u658b_u5fd7_u5f02:id33}]{\sphinxcrossref{1.1.31   叶 生}}}

\item {} 
\phantomsection\label{\detokenize{p00_u5176_u5b83/_u767d_u8bdd_u804a_u658b_u5fd7_u5f02:id544}}{\hyperref[\detokenize{p00_u5176_u5b83/_u767d_u8bdd_u804a_u658b_u5fd7_u5f02:id34}]{\sphinxcrossref{1.1.32   四 十 千}}}

\item {} 
\phantomsection\label{\detokenize{p00_u5176_u5b83/_u767d_u8bdd_u804a_u658b_u5fd7_u5f02:id545}}{\hyperref[\detokenize{p00_u5176_u5b83/_u767d_u8bdd_u804a_u658b_u5fd7_u5f02:id35}]{\sphinxcrossref{1.1.33   成 仙}}}

\item {} 
\phantomsection\label{\detokenize{p00_u5176_u5b83/_u767d_u8bdd_u804a_u658b_u5fd7_u5f02:id546}}{\hyperref[\detokenize{p00_u5176_u5b83/_u767d_u8bdd_u804a_u658b_u5fd7_u5f02:id36}]{\sphinxcrossref{1.1.34   新 郎}}}

\item {} 
\phantomsection\label{\detokenize{p00_u5176_u5b83/_u767d_u8bdd_u804a_u658b_u5fd7_u5f02:id547}}{\hyperref[\detokenize{p00_u5176_u5b83/_u767d_u8bdd_u804a_u658b_u5fd7_u5f02:id37}]{\sphinxcrossref{1.1.35   灵 官}}}

\item {} 
\phantomsection\label{\detokenize{p00_u5176_u5b83/_u767d_u8bdd_u804a_u658b_u5fd7_u5f02:id548}}{\hyperref[\detokenize{p00_u5176_u5b83/_u767d_u8bdd_u804a_u658b_u5fd7_u5f02:id38}]{\sphinxcrossref{1.1.36   王 兰}}}

\item {} 
\phantomsection\label{\detokenize{p00_u5176_u5b83/_u767d_u8bdd_u804a_u658b_u5fd7_u5f02:id549}}{\hyperref[\detokenize{p00_u5176_u5b83/_u767d_u8bdd_u804a_u658b_u5fd7_u5f02:id39}]{\sphinxcrossref{1.1.37   鹰 虎 神}}}

\item {} 
\phantomsection\label{\detokenize{p00_u5176_u5b83/_u767d_u8bdd_u804a_u658b_u5fd7_u5f02:id550}}{\hyperref[\detokenize{p00_u5176_u5b83/_u767d_u8bdd_u804a_u658b_u5fd7_u5f02:id40}]{\sphinxcrossref{1.1.38   王 成}}}

\item {} 
\phantomsection\label{\detokenize{p00_u5176_u5b83/_u767d_u8bdd_u804a_u658b_u5fd7_u5f02:id551}}{\hyperref[\detokenize{p00_u5176_u5b83/_u767d_u8bdd_u804a_u658b_u5fd7_u5f02:id41}]{\sphinxcrossref{1.1.39   青 凤}}}

\item {} 
\phantomsection\label{\detokenize{p00_u5176_u5b83/_u767d_u8bdd_u804a_u658b_u5fd7_u5f02:id552}}{\hyperref[\detokenize{p00_u5176_u5b83/_u767d_u8bdd_u804a_u658b_u5fd7_u5f02:id42}]{\sphinxcrossref{1.1.40   画 皮}}}

\item {} 
\phantomsection\label{\detokenize{p00_u5176_u5b83/_u767d_u8bdd_u804a_u658b_u5fd7_u5f02:id553}}{\hyperref[\detokenize{p00_u5176_u5b83/_u767d_u8bdd_u804a_u658b_u5fd7_u5f02:id43}]{\sphinxcrossref{1.1.41   贾 儿}}}

\item {} 
\phantomsection\label{\detokenize{p00_u5176_u5b83/_u767d_u8bdd_u804a_u658b_u5fd7_u5f02:id554}}{\hyperref[\detokenize{p00_u5176_u5b83/_u767d_u8bdd_u804a_u658b_u5fd7_u5f02:id44}]{\sphinxcrossref{1.1.42   蛇 癖}}}

\end{itemize}

\item {} 
\phantomsection\label{\detokenize{p00_u5176_u5b83/_u767d_u8bdd_u804a_u658b_u5fd7_u5f02:id555}}{\hyperref[\detokenize{p00_u5176_u5b83/_u767d_u8bdd_u804a_u658b_u5fd7_u5f02:id45}]{\sphinxcrossref{1.2   卷 二}}}
\begin{itemize}
\item {} 
\phantomsection\label{\detokenize{p00_u5176_u5b83/_u767d_u8bdd_u804a_u658b_u5fd7_u5f02:id556}}{\hyperref[\detokenize{p00_u5176_u5b83/_u767d_u8bdd_u804a_u658b_u5fd7_u5f02:id46}]{\sphinxcrossref{1.2.1   金 世 成}}}

\item {} 
\phantomsection\label{\detokenize{p00_u5176_u5b83/_u767d_u8bdd_u804a_u658b_u5fd7_u5f02:id557}}{\hyperref[\detokenize{p00_u5176_u5b83/_u767d_u8bdd_u804a_u658b_u5fd7_u5f02:id47}]{\sphinxcrossref{1.2.2   董 生}}}

\item {} 
\phantomsection\label{\detokenize{p00_u5176_u5b83/_u767d_u8bdd_u804a_u658b_u5fd7_u5f02:id558}}{\hyperref[\detokenize{p00_u5176_u5b83/_u767d_u8bdd_u804a_u658b_u5fd7_u5f02:id48}]{\sphinxcrossref{1.2.3   龁 石}}}

\item {} 
\phantomsection\label{\detokenize{p00_u5176_u5b83/_u767d_u8bdd_u804a_u658b_u5fd7_u5f02:id559}}{\hyperref[\detokenize{p00_u5176_u5b83/_u767d_u8bdd_u804a_u658b_u5fd7_u5f02:id49}]{\sphinxcrossref{1.2.4   庙 鬼}}}

\item {} 
\phantomsection\label{\detokenize{p00_u5176_u5b83/_u767d_u8bdd_u804a_u658b_u5fd7_u5f02:id560}}{\hyperref[\detokenize{p00_u5176_u5b83/_u767d_u8bdd_u804a_u658b_u5fd7_u5f02:id50}]{\sphinxcrossref{1.2.5   陆 判}}}

\item {} 
\phantomsection\label{\detokenize{p00_u5176_u5b83/_u767d_u8bdd_u804a_u658b_u5fd7_u5f02:id561}}{\hyperref[\detokenize{p00_u5176_u5b83/_u767d_u8bdd_u804a_u658b_u5fd7_u5f02:id51}]{\sphinxcrossref{1.2.6   婴 宁}}}

\item {} 
\phantomsection\label{\detokenize{p00_u5176_u5b83/_u767d_u8bdd_u804a_u658b_u5fd7_u5f02:id562}}{\hyperref[\detokenize{p00_u5176_u5b83/_u767d_u8bdd_u804a_u658b_u5fd7_u5f02:id52}]{\sphinxcrossref{1.2.7   聂 小 倩}}}

\item {} 
\phantomsection\label{\detokenize{p00_u5176_u5b83/_u767d_u8bdd_u804a_u658b_u5fd7_u5f02:id563}}{\hyperref[\detokenize{p00_u5176_u5b83/_u767d_u8bdd_u804a_u658b_u5fd7_u5f02:id53}]{\sphinxcrossref{1.2.8   义 鼠}}}

\item {} 
\phantomsection\label{\detokenize{p00_u5176_u5b83/_u767d_u8bdd_u804a_u658b_u5fd7_u5f02:id564}}{\hyperref[\detokenize{p00_u5176_u5b83/_u767d_u8bdd_u804a_u658b_u5fd7_u5f02:id54}]{\sphinxcrossref{1.2.9   地 震}}}

\item {} 
\phantomsection\label{\detokenize{p00_u5176_u5b83/_u767d_u8bdd_u804a_u658b_u5fd7_u5f02:id565}}{\hyperref[\detokenize{p00_u5176_u5b83/_u767d_u8bdd_u804a_u658b_u5fd7_u5f02:id55}]{\sphinxcrossref{1.2.10   海 公 子}}}

\item {} 
\phantomsection\label{\detokenize{p00_u5176_u5b83/_u767d_u8bdd_u804a_u658b_u5fd7_u5f02:id566}}{\hyperref[\detokenize{p00_u5176_u5b83/_u767d_u8bdd_u804a_u658b_u5fd7_u5f02:id56}]{\sphinxcrossref{1.2.11   丁 前 溪}}}

\item {} 
\phantomsection\label{\detokenize{p00_u5176_u5b83/_u767d_u8bdd_u804a_u658b_u5fd7_u5f02:id567}}{\hyperref[\detokenize{p00_u5176_u5b83/_u767d_u8bdd_u804a_u658b_u5fd7_u5f02:id57}]{\sphinxcrossref{1.2.12   海 大 鱼}}}

\item {} 
\phantomsection\label{\detokenize{p00_u5176_u5b83/_u767d_u8bdd_u804a_u658b_u5fd7_u5f02:id568}}{\hyperref[\detokenize{p00_u5176_u5b83/_u767d_u8bdd_u804a_u658b_u5fd7_u5f02:id58}]{\sphinxcrossref{1.2.13   张 老 相 公}}}

\item {} 
\phantomsection\label{\detokenize{p00_u5176_u5b83/_u767d_u8bdd_u804a_u658b_u5fd7_u5f02:id569}}{\hyperref[\detokenize{p00_u5176_u5b83/_u767d_u8bdd_u804a_u658b_u5fd7_u5f02:id59}]{\sphinxcrossref{1.2.14   水 莽 草}}}

\item {} 
\phantomsection\label{\detokenize{p00_u5176_u5b83/_u767d_u8bdd_u804a_u658b_u5fd7_u5f02:id570}}{\hyperref[\detokenize{p00_u5176_u5b83/_u767d_u8bdd_u804a_u658b_u5fd7_u5f02:id60}]{\sphinxcrossref{1.2.15   造 畜}}}

\item {} 
\phantomsection\label{\detokenize{p00_u5176_u5b83/_u767d_u8bdd_u804a_u658b_u5fd7_u5f02:id571}}{\hyperref[\detokenize{p00_u5176_u5b83/_u767d_u8bdd_u804a_u658b_u5fd7_u5f02:id61}]{\sphinxcrossref{1.2.16   凤 阳 士 人}}}

\item {} 
\phantomsection\label{\detokenize{p00_u5176_u5b83/_u767d_u8bdd_u804a_u658b_u5fd7_u5f02:id572}}{\hyperref[\detokenize{p00_u5176_u5b83/_u767d_u8bdd_u804a_u658b_u5fd7_u5f02:id62}]{\sphinxcrossref{1.2.17   耿 十 八}}}

\item {} 
\phantomsection\label{\detokenize{p00_u5176_u5b83/_u767d_u8bdd_u804a_u658b_u5fd7_u5f02:id573}}{\hyperref[\detokenize{p00_u5176_u5b83/_u767d_u8bdd_u804a_u658b_u5fd7_u5f02:id63}]{\sphinxcrossref{1.2.18   珠 儿}}}

\item {} 
\phantomsection\label{\detokenize{p00_u5176_u5b83/_u767d_u8bdd_u804a_u658b_u5fd7_u5f02:id574}}{\hyperref[\detokenize{p00_u5176_u5b83/_u767d_u8bdd_u804a_u658b_u5fd7_u5f02:id64}]{\sphinxcrossref{1.2.19   小 官 人}}}

\item {} 
\phantomsection\label{\detokenize{p00_u5176_u5b83/_u767d_u8bdd_u804a_u658b_u5fd7_u5f02:id575}}{\hyperref[\detokenize{p00_u5176_u5b83/_u767d_u8bdd_u804a_u658b_u5fd7_u5f02:id65}]{\sphinxcrossref{1.2.20   胡 四 姐}}}

\item {} 
\phantomsection\label{\detokenize{p00_u5176_u5b83/_u767d_u8bdd_u804a_u658b_u5fd7_u5f02:id576}}{\hyperref[\detokenize{p00_u5176_u5b83/_u767d_u8bdd_u804a_u658b_u5fd7_u5f02:id66}]{\sphinxcrossref{1.2.21   祝 翁}}}

\item {} 
\phantomsection\label{\detokenize{p00_u5176_u5b83/_u767d_u8bdd_u804a_u658b_u5fd7_u5f02:id577}}{\hyperref[\detokenize{p00_u5176_u5b83/_u767d_u8bdd_u804a_u658b_u5fd7_u5f02:id67}]{\sphinxcrossref{1.2.22   猪 婆 龙}}}

\item {} 
\phantomsection\label{\detokenize{p00_u5176_u5b83/_u767d_u8bdd_u804a_u658b_u5fd7_u5f02:id578}}{\hyperref[\detokenize{p00_u5176_u5b83/_u767d_u8bdd_u804a_u658b_u5fd7_u5f02:id68}]{\sphinxcrossref{1.2.23   某 公}}}

\item {} 
\phantomsection\label{\detokenize{p00_u5176_u5b83/_u767d_u8bdd_u804a_u658b_u5fd7_u5f02:id579}}{\hyperref[\detokenize{p00_u5176_u5b83/_u767d_u8bdd_u804a_u658b_u5fd7_u5f02:id69}]{\sphinxcrossref{1.2.24   快 刀}}}

\item {} 
\phantomsection\label{\detokenize{p00_u5176_u5b83/_u767d_u8bdd_u804a_u658b_u5fd7_u5f02:id580}}{\hyperref[\detokenize{p00_u5176_u5b83/_u767d_u8bdd_u804a_u658b_u5fd7_u5f02:id70}]{\sphinxcrossref{1.2.25   侠 女}}}

\item {} 
\phantomsection\label{\detokenize{p00_u5176_u5b83/_u767d_u8bdd_u804a_u658b_u5fd7_u5f02:id581}}{\hyperref[\detokenize{p00_u5176_u5b83/_u767d_u8bdd_u804a_u658b_u5fd7_u5f02:id71}]{\sphinxcrossref{1.2.26   酒 友}}}

\item {} 
\phantomsection\label{\detokenize{p00_u5176_u5b83/_u767d_u8bdd_u804a_u658b_u5fd7_u5f02:id582}}{\hyperref[\detokenize{p00_u5176_u5b83/_u767d_u8bdd_u804a_u658b_u5fd7_u5f02:id72}]{\sphinxcrossref{1.2.27   莲 香}}}

\item {} 
\phantomsection\label{\detokenize{p00_u5176_u5b83/_u767d_u8bdd_u804a_u658b_u5fd7_u5f02:id583}}{\hyperref[\detokenize{p00_u5176_u5b83/_u767d_u8bdd_u804a_u658b_u5fd7_u5f02:id73}]{\sphinxcrossref{1.2.28   阿 宝}}}

\item {} 
\phantomsection\label{\detokenize{p00_u5176_u5b83/_u767d_u8bdd_u804a_u658b_u5fd7_u5f02:id584}}{\hyperref[\detokenize{p00_u5176_u5b83/_u767d_u8bdd_u804a_u658b_u5fd7_u5f02:id74}]{\sphinxcrossref{1.2.29   九 山 王}}}

\item {} 
\phantomsection\label{\detokenize{p00_u5176_u5b83/_u767d_u8bdd_u804a_u658b_u5fd7_u5f02:id585}}{\hyperref[\detokenize{p00_u5176_u5b83/_u767d_u8bdd_u804a_u658b_u5fd7_u5f02:id75}]{\sphinxcrossref{1.2.30   遵 化 署 狐}}}

\item {} 
\phantomsection\label{\detokenize{p00_u5176_u5b83/_u767d_u8bdd_u804a_u658b_u5fd7_u5f02:id586}}{\hyperref[\detokenize{p00_u5176_u5b83/_u767d_u8bdd_u804a_u658b_u5fd7_u5f02:id76}]{\sphinxcrossref{1.2.31   张 诚}}}

\item {} 
\phantomsection\label{\detokenize{p00_u5176_u5b83/_u767d_u8bdd_u804a_u658b_u5fd7_u5f02:id587}}{\hyperref[\detokenize{p00_u5176_u5b83/_u767d_u8bdd_u804a_u658b_u5fd7_u5f02:id77}]{\sphinxcrossref{1.2.32   汾 州 狐}}}

\item {} 
\phantomsection\label{\detokenize{p00_u5176_u5b83/_u767d_u8bdd_u804a_u658b_u5fd7_u5f02:id588}}{\hyperref[\detokenize{p00_u5176_u5b83/_u767d_u8bdd_u804a_u658b_u5fd7_u5f02:id78}]{\sphinxcrossref{1.2.33   巧 娘}}}

\item {} 
\phantomsection\label{\detokenize{p00_u5176_u5b83/_u767d_u8bdd_u804a_u658b_u5fd7_u5f02:id589}}{\hyperref[\detokenize{p00_u5176_u5b83/_u767d_u8bdd_u804a_u658b_u5fd7_u5f02:id79}]{\sphinxcrossref{1.2.34   吴 令}}}

\item {} 
\phantomsection\label{\detokenize{p00_u5176_u5b83/_u767d_u8bdd_u804a_u658b_u5fd7_u5f02:id590}}{\hyperref[\detokenize{p00_u5176_u5b83/_u767d_u8bdd_u804a_u658b_u5fd7_u5f02:id80}]{\sphinxcrossref{1.2.35   口 技}}}

\item {} 
\phantomsection\label{\detokenize{p00_u5176_u5b83/_u767d_u8bdd_u804a_u658b_u5fd7_u5f02:id591}}{\hyperref[\detokenize{p00_u5176_u5b83/_u767d_u8bdd_u804a_u658b_u5fd7_u5f02:id81}]{\sphinxcrossref{1.2.36   狐 联}}}

\item {} 
\phantomsection\label{\detokenize{p00_u5176_u5b83/_u767d_u8bdd_u804a_u658b_u5fd7_u5f02:id592}}{\hyperref[\detokenize{p00_u5176_u5b83/_u767d_u8bdd_u804a_u658b_u5fd7_u5f02:id82}]{\sphinxcrossref{1.2.37   潍 水 狐}}}

\item {} 
\phantomsection\label{\detokenize{p00_u5176_u5b83/_u767d_u8bdd_u804a_u658b_u5fd7_u5f02:id593}}{\hyperref[\detokenize{p00_u5176_u5b83/_u767d_u8bdd_u804a_u658b_u5fd7_u5f02:id83}]{\sphinxcrossref{1.2.38   红 玉}}}

\item {} 
\phantomsection\label{\detokenize{p00_u5176_u5b83/_u767d_u8bdd_u804a_u658b_u5fd7_u5f02:id594}}{\hyperref[\detokenize{p00_u5176_u5b83/_u767d_u8bdd_u804a_u658b_u5fd7_u5f02:id84}]{\sphinxcrossref{1.2.39   龙}}}

\item {} 
\phantomsection\label{\detokenize{p00_u5176_u5b83/_u767d_u8bdd_u804a_u658b_u5fd7_u5f02:id595}}{\hyperref[\detokenize{p00_u5176_u5b83/_u767d_u8bdd_u804a_u658b_u5fd7_u5f02:id85}]{\sphinxcrossref{1.2.40   林 四 娘}}}

\end{itemize}

\item {} 
\phantomsection\label{\detokenize{p00_u5176_u5b83/_u767d_u8bdd_u804a_u658b_u5fd7_u5f02:id596}}{\hyperref[\detokenize{p00_u5176_u5b83/_u767d_u8bdd_u804a_u658b_u5fd7_u5f02:id86}]{\sphinxcrossref{1.3   卷 三}}}
\begin{itemize}
\item {} 
\phantomsection\label{\detokenize{p00_u5176_u5b83/_u767d_u8bdd_u804a_u658b_u5fd7_u5f02:id597}}{\hyperref[\detokenize{p00_u5176_u5b83/_u767d_u8bdd_u804a_u658b_u5fd7_u5f02:id87}]{\sphinxcrossref{1.3.1   江 中}}}

\item {} 
\phantomsection\label{\detokenize{p00_u5176_u5b83/_u767d_u8bdd_u804a_u658b_u5fd7_u5f02:id598}}{\hyperref[\detokenize{p00_u5176_u5b83/_u767d_u8bdd_u804a_u658b_u5fd7_u5f02:id88}]{\sphinxcrossref{1.3.2   鲁 公 女}}}

\item {} 
\phantomsection\label{\detokenize{p00_u5176_u5b83/_u767d_u8bdd_u804a_u658b_u5fd7_u5f02:id599}}{\hyperref[\detokenize{p00_u5176_u5b83/_u767d_u8bdd_u804a_u658b_u5fd7_u5f02:id89}]{\sphinxcrossref{1.3.3   道 士}}}

\item {} 
\phantomsection\label{\detokenize{p00_u5176_u5b83/_u767d_u8bdd_u804a_u658b_u5fd7_u5f02:id600}}{\hyperref[\detokenize{p00_u5176_u5b83/_u767d_u8bdd_u804a_u658b_u5fd7_u5f02:id90}]{\sphinxcrossref{1.3.4   胡 氏}}}

\item {} 
\phantomsection\label{\detokenize{p00_u5176_u5b83/_u767d_u8bdd_u804a_u658b_u5fd7_u5f02:id601}}{\hyperref[\detokenize{p00_u5176_u5b83/_u767d_u8bdd_u804a_u658b_u5fd7_u5f02:id91}]{\sphinxcrossref{1.3.5   戏 术}}}

\item {} 
\phantomsection\label{\detokenize{p00_u5176_u5b83/_u767d_u8bdd_u804a_u658b_u5fd7_u5f02:id602}}{\hyperref[\detokenize{p00_u5176_u5b83/_u767d_u8bdd_u804a_u658b_u5fd7_u5f02:id92}]{\sphinxcrossref{1.3.6   丐 僧}}}

\item {} 
\phantomsection\label{\detokenize{p00_u5176_u5b83/_u767d_u8bdd_u804a_u658b_u5fd7_u5f02:id603}}{\hyperref[\detokenize{p00_u5176_u5b83/_u767d_u8bdd_u804a_u658b_u5fd7_u5f02:id93}]{\sphinxcrossref{1.3.7   伏 狐}}}

\item {} 
\phantomsection\label{\detokenize{p00_u5176_u5b83/_u767d_u8bdd_u804a_u658b_u5fd7_u5f02:id604}}{\hyperref[\detokenize{p00_u5176_u5b83/_u767d_u8bdd_u804a_u658b_u5fd7_u5f02:id94}]{\sphinxcrossref{1.3.8   蛰 龙}}}

\item {} 
\phantomsection\label{\detokenize{p00_u5176_u5b83/_u767d_u8bdd_u804a_u658b_u5fd7_u5f02:id605}}{\hyperref[\detokenize{p00_u5176_u5b83/_u767d_u8bdd_u804a_u658b_u5fd7_u5f02:id95}]{\sphinxcrossref{1.3.9   苏 仙}}}

\item {} 
\phantomsection\label{\detokenize{p00_u5176_u5b83/_u767d_u8bdd_u804a_u658b_u5fd7_u5f02:id606}}{\hyperref[\detokenize{p00_u5176_u5b83/_u767d_u8bdd_u804a_u658b_u5fd7_u5f02:id96}]{\sphinxcrossref{1.3.10   李 伯 言}}}

\item {} 
\phantomsection\label{\detokenize{p00_u5176_u5b83/_u767d_u8bdd_u804a_u658b_u5fd7_u5f02:id607}}{\hyperref[\detokenize{p00_u5176_u5b83/_u767d_u8bdd_u804a_u658b_u5fd7_u5f02:id97}]{\sphinxcrossref{1.3.11   黄 九 郎}}}

\item {} 
\phantomsection\label{\detokenize{p00_u5176_u5b83/_u767d_u8bdd_u804a_u658b_u5fd7_u5f02:id608}}{\hyperref[\detokenize{p00_u5176_u5b83/_u767d_u8bdd_u804a_u658b_u5fd7_u5f02:id98}]{\sphinxcrossref{1.3.12   金 陵 女 子}}}

\item {} 
\phantomsection\label{\detokenize{p00_u5176_u5b83/_u767d_u8bdd_u804a_u658b_u5fd7_u5f02:id609}}{\hyperref[\detokenize{p00_u5176_u5b83/_u767d_u8bdd_u804a_u658b_u5fd7_u5f02:id99}]{\sphinxcrossref{1.3.13   汤 公}}}

\item {} 
\phantomsection\label{\detokenize{p00_u5176_u5b83/_u767d_u8bdd_u804a_u658b_u5fd7_u5f02:id610}}{\hyperref[\detokenize{p00_u5176_u5b83/_u767d_u8bdd_u804a_u658b_u5fd7_u5f02:id100}]{\sphinxcrossref{1.3.14   阎 罗}}}

\item {} 
\phantomsection\label{\detokenize{p00_u5176_u5b83/_u767d_u8bdd_u804a_u658b_u5fd7_u5f02:id611}}{\hyperref[\detokenize{p00_u5176_u5b83/_u767d_u8bdd_u804a_u658b_u5fd7_u5f02:id101}]{\sphinxcrossref{1.3.15   连 琐}}}

\item {} 
\phantomsection\label{\detokenize{p00_u5176_u5b83/_u767d_u8bdd_u804a_u658b_u5fd7_u5f02:id612}}{\hyperref[\detokenize{p00_u5176_u5b83/_u767d_u8bdd_u804a_u658b_u5fd7_u5f02:id102}]{\sphinxcrossref{1.3.16   单 道 士}}}

\item {} 
\phantomsection\label{\detokenize{p00_u5176_u5b83/_u767d_u8bdd_u804a_u658b_u5fd7_u5f02:id613}}{\hyperref[\detokenize{p00_u5176_u5b83/_u767d_u8bdd_u804a_u658b_u5fd7_u5f02:id103}]{\sphinxcrossref{1.3.17   白 于 玉}}}

\item {} 
\phantomsection\label{\detokenize{p00_u5176_u5b83/_u767d_u8bdd_u804a_u658b_u5fd7_u5f02:id614}}{\hyperref[\detokenize{p00_u5176_u5b83/_u767d_u8bdd_u804a_u658b_u5fd7_u5f02:id104}]{\sphinxcrossref{1.3.18   夜 叉 国}}}

\item {} 
\phantomsection\label{\detokenize{p00_u5176_u5b83/_u767d_u8bdd_u804a_u658b_u5fd7_u5f02:id615}}{\hyperref[\detokenize{p00_u5176_u5b83/_u767d_u8bdd_u804a_u658b_u5fd7_u5f02:id105}]{\sphinxcrossref{1.3.19   小 髻}}}

\item {} 
\phantomsection\label{\detokenize{p00_u5176_u5b83/_u767d_u8bdd_u804a_u658b_u5fd7_u5f02:id616}}{\hyperref[\detokenize{p00_u5176_u5b83/_u767d_u8bdd_u804a_u658b_u5fd7_u5f02:id106}]{\sphinxcrossref{1.3.20   西 僧}}}

\item {} 
\phantomsection\label{\detokenize{p00_u5176_u5b83/_u767d_u8bdd_u804a_u658b_u5fd7_u5f02:id617}}{\hyperref[\detokenize{p00_u5176_u5b83/_u767d_u8bdd_u804a_u658b_u5fd7_u5f02:id107}]{\sphinxcrossref{1.3.21   老 饕}}}

\item {} 
\phantomsection\label{\detokenize{p00_u5176_u5b83/_u767d_u8bdd_u804a_u658b_u5fd7_u5f02:id618}}{\hyperref[\detokenize{p00_u5176_u5b83/_u767d_u8bdd_u804a_u658b_u5fd7_u5f02:id108}]{\sphinxcrossref{1.3.22   连 城}}}

\item {} 
\phantomsection\label{\detokenize{p00_u5176_u5b83/_u767d_u8bdd_u804a_u658b_u5fd7_u5f02:id619}}{\hyperref[\detokenize{p00_u5176_u5b83/_u767d_u8bdd_u804a_u658b_u5fd7_u5f02:id109}]{\sphinxcrossref{1.3.23   霍 生}}}

\item {} 
\phantomsection\label{\detokenize{p00_u5176_u5b83/_u767d_u8bdd_u804a_u658b_u5fd7_u5f02:id620}}{\hyperref[\detokenize{p00_u5176_u5b83/_u767d_u8bdd_u804a_u658b_u5fd7_u5f02:id110}]{\sphinxcrossref{1.3.24   汪 士 秀}}}

\item {} 
\phantomsection\label{\detokenize{p00_u5176_u5b83/_u767d_u8bdd_u804a_u658b_u5fd7_u5f02:id621}}{\hyperref[\detokenize{p00_u5176_u5b83/_u767d_u8bdd_u804a_u658b_u5fd7_u5f02:id111}]{\sphinxcrossref{1.3.25   商 三 官}}}

\item {} 
\phantomsection\label{\detokenize{p00_u5176_u5b83/_u767d_u8bdd_u804a_u658b_u5fd7_u5f02:id622}}{\hyperref[\detokenize{p00_u5176_u5b83/_u767d_u8bdd_u804a_u658b_u5fd7_u5f02:id112}]{\sphinxcrossref{1.3.26   于 江}}}

\item {} 
\phantomsection\label{\detokenize{p00_u5176_u5b83/_u767d_u8bdd_u804a_u658b_u5fd7_u5f02:id623}}{\hyperref[\detokenize{p00_u5176_u5b83/_u767d_u8bdd_u804a_u658b_u5fd7_u5f02:id113}]{\sphinxcrossref{1.3.27   小 二}}}

\item {} 
\phantomsection\label{\detokenize{p00_u5176_u5b83/_u767d_u8bdd_u804a_u658b_u5fd7_u5f02:id624}}{\hyperref[\detokenize{p00_u5176_u5b83/_u767d_u8bdd_u804a_u658b_u5fd7_u5f02:id114}]{\sphinxcrossref{1.3.28   庚 娘}}}

\item {} 
\phantomsection\label{\detokenize{p00_u5176_u5b83/_u767d_u8bdd_u804a_u658b_u5fd7_u5f02:id625}}{\hyperref[\detokenize{p00_u5176_u5b83/_u767d_u8bdd_u804a_u658b_u5fd7_u5f02:id115}]{\sphinxcrossref{1.3.29   宫 梦 弼}}}

\item {} 
\phantomsection\label{\detokenize{p00_u5176_u5b83/_u767d_u8bdd_u804a_u658b_u5fd7_u5f02:id626}}{\hyperref[\detokenize{p00_u5176_u5b83/_u767d_u8bdd_u804a_u658b_u5fd7_u5f02:id116}]{\sphinxcrossref{1.3.30   鸲 鹆}}}

\item {} 
\phantomsection\label{\detokenize{p00_u5176_u5b83/_u767d_u8bdd_u804a_u658b_u5fd7_u5f02:id627}}{\hyperref[\detokenize{p00_u5176_u5b83/_u767d_u8bdd_u804a_u658b_u5fd7_u5f02:id117}]{\sphinxcrossref{1.3.31   刘 海 石}}}

\item {} 
\phantomsection\label{\detokenize{p00_u5176_u5b83/_u767d_u8bdd_u804a_u658b_u5fd7_u5f02:id628}}{\hyperref[\detokenize{p00_u5176_u5b83/_u767d_u8bdd_u804a_u658b_u5fd7_u5f02:id118}]{\sphinxcrossref{1.3.32   谕 鬼}}}

\item {} 
\phantomsection\label{\detokenize{p00_u5176_u5b83/_u767d_u8bdd_u804a_u658b_u5fd7_u5f02:id629}}{\hyperref[\detokenize{p00_u5176_u5b83/_u767d_u8bdd_u804a_u658b_u5fd7_u5f02:id119}]{\sphinxcrossref{1.3.33   泥 鬼}}}

\item {} 
\phantomsection\label{\detokenize{p00_u5176_u5b83/_u767d_u8bdd_u804a_u658b_u5fd7_u5f02:id630}}{\hyperref[\detokenize{p00_u5176_u5b83/_u767d_u8bdd_u804a_u658b_u5fd7_u5f02:id120}]{\sphinxcrossref{1.3.34   梦 别}}}

\item {} 
\phantomsection\label{\detokenize{p00_u5176_u5b83/_u767d_u8bdd_u804a_u658b_u5fd7_u5f02:id631}}{\hyperref[\detokenize{p00_u5176_u5b83/_u767d_u8bdd_u804a_u658b_u5fd7_u5f02:id121}]{\sphinxcrossref{1.3.35   犬 灯}}}

\item {} 
\phantomsection\label{\detokenize{p00_u5176_u5b83/_u767d_u8bdd_u804a_u658b_u5fd7_u5f02:id632}}{\hyperref[\detokenize{p00_u5176_u5b83/_u767d_u8bdd_u804a_u658b_u5fd7_u5f02:id122}]{\sphinxcrossref{1.3.36   番 僧}}}

\item {} 
\phantomsection\label{\detokenize{p00_u5176_u5b83/_u767d_u8bdd_u804a_u658b_u5fd7_u5f02:id633}}{\hyperref[\detokenize{p00_u5176_u5b83/_u767d_u8bdd_u804a_u658b_u5fd7_u5f02:id123}]{\sphinxcrossref{1.3.37   狐 妾}}}

\item {} 
\phantomsection\label{\detokenize{p00_u5176_u5b83/_u767d_u8bdd_u804a_u658b_u5fd7_u5f02:id634}}{\hyperref[\detokenize{p00_u5176_u5b83/_u767d_u8bdd_u804a_u658b_u5fd7_u5f02:id124}]{\sphinxcrossref{1.3.38   雷 曹}}}

\item {} 
\phantomsection\label{\detokenize{p00_u5176_u5b83/_u767d_u8bdd_u804a_u658b_u5fd7_u5f02:id635}}{\hyperref[\detokenize{p00_u5176_u5b83/_u767d_u8bdd_u804a_u658b_u5fd7_u5f02:id125}]{\sphinxcrossref{1.3.39   赌 符}}}

\item {} 
\phantomsection\label{\detokenize{p00_u5176_u5b83/_u767d_u8bdd_u804a_u658b_u5fd7_u5f02:id636}}{\hyperref[\detokenize{p00_u5176_u5b83/_u767d_u8bdd_u804a_u658b_u5fd7_u5f02:id126}]{\sphinxcrossref{1.3.40   阿 霞}}}

\item {} 
\phantomsection\label{\detokenize{p00_u5176_u5b83/_u767d_u8bdd_u804a_u658b_u5fd7_u5f02:id637}}{\hyperref[\detokenize{p00_u5176_u5b83/_u767d_u8bdd_u804a_u658b_u5fd7_u5f02:id127}]{\sphinxcrossref{1.3.41   李 司 鉴}}}

\item {} 
\phantomsection\label{\detokenize{p00_u5176_u5b83/_u767d_u8bdd_u804a_u658b_u5fd7_u5f02:id638}}{\hyperref[\detokenize{p00_u5176_u5b83/_u767d_u8bdd_u804a_u658b_u5fd7_u5f02:id128}]{\sphinxcrossref{1.3.42   五 羖 大 夫}}}

\item {} 
\phantomsection\label{\detokenize{p00_u5176_u5b83/_u767d_u8bdd_u804a_u658b_u5fd7_u5f02:id639}}{\hyperref[\detokenize{p00_u5176_u5b83/_u767d_u8bdd_u804a_u658b_u5fd7_u5f02:id129}]{\sphinxcrossref{1.3.43   毛 狐}}}

\item {} 
\phantomsection\label{\detokenize{p00_u5176_u5b83/_u767d_u8bdd_u804a_u658b_u5fd7_u5f02:id640}}{\hyperref[\detokenize{p00_u5176_u5b83/_u767d_u8bdd_u804a_u658b_u5fd7_u5f02:id130}]{\sphinxcrossref{1.3.44   翩 翩}}}

\item {} 
\phantomsection\label{\detokenize{p00_u5176_u5b83/_u767d_u8bdd_u804a_u658b_u5fd7_u5f02:id641}}{\hyperref[\detokenize{p00_u5176_u5b83/_u767d_u8bdd_u804a_u658b_u5fd7_u5f02:id131}]{\sphinxcrossref{1.3.45   黑 兽}}}

\end{itemize}

\item {} 
\phantomsection\label{\detokenize{p00_u5176_u5b83/_u767d_u8bdd_u804a_u658b_u5fd7_u5f02:id642}}{\hyperref[\detokenize{p00_u5176_u5b83/_u767d_u8bdd_u804a_u658b_u5fd7_u5f02:id132}]{\sphinxcrossref{1.4   卷 四}}}
\begin{itemize}
\item {} 
\phantomsection\label{\detokenize{p00_u5176_u5b83/_u767d_u8bdd_u804a_u658b_u5fd7_u5f02:id643}}{\hyperref[\detokenize{p00_u5176_u5b83/_u767d_u8bdd_u804a_u658b_u5fd7_u5f02:id133}]{\sphinxcrossref{1.4.1   余 德}}}

\item {} 
\phantomsection\label{\detokenize{p00_u5176_u5b83/_u767d_u8bdd_u804a_u658b_u5fd7_u5f02:id644}}{\hyperref[\detokenize{p00_u5176_u5b83/_u767d_u8bdd_u804a_u658b_u5fd7_u5f02:id134}]{\sphinxcrossref{1.4.2   杨 千 总}}}

\item {} 
\phantomsection\label{\detokenize{p00_u5176_u5b83/_u767d_u8bdd_u804a_u658b_u5fd7_u5f02:id645}}{\hyperref[\detokenize{p00_u5176_u5b83/_u767d_u8bdd_u804a_u658b_u5fd7_u5f02:id135}]{\sphinxcrossref{1.4.3   瓜 异}}}

\item {} 
\phantomsection\label{\detokenize{p00_u5176_u5b83/_u767d_u8bdd_u804a_u658b_u5fd7_u5f02:id646}}{\hyperref[\detokenize{p00_u5176_u5b83/_u767d_u8bdd_u804a_u658b_u5fd7_u5f02:id136}]{\sphinxcrossref{1.4.4   青 梅}}}

\item {} 
\phantomsection\label{\detokenize{p00_u5176_u5b83/_u767d_u8bdd_u804a_u658b_u5fd7_u5f02:id647}}{\hyperref[\detokenize{p00_u5176_u5b83/_u767d_u8bdd_u804a_u658b_u5fd7_u5f02:id137}]{\sphinxcrossref{1.4.5   罗 刹 海 市}}}

\item {} 
\phantomsection\label{\detokenize{p00_u5176_u5b83/_u767d_u8bdd_u804a_u658b_u5fd7_u5f02:id648}}{\hyperref[\detokenize{p00_u5176_u5b83/_u767d_u8bdd_u804a_u658b_u5fd7_u5f02:id138}]{\sphinxcrossref{1.4.6   田 七 郎}}}

\item {} 
\phantomsection\label{\detokenize{p00_u5176_u5b83/_u767d_u8bdd_u804a_u658b_u5fd7_u5f02:id649}}{\hyperref[\detokenize{p00_u5176_u5b83/_u767d_u8bdd_u804a_u658b_u5fd7_u5f02:id139}]{\sphinxcrossref{1.4.7   产 龙}}}

\item {} 
\phantomsection\label{\detokenize{p00_u5176_u5b83/_u767d_u8bdd_u804a_u658b_u5fd7_u5f02:id650}}{\hyperref[\detokenize{p00_u5176_u5b83/_u767d_u8bdd_u804a_u658b_u5fd7_u5f02:id140}]{\sphinxcrossref{1.4.8   保 住}}}

\item {} 
\phantomsection\label{\detokenize{p00_u5176_u5b83/_u767d_u8bdd_u804a_u658b_u5fd7_u5f02:id651}}{\hyperref[\detokenize{p00_u5176_u5b83/_u767d_u8bdd_u804a_u658b_u5fd7_u5f02:id141}]{\sphinxcrossref{1.4.9   公 孙 九 娘}}}

\item {} 
\phantomsection\label{\detokenize{p00_u5176_u5b83/_u767d_u8bdd_u804a_u658b_u5fd7_u5f02:id652}}{\hyperref[\detokenize{p00_u5176_u5b83/_u767d_u8bdd_u804a_u658b_u5fd7_u5f02:id142}]{\sphinxcrossref{1.4.10   促 织}}}

\item {} 
\phantomsection\label{\detokenize{p00_u5176_u5b83/_u767d_u8bdd_u804a_u658b_u5fd7_u5f02:id653}}{\hyperref[\detokenize{p00_u5176_u5b83/_u767d_u8bdd_u804a_u658b_u5fd7_u5f02:id143}]{\sphinxcrossref{1.4.11   柳 秀 才}}}

\item {} 
\phantomsection\label{\detokenize{p00_u5176_u5b83/_u767d_u8bdd_u804a_u658b_u5fd7_u5f02:id654}}{\hyperref[\detokenize{p00_u5176_u5b83/_u767d_u8bdd_u804a_u658b_u5fd7_u5f02:id144}]{\sphinxcrossref{1.4.12   水 灾}}}

\item {} 
\phantomsection\label{\detokenize{p00_u5176_u5b83/_u767d_u8bdd_u804a_u658b_u5fd7_u5f02:id655}}{\hyperref[\detokenize{p00_u5176_u5b83/_u767d_u8bdd_u804a_u658b_u5fd7_u5f02:id145}]{\sphinxcrossref{1.4.13   诸 城 某 甲}}}

\item {} 
\phantomsection\label{\detokenize{p00_u5176_u5b83/_u767d_u8bdd_u804a_u658b_u5fd7_u5f02:id656}}{\hyperref[\detokenize{p00_u5176_u5b83/_u767d_u8bdd_u804a_u658b_u5fd7_u5f02:id146}]{\sphinxcrossref{1.4.14   库 官}}}

\item {} 
\phantomsection\label{\detokenize{p00_u5176_u5b83/_u767d_u8bdd_u804a_u658b_u5fd7_u5f02:id657}}{\hyperref[\detokenize{p00_u5176_u5b83/_u767d_u8bdd_u804a_u658b_u5fd7_u5f02:id147}]{\sphinxcrossref{1.4.15   酆 都 御 史}}}

\item {} 
\phantomsection\label{\detokenize{p00_u5176_u5b83/_u767d_u8bdd_u804a_u658b_u5fd7_u5f02:id658}}{\hyperref[\detokenize{p00_u5176_u5b83/_u767d_u8bdd_u804a_u658b_u5fd7_u5f02:id148}]{\sphinxcrossref{1.4.16   龙 无 目}}}

\item {} 
\phantomsection\label{\detokenize{p00_u5176_u5b83/_u767d_u8bdd_u804a_u658b_u5fd7_u5f02:id659}}{\hyperref[\detokenize{p00_u5176_u5b83/_u767d_u8bdd_u804a_u658b_u5fd7_u5f02:id149}]{\sphinxcrossref{1.4.17   狐 谐}}}

\item {} 
\phantomsection\label{\detokenize{p00_u5176_u5b83/_u767d_u8bdd_u804a_u658b_u5fd7_u5f02:id660}}{\hyperref[\detokenize{p00_u5176_u5b83/_u767d_u8bdd_u804a_u658b_u5fd7_u5f02:id150}]{\sphinxcrossref{1.4.18   雨 钱}}}

\item {} 
\phantomsection\label{\detokenize{p00_u5176_u5b83/_u767d_u8bdd_u804a_u658b_u5fd7_u5f02:id661}}{\hyperref[\detokenize{p00_u5176_u5b83/_u767d_u8bdd_u804a_u658b_u5fd7_u5f02:id151}]{\sphinxcrossref{1.4.19   妾 击 贼}}}

\item {} 
\phantomsection\label{\detokenize{p00_u5176_u5b83/_u767d_u8bdd_u804a_u658b_u5fd7_u5f02:id662}}{\hyperref[\detokenize{p00_u5176_u5b83/_u767d_u8bdd_u804a_u658b_u5fd7_u5f02:id152}]{\sphinxcrossref{1.4.20   驱 怪}}}

\item {} 
\phantomsection\label{\detokenize{p00_u5176_u5b83/_u767d_u8bdd_u804a_u658b_u5fd7_u5f02:id663}}{\hyperref[\detokenize{p00_u5176_u5b83/_u767d_u8bdd_u804a_u658b_u5fd7_u5f02:id153}]{\sphinxcrossref{1.4.21   姊 妹 易 嫁}}}

\item {} 
\phantomsection\label{\detokenize{p00_u5176_u5b83/_u767d_u8bdd_u804a_u658b_u5fd7_u5f02:id664}}{\hyperref[\detokenize{p00_u5176_u5b83/_u767d_u8bdd_u804a_u658b_u5fd7_u5f02:id154}]{\sphinxcrossref{1.4.22   续 黄 粱}}}

\item {} 
\phantomsection\label{\detokenize{p00_u5176_u5b83/_u767d_u8bdd_u804a_u658b_u5fd7_u5f02:id665}}{\hyperref[\detokenize{p00_u5176_u5b83/_u767d_u8bdd_u804a_u658b_u5fd7_u5f02:id155}]{\sphinxcrossref{1.4.23   龙 取 水}}}

\item {} 
\phantomsection\label{\detokenize{p00_u5176_u5b83/_u767d_u8bdd_u804a_u658b_u5fd7_u5f02:id666}}{\hyperref[\detokenize{p00_u5176_u5b83/_u767d_u8bdd_u804a_u658b_u5fd7_u5f02:id156}]{\sphinxcrossref{1.4.24   小 猎 犬}}}

\item {} 
\phantomsection\label{\detokenize{p00_u5176_u5b83/_u767d_u8bdd_u804a_u658b_u5fd7_u5f02:id667}}{\hyperref[\detokenize{p00_u5176_u5b83/_u767d_u8bdd_u804a_u658b_u5fd7_u5f02:id157}]{\sphinxcrossref{1.4.25   棋 鬼}}}

\item {} 
\phantomsection\label{\detokenize{p00_u5176_u5b83/_u767d_u8bdd_u804a_u658b_u5fd7_u5f02:id668}}{\hyperref[\detokenize{p00_u5176_u5b83/_u767d_u8bdd_u804a_u658b_u5fd7_u5f02:id158}]{\sphinxcrossref{1.4.26   辛 十 四 娘}}}

\item {} 
\phantomsection\label{\detokenize{p00_u5176_u5b83/_u767d_u8bdd_u804a_u658b_u5fd7_u5f02:id669}}{\hyperref[\detokenize{p00_u5176_u5b83/_u767d_u8bdd_u804a_u658b_u5fd7_u5f02:id159}]{\sphinxcrossref{1.4.27   白 莲 教}}}

\item {} 
\phantomsection\label{\detokenize{p00_u5176_u5b83/_u767d_u8bdd_u804a_u658b_u5fd7_u5f02:id670}}{\hyperref[\detokenize{p00_u5176_u5b83/_u767d_u8bdd_u804a_u658b_u5fd7_u5f02:id160}]{\sphinxcrossref{1.4.28   双 灯}}}

\item {} 
\phantomsection\label{\detokenize{p00_u5176_u5b83/_u767d_u8bdd_u804a_u658b_u5fd7_u5f02:id671}}{\hyperref[\detokenize{p00_u5176_u5b83/_u767d_u8bdd_u804a_u658b_u5fd7_u5f02:id161}]{\sphinxcrossref{1.4.29   捉 鬼 射 狐}}}

\item {} 
\phantomsection\label{\detokenize{p00_u5176_u5b83/_u767d_u8bdd_u804a_u658b_u5fd7_u5f02:id672}}{\hyperref[\detokenize{p00_u5176_u5b83/_u767d_u8bdd_u804a_u658b_u5fd7_u5f02:id162}]{\sphinxcrossref{1.4.30   蹇 偿 债}}}

\item {} 
\phantomsection\label{\detokenize{p00_u5176_u5b83/_u767d_u8bdd_u804a_u658b_u5fd7_u5f02:id673}}{\hyperref[\detokenize{p00_u5176_u5b83/_u767d_u8bdd_u804a_u658b_u5fd7_u5f02:id163}]{\sphinxcrossref{1.4.31   头 滚}}}

\item {} 
\phantomsection\label{\detokenize{p00_u5176_u5b83/_u767d_u8bdd_u804a_u658b_u5fd7_u5f02:id674}}{\hyperref[\detokenize{p00_u5176_u5b83/_u767d_u8bdd_u804a_u658b_u5fd7_u5f02:id164}]{\sphinxcrossref{1.4.32   鬼 作 筵}}}

\item {} 
\phantomsection\label{\detokenize{p00_u5176_u5b83/_u767d_u8bdd_u804a_u658b_u5fd7_u5f02:id675}}{\hyperref[\detokenize{p00_u5176_u5b83/_u767d_u8bdd_u804a_u658b_u5fd7_u5f02:id165}]{\sphinxcrossref{1.4.33   胡 四 相 公}}}

\item {} 
\phantomsection\label{\detokenize{p00_u5176_u5b83/_u767d_u8bdd_u804a_u658b_u5fd7_u5f02:id676}}{\hyperref[\detokenize{p00_u5176_u5b83/_u767d_u8bdd_u804a_u658b_u5fd7_u5f02:id166}]{\sphinxcrossref{1.4.34   念 秧}}}

\item {} 
\phantomsection\label{\detokenize{p00_u5176_u5b83/_u767d_u8bdd_u804a_u658b_u5fd7_u5f02:id677}}{\hyperref[\detokenize{p00_u5176_u5b83/_u767d_u8bdd_u804a_u658b_u5fd7_u5f02:id167}]{\sphinxcrossref{1.4.35   蛙 曲}}}

\item {} 
\phantomsection\label{\detokenize{p00_u5176_u5b83/_u767d_u8bdd_u804a_u658b_u5fd7_u5f02:id678}}{\hyperref[\detokenize{p00_u5176_u5b83/_u767d_u8bdd_u804a_u658b_u5fd7_u5f02:id168}]{\sphinxcrossref{1.4.36   鼠 戏}}}

\item {} 
\phantomsection\label{\detokenize{p00_u5176_u5b83/_u767d_u8bdd_u804a_u658b_u5fd7_u5f02:id679}}{\hyperref[\detokenize{p00_u5176_u5b83/_u767d_u8bdd_u804a_u658b_u5fd7_u5f02:id169}]{\sphinxcrossref{1.4.37   泥 书 生}}}

\item {} 
\phantomsection\label{\detokenize{p00_u5176_u5b83/_u767d_u8bdd_u804a_u658b_u5fd7_u5f02:id680}}{\hyperref[\detokenize{p00_u5176_u5b83/_u767d_u8bdd_u804a_u658b_u5fd7_u5f02:id170}]{\sphinxcrossref{1.4.38   土 地 夫 人}}}

\item {} 
\phantomsection\label{\detokenize{p00_u5176_u5b83/_u767d_u8bdd_u804a_u658b_u5fd7_u5f02:id681}}{\hyperref[\detokenize{p00_u5176_u5b83/_u767d_u8bdd_u804a_u658b_u5fd7_u5f02:id171}]{\sphinxcrossref{1.4.39   寒 月 芙 蕖}}}

\item {} 
\phantomsection\label{\detokenize{p00_u5176_u5b83/_u767d_u8bdd_u804a_u658b_u5fd7_u5f02:id682}}{\hyperref[\detokenize{p00_u5176_u5b83/_u767d_u8bdd_u804a_u658b_u5fd7_u5f02:id172}]{\sphinxcrossref{1.4.40   酒 狂}}}

\end{itemize}

\item {} 
\phantomsection\label{\detokenize{p00_u5176_u5b83/_u767d_u8bdd_u804a_u658b_u5fd7_u5f02:id683}}{\hyperref[\detokenize{p00_u5176_u5b83/_u767d_u8bdd_u804a_u658b_u5fd7_u5f02:id173}]{\sphinxcrossref{1.5   卷 五}}}
\begin{itemize}
\item {} 
\phantomsection\label{\detokenize{p00_u5176_u5b83/_u767d_u8bdd_u804a_u658b_u5fd7_u5f02:id684}}{\hyperref[\detokenize{p00_u5176_u5b83/_u767d_u8bdd_u804a_u658b_u5fd7_u5f02:id174}]{\sphinxcrossref{1.5.1   阳 武 侯}}}

\item {} 
\phantomsection\label{\detokenize{p00_u5176_u5b83/_u767d_u8bdd_u804a_u658b_u5fd7_u5f02:id685}}{\hyperref[\detokenize{p00_u5176_u5b83/_u767d_u8bdd_u804a_u658b_u5fd7_u5f02:id175}]{\sphinxcrossref{1.5.2   赵 城 虎}}}

\item {} 
\phantomsection\label{\detokenize{p00_u5176_u5b83/_u767d_u8bdd_u804a_u658b_u5fd7_u5f02:id686}}{\hyperref[\detokenize{p00_u5176_u5b83/_u767d_u8bdd_u804a_u658b_u5fd7_u5f02:id176}]{\sphinxcrossref{1.5.3   螳 螂 捕 蛇}}}

\item {} 
\phantomsection\label{\detokenize{p00_u5176_u5b83/_u767d_u8bdd_u804a_u658b_u5fd7_u5f02:id687}}{\hyperref[\detokenize{p00_u5176_u5b83/_u767d_u8bdd_u804a_u658b_u5fd7_u5f02:id177}]{\sphinxcrossref{1.5.4   武 技}}}

\item {} 
\phantomsection\label{\detokenize{p00_u5176_u5b83/_u767d_u8bdd_u804a_u658b_u5fd7_u5f02:id688}}{\hyperref[\detokenize{p00_u5176_u5b83/_u767d_u8bdd_u804a_u658b_u5fd7_u5f02:id178}]{\sphinxcrossref{1.5.5   小 人}}}

\item {} 
\phantomsection\label{\detokenize{p00_u5176_u5b83/_u767d_u8bdd_u804a_u658b_u5fd7_u5f02:id689}}{\hyperref[\detokenize{p00_u5176_u5b83/_u767d_u8bdd_u804a_u658b_u5fd7_u5f02:id179}]{\sphinxcrossref{1.5.6   秦 生}}}

\item {} 
\phantomsection\label{\detokenize{p00_u5176_u5b83/_u767d_u8bdd_u804a_u658b_u5fd7_u5f02:id690}}{\hyperref[\detokenize{p00_u5176_u5b83/_u767d_u8bdd_u804a_u658b_u5fd7_u5f02:id180}]{\sphinxcrossref{1.5.7   鸦 头}}}

\item {} 
\phantomsection\label{\detokenize{p00_u5176_u5b83/_u767d_u8bdd_u804a_u658b_u5fd7_u5f02:id691}}{\hyperref[\detokenize{p00_u5176_u5b83/_u767d_u8bdd_u804a_u658b_u5fd7_u5f02:id181}]{\sphinxcrossref{1.5.8   酒 虫}}}

\item {} 
\phantomsection\label{\detokenize{p00_u5176_u5b83/_u767d_u8bdd_u804a_u658b_u5fd7_u5f02:id692}}{\hyperref[\detokenize{p00_u5176_u5b83/_u767d_u8bdd_u804a_u658b_u5fd7_u5f02:id182}]{\sphinxcrossref{1.5.9   木 雕 美 人}}}

\item {} 
\phantomsection\label{\detokenize{p00_u5176_u5b83/_u767d_u8bdd_u804a_u658b_u5fd7_u5f02:id693}}{\hyperref[\detokenize{p00_u5176_u5b83/_u767d_u8bdd_u804a_u658b_u5fd7_u5f02:id183}]{\sphinxcrossref{1.5.10   封 三 娘}}}

\item {} 
\phantomsection\label{\detokenize{p00_u5176_u5b83/_u767d_u8bdd_u804a_u658b_u5fd7_u5f02:id694}}{\hyperref[\detokenize{p00_u5176_u5b83/_u767d_u8bdd_u804a_u658b_u5fd7_u5f02:id184}]{\sphinxcrossref{1.5.11   狐 梦}}}

\item {} 
\phantomsection\label{\detokenize{p00_u5176_u5b83/_u767d_u8bdd_u804a_u658b_u5fd7_u5f02:id695}}{\hyperref[\detokenize{p00_u5176_u5b83/_u767d_u8bdd_u804a_u658b_u5fd7_u5f02:id185}]{\sphinxcrossref{1.5.12   布 客}}}

\item {} 
\phantomsection\label{\detokenize{p00_u5176_u5b83/_u767d_u8bdd_u804a_u658b_u5fd7_u5f02:id696}}{\hyperref[\detokenize{p00_u5176_u5b83/_u767d_u8bdd_u804a_u658b_u5fd7_u5f02:id186}]{\sphinxcrossref{1.5.13   农 人}}}

\item {} 
\phantomsection\label{\detokenize{p00_u5176_u5b83/_u767d_u8bdd_u804a_u658b_u5fd7_u5f02:id697}}{\hyperref[\detokenize{p00_u5176_u5b83/_u767d_u8bdd_u804a_u658b_u5fd7_u5f02:id187}]{\sphinxcrossref{1.5.14   章 阿 端}}}

\item {} 
\phantomsection\label{\detokenize{p00_u5176_u5b83/_u767d_u8bdd_u804a_u658b_u5fd7_u5f02:id698}}{\hyperref[\detokenize{p00_u5176_u5b83/_u767d_u8bdd_u804a_u658b_u5fd7_u5f02:id188}]{\sphinxcrossref{1.5.15   馎 饦 媪}}}

\item {} 
\phantomsection\label{\detokenize{p00_u5176_u5b83/_u767d_u8bdd_u804a_u658b_u5fd7_u5f02:id699}}{\hyperref[\detokenize{p00_u5176_u5b83/_u767d_u8bdd_u804a_u658b_u5fd7_u5f02:id189}]{\sphinxcrossref{1.5.16   金 永 年}}}

\item {} 
\phantomsection\label{\detokenize{p00_u5176_u5b83/_u767d_u8bdd_u804a_u658b_u5fd7_u5f02:id700}}{\hyperref[\detokenize{p00_u5176_u5b83/_u767d_u8bdd_u804a_u658b_u5fd7_u5f02:id190}]{\sphinxcrossref{1.5.17   花 姑 子}}}

\item {} 
\phantomsection\label{\detokenize{p00_u5176_u5b83/_u767d_u8bdd_u804a_u658b_u5fd7_u5f02:id701}}{\hyperref[\detokenize{p00_u5176_u5b83/_u767d_u8bdd_u804a_u658b_u5fd7_u5f02:id191}]{\sphinxcrossref{1.5.18   武 孝 廉}}}

\item {} 
\phantomsection\label{\detokenize{p00_u5176_u5b83/_u767d_u8bdd_u804a_u658b_u5fd7_u5f02:id702}}{\hyperref[\detokenize{p00_u5176_u5b83/_u767d_u8bdd_u804a_u658b_u5fd7_u5f02:id192}]{\sphinxcrossref{1.5.19   西 湖 主}}}

\item {} 
\phantomsection\label{\detokenize{p00_u5176_u5b83/_u767d_u8bdd_u804a_u658b_u5fd7_u5f02:id703}}{\hyperref[\detokenize{p00_u5176_u5b83/_u767d_u8bdd_u804a_u658b_u5fd7_u5f02:id193}]{\sphinxcrossref{1.5.20   孝 子}}}

\item {} 
\phantomsection\label{\detokenize{p00_u5176_u5b83/_u767d_u8bdd_u804a_u658b_u5fd7_u5f02:id704}}{\hyperref[\detokenize{p00_u5176_u5b83/_u767d_u8bdd_u804a_u658b_u5fd7_u5f02:id194}]{\sphinxcrossref{1.5.21   狮 子}}}

\item {} 
\phantomsection\label{\detokenize{p00_u5176_u5b83/_u767d_u8bdd_u804a_u658b_u5fd7_u5f02:id705}}{\hyperref[\detokenize{p00_u5176_u5b83/_u767d_u8bdd_u804a_u658b_u5fd7_u5f02:id195}]{\sphinxcrossref{1.5.22   阎 王}}}

\item {} 
\phantomsection\label{\detokenize{p00_u5176_u5b83/_u767d_u8bdd_u804a_u658b_u5fd7_u5f02:id706}}{\hyperref[\detokenize{p00_u5176_u5b83/_u767d_u8bdd_u804a_u658b_u5fd7_u5f02:id196}]{\sphinxcrossref{1.5.23   土 偶}}}

\item {} 
\phantomsection\label{\detokenize{p00_u5176_u5b83/_u767d_u8bdd_u804a_u658b_u5fd7_u5f02:id707}}{\hyperref[\detokenize{p00_u5176_u5b83/_u767d_u8bdd_u804a_u658b_u5fd7_u5f02:id197}]{\sphinxcrossref{1.5.24   长 治 女 子}}}

\item {} 
\phantomsection\label{\detokenize{p00_u5176_u5b83/_u767d_u8bdd_u804a_u658b_u5fd7_u5f02:id708}}{\hyperref[\detokenize{p00_u5176_u5b83/_u767d_u8bdd_u804a_u658b_u5fd7_u5f02:id198}]{\sphinxcrossref{1.5.25   义 犬}}}

\item {} 
\phantomsection\label{\detokenize{p00_u5176_u5b83/_u767d_u8bdd_u804a_u658b_u5fd7_u5f02:id709}}{\hyperref[\detokenize{p00_u5176_u5b83/_u767d_u8bdd_u804a_u658b_u5fd7_u5f02:id199}]{\sphinxcrossref{1.5.26   鄱 阳 神}}}

\item {} 
\phantomsection\label{\detokenize{p00_u5176_u5b83/_u767d_u8bdd_u804a_u658b_u5fd7_u5f02:id710}}{\hyperref[\detokenize{p00_u5176_u5b83/_u767d_u8bdd_u804a_u658b_u5fd7_u5f02:id200}]{\sphinxcrossref{1.5.27   伍 秋 月}}}

\item {} 
\phantomsection\label{\detokenize{p00_u5176_u5b83/_u767d_u8bdd_u804a_u658b_u5fd7_u5f02:id711}}{\hyperref[\detokenize{p00_u5176_u5b83/_u767d_u8bdd_u804a_u658b_u5fd7_u5f02:id201}]{\sphinxcrossref{1.5.28   莲 花 公 主}}}

\item {} 
\phantomsection\label{\detokenize{p00_u5176_u5b83/_u767d_u8bdd_u804a_u658b_u5fd7_u5f02:id712}}{\hyperref[\detokenize{p00_u5176_u5b83/_u767d_u8bdd_u804a_u658b_u5fd7_u5f02:id202}]{\sphinxcrossref{1.5.29   绿 衣 女}}}

\item {} 
\phantomsection\label{\detokenize{p00_u5176_u5b83/_u767d_u8bdd_u804a_u658b_u5fd7_u5f02:id713}}{\hyperref[\detokenize{p00_u5176_u5b83/_u767d_u8bdd_u804a_u658b_u5fd7_u5f02:id203}]{\sphinxcrossref{1.5.30   黎 氏}}}

\item {} 
\phantomsection\label{\detokenize{p00_u5176_u5b83/_u767d_u8bdd_u804a_u658b_u5fd7_u5f02:id714}}{\hyperref[\detokenize{p00_u5176_u5b83/_u767d_u8bdd_u804a_u658b_u5fd7_u5f02:id204}]{\sphinxcrossref{1.5.31   荷 花 三 娘 子}}}

\item {} 
\phantomsection\label{\detokenize{p00_u5176_u5b83/_u767d_u8bdd_u804a_u658b_u5fd7_u5f02:id715}}{\hyperref[\detokenize{p00_u5176_u5b83/_u767d_u8bdd_u804a_u658b_u5fd7_u5f02:id205}]{\sphinxcrossref{1.5.32   骂 鸭}}}

\item {} 
\phantomsection\label{\detokenize{p00_u5176_u5b83/_u767d_u8bdd_u804a_u658b_u5fd7_u5f02:id716}}{\hyperref[\detokenize{p00_u5176_u5b83/_u767d_u8bdd_u804a_u658b_u5fd7_u5f02:id206}]{\sphinxcrossref{1.5.33   柳 氏 子}}}

\item {} 
\phantomsection\label{\detokenize{p00_u5176_u5b83/_u767d_u8bdd_u804a_u658b_u5fd7_u5f02:id717}}{\hyperref[\detokenize{p00_u5176_u5b83/_u767d_u8bdd_u804a_u658b_u5fd7_u5f02:id207}]{\sphinxcrossref{1.5.34   上 仙}}}

\item {} 
\phantomsection\label{\detokenize{p00_u5176_u5b83/_u767d_u8bdd_u804a_u658b_u5fd7_u5f02:id718}}{\hyperref[\detokenize{p00_u5176_u5b83/_u767d_u8bdd_u804a_u658b_u5fd7_u5f02:id208}]{\sphinxcrossref{1.5.35   侯 静 山}}}

\item {} 
\phantomsection\label{\detokenize{p00_u5176_u5b83/_u767d_u8bdd_u804a_u658b_u5fd7_u5f02:id719}}{\hyperref[\detokenize{p00_u5176_u5b83/_u767d_u8bdd_u804a_u658b_u5fd7_u5f02:id209}]{\sphinxcrossref{1.5.36   钱 流}}}

\item {} 
\phantomsection\label{\detokenize{p00_u5176_u5b83/_u767d_u8bdd_u804a_u658b_u5fd7_u5f02:id720}}{\hyperref[\detokenize{p00_u5176_u5b83/_u767d_u8bdd_u804a_u658b_u5fd7_u5f02:id210}]{\sphinxcrossref{1.5.37   郭 生}}}

\item {} 
\phantomsection\label{\detokenize{p00_u5176_u5b83/_u767d_u8bdd_u804a_u658b_u5fd7_u5f02:id721}}{\hyperref[\detokenize{p00_u5176_u5b83/_u767d_u8bdd_u804a_u658b_u5fd7_u5f02:id211}]{\sphinxcrossref{1.5.38   金 生 色}}}

\item {} 
\phantomsection\label{\detokenize{p00_u5176_u5b83/_u767d_u8bdd_u804a_u658b_u5fd7_u5f02:id722}}{\hyperref[\detokenize{p00_u5176_u5b83/_u767d_u8bdd_u804a_u658b_u5fd7_u5f02:id212}]{\sphinxcrossref{1.5.39   彭 海 秋}}}

\item {} 
\phantomsection\label{\detokenize{p00_u5176_u5b83/_u767d_u8bdd_u804a_u658b_u5fd7_u5f02:id723}}{\hyperref[\detokenize{p00_u5176_u5b83/_u767d_u8bdd_u804a_u658b_u5fd7_u5f02:id213}]{\sphinxcrossref{1.5.40   堪 舆}}}

\item {} 
\phantomsection\label{\detokenize{p00_u5176_u5b83/_u767d_u8bdd_u804a_u658b_u5fd7_u5f02:id724}}{\hyperref[\detokenize{p00_u5176_u5b83/_u767d_u8bdd_u804a_u658b_u5fd7_u5f02:id214}]{\sphinxcrossref{1.5.41   窦 氏}}}

\item {} 
\phantomsection\label{\detokenize{p00_u5176_u5b83/_u767d_u8bdd_u804a_u658b_u5fd7_u5f02:id725}}{\hyperref[\detokenize{p00_u5176_u5b83/_u767d_u8bdd_u804a_u658b_u5fd7_u5f02:id215}]{\sphinxcrossref{1.5.42   梁 彦}}}

\item {} 
\phantomsection\label{\detokenize{p00_u5176_u5b83/_u767d_u8bdd_u804a_u658b_u5fd7_u5f02:id726}}{\hyperref[\detokenize{p00_u5176_u5b83/_u767d_u8bdd_u804a_u658b_u5fd7_u5f02:id216}]{\sphinxcrossref{1.5.43   龙 肉}}}

\end{itemize}

\item {} 
\phantomsection\label{\detokenize{p00_u5176_u5b83/_u767d_u8bdd_u804a_u658b_u5fd7_u5f02:id727}}{\hyperref[\detokenize{p00_u5176_u5b83/_u767d_u8bdd_u804a_u658b_u5fd7_u5f02:id217}]{\sphinxcrossref{1.6   卷 六}}}
\begin{itemize}
\item {} 
\phantomsection\label{\detokenize{p00_u5176_u5b83/_u767d_u8bdd_u804a_u658b_u5fd7_u5f02:id728}}{\hyperref[\detokenize{p00_u5176_u5b83/_u767d_u8bdd_u804a_u658b_u5fd7_u5f02:id218}]{\sphinxcrossref{1.6.1   潞 令}}}

\item {} 
\phantomsection\label{\detokenize{p00_u5176_u5b83/_u767d_u8bdd_u804a_u658b_u5fd7_u5f02:id729}}{\hyperref[\detokenize{p00_u5176_u5b83/_u767d_u8bdd_u804a_u658b_u5fd7_u5f02:id219}]{\sphinxcrossref{1.6.2   马 介 甫}}}

\item {} 
\phantomsection\label{\detokenize{p00_u5176_u5b83/_u767d_u8bdd_u804a_u658b_u5fd7_u5f02:id730}}{\hyperref[\detokenize{p00_u5176_u5b83/_u767d_u8bdd_u804a_u658b_u5fd7_u5f02:id220}]{\sphinxcrossref{1.6.3   魁 星}}}

\item {} 
\phantomsection\label{\detokenize{p00_u5176_u5b83/_u767d_u8bdd_u804a_u658b_u5fd7_u5f02:id731}}{\hyperref[\detokenize{p00_u5176_u5b83/_u767d_u8bdd_u804a_u658b_u5fd7_u5f02:id221}]{\sphinxcrossref{1.6.4   厍 将 军}}}

\item {} 
\phantomsection\label{\detokenize{p00_u5176_u5b83/_u767d_u8bdd_u804a_u658b_u5fd7_u5f02:id732}}{\hyperref[\detokenize{p00_u5176_u5b83/_u767d_u8bdd_u804a_u658b_u5fd7_u5f02:id222}]{\sphinxcrossref{1.6.5   绛 妃}}}

\item {} 
\phantomsection\label{\detokenize{p00_u5176_u5b83/_u767d_u8bdd_u804a_u658b_u5fd7_u5f02:id733}}{\hyperref[\detokenize{p00_u5176_u5b83/_u767d_u8bdd_u804a_u658b_u5fd7_u5f02:id223}]{\sphinxcrossref{1.6.6   河 间 生}}}

\item {} 
\phantomsection\label{\detokenize{p00_u5176_u5b83/_u767d_u8bdd_u804a_u658b_u5fd7_u5f02:id734}}{\hyperref[\detokenize{p00_u5176_u5b83/_u767d_u8bdd_u804a_u658b_u5fd7_u5f02:id224}]{\sphinxcrossref{1.6.7   云 翠 仙}}}

\item {} 
\phantomsection\label{\detokenize{p00_u5176_u5b83/_u767d_u8bdd_u804a_u658b_u5fd7_u5f02:id735}}{\hyperref[\detokenize{p00_u5176_u5b83/_u767d_u8bdd_u804a_u658b_u5fd7_u5f02:id225}]{\sphinxcrossref{1.6.8   跳 神}}}

\item {} 
\phantomsection\label{\detokenize{p00_u5176_u5b83/_u767d_u8bdd_u804a_u658b_u5fd7_u5f02:id736}}{\hyperref[\detokenize{p00_u5176_u5b83/_u767d_u8bdd_u804a_u658b_u5fd7_u5f02:id226}]{\sphinxcrossref{1.6.9   铁 布 衫 法}}}

\item {} 
\phantomsection\label{\detokenize{p00_u5176_u5b83/_u767d_u8bdd_u804a_u658b_u5fd7_u5f02:id737}}{\hyperref[\detokenize{p00_u5176_u5b83/_u767d_u8bdd_u804a_u658b_u5fd7_u5f02:id227}]{\sphinxcrossref{1.6.10   大 力 将 军}}}

\item {} 
\phantomsection\label{\detokenize{p00_u5176_u5b83/_u767d_u8bdd_u804a_u658b_u5fd7_u5f02:id738}}{\hyperref[\detokenize{p00_u5176_u5b83/_u767d_u8bdd_u804a_u658b_u5fd7_u5f02:id228}]{\sphinxcrossref{1.6.11   白 莲 教}}}

\item {} 
\phantomsection\label{\detokenize{p00_u5176_u5b83/_u767d_u8bdd_u804a_u658b_u5fd7_u5f02:id739}}{\hyperref[\detokenize{p00_u5176_u5b83/_u767d_u8bdd_u804a_u658b_u5fd7_u5f02:id229}]{\sphinxcrossref{1.6.12   颜 氏}}}

\item {} 
\phantomsection\label{\detokenize{p00_u5176_u5b83/_u767d_u8bdd_u804a_u658b_u5fd7_u5f02:id740}}{\hyperref[\detokenize{p00_u5176_u5b83/_u767d_u8bdd_u804a_u658b_u5fd7_u5f02:id230}]{\sphinxcrossref{1.6.13   杜 翁}}}

\item {} 
\phantomsection\label{\detokenize{p00_u5176_u5b83/_u767d_u8bdd_u804a_u658b_u5fd7_u5f02:id741}}{\hyperref[\detokenize{p00_u5176_u5b83/_u767d_u8bdd_u804a_u658b_u5fd7_u5f02:id231}]{\sphinxcrossref{1.6.14   小 谢}}}

\item {} 
\phantomsection\label{\detokenize{p00_u5176_u5b83/_u767d_u8bdd_u804a_u658b_u5fd7_u5f02:id742}}{\hyperref[\detokenize{p00_u5176_u5b83/_u767d_u8bdd_u804a_u658b_u5fd7_u5f02:id232}]{\sphinxcrossref{1.6.15   缢 鬼}}}

\item {} 
\phantomsection\label{\detokenize{p00_u5176_u5b83/_u767d_u8bdd_u804a_u658b_u5fd7_u5f02:id743}}{\hyperref[\detokenize{p00_u5176_u5b83/_u767d_u8bdd_u804a_u658b_u5fd7_u5f02:id233}]{\sphinxcrossref{1.6.16   吴 门 画 工}}}

\item {} 
\phantomsection\label{\detokenize{p00_u5176_u5b83/_u767d_u8bdd_u804a_u658b_u5fd7_u5f02:id744}}{\hyperref[\detokenize{p00_u5176_u5b83/_u767d_u8bdd_u804a_u658b_u5fd7_u5f02:id234}]{\sphinxcrossref{1.6.17   林 氏}}}

\item {} 
\phantomsection\label{\detokenize{p00_u5176_u5b83/_u767d_u8bdd_u804a_u658b_u5fd7_u5f02:id745}}{\hyperref[\detokenize{p00_u5176_u5b83/_u767d_u8bdd_u804a_u658b_u5fd7_u5f02:id235}]{\sphinxcrossref{1.6.18   胡 大 姑}}}

\item {} 
\phantomsection\label{\detokenize{p00_u5176_u5b83/_u767d_u8bdd_u804a_u658b_u5fd7_u5f02:id746}}{\hyperref[\detokenize{p00_u5176_u5b83/_u767d_u8bdd_u804a_u658b_u5fd7_u5f02:id236}]{\sphinxcrossref{1.6.19   细 侯}}}

\item {} 
\phantomsection\label{\detokenize{p00_u5176_u5b83/_u767d_u8bdd_u804a_u658b_u5fd7_u5f02:id747}}{\hyperref[\detokenize{p00_u5176_u5b83/_u767d_u8bdd_u804a_u658b_u5fd7_u5f02:id237}]{\sphinxcrossref{1.6.20   狼 三 则}}}

\item {} 
\phantomsection\label{\detokenize{p00_u5176_u5b83/_u767d_u8bdd_u804a_u658b_u5fd7_u5f02:id748}}{\hyperref[\detokenize{p00_u5176_u5b83/_u767d_u8bdd_u804a_u658b_u5fd7_u5f02:id238}]{\sphinxcrossref{1.6.21   美 人 首}}}

\item {} 
\phantomsection\label{\detokenize{p00_u5176_u5b83/_u767d_u8bdd_u804a_u658b_u5fd7_u5f02:id749}}{\hyperref[\detokenize{p00_u5176_u5b83/_u767d_u8bdd_u804a_u658b_u5fd7_u5f02:id239}]{\sphinxcrossref{1.6.22   刘 亮 采}}}

\item {} 
\phantomsection\label{\detokenize{p00_u5176_u5b83/_u767d_u8bdd_u804a_u658b_u5fd7_u5f02:id750}}{\hyperref[\detokenize{p00_u5176_u5b83/_u767d_u8bdd_u804a_u658b_u5fd7_u5f02:id240}]{\sphinxcrossref{1.6.23   蕙 芳}}}

\item {} 
\phantomsection\label{\detokenize{p00_u5176_u5b83/_u767d_u8bdd_u804a_u658b_u5fd7_u5f02:id751}}{\hyperref[\detokenize{p00_u5176_u5b83/_u767d_u8bdd_u804a_u658b_u5fd7_u5f02:id241}]{\sphinxcrossref{1.6.24   山 神}}}

\item {} 
\phantomsection\label{\detokenize{p00_u5176_u5b83/_u767d_u8bdd_u804a_u658b_u5fd7_u5f02:id752}}{\hyperref[\detokenize{p00_u5176_u5b83/_u767d_u8bdd_u804a_u658b_u5fd7_u5f02:id242}]{\sphinxcrossref{1.6.25   萧 七}}}

\item {} 
\phantomsection\label{\detokenize{p00_u5176_u5b83/_u767d_u8bdd_u804a_u658b_u5fd7_u5f02:id753}}{\hyperref[\detokenize{p00_u5176_u5b83/_u767d_u8bdd_u804a_u658b_u5fd7_u5f02:id243}]{\sphinxcrossref{1.6.26   乱 离 二 则}}}

\item {} 
\phantomsection\label{\detokenize{p00_u5176_u5b83/_u767d_u8bdd_u804a_u658b_u5fd7_u5f02:id754}}{\hyperref[\detokenize{p00_u5176_u5b83/_u767d_u8bdd_u804a_u658b_u5fd7_u5f02:id244}]{\sphinxcrossref{1.6.27   豢 蛇}}}

\item {} 
\phantomsection\label{\detokenize{p00_u5176_u5b83/_u767d_u8bdd_u804a_u658b_u5fd7_u5f02:id755}}{\hyperref[\detokenize{p00_u5176_u5b83/_u767d_u8bdd_u804a_u658b_u5fd7_u5f02:id245}]{\sphinxcrossref{1.6.28   雷 公}}}

\item {} 
\phantomsection\label{\detokenize{p00_u5176_u5b83/_u767d_u8bdd_u804a_u658b_u5fd7_u5f02:id756}}{\hyperref[\detokenize{p00_u5176_u5b83/_u767d_u8bdd_u804a_u658b_u5fd7_u5f02:id246}]{\sphinxcrossref{1.6.29   菱 角}}}

\item {} 
\phantomsection\label{\detokenize{p00_u5176_u5b83/_u767d_u8bdd_u804a_u658b_u5fd7_u5f02:id757}}{\hyperref[\detokenize{p00_u5176_u5b83/_u767d_u8bdd_u804a_u658b_u5fd7_u5f02:id247}]{\sphinxcrossref{1.6.30   饿 鬼}}}

\item {} 
\phantomsection\label{\detokenize{p00_u5176_u5b83/_u767d_u8bdd_u804a_u658b_u5fd7_u5f02:id758}}{\hyperref[\detokenize{p00_u5176_u5b83/_u767d_u8bdd_u804a_u658b_u5fd7_u5f02:id248}]{\sphinxcrossref{1.6.31   考 弊 司}}}

\item {} 
\phantomsection\label{\detokenize{p00_u5176_u5b83/_u767d_u8bdd_u804a_u658b_u5fd7_u5f02:id759}}{\hyperref[\detokenize{p00_u5176_u5b83/_u767d_u8bdd_u804a_u658b_u5fd7_u5f02:id249}]{\sphinxcrossref{1.6.32   阎 罗}}}

\item {} 
\phantomsection\label{\detokenize{p00_u5176_u5b83/_u767d_u8bdd_u804a_u658b_u5fd7_u5f02:id760}}{\hyperref[\detokenize{p00_u5176_u5b83/_u767d_u8bdd_u804a_u658b_u5fd7_u5f02:id250}]{\sphinxcrossref{1.6.33   大 人}}}

\item {} 
\phantomsection\label{\detokenize{p00_u5176_u5b83/_u767d_u8bdd_u804a_u658b_u5fd7_u5f02:id761}}{\hyperref[\detokenize{p00_u5176_u5b83/_u767d_u8bdd_u804a_u658b_u5fd7_u5f02:id251}]{\sphinxcrossref{1.6.34   向 杲}}}

\item {} 
\phantomsection\label{\detokenize{p00_u5176_u5b83/_u767d_u8bdd_u804a_u658b_u5fd7_u5f02:id762}}{\hyperref[\detokenize{p00_u5176_u5b83/_u767d_u8bdd_u804a_u658b_u5fd7_u5f02:id252}]{\sphinxcrossref{1.6.35   董 公 子}}}

\item {} 
\phantomsection\label{\detokenize{p00_u5176_u5b83/_u767d_u8bdd_u804a_u658b_u5fd7_u5f02:id763}}{\hyperref[\detokenize{p00_u5176_u5b83/_u767d_u8bdd_u804a_u658b_u5fd7_u5f02:id253}]{\sphinxcrossref{1.6.36   周 三}}}

\item {} 
\phantomsection\label{\detokenize{p00_u5176_u5b83/_u767d_u8bdd_u804a_u658b_u5fd7_u5f02:id764}}{\hyperref[\detokenize{p00_u5176_u5b83/_u767d_u8bdd_u804a_u658b_u5fd7_u5f02:id254}]{\sphinxcrossref{1.6.37   鸽 异}}}

\item {} 
\phantomsection\label{\detokenize{p00_u5176_u5b83/_u767d_u8bdd_u804a_u658b_u5fd7_u5f02:id765}}{\hyperref[\detokenize{p00_u5176_u5b83/_u767d_u8bdd_u804a_u658b_u5fd7_u5f02:id255}]{\sphinxcrossref{1.6.38   聂 政}}}

\item {} 
\phantomsection\label{\detokenize{p00_u5176_u5b83/_u767d_u8bdd_u804a_u658b_u5fd7_u5f02:id766}}{\hyperref[\detokenize{p00_u5176_u5b83/_u767d_u8bdd_u804a_u658b_u5fd7_u5f02:id256}]{\sphinxcrossref{1.6.39   冷 生}}}

\item {} 
\phantomsection\label{\detokenize{p00_u5176_u5b83/_u767d_u8bdd_u804a_u658b_u5fd7_u5f02:id767}}{\hyperref[\detokenize{p00_u5176_u5b83/_u767d_u8bdd_u804a_u658b_u5fd7_u5f02:id257}]{\sphinxcrossref{1.6.40   狐 惩 淫}}}

\item {} 
\phantomsection\label{\detokenize{p00_u5176_u5b83/_u767d_u8bdd_u804a_u658b_u5fd7_u5f02:id768}}{\hyperref[\detokenize{p00_u5176_u5b83/_u767d_u8bdd_u804a_u658b_u5fd7_u5f02:id258}]{\sphinxcrossref{1.6.41   山 市}}}

\item {} 
\phantomsection\label{\detokenize{p00_u5176_u5b83/_u767d_u8bdd_u804a_u658b_u5fd7_u5f02:id769}}{\hyperref[\detokenize{p00_u5176_u5b83/_u767d_u8bdd_u804a_u658b_u5fd7_u5f02:id259}]{\sphinxcrossref{1.6.42   江 城}}}

\item {} 
\phantomsection\label{\detokenize{p00_u5176_u5b83/_u767d_u8bdd_u804a_u658b_u5fd7_u5f02:id770}}{\hyperref[\detokenize{p00_u5176_u5b83/_u767d_u8bdd_u804a_u658b_u5fd7_u5f02:id260}]{\sphinxcrossref{1.6.43   孙 生}}}

\item {} 
\phantomsection\label{\detokenize{p00_u5176_u5b83/_u767d_u8bdd_u804a_u658b_u5fd7_u5f02:id771}}{\hyperref[\detokenize{p00_u5176_u5b83/_u767d_u8bdd_u804a_u658b_u5fd7_u5f02:id261}]{\sphinxcrossref{1.6.44   八 大 王}}}

\item {} 
\phantomsection\label{\detokenize{p00_u5176_u5b83/_u767d_u8bdd_u804a_u658b_u5fd7_u5f02:id772}}{\hyperref[\detokenize{p00_u5176_u5b83/_u767d_u8bdd_u804a_u658b_u5fd7_u5f02:id262}]{\sphinxcrossref{1.6.45   戏 缢}}}

\end{itemize}

\item {} 
\phantomsection\label{\detokenize{p00_u5176_u5b83/_u767d_u8bdd_u804a_u658b_u5fd7_u5f02:id773}}{\hyperref[\detokenize{p00_u5176_u5b83/_u767d_u8bdd_u804a_u658b_u5fd7_u5f02:id263}]{\sphinxcrossref{1.7   卷 七}}}
\begin{itemize}
\item {} 
\phantomsection\label{\detokenize{p00_u5176_u5b83/_u767d_u8bdd_u804a_u658b_u5fd7_u5f02:id774}}{\hyperref[\detokenize{p00_u5176_u5b83/_u767d_u8bdd_u804a_u658b_u5fd7_u5f02:id264}]{\sphinxcrossref{1.7.1   罗 祖}}}

\item {} 
\phantomsection\label{\detokenize{p00_u5176_u5b83/_u767d_u8bdd_u804a_u658b_u5fd7_u5f02:id775}}{\hyperref[\detokenize{p00_u5176_u5b83/_u767d_u8bdd_u804a_u658b_u5fd7_u5f02:id265}]{\sphinxcrossref{1.7.2   刘 姓}}}

\item {} 
\phantomsection\label{\detokenize{p00_u5176_u5b83/_u767d_u8bdd_u804a_u658b_u5fd7_u5f02:id776}}{\hyperref[\detokenize{p00_u5176_u5b83/_u767d_u8bdd_u804a_u658b_u5fd7_u5f02:id266}]{\sphinxcrossref{1.7.3   邵 女}}}

\item {} 
\phantomsection\label{\detokenize{p00_u5176_u5b83/_u767d_u8bdd_u804a_u658b_u5fd7_u5f02:id777}}{\hyperref[\detokenize{p00_u5176_u5b83/_u767d_u8bdd_u804a_u658b_u5fd7_u5f02:id267}]{\sphinxcrossref{1.7.4   巩 仙}}}

\item {} 
\phantomsection\label{\detokenize{p00_u5176_u5b83/_u767d_u8bdd_u804a_u658b_u5fd7_u5f02:id778}}{\hyperref[\detokenize{p00_u5176_u5b83/_u767d_u8bdd_u804a_u658b_u5fd7_u5f02:id268}]{\sphinxcrossref{1.7.5   二 商}}}

\item {} 
\phantomsection\label{\detokenize{p00_u5176_u5b83/_u767d_u8bdd_u804a_u658b_u5fd7_u5f02:id779}}{\hyperref[\detokenize{p00_u5176_u5b83/_u767d_u8bdd_u804a_u658b_u5fd7_u5f02:id269}]{\sphinxcrossref{1.7.6   沂 水 秀 才}}}

\item {} 
\phantomsection\label{\detokenize{p00_u5176_u5b83/_u767d_u8bdd_u804a_u658b_u5fd7_u5f02:id780}}{\hyperref[\detokenize{p00_u5176_u5b83/_u767d_u8bdd_u804a_u658b_u5fd7_u5f02:id270}]{\sphinxcrossref{1.7.7   梅 女}}}

\item {} 
\phantomsection\label{\detokenize{p00_u5176_u5b83/_u767d_u8bdd_u804a_u658b_u5fd7_u5f02:id781}}{\hyperref[\detokenize{p00_u5176_u5b83/_u767d_u8bdd_u804a_u658b_u5fd7_u5f02:id271}]{\sphinxcrossref{1.7.8   郭 秀 才}}}

\item {} 
\phantomsection\label{\detokenize{p00_u5176_u5b83/_u767d_u8bdd_u804a_u658b_u5fd7_u5f02:id782}}{\hyperref[\detokenize{p00_u5176_u5b83/_u767d_u8bdd_u804a_u658b_u5fd7_u5f02:id272}]{\sphinxcrossref{1.7.9   死 僧}}}

\item {} 
\phantomsection\label{\detokenize{p00_u5176_u5b83/_u767d_u8bdd_u804a_u658b_u5fd7_u5f02:id783}}{\hyperref[\detokenize{p00_u5176_u5b83/_u767d_u8bdd_u804a_u658b_u5fd7_u5f02:id273}]{\sphinxcrossref{1.7.10   阿 英}}}

\item {} 
\phantomsection\label{\detokenize{p00_u5176_u5b83/_u767d_u8bdd_u804a_u658b_u5fd7_u5f02:id784}}{\hyperref[\detokenize{p00_u5176_u5b83/_u767d_u8bdd_u804a_u658b_u5fd7_u5f02:id274}]{\sphinxcrossref{1.7.11   桔 树}}}

\item {} 
\phantomsection\label{\detokenize{p00_u5176_u5b83/_u767d_u8bdd_u804a_u658b_u5fd7_u5f02:id785}}{\hyperref[\detokenize{p00_u5176_u5b83/_u767d_u8bdd_u804a_u658b_u5fd7_u5f02:id275}]{\sphinxcrossref{1.7.12   赤 字}}}

\item {} 
\phantomsection\label{\detokenize{p00_u5176_u5b83/_u767d_u8bdd_u804a_u658b_u5fd7_u5f02:id786}}{\hyperref[\detokenize{p00_u5176_u5b83/_u767d_u8bdd_u804a_u658b_u5fd7_u5f02:id276}]{\sphinxcrossref{1.7.13   牛 成 章}}}

\item {} 
\phantomsection\label{\detokenize{p00_u5176_u5b83/_u767d_u8bdd_u804a_u658b_u5fd7_u5f02:id787}}{\hyperref[\detokenize{p00_u5176_u5b83/_u767d_u8bdd_u804a_u658b_u5fd7_u5f02:id277}]{\sphinxcrossref{1.7.14   青 娥}}}

\item {} 
\phantomsection\label{\detokenize{p00_u5176_u5b83/_u767d_u8bdd_u804a_u658b_u5fd7_u5f02:id788}}{\hyperref[\detokenize{p00_u5176_u5b83/_u767d_u8bdd_u804a_u658b_u5fd7_u5f02:id278}]{\sphinxcrossref{1.7.15   镜 听}}}

\item {} 
\phantomsection\label{\detokenize{p00_u5176_u5b83/_u767d_u8bdd_u804a_u658b_u5fd7_u5f02:id789}}{\hyperref[\detokenize{p00_u5176_u5b83/_u767d_u8bdd_u804a_u658b_u5fd7_u5f02:id279}]{\sphinxcrossref{1.7.16   牛 癀}}}

\item {} 
\phantomsection\label{\detokenize{p00_u5176_u5b83/_u767d_u8bdd_u804a_u658b_u5fd7_u5f02:id790}}{\hyperref[\detokenize{p00_u5176_u5b83/_u767d_u8bdd_u804a_u658b_u5fd7_u5f02:id280}]{\sphinxcrossref{1.7.17   金 姑 夫}}}

\item {} 
\phantomsection\label{\detokenize{p00_u5176_u5b83/_u767d_u8bdd_u804a_u658b_u5fd7_u5f02:id791}}{\hyperref[\detokenize{p00_u5176_u5b83/_u767d_u8bdd_u804a_u658b_u5fd7_u5f02:id281}]{\sphinxcrossref{1.7.18   梓 潼 令}}}

\item {} 
\phantomsection\label{\detokenize{p00_u5176_u5b83/_u767d_u8bdd_u804a_u658b_u5fd7_u5f02:id792}}{\hyperref[\detokenize{p00_u5176_u5b83/_u767d_u8bdd_u804a_u658b_u5fd7_u5f02:id282}]{\sphinxcrossref{1.7.19   鬼 津}}}

\item {} 
\phantomsection\label{\detokenize{p00_u5176_u5b83/_u767d_u8bdd_u804a_u658b_u5fd7_u5f02:id793}}{\hyperref[\detokenize{p00_u5176_u5b83/_u767d_u8bdd_u804a_u658b_u5fd7_u5f02:id283}]{\sphinxcrossref{1.7.20   仙 人 岛}}}

\item {} 
\phantomsection\label{\detokenize{p00_u5176_u5b83/_u767d_u8bdd_u804a_u658b_u5fd7_u5f02:id794}}{\hyperref[\detokenize{p00_u5176_u5b83/_u767d_u8bdd_u804a_u658b_u5fd7_u5f02:id284}]{\sphinxcrossref{1.7.21   阎 罗 薨}}}

\item {} 
\phantomsection\label{\detokenize{p00_u5176_u5b83/_u767d_u8bdd_u804a_u658b_u5fd7_u5f02:id795}}{\hyperref[\detokenize{p00_u5176_u5b83/_u767d_u8bdd_u804a_u658b_u5fd7_u5f02:id285}]{\sphinxcrossref{1.7.22   颠 道 人}}}

\item {} 
\phantomsection\label{\detokenize{p00_u5176_u5b83/_u767d_u8bdd_u804a_u658b_u5fd7_u5f02:id796}}{\hyperref[\detokenize{p00_u5176_u5b83/_u767d_u8bdd_u804a_u658b_u5fd7_u5f02:id286}]{\sphinxcrossref{1.7.23   胡 四 娘}}}

\item {} 
\phantomsection\label{\detokenize{p00_u5176_u5b83/_u767d_u8bdd_u804a_u658b_u5fd7_u5f02:id797}}{\hyperref[\detokenize{p00_u5176_u5b83/_u767d_u8bdd_u804a_u658b_u5fd7_u5f02:id287}]{\sphinxcrossref{1.7.24   僧 术}}}

\item {} 
\phantomsection\label{\detokenize{p00_u5176_u5b83/_u767d_u8bdd_u804a_u658b_u5fd7_u5f02:id798}}{\hyperref[\detokenize{p00_u5176_u5b83/_u767d_u8bdd_u804a_u658b_u5fd7_u5f02:id288}]{\sphinxcrossref{1.7.25   禄 数}}}

\item {} 
\phantomsection\label{\detokenize{p00_u5176_u5b83/_u767d_u8bdd_u804a_u658b_u5fd7_u5f02:id799}}{\hyperref[\detokenize{p00_u5176_u5b83/_u767d_u8bdd_u804a_u658b_u5fd7_u5f02:id289}]{\sphinxcrossref{1.7.26   柳 生}}}

\item {} 
\phantomsection\label{\detokenize{p00_u5176_u5b83/_u767d_u8bdd_u804a_u658b_u5fd7_u5f02:id800}}{\hyperref[\detokenize{p00_u5176_u5b83/_u767d_u8bdd_u804a_u658b_u5fd7_u5f02:id290}]{\sphinxcrossref{1.7.27   冤 狱}}}

\item {} 
\phantomsection\label{\detokenize{p00_u5176_u5b83/_u767d_u8bdd_u804a_u658b_u5fd7_u5f02:id801}}{\hyperref[\detokenize{p00_u5176_u5b83/_u767d_u8bdd_u804a_u658b_u5fd7_u5f02:id291}]{\sphinxcrossref{1.7.28   鬼 令}}}

\item {} 
\phantomsection\label{\detokenize{p00_u5176_u5b83/_u767d_u8bdd_u804a_u658b_u5fd7_u5f02:id802}}{\hyperref[\detokenize{p00_u5176_u5b83/_u767d_u8bdd_u804a_u658b_u5fd7_u5f02:id292}]{\sphinxcrossref{1.7.29   甄 后}}}

\item {} 
\phantomsection\label{\detokenize{p00_u5176_u5b83/_u767d_u8bdd_u804a_u658b_u5fd7_u5f02:id803}}{\hyperref[\detokenize{p00_u5176_u5b83/_u767d_u8bdd_u804a_u658b_u5fd7_u5f02:id293}]{\sphinxcrossref{1.7.30   宦 娘}}}

\item {} 
\phantomsection\label{\detokenize{p00_u5176_u5b83/_u767d_u8bdd_u804a_u658b_u5fd7_u5f02:id804}}{\hyperref[\detokenize{p00_u5176_u5b83/_u767d_u8bdd_u804a_u658b_u5fd7_u5f02:id294}]{\sphinxcrossref{1.7.31   阿 绣}}}

\item {} 
\phantomsection\label{\detokenize{p00_u5176_u5b83/_u767d_u8bdd_u804a_u658b_u5fd7_u5f02:id805}}{\hyperref[\detokenize{p00_u5176_u5b83/_u767d_u8bdd_u804a_u658b_u5fd7_u5f02:id295}]{\sphinxcrossref{1.7.32   杨 疤 眼}}}

\item {} 
\phantomsection\label{\detokenize{p00_u5176_u5b83/_u767d_u8bdd_u804a_u658b_u5fd7_u5f02:id806}}{\hyperref[\detokenize{p00_u5176_u5b83/_u767d_u8bdd_u804a_u658b_u5fd7_u5f02:id296}]{\sphinxcrossref{1.7.33   小 翠}}}

\item {} 
\phantomsection\label{\detokenize{p00_u5176_u5b83/_u767d_u8bdd_u804a_u658b_u5fd7_u5f02:id807}}{\hyperref[\detokenize{p00_u5176_u5b83/_u767d_u8bdd_u804a_u658b_u5fd7_u5f02:id297}]{\sphinxcrossref{1.7.34   金 和 尚}}}

\item {} 
\phantomsection\label{\detokenize{p00_u5176_u5b83/_u767d_u8bdd_u804a_u658b_u5fd7_u5f02:id808}}{\hyperref[\detokenize{p00_u5176_u5b83/_u767d_u8bdd_u804a_u658b_u5fd7_u5f02:id298}]{\sphinxcrossref{1.7.35   龙 戏 蛛}}}

\item {} 
\phantomsection\label{\detokenize{p00_u5176_u5b83/_u767d_u8bdd_u804a_u658b_u5fd7_u5f02:id809}}{\hyperref[\detokenize{p00_u5176_u5b83/_u767d_u8bdd_u804a_u658b_u5fd7_u5f02:id299}]{\sphinxcrossref{1.7.36   商 妇}}}

\item {} 
\phantomsection\label{\detokenize{p00_u5176_u5b83/_u767d_u8bdd_u804a_u658b_u5fd7_u5f02:id810}}{\hyperref[\detokenize{p00_u5176_u5b83/_u767d_u8bdd_u804a_u658b_u5fd7_u5f02:id300}]{\sphinxcrossref{1.7.37   阎 罗 宴}}}

\item {} 
\phantomsection\label{\detokenize{p00_u5176_u5b83/_u767d_u8bdd_u804a_u658b_u5fd7_u5f02:id811}}{\hyperref[\detokenize{p00_u5176_u5b83/_u767d_u8bdd_u804a_u658b_u5fd7_u5f02:id301}]{\sphinxcrossref{1.7.38   役 鬼}}}

\item {} 
\phantomsection\label{\detokenize{p00_u5176_u5b83/_u767d_u8bdd_u804a_u658b_u5fd7_u5f02:id812}}{\hyperref[\detokenize{p00_u5176_u5b83/_u767d_u8bdd_u804a_u658b_u5fd7_u5f02:id302}]{\sphinxcrossref{1.7.39   细 柳}}}

\end{itemize}

\item {} 
\phantomsection\label{\detokenize{p00_u5176_u5b83/_u767d_u8bdd_u804a_u658b_u5fd7_u5f02:id813}}{\hyperref[\detokenize{p00_u5176_u5b83/_u767d_u8bdd_u804a_u658b_u5fd7_u5f02:id303}]{\sphinxcrossref{1.8   卷 八}}}
\begin{itemize}
\item {} 
\phantomsection\label{\detokenize{p00_u5176_u5b83/_u767d_u8bdd_u804a_u658b_u5fd7_u5f02:id814}}{\hyperref[\detokenize{p00_u5176_u5b83/_u767d_u8bdd_u804a_u658b_u5fd7_u5f02:id304}]{\sphinxcrossref{1.8.1   画 马}}}

\item {} 
\phantomsection\label{\detokenize{p00_u5176_u5b83/_u767d_u8bdd_u804a_u658b_u5fd7_u5f02:id815}}{\hyperref[\detokenize{p00_u5176_u5b83/_u767d_u8bdd_u804a_u658b_u5fd7_u5f02:id305}]{\sphinxcrossref{1.8.2   局 诈}}}

\item {} 
\phantomsection\label{\detokenize{p00_u5176_u5b83/_u767d_u8bdd_u804a_u658b_u5fd7_u5f02:id816}}{\hyperref[\detokenize{p00_u5176_u5b83/_u767d_u8bdd_u804a_u658b_u5fd7_u5f02:id306}]{\sphinxcrossref{1.8.3   放 蝶}}}

\item {} 
\phantomsection\label{\detokenize{p00_u5176_u5b83/_u767d_u8bdd_u804a_u658b_u5fd7_u5f02:id817}}{\hyperref[\detokenize{p00_u5176_u5b83/_u767d_u8bdd_u804a_u658b_u5fd7_u5f02:id307}]{\sphinxcrossref{1.8.4   男 生 子}}}

\item {} 
\phantomsection\label{\detokenize{p00_u5176_u5b83/_u767d_u8bdd_u804a_u658b_u5fd7_u5f02:id818}}{\hyperref[\detokenize{p00_u5176_u5b83/_u767d_u8bdd_u804a_u658b_u5fd7_u5f02:id308}]{\sphinxcrossref{1.8.5   钟 生}}}

\item {} 
\phantomsection\label{\detokenize{p00_u5176_u5b83/_u767d_u8bdd_u804a_u658b_u5fd7_u5f02:id819}}{\hyperref[\detokenize{p00_u5176_u5b83/_u767d_u8bdd_u804a_u658b_u5fd7_u5f02:id309}]{\sphinxcrossref{1.8.6   鬼 妻}}}

\item {} 
\phantomsection\label{\detokenize{p00_u5176_u5b83/_u767d_u8bdd_u804a_u658b_u5fd7_u5f02:id820}}{\hyperref[\detokenize{p00_u5176_u5b83/_u767d_u8bdd_u804a_u658b_u5fd7_u5f02:id310}]{\sphinxcrossref{1.8.7   黄 将 军}}}

\item {} 
\phantomsection\label{\detokenize{p00_u5176_u5b83/_u767d_u8bdd_u804a_u658b_u5fd7_u5f02:id821}}{\hyperref[\detokenize{p00_u5176_u5b83/_u767d_u8bdd_u804a_u658b_u5fd7_u5f02:id311}]{\sphinxcrossref{1.8.8   三 朝 元 老}}}

\item {} 
\phantomsection\label{\detokenize{p00_u5176_u5b83/_u767d_u8bdd_u804a_u658b_u5fd7_u5f02:id822}}{\hyperref[\detokenize{p00_u5176_u5b83/_u767d_u8bdd_u804a_u658b_u5fd7_u5f02:id312}]{\sphinxcrossref{1.8.9   医 术}}}

\item {} 
\phantomsection\label{\detokenize{p00_u5176_u5b83/_u767d_u8bdd_u804a_u658b_u5fd7_u5f02:id823}}{\hyperref[\detokenize{p00_u5176_u5b83/_u767d_u8bdd_u804a_u658b_u5fd7_u5f02:id313}]{\sphinxcrossref{1.8.10   藏 虱}}}

\item {} 
\phantomsection\label{\detokenize{p00_u5176_u5b83/_u767d_u8bdd_u804a_u658b_u5fd7_u5f02:id824}}{\hyperref[\detokenize{p00_u5176_u5b83/_u767d_u8bdd_u804a_u658b_u5fd7_u5f02:id314}]{\sphinxcrossref{1.8.11   梦 狼}}}

\item {} 
\phantomsection\label{\detokenize{p00_u5176_u5b83/_u767d_u8bdd_u804a_u658b_u5fd7_u5f02:id825}}{\hyperref[\detokenize{p00_u5176_u5b83/_u767d_u8bdd_u804a_u658b_u5fd7_u5f02:id315}]{\sphinxcrossref{1.8.12   夜 明}}}

\item {} 
\phantomsection\label{\detokenize{p00_u5176_u5b83/_u767d_u8bdd_u804a_u658b_u5fd7_u5f02:id826}}{\hyperref[\detokenize{p00_u5176_u5b83/_u767d_u8bdd_u804a_u658b_u5fd7_u5f02:id316}]{\sphinxcrossref{1.8.13   夏 雪}}}

\item {} 
\phantomsection\label{\detokenize{p00_u5176_u5b83/_u767d_u8bdd_u804a_u658b_u5fd7_u5f02:id827}}{\hyperref[\detokenize{p00_u5176_u5b83/_u767d_u8bdd_u804a_u658b_u5fd7_u5f02:id317}]{\sphinxcrossref{1.8.14   化 男}}}

\item {} 
\phantomsection\label{\detokenize{p00_u5176_u5b83/_u767d_u8bdd_u804a_u658b_u5fd7_u5f02:id828}}{\hyperref[\detokenize{p00_u5176_u5b83/_u767d_u8bdd_u804a_u658b_u5fd7_u5f02:id318}]{\sphinxcrossref{1.8.15   禽 侠}}}

\item {} 
\phantomsection\label{\detokenize{p00_u5176_u5b83/_u767d_u8bdd_u804a_u658b_u5fd7_u5f02:id829}}{\hyperref[\detokenize{p00_u5176_u5b83/_u767d_u8bdd_u804a_u658b_u5fd7_u5f02:id319}]{\sphinxcrossref{1.8.16   鸿}}}

\item {} 
\phantomsection\label{\detokenize{p00_u5176_u5b83/_u767d_u8bdd_u804a_u658b_u5fd7_u5f02:id830}}{\hyperref[\detokenize{p00_u5176_u5b83/_u767d_u8bdd_u804a_u658b_u5fd7_u5f02:id320}]{\sphinxcrossref{1.8.17   象}}}

\item {} 
\phantomsection\label{\detokenize{p00_u5176_u5b83/_u767d_u8bdd_u804a_u658b_u5fd7_u5f02:id831}}{\hyperref[\detokenize{p00_u5176_u5b83/_u767d_u8bdd_u804a_u658b_u5fd7_u5f02:id321}]{\sphinxcrossref{1.8.18   负 尸}}}

\item {} 
\phantomsection\label{\detokenize{p00_u5176_u5b83/_u767d_u8bdd_u804a_u658b_u5fd7_u5f02:id832}}{\hyperref[\detokenize{p00_u5176_u5b83/_u767d_u8bdd_u804a_u658b_u5fd7_u5f02:id322}]{\sphinxcrossref{1.8.19   紫 花 和 尚}}}

\item {} 
\phantomsection\label{\detokenize{p00_u5176_u5b83/_u767d_u8bdd_u804a_u658b_u5fd7_u5f02:id833}}{\hyperref[\detokenize{p00_u5176_u5b83/_u767d_u8bdd_u804a_u658b_u5fd7_u5f02:id323}]{\sphinxcrossref{1.8.20   周 克 昌}}}

\item {} 
\phantomsection\label{\detokenize{p00_u5176_u5b83/_u767d_u8bdd_u804a_u658b_u5fd7_u5f02:id834}}{\hyperref[\detokenize{p00_u5176_u5b83/_u767d_u8bdd_u804a_u658b_u5fd7_u5f02:id324}]{\sphinxcrossref{1.8.21   嫦 娥}}}

\item {} 
\phantomsection\label{\detokenize{p00_u5176_u5b83/_u767d_u8bdd_u804a_u658b_u5fd7_u5f02:id835}}{\hyperref[\detokenize{p00_u5176_u5b83/_u767d_u8bdd_u804a_u658b_u5fd7_u5f02:id325}]{\sphinxcrossref{1.8.22   鞠 药 如}}}

\item {} 
\phantomsection\label{\detokenize{p00_u5176_u5b83/_u767d_u8bdd_u804a_u658b_u5fd7_u5f02:id836}}{\hyperref[\detokenize{p00_u5176_u5b83/_u767d_u8bdd_u804a_u658b_u5fd7_u5f02:id326}]{\sphinxcrossref{1.8.23   褚 生}}}

\item {} 
\phantomsection\label{\detokenize{p00_u5176_u5b83/_u767d_u8bdd_u804a_u658b_u5fd7_u5f02:id837}}{\hyperref[\detokenize{p00_u5176_u5b83/_u767d_u8bdd_u804a_u658b_u5fd7_u5f02:id327}]{\sphinxcrossref{1.8.24   盗 户}}}

\item {} 
\phantomsection\label{\detokenize{p00_u5176_u5b83/_u767d_u8bdd_u804a_u658b_u5fd7_u5f02:id838}}{\hyperref[\detokenize{p00_u5176_u5b83/_u767d_u8bdd_u804a_u658b_u5fd7_u5f02:id328}]{\sphinxcrossref{1.8.25   某 乙}}}

\item {} 
\phantomsection\label{\detokenize{p00_u5176_u5b83/_u767d_u8bdd_u804a_u658b_u5fd7_u5f02:id839}}{\hyperref[\detokenize{p00_u5176_u5b83/_u767d_u8bdd_u804a_u658b_u5fd7_u5f02:id329}]{\sphinxcrossref{1.8.26   霍 女}}}

\item {} 
\phantomsection\label{\detokenize{p00_u5176_u5b83/_u767d_u8bdd_u804a_u658b_u5fd7_u5f02:id840}}{\hyperref[\detokenize{p00_u5176_u5b83/_u767d_u8bdd_u804a_u658b_u5fd7_u5f02:id330}]{\sphinxcrossref{1.8.27   司 文 郎}}}

\item {} 
\phantomsection\label{\detokenize{p00_u5176_u5b83/_u767d_u8bdd_u804a_u658b_u5fd7_u5f02:id841}}{\hyperref[\detokenize{p00_u5176_u5b83/_u767d_u8bdd_u804a_u658b_u5fd7_u5f02:id331}]{\sphinxcrossref{1.8.28   丑 狐}}}

\item {} 
\phantomsection\label{\detokenize{p00_u5176_u5b83/_u767d_u8bdd_u804a_u658b_u5fd7_u5f02:id842}}{\hyperref[\detokenize{p00_u5176_u5b83/_u767d_u8bdd_u804a_u658b_u5fd7_u5f02:id332}]{\sphinxcrossref{1.8.29   吕 无 病}}}

\item {} 
\phantomsection\label{\detokenize{p00_u5176_u5b83/_u767d_u8bdd_u804a_u658b_u5fd7_u5f02:id843}}{\hyperref[\detokenize{p00_u5176_u5b83/_u767d_u8bdd_u804a_u658b_u5fd7_u5f02:id333}]{\sphinxcrossref{1.8.30   钱 卜 巫}}}

\item {} 
\phantomsection\label{\detokenize{p00_u5176_u5b83/_u767d_u8bdd_u804a_u658b_u5fd7_u5f02:id844}}{\hyperref[\detokenize{p00_u5176_u5b83/_u767d_u8bdd_u804a_u658b_u5fd7_u5f02:id334}]{\sphinxcrossref{1.8.31   姚 安}}}

\item {} 
\phantomsection\label{\detokenize{p00_u5176_u5b83/_u767d_u8bdd_u804a_u658b_u5fd7_u5f02:id845}}{\hyperref[\detokenize{p00_u5176_u5b83/_u767d_u8bdd_u804a_u658b_u5fd7_u5f02:id335}]{\sphinxcrossref{1.8.32   采 薇 翁}}}

\item {} 
\phantomsection\label{\detokenize{p00_u5176_u5b83/_u767d_u8bdd_u804a_u658b_u5fd7_u5f02:id846}}{\hyperref[\detokenize{p00_u5176_u5b83/_u767d_u8bdd_u804a_u658b_u5fd7_u5f02:id336}]{\sphinxcrossref{1.8.33   崔 猛}}}

\item {} 
\phantomsection\label{\detokenize{p00_u5176_u5b83/_u767d_u8bdd_u804a_u658b_u5fd7_u5f02:id847}}{\hyperref[\detokenize{p00_u5176_u5b83/_u767d_u8bdd_u804a_u658b_u5fd7_u5f02:id337}]{\sphinxcrossref{1.8.34   诗 谳}}}

\item {} 
\phantomsection\label{\detokenize{p00_u5176_u5b83/_u767d_u8bdd_u804a_u658b_u5fd7_u5f02:id848}}{\hyperref[\detokenize{p00_u5176_u5b83/_u767d_u8bdd_u804a_u658b_u5fd7_u5f02:id338}]{\sphinxcrossref{1.8.35   鹿 草}}}

\item {} 
\phantomsection\label{\detokenize{p00_u5176_u5b83/_u767d_u8bdd_u804a_u658b_u5fd7_u5f02:id849}}{\hyperref[\detokenize{p00_u5176_u5b83/_u767d_u8bdd_u804a_u658b_u5fd7_u5f02:id339}]{\sphinxcrossref{1.8.36   小 棺}}}

\item {} 
\phantomsection\label{\detokenize{p00_u5176_u5b83/_u767d_u8bdd_u804a_u658b_u5fd7_u5f02:id850}}{\hyperref[\detokenize{p00_u5176_u5b83/_u767d_u8bdd_u804a_u658b_u5fd7_u5f02:id340}]{\sphinxcrossref{1.8.37   邢 子 仪}}}

\item {} 
\phantomsection\label{\detokenize{p00_u5176_u5b83/_u767d_u8bdd_u804a_u658b_u5fd7_u5f02:id851}}{\hyperref[\detokenize{p00_u5176_u5b83/_u767d_u8bdd_u804a_u658b_u5fd7_u5f02:id341}]{\sphinxcrossref{1.8.38   李 生}}}

\item {} 
\phantomsection\label{\detokenize{p00_u5176_u5b83/_u767d_u8bdd_u804a_u658b_u5fd7_u5f02:id852}}{\hyperref[\detokenize{p00_u5176_u5b83/_u767d_u8bdd_u804a_u658b_u5fd7_u5f02:id342}]{\sphinxcrossref{1.8.39   陆 押 官}}}

\item {} 
\phantomsection\label{\detokenize{p00_u5176_u5b83/_u767d_u8bdd_u804a_u658b_u5fd7_u5f02:id853}}{\hyperref[\detokenize{p00_u5176_u5b83/_u767d_u8bdd_u804a_u658b_u5fd7_u5f02:id343}]{\sphinxcrossref{1.8.40   蒋 太 史}}}

\item {} 
\phantomsection\label{\detokenize{p00_u5176_u5b83/_u767d_u8bdd_u804a_u658b_u5fd7_u5f02:id854}}{\hyperref[\detokenize{p00_u5176_u5b83/_u767d_u8bdd_u804a_u658b_u5fd7_u5f02:id344}]{\sphinxcrossref{1.8.41   邵 士 梅}}}

\item {} 
\phantomsection\label{\detokenize{p00_u5176_u5b83/_u767d_u8bdd_u804a_u658b_u5fd7_u5f02:id855}}{\hyperref[\detokenize{p00_u5176_u5b83/_u767d_u8bdd_u804a_u658b_u5fd7_u5f02:id345}]{\sphinxcrossref{1.8.42   顾 生}}}

\item {} 
\phantomsection\label{\detokenize{p00_u5176_u5b83/_u767d_u8bdd_u804a_u658b_u5fd7_u5f02:id856}}{\hyperref[\detokenize{p00_u5176_u5b83/_u767d_u8bdd_u804a_u658b_u5fd7_u5f02:id346}]{\sphinxcrossref{1.8.43   陈 锡 九}}}

\end{itemize}

\item {} 
\phantomsection\label{\detokenize{p00_u5176_u5b83/_u767d_u8bdd_u804a_u658b_u5fd7_u5f02:id857}}{\hyperref[\detokenize{p00_u5176_u5b83/_u767d_u8bdd_u804a_u658b_u5fd7_u5f02:id347}]{\sphinxcrossref{1.9   卷 九}}}
\begin{itemize}
\item {} 
\phantomsection\label{\detokenize{p00_u5176_u5b83/_u767d_u8bdd_u804a_u658b_u5fd7_u5f02:id858}}{\hyperref[\detokenize{p00_u5176_u5b83/_u767d_u8bdd_u804a_u658b_u5fd7_u5f02:id348}]{\sphinxcrossref{1.9.1   邵 临 淄}}}

\item {} 
\phantomsection\label{\detokenize{p00_u5176_u5b83/_u767d_u8bdd_u804a_u658b_u5fd7_u5f02:id859}}{\hyperref[\detokenize{p00_u5176_u5b83/_u767d_u8bdd_u804a_u658b_u5fd7_u5f02:id349}]{\sphinxcrossref{1.9.2   于 去 恶}}}

\item {} 
\phantomsection\label{\detokenize{p00_u5176_u5b83/_u767d_u8bdd_u804a_u658b_u5fd7_u5f02:id860}}{\hyperref[\detokenize{p00_u5176_u5b83/_u767d_u8bdd_u804a_u658b_u5fd7_u5f02:id350}]{\sphinxcrossref{1.9.3   狂 生}}}

\item {} 
\phantomsection\label{\detokenize{p00_u5176_u5b83/_u767d_u8bdd_u804a_u658b_u5fd7_u5f02:id861}}{\hyperref[\detokenize{p00_u5176_u5b83/_u767d_u8bdd_u804a_u658b_u5fd7_u5f02:id351}]{\sphinxcrossref{1.9.4   澂 俗}}}

\item {} 
\phantomsection\label{\detokenize{p00_u5176_u5b83/_u767d_u8bdd_u804a_u658b_u5fd7_u5f02:id862}}{\hyperref[\detokenize{p00_u5176_u5b83/_u767d_u8bdd_u804a_u658b_u5fd7_u5f02:id352}]{\sphinxcrossref{1.9.5   凤 仙}}}

\item {} 
\phantomsection\label{\detokenize{p00_u5176_u5b83/_u767d_u8bdd_u804a_u658b_u5fd7_u5f02:id863}}{\hyperref[\detokenize{p00_u5176_u5b83/_u767d_u8bdd_u804a_u658b_u5fd7_u5f02:id353}]{\sphinxcrossref{1.9.6   佟 客}}}

\item {} 
\phantomsection\label{\detokenize{p00_u5176_u5b83/_u767d_u8bdd_u804a_u658b_u5fd7_u5f02:id864}}{\hyperref[\detokenize{p00_u5176_u5b83/_u767d_u8bdd_u804a_u658b_u5fd7_u5f02:id354}]{\sphinxcrossref{1.9.7   辽 阳 军}}}

\item {} 
\phantomsection\label{\detokenize{p00_u5176_u5b83/_u767d_u8bdd_u804a_u658b_u5fd7_u5f02:id865}}{\hyperref[\detokenize{p00_u5176_u5b83/_u767d_u8bdd_u804a_u658b_u5fd7_u5f02:id355}]{\sphinxcrossref{1.9.8   张 贡 士}}}

\item {} 
\phantomsection\label{\detokenize{p00_u5176_u5b83/_u767d_u8bdd_u804a_u658b_u5fd7_u5f02:id866}}{\hyperref[\detokenize{p00_u5176_u5b83/_u767d_u8bdd_u804a_u658b_u5fd7_u5f02:id356}]{\sphinxcrossref{1.9.9   爱 奴}}}

\item {} 
\phantomsection\label{\detokenize{p00_u5176_u5b83/_u767d_u8bdd_u804a_u658b_u5fd7_u5f02:id867}}{\hyperref[\detokenize{p00_u5176_u5b83/_u767d_u8bdd_u804a_u658b_u5fd7_u5f02:id357}]{\sphinxcrossref{1.9.10   单 父 宰}}}

\item {} 
\phantomsection\label{\detokenize{p00_u5176_u5b83/_u767d_u8bdd_u804a_u658b_u5fd7_u5f02:id868}}{\hyperref[\detokenize{p00_u5176_u5b83/_u767d_u8bdd_u804a_u658b_u5fd7_u5f02:id358}]{\sphinxcrossref{1.9.11   孙 必 振}}}

\item {} 
\phantomsection\label{\detokenize{p00_u5176_u5b83/_u767d_u8bdd_u804a_u658b_u5fd7_u5f02:id869}}{\hyperref[\detokenize{p00_u5176_u5b83/_u767d_u8bdd_u804a_u658b_u5fd7_u5f02:id359}]{\sphinxcrossref{1.9.12   邑 人}}}

\item {} 
\phantomsection\label{\detokenize{p00_u5176_u5b83/_u767d_u8bdd_u804a_u658b_u5fd7_u5f02:id870}}{\hyperref[\detokenize{p00_u5176_u5b83/_u767d_u8bdd_u804a_u658b_u5fd7_u5f02:id360}]{\sphinxcrossref{1.9.13   元 宝}}}

\item {} 
\phantomsection\label{\detokenize{p00_u5176_u5b83/_u767d_u8bdd_u804a_u658b_u5fd7_u5f02:id871}}{\hyperref[\detokenize{p00_u5176_u5b83/_u767d_u8bdd_u804a_u658b_u5fd7_u5f02:id361}]{\sphinxcrossref{1.9.14   研 石}}}

\item {} 
\phantomsection\label{\detokenize{p00_u5176_u5b83/_u767d_u8bdd_u804a_u658b_u5fd7_u5f02:id872}}{\hyperref[\detokenize{p00_u5176_u5b83/_u767d_u8bdd_u804a_u658b_u5fd7_u5f02:id362}]{\sphinxcrossref{1.9.15   武 夷}}}

\item {} 
\phantomsection\label{\detokenize{p00_u5176_u5b83/_u767d_u8bdd_u804a_u658b_u5fd7_u5f02:id873}}{\hyperref[\detokenize{p00_u5176_u5b83/_u767d_u8bdd_u804a_u658b_u5fd7_u5f02:id363}]{\sphinxcrossref{1.9.16   大 鼠}}}

\item {} 
\phantomsection\label{\detokenize{p00_u5176_u5b83/_u767d_u8bdd_u804a_u658b_u5fd7_u5f02:id874}}{\hyperref[\detokenize{p00_u5176_u5b83/_u767d_u8bdd_u804a_u658b_u5fd7_u5f02:id364}]{\sphinxcrossref{1.9.17   张 不 量}}}

\item {} 
\phantomsection\label{\detokenize{p00_u5176_u5b83/_u767d_u8bdd_u804a_u658b_u5fd7_u5f02:id875}}{\hyperref[\detokenize{p00_u5176_u5b83/_u767d_u8bdd_u804a_u658b_u5fd7_u5f02:id365}]{\sphinxcrossref{1.9.18   牧 竖}}}

\item {} 
\phantomsection\label{\detokenize{p00_u5176_u5b83/_u767d_u8bdd_u804a_u658b_u5fd7_u5f02:id876}}{\hyperref[\detokenize{p00_u5176_u5b83/_u767d_u8bdd_u804a_u658b_u5fd7_u5f02:id366}]{\sphinxcrossref{1.9.19   富 翁}}}

\item {} 
\phantomsection\label{\detokenize{p00_u5176_u5b83/_u767d_u8bdd_u804a_u658b_u5fd7_u5f02:id877}}{\hyperref[\detokenize{p00_u5176_u5b83/_u767d_u8bdd_u804a_u658b_u5fd7_u5f02:id367}]{\sphinxcrossref{1.9.20   王 司 马}}}

\item {} 
\phantomsection\label{\detokenize{p00_u5176_u5b83/_u767d_u8bdd_u804a_u658b_u5fd7_u5f02:id878}}{\hyperref[\detokenize{p00_u5176_u5b83/_u767d_u8bdd_u804a_u658b_u5fd7_u5f02:id368}]{\sphinxcrossref{1.9.21   岳 神}}}

\item {} 
\phantomsection\label{\detokenize{p00_u5176_u5b83/_u767d_u8bdd_u804a_u658b_u5fd7_u5f02:id879}}{\hyperref[\detokenize{p00_u5176_u5b83/_u767d_u8bdd_u804a_u658b_u5fd7_u5f02:id369}]{\sphinxcrossref{1.9.22   小 梅}}}

\item {} 
\phantomsection\label{\detokenize{p00_u5176_u5b83/_u767d_u8bdd_u804a_u658b_u5fd7_u5f02:id880}}{\hyperref[\detokenize{p00_u5176_u5b83/_u767d_u8bdd_u804a_u658b_u5fd7_u5f02:id370}]{\sphinxcrossref{1.9.23   药 僧}}}

\item {} 
\phantomsection\label{\detokenize{p00_u5176_u5b83/_u767d_u8bdd_u804a_u658b_u5fd7_u5f02:id881}}{\hyperref[\detokenize{p00_u5176_u5b83/_u767d_u8bdd_u804a_u658b_u5fd7_u5f02:id371}]{\sphinxcrossref{1.9.24   于 中 丞}}}

\item {} 
\phantomsection\label{\detokenize{p00_u5176_u5b83/_u767d_u8bdd_u804a_u658b_u5fd7_u5f02:id882}}{\hyperref[\detokenize{p00_u5176_u5b83/_u767d_u8bdd_u804a_u658b_u5fd7_u5f02:id372}]{\sphinxcrossref{1.9.25   皂 隶}}}

\item {} 
\phantomsection\label{\detokenize{p00_u5176_u5b83/_u767d_u8bdd_u804a_u658b_u5fd7_u5f02:id883}}{\hyperref[\detokenize{p00_u5176_u5b83/_u767d_u8bdd_u804a_u658b_u5fd7_u5f02:id373}]{\sphinxcrossref{1.9.26   绩 女}}}

\item {} 
\phantomsection\label{\detokenize{p00_u5176_u5b83/_u767d_u8bdd_u804a_u658b_u5fd7_u5f02:id884}}{\hyperref[\detokenize{p00_u5176_u5b83/_u767d_u8bdd_u804a_u658b_u5fd7_u5f02:id374}]{\sphinxcrossref{1.9.27   红 毛 毡}}}

\item {} 
\phantomsection\label{\detokenize{p00_u5176_u5b83/_u767d_u8bdd_u804a_u658b_u5fd7_u5f02:id885}}{\hyperref[\detokenize{p00_u5176_u5b83/_u767d_u8bdd_u804a_u658b_u5fd7_u5f02:id375}]{\sphinxcrossref{1.9.28   抽 肠}}}

\item {} 
\phantomsection\label{\detokenize{p00_u5176_u5b83/_u767d_u8bdd_u804a_u658b_u5fd7_u5f02:id886}}{\hyperref[\detokenize{p00_u5176_u5b83/_u767d_u8bdd_u804a_u658b_u5fd7_u5f02:id376}]{\sphinxcrossref{1.9.29   张 鸿 渐}}}

\item {} 
\phantomsection\label{\detokenize{p00_u5176_u5b83/_u767d_u8bdd_u804a_u658b_u5fd7_u5f02:id887}}{\hyperref[\detokenize{p00_u5176_u5b83/_u767d_u8bdd_u804a_u658b_u5fd7_u5f02:id377}]{\sphinxcrossref{1.9.30   太 医}}}

\item {} 
\phantomsection\label{\detokenize{p00_u5176_u5b83/_u767d_u8bdd_u804a_u658b_u5fd7_u5f02:id888}}{\hyperref[\detokenize{p00_u5176_u5b83/_u767d_u8bdd_u804a_u658b_u5fd7_u5f02:id378}]{\sphinxcrossref{1.9.31   牛 飞}}}

\item {} 
\phantomsection\label{\detokenize{p00_u5176_u5b83/_u767d_u8bdd_u804a_u658b_u5fd7_u5f02:id889}}{\hyperref[\detokenize{p00_u5176_u5b83/_u767d_u8bdd_u804a_u658b_u5fd7_u5f02:id379}]{\sphinxcrossref{1.9.32   王 子 安}}}

\item {} 
\phantomsection\label{\detokenize{p00_u5176_u5b83/_u767d_u8bdd_u804a_u658b_u5fd7_u5f02:id890}}{\hyperref[\detokenize{p00_u5176_u5b83/_u767d_u8bdd_u804a_u658b_u5fd7_u5f02:id380}]{\sphinxcrossref{1.9.33   刁 姓}}}

\item {} 
\phantomsection\label{\detokenize{p00_u5176_u5b83/_u767d_u8bdd_u804a_u658b_u5fd7_u5f02:id891}}{\hyperref[\detokenize{p00_u5176_u5b83/_u767d_u8bdd_u804a_u658b_u5fd7_u5f02:id381}]{\sphinxcrossref{1.9.34   农 妇}}}

\item {} 
\phantomsection\label{\detokenize{p00_u5176_u5b83/_u767d_u8bdd_u804a_u658b_u5fd7_u5f02:id892}}{\hyperref[\detokenize{p00_u5176_u5b83/_u767d_u8bdd_u804a_u658b_u5fd7_u5f02:id382}]{\sphinxcrossref{1.9.35   金 陵 乙}}}

\item {} 
\phantomsection\label{\detokenize{p00_u5176_u5b83/_u767d_u8bdd_u804a_u658b_u5fd7_u5f02:id893}}{\hyperref[\detokenize{p00_u5176_u5b83/_u767d_u8bdd_u804a_u658b_u5fd7_u5f02:id383}]{\sphinxcrossref{1.9.36   郭 安}}}

\item {} 
\phantomsection\label{\detokenize{p00_u5176_u5b83/_u767d_u8bdd_u804a_u658b_u5fd7_u5f02:id894}}{\hyperref[\detokenize{p00_u5176_u5b83/_u767d_u8bdd_u804a_u658b_u5fd7_u5f02:id384}]{\sphinxcrossref{1.9.37   折 狱}}}

\item {} 
\phantomsection\label{\detokenize{p00_u5176_u5b83/_u767d_u8bdd_u804a_u658b_u5fd7_u5f02:id895}}{\hyperref[\detokenize{p00_u5176_u5b83/_u767d_u8bdd_u804a_u658b_u5fd7_u5f02:id385}]{\sphinxcrossref{1.9.38   义 犬}}}

\item {} 
\phantomsection\label{\detokenize{p00_u5176_u5b83/_u767d_u8bdd_u804a_u658b_u5fd7_u5f02:id896}}{\hyperref[\detokenize{p00_u5176_u5b83/_u767d_u8bdd_u804a_u658b_u5fd7_u5f02:id386}]{\sphinxcrossref{1.9.39   杨 大 洪}}}

\item {} 
\phantomsection\label{\detokenize{p00_u5176_u5b83/_u767d_u8bdd_u804a_u658b_u5fd7_u5f02:id897}}{\hyperref[\detokenize{p00_u5176_u5b83/_u767d_u8bdd_u804a_u658b_u5fd7_u5f02:id387}]{\sphinxcrossref{1.9.40   查 牙 山 洞}}}

\item {} 
\phantomsection\label{\detokenize{p00_u5176_u5b83/_u767d_u8bdd_u804a_u658b_u5fd7_u5f02:id898}}{\hyperref[\detokenize{p00_u5176_u5b83/_u767d_u8bdd_u804a_u658b_u5fd7_u5f02:id388}]{\sphinxcrossref{1.9.41   安 期 岛}}}

\item {} 
\phantomsection\label{\detokenize{p00_u5176_u5b83/_u767d_u8bdd_u804a_u658b_u5fd7_u5f02:id899}}{\hyperref[\detokenize{p00_u5176_u5b83/_u767d_u8bdd_u804a_u658b_u5fd7_u5f02:id389}]{\sphinxcrossref{1.9.42   沅 俗}}}

\item {} 
\phantomsection\label{\detokenize{p00_u5176_u5b83/_u767d_u8bdd_u804a_u658b_u5fd7_u5f02:id900}}{\hyperref[\detokenize{p00_u5176_u5b83/_u767d_u8bdd_u804a_u658b_u5fd7_u5f02:id390}]{\sphinxcrossref{1.9.43   云 萝 公 主}}}

\item {} 
\phantomsection\label{\detokenize{p00_u5176_u5b83/_u767d_u8bdd_u804a_u658b_u5fd7_u5f02:id901}}{\hyperref[\detokenize{p00_u5176_u5b83/_u767d_u8bdd_u804a_u658b_u5fd7_u5f02:id391}]{\sphinxcrossref{1.9.44   鸟 语}}}

\item {} 
\phantomsection\label{\detokenize{p00_u5176_u5b83/_u767d_u8bdd_u804a_u658b_u5fd7_u5f02:id902}}{\hyperref[\detokenize{p00_u5176_u5b83/_u767d_u8bdd_u804a_u658b_u5fd7_u5f02:id392}]{\sphinxcrossref{1.9.45   天 宫}}}

\item {} 
\phantomsection\label{\detokenize{p00_u5176_u5b83/_u767d_u8bdd_u804a_u658b_u5fd7_u5f02:id903}}{\hyperref[\detokenize{p00_u5176_u5b83/_u767d_u8bdd_u804a_u658b_u5fd7_u5f02:id393}]{\sphinxcrossref{1.9.46   乔 女}}}

\item {} 
\phantomsection\label{\detokenize{p00_u5176_u5b83/_u767d_u8bdd_u804a_u658b_u5fd7_u5f02:id904}}{\hyperref[\detokenize{p00_u5176_u5b83/_u767d_u8bdd_u804a_u658b_u5fd7_u5f02:id394}]{\sphinxcrossref{1.9.47   蛤}}}

\item {} 
\phantomsection\label{\detokenize{p00_u5176_u5b83/_u767d_u8bdd_u804a_u658b_u5fd7_u5f02:id905}}{\hyperref[\detokenize{p00_u5176_u5b83/_u767d_u8bdd_u804a_u658b_u5fd7_u5f02:id395}]{\sphinxcrossref{1.9.48   刘 夫 人}}}

\item {} 
\phantomsection\label{\detokenize{p00_u5176_u5b83/_u767d_u8bdd_u804a_u658b_u5fd7_u5f02:id906}}{\hyperref[\detokenize{p00_u5176_u5b83/_u767d_u8bdd_u804a_u658b_u5fd7_u5f02:id396}]{\sphinxcrossref{1.9.49   陵 县 狐}}}

\end{itemize}

\item {} 
\phantomsection\label{\detokenize{p00_u5176_u5b83/_u767d_u8bdd_u804a_u658b_u5fd7_u5f02:id907}}{\hyperref[\detokenize{p00_u5176_u5b83/_u767d_u8bdd_u804a_u658b_u5fd7_u5f02:id397}]{\sphinxcrossref{1.10   卷 十}}}
\begin{itemize}
\item {} 
\phantomsection\label{\detokenize{p00_u5176_u5b83/_u767d_u8bdd_u804a_u658b_u5fd7_u5f02:id908}}{\hyperref[\detokenize{p00_u5176_u5b83/_u767d_u8bdd_u804a_u658b_u5fd7_u5f02:id398}]{\sphinxcrossref{1.10.1   王 货 郎}}}

\item {} 
\phantomsection\label{\detokenize{p00_u5176_u5b83/_u767d_u8bdd_u804a_u658b_u5fd7_u5f02:id909}}{\hyperref[\detokenize{p00_u5176_u5b83/_u767d_u8bdd_u804a_u658b_u5fd7_u5f02:id399}]{\sphinxcrossref{1.10.2   疲 龙}}}

\item {} 
\phantomsection\label{\detokenize{p00_u5176_u5b83/_u767d_u8bdd_u804a_u658b_u5fd7_u5f02:id910}}{\hyperref[\detokenize{p00_u5176_u5b83/_u767d_u8bdd_u804a_u658b_u5fd7_u5f02:id400}]{\sphinxcrossref{1.10.3   真 生}}}

\item {} 
\phantomsection\label{\detokenize{p00_u5176_u5b83/_u767d_u8bdd_u804a_u658b_u5fd7_u5f02:id911}}{\hyperref[\detokenize{p00_u5176_u5b83/_u767d_u8bdd_u804a_u658b_u5fd7_u5f02:id401}]{\sphinxcrossref{1.10.4   布 商}}}

\item {} 
\phantomsection\label{\detokenize{p00_u5176_u5b83/_u767d_u8bdd_u804a_u658b_u5fd7_u5f02:id912}}{\hyperref[\detokenize{p00_u5176_u5b83/_u767d_u8bdd_u804a_u658b_u5fd7_u5f02:id402}]{\sphinxcrossref{1.10.5   彭 二 挣}}}

\item {} 
\phantomsection\label{\detokenize{p00_u5176_u5b83/_u767d_u8bdd_u804a_u658b_u5fd7_u5f02:id913}}{\hyperref[\detokenize{p00_u5176_u5b83/_u767d_u8bdd_u804a_u658b_u5fd7_u5f02:id403}]{\sphinxcrossref{1.10.6   何 仙}}}

\item {} 
\phantomsection\label{\detokenize{p00_u5176_u5b83/_u767d_u8bdd_u804a_u658b_u5fd7_u5f02:id914}}{\hyperref[\detokenize{p00_u5176_u5b83/_u767d_u8bdd_u804a_u658b_u5fd7_u5f02:id404}]{\sphinxcrossref{1.10.7   牛 同 人}}}

\item {} 
\phantomsection\label{\detokenize{p00_u5176_u5b83/_u767d_u8bdd_u804a_u658b_u5fd7_u5f02:id915}}{\hyperref[\detokenize{p00_u5176_u5b83/_u767d_u8bdd_u804a_u658b_u5fd7_u5f02:id405}]{\sphinxcrossref{1.10.8   神 女}}}

\item {} 
\phantomsection\label{\detokenize{p00_u5176_u5b83/_u767d_u8bdd_u804a_u658b_u5fd7_u5f02:id916}}{\hyperref[\detokenize{p00_u5176_u5b83/_u767d_u8bdd_u804a_u658b_u5fd7_u5f02:id406}]{\sphinxcrossref{1.10.9   湘 裙}}}

\item {} 
\phantomsection\label{\detokenize{p00_u5176_u5b83/_u767d_u8bdd_u804a_u658b_u5fd7_u5f02:id917}}{\hyperref[\detokenize{p00_u5176_u5b83/_u767d_u8bdd_u804a_u658b_u5fd7_u5f02:id407}]{\sphinxcrossref{1.10.10   三 生}}}

\item {} 
\phantomsection\label{\detokenize{p00_u5176_u5b83/_u767d_u8bdd_u804a_u658b_u5fd7_u5f02:id918}}{\hyperref[\detokenize{p00_u5176_u5b83/_u767d_u8bdd_u804a_u658b_u5fd7_u5f02:id408}]{\sphinxcrossref{1.10.11   长 亭}}}

\item {} 
\phantomsection\label{\detokenize{p00_u5176_u5b83/_u767d_u8bdd_u804a_u658b_u5fd7_u5f02:id919}}{\hyperref[\detokenize{p00_u5176_u5b83/_u767d_u8bdd_u804a_u658b_u5fd7_u5f02:id409}]{\sphinxcrossref{1.10.12   席 方 平}}}

\item {} 
\phantomsection\label{\detokenize{p00_u5176_u5b83/_u767d_u8bdd_u804a_u658b_u5fd7_u5f02:id920}}{\hyperref[\detokenize{p00_u5176_u5b83/_u767d_u8bdd_u804a_u658b_u5fd7_u5f02:id410}]{\sphinxcrossref{1.10.13   素 秋}}}

\item {} 
\phantomsection\label{\detokenize{p00_u5176_u5b83/_u767d_u8bdd_u804a_u658b_u5fd7_u5f02:id921}}{\hyperref[\detokenize{p00_u5176_u5b83/_u767d_u8bdd_u804a_u658b_u5fd7_u5f02:id411}]{\sphinxcrossref{1.10.14   贾 奉 雉}}}

\item {} 
\phantomsection\label{\detokenize{p00_u5176_u5b83/_u767d_u8bdd_u804a_u658b_u5fd7_u5f02:id922}}{\hyperref[\detokenize{p00_u5176_u5b83/_u767d_u8bdd_u804a_u658b_u5fd7_u5f02:id412}]{\sphinxcrossref{1.10.15   胭 脂}}}

\item {} 
\phantomsection\label{\detokenize{p00_u5176_u5b83/_u767d_u8bdd_u804a_u658b_u5fd7_u5f02:id923}}{\hyperref[\detokenize{p00_u5176_u5b83/_u767d_u8bdd_u804a_u658b_u5fd7_u5f02:id413}]{\sphinxcrossref{1.10.16   阿 纤}}}

\item {} 
\phantomsection\label{\detokenize{p00_u5176_u5b83/_u767d_u8bdd_u804a_u658b_u5fd7_u5f02:id924}}{\hyperref[\detokenize{p00_u5176_u5b83/_u767d_u8bdd_u804a_u658b_u5fd7_u5f02:id414}]{\sphinxcrossref{1.10.17   瑞 云}}}

\item {} 
\phantomsection\label{\detokenize{p00_u5176_u5b83/_u767d_u8bdd_u804a_u658b_u5fd7_u5f02:id925}}{\hyperref[\detokenize{p00_u5176_u5b83/_u767d_u8bdd_u804a_u658b_u5fd7_u5f02:id415}]{\sphinxcrossref{1.10.18   仇 大 娘}}}

\item {} 
\phantomsection\label{\detokenize{p00_u5176_u5b83/_u767d_u8bdd_u804a_u658b_u5fd7_u5f02:id926}}{\hyperref[\detokenize{p00_u5176_u5b83/_u767d_u8bdd_u804a_u658b_u5fd7_u5f02:id416}]{\sphinxcrossref{1.10.19   曹 操 冢}}}

\item {} 
\phantomsection\label{\detokenize{p00_u5176_u5b83/_u767d_u8bdd_u804a_u658b_u5fd7_u5f02:id927}}{\hyperref[\detokenize{p00_u5176_u5b83/_u767d_u8bdd_u804a_u658b_u5fd7_u5f02:id417}]{\sphinxcrossref{1.10.20   龙 飞 相 公}}}

\item {} 
\phantomsection\label{\detokenize{p00_u5176_u5b83/_u767d_u8bdd_u804a_u658b_u5fd7_u5f02:id928}}{\hyperref[\detokenize{p00_u5176_u5b83/_u767d_u8bdd_u804a_u658b_u5fd7_u5f02:id418}]{\sphinxcrossref{1.10.21   珊 瑚}}}

\item {} 
\phantomsection\label{\detokenize{p00_u5176_u5b83/_u767d_u8bdd_u804a_u658b_u5fd7_u5f02:id929}}{\hyperref[\detokenize{p00_u5176_u5b83/_u767d_u8bdd_u804a_u658b_u5fd7_u5f02:id419}]{\sphinxcrossref{1.10.22   五 通}}}

\item {} 
\phantomsection\label{\detokenize{p00_u5176_u5b83/_u767d_u8bdd_u804a_u658b_u5fd7_u5f02:id930}}{\hyperref[\detokenize{p00_u5176_u5b83/_u767d_u8bdd_u804a_u658b_u5fd7_u5f02:id420}]{\sphinxcrossref{1.10.23   申 氏}}}

\item {} 
\phantomsection\label{\detokenize{p00_u5176_u5b83/_u767d_u8bdd_u804a_u658b_u5fd7_u5f02:id931}}{\hyperref[\detokenize{p00_u5176_u5b83/_u767d_u8bdd_u804a_u658b_u5fd7_u5f02:id421}]{\sphinxcrossref{1.10.24   恒 娘}}}

\item {} 
\phantomsection\label{\detokenize{p00_u5176_u5b83/_u767d_u8bdd_u804a_u658b_u5fd7_u5f02:id932}}{\hyperref[\detokenize{p00_u5176_u5b83/_u767d_u8bdd_u804a_u658b_u5fd7_u5f02:id422}]{\sphinxcrossref{1.10.25   葛 巾}}}

\end{itemize}

\item {} 
\phantomsection\label{\detokenize{p00_u5176_u5b83/_u767d_u8bdd_u804a_u658b_u5fd7_u5f02:id933}}{\hyperref[\detokenize{p00_u5176_u5b83/_u767d_u8bdd_u804a_u658b_u5fd7_u5f02:id423}]{\sphinxcrossref{1.11   卷 十 一}}}
\begin{itemize}
\item {} 
\phantomsection\label{\detokenize{p00_u5176_u5b83/_u767d_u8bdd_u804a_u658b_u5fd7_u5f02:id934}}{\hyperref[\detokenize{p00_u5176_u5b83/_u767d_u8bdd_u804a_u658b_u5fd7_u5f02:id424}]{\sphinxcrossref{1.11.1   冯 木 匠}}}

\item {} 
\phantomsection\label{\detokenize{p00_u5176_u5b83/_u767d_u8bdd_u804a_u658b_u5fd7_u5f02:id935}}{\hyperref[\detokenize{p00_u5176_u5b83/_u767d_u8bdd_u804a_u658b_u5fd7_u5f02:id425}]{\sphinxcrossref{1.11.2   黄 英}}}

\item {} 
\phantomsection\label{\detokenize{p00_u5176_u5b83/_u767d_u8bdd_u804a_u658b_u5fd7_u5f02:id936}}{\hyperref[\detokenize{p00_u5176_u5b83/_u767d_u8bdd_u804a_u658b_u5fd7_u5f02:id426}]{\sphinxcrossref{1.11.3   书 痴}}}

\item {} 
\phantomsection\label{\detokenize{p00_u5176_u5b83/_u767d_u8bdd_u804a_u658b_u5fd7_u5f02:id937}}{\hyperref[\detokenize{p00_u5176_u5b83/_u767d_u8bdd_u804a_u658b_u5fd7_u5f02:id427}]{\sphinxcrossref{1.11.4   齐 天 大 圣}}}

\item {} 
\phantomsection\label{\detokenize{p00_u5176_u5b83/_u767d_u8bdd_u804a_u658b_u5fd7_u5f02:id938}}{\hyperref[\detokenize{p00_u5176_u5b83/_u767d_u8bdd_u804a_u658b_u5fd7_u5f02:id428}]{\sphinxcrossref{1.11.5   青 蛙 神}}}

\item {} 
\phantomsection\label{\detokenize{p00_u5176_u5b83/_u767d_u8bdd_u804a_u658b_u5fd7_u5f02:id939}}{\hyperref[\detokenize{p00_u5176_u5b83/_u767d_u8bdd_u804a_u658b_u5fd7_u5f02:id429}]{\sphinxcrossref{1.11.6   任 秀}}}

\item {} 
\phantomsection\label{\detokenize{p00_u5176_u5b83/_u767d_u8bdd_u804a_u658b_u5fd7_u5f02:id940}}{\hyperref[\detokenize{p00_u5176_u5b83/_u767d_u8bdd_u804a_u658b_u5fd7_u5f02:id430}]{\sphinxcrossref{1.11.7   晚 霞}}}

\item {} 
\phantomsection\label{\detokenize{p00_u5176_u5b83/_u767d_u8bdd_u804a_u658b_u5fd7_u5f02:id941}}{\hyperref[\detokenize{p00_u5176_u5b83/_u767d_u8bdd_u804a_u658b_u5fd7_u5f02:id431}]{\sphinxcrossref{1.11.8   白 秋 练}}}

\item {} 
\phantomsection\label{\detokenize{p00_u5176_u5b83/_u767d_u8bdd_u804a_u658b_u5fd7_u5f02:id942}}{\hyperref[\detokenize{p00_u5176_u5b83/_u767d_u8bdd_u804a_u658b_u5fd7_u5f02:id432}]{\sphinxcrossref{1.11.9   王 者}}}

\item {} 
\phantomsection\label{\detokenize{p00_u5176_u5b83/_u767d_u8bdd_u804a_u658b_u5fd7_u5f02:id943}}{\hyperref[\detokenize{p00_u5176_u5b83/_u767d_u8bdd_u804a_u658b_u5fd7_u5f02:id433}]{\sphinxcrossref{1.11.10   某 甲}}}

\item {} 
\phantomsection\label{\detokenize{p00_u5176_u5b83/_u767d_u8bdd_u804a_u658b_u5fd7_u5f02:id944}}{\hyperref[\detokenize{p00_u5176_u5b83/_u767d_u8bdd_u804a_u658b_u5fd7_u5f02:id434}]{\sphinxcrossref{1.11.11   衢 州 三 怪}}}

\item {} 
\phantomsection\label{\detokenize{p00_u5176_u5b83/_u767d_u8bdd_u804a_u658b_u5fd7_u5f02:id945}}{\hyperref[\detokenize{p00_u5176_u5b83/_u767d_u8bdd_u804a_u658b_u5fd7_u5f02:id435}]{\sphinxcrossref{1.11.12   拆 楼 人}}}

\item {} 
\phantomsection\label{\detokenize{p00_u5176_u5b83/_u767d_u8bdd_u804a_u658b_u5fd7_u5f02:id946}}{\hyperref[\detokenize{p00_u5176_u5b83/_u767d_u8bdd_u804a_u658b_u5fd7_u5f02:id436}]{\sphinxcrossref{1.11.13   大 蝎}}}

\item {} 
\phantomsection\label{\detokenize{p00_u5176_u5b83/_u767d_u8bdd_u804a_u658b_u5fd7_u5f02:id947}}{\hyperref[\detokenize{p00_u5176_u5b83/_u767d_u8bdd_u804a_u658b_u5fd7_u5f02:id437}]{\sphinxcrossref{1.11.14   陈 云 栖}}}

\item {} 
\phantomsection\label{\detokenize{p00_u5176_u5b83/_u767d_u8bdd_u804a_u658b_u5fd7_u5f02:id948}}{\hyperref[\detokenize{p00_u5176_u5b83/_u767d_u8bdd_u804a_u658b_u5fd7_u5f02:id438}]{\sphinxcrossref{1.11.15   司 札 吏}}}

\item {} 
\phantomsection\label{\detokenize{p00_u5176_u5b83/_u767d_u8bdd_u804a_u658b_u5fd7_u5f02:id949}}{\hyperref[\detokenize{p00_u5176_u5b83/_u767d_u8bdd_u804a_u658b_u5fd7_u5f02:id439}]{\sphinxcrossref{1.11.16   蛐 蜒}}}

\item {} 
\phantomsection\label{\detokenize{p00_u5176_u5b83/_u767d_u8bdd_u804a_u658b_u5fd7_u5f02:id950}}{\hyperref[\detokenize{p00_u5176_u5b83/_u767d_u8bdd_u804a_u658b_u5fd7_u5f02:id440}]{\sphinxcrossref{1.11.17   司 训}}}

\item {} 
\phantomsection\label{\detokenize{p00_u5176_u5b83/_u767d_u8bdd_u804a_u658b_u5fd7_u5f02:id951}}{\hyperref[\detokenize{p00_u5176_u5b83/_u767d_u8bdd_u804a_u658b_u5fd7_u5f02:id441}]{\sphinxcrossref{1.11.18   黑 鬼}}}

\item {} 
\phantomsection\label{\detokenize{p00_u5176_u5b83/_u767d_u8bdd_u804a_u658b_u5fd7_u5f02:id952}}{\hyperref[\detokenize{p00_u5176_u5b83/_u767d_u8bdd_u804a_u658b_u5fd7_u5f02:id442}]{\sphinxcrossref{1.11.19   织 成}}}

\item {} 
\phantomsection\label{\detokenize{p00_u5176_u5b83/_u767d_u8bdd_u804a_u658b_u5fd7_u5f02:id953}}{\hyperref[\detokenize{p00_u5176_u5b83/_u767d_u8bdd_u804a_u658b_u5fd7_u5f02:id443}]{\sphinxcrossref{1.11.20   竹 青}}}

\item {} 
\phantomsection\label{\detokenize{p00_u5176_u5b83/_u767d_u8bdd_u804a_u658b_u5fd7_u5f02:id954}}{\hyperref[\detokenize{p00_u5176_u5b83/_u767d_u8bdd_u804a_u658b_u5fd7_u5f02:id444}]{\sphinxcrossref{1.11.21   段 氏}}}

\item {} 
\phantomsection\label{\detokenize{p00_u5176_u5b83/_u767d_u8bdd_u804a_u658b_u5fd7_u5f02:id955}}{\hyperref[\detokenize{p00_u5176_u5b83/_u767d_u8bdd_u804a_u658b_u5fd7_u5f02:id445}]{\sphinxcrossref{1.11.22   狐 女}}}

\item {} 
\phantomsection\label{\detokenize{p00_u5176_u5b83/_u767d_u8bdd_u804a_u658b_u5fd7_u5f02:id956}}{\hyperref[\detokenize{p00_u5176_u5b83/_u767d_u8bdd_u804a_u658b_u5fd7_u5f02:id446}]{\sphinxcrossref{1.11.23   张 氏 妇}}}

\item {} 
\phantomsection\label{\detokenize{p00_u5176_u5b83/_u767d_u8bdd_u804a_u658b_u5fd7_u5f02:id957}}{\hyperref[\detokenize{p00_u5176_u5b83/_u767d_u8bdd_u804a_u658b_u5fd7_u5f02:id447}]{\sphinxcrossref{1.11.24   于 子 游}}}

\item {} 
\phantomsection\label{\detokenize{p00_u5176_u5b83/_u767d_u8bdd_u804a_u658b_u5fd7_u5f02:id958}}{\hyperref[\detokenize{p00_u5176_u5b83/_u767d_u8bdd_u804a_u658b_u5fd7_u5f02:id448}]{\sphinxcrossref{1.11.25   男 妾}}}

\item {} 
\phantomsection\label{\detokenize{p00_u5176_u5b83/_u767d_u8bdd_u804a_u658b_u5fd7_u5f02:id959}}{\hyperref[\detokenize{p00_u5176_u5b83/_u767d_u8bdd_u804a_u658b_u5fd7_u5f02:id449}]{\sphinxcrossref{1.11.26   汪 可 受}}}

\item {} 
\phantomsection\label{\detokenize{p00_u5176_u5b83/_u767d_u8bdd_u804a_u658b_u5fd7_u5f02:id960}}{\hyperref[\detokenize{p00_u5176_u5b83/_u767d_u8bdd_u804a_u658b_u5fd7_u5f02:id450}]{\sphinxcrossref{1.11.27   牛 犊}}}

\item {} 
\phantomsection\label{\detokenize{p00_u5176_u5b83/_u767d_u8bdd_u804a_u658b_u5fd7_u5f02:id961}}{\hyperref[\detokenize{p00_u5176_u5b83/_u767d_u8bdd_u804a_u658b_u5fd7_u5f02:id451}]{\sphinxcrossref{1.11.28   王 大}}}

\item {} 
\phantomsection\label{\detokenize{p00_u5176_u5b83/_u767d_u8bdd_u804a_u658b_u5fd7_u5f02:id962}}{\hyperref[\detokenize{p00_u5176_u5b83/_u767d_u8bdd_u804a_u658b_u5fd7_u5f02:id452}]{\sphinxcrossref{1.11.29   乐 仲}}}

\item {} 
\phantomsection\label{\detokenize{p00_u5176_u5b83/_u767d_u8bdd_u804a_u658b_u5fd7_u5f02:id963}}{\hyperref[\detokenize{p00_u5176_u5b83/_u767d_u8bdd_u804a_u658b_u5fd7_u5f02:id453}]{\sphinxcrossref{1.11.30   香 玉}}}

\item {} 
\phantomsection\label{\detokenize{p00_u5176_u5b83/_u767d_u8bdd_u804a_u658b_u5fd7_u5f02:id964}}{\hyperref[\detokenize{p00_u5176_u5b83/_u767d_u8bdd_u804a_u658b_u5fd7_u5f02:id454}]{\sphinxcrossref{1.11.31   三 仙}}}

\item {} 
\phantomsection\label{\detokenize{p00_u5176_u5b83/_u767d_u8bdd_u804a_u658b_u5fd7_u5f02:id965}}{\hyperref[\detokenize{p00_u5176_u5b83/_u767d_u8bdd_u804a_u658b_u5fd7_u5f02:id455}]{\sphinxcrossref{1.11.32   鬼 隶}}}

\item {} 
\phantomsection\label{\detokenize{p00_u5176_u5b83/_u767d_u8bdd_u804a_u658b_u5fd7_u5f02:id966}}{\hyperref[\detokenize{p00_u5176_u5b83/_u767d_u8bdd_u804a_u658b_u5fd7_u5f02:id456}]{\sphinxcrossref{1.11.33   王 十}}}

\item {} 
\phantomsection\label{\detokenize{p00_u5176_u5b83/_u767d_u8bdd_u804a_u658b_u5fd7_u5f02:id967}}{\hyperref[\detokenize{p00_u5176_u5b83/_u767d_u8bdd_u804a_u658b_u5fd7_u5f02:id457}]{\sphinxcrossref{1.11.34   大 男}}}

\item {} 
\phantomsection\label{\detokenize{p00_u5176_u5b83/_u767d_u8bdd_u804a_u658b_u5fd7_u5f02:id968}}{\hyperref[\detokenize{p00_u5176_u5b83/_u767d_u8bdd_u804a_u658b_u5fd7_u5f02:id458}]{\sphinxcrossref{1.11.35   外 国 人}}}

\item {} 
\phantomsection\label{\detokenize{p00_u5176_u5b83/_u767d_u8bdd_u804a_u658b_u5fd7_u5f02:id969}}{\hyperref[\detokenize{p00_u5176_u5b83/_u767d_u8bdd_u804a_u658b_u5fd7_u5f02:id459}]{\sphinxcrossref{1.11.36   韦 公 子}}}

\item {} 
\phantomsection\label{\detokenize{p00_u5176_u5b83/_u767d_u8bdd_u804a_u658b_u5fd7_u5f02:id970}}{\hyperref[\detokenize{p00_u5176_u5b83/_u767d_u8bdd_u804a_u658b_u5fd7_u5f02:id460}]{\sphinxcrossref{1.11.37   石 清 虚}}}

\item {} 
\phantomsection\label{\detokenize{p00_u5176_u5b83/_u767d_u8bdd_u804a_u658b_u5fd7_u5f02:id971}}{\hyperref[\detokenize{p00_u5176_u5b83/_u767d_u8bdd_u804a_u658b_u5fd7_u5f02:id461}]{\sphinxcrossref{1.11.38   曾 友 于}}}

\item {} 
\phantomsection\label{\detokenize{p00_u5176_u5b83/_u767d_u8bdd_u804a_u658b_u5fd7_u5f02:id972}}{\hyperref[\detokenize{p00_u5176_u5b83/_u767d_u8bdd_u804a_u658b_u5fd7_u5f02:id462}]{\sphinxcrossref{1.11.39   嘉 平 公 子}}}

\end{itemize}

\item {} 
\phantomsection\label{\detokenize{p00_u5176_u5b83/_u767d_u8bdd_u804a_u658b_u5fd7_u5f02:id973}}{\hyperref[\detokenize{p00_u5176_u5b83/_u767d_u8bdd_u804a_u658b_u5fd7_u5f02:id463}]{\sphinxcrossref{1.12   卷 十 二}}}
\begin{itemize}
\item {} 
\phantomsection\label{\detokenize{p00_u5176_u5b83/_u767d_u8bdd_u804a_u658b_u5fd7_u5f02:id974}}{\hyperref[\detokenize{p00_u5176_u5b83/_u767d_u8bdd_u804a_u658b_u5fd7_u5f02:id464}]{\sphinxcrossref{1.12.1   二 班}}}

\item {} 
\phantomsection\label{\detokenize{p00_u5176_u5b83/_u767d_u8bdd_u804a_u658b_u5fd7_u5f02:id975}}{\hyperref[\detokenize{p00_u5176_u5b83/_u767d_u8bdd_u804a_u658b_u5fd7_u5f02:id465}]{\sphinxcrossref{1.12.2   车 夫}}}

\item {} 
\phantomsection\label{\detokenize{p00_u5176_u5b83/_u767d_u8bdd_u804a_u658b_u5fd7_u5f02:id976}}{\hyperref[\detokenize{p00_u5176_u5b83/_u767d_u8bdd_u804a_u658b_u5fd7_u5f02:id466}]{\sphinxcrossref{1.12.3   乩 仙}}}

\item {} 
\phantomsection\label{\detokenize{p00_u5176_u5b83/_u767d_u8bdd_u804a_u658b_u5fd7_u5f02:id977}}{\hyperref[\detokenize{p00_u5176_u5b83/_u767d_u8bdd_u804a_u658b_u5fd7_u5f02:id467}]{\sphinxcrossref{1.12.4   苗 生}}}

\item {} 
\phantomsection\label{\detokenize{p00_u5176_u5b83/_u767d_u8bdd_u804a_u658b_u5fd7_u5f02:id978}}{\hyperref[\detokenize{p00_u5176_u5b83/_u767d_u8bdd_u804a_u658b_u5fd7_u5f02:id468}]{\sphinxcrossref{1.12.5   蝎 客}}}

\item {} 
\phantomsection\label{\detokenize{p00_u5176_u5b83/_u767d_u8bdd_u804a_u658b_u5fd7_u5f02:id979}}{\hyperref[\detokenize{p00_u5176_u5b83/_u767d_u8bdd_u804a_u658b_u5fd7_u5f02:id469}]{\sphinxcrossref{1.12.6   杜 小 雷}}}

\item {} 
\phantomsection\label{\detokenize{p00_u5176_u5b83/_u767d_u8bdd_u804a_u658b_u5fd7_u5f02:id980}}{\hyperref[\detokenize{p00_u5176_u5b83/_u767d_u8bdd_u804a_u658b_u5fd7_u5f02:id470}]{\sphinxcrossref{1.12.7   毛 大 福}}}

\item {} 
\phantomsection\label{\detokenize{p00_u5176_u5b83/_u767d_u8bdd_u804a_u658b_u5fd7_u5f02:id981}}{\hyperref[\detokenize{p00_u5176_u5b83/_u767d_u8bdd_u804a_u658b_u5fd7_u5f02:id471}]{\sphinxcrossref{1.12.8   雹 神}}}

\item {} 
\phantomsection\label{\detokenize{p00_u5176_u5b83/_u767d_u8bdd_u804a_u658b_u5fd7_u5f02:id982}}{\hyperref[\detokenize{p00_u5176_u5b83/_u767d_u8bdd_u804a_u658b_u5fd7_u5f02:id472}]{\sphinxcrossref{1.12.9   李 八 缸}}}

\item {} 
\phantomsection\label{\detokenize{p00_u5176_u5b83/_u767d_u8bdd_u804a_u658b_u5fd7_u5f02:id983}}{\hyperref[\detokenize{p00_u5176_u5b83/_u767d_u8bdd_u804a_u658b_u5fd7_u5f02:id473}]{\sphinxcrossref{1.12.10   老 龙 舡 户}}}

\item {} 
\phantomsection\label{\detokenize{p00_u5176_u5b83/_u767d_u8bdd_u804a_u658b_u5fd7_u5f02:id984}}{\hyperref[\detokenize{p00_u5176_u5b83/_u767d_u8bdd_u804a_u658b_u5fd7_u5f02:id474}]{\sphinxcrossref{1.12.11   青 城 妇}}}

\item {} 
\phantomsection\label{\detokenize{p00_u5176_u5b83/_u767d_u8bdd_u804a_u658b_u5fd7_u5f02:id985}}{\hyperref[\detokenize{p00_u5176_u5b83/_u767d_u8bdd_u804a_u658b_u5fd7_u5f02:id475}]{\sphinxcrossref{1.12.12   鸮 鸟}}}

\item {} 
\phantomsection\label{\detokenize{p00_u5176_u5b83/_u767d_u8bdd_u804a_u658b_u5fd7_u5f02:id986}}{\hyperref[\detokenize{p00_u5176_u5b83/_u767d_u8bdd_u804a_u658b_u5fd7_u5f02:id476}]{\sphinxcrossref{1.12.13   古 瓶}}}

\item {} 
\phantomsection\label{\detokenize{p00_u5176_u5b83/_u767d_u8bdd_u804a_u658b_u5fd7_u5f02:id987}}{\hyperref[\detokenize{p00_u5176_u5b83/_u767d_u8bdd_u804a_u658b_u5fd7_u5f02:id477}]{\sphinxcrossref{1.12.14   元 少 先 生}}}

\item {} 
\phantomsection\label{\detokenize{p00_u5176_u5b83/_u767d_u8bdd_u804a_u658b_u5fd7_u5f02:id988}}{\hyperref[\detokenize{p00_u5176_u5b83/_u767d_u8bdd_u804a_u658b_u5fd7_u5f02:id478}]{\sphinxcrossref{1.12.15   薛 蔚 娘}}}

\item {} 
\phantomsection\label{\detokenize{p00_u5176_u5b83/_u767d_u8bdd_u804a_u658b_u5fd7_u5f02:id989}}{\hyperref[\detokenize{p00_u5176_u5b83/_u767d_u8bdd_u804a_u658b_u5fd7_u5f02:id479}]{\sphinxcrossref{1.12.16   田 子 成}}}

\item {} 
\phantomsection\label{\detokenize{p00_u5176_u5b83/_u767d_u8bdd_u804a_u658b_u5fd7_u5f02:id990}}{\hyperref[\detokenize{p00_u5176_u5b83/_u767d_u8bdd_u804a_u658b_u5fd7_u5f02:id480}]{\sphinxcrossref{1.12.17   王 桂 庵}}}

\item {} 
\phantomsection\label{\detokenize{p00_u5176_u5b83/_u767d_u8bdd_u804a_u658b_u5fd7_u5f02:id991}}{\hyperref[\detokenize{p00_u5176_u5b83/_u767d_u8bdd_u804a_u658b_u5fd7_u5f02:id481}]{\sphinxcrossref{1.12.18   寄 生}}}

\item {} 
\phantomsection\label{\detokenize{p00_u5176_u5b83/_u767d_u8bdd_u804a_u658b_u5fd7_u5f02:id992}}{\hyperref[\detokenize{p00_u5176_u5b83/_u767d_u8bdd_u804a_u658b_u5fd7_u5f02:id482}]{\sphinxcrossref{1.12.19   周 生}}}

\item {} 
\phantomsection\label{\detokenize{p00_u5176_u5b83/_u767d_u8bdd_u804a_u658b_u5fd7_u5f02:id993}}{\hyperref[\detokenize{p00_u5176_u5b83/_u767d_u8bdd_u804a_u658b_u5fd7_u5f02:id483}]{\sphinxcrossref{1.12.20   褚 遂 良}}}

\item {} 
\phantomsection\label{\detokenize{p00_u5176_u5b83/_u767d_u8bdd_u804a_u658b_u5fd7_u5f02:id994}}{\hyperref[\detokenize{p00_u5176_u5b83/_u767d_u8bdd_u804a_u658b_u5fd7_u5f02:id484}]{\sphinxcrossref{1.12.21   刘 全}}}

\item {} 
\phantomsection\label{\detokenize{p00_u5176_u5b83/_u767d_u8bdd_u804a_u658b_u5fd7_u5f02:id995}}{\hyperref[\detokenize{p00_u5176_u5b83/_u767d_u8bdd_u804a_u658b_u5fd7_u5f02:id485}]{\sphinxcrossref{1.12.22   土 化 兔}}}

\item {} 
\phantomsection\label{\detokenize{p00_u5176_u5b83/_u767d_u8bdd_u804a_u658b_u5fd7_u5f02:id996}}{\hyperref[\detokenize{p00_u5176_u5b83/_u767d_u8bdd_u804a_u658b_u5fd7_u5f02:id486}]{\sphinxcrossref{1.12.23   鸟 使}}}

\item {} 
\phantomsection\label{\detokenize{p00_u5176_u5b83/_u767d_u8bdd_u804a_u658b_u5fd7_u5f02:id997}}{\hyperref[\detokenize{p00_u5176_u5b83/_u767d_u8bdd_u804a_u658b_u5fd7_u5f02:id487}]{\sphinxcrossref{1.12.24   姬 生}}}

\item {} 
\phantomsection\label{\detokenize{p00_u5176_u5b83/_u767d_u8bdd_u804a_u658b_u5fd7_u5f02:id998}}{\hyperref[\detokenize{p00_u5176_u5b83/_u767d_u8bdd_u804a_u658b_u5fd7_u5f02:id488}]{\sphinxcrossref{1.12.25   果 报}}}

\item {} 
\phantomsection\label{\detokenize{p00_u5176_u5b83/_u767d_u8bdd_u804a_u658b_u5fd7_u5f02:id999}}{\hyperref[\detokenize{p00_u5176_u5b83/_u767d_u8bdd_u804a_u658b_u5fd7_u5f02:id489}]{\sphinxcrossref{1.12.26   公 孙 夏}}}

\item {} 
\phantomsection\label{\detokenize{p00_u5176_u5b83/_u767d_u8bdd_u804a_u658b_u5fd7_u5f02:id1000}}{\hyperref[\detokenize{p00_u5176_u5b83/_u767d_u8bdd_u804a_u658b_u5fd7_u5f02:id490}]{\sphinxcrossref{1.12.27   韩 方}}}

\item {} 
\phantomsection\label{\detokenize{p00_u5176_u5b83/_u767d_u8bdd_u804a_u658b_u5fd7_u5f02:id1001}}{\hyperref[\detokenize{p00_u5176_u5b83/_u767d_u8bdd_u804a_u658b_u5fd7_u5f02:id491}]{\sphinxcrossref{1.12.28   纫 针}}}

\item {} 
\phantomsection\label{\detokenize{p00_u5176_u5b83/_u767d_u8bdd_u804a_u658b_u5fd7_u5f02:id1002}}{\hyperref[\detokenize{p00_u5176_u5b83/_u767d_u8bdd_u804a_u658b_u5fd7_u5f02:id492}]{\sphinxcrossref{1.12.29   桓 侯}}}

\item {} 
\phantomsection\label{\detokenize{p00_u5176_u5b83/_u767d_u8bdd_u804a_u658b_u5fd7_u5f02:id1003}}{\hyperref[\detokenize{p00_u5176_u5b83/_u767d_u8bdd_u804a_u658b_u5fd7_u5f02:id493}]{\sphinxcrossref{1.12.30   粉 蝶}}}

\item {} 
\phantomsection\label{\detokenize{p00_u5176_u5b83/_u767d_u8bdd_u804a_u658b_u5fd7_u5f02:id1004}}{\hyperref[\detokenize{p00_u5176_u5b83/_u767d_u8bdd_u804a_u658b_u5fd7_u5f02:id494}]{\sphinxcrossref{1.12.31   李 檀 斯}}}

\item {} 
\phantomsection\label{\detokenize{p00_u5176_u5b83/_u767d_u8bdd_u804a_u658b_u5fd7_u5f02:id1005}}{\hyperref[\detokenize{p00_u5176_u5b83/_u767d_u8bdd_u804a_u658b_u5fd7_u5f02:id495}]{\sphinxcrossref{1.12.32   锦 瑟}}}

\item {} 
\phantomsection\label{\detokenize{p00_u5176_u5b83/_u767d_u8bdd_u804a_u658b_u5fd7_u5f02:id1006}}{\hyperref[\detokenize{p00_u5176_u5b83/_u767d_u8bdd_u804a_u658b_u5fd7_u5f02:id496}]{\sphinxcrossref{1.12.33   太 原 狱}}}

\item {} 
\phantomsection\label{\detokenize{p00_u5176_u5b83/_u767d_u8bdd_u804a_u658b_u5fd7_u5f02:id1007}}{\hyperref[\detokenize{p00_u5176_u5b83/_u767d_u8bdd_u804a_u658b_u5fd7_u5f02:id497}]{\sphinxcrossref{1.12.34   新 郑 讼}}}

\item {} 
\phantomsection\label{\detokenize{p00_u5176_u5b83/_u767d_u8bdd_u804a_u658b_u5fd7_u5f02:id1008}}{\hyperref[\detokenize{p00_u5176_u5b83/_u767d_u8bdd_u804a_u658b_u5fd7_u5f02:id498}]{\sphinxcrossref{1.12.35   李 象 先}}}

\item {} 
\phantomsection\label{\detokenize{p00_u5176_u5b83/_u767d_u8bdd_u804a_u658b_u5fd7_u5f02:id1009}}{\hyperref[\detokenize{p00_u5176_u5b83/_u767d_u8bdd_u804a_u658b_u5fd7_u5f02:id499}]{\sphinxcrossref{1.12.36   房 文 淑}}}

\item {} 
\phantomsection\label{\detokenize{p00_u5176_u5b83/_u767d_u8bdd_u804a_u658b_u5fd7_u5f02:id1010}}{\hyperref[\detokenize{p00_u5176_u5b83/_u767d_u8bdd_u804a_u658b_u5fd7_u5f02:id500}]{\sphinxcrossref{1.12.37   秦 桧}}}

\item {} 
\phantomsection\label{\detokenize{p00_u5176_u5b83/_u767d_u8bdd_u804a_u658b_u5fd7_u5f02:id1011}}{\hyperref[\detokenize{p00_u5176_u5b83/_u767d_u8bdd_u804a_u658b_u5fd7_u5f02:id501}]{\sphinxcrossref{1.12.38   浙 东 生}}}

\item {} 
\phantomsection\label{\detokenize{p00_u5176_u5b83/_u767d_u8bdd_u804a_u658b_u5fd7_u5f02:id1012}}{\hyperref[\detokenize{p00_u5176_u5b83/_u767d_u8bdd_u804a_u658b_u5fd7_u5f02:id502}]{\sphinxcrossref{1.12.39   博 兴 女}}}

\item {} 
\phantomsection\label{\detokenize{p00_u5176_u5b83/_u767d_u8bdd_u804a_u658b_u5fd7_u5f02:id1013}}{\hyperref[\detokenize{p00_u5176_u5b83/_u767d_u8bdd_u804a_u658b_u5fd7_u5f02:id503}]{\sphinxcrossref{1.12.40   一 员 官}}}

\item {} 
\phantomsection\label{\detokenize{p00_u5176_u5b83/_u767d_u8bdd_u804a_u658b_u5fd7_u5f02:id1014}}{\hyperref[\detokenize{p00_u5176_u5b83/_u767d_u8bdd_u804a_u658b_u5fd7_u5f02:id504}]{\sphinxcrossref{1.12.41   丐 仙}}}

\item {} 
\phantomsection\label{\detokenize{p00_u5176_u5b83/_u767d_u8bdd_u804a_u658b_u5fd7_u5f02:id1015}}{\hyperref[\detokenize{p00_u5176_u5b83/_u767d_u8bdd_u804a_u658b_u5fd7_u5f02:id505}]{\sphinxcrossref{1.12.42   人 妖}}}

\item {} 
\phantomsection\label{\detokenize{p00_u5176_u5b83/_u767d_u8bdd_u804a_u658b_u5fd7_u5f02:id1016}}{\hyperref[\detokenize{p00_u5176_u5b83/_u767d_u8bdd_u804a_u658b_u5fd7_u5f02:id506}]{\sphinxcrossref{1.12.43   附录}}}

\item {} 
\phantomsection\label{\detokenize{p00_u5176_u5b83/_u767d_u8bdd_u804a_u658b_u5fd7_u5f02:id1017}}{\hyperref[\detokenize{p00_u5176_u5b83/_u767d_u8bdd_u804a_u658b_u5fd7_u5f02:id507}]{\sphinxcrossref{1.12.44   蛰 蛇}}}

\item {} 
\phantomsection\label{\detokenize{p00_u5176_u5b83/_u767d_u8bdd_u804a_u658b_u5fd7_u5f02:id1018}}{\hyperref[\detokenize{p00_u5176_u5b83/_u767d_u8bdd_u804a_u658b_u5fd7_u5f02:id508}]{\sphinxcrossref{1.12.45   晋 人}}}

\item {} 
\phantomsection\label{\detokenize{p00_u5176_u5b83/_u767d_u8bdd_u804a_u658b_u5fd7_u5f02:id1019}}{\hyperref[\detokenize{p00_u5176_u5b83/_u767d_u8bdd_u804a_u658b_u5fd7_u5f02:id509}]{\sphinxcrossref{1.12.46   龙}}}

\item {} 
\phantomsection\label{\detokenize{p00_u5176_u5b83/_u767d_u8bdd_u804a_u658b_u5fd7_u5f02:id1020}}{\hyperref[\detokenize{p00_u5176_u5b83/_u767d_u8bdd_u804a_u658b_u5fd7_u5f02:id510}]{\sphinxcrossref{1.12.47   爱 才}}}

\end{itemize}

\end{itemize}

\end{itemize}
\end{sphinxShadowBox}

此书以张友鹤先生的“三会本”《聊斋志异》为翻译底本,个别地方对照参考了其他版本,择善而从;翻译力求客观再现原著风貌;惟原文每篇故事后的“异史氏曰”,因系作者感慨议论,与故事本身联系不大,姑删去未译。

附录的四个故事《蛰蛇》、《晋人》、《龙》、《爱才》,书中没有,故用文言文代替。


\section{1.1   卷 一}
\label{\detokenize{p00_u5176_u5b83/_u767d_u8bdd_u804a_u658b_u5fd7_u5f02:id2}}

\subsection{1.1.1   考 城 隍}
\label{\detokenize{p00_u5176_u5b83/_u767d_u8bdd_u804a_u658b_u5fd7_u5f02:id3}}
我姐夫的祖父,名叫宋焘,是本县的廪生。有一天,他生病卧床,见一个小官吏,拿着帖子,牵着一匹额上有白毛的马来找他,对他说:“请你去考试。”宋公说: “考官还没来,为什么马上就考试?”来的官吏也不多说,只是催宋公上路。宋公没办法,只好带病骑上马跟他走了。

走的这一路很生疏,到了一座城郭,好像是一个国王的国都。一霎时他就跟那人进入了王府,只见王府内的宫殿非常辉煌华丽。正面大殿内坐着十几位官员,都不认得是什么人,唯有关帝神他认得。殿外屋檐下摆着两张桌子,两个坐墩,已经有一个秀才坐在那里,宋公便与这人并肩坐下。桌上分别放着笔和纸。

不多时,就发下试题来,一看上面有八个字:“一人二人,有心无心。”一会儿,两人的文章就作完了,呈交殿上。宋公文章中有这样的句子:“有心为善,虽善不赏;无心为恶,虽恶不罚。”诸位神人传着看完,称赞不已。便传叫宋公上殿。下令说:“河南缺一个城隍神,你很称职。”宋公听了,才恍然大悟,随即叩头在地,哭着说:“大神错爱我,叫我去当城隍,不敢推辞。只是我家有老母,七十多岁了,无人奉养,请求大神准我侍候母亲去世后,再去上任。”正面坐着一位像帝王的人,叫取宋公母亲的寿命簿来查看。一个长着胡子的官吏捧过簿子来翻看一遍,禀告说:“还有阳寿九年。”诸神都犹豫了,一时拿不出主意,关帝神说:“不妨先叫张生代理九年吧!”便对宋公说:“本应叫你马上去上任,念你有孝心,给你九年假期,到时再叫你来。”接着关帝神又勉励了秀才几句话,两个考生便叩头下殿。

秀才握着宋公手送到郊外,自己介绍说是长山县人,姓张,还给宋公作送别诗一首。原文都忘记了,只记得有这样的句子:“有花有酒春常在,无烛无灯夜自明。”宋公便上马作别而回。

宋公到了家,像是做了一个梦醒来,那时他已死了三天了。他母亲听见棺材中有呻吟声,打开棺材见他醒了过来,就把他扶出来,呆了半天才会说话。后来到长山县打听,果然有个姓张的秀才在这一天死去。

九年后,宋公的母亲果然去世,宋公料理完了丧事,洗了个澡,穿上新衣服,进屋就死了。

他的岳父家住城里西门里。一天,忽然见宋公骑着红缨大马,带着许多车马,到他家拜别。一家人都非常惊疑,不知道他已成了神人了。急忙跑到宋公家一问,才知道宋公已死了。

宋公自已记有小传,可惜兵慌马乱中没有存下来。这里的记载只是个大概而已。


\subsection{1.1.2   耳 中 人}
\label{\detokenize{p00_u5176_u5b83/_u767d_u8bdd_u804a_u658b_u5fd7_u5f02:id4}}
谭晋玄,是本县的一名秀才。他很相信一种气功之术,每日练习,冬夏不停。练了好几个月,自己觉得好像有些收获。有一天,他正盘腿而坐,听到耳中有很小的说话声,就像苍蝇叫一般,说:“可以见吗?”他一睁眼,就再也听不见了。他又重新闭上眼、息住气听,又听到方才的声音。他想:这可能是功已练成,心里暗暗高兴。

从此,他每日坐下就听,心里想,等耳中再说话时,应当答应一声并睁眼看看是什么东西。有一天,果然又听到那“可以见吗?”的小小说话声,他就小声答应:“可以见了。”很快觉得耳朵中有窸窸窸窸的声音,像有东西爬出来。他慢慢地睁开眼偷看,果然看到一个小人,高三寸多,面貌狰狞,丑恶得像夜叉一样,在地上转着走。他心里暗自惊异,心想不管怎么样,先看他有什么变化再说。正看着,忽听邻居有人来借东西叫门呼唤。小人听到后,样子很恐慌,围着屋内乱转,好像老鼠找不到窝一样。谭秀才也觉得神志不清,像掉了魂,不知道小人到哪里去了。随后他便得了疯癫病,哭叫不停。家人为他请医吃药,治了半年,才渐渐好了。


\subsection{1.1.3   尸 变}
\label{\detokenize{p00_u5176_u5b83/_u767d_u8bdd_u804a_u658b_u5fd7_u5f02:id5}}
阳信县某老翁,家住本县蔡店。这个村离县城五六里路。他们父子开了一个路边小店,专供过往行商的人住宿。有几个车夫,来往贩卖东西,经常住在这个店里。一天日落西山时,四个车夫来投店住宿,但店里已住满了人。他们估计没处可去了,坚决要求住下。老翁想了一下,想到了有个地方可住,但恐怕客人不满意。客人表示:“随便一间小屋都行,不敢挑拣。”当时,老翁的儿媳刚死,尸体停在一间小屋里,儿子出门买棺材还没回来。老翁就穿过街巷,把客人领到这间小房子里。

客人进屋,见桌案上有盏昏暗的油灯,桌案后有顶帐子,纸被子盖着死者。又看他们的住处,是在小里间里的大通铺上。他们四人一路奔波疲劳,很是困乏,头刚刚放在枕头上,就睡着了。其中唯有一人还朦朦胧胧地没有睡熟,忽听见灵床上嚓嚓有声响,赶快睁眼一看,见灵前灯火明亮,看的东西清清楚楚。就见女尸掀开被子起来,接着下床慢慢地进了他们的住室。女尸面呈淡金色,额上扎着生丝绸子,走到铺前,俯身对着每人吹了三口气。这客人吓得不得了,唯恐吹到自已,就偷偷将被子蒙住头,连气也不敢喘,静静听着。不多时,女尸果然过来,像吹别人一样也吹了他三口。他觉得女尸已走出房门,又听到纸被声响,才伸出头来偷看,见女尸如原样躺在那里。这个客人害怕极了,不敢作声,偷偷用脚蹬其他三人,那三人却一动不动。他无计可施,心想不如穿上衣服逃跑了吧!刚起来拿衣服,嚓嚓声又响了。这个客人赶快把头缩回被子里,觉得女尸又过来,连续吹了他好几口气才走。少待一会,听见灵床又响,知道女尸又躺下了。他就慢慢地在被子里摸到衣服穿好,猛地起来,光着脚就向外跑。这时女尸也起来了,像是要追他。等她离开帐子时,客人已开门跑出来,随后女尸也跟了出来。

客人边跑边喊,但村里人没有一人听见。想去敲店主的门,又怕来不及被女尸追上,所以就顺着通向县城的路尽力快跑。到了东郊,看见一座寺庙,听见有敲木鱼的声音,客人就急急敲打庙门。可道士在惊讶之中,认为情况异常,不肯及时开门让他进去。他回过身来,女尸已追到了,还只距离一尺远。客人怕得更厉害了。庙门外有一棵大白杨树,树围有四五尺,他就用树挡着身子。女尸从右来他就往左躲,从左来就往右躲,女尸越怒。这时双方都汗流浃背,非常疲倦了。女尸顿时站住,客人也气喘不止,避在树后。忽然,女尸暴起,伸开两臂隔着树捉那客商。客人当即被吓倒了。女尸没能捉住人,抱着树僵立在那里。

道士听了很长时间,听庙外没了动静,才慢慢走出庙门。见客人躺在地上,拿灯一照,已经死了。但摸摸心,仍有一点搏动,就背到庙里,整整一夜,客人才醒过来。喂了一些汤水,问是怎么回事。客人原原本本地说了一遍。这时寺庙晨钟已敲过,天已蒙蒙亮了。道士出门再看树旁,果然见一女尸僵立在那里。道士大惊失色,马上报告了县官。县官亲自来验尸,叫人拔女尸的两手,插得牢牢的拔不出来。仔细一看,女尸左右两手的四个指头都像钢钩一样深深地抓入树里,连指甲都插进去了。又叫几个人使劲拔,才拔了出来,只见她指甲插的痕迹像凿的孔一样。县官命衙役去老翁店里打听,才知道女尸没有了,住宿的其他三个客人已死了,人们正议论纷纷。衙役向老翁说了缘故,老翁便跟随衙役来到庙前,把女尸抬回。

客人哭着对县官说:“我们四个人一起出来的,现在我一人回去,怎么能让乡亲们相信我呢?”县官便给他写了一封证明信,并给了他些银子送他回去了。


\subsection{1.1.4   喷 水}
\label{\detokenize{p00_u5176_u5b83/_u767d_u8bdd_u804a_u658b_u5fd7_u5f02:id6}}
莱阳有个叫宋玉叔的先生,当部曹官的时候,租赁了一套宅院,很是荒凉。有一天夜里,两个丫鬟侍奉着宋先生的母亲睡在正屋,听到院里有扑扑的声音,就像裁缝向衣服上喷水一样。宋母催促丫鬟起来,叫他们把窗纸捅破个小孔偷偷地往外看看。只见院子里有个老婆子,身体很矮、驼着背,雪白的头发和扫帚一样,挽着一个二尺长的发髻,正围着院子走;一躬身一躬身像鹤走路的样子,一边走一边喷着水,总也喷不完。丫鬟非常惊愕,急忙回去告诉宋母。宋母也非常惊奇地起了床,让两个丫鬟搀扶着到窗边一起观看。忽然,那老婆子逼近窗前,直冲着窗子喷来,水柱冲破窗纸溅了进来,三个人一齐倒在地上,而其他家人们都不知道。

清晨日出时,家人们都来到正屋,敲门却没有人答应,才开始害怕。撬开门进到屋里,见宋母和两个丫鬟都死在地上。摸一摸,发现其中一个丫鬟还有体温,随即扶她起来用水灌,不多时醒了过来,说出了见到的情形。宋先生闻讯而来,悲愤得要死。细问了丫鬟那老婆子隐没的地方,便命家人们在那地方往下挖。挖到三尺多深时,渐渐地露出了白发。继续往下挖,随即露出了一个囫囵尸首,和丫鬟看见的完全一样,脸面丰满如同活人。宋先生命家人砸她,砸烂骨肉后,发现皮肉内全都是清水。


\subsection{1.1.5   瞳 人 语}
\label{\detokenize{p00_u5176_u5b83/_u767d_u8bdd_u804a_u658b_u5fd7_u5f02:id7}}
书生方栋,在长安城里很有点名气,但他为人很轻佻,不守礼节。每在郊外遇到游玩的女子,就很不礼貌地尾随在后头。

清明节的前一天,他偶然到城郊游玩,见到一辆小车子,挂着朱红色的穸帘,周着绣花簇锦的车帷,几位女婢骑着马跟在车后。其中一个婢女,骑着匹小马,容貌美丽极了。方栋稍向前凑近,偷眼一看,见车的帷幔拉开着,车里坐着一位十五六岁的女郎,她妆梳非常艳丽,真是生平从未见到过。方栋目光缭乱,神志昏昏,跟在车的前前后后,舍不得离开,这样跟着走了好几里。忽听车中女郎把婢女叫到车边,说:“给我把帘子放下来。哪里来的这么一个狂妄书生,频频地来偷看。”婢女把穸帘放下,回过头愤怒地看着方栋说:“这是芙蓉城里七郎的新妇回娘家,不是一个乡下女子,随便让秀才偷看的。”说完,就从车道上捧起一把土,朝着方栋扬去。

方栋眯眼睁不开,刚刚用手擦试眼睛,女郎的车马已经远去了。他惊恐疑惑地回到家里,总觉得眼睛里不舒服。请人扒开眼睑一看,眼球上生出了一层薄膜。过了一宿,越发严重,眼泪不止地簌簌流下来。白色的翳膜渐渐大起来,又过了几天,就像个铜钱那么厚。右边的那个眼球上,起了如同螺旋状的厚翳膜,用各种药物医治,都不见效。这时,方栋心中懊悔极了,很愧悔自己作法不当。他听说佛家的《光明经》能消除灾难,就手拿一卷,请别人教诵。最初,读时心情很烦躁,时间久了,渐渐地就习惯了。一天早晚无别的事可作,只盘腿坐着捻珠诵经。就这样他持续了一年,什么杂乱的念头也没有了。忽然,听到左边眼睛中,有如小蝇的声音,说:“黑如漆,真难受死了。”右边眼睛中应声说:“可以一同出去游玩一会儿,出出这口闷气。”方栋渐渐觉得两鼻孔中,蠕蠕动弹,很痒,好像有东西从鼻孔里面爬出来。过了一段时间,又返回来,又从鼻孔进到眼眶里。它们又说:“好长时间没能看看园中的亭台了,那珍珠兰快要枯死了。”

方栋生平很喜欢兰花,园中种植了许多兰花,以前自己常去灌水,自从两眼失明,长久没再过问。忽然听到这话,急忙问他的妻子:“兰花怎么弄得快干死了?”妻子问方栋怎么知道的,方拣就把实情告诉妻子。妻子到花园中一看,果然兰花枯萎了。妻子感到惊异,静静躲在屋里看个究竟,见有小人从方栋的鼻子中出来,大小不如一粒豆子,转转悠悠地竟到门外去了,越走越远,接着就看不清了。一会儿,两个小人又挎着胳膊回来,飞到方栋的脸上,好像蜜蜂和蚂蚁回窝一样。就这样倒腾了二三天。

方栋又听左眼中小人说: “这条隧道弯弯曲曲,来来去去很是不方便,还不如自己另开一个门。”右眼睛中小人说:“我这里的洞壁太厚,要开门不太容易。”左边的说:“我来试试看,若能开开,咱俩就住到一块算了。”方栋接着感到左眼眶内隐隐地痛似抓裂一样。一会,睁开眼一看,突然屋里的桌椅等物看得很清楚。方栋很高兴地告诉妻子。妻子仔细查看,左眼中那层小脂膜破开一个小孔,露出亮晶晶的黑色眼球,才有半个胡椒粒大。过了一宿,那层翳膜全消退了。细细一看,竟然是两个瞳人。而右眼厚厚的翳膜,仍是老样子,这才知两个瞳人合居在一个眼眶里了。方栋虽然瞎了一只眼睛,但比以前两个眼睛时看东西更清楚。自这以后,他对自己的行为,就更检点约束了,乡亲们都称赞他的品德好。


\subsection{1.1.6   画 壁}
\label{\detokenize{p00_u5176_u5b83/_u767d_u8bdd_u804a_u658b_u5fd7_u5f02:id8}}
江西的孟龙潭,与朱举人客居在京城。他们偶然来到一座寺院,见殿堂僧舍,都不太宽敞,只有一位云游四方的老僧暂住在里面。老僧见有客人进门,便整理了一下衣服出来迎接,引导他俩在寺内游览。大殿中塑着手足都作鸟爪形状的志公像。两边墙上的壁画非常精妙,上面的人物栩栩如生。东边墙壁上画着好多散花的天女,她们中间有一个垂发少女,手拈鲜花面带微笑,樱桃小嘴像要说话,眼睛也像要转动起来。朱举人紧盯着她看了很久,不觉神摇意动,顿时沉浸在倾心爱慕的凝思之中。

忽然间他感到自己的身子飘飘悠悠,像是驾着云雾,已经来到了壁画中。见殿堂楼阁重重迭迭,不再是人间的景象。有一位老僧在座上宣讲佛法,四周众多僧人围绕着听讲。朱举人也掺杂站立其中。不一会儿,好像有人偷偷牵他的衣襟。回头一看,原来是那个垂发少女,正微笑着走开。朱举人便立即跟在她的身后。过了曲曲折折的栅栏,少女进了一间小房舍,朱举人停下脚步不敢再往前走。少女回过头来,举起手中的花,远远地向他打招呼,朱举人这才跟了进去。见房子里寂静无人,他就去拥抱少女,少女也不太抗拒,于是和她亲热起来。不久少女关上门出去,嘱咐朱举人不要咳嗽弄出动静。夜里她又来到。这样过了两天,女伴发觉了,一块把朱举人搜了出来,对少女开玩笑说:“腹内的小儿已多大了,还想垂发学处女吗?”都拿来头簪耳环,催促她改梳成少妇发型。少女羞得说不出话来。一个女伴说:“姊妹们,我们不要在这里久待,恐怕人家不高兴。”众女伴笑着离去。朱举人看了看少女,像云一样形状的发髻高耸着,束发髻的凤钗低垂着,比垂发时更加艳绝人寰。他见四周无人,便渐渐地和少女亲昵起来,兰花麝香的气味沁人心脾,两人沉浸在欢乐之中。

忽然听到猛烈的皮靴走路的铿铿声,并伴随着绳锁哗哗啦啦的声响。旋即又传来乱纷纷的喧哗争辩的声音。少女惊起,与朱举人一起偷偷地往外看去,就见有个穿着金甲的神人,黑脸如漆,手握绳锁,提着大槌,很多女子围绕着他。金甲神说:“全到了没有?”众女回答:“已经全到了。”他又说:“若有藏匿下界凡人的,你们要立即告发,不要自己找罪受!”众女子同声说:“没有。”金甲神反转身来像鱼鹰一样凶狠地看着周围,像要进行搜查。少女非常害怕,吓得面如死灰,慌张失措地对朱举人说:“赶快藏到床底下。”她自己则开开墙上的小门,仓皇逃去,朱举人趴在床底下,大气不敢出。不久听到皮靴声来到房内,又走了出去。一会儿,众人的喧闹声渐渐远去,朱举人的心情才稍稍安稳了一点。然而门外总是有来往说话议论的人。他心神不宁地趴了很久,觉得耳如蝉鸣,眼里冒火,几乎没法忍耐。但也只有静静听着,等待少女归来,竟然不再记得自已是从哪里来的了。

当时孟龙潭在大殿中,转眼不见了朱举人,便很奇怪地问老僧。老僧笑着说:“去听宣讲佛法去了。”孟龙潭问道:“在什么地方?”老僧回答说:“不远。”过了一会儿,老僧用手指弹着墙壁呼唤说:“朱施主游玩这么久了,怎么还不归来?”立即见壁画上出现了朱举人的像,他侧耳站立,像是听见了。老僧又呼唤说:“你的游伴久等了。”朱举人于是飘飘忽忽从墙壁上下来,灰心呆立,目瞪足软。孟龙潭大为吃惊,慢慢问他。原来朱举人刚才正伏在床下,听到叩墙声如雷,因此出房来听听看看。这时他们再看壁画上那个拈花少女,已是螺髻高翘,不再垂发了。朱举人很惊异地向老僧行礼,问他这是怎么回事。老僧笑着说:“幻觉生自人心,贫僧怎么能解呢!”朱举人胸中郁闷不舒,孟龙潭心中则惊骇无主。两人立即起身告辞,顺阶而下出门离去。


\subsection{1.1.7   山 魈}
\label{\detokenize{p00_u5176_u5b83/_u767d_u8bdd_u804a_u658b_u5fd7_u5f02:id9}}
孙太白曾说过这么件事,他的曾祖父以前在南山柳沟寺读书,麦秋时节回家,过了十天又返回寺里。孙公打开他住的房门,见桌案上满是尘土,窗户上也有了蜘蛛网,便命仆人打扫清除。到了晚上才觉得清爽些,可以休息休息了。于是他扫扫床,铺开被褥,关门睡觉。

这时,月光照满窗,他躺在床上翻来复去多时,没睡着,觉得万籁俱寂。忽然间听到风声呼啸,山门被风刮得咣当咣当直响,孙公心想可能是和尚没关好门。他正寻思间,风声逐渐接近住房,一霎时,房门也被刮开了。他更心疑了,还设想过来是怎么回事,风声已入屋内,并伴有铿铿的靴声,逐渐靠近卧室门口。这时他心里才害怕起来。霎时门开了,他急忙一看,一个大鬼弓着身子塞了进来,矗立在床前,头几乎触着梁,面似老瓜皮色,目光闪闪,向屋内四面环视。张开如盆大口,牙齿稀疏,长三寸多。哇啦哇啦乱叫,声音震得四面墙壁山响。

孙公害怕极了,心想在这咫尺的小房子里,势必无法逃避,不如与它拼了。于是暗暗去抽枕下的佩刀,猛地拔出向大鬼砍去,正砍中了它的肚子,发出像砍石头样的声音。鬼大怒,伸出大爪子抓他。孙公稍微缩了缩身子,被鬼抓住了被子,揪着忿忿地走了。孙公随被子掉到了地上,趴在地上大叫。家人都拿着火把赶来,见门依然关着,如以前一样,只得推开窗户进来。一见孙公的样子,众人都很惊讶。把他抬到床上,他才把事情的前后说了一遍。共同检查一下,才看到被子夹在寝室的门缝里。开门用火把照着检查,见有爪痕,大如簸箕,五个指爪碰到哪里哪里就被穿透。天明,孙公再也不敢留在这里,于是便背起书箱回家了。后来再问寺里的和尚,他们说再没有异常事情发生。


\subsection{1.1.8   咬 鬼}
\label{\detokenize{p00_u5176_u5b83/_u767d_u8bdd_u804a_u658b_u5fd7_u5f02:id10}}
沈麟生说:他的朋友某翁,夏天午睡,朦朦胧胧之中,见一个女子掀帘进屋,头上裹着白布,穿着丧服,竟向里屋走去。老翁心想,可能是邻居家妇女来找自己妻子。可又一想,为什么穿着不吉利的衣服到人家里去呢?正自疑惑间,那女子已从里屋走出。他仔细一看,这女子大约有三十多岁,脸色发黄膨肿,眉眼很不舒展,神情可怕。女子犹豫着不走,渐渐靠近老翁的床前。老翁假装睡着,看要发生什么事。

不多时,女子穿着衣服上了床,压在老翁的肚子上,老翁感觉有几百斤重。心里虽然什么都明白,但想举手,手如被捆绑;想抬脚,脚无力不能动。急得想呼喊求救,又苦于喊不出声来。接着,女子用嘴去嗅他的脸,腮、鼻、眉、额,都嗅了一遍。老翁觉得她的嘴如凉冰,寒气透骨。他急中生智,想等她嗅到腮边时,狠狠咬她一口。没有多大会儿,果然嗅到腮边,老翁趁势猛力咬住了她的颧骨,牙都咬进肉里去了。女子觉得疼,想赶紧离开,一面挣扎,一面哭叫。但老翁越是使劲咬住,直觉血水流过面颊,浸湿了枕头。

正在两相苦挣之际,听到院子里妻子的声音,老翁急喊:“有鬼!”一松口,女子已飘然逃走。妻子跑进屋里,什么也没看见,笑他做了个恶梦罢了。老翁详细说了这件怪事,并说有枕头上的血迹为证。两人查看,果然有像屋上漏的水一样的东西,淌湿了枕头和席子。趴下嗅一嗅,腥臭异常。老翁恶心得大吐,过了几天,口中还有残余的臭味。


\subsection{1.1.9   捉 狐}
\label{\detokenize{p00_u5176_u5b83/_u767d_u8bdd_u804a_u658b_u5fd7_u5f02:id11}}
孙老翁,是我亲家孙清服的伯父,一向很有胆量。一个白天,他正躺着休息,觉得仿佛有什么东西爬上了床,接着感觉身子摇摇晃晃,如同腾云驾雾。他心中暗想,难道是被狐狸精魇住了?便眯缝着眼悄悄地偷看,见一物大如猫,一身黄毛,却长着绿色的嘴巴,正从脚边慢慢地爬来。它轻轻地蠕动着,像是怕惊醒了老翁似的。一会儿,就贴到孙老翁的身上,挨着脚,脚瘫;靠着腿,腿软。待它刚刚爬到腹部,孙老翁突然坐了起来,猛地按下,把它捉住,两手掐住它的脖子。它急得嗥叫,却不能挣脱。

孙老翁急忙把夫人喊来,用绳子捆起它的腰,勒紧绳子两头,笑着说:“听说你善于变化,今天我在这里盯着你,看你怎么个变法。”说话间,它忽然把肚子缩得像细管,几乎把绳子脱去逃掉。孙老翁大惊,急忙用力勒紧绳子。可它又鼓起肚子,像碗口一样粗,再也勒不下去。孙老翁气力稍一松,它又缩了下去。

孙老翁怕它跑了,叫夫人赶快拿刀来把它杀掉。老夫人惊慌地四处寻找,竟不知刀放在什么地方。孙老翁向左摇头,目示放刀的位置。等回过头来,手中只剩下一个如环样的空绳套子,而那狐狸已经不知去向了。


\subsection{1.1.10   荍 中 怪}
\label{\detokenize{p00_u5176_u5b83/_u767d_u8bdd_u804a_u658b_u5fd7_u5f02:id12}}
长山县有一个老翁,姓安,生性喜欢务农。有一年秋天,他种的荞麦熟了,割了堆到地边。因怕邻村偷庄稼的贼,安老翁就命令佃户趁着月光用车运到场上。等佃户装车推走后,他自己留下守护还没运走的庄稼,头下枕着长矛,露天躺在地上,稍稍闭着眼休息。

猛然间他听到有人踏着荞麦根走来,吱吱咯咯地响。他心想可能有贼,猛一抬头,见一个大鬼,身高一丈多,红头发,乱胡须,已走到身前。安老头很害怕,来不及想别的办法,猛地跳起用长矛狠狠刺去。鬼大叫一声,如打雷一般,随即不见了。他怕鬼再回来,就扛起矛回村。走到半路,遇到佃户们,安老翁把刚才的事一五一十地告诉了他们,并告诫他们不要再去了。大伙还有点不大相信。

到了第二天,把荞麦晒在场上,忽然听到空中有声。安老翁大惊,喊道:“鬼来了!”喊罢就跑,大伙也跟着跑。过了一会儿,没有事,又纷纷回来。安老翁命大伙多准备一些弓箭,等候鬼来。又过了一天,鬼果然又来了,大伙乱箭齐发,鬼被吓跑了。此后两三天没有再来。

荞麦晒打完毕入了仓,场上仍有乱麦秸杆。老翁命佃户收积起来堆成垛,他在垛顶上用脚踩实。等垛高数尺时,他忽然在垛顶上望着远处高呼:“鬼来了。”大伙急着找弓箭时,鬼已到老翁身边,老翁倒在了垛上,鬼啃了他的前额一口就走了。大伙都到垛上去看时,老翁的前额已被那鬼啃去了手掌大的一块皮肉。老翁昏迷不醒人事,大伙抬他回家,很快就死了。以后那怪物没有再来,也没有人知道那是什么怪物。


\subsection{1.1.11   宅 妖}
\label{\detokenize{p00_u5176_u5b83/_u767d_u8bdd_u804a_u658b_u5fd7_u5f02:id13}}
长山县李公,是李大司寇的侄子,他家里经常有妖异出观,一次,李公见厅上有条长板凳,呈肉红色,非常细润。他因为以前没有见过这东西,所以走近摸了摸。一摸,板凳随手弯曲起来,和肉一样软。李公吓了一跳,拔腿就走。边走边同头看,那东西四腿动了起来,渐渐地隐入墙壁中去了。又有一次,李公见墙壁上竖着一根白色细长的木杖,非常光滑干净。他走近用手一扶,木杖便软绵绵地倒下,像蛇一样弯曲地钻向墙内,一会儿也看不见了。

康熙十七年,有一个书生王俊升在李公家教书。一日黄昏时候,刚点上灯,王先生穿着鞋躺在床上。忽然看见一个小人,长三寸多,从门外进来,稍微打了个转就又出去了。过了一会儿,小人拿了两只小凳来,放在屋正中,像小孩用高梁秸做的玩具小凳一样。又过了一会儿,两个小人抬了一口棺材进来,不过四寸多长,放在两只小凳上。安排还没就绪,又见一女子带领几个丫鬟佣人进来,都像先前小人一样的细小。女子身穿孝服,腰扎麻绳,头裹白布,用袖子捂着嘴,细声细气地啼哭,那声音就象大苍蝇叫一般。王先生偷看了很长时间,吓得毛骨悚然,浑身像霜打了一样凉。他大叫一声,拔腿就跑,可是没能跑掉反而跌倒在床下,浑身颤抖,站不起来。当馆里的人们听到喊叫声急忙跑来看时,屋里的小人和小物全都不见了。


\subsection{1.1.12   王 六 郎}
\label{\detokenize{p00_u5176_u5b83/_u767d_u8bdd_u804a_u658b_u5fd7_u5f02:id14}}
有个姓许的,家住淄川县城北,以打鱼为生。他每天傍晚总要带酒到河边去,边喝酒边打鱼。而喝酒前,又总是先斟上一盅祭奠一下,并祷告说:“河中的溺鬼,请来喝酒吧!”这样便习以为常。其他人往往打鱼很少,而他每天都打满筐的鱼。

一天傍晚,许某刚刚独自饮酒,见一少年走来,在他身边转来转去。许某让他同饮,少年也不推辞,二人便对饮起来。这一夜竟连一条鱼也未能打到,许某很有些丧气。少年起立躬身说: “我到下游为你赶鱼。”说罢,朝下游飘然走去。一会儿,少年回来说:“大群鱼来了!”果然听到有许多鱼吞吃饵食的声音。许某便撒网,一网捕了十数尾尺把长的大鱼。他非常高兴,对少年深表感谢。少年欲走,许送鱼给他,少年不要,并说:“屡次喝你的好酒,这点小事怎能提到感谢呢?如您不嫌麻烦,我将常来找您。”许某说:“才相见一晚,怎说多次?你如愿来相助,我是求之不得,可我怎样报答你的情意呢?”于是便问少年姓名。少年说:“我姓王,没有名字,你见面就叫我王六郎吧。”说罢,便告辞而去。

次日,许某将鱼卖掉,顺便多买了些酒。当晚,许某来到河边时,六郎早已先在等候,二人便开怀畅饮。饮几杯后,六郎便为许某赶鱼。就这样半年过去了。一天,六郎忽然对许说:“你我相识,情同手足,可是,咱们马上就要分别了。”说得很是悲伤。许某甚为诧异,问六郎为何这样,六郎考虑再三,才说:“你我既然亲如兄弟,我说了你也不必惊讶。如今将要分别,无妨如实告知:我实际是一鬼,只因生前饮酒过量,醉后溺水而死,已经好几年了。以前你之所以捕到比别人更多的鱼,都是我暗中帮你驱赶,以此来酬谢奠酒之情。明日我的期限已满,将有人来代替我,我将要投生于人间,你我相聚只有今晚了,所以我不能平静。”许某听了起初了分害怕,然而,因为长期相处,不再恐怖,反而难过起来。于是,他满满斟了一杯酒捧在手中说:“六郎,我敬你这杯酒!望你饮了不要难过。你我从此不能相见,虽很伤心,但你由此解脱灾难,我应该祝贺你。不要悲伤,应该高兴才是!”于是,二人继续畅饮。许问六郎:“何人来相替?”六郎说:“兄长明天可在河边阴处等候,正当午时,有一女子渡河,溺水而死,即是替我之人。”二人听到村鸡鸣叫,方洒泪而别。

次日,许在河边暗暗观看,会发生什么事情。中午时,果有一怀抱婴儿的妇女,到河边便坠入水中。婴儿被抛在岸上,举手蹬脚地啼哭。妇女几次浮上沉下,后竟又水淋淋地爬上河岸,坐在地上稍稍休息后,抱起婴儿走了。

当许某看到妇女掉入水中时,很不忍心,想去相救,但一想这是六郎的替身,才打消救人的念头。当又看到妇人未溺死,心中怀疑六郎所言有些荒唐。

当晚,许某仍到原地去打鱼,而六郎早已在那里,说:“现在又相聚了,可暂先不说分别的事。”许某问六郎白天的事,六郎说:“本来那女子是替我的,但我怜她怀中婴儿,不忍心为了自己一人而伤两个人的性命。因此,我决定舍弃这个机会,但又不知何时再有替死的人。也许是你我缘分未尽啊。”许某慨叹地说:“你这种仁慈之心,总可感动上帝的。”从此,二人一如既往,饮酒捕鱼。

过了几天,六郎又来向许某告别,许以为又有替六郎之人。六郎说:“不是的,我前次之好心果然感动了上帝,因而招我为招远县邬镇的土地。明日要去赴任,如你不忘咱俩的交情,不要嫌路远,去招远看我。”许某祝贺说:“贤弟行为正直而做了神,我感到十分欣慰。但人和神之间相隔遥远,即使我不怕路远,又怎样才能见到你呢?”六郎说:“只管前往,不要顾虑。”再三嘱咐而去。

许某回到家,便要骨办行装东下招远。他妻子笑着说:“这一去几百里路,即使有这个地方,恐怕和一个泥偶象也无法交谈。”许某不听,竟然去了招远。问当地居民,果然有个邬镇。他找到了邬镇,便住进一个客店,向主人打听土地祠在什么地方。主人惊异地说:“客人莫非姓许?”许某说:“是的,但是您怎么知道?”店主人又问:“客人莫非是淄川人?”许某说:“是的,然则您又是怎么知道的?”店主人并不回答,很快地走出去。过了一会,只见丈夫抱着小儿,大姑娘小媳妇在门外偷看,村里人纷纷到来,围看许某,如四面围墙一般。许某更为惊异。大家告诉他说:“前几夜,梦见神人来告知:有一个淄川姓许的人将来此地,可以给些资助。因而在此等候多时。”许某甚为奇怪,便到土地祠祭祀六郎,祷告说:“自从与你分别后,睡梦中都铭记在心,为此远道而来赴昔日之约。又蒙你托梦告知村里人,心中十分感谢。很惭愧我没有厚礼可赠,只有一杯薄酒,如不嫌弃,当如过去在河边那样对饮一番。”祷告毕,又烧了些纸钱。顷刻见到一阵旋风起于神座之后,旋转许久才散去。

当夜,许某梦到六郎来到,衣冠楚楚的,与过去大不相同。六郎致谢道:“有劳你远道而来看望我,使我又欢喜又悲伤。但我现在有职务在身,不便与你相会,近在咫尺,却如远隔山河,心中十分凄怆。村中人有微薄的礼物相赠,就算代我酬谢一下旧日的好友。当你回去的时候,我必来相送。”

许某住了几天,打算回家,大家殷勤挽留,每天早晚都轮流作东道主为许某饯行。许坚决告辞,村中人争着送来许多礼物,为他充实行装。不到一天,送的礼物装满行囊,男女老少都聚集来进许出村。忽然刮起一阵旋风,跟随许某十余里路。许对着旋风再拜说:“六郎珍重,不要远送了。你心怀仁爱,自然能为一方百姓造福,无需老朋友嘱咐了。” 旋风又盘旋许久,才离去。村中的人也都嗟叹着返回了。

许某回到家里,家境稍稍宽裕些,便不再打鱼了。后来见到招远的人,向他们打听土地的情况,据说灵验得像传说的那样,远近闻名。


\subsection{1.1.13   偷 桃}
\label{\detokenize{p00_u5176_u5b83/_u767d_u8bdd_u804a_u658b_u5fd7_u5f02:id15}}
我童年的时候,一次到济南府参加考试,正巧遇到过春节。接旧风俗,春节的前一天,城里的各行各业作生意的,要抬着彩楼,吹吹打打地到布政司衙门去祝贺春节,这叫做“演春”。我也跟着朋友到那里去看热闹。

那天,游人很多,人们把四面围得像堵墙,水泄不通。大堂上坐着四位官员,身上都穿着红袍,东西面对坐着。那时我年纪还小,也不懂得堂上是什么官。只听得人声嘈杂,鼓乐喧天,震耳欲聋。忽然有一个人,领着一个披头散发的童子,挑着一副担子,走上堂来,好像说了一些话,只是人声鼎沸,也听不见他说了些什么,只见大堂上的人在笑。接着,就有个穿黑色衣服的衙役传话说,让他们演戏。那人答应了,刚要表演,又问道:“耍什么戏法?”堂上的人相互商量了几句,就见有个衙役走下堂来,问他有什么拿手的好戏法。那人回答道:“我能颠倒生物的时令,生长出各种各样的东西。”衙役回到堂上禀报后,又走下来,说叫他表演取桃子。

耍戏法的点头答应了,脱下衣服盖在竹箱上,故意装出一副埋怨的样子说:“官长们委实不明白事理,眼下冰还没有化,叫我哪里去取桃子呢?不去取吧,怕惹得官长生气,这可叫我怎么办?”他的儿子说:“父亲已经答应了,又怎么好推辞呢?”耍戏法的人为难了一阵子,说道:“我认真想过了,眼下还是初春天气,冰雪还未融化,在人间哪里能找到挑子啊?只有王母娘娘那蟠桃园里,四季如春,兴许会有桃子。可是,必须到天上去偷,才能得到桃子。”儿子说:“嘻!天可以像有台阶似地走上去吗?”耍戏法的说:“我自有办法。”说完,就打开竹箱子,从里面取出一团绳子,大约有几十丈长。他理出一个绳头,向空中一抛,绳子竟然挂在半空,好像有什么东西牵着似的。眼看着绳子不断上升,愈升愈高,隐隐约约地升到云端,手中的绳子也用完了。这时,他把儿子叫到身边,说:“孩子你来,我老了,身体疲乏、笨拙,上不去,你替我走一趟吧。”接着就把绳子头交给儿子,说:“抓着这根绳子就可登上去。”

儿子接过绳子,脸上显出很为难的样子,埋怨说:“爹爹真是老糊涂了,这样一条细细的绳子,就叫我顺着它爬上万丈高天。假若中途绳子断了,掉下来也是粉身碎骨。”父亲哄着而又严肃地说:“我已经出口答应人家,后悔也来不及了,还是麻烦儿子去走一趟。不要怕苦,万一能偷得来桃子,一定能得到百金的赏赐,那时我一定给你娶个漂亮的媳妇。”儿子无奈,用手拉住绳子,盘旋着向上攀去;脚随着手向上移动,活像蜘蛛走丝网那样,渐渐没入云端,看不见了。过了一会,从天上掉下一个桃子,像碗口那么大。耍戏法的很高兴,用双手捧着桃子,献到堂上。堂上的官员看了老半天,也说不清是真是假。这时,绳子忽然从天上落下来,耍戏法的惊惶失色地喊道: “糟了!天上有人把绳子砍断了,我儿子可怎么下来啊?”又过了一会儿,又掉下个东西来,一看,原来是他儿子的头。他捧着儿子的头哭着说:“这一定是偷桃时,被那看守人发现了,我的儿子算完了。”正哭得伤心时,从天上又掉下一只脚来;不一会,肢体、躯干都纷纷落下来。

耍戏法的人很是伤心,一件一件地都捡起来装进箱子,然后加上盖说:“老汉只有这么个儿子,每天跟我走南闯北。今天遵照官长的严命,没有料到遭到这样的惨祸,只好把他背回去安葬。”于是,他走到堂上,跪下哀求说:“为了去偷桃子,我儿子被杀害了!大人们可怜小人,请赏给几个钱,也好收拾儿子尸骨。日后,我死了也当报答各位官长的恩情。”

堂上的官员很惊骇,各自拿出许多银钱赏他。他接过钱缠到腰上,从堂上走下来,用手拍打着箱子,招呼说:“八八儿啊,不赶快出来谢谢各位大人的赏钱,还等到什么时候!”忽然,一个披头散发的小孩用头顶开箱盖,从箱子里走出来,朝堂上叩头。一看,原来就是他的儿子。

因为这个戏法耍得太神奇了,直到现在我还记得很深刻。后来听人说,白莲教能表演这个法术。我想,这可能就是他们的后代吧?


\subsection{1.1.14   种 梨}
\label{\detokenize{p00_u5176_u5b83/_u767d_u8bdd_u804a_u658b_u5fd7_u5f02:id16}}
有个乡下人,在集市上卖梨。梨的味道非常香甜,但价钱很贵。有个道士,戴着破头巾,穿着破烂道袍,在车前伸手向乡下人乞讨。乡下人呵斥他,他也不走。乡下人生气了,大声地辱骂起来。道士说:“你这一车梨有好几百个,贫道只讨你一个,对你来说没多大损失,为什么还要发这么大的脾气呢?”观看的人劝乡下人拿一个不好的梨给老道士,打发他走算了,乡下人坚决不肯。路旁店铺里的一个伙计,见他们吵得不成样子,就拿出钱买了一个梨,给了道士。道士拜谢,然后对着众人说:“出家人不知道吝惜东西。我有好梨,请大家品尝。”有人问:“你既然有梨,为什么不吃自己的?”道士说:“我是需要这个梨核做种子。”于是捧着梨大口大口地吃了起来。

道士吃完梨,把核放在手里,取下背在肩上的小铁铲,在地上挖了个几寸深的坑,然后放进梨核,盖上土,向旁边的人要点热水浇灌。有好事的人便到路边店铺中提来一壶滚开的水,道士接过开水浇进了坑里。大家都瞪着眼看着,见一棵嫩芽儿冒了出来,并渐渐长大,一会儿就长成了一棵枝繁叶茂的大树;转眼间开花、结果,又大又香的梨子挂满了枝头。道士从树上摘下梨子,分给围观的人吃,一会儿功夫就吃光了。然后,道士就用铁铲砍树,叮叮当当地砍了好长时间方才砍断。道士把满带枝叶的梨树扛在肩上,不慌不忙地走了。

一开始,道士做戏法时,那个乡下人也杂在人群中,伸着脖子瞪着眼看,竟忘记了自己的营生。道士走了以后,他才回来去看顾他车上的梨,却已经一个也没有了。他这才恍然大悟,道士刚才分的梨子都是他的;再细细一看,一根车把没有了,碴口是新砍断的。乡下人心里非常气愤,急忙去追赶道士。转过一个墙角,见砍断的车把扔在墙角下,这才知道道士刚才砍的那棵梨树,就是他的车把,而道士却已经不知去向了。满集市上的人都笑得合不上嘴。


\subsection{1.1.15   劳 山 道 士}
\label{\detokenize{p00_u5176_u5b83/_u767d_u8bdd_u804a_u658b_u5fd7_u5f02:id17}}
县里有个姓王的书生,排行第七,是官宦之家的子弟,从小就羡慕道术。他听说崂山上仙人很多,就背上行李,前去寻仙访道。

他登上一座山顶,看见一所道观,环境非常幽静。有一个道士坐在蒲团上,满头白发披肩,两眼奕奕有神。王生上前见过礼并与他交谈起来,觉得道士讲的道理非常玄妙,便请求道士收他为徒。道士说:“恐怕你娇气懒惰惯了,不能吃苦。”王生回答说:“我能吃苦。”

道士的徒弟很多,傍晚的时候都集拢来了。王生一一向他们行过见面礼,就留在道观中。

第二天凌晨,道士把王生叫去,交给他一把斧头,让他随众道徒一起去砍柴。王生恭恭敬敬地答应了。过了一个月,王生的手脚都磨出了厚厚的老茧,他再也忍受不了这样的苦累,暗暗产生了回家的念头。

有一天傍晚,他回到观里,看见两个客人与师傅共坐饮酒。天已经晚了,还没有点上蜡烛。师傅就剪了一张像镜子形状的纸,贴在墙了。一会儿,那纸变成一轮明月照亮室内,光芒四射。各位弟子都在周围奔走侍候。

一个客人说:“良宵美景,其乐无穷,不能不共同享受。”于是,从桌上拿起酒壶,把酒分赏给众弟子,并且嘱咐可以尽情地畅饮。王生心里想,七八个人,一壶酒怎么能够喝?于是,各人寻杯觅碗,争先抢喝,惟恐壶里的酒干了。然而众人往来不断地倒,那壶里的酒竟一点儿也不少。王生心里非常纳闷。

过了一会儿,一个客人说:“承蒙赐给我们月光来照明,但这样饮酒还是有些寂寞,为什么不叫嫦娥来呢?”于是就把筷子向月亮中扔去。只见一个美女,从月光中飘出,起初不到一尺,等落到地上,便和平常人一样了。她扭动纤细的腰身、秀美的颈项,翩翩地跳起“霓裳舞”。接着唱道:“神仙啊,你回到人间,而为什么把我幽禁在广寒宫!”那歌声清脆悠扬,美妙如同吹奏箫管。唱完歌后,盘旋着飘然而起,跳到了桌子上,大家惊奇地观望之间,已还原为筷子。师傅与两位客人开怀大笑。

又一位客人说:“今晚最高兴了,然而我已经快喝醉了,二位陪伴我到月宫里喝杯饯行酒好吗?”于是三人移动席位,渐渐进入月宫中。众弟子仰望三个人,坐在月宫中饮酒,胡须眉毛全都看得清清楚楚,就像人照在镜子里的影子一样。

过了一会儿,月亮的光渐渐暗淡下来,弟子点上蜡烛来,只见道士独自坐在那里,而客人已不知去向。桌子上菜肴果核还残存在那里。那墙上的月亮,只不过是一张像镜子一样的圆的纸罢了。道士问众弟子:“喝够了吗?”大家回答说:“够了。”道士说:“喝够了就早去睡觉,不要耽误了明天打柴。”众弟子答应着退了出去。王生心里惊喜羡慕,回家的念头随即打消了。

又过了一个月,王生实在忍受不了这种苦累,而道士还是连一个法术也不传授,他心里实在憋不住,就向道士辞行说:“弟子不远数百里来拜仙师学习,即使不能得到长生不老的法术,若能学习点小法术,也可安慰我求教的心情。如今过了两三个月,不过早上出去打柴,晚上回来睡觉。弟子在家中,从没吃过这种苦。”道士笑着说:“我本来就说你不能吃苦,现在果然如此。明天早晨就送你回去。”王生说:“弟子在这里劳作了多日,请师傅稍微教我一点儿小法术,我这次来也算没白跑一趟。”道士问:“你要求学点什么法术?”王生说:“平常我见师傅所到之外,墙壁也不能阻挡,只要能学到这个法术,我就知足了。”道士笑着答应了。于是就传授他秘诀,让他自己念完了,道士大声说:“进墙去!”王生面对着墙不敢进去。道士又说:“你试着往里走。”王生就从容地向前走,到了墙跟前,被墙挡住。道士说:“低头猛进,不要犹豫!”王生果然离开墙数步,奔跑着冲过去,过墙时,像空虚无物;回头一看,身子果然在墙外了。王生非常高兴,回去拜谢了师傅。道士说:“回去后要洁持自爱,否则法术就不灵验。”于是就给他些路费,打发他回去了。

王生回到家里,自己夸耀遇到了仙道,坚固的墙壁也不能阻挡他。他的妻子不相信。王生便仿效起那天的一举一动,离墙数尺,奔跑着冲去,头撞到坚硬的墙上,猛然跌倒在地。妻子扶起他来一看,额头上鼓起大包,像个大鸡蛋一样。妻子讥笑他,王生又惭愧又气愤,骂老道士没安好心。


\subsection{1.1.16   长 清 僧}
\label{\detokenize{p00_u5176_u5b83/_u767d_u8bdd_u804a_u658b_u5fd7_u5f02:id18}}
山东长清地方,有位道业高深、品行纯洁的老僧,八十多岁了还很康健。一天,他突然跌倒起不来了,寺里的僧人跑过去抢救,一看已经圆寂了;而他并不知道自己已死,灵魂飘然而去,到了河南地界。

河南有个旧官宦世家的子弟,这天率领十几个骑马的侍从,架着猎鹰打兔子。忽然马受惊狂奔不止,公子从马上掉下来摔死了。这时老僧的灵魂恰好与公子的尸体相遇,倏忽而合,公子竟然渐渐苏醒过来。奴仆们围着他问讯,他睁开眼说:“怎么来到这里!”众人扶着他回了家。

公子进门,搽粉描眉的姬妾们,纷纷聚集过来看望慰问。他大惊说:“我是僧人,怎么来到了这里!”家人以为太荒唐,都扯着他的耳朵恳切开导,促使他醒悟。他也不自我辩解,只是闭着眼不再说话。给他粗米饭才吃,凡是酒肉却拒绝。夜里他独自睡觉,不和妻妾在一起。几天后,他忽然想稍微走动一下。家人都很高兴。出了房门后,他刚刚站定,就有几个仆人来到,拿着钱粮帐册,纷纷请他审理收支情况。公子推托因为有病倦怠,全都拒绝办理,惟独问道:“山东的长清县,知道在哪里吗?”仆人们都回答说:“知道。”公子说:“我烦闷无聊,要去那里游览一下,快备办行装。”众人说他病才痊愈,不应出远门,但他不听,第二天就出门上路了。

到了长清,他见当地的风光景物犹如昨天一样。不用烦劳问路,竟然到了佛寺。那老僧的好几个弟子见贵客来到,都非常恭敬地前来拜见。公子就问道:“原来的老僧到哪里去了?”他们回答说:“我们的师父前些时候已经去世了。”公子又问老僧的墓地。众僧引导着他前去,看了看那三尺孤坟,荒草还没长满。僧人们都不知这位公子是什么意思。不久公子备马要走,嘱咐说:“你们的师父是个恪守戒律的僧人,他遗留下的手迹,应当谨慎地守护好,不要使它受到损害。”众僧很恭敬地答应了,公子这才离去。回到家后,他木然呆坐,一点也不过问家务。

过了几个月,公子出门自己走去,直到长清旧寺。他对弟子们说:“我就是你们的师父。”众僧怀疑他说得荒唐,相视而笑。老僧于是叙述了他还魂的经过,又说了自己生前的所作所为,全都符合事实。众僧这才信以为真,让他睡在原来的床上,仍像过去那样侍奉他。

后来公子家里屡次派车马来,苦苦地请他回家,他丝毫都不理会。又过了一年多,公子的夫人派管家来到长清寺院,赠送了很多东西。凡是金银绸缎他一概不要,只收下一件布袍而已。公子的朋友中有人到了长清,去寺院拜访他。见他默然处之,心志坚定;虽年仅三十多岁,却总说他八十多年所经历的事情。


\subsection{1.1.17   蛇 人}
\label{\detokenize{p00_u5176_u5b83/_u767d_u8bdd_u804a_u658b_u5fd7_u5f02:id19}}
东郡有个人,以耍蛇为生。他曾经驯养着两条蛇,都是青色的,把大的叫大青,小的叫二青。二青的前额上长有红点,尤其聪明驯服,指挥它盘旋表演无不如意。因此,蛇人对它的宠爱,超过了其它的蛇。

过了一年,大青死了,蛇人想再找一条来补上空缺,但一直没顾得上。一天晚上,他寄宿在山里的一所寺院。天明,打开竹箱一看,二青也不见了。蛇人懊恼得要死,明处暗处搜寻呼叫,始终连个影子也没见到。先前每到草木丰盛的地方,就把蛇放出去,让它们自由自在一番,不久自己就会回来。由于这个原因,蛇人还希望它自己能回来,便坐着等待。直到太阳升起很高,自己也绝望了,才怏怏不乐地离开。

出门刚走了几步,蛇人忽然听见杂乱的草丛中,传米窸窸窣窣的声音。他停下脚步惊愕地一看,是二青回来了。蛇人非常高兴,像得了无价之宝似的。把担子放在路边,二青也停下来。再一看它的后边,还跟着一条小蛇。他抚摸着二青说道:“我还以为你跑了呢。那小家伙是你推荐来的吗?”说着就拿出饲料来喂它,同时也给小蛇一些。小蛇虽然不离开,但畏缩在那里不敢来吃。二青用嘴含着饲料喂它,好像主人招待客人似的。蛇人再喂它,它才吃了。吃完,小蛇跟随二青一块钻进了竹箱中。

蛇人挑回去训练,小蛇盘旋弯曲都合要求,与二青没有多少差别。因此给它取名叫小青。蛇人带着它俩,四方表演献技,赚了不少钱。

一般耍蛇人耍弄的蛇,不超过二尺,再大就太重了,就得更换一条。因为二青很驯良,所以蛇人没有马上把它换掉。又过了二三年,二青已长到三尺多长了,卧进竹箱里,竹箱被塞得满满的,于是蛇人决定把它放走。

一天,蛇人来到淄川县东山里,拿出最好的食物喂二青,向它祝福一番后便把它放了。二青走了,一会儿却又回来了,围着竹箱蜿蜒地爬。蛇人挥手赶它说:“走吧!世上没有百年不散的宴席。从此以后,你隐身在深山大谷中,将来一定能修练成一条神龙。竹箱怎么可以长期居住呢?”二青才离去,蛇人目送它离开。但一会儿二青又回来,蛇人怎么赶它也不走,还用头碰竹箱,小青在竹箱里也不停地窜动。蛇人恍然大悟说:“你是不是想和小青告别呀?”说着就打开竹箱。小青从竹箱里径直窜出来,二青与它交头吐舌,好像互相嘱咐话语。接着两条蛇依偎着一起走了。蛇人正在想小青不会回来了,一会儿小青却又独自回来,爬进竹箱卧下。

从此,蛇人随时都在寻找物色新蛇,但一直没有合适的。而小青也渐渐长大,不便于表演了。后来蛇人得到一条蛇,也很驯服,然而到底不如小青出色。这时小青已经长得比小孩的胳膊还要粗了。

先前,二青在山中,打柴的人经常见到它。又过了几年,二青长得好几尺长,碗口那么粗,渐渐地出来追赶人。因此,行人旅客都互相告诫,不敢从它出没的那条路走。一天,蛇人经过那里,一条蛇猛然窜出,行如骤风。蛇人大为惊恐,拼命奔跑。蛇追得更急。他回头一看已经快追上了,突然看见蛇头上俨然有一个红点,这才明白这就是二青。他放下担子,高声叫道:“二青,二青!”那蛇顿时停住,昂起头来呆了很久,纵身上前把蛇人缠住,就像以前表演的样子。蛇人察觉到二青并没有害他的意思,只是身躯太重,自己经不起它缠绕。只好倒在地上高声祈祷,于是二青就放开了他。二青又用头去碰竹箱子。蛇人明白了它的意思,打开竹箱放出小青。两条蛇一相见,立即紧紧交缠得像饴糖一样粘在一起,很久才分开。蛇人祝福小青说:“我早就想和你分别,今天你有伴了。”又对二青说:“小青原本是你引来的,还可以领它走。我再叮嘱你一句话,深山里不缺你的吃喝,不要惊扰过路行人,免得遭受上天的惩罚。”二条蛇都垂下头,好像接受了他的劝告,马上窜起离去,二青在前,小青在后,所过之处,树木草丛都被从中分开,向两边倒去。蛇人久久地站在那里望着,直到看不见了才离开。从此以后,行人经过那一带像先前一样平安无事,不知那两条蛇到哪里去了。


\subsection{1.1.18   斫 蟒}
\label{\detokenize{p00_u5176_u5b83/_u767d_u8bdd_u804a_u658b_u5fd7_u5f02:id20}}
胡田村有家姓胡的,兄弟二人到山上砍柴,无意中走到深山峡谷中。突然遇到一条大蟒,长兄走在前边,被大蟒咬住。弟弟在后面见了,最初惊吓得想逃跑,见到哥哥被蟒咬住向下吞,就奋不顾身地抽出砍柴的斧头,向大蟒的头砍去。大蟒虽然受了伤,但仍然咬住不放。长兄的头虽说被吞进去,幸而肩膀吞不下去。弟弟在紧急中,没有别的办法可施,就用两只手攥住兄的两只脚,用力与蟒争夺,竟然把兄从蟒的口中拖了出来。大蟒也因受伤负痛走了。细细一看长兄,鼻子耳朵都已经化掉,气息奄奄,很是危险。他用肩扛起长兄往回走,一路上歇息了十几次,才背回家。请医生给医治,在家养了半年才好。到现在,满脸上全是瘢痕,长鼻子耳朵的地方,只有窟窿了。哎,在农人中,竟有这样的弟弟!有的说:“大蟒没有杀死他的长兄,那是被他弟弟的德行与义气所感化。”的确是这样!


\subsection{1.1.19   犬 奸}
\label{\detokenize{p00_u5176_u5b83/_u767d_u8bdd_u804a_u658b_u5fd7_u5f02:id21}}
青州有一个商人,经商在外,经常一年都不回家一次。家里养着一只白狗,他的妻子就引着它与自己性交,狗便习以为常了。
一天,丈夫回来,与妻子同睡一床。白狗突然进屋窜上床,竟把商人咬死了。

后来,邻居们稍稍听到一点这事的经过,都抱不平,于是告了官。官府拷打这妇人,妇人就是不招供,便将她押进了监牢。接着官府又命衙役把狗牵来,狗来了又把妇人叫出来。狗见了妇人,径直跑到妇人身前撕碎衣服做出性交的姿势。这时,妇人才没有话可说了。

官府差两个衙役押着妇人和狗上解部院,一个押解妇人,一个押着狗。一路上有愿看人、狗性交的,就敛钱贿赂差役,差役便叫狗与那妇人交配。所到处,看的人常有几百之多,差役因此也大发其财。后来,妇人和狗都判了刑,被一寸一寸地割死了。

唉!天地之大,真是无奇不有。但是长着人样却与狗相交的,又岂止这妇人一个呢?


\subsection{1.1.20   雹 神}
\label{\detokenize{p00_u5176_u5b83/_u767d_u8bdd_u804a_u658b_u5fd7_u5f02:id22}}
王筠苍公,到楚中上任做官。一到任,他就登龙虎山拜谒张天师。到了湖边,王公刚上船,就见一人驾一叶小舟而来。来人到了王公船前,就叫船上的人通报王公。王公出来接见,见此人相貌高大魁伟,很是不凡。那人见了王公,马上从怀中拿出张天师的帖子呈上,说:“天师知道大人带着护从来了,特派我来迎接带路。”王公惊讶天师早有知晓,心中越发崇敬,因此,更加虔诚地前去拜谒。

到了天师处,天师摆下宴席招待。在一边侍奉的人,穿的衣服,长的相貌,都不像平常人一样。迎接王公的那位官员,也站在一边侍卫。一会儿,他走到天师耳边小声说了几句话。天师便对王公说:“此人是先生的老乡,不认得吗?”王公表示不认得,问是谁,天师说:“他就是世上传说的雹神李左车将军。”王公非常愕然,马上另眼相看。天师说: “他刚才说奉旨要去降雹子,特来告辞。”王公问:“到哪里去?”天师说:“章丘。”王公因为章丘是淄川的近邻,忙离席下拜请求不要去降冰雹。天师说:“这是上帝的玉旨,降多少雹子都是有数的,哪能徇私情?”王公哀求不止。天师低头想了半天,就对雹神嘱咐说:“你可多把雹子下到山谷里,不要伤害庄稼就是了。”又说:“这里贵客在座,走的时候注意斯文一点,不要惊动人。”

雹神就走到院中,忽然脚下生烟,云雾绕地,过了一刻多钟后,他便极力飞腾,开始和树一样高;再一飞腾,就与楼阁一样高,最后霹雳一声,向北飞去。房屋震动,桌上的餐具也摇摇晃晃。王公害怕地说:“他这是去打雷吗?”天师对王公说:“这还是刚才我告诫了他,行动还算缓慢。不然的话,平地一声就去了。”

王公拜别天师回到官署后,记录下这事的时间。后来派人去章丘查询,果然这一天下了雹子,下得沟满壕平,可庄稼地里只下了几粒。


\subsection{1.1.21   狐 嫁 女}
\label{\detokenize{p00_u5176_u5b83/_u767d_u8bdd_u804a_u658b_u5fd7_u5f02:id23}}
山东历城的殷尚书,年轻时家里很贫寒,但是他却很有胆量才略。县里有个世族大家的宅院,方圆几十亩地,楼房相连成片。因为经常出现怪异现象,所以被废弃,无人再住。时间长了,里面渐渐长满了蓬蒿,即使是大白天也没人敢进去了。

正巧殷公和同窗学友们一起饮酒,其中有人开玩笑说:“有能在这个院子里睡上一宿的,咱们大家共同出钱请客。”殷公一跃而起,说道:“这有什么难的!”便带上一张席子去了。众人把他送到那家大门口,戏弄地说:“我们暂时在这里等着你,如果见到妖怪,就赶紧喊叫。”殷公笑着说:“若有鬼狐的话,我一定捉住它作个证明。”说完就进了门。

走进院子,见长长的莎草掩没了路径,艾蒿如麻一样多。这时正是月初,幸好有昏黄的月光,门户还能辨认出来。殷公摸索着过了几重院落,这才到了后楼。登上月台,见上面光洁可爱,就停住了脚步。看了看西边的月亮,已落到山后,只剩下一线余辉。坐了很久,见没出现什么怪事,便暗笑传言的荒谬。就地枕着块石头,仰面躺着观赏起天上的牛郎织女星来。

一更将尽的时候,殷公迷迷糊糊想睡。忽然听见楼下有脚步声,纷纷从下面上来。他便假装睡着,斜眼看去,见一个穿青衣的人,挑着一盏莲花灯上来。突然发现了殷公,她大吃一惊往后退却,对后边的人说道:“有生人在上边。”下面的人问:“是谁呀?”青衣人回答说:“不认识。”顷刻间一个老翁上来,对着殷公仔细看了看,说:“这是殷尚书,他已经睡熟了。只管办我们的事,殷相公不拘俗礼,或许不会责怪。”于是便领着人相继上了楼,把楼上的门都打开了。过了一会儿,进出往来的人更多了。楼上灯火辉煌,就像白天一样。殷公略微翻了翻身,打了个喷嚏。老翁听见他醒了,于是出来,跪下说道:“小人有个女儿,今夜出嫁。没想到触犯贵人,万望不要怪罪。”殷公起身,拉起老翁说:“不知今夜贵府有大喜事,很惭愧没有贺礼奉上。”老翁说:“贵人光临,压除凶神恶煞,就很有幸了。麻烦您陪坐一会儿,小人全家倍加光荣。”殷公很高兴,便答应了。

殷公进楼一看,里面摆设得很华丽。这时就有个妇人出来拜见,年纪约有四十多岁。老翁说:“这是我的妻子。”殷公向她拱手还礼。顷刻间听到笙管鼓乐震耳齐鸣,有人跑上来说:“来了!”老翁急忙出门去迎接,殷公也站起来等候。不一会儿,有好多纱灯引导着新郎进来了。新郎大约有十七八岁,相貌俊雅。老翁让他先给殷公行了礼。新郎两眼看着殷公。殷公就像婚礼主持人一样,还了半主礼。紧接着翁婿互拜,拜完后,就入席。一会儿,年轻的丫鬟侍女们一个接着一个,送来热气蒸腾的佳肴美酒,玉碗金杯,映照得桌子发亮。酒过数巡,老翁叫侍女去请小姐来。侍女应声而去。过了很久没见出来。老翁起身,自己掀开帏幔去催促。

过了片刻,几个丫鬟仆妇,簇拥着新娘子出来,环佩叮当作响,兰麝熏香四散。老翁叫女儿向上面行礼。起来后,她就坐到了母亲的旁边。殷公稍微看了一眼,只见她髻插翡翠凤钗,戴着明珠耳坠,容貌艳丽,绝世无双。

尔后改用金爵斟酒,金爵很大,能盛数斗。殷公自思这东西可以拿给同学作证,就偷偷地放进衣袖中。他假装酒醉趴在桌子上,像是睡着了。席上的人都说:“殷相公醉了。”不多时,听新郎说要走。笙管鼓乐猛然间响了起来,人们纷纷离席下楼走了。随后主人收拾酒具,发现少了一只金爵,怎么找也找不到。有人暗中议论金爵可能在醉卧的殷公手里。老翁听说急忙告诫人们不要乱讲,惟恐殷公听见。过了一阵,内外都没了动静,殷公才起来。四周围暗无灯光,只有脂粉的芳香和浓郁的酒气,充满整个屋内。见东方已经发白,殷公便慢慢地下了楼。伸手摸了摸袖中,金爵仍然还在里面。

殷公到了大门口,学友们先在那里等候了,都怀疑他是夜里出来早晨又进去的。殷公拿出金爵让大家看。众人惊讶地询问来历,殷公就把夜里的情形说了一遍。大家都认为这样贵重的东西不是贫寒的读书人所能有的,于是就相信了他的话。

后来殷公考中了进士,被派到河北广平府肥丘县当县令。当地的官宦世家朱某宴请殷公,叫家人去拿大酒杯,过了很久没拿来。有个小僮捂着嘴小声和主人说了些什么话,主人脸上有了怒色。不一会儿捧来金爵劝殷公喝酒。殷公仔细看去,金爵的样式和上面雕刻的图象,与狐狸的金爵毫无区别,大为惊奇,便问是什么地方制造的。朱某回答说:“这样的金爵家里共有八只,是先父当京官时找精巧的匠工监制的。这是家传的贵重物品,层层包裹珍藏已经很久了。因为县尊大人光临,刚才从竹箱里取出来,竟然仅存七只,怀疑是家人偷了去,但包裹上十年来的尘土厚积着,依然是原样没动过,实在没法解释。”殷公笑着说:“你那只金爵成仙飞升了。然而世传的珍宝不可丢失,我也有一只,和您的金爵非常近似,一定奉赠给您。”

散了席殷公回到官署,找出金爵差人速送朱家。朱某拿着反复查看后,大为惊异。他亲到官署感谢殷公,并问金爵的来历。殷公于是叙述了事情的始末。这才知道千里以外的物品,狐狸也能摄取到手,但是却不敢最终留在自己的手里。


\subsection{1.1.22   娇 娜}
\label{\detokenize{p00_u5176_u5b83/_u767d_u8bdd_u804a_u658b_u5fd7_u5f02:id24}}
书生孔雪笠,是孔圣人的后裔,为人宽厚有涵养,善于作诗。他有位挚友在浙江天台当县令,来信请他去。孔生应邀前往,而县令恰恰去世了。他飘泊无依,穷困潦倒,回不了家,只好寄居在菩陀寺,被寺僧雇佣,抄录经文。

菩陀寺西面百步开外,有单先生家的宅院。单先生是世家子弟,因为打了一场大官司,家境败落,人口也少了,便迁移到乡下居住,这座宅子于是空闲起来。有一天,大雪纷飞,道上静悄悄的没有行人。孔生偶然经过单家门口,看见一个少年从里面出来,容貌美好,仪态风雅。少年看到孔生,便过来向他行礼,略致问候以后,就邀请他进家说话。孔生很喜欢他,非常高兴地跟他进了门。见房屋虽然不太宽敞,但是处处悬着锦缎帏幔,墙壁上挂着许多古人的字画。案头上有一册书,封面题名《瑯嬛琐记》。他翻阅了一下,内容都是过去从未见过的。

孔生见少年住在这座宅院,以为他是单家的主人,也就不再问他的姓氏家族了。少年详细地询问了孔生的经历,很同情他,劝他设馆教书。孔生叹息道:“我这流落在外的人,谁能推荐我呢?”少年说:“如果不嫌弃我拙劣,我愿意拜您为师。”孔生大喜,不敢当少年的老师,请他以朋友相待。便问少年说:“您家里为什么老关着大门?”少年回答道:“这是单家的宅子,以前因为单公子回乡居住,所以空闲了很久。我姓皇甫,祖先住在陕西。因为家宅被野火烧了,暂且借居安顿在这里。”孔生这才知道少年不是单家的主人。当晚,两人谈笑风生,非常高兴,少年就留下孔生和他同床睡了。

第二天一大早,就有个小书僮进屋来生着了炭火。少年先起床进了内宅,孔生还围着被子在床上坐着。书僮进来说:“太公来了。”孔生大惊,急忙起床。一位白发老人进来,向孔生殷切地感谢说:“先生不嫌弃我那愚顽小子,愿意教他念书。他才初学读书习字,请不要因为朋友的关系,而按同辈看待他。”说完后,送上一套锦缎衣服,一顶貂皮帽子,鞋和袜子各一双。老人看孔生梳洗完了,于是吩咐上酒上菜。房内摆设的桌椅和人们穿着的衣裙光彩耀眼,不知道是什么东西做成的。酒过数巡,老人起身告辞,提上拐杖走了。

吃完了饭,皇甫公子送上所学的功课,都是些古文诗词,并无当时的八股文。孔生问他是何缘故,公子笑着回答说:“我不是为了求取功名。”到了傍晚,公子又摆上酒菜说道:“今夜尽情欢饮,明天便不允许这样了。”又喊书僮说:“看看太公睡了没有?如果睡了,可悄悄把香奴叫来。”书僮去不久,先用绣囊把琵琶带了回来。过了片刻,一个侍女进来,身穿红装,艳丽无比。公子让她弹奏《湘妃》曲,香奴用象牙拨子勾动琴弦,旋律激扬哀烈,节拍不像以前所听到的。又让她用大杯斟酒,二人一直喝到三更天才罢。

第二天,两人早起一同读书。公子非常聪慧,过目成诵。两三个月后,下笔成文,令人惊叹叫绝。他们约好每五天饮酒一次,每次饮酒必定叫香奴来陪。一天晚上,喝到半醉的时候,孔生的两只眼睛紧紧地盯住了香奴。公子已经明白了他的心意,说:“这个侍女是老父亲抚养的。您离家既远又无妻室,我替您日夜筹划已经很久了,想为您找一位美貌的妻子。”孔生说:“假若真要帮我的忙,必须找一个像香奴这样的。”公子笑着说:“您真正成了‘少见而多怪’的人了,要是认为香奴漂亮的话,那您的心愿也太容易满足了。”

过了半年多,孔生想到郊野去游玩,到了大门口,见两扇门板外边上着锁,便问公子是什么原因,公子说:“家父恐怕结交一些朋友扰乱心绪,所以闭门谢客。”孔生听说后也就安下心来。

当时正值盛夏湿热季节,他们便把书房移到园亭中。孔生的胸膛上突然肿起一个像桃样的疮疖,过了一夜竟然长得像碗一样大了,他疼痛难忍,呻吟不止。公子朝夕探望,连吃饭睡觉都顾不上。又过了几天,孔生痛得更加厉害,渐渐不能吃喝了。太公也来探望,父子相对叹息。公子说:“我前天夜里考虑,先生的病情,只有娇娜妹妹能冶疗。已派人到外祖母家去叫她了,怎么这么久还没到来?”话刚说完,书僮进来说道:“娜姑到了,姨婆和松姑也一同来了。”父子俩急忙进了内宅。一霎时,公子领着妹妹娇娜来看孔生。娇娜年约十三四岁,美艳聪慧,窈窕多姿。孔生一见到她的美貌,顿时忘记了呻吟,精神也为之一爽。公子便对妹妹说:“这是我的好朋友,我们不亚于同胞兄弟,妹妹要好好为他医治。”娇娜于是收起自己的羞容,垂着长袖,靠在床上为孔生诊断病情。手把手之间,孔生闻到娇娜身上散发着的芳香胜于兰花。娇娜笑着说:“应该得这种病,心脉都动了。病情虽然危急,但是还可医治;只是皮肤疮块已经凝结,非割皮削肉不可。”说完就脱下手臂上的金镯安放到孔生的患处,慢慢压了下去。疮疖突起一寸多,高出金镯以外,而疮根的红肿部位,都被收在镯内,不像以前如碗那样大了。娇娜又用另一只手掀起衣襟,解下佩刀,刀刃比纸还薄。她一手按镯一手握刀,轻轻沿着疮根割去。紫血顺着刀流出来,沾染了床席。孔生贪恋娇娜的美姿,不仅不觉得疼痛,反而还怕早早割完,没法再和她多偎傍一会儿。不多时,把疮上的烂肉都割了下来,圆团团的就像树上削下来的瘤子。娇娜又叫拿水来,把割开的伤口洗净。然后从嘴里吐出一粒红丸,像弹丸一样大小,放到割去了疮疖的肉上,用手按着它旋转。才转了一圈,孔生就觉得热火蒸腾;再一圈,便觉得习习发痒;转完三圈,已是浑身清凉,透入骨髓。娇娜收起红丸放回嘴里,说:“治好了!”说完便快步走了。孔生一跃起身追出门外感谢,觉得长时间的病痛像是一下子全没了。而心里却挂念苦想着娇娜的美貌,再也无法控制自己。

从此孔生闭卷呆坐,百无聊赖。公子已经看出他的心事,说:“我为您物色了很久,终于选得一位好姑娘。”孔生问:“是谁呀?”公子回答说:“也是我的亲属。”孔生苦想了好长时间,只是说:“不必要了。”然后面对墙壁吟诵元稹的诗句道:“曾经沧海难为水,除却巫山不是云。”公子领会了他的意思。说:“家父仰慕您的大才,常想联为婚姻。只是我仅有一个小妹娇娜,年龄又太小。我还有个姨表姐阿松,已十八岁了,长相不俗。如果不信的话,松表姐天天都来游园亭,您等候在前厢房,可以望见她。”孔生便按公子说的到了那里,果然见娇娜和一个美人一起来了。这女子画眉弯如蚕蛾的触须,纤瘦的小脚穿着凤头绣鞋,与娇娜难分上下。孔生大喜,便求公子作媒。

第二天公子从内宅出来,向孔生祝贺说:“事情办好了。”于是清扫另一个院子,为孔生举行婚礼。这天夜里,鼓乐齐鸣,热闹异常。孔生觉得好似月亮中的仙女忽然来和他同衾而卧,竟然怀疑广寒宫殿即在眼前。未必在云霄之上了。结婚之后,孔生心里非常满足。

一天夜里,公子对孔生说:“您对我增长学问的指点我永远不会忘怀。只是最近单公子解除官司回来,索要宅子很急。我家想要离开此地西去。看样子已很难再相聚,因而离情别绪搅得心里非常难受。”孔生愿意跟随他家西行。公子劝他还是回山东故乡,孔生感到很为难。公子说:“不用忧虑,可立即送您走。”

不多时,太公领着松娘来到,拿出一百两黄金赠送给孔生。公子伸出两手紧握着孔生夫妇的手,叮嘱二人闭上眼睛不要看。他们飘然腾空,只觉得耳边的风声呼呼地响。过了很久,公子说: “到了。”孔生睁开眼,见果然回到了家乡。这才知道公子并非人类。他高兴地叫开家门。母亲出乎意料,又看到漂亮的儿媳,全家都非常喜悦。等到回头一看,公子早已无影无踪了。松娘侍奉婆母很孝顺,她的美貌和贤惠的名声,传诵远近。

后来孔生考中了进士,被授予延安府司理官职,携带着家眷上任了。他的母亲因为路远没一同去。松娘生了个男孩,取名叫小宦。孔生后来因冒犯了御史行台而被罢官,受阻回不了家乡。有一次他偶然到郊外打猎,碰见了一位美貌少年,骑着匹黑马驹,频频回头看他,孔生仔细看了看,原来是皇甫公子。急忙收缰勒马,两人相认,悲喜交加。公子邀请孔生跟他一起回家去。他们走到一村,树木茂密,浓荫蔽日。进了公子家,见门上饰有金色的泡钉,仿佛世族大家。孔生问娇娜妹子的近况,知道她已经出嫁了;又知岳母也已去世,非常感慨伤心。他住了一宿回去,又和妻子一同返回来。这时,正好娇娜也来了,她抱过孔生的儿子上下抛逗着玩,说:“姐姐乱了我家的种了。”孔生拜谢她先前的恩德,娇娜笑道:“姐夫显贵了,疮口已经好了,没忘记疼吧?”她的丈夫吴郎,也来拜见。在这里住了两夜才离去。

一天,皇甫公子忽带忧愁的神色,对孔生说道:“天降灾祸,您能相救吗?”孔生虽然不知将要发生什么事,但却立即表示自己甘愿承当。公子急忙出去,招呼全家人来到,排列在堂上向孔生礼拜。孔生大为惊异,急问缘故。公子说:“我们不是人类,而是狐狸。今有雷霆劫难,您愿意以身抵挡,我们就都能生存;不然的话,请您抱着孩子走吧,免得让您受牵累。”孔生发誓与公子全家共存亡。于是公子让孔生手执利剑站立在门口,叮嘱他说:“霹雳轰击,也不要动!”孔生按公子说的去办。果然见阴云密布,白昼如夜,昏天黑地。回头一看住过的地方,宽大的房舍没有了,只有一座高大的坟冢,有个深不见底的大洞穴。正在惊异不定的时候,霹雳一声巨响,震撼山岳;狂风暴雨骤起,把老树都连根拔出。孔生虽然感到耳聋眼花,却依然屹立在那里一动不动。在浓烟黑雾之中,忽见有个鬼样的怪物,尖嘴长爪,从深洞中抓出一个人来,随着烟雾上升。孔生瞥了一眼那人的衣裳鞋子,觉得很像娇娜。急忙一跃而起,用利剑向怪物剌去,随手堕落一物。突然又一个炸雷爆裂,孔生被震倒在地,竟然昏死过去。

过了一会儿,天晴云散,娇娜自己慢慢苏醒过来。当她看到孔生死在身旁,便大哭着说道:“孔郎为我而死,我为什么还活着!”松娘也从洞内出来,一起把孔生抬了回去。娇娜让松娘捧着孔生的头,让公子用金簪拨开孔生的牙齿;她自己两手撮着孔生的腮,用舌头把口里的红丸送到他的嘴里,又口对口地往里吹气。红丸随着气进入孔生的喉咙,发出格格的响声。不一会儿,孔生竟苏醒过来。见亲属们都在面前,仿佛如梦中醒来。于是一家团圆,不再惊慌,万分喜悦。

孔生认为墓穴不可久住,提议让大家和他一同回自己的故乡。满屋的人都交口称赞,只有娇娜不高兴。孔生请她与吴郎一起去,娇娜又怕公婆不肯离开幼子,一整天也没商量出结果。忽然见吴家的一个小仆人,汗流满面气喘吁吁地来到。大家惊慌地再三追问他,才知道吴郎家也在同一天遭难,全家都死了。娇娜听说,顿足悲伤,啼哭不止。大家一起慰劝她。直到这时,大家一同随孔生回归故乡的计划才算定下来。孔生进城料理了几天,回来就连夜催促整理行装。

孔生回到家乡后,把自己的一处闲弃的园子给皇甫公子一家住,平常反锁着园门;只有孔生和松娘来到,才开门。孔生与公子、娇娜兄妹在一起,下棋、饮酒、谈天、聚会,亲密得就像一家人。孔生的儿子小宦长大了,容貌美好,有狐狸的神情。他到城里去游玩,人们都知道他是狐狸生的儿子。


\subsection{1.1.23   僧 孽}
\label{\detokenize{p00_u5176_u5b83/_u767d_u8bdd_u804a_u658b_u5fd7_u5f02:id25}}
有一个姓张的人,突然死了,跟着鬼使去见阎王。阎王拿生死簿一查,训斥鬼使捉错了人,命令将他送回去。姓张的下了阎王殿,私下托请鬼使,请求他带自己在阴曹地府参观参观。鬼使领他游遍了九层地狱,刀山、剑树都一一指给他看。最后到了一处,见有一个僧人被绳子穿过大腿倒挂在那里,痛得直喊要死。走近一看,竟是他哥哥。姓张的见了很是害怕,问鬼使:“犯了什么罪能到这个地步?”鬼使说:“这个和尚,到处募捐钱财,供他嫖赌,因此罚他。要想摆脱此罪,必须改过自新。”

姓张的苏醒过来后,怀疑他哥哥已死,便去他哥哥当和尚的兴福寺里打听。进门,便听到有人喊痛的声音。进屋一看,见哥哥腚上生疮,脓血渍流,身子倒挂在墙上,就像在阴曹看到的一样。他惊问这是怎么回事,哥哥说:“挂着还可以忍受,不然就痛彻心肺。”姓张的告诉哥哥他在阴曹所见的一切,他哥哥当真才害怕。从此,他戒酒、戒赌、戒嫖,虔诚地诵读经文。过了半月,身体才好了。此后,他就成了一个戒僧。


\subsection{1.1.24   妖 术}
\label{\detokenize{p00_u5176_u5b83/_u767d_u8bdd_u804a_u658b_u5fd7_u5f02:id26}}
有位于公,年轻时行侠仗义,喜欢练拳比武,力气大得能把高脚的漏壶举起,旋风般地舞动。

明朝崇祯年间,他在京都参加殿试,因仆人得病卧床不起而十分忧虑。正好集市上有个精于算卦的人,能够算出人的生死命运。他准备替仆人去问一问病的吉凶。

于公来到算卦人的跟前,还没有开口,算卦的就说:“你是不是想问仆人的病呀?”于公吃惊地点头称是。算卦的又说:“病人没事,而你却很危险。”于公便请他给自己算一卦。算卦的卜完卦后惊愕地说:“你三天之内就会死。”于公听了惊诧半天。算卦的从容地说:“我有小小的法术,送我十两银子,就可以替你消灾。”于公自己思忖,生死已经注定,小小法术怎么能解除?他没有答应,起身要走。算卦的说:“吝惜这点钱,不要后悔,不要后悔!”爱护于公的人都为他担心,劝他把所有的钱都拿出来,哀求算卦的人为他消灾,于公不听。

转眼到了第三天,于公端端正正地坐在旅店里,静静地观察动静,但一整天都没什么意外。到了夜晚,于公关上门挑亮了灯,靠着宝剑端坐在室中。一更将过,根本没有死的征兆,就想躺下睡觉。忽然听到窗缝里有窸窸索索的响声,急忙一看,有一个小人肩上扛着矛戈进来,刚落地,就变得和平常人一样高。于公拔剑而起,急向小人砍去,但飘忽未能击中。小人急剧变小,又去找窗缝,想要逃跑。于公飞快地砍去,那小人应手而倒。拿灯一照,是个纸人,已被拦腰砍断。于公不敢睡了,坐在那里等待。

过了一会儿,一个怪物穿窗进来,面目狰狞如鬼。刚落地,于公急忙向它击去,砍为两截,都在地上蠕动。恐怕它再起来,又连连击去,剑剑都中。发出的声音,不像是软的肉体,仔细一看,是个泥偶,一片片碎落在地上。

于是于公就移坐到窗下,眼睛注视着窗缝。过了很长时间,听到窗外有像牛喘一样的声音,有个怪物来推窗棂,房间的墙壁被震摇,看上去像是要被推倒的样子。于公害怕被压倒在墙下,心里合计不如冲出去和它斗,便猛然打开门,飞奔而出。只见一个巨鬼,有房檐一样高。在昏暗的月光中,面孔黑得像煤炭,眼睛里闪烁着黄光,上身没穿衣服,脚下没穿鞋子,手持一张弓,腰里插着箭。于公正在惊愕间,鬼已经弯弓射来一箭,于公急忙用剑拨开,箭落到地上。刚要奔过去,鬼又射来一箭,于公急忙跳跃躲开,箭穿透墙壁,咔咔作响。鬼非常恼怒,又拔出佩刀,挥舞如风,向于公猛力劈来。于公像猴子似地纵身往前一跃,刀砍在院中的石头上,石头立刻断裂。于公乘机钻到鬼的两腿间,挥剑砍削鬼的脚脖子,发出铿然之声。鬼更加愤怒,吼声如雷,转身再剁。于公又伏身向前一钻,鬼的刀落下来,砍下一截他的裙袍。而于公已到了鬼的肋下,挥剑猛砍,也是铿然作响,鬼仆倒在地不动了。于公又挥剑乱砍,声音脆裂像砍木头一样。用灯一照,原来是个木偶,高大如同平常人一样。弓箭还缠在腰间,脸谱刻画得狰狞可怖,凡是被剑砍的地方,都有血流出。于公怕再来鬼物,便手持烛灯坐等天明。这才悟出鬼物都是那个算卦的人派来的,想把人吓死,以证明他的法术神灵。

第二天,于公遍告所有的朋友,约好了一起去算卦人的住所。算卦的人老远看见于公,转眼间就不见了。有人说:“这是隐形术,用狗血可破。”于公按那人说的准备好了再次前往。算卦人又像上次那样隐匿起来。于公急忙用狗血浇他站的地方,只见算卦人头上脸上狗血模糊,目光一闪一闪的像个鬼一样站在那里。于是就把它押送到衙门处死了。


\subsection{1.1.25   野 狗}
\label{\detokenize{p00_u5176_u5b83/_u767d_u8bdd_u804a_u658b_u5fd7_u5f02:id27}}
于七之乱,杀人很多。乡下人李化龙,从山中逃回来,正碰上晚上过大兵。为以免被大兵杀害,他急切间无处藏身,便僵卧到死人堆里佯装死人。大兵过完后,李化龙还没敢爬起来,睁眼一看,忽然见掉了头断了胳膊的尸体,都站了起来,像小树林一样。其中一具尸体,已经断了的头仍连在肩膀上,嘴里说道:“野狗子来了,怎么办?”其它尸体也一起乱嘈嘈地说:“怎么办?”一霎时,都扑哧扑哧倒下了,随即一点声音也没了。

李化龙战战兢兢地才想爬起来,就见一个兽头人身的怪物,正趴在死尸堆里吃人头,挨个吸人的脑子。他害怕被吃,便把头藏在尸体底下。怪物来拨弄他的肩膀,想吃他的头,李就用力趴在地上。怪物几次都没能得到他的头,就推去盖在李头上的尸体,使他的头露了出来。李害怕万分,慢慢用手摸索腰下,摸到一块石头,有碗那样大,握在手里。怪物找到了李的头趴下就想啃。李突然跳起,大喊一声,用石头猛击怪物的头,结果打中了它的嘴。怪物像猫头鹰那样大叫了一声,捂着嘴负痛跑了。它路上吐了一些血,李化龙就地查看,在血里找到了两颗牙齿,中间弯曲,末端锐利,长四寸多。拿回村给别人看,谁都不知道那是什么怪物。


\subsection{1.1.26   三 生}
\label{\detokenize{p00_u5176_u5b83/_u767d_u8bdd_u804a_u658b_u5fd7_u5f02:id28}}
刘孝廉这个人,能记得前生的经历。与我过世的文贲兄是同榜考中的举人。他曾把前世的经历清清楚楚地说出来。

他说他前一世为绅士,行为不端,六十二岁那年就死了。死后初见阎王,阎王很客气,拿他当乡下有声望的人对待,先是赐坐,后是倒茶。他坐下后,看到阎王茶杯里的茶,色清透明;而自己杯里的荼,却浑得像浊酒。他心里暗想:莫非这便是迷魂汤?他没喝,趁阎王不注意时,把茶倒在了桌子底下,假装已经喝光了。

待了一会儿,阎王查知刘的生前恶行,大怒,命令群鬼将他拉下去,罚脱生为马。接着有个厉鬼牵着他就走。走到一家人家,大门坎太高,不好迈过。他在犹豫徘徊时,鬼用力打了他一下,痛得他跌倒在地。自己看了一下自己,已身在马槽下边了,耳听有人说话:“大黑马生小马驹了,是公的。”他心里十分明白,但不能说话。一时肚里觉得很饥饿,不得已去母马肚下吃奶。

过了四五年,小马长得高大健壮。但很怕挨打,见鞭子就跑。主人骑它时,厚厚地垫好鞍子,慢慢走,还不算苦。惟有奴仆们和喂马的人骑它时,都是不加鞍垫,两腿一夹就叫它跑,真是痛彻肺腑。它很气愤,绝食三天就死了。

又回到阴间,阎王查他的罚期还没有满,责备他逃避惩罚,就又命令小鬼剥去他的皮,罚它托生为狗。他觉得非常懊悔,不愿去托生。众多小鬼就乱打他。它痛极了,跑到了野外,自己想:还不如死了好,气忿忿地一头投下悬崖,跌得爬也爬不起来。自己一看,原来已在狗洞里了,母狗正在抚爱地用舌头舐它,才知道自己又托生为狗了。

托生成狗后,稍稍长大了点,见了屎和尿,也知道脏,但用鼻子一闻,却觉得很香,但是下决心不吃它。当了一年狗,常常忿恨得想死,又怕罚期不到再罪加一等。而主人又喂养着不杀他,没有别的办法,就故意咬主人,使主人皮破露骨。主人大怒,就把狗杀了。

他再次回到阴间,阎王审问后,嫌他太疯狂,命令小鬼打他数百棍,罚他托生为蛇。把它囚禁在黑屋子里,成天不见天日。它感到闷得慌,便顺着墙向上爬,打了个洞钻出屋来。自己一看已身在草丛里了,变成了一条蛇。从此,下决心不残害生灵,饿了就吃果实。

当了一年多蛇,它每每心想:自尽不可以,害人致死也不可以,怎么能求得一个好死的良策呢?一直没有想出个好办法来。一天,他正趴在草丛里,听见有车子路过身边,它猛地爬出来挡住车的路,结果车轮压过,把它的身子压为两截,蛇死了。

他又一次回到阴间,阎王很惊讶,奇怪它怎么这么快就回来了。他匍匐在地作了表白。阎王听了,认为这是无罪被杀,宽恕了他,准它服罪期满复生为人,这就是现在的刘孝廉。

刘孝廉一生下来就会说话,文章书籍一看就能背诵,辛酉年中了举人。他常劝人:骑马必须把鞍下垫得厚厚的,骑光腚马,马被两腿一夹,比鞭子抽打还疼呢。


\subsection{1.1.27   狐 入 瓶}
\label{\detokenize{p00_u5176_u5b83/_u767d_u8bdd_u804a_u658b_u5fd7_u5f02:id29}}
万村石家的媳妇,被狐狸精缠上,一家人很但担忧,却打发不走它。妇人门后有个瓶,每次听见妇人的公公回来,狐精就藏入瓶内。妇人多次看在眼里,便记在心里,也不吭气。

一次,狐又钻入瓶内,妇人急忙用棉絮塞住瓶口,把瓶放到锅里煮。瓶热后狐狸在瓶内喊:“太热了,别胡闹!”妇人不答话,继续煮。狐精在瓶里喊得更急,时间一长就听不到动静了。妇人拔开塞子看时,仅有一堆毛和几滴血而已。


\subsection{1.1.28   鬼 哭}
\label{\detokenize{p00_u5176_u5b83/_u767d_u8bdd_u804a_u658b_u5fd7_u5f02:id30}}
谢迁造反时,官宦人家的宅第都被贼占据着,成了贼窝子。有个叫王七襄的学使,家里住的贼尤其多。官兵破城后,扫荡群贼,死尸都填满了台阶,血顺门而流。

王学使进了城,回到家里,命人把盗贼的尸首抬出去,把血迹洗刷干净,这才住下。但是大白天就往往见到鬼,夜晚床下磷火乱飞,墙角还时常有鬼哭,很不安宁。

一天,有个叫王皡迪的书生,借住在王公家。夜里听到床下有小声连连叫:“皡迪!皡迪!”过了一会儿,声音渐大,并说:“我死得好苦呀!”随后就哭起来,接着满院子里都有哭声。王公听见后,手持宝剑到王生屋里,大声说:“你们不知道我是王学院吗?”只听见众鬼嗤嗤冷笑。

王公不得已,于是设了水陸道场,命和尚、道士念经超度,夜里做了饭抛到院子里让群鬼吃。这时就见院子里磷火点点,到处都是。

先前一个为王公看大门的姓王的人,病得很厉害,已经昏迷几天不知人事了。闹鬼的这天,他忽然伸了伸身子,像是醒过来了。他老婆见这情形就给他端来饭,他却说:“刚才主人不知为什么在院子里施饭,我也跟大伙一块吃,这不才吃饱了回来,所以不觉得饿。”

自此以后,鬼都绝迹了。难道道士奏乐,和尚超度,施舍饭食,果然灵验吗?


\subsection{1.1.29   真 定 女}
\label{\detokenize{p00_u5176_u5b83/_u767d_u8bdd_u804a_u658b_u5fd7_u5f02:id31}}
真定界内,有一个孤女,年纪方六七岁,就当了童养媳。一两年后,丈夫引诱她同了房,此后就怀孕了,肚子渐渐胀大。自己以为得了病,便告诉婆母。婆母问:“动不动?”回答说:“动。”婆母觉得很奇怪,但因女孩年纪太小,不敢断定。没多长时间,果然生了个男孩。婆母叹口气说:“没料想拳头大的小母亲竟生了个锥子大的小孩子!”


\subsection{1.1.30   焦 螟}
\label{\detokenize{p00_u5176_u5b83/_u767d_u8bdd_u804a_u658b_u5fd7_u5f02:id32}}
董默庵在朝中当侍读官。他家里被狐精扰乱,砖瓦石沙经常像下雹子一样从天上落下来。全家人拖老带小纷纷奔逃躲藏,等平静了才再出来干活。董公对此深感忧虑,于是借了司马孙怍庭的宅子暂住,然而狐精仍旧扰乱,和在家时一样。

一天,董公在待漏院等待上朝时,与同事们说出这件奇怪的事。有一位大臣说:“关东道士焦螟,现在内城住着,主持降妖的法术,听说很灵验。”于是董公就登门拜访焦道士请他帮助降妖。焦道士用朱笔写了一道符,叫董公回家贴到墙上。董公回家照办后,狐精一点不怕,抛掷砖石反而更加厉害了。不得已,董公只好又去告诉道士。焦道士大怒,亲自去董府,筑坛台作法术。他作法不多时,见一个大狐趴在坛下。董府家人受害很长时间了,早就恨得咬牙切齿,一个丫鬟上去就打了狐狸一下,这丫鬟却忽然倒在地上断了气。道士说:“这个东西很猖獗,我都不能一下子降服它,这女子怎敢轻易冒犯它呢?”接着又说:“正好,我可以借这女子之口向狐狸问话。”便用手指着丫鬟,口中念咒,丫鬟忽地起来跪在坛下。道士问它住哪里?丫鬟口里说出狐狸的话:“我是西域生的,来京城已十八辈子了。”道士又说:“这是朝廷住的京城,怎么能容你们这些东西长久住下去?赶快走吧!”狐狸不回答。道士大怒,拍着桌子说:“你还想违抗我的命令吗?若再迟延,道法可不容你!”狐狸这才皱起眉头有点害怕的样子,表示愿奉教命。道士又催它快走。这时丫鬟又倒下没气了,过了很长时间,才苏醒过来。接着见四五块白团滚滚如圆球,顺着屋檐滚动,一个跟着一个,一转眼的功夫就都滚走了。从此,董公家才安定无事。


\subsection{1.1.31   叶 生}
\label{\detokenize{p00_u5176_u5b83/_u767d_u8bdd_u804a_u658b_u5fd7_u5f02:id33}}
河南淮阳有个姓叶的秀才,不知道他的名字。他的文章词赋,在当时首屈一指;但是命运不济,始终未能考中举人。

恰巧关东的丁乘鹤,来担任淮阳县令。他见到叶生的文章,认为不同寻常,便召叶生来谈话,结果非常高兴,便让叶生在官府读书,并资助他学习费用;还时常拿钱粮救济他家。到了开科考试的时候,丁公在学使面前称赞叶生,使他得了科试第一名。丁公对叶生的前途寄予极大的希望。乡试考完,丁公要叶生的文稿来阅读,拍案叫好。没料想时运限人,文章虽好命不佳,发榜后,叶生仍旧名落孙山。他垂头丧气地回到家,感到辜负了丁公的期望,很惭愧,身形消瘦,呆如木偶。丁公听说,召他来劝慰了一番,叶生泪落不止。丁公很同情他,约好等自己三年任满进京,带着他一起北上。叶生非常感激。辞别丁公回家,从此闭门不出。

没过多久,叶生病倒在床上。丁公经常送东西慰问他;可是叶生服用了一百多副药,根本不见效。丁公正巧因冒犯上司被免了官职,将要离任回乡。他给叶生写了封信,大致意思说:“我东归的日期已经定了,所以迟迟不走的原因,是为了等待您。您若早晨来到,我晚上就可以上路了。”信被送到了病床上,叶生看着信哭得非常伤心,他让送信人捎话给丁公说:“我的病很重,很难立即痊愈,请先动身吧。”送信人回去如实说了。丁公不忍心就走,仍慢慢等着他。

过了几天,看门的人忽然通报说叶生来了。丁公大喜,迎上前来慰问他。叶生说:“因为小人的病,有劳先生您久等,心里怎么也不安宁。今天有幸可以跟随在您身边了。”丁公于是整理行装赶早上路。

丁公回到家,让儿子拜叶生为师,并让好好伺候,早晚都和他住在一起。丁公子名叫再昌,当时十六岁,还不能写文章。但是却特别聪慧,文章看上两三遍,就不会再忘记。过了一年,公子便能落笔成文。加上丁公的力量,于是他进了县学成为秀才,叶生把自己过去考举人的范文习作,全部抄下来教公子诵读。结果乡试出的七个题目,都在准备的习作中,无一脱漏,公子考了个第二名。

一天,丁公对叶生说: “您拿出自己学问的剩余部分,就使我的儿子成了名。然而您这贤才却被长期埋没,有什么办法呢!”叶生说:“这恐怕是命中注定的吧。不过能托您家的福为文章吐口气,让天下人知道我半生的沦落,不是因为文章低劣,我的心愿也就足了。况且读书之人能得一知己,也没什么遗憾了。何必非要穿上官服,抛掉秀才衣裳,才说是发迹走运呢!”丁公认为叶生长期客居外省,怕他耽误了参加岁试,便劝他回家。叶生听说后脸上现出了凄惨不乐的神色。丁公不忍心强让他走,就叮嘱公子到京城参加会试时,一定要为叶生稍纳个监生。

丁公子考中了进士,被授部中主政。上任时带着叶生,并送他进太学国子监读书,与他早晚在一起。过了一年,叶生参加顺天府乡试,终于考中了举人。正遇上丁公子奉派主管南河公务,他就对叶生说:“此去离您的家乡不远。先生已经功成名就,衣锦还乡该何等令人高兴。”叶生也很喜悦。他们择定吉日上路。到了淮阳县界,丁公子派仆人用马车护送叶生回了家。

叶生到家下车,看见自己的门户很萧条,心里非常难过。他慢慢地走到院子里。妻子正好拿着簸箕从屋里出来,猛然看到叶生,吓得扔了簸箕就走。叶生凄惨地说:“我现在已经中了举人了。才三四年不见,怎么竟不认识我了?”妻子站在远处对他说:“您死了已经很久了,怎么又说显贵了呢?之所以一直停放着您的棺木没有埋葬,是因为家里贫穷和儿子太小的缘故。如今儿子阿大已经成人,正要选择墓地为您安葬。请不要作怪来惊吓活人。”叶生听完这些话,显得非常伤感和懊恼。他慢慢进了屋,见自已的棺材还停放在那里,便一下扑到地上没了踪影。妻子惊恐地看了看,只见叶生的衣帽鞋袜脱落在地上。她悲痛极了,抱起地上的衣服伤心地大哭起来。儿子从学堂中回来,看见门前拴着马车。他问明赶车人的来历,吓得急忙跑去告诉母亲。母亲便流着眼泪把见到的情景告诉了儿子。娘俩又仔细询问了护送叶生的仆人,才得知事情的始末。

仆人返回,如实报告了主人。丁公子听说,泪水浸湿了胸前的衣服。他立即乘着马车哭奔到叶生的灵堂祭拜;出钱修墓办理丧事,用举人的葬礼安葬了叶生。又送了很多钱财给叶生的儿子,并为他请了老师教读。后来丁公子向学使推荐,使叶生的儿子第二年入县学成了秀才。


\subsection{1.1.32   四 十 千}
\label{\detokenize{p00_u5176_u5b83/_u767d_u8bdd_u804a_u658b_u5fd7_u5f02:id34}}
新城王大司马,家里有管家仆人,很是富有。一天,他忽然梦见一个人进来对他说:“你欠我四十千钱,现在应该还我了。”他惊讶地询问缘故,那人也不回答,径直向里屋走去。他一下子醒来,妻子正好生了一个男孩。他知道这孩子是来要前生的帐的,就拿出四十千钱单独放在一个房间。凡是孩子的一切衣食、医药费用,都从这四十千里开支。

过了三四年的功夫,看看那四十千钱只剩七百了。这天,奶娘正抱着孩子在一边玩耍,王大司马便叫过孩子来,对孩子说:“四十千快用完了,你该走了。”话刚说完,小孩的脸色就变了,接着头向后一仰就瞪了眼,摸了摸鼻子,已经没气了。于是就把剩下的钱买了治丧的物件,把小孩埋了。

这件事,欠帐的人可以引以为戒。从前曾有个老来无子的人,询问高僧这是为什么?高僧回答说:“你不欠人家的债,人家也不欠你的债,哪能得孩子?”所以说:生好孩子是来报恩的;生坏孩子,是来讨帐的。生死由命,生了孩子的不要过于欢喜,孩子死了也不要过于悲哀。


\subsection{1.1.33   成 仙}
\label{\detokenize{p00_u5176_u5b83/_u767d_u8bdd_u804a_u658b_u5fd7_u5f02:id35}}
文登一个姓周的书生,与一个姓成的书生小时候在一个书桌上读书、写字,成为知己好友。成生家中贫穷,一年到头都依靠周生接济。周生比成生大,所以成生管周生的妻子叫嫂嫂。逢年过节都去拜访,像一家人一样。

后来,周生的妻子因生孩子,产后得急病死了,周生接着又娶了个后妻王氏。成生因为新嫂嫂比自己年纪小,所以从没要求周生让自己见见她。

一天,王氏的弟弟来看望姐姐,周生便在卧室里设宴招待。正好成生来了,仆人来通报,周生坐在宴席上命人快请他进来。成生不进,告辞要走。周生便将酒席移到外间,将成生追了回来。刚刚坐下,就有人来禀告,一个庄园里的仆人被县太爷重打了。原因是黄吏部家有个放牛的,放牛时踩了周家的田,两家仆人发生争吵、谩骂。黄家放牛的回去告诉了主人,周家仆人就被捉去送官,所以挨了重打。周生听说,很气愤地骂道:“黄某这个放猪奴,怎敢这样!他前辈是我家祖上的奴才,刚得志就目中无人了!”周生气满胸膛,忿忿地起来要去找黄家。成生按住他制止说:“强梁世界,本来没有青红皂白!况且今日的官府一半是不打旗子的强盗呢!”周生不听,成生再三劝说,以至掉了泪,周生才勉强忍下。

但是,周生的怒气终不能消除,一夜翻来覆去没有睡着,对家人说:“黄家欺侮我们,是我们的仇家,这先不说,县官是朝廷的命官,并不是有势力人家的官,就是互有争端,也应传两家对质,何至于像哈叭狗一样跟着叫?我也去告他家的仆人,看县官怎么处置他们?”家人们也鼓动他,于是他就写了呈子送到县衙。可是县官只看了一眼就把呈子撕了扔在地下。周生气极了,顺口说了几句不好听的话,冒犯了县官。县官恼羞成怒,就把周生拘捕了。

这天早饭后,成生又去找周生,才知道周生去县城告状去了。他急忙追去想劝止,不料周生却已在监狱里了。急得他直跺脚,无计可施。

这时,官府正抓了三个海盗。县官与黄吏部用钱买通了海盗,让他们捏造周生是同党,然后根据假证词,革去了周生的功名,更加残酷地拷打他。成生来看他,两人抱头痛哭。他二人偷着商量还得上告。周生说:“我身在监牢,像鸟在笼子里。家里虽有一个弟弟,也只能给我送点饭来,谁能替我上告呢?”成生表示愿一人承担,说:“这是我应尽的责任,朋友有难而不能急救,还算什么朋友?”说罢就走。周生的弟弟打算送路费给他时,他已经走远了。

成生到了京城,上告无门,正急得不得了的时候,听人传说皇帝要出城打猎。成生就暗藏在木市中。待了不多时,皇帝的大队人马果然从这里经过。成生趴在地上大声喊冤,皇帝问明了原因,准了他的状,叫他等着,并把他的状子批到部院,命部院复审上奏。

此时,距周生入狱已十多个月了,周生已受刑不过,屈打成招,定了罪名。部院官员接到皇上御批,非常惊惧,打算亲自复审。黄家知道后也很害怕,就计划暗中谋害周生。首先买通看监的狱卒不给周生饭吃。周生的弟弟来送饭,也不让他们见面。成生又到部院喊冤,部院才提审。这时周生已饿得站不起来了。部院宫员见了大怒,喝令将狱卒打死。黄吏部更害怕,就拿几千两银子托人为他说情。部院官员才打了个马虎眼,免了黄吏部的罪。县官因为枉法,被判流放。

周生被放归,越发对成生感激不尽。成生经过这场官司,也厌世了。因此,就与周生商量一起隐居。然而周生因为有年轻的妻子,不忍离去,一直以言笑推托。成生见周生态度不明,虽然没再说什么,自己决心已定,准备出走。

两人分别以后,成生一连几天没有来找周生。周生就派人到成生家去打听。而成家还认为在周家呢,这才知道成生不见了。周生心里明白,急忙派人到处找,所有远近寺观、沟谷都找遍了,还是不见成生的踪影。周生只好经常送钱、送粮给成的儿子,帮助成家过日子。

又过了八九年的工夫,成生忽然自己回来了。他头戴黄冠,身穿大氅,一副仙风道骨的样子。周生见了,亲热得不得了,一把拉住成生的胳膊说:“你到哪里去了,让我们到处找?”成生笑着说:“孤云野鹤,哪有一定的地方?分别后幸亏还康健就好。”周生赶快命家人摆酒席招待,略说几句客套话以后,周生就催着成生换下道服来。成生只笑不说话。周生说:“你真傻!为什么不要老婆孩子,把他们像旧鞋一样扔掉呢?”成生笑着回答说:“不对!是别人抛弃了我,哪里是我抛弃别人呢?”周生又问成生住哪里,成生说在崂山清宫。

两人当夜就抵足睡了。正睡间,周生梦见成生光着身子压在自己胸上,压得喘不过气来。他惊讶地问这是为什么,成生也不回答。忽然就醒了,喊成生不答应,坐起来找成生,却不知哪里去了。定了定神,才发现自己是在成生睡的地方,他惊骇地自言自语:“昨晚没有喝醉,为什么糊涂到这个地步?”于是叫家人拿灯来照,家人只见成生坐在那里,周生不见了。周生本来胡子很多,此时他用手一捋,稀稀拉拉地没有几根了。拿镜子一照,周生大惊失色地说:“成生在这里,我哪里去了呢?”接着一想,才恍然大悟:原来这是成生用幻术招他去隐居。他想进卧室去找妻子,他弟弟因他已变为成生了,不让他进去。他自己也无法说明白,只好不进去。

别无它法,周生只好叫仆人备了马,主仆二人前去崂山找成生。走了好几天,才到了崂山。周生骑马走得快,仆人在后面一时没有跟上来,他就坐在树下休息。但见这里道士来去不断,内中一个道士看了他一眼,周生就顺势问他知不知道成生。道士笑着说:“听说过这个人,好像是在上清宫。”说罢就走了。周生目送那道士,见他走出一箭地之外,又与另一人说话,也不过说了几句,那人就走了过来。一看,原来是同学。那人见了周生以为是成生,吃惊地说:“几年不见了,听别人说你已在名山学道,为什么还游戏在人间呢?”周生知道他把自己当成成生了,于是就把自己的事说了一遍。那人惊讶地说:“我刚才还遇见他,以为是你呢!才走了不多时,或者没有走远。” 周生觉得很奇怪,说:“怪呀!我为什么见了自已的面目还不认得呢?”

过了一会儿,仆人追上来,他们急忙快走。可是走了半天,路上连个人影也看不见。前面的路一望无际,遥远得很,拿不定主意是走还是回去。可是转又一想,已经没有回去的可能了,只有向前走追上成生才行。但路却越发险恶难行,马也不能再骑了。周生就把马交给仆人,叫他转回去,自己沿着崎岖的山道一步步走去。

走了一段路,远远看见一个小道童坐在那里,周生便走向前去问路,并说来找什么人。道童说自已是成生的弟子,并帮周生拿着行李,领他一块走。他们一路风餐露宿,往很远很远的地方走去。

走了三天三夜,才到一个地方,但这里又不是世上传说的上清宫。当时是十月天气,可山路两边却山花烂漫,一点不像是初冬。道童进去禀报,成生很快就出来迎接,周生这才认出自已的面貌。两人手拉手进了大殿,接着就摆上酒席,饮酒谈心。但见珍奇的小鸟,飞来飞去,一点也不怕人,叫的声音像音乐一样好听,不时还到桌上叫几声,周生心里非常惊奇。然而他仍然思念尘世返乡心切,无意在这里呆下去。饮完了酒,见地上有两个蒲团,成生拉周生并坐在上面。约二更以后,万籁俱寂,周生忽然打了一个盹,觉得自己与成生换了个位置,心里很奇怪。自己随便用手摸了一下下颔,胡子已和从前一样了。

天亮了,周生回家心切,要求走,成生坚持留他多住几天。又住了三天后,成生对周生说:“请你稍闭一下眼,我送你回家。”周生刚一合眼,就听见成生叫着说:“行装都已齐备。”于是周生起来跟着就走。一路走的并不是原道,但走了不多时,就看到家乡了。成生坐在路旁等着,叫周生自己回家。周生强邀成生一块回家,成生执意不肯。周生就一个人回到了家门。他见大门关着,就叫了几声,里面没有答声。刚想跳墙,就觉自己的身子像树叶一样,轻飘飘进了院子。又跳了几道墙才到了卧房。见卧室内灯光昏暗,妻子还没有睡觉,听到屋里咕咕哝哝好像有人说话。他悄悄舔开窗纸往里一看,见妻子正与一个仆人用一个杯子喝酒,样子非常亲密。周生大怒,想立即进屋捉住他们。可又怕自己一人难以对付他们两人,就悄悄出门回去请成生来帮忙。成生慷慨答应,立即跟周生一直到了卧室。周生拿石头砸门,屋内二人吓慌了神,砸得越急门关得越紧。成生用剑拨门,一下两扇门都开了。周生跑进去捉人,那个仆人冲出门向外跑。成生在门外一剑砍去,砍下了仆人一条臂膀。周生进屋捉住妻子拷问,才知道刚娶她进门时她就与仆人私通了。周生拿过成生的剑,割下妻子的头,挑出她的肠子挂在院里的树上,才跟着成生原路返回。周生忽然一觉醒来,原来身子还在床上,惊异地说:“怪梦七长八短,真使人怕死了!”成生一旁笑着说:“是梦,兄却以为是真;而真,兄却以为是梦。”周生不明白是什么道理,就问成生。成生拿出剑来给他看,剑上的血迹仍在。周生吓得要死,暗暗疑惑成生已会幻术了。成生也知道周生的心思,就催他整理行装,送他回家去。

二人辗转走到了家门,成生对周生说:“那天夜里我倚着剑等你,不是在这里吗?我厌恶看见污浊,还在这里等你。如果过了申时不回来,我就自已回去了。”

周生到了家门,门庭冷冷清清,好像没有人住一样。又到了弟弟家里,弟弟见了他,双泪交流,对他说:“哥哥你走后,贼夜里来杀了嫂嫂,还把肠子挂在树上,真是可怕。至今官府还没有破案。”周生才大梦方醒,把一切事情告诉了弟弟,并嘱咐他不要再追究了。他弟弟吓呆了很长时间。周生问起孩子,弟弟叫奶妈抱来。周生看了说:“这孩子是咱家的后代,请你好好照看,兄要告辞人世了。”说罢起身就走。弟弟哭着追出挽留,周生笑着走了,连头也没回。到了郊外,见了成生,二人一起上了路,远远地回过头来说:“能忍就是最大的乐事。”他弟弟追着想再说几句话,成生一举袖子,就无影无踪了。弟弟呆立多时,哭着回了家。

周生的弟弟忠厚老实,但没有能力,不会治理家务。过了几年,家里越发穷了。周生的孩子渐渐长大,没有钱请老师教学,他就亲自教侄子读书。

一天,清早到书房里,见桌子上放着一封信,封口粘得很结实,信封上写着“二弟启”。细看是他哥哥的笔迹。拆开信一看,里面什么也没有,只有一个爪甲,有二指来长,心里觉得很奇怪。他把爪甲放在砚台上,出来问家人这信是哪里送来的,家人们都不知道。回到屋里一看,砚台闪闪发光,已变成了黄金。他更加惊奇,又放在铜铁上试试,都变成了黄金。从此,他家大富起来。他拿出千金给成生的孩子。后来相传两家都有点石成金的法术。


\subsection{1.1.34   新 郎}
\label{\detokenize{p00_u5176_u5b83/_u767d_u8bdd_u804a_u658b_u5fd7_u5f02:id36}}
江南有个孝廉,名叫梅耦长,他说他同乡有个孙翁,在德州当官的时候,审问了一桩奇案。

事情是这样的:当初,有个村民为儿子娶媳妇。新媳妇过了门,庄里乡亲都来贺喜。喜酒喝到一更多天,新郎出房,看到新娘子穿着耀眼的衣服走向屋后。新郎好生怀疑,就跟在后面看是怎么回事。宅子后面有一条长长的小河,上面有一小桥可以通过。他看见新娘子过了桥一直走去,越发怀疑,就在后面喊她。新娘也不答应,只是远远招手。新郎急忙赶过去,相距也就有尺多远,但手却一直捉不到她。

走了几里路,进了一个村子。新娘站住了,对女婿说:“你家寂寞,我住不惯,请郎君暂住我家几天,咱们再一起回家看望二老。”说罢,抽出簪子敲门,门吱呀一下就开了。有个女僮出来迎接。新娘先进去,新郎不得已也跟着进去。一进门,岳父岳母部在堂上坐着,对女婿说:“我女儿从小娇惯,没有一时离开过我。一旦离开家,心里总是不痛快。今日与你一起回来,我们很放心,住几天就送你们回去。”于是就叫丫鬟扫屋子、铺被褥,两人就住下了。

新郎家中的客人,见新郎出去多时不回来,就到处找。新房里只有新娘子在等待,新郎却不知到哪里去了。大家就四处查询,一点消息也没有。公公、婆婆都哭得很伤心,说是必死无疑。

过了半年,媳妇娘家怕女儿守寡,就与新郎家父母商量,打算给女儿另找婆家。新郎父母越发悲伤,说:“尸骨衣物,都还没有找到,怎么知道我儿一定死了呢?就算死了,过一年再另嫁也不晚,为什么这么急呢?”新娘父亲更加怨恨,于是告了官府。孙公受理了这个案子,他觉得十分奇怪,但又没有头绪,暂判女家等待三年再说。案卷存档,人们先各自回家。

再说新郎住在另一个新娘家,全家人都对他很好。他时常与媳妇商量回家,媳妇也满口答应,就是迟迟不动身。住了半年多,新郎心里就犯了嘀咕,整天焦虑不安。想自己单独回家,媳妇又坚决不让。一天,她们全家惶惶不安,似乎有大难临头。新娘父母急匆匆地对女婿说:“本来打算三两日内叫你们夫妇一起回家,没想到行李用具还没有准备齐全,忽然碰到点麻烦事。不得已,就先送你一人回去吧。”说罢就把新郎送出门来,转身急忙回去了,虽周旋了几句话,也很匆忙草率。

新郎出了大门,刚想找路行走,回头一看房子、院子都没有了,只有—个高大的坟墓,心里非常害怕,急急忙忙找路回家。到了家里,从头到尾说了他的经过,并到官府与孙公说明情况。孙公传新娘的父亲到案,令他送女儿回婆家,于是才正式合婚。


\subsection{1.1.35   灵 官}
\label{\detokenize{p00_u5176_u5b83/_u767d_u8bdd_u804a_u658b_u5fd7_u5f02:id37}}
朝天观有一个道士,喜欢吐纳法术。有一个老翁借住在他的观中,正巧与他爱好相同,于是他俩便成了道友。住了几年,每逢香火大会祭祀神灵的时候,老翁头十天就走开;祭祀完了,他才回来。道士怀疑地问他,老翁说:“我们两人已是莫逆之交,不妨与你实说。我是个狐,祭祀的时候,诸位神仙下界清理污秽,我没处去,只好到别处去藏身。”

又一年,到了祭祀的时候,老翁又走了,这次很久没有回来,道士很怀疑。一天他忽然回来了,道士问他是什么原因,老翁说:“我差点见不到你了。上次祭祀时,本应照样远避,但又懒得走,见阴沟很隐蔽,就暂时藏在卷瓮底下。想不到灵官清除到了这里,一下看见了我,气得就要用鞭打我。我很害怕,急忙逃跑,灵官追我很急。到了黄河沿岸,眼看就追到水边,我没办法,就一头扎进一个大厕坑里,灵官嫌脏,才返身走了。我爬了出来,沾了一身臭气,不能再游历人世间,就到水里洗了一下,隐藏在洞里。过了几百天,一身脏东西才干净了。今天我来告别,并且告诉你,你也应到别处去躲躲,大劫的日子就要到了,这里不是福地。”说完,就告辞而去。

道士依照老头的话也搬到别处去了。没过多长时间,便发生甲申之变。


\subsection{1.1.36   王 兰}
\label{\detokenize{p00_u5176_u5b83/_u767d_u8bdd_u804a_u658b_u5fd7_u5f02:id38}}
利津县有个叫王兰的人,生急病死了。阎王复查生死簿,王兰不该死,是鬼卒错把他抓了来的,就责令鬼卒送他还生。但王兰的尸体已经腐烂,鬼卒怕他不能还生阎王治罪,就与王兰商量说:“人成了鬼受苦,鬼成了仙就享乐。只要有乐享,何必再还生为人呢?”王兰认为很对,就同意了。小鬼对王兰说:“这地方有个狐,成天炼丹,现在已经炼成。我领你去偷那丹来吃,你的魂就不会散,可以长存于世,想干什么就能干什么,没有办不到的事。你愿意不愿意?”王兰听了表示同意。

那鬼卒就领王兰走进一个高大的院落。见院内楼阁整齐,清静幽雅,静悄悄的一个人也没有。只有一只狐,在月光下仰头朝天,从口中呼出一粒丹丸,径直飞入月中;一吸气,那丹丸又落入狐口中。这样一呼一吸接连不断。鬼卒悄悄等在狐的身旁。等狐又呼出时,急忙用手抢来,交给王兰叫他赶快咽下去。狐大吃一惊!怒气冲冲地走过来,一看是两个鬼,怕斗不过他们,就气愤地走了。

王兰与鬼卒告别,回到自己家中。他的妻子见了他就跑,王兰叫住她,告诉妻子前后经过,他妻子才渐渐不害怕了。从此他夫妻住在一起,和往常一样生活。

王兰有个朋友,姓张,听说王兰回来了,就来看他。见面后互相问好,王兰便对张说:“我与你家素来都很穷,现在有办法可以发财了。你能跟我出去游历一番吗?”姓张的没有表态。王兰又说:“我能不用药就治病,不用卜算就知道人的吉凶。我想现原形,又恐认识我的人害怕。所以,我只有附在你身上,咱两人在一起,才能办事。你说行不行?” 姓张的这才答应了。于是两人当天就打点行装出发了。

他俩到了山西地界,听说当地一个财主的女儿生了急病,眼看要死了,前后不知请过多少医生术士都没治好。姓张的带了王兰的魂访到财主家,自称有办法治病,保证起死回生。这个财主只有一个女儿,爱如掌上明珠,治病心切,愿出千金报答。张要求看看小姐的病,随财主到小姐房里,见女子躺在那里,双眼紧闭。掀开被子,用手摸摸身子,也没有知觉,和死了一样,只剩一口气。王兰附在张身上说:“这女子的魂已出舍了,应快找回来。”于是姓张的就告诉财主:“你女儿十分危险,但能治好。”财主问他:“需要什么药?”张说:“什么药也不要,是小姐的魂跑了,我已派神仙去找了。”

过了一个时辰,王兰回来,附在张的耳朵上说,女子魂已找回来了。姓张的请财主再进屋看看,他又摸了一下女子,一会儿,女子伸了伸腰,一下就睁开眼了。财主大喜,马上安慰女儿,并问她情况。女子说:“前几天我去园子里玩,见一个少年用弹弓打麻雀;几个人牵着高头大马跟在他后面。我急着想躲起来,被他们挡住了。少年拿弓给我,教我打弹弓,我觉得害羞,说了他几句,他就捉我上了马,笑着对我说:‘我乐意与你玩,你不要害羞。’走了几里路,进了山。我在马上一面喊一面骂,少年生气,把我从马上推下来。我想回家,又找不到路。正没办法时,一个人来捉住我的胳膊一路小跑,转眼就到了家,只觉恍恍忽忽像做了个恶梦。”财主一听,认为太神奇了,果然拿出千金作为报酬。

王兰与姓张的当夜商量,把得到的千金报酬留下二百两作为他俩的路费,余下的全部由王兰送回家去,交给王兰的儿子,再命儿子给姓张的妻子三百两。王兰办完了当夜又返回来。第二天与财主告别时,财主不见姓张的带着那千金,觉得他更加神奇,又送了些重礼给他。

过了几天,姓张的在郊外遇到同乡贺才。这个贺才整日喝酒赌博,不务正业,穷得和要饭的一样。贺才听说姓张的有发财的法术,得了许多金钱,就到处找他。王兰暗中劝姓张的稍稍给贺才几个钱打发他走。可是贺才改不了老毛病,十天就把钱用光了,还要来找张。王兰已经知道了,就再次对张说:“贺才放荡疯狂,不能长与他相处。只宜给些钱叫他走,恐惹祸还少。”过了几天,贺才果然又来找张,强要和张合作。张就对贺才说:“我就知道你还会来找我!你天天酗酒、赌博,千金也满足不了你的无底洞。你要真心改过自新,我就给你一百两银子,你自谋生路。”贺才高兴得满口答应。张就倒了倒口袋的钱,都给了贺才。贺才有百两银子,反而赌得更厉害,又添了嫖妓的毛病,挥金如土。县里的衙役见他花钱那么容易,怀疑他的钱来路不明,就逮捕了他。贺才到大堂被拷打审问,受刑不过就说了实话,供出钱的来历。县官派人带着贺才去捉姓张的。几天后贺才棒伤溃烂,死在路上。但贺才的魂还没有忘记姓张的,又去找到他附在他身上,与王兰在一起。

一天,张、贺、王三人聚在烟墩喝酒,贺才醉了大喊大叫,王兰制止他,他不听。正遇上巡方御史从这里经过,听到有人大叫就命人搜查,抓住了姓张的。张害怕,就说了实话。御史听了大怒,打了张一顿板子,并写了牒文报告神灵。御史当夜做了个梦,见金甲神人告诉他:“经查王兰是无辜而死,今为鬼仙,从医也是仁术,不能按妖魅治罪。今奉上帝旨意,授为清道史。贺才邪荡,已罚他到铁围山。张某无罪,应即释放。”御史醒来,觉得好生奇怪,就按梦中神人所说,放了姓张的。

张某治理行装回到家里,口袋还存着几百两银子,把一半恭送到王兰家。王兰家的儿孙们从此就富了起来。


\subsection{1.1.37   鹰 虎 神}
\label{\detokenize{p00_u5176_u5b83/_u767d_u8bdd_u804a_u658b_u5fd7_u5f02:id39}}
郡城东岳庙,在南郊。庙的大门两边有神像,身高一丈多,面目狰狞可怕。人们称他鹰虎神。

这个庙里住着一个道士,姓任。他每天鸡叫时就起来烧香念经。这天,有一个小偷一早就藏在走廊里,等道士起来去烧香后,他就进入道士的寝室,到处搜找财物。怎奈这道士很穷,屋里没有什么好东西可偷。小偷找了一遍,只在草垫子底下找到三百钱,就掖在腰里,拨开门闩逃出来,准备爬上千佛山。向南跑了多时,才到了千佛山下。

正走间,遇到一个巨人正从山上走下来,右胳膊上站着个苍鹰,正好与小偷走了个对面。小偷走近前一看,这巨人面如青铜色,模模糊糊好像庙门里常见过的神像一样。小偷大为害怕,蹲在地上直打颤。大神责备他说:“你偷了钱要往哪里去?”小偷更加害怕,不住地叩头。大神伸手揪住他叫他回庙,让他倒出所偷的钱,并叫他跪在那里守着。道士念完经,回头一看,大吃一惊!小偷自己清清楚楚说了是怎么回事。道士收起钱来,打发小偷走了。


\subsection{1.1.38   王 成}
\label{\detokenize{p00_u5176_u5b83/_u767d_u8bdd_u804a_u658b_u5fd7_u5f02:id40}}
王成,原是平原县一个旧官僚家的子弟。他生性懒惰,生活越来越没着落。后来只剩下几间破屋,与妻子睡在破草席上,经常互相怨骂,难以度日。

当时正是炎热的夏季,村子外边原来有个周家的花园,已经墙倒屋塌,只剩下一个亭子。村里有许多人来这里住宿乘凉,王成也在其中。有一天,天亮后,睡在这里的人都走了。太阳升起三杆高了,王成才起来,摇摇晃晃地想要回家。忽然看见草丛中有一股金钗,他拾起来一看,上面刻着“仪宾府造”一行小字。王成的祖父原来是衡恭王府的仪宾,家里的旧物,很多都是这种款式,因此王成拿着金钗踌躇了半天。这时有个老婆婆来寻金钗,王成虽然很穷,但秉性耿直,急忙拿出来交给了她。老婆婆很高兴,极力称赞王成的品德,说:“金钗不值几个钱,可这是已故丈夫的遗物。”王成问:“您夫君是谁呀?”老婆婆回答说:“是已故仪宾王柬之。”王成吃惊地说:“那是我祖父!你们怎么认识的?”老婆婆也惊讶地说:“你就是王柬之的孙子吗?我是狐仙。一百年前,我同你祖父相好。你祖父死后,我就隐居起来了。今天经过这里时遗失了金钗,恰好被你拾到,这不是上天的安排吗!”王成也曾听说过祖父有个狐妻,便相信了老婆婆的话,邀请她到家中坐。老婆婆跟他去了。王成叫妻子出来相见,只见她穿着破烂衣服,面黄肌瘦。老婆婆叹息说:“咳!王柬之的孙子,竟然穷到这种地步!”又见破锅旧灶没有一丝烟火,老婆婆说:“家境如此,你们靠什么生活呢?”王妻就把贫苦的状况细细地述说给老婆婆听,忍不住呜呜咽咽哭泣起来。老婆婆把金钗交给王妻,让她到市上当了钱买些米来暂且度日,三天以后再来相见。王成挽留她,老婆婆说:“一个妻子你还养活不了,我在这里,你只能仰望屋顶,无可奈何,有什么用呢?”说完径自去了。王成对妻子讲了老婆婆的来历,妻子很害怕。王成称颂她的仁义,让妻子像待婆母那样侍奉她,妻子答应了。三天后,老婆婆果然来了。拿出一些银子,让王成买米、面各一石。夜里她就同王成的妻子一块睡在短床上。妻子开始很害怕,但后来看到她心意诚恳,就不再疑心了。

第二天,老婆婆对王成说:“孙子不要再懒惰了,应该做点小买卖。坐吃山空怎么能长久呢?”王成告诉她没有本钱。老婆婆说:“你祖父在时,金银绸缎任凭我取。我因自己是世外之人,不需要这些东西,所以没有多拿过。只积攒下买花粉的四十两银子,至今还存着。长久放在我那儿也没用处,你可以拿去全买成葛布,立即赶到京城卖掉,可赚点利钱。”王成听从了她的话,买了五十多匹葛布回来。老婆婆让他马上收拾行装,估计六七天就可以到京城。并嘱咐王成,“要勤不要懒;要快不能慢。如果晚到一天,后悔就晚了。”王成恭敬地答应了,带着货物上了路。

王成中途遇雨,衣服鞋子全湿透了。他平生从未经历过风霜之苦,疲倦不堪,就决定暂时在旅店歇息。不想大雨下了一整夜,房檐雨流如绳。过了一夜,道路更加泥泞难走。王成见来往的行人,积水没过脚脖,心中怕苦。等到中午,雨才不下了。但一会儿,阴云密布,又下起大雨,王成只好又住了一宿才走。快到京城时,听说葛布价格飞涨,王成心中暗暗高兴。进京后,来到客店解下行装,店主非常惋惜他来晚了。原来,南方的道路刚开通,葛布运至京城的极少;贝勒府又急着购买,价格顿时上涨,比平时贵三倍,前一天才刚购满数额。后来的人都很失望。店主人把缘故告诉王成,王成闷闷不乐。过了一天,葛布运到京城的越来越多,价格更下跌了。王成因为没有利润不肯出售,迟延了十余天,算计食宿花费很多,更加烦闷忧愁。店主人劝他把葛布贱卖掉,改作别的买卖,王成只好听从了,亏了十几两银子,把布全部脱了手。早晨起来,王成准备回去,打开行囊一看,银子全没了。王成惊慌地告诉店主人,主人也没有办法。有人让王成报告官府,要店主偿还。王成叹息说:“这是我命该如此,和店主有什么关系?”店主听说后很感激他,赠送他五两银子,劝慰他让他回去。王成自己考虑着没脸回去见祖母,里里外外地犹豫徘徊,进退两难。

一天,王成恰好看见有斗鹌鹑的,一赌就是几千文钱。每买一只鹌鹑,常常花费不止一百文。他忽然心中一动,算了算行囊中的钱,仅够贩卖鹌鹑的,就回去同店主人商议。店主人极力怂恿他,并且约好让他借住店中,管饭吃,不收他钱。王成很高兴,就上路了。他买了满满一担鹌鹑,又回到京城。店主人很高兴,祝他早点卖光。到了夜里,大雨一直下到天明。天亮后,街上水流如河,雨还是没停。王成只好住在店里等待晴天。可是雨一连下了好几天不停。看看笼中,鹌鹑慢慢死了一些。王成害怕极了,不知怎么办才好。又过了一天,死的更多,仅剩下几只,合并到一个笼子内养着。过了一夜又去看,只有一只还活着。王成告诉了店主人,忍不住泪流满面。店主人也为他振臂叹息。王成觉得银两亏尽,有家难回,只想寻死。店主人劝慰他,同他一块去看那只活下来的鹌鹑。店主人仔细审视一番后说:“这只鹌鹑好像不同寻常。那些死了的鹌鹑,未必不是被它斗杀的。你现在也闲着没事,就训练训练它,如果是个良种,用它来赌博也可以谋生。”王成遵照店主人的意思去做了。驯好以后,店主人让他拿着到街头,赌些酒饭吃。这只鹌鹑十分健壮,几次都赢了。店主人很欢喜,交给王成些银子,让他去与富家子弟赌,又是屡赌屡胜。过了半年多,王成积攒了二十两银子,心里渐感宽慰,把这只鹌鹑看作性命一般。起先,有个大亲王好斗鹌鹑。每逢元宵节,就放民间养鹌鹑的进王府与他的鹌鹑角斗。店主人告诉王成说:“现在发财可以说很容易,所不知道的就是你的运气如何了。”于是就把大亲王府斗鹌鹑的事告诉他,带他一起前去,嘱咐说:“如果败了,就自认丧气出来;倘若万一斗胜了,大亲王肯定要买下来,你不要答应。如果他强买,你看我的脸色行事,等我点头后再答应他。”王成说:“行。”

来到王府,来斗鹌鹑的人已经拥挤在殿阶下。不一会儿,亲王走出御殿,左右随从宣告说:“有愿斗的上来。”立即有一个人手把鹌鹑,快步上去。亲王命令放出王府的鹌鹑,客人也放出自己的,两只鹌鹑刚一搏斗,客方已经败了,亲王大笑。不一会儿,登台败下来的已有好几个人。店主人说:“可以了。”和王成登上台。亲王端详了一下王成的鹌鹑,说:“眼睛里有怒脉,这是只凶猛善斗的鸟,不可轻敌!”命取一只叫铁嘴的鹌鹑来对阵。经过一番跃腾搏斗,王府的鹌鹑败下阵来。又选出更好的,但换一只败一只。亲王急忙命取来宫中的玉鹑。片刻功夫,有人把着这只鹌鹑出来。只见它全身雪白,像鹭鸟一样,神骏不凡。王成胆怯了,跪下请求罢体,说:“大王的鹌鹑是神物;我怕伤了我的鸟,砸了我的饭碗。”亲王笑着说:“放出来吧!如果你的斗死了,我会重重地赔偿你的。”王成这才放出鹌鹑,亲王的玉鹑直扑过来。这时王成的那只正像怒鸡一样伏在那里严阵以待。玉鹑猛地一啄,王成的鹌鹑突然飞起,像仙鹤似地攻击它。两只鹌鹑上下飞腾,相持了很久,玉鹑渐渐不支了。而王成的却更加气盛勇猛,越斗越急,不一会儿玉鹑雪白的羽毛纷纷被啄落,垂翅而逃。周围观看的上千人无不赞叹羡慕王成的鹌鹑。

亲王于是把这鹌鹑要过来放在手上亲自把着它,从嘴到爪,审视一遍,问王成说:“你的鹌鹑卖吗?”王成回答说:“小人没什么产业,与它相依为命,不愿卖它。”亲王说:“赐你好价钱,中等人家的财产马上可以到手,你愿意吗?”王成低头思索了许久说:“本来不愿意卖,大王既然这么喜欢它,如大王真能让我得到一份衣食不愁的产业,我还有什么可求的呢?”亲王便问价钱,王成回答说一千两银子。亲王笑着说:“痴男子!这是什么珍宝,能值一千两银子?”王成说:“大王不认为它是宝,臣民我却认为价值连城的宝玉也没它值钱。”亲王说:“为什么?”王成说:“小人拿着它到市上去赌斗,每次能得几两银子,换成米,一家十几口人指望它吃饭,没有挨饿受冻之忧,什么宝物能比得上它?”亲王说:“我不亏待你,给你二百两银子”。王成摇头。亲王又加百两。王成看了店主人一眼,见店主人没动声色,便说:“承蒙大王愿买,我愿减一百两,九百两银子卖了。”亲王说:“算了吧,谁肯用九百两银子换一只鹌鹑!”王成装起鹌鹑就要走,亲王忙喊:“养鹌鹑的人回来!养鹌鹑的人回来!我实实在在给你六百两银子,肯就卖,否则就算了!”王成又看店主人,店主人仍没什么表情。王成心中已非常满足,惟恐失掉这次机会,说:“以这个数卖给你,心中实在不情愿。但讨还了半天价买卖若不成,得罪了王爷我担当不起。没别的办法,只好按王爷的意思办!”王爷很高兴,立刻秤出银子交给他。王成装好银子,拜谢赏赐出来。店主人埋怨说:“我怎么说的?你这样急着自己作主卖了。再还一下价,八百两银子到手了。”王成回去后,把银子扔在桌上,请店主人自己拿,店主人不要。王成再三相让,店主人才把他的饭钱算清收下。

王成整治好行装回到家,详细述说了自己的经历,拿出银子让大家共享快乐。老婆婆让他买了三百亩良田,盖房子置家具,居然又恢复了祖上的世家景象。老婆婆每天很早就起床,让王成督促佣工耕种;王成的妻子督促家人纺织。稍有懒惰,老婆婆就斥责他俩。夫妇两人安守本分,不敢有怨言。过了三年,家里更富了,老婆婆辞别要走。夫妻二人共同挽留她,直到难过地流泪,老婆婆才留了下来。可第二天早晨,夫妻二人去问安时,老婆婆已经杳无踪影了。


\subsection{1.1.39   青 凤}
\label{\detokenize{p00_u5176_u5b83/_u767d_u8bdd_u804a_u658b_u5fd7_u5f02:id41}}
山西太原耿家,原来是官宦世家,宅院宽阔,气势弘大。后来家势衰落,接连成片的楼房瓦舍,大多都空废着,于是发生了许多奇怪的事情。屋门总是自开自关,家人常常半夜里惊醒呼喊。耿家房主对此很担忧,便搬到别墅里去住,只留下一个老翁看着门。从此宅院更加荒凉败落,有时还能听到里面说笑唱歌吹奏乐器的声音。

耿家房主的侄子叫耿去病,性格狂放不羁。他嘱咐看门的老翁只要听见或看到了什么,就跑去告诉他。到了夜里,老翁见楼上灯光闪烁,就去告诉了他。耿生要去看看是什么东西在作怪,老翁劝阻他,不听。耿生本来就很熟悉院内的房屋门户,便手拔蓬蒿,顺着曲折的路径进了院子。他登上楼房,没看见有什么奇怪的情景。穿过这座楼再往后走,听见有轻微的说话声。偷偷看去,见两只巨大的蜡烛燃烧着,照得四周通明如同白昼。一位头戴儒冠的老头朝南坐着,一位老妇人坐在他的对面,二人都在四十以上的年纪。朝东坐着一位年轻人,约有二十多岁;右边坐着一位女郎,才刚十五六岁的样子。酒菜摆了满满一桌。四人正围坐着说笑。

耿生突然走进房内,笑着喊道:“有一个不速之客来到!”里面的人大为惊慌,奔逃躲避。只有老头出来喝叱道:“是谁闯进人家的内室来了?”耿生说:“这是我家的内室,却被您占了。美酒自己饮,也不邀请主人,岂不有点太吝啬?”老头仔细看了看他说:“你不是这里的主人。”耿生说:“我是狂生耿去病,主人的侄子。”老头致敬说:“久仰大名!久仰大名!”作揖请耿生入坐,喊家人撤换酒肴。耿生不让他换,老头就为耿生斟上酒。耿生说:“咱们是老世交了,刚才酒席上的人没必要回避,还请他们来一起喝酒吧。”老头喊道:“孝儿!”不一会儿,年轻人从外面进来了。老头对耿生说:“这是我的儿子。”孝儿行了个拱手礼坐下。耿生大致问了一下他们的家族姓氏,老头说道:“我叫义君,姓胡。”耿生一向豪爽,谈笑风生。孝儿也很超脱,不同凡俗。两人倾怀畅谈,意气相投,非常喜悦。耿生二十一岁,比孝儿大两岁,就称他为弟。胡叟说道:“听说您的祖父曾经编纂过一部《涂山外传》,您知道吗?”耿生回答说:“知道。”胡叟说:“我是涂山氏的后裔。自唐朝以后的家谱世系我仍然记得,五代以上的就失传了。希望公子能够指教。”耿生大致叙述了涂山女嫁给大禹并帮助他治水的功劳,言谈中丽词妙语,犹如泉涌。胡叟听了大喜,对孝儿说道:“今天有幸听到了以前从未听到过的事情。公子也不是外人,可请你母亲和青凤一起来听听,也好让她们知道我们祖宗的功德。”孝儿便走进了帐幔里面。

一会儿,老妇人带着女郎出来了。耿生仔细看去,女郎柔弱的身姿现出万般娇态,美丽的眼睛流露出聪慧的神色,人间再也找不出比她更漂亮的女子了。胡叟指着妇人说:“这是我的老妻,”又指着女郎说:“这是青凤,是我的侄女,很聪明,所见所闻总是牢记不忘,因此叫来让她听听这些事。”耿生叙述完了又喝酒,两眼紧紧盯着青凤,连眼珠子都不转了。青凤察觉了,就低下了头。耿生暗中去踩青凤的脚,青凤急忙把脚往后缩,脸上也没有怒色。耿生神摇意动,控制不住自已,拍案大声说道:“若得到像青凤这样的妻子,南面为王都不换!”妇人见耿生渐醉越狂,便急忙和青凤一同起身,撩开帏幔走了。耿生很失望,便辞别了胡叟出来。但心里老挂念着青凤,时刻都忘不了。到了夜里,耿生又登上楼去,里面兰麝芳香仍存。凝神等待了一整夜,始终寂静无声。他回家和妻子商议,想把家搬到楼上去住,盼望能再遇见青凤。妻子不同意,耿生于是自己前去,住在楼下读书。

夜里,耿生刚刚靠在桌子上,只见一个鬼披头散发地进了门,脸黑如漆,瞪着两眼看着耿生。耿生笑了笑,用手指蘸着墨汁涂黑自己的脸,目光灼灼地和鬼对视,那鬼很羞惭地走了。第二天晚上,夜已经很深了,耿生吹灭了蜡烛正想睡觉,忽然听见楼后面的门插销发出呯的一声响。耿生急忙起来过去探看,原来门扇半开了。不一会儿听到细碎的脚步声,有人拿着点燃的蜡烛从房子里出来。一看,竟是青凤。青凤猛然看见耿生,吓得往后便退,急忙回去把两扇门关上。耿生直挺挺地跪下,对门内的青凤说:“小生冒着险恶而来,确实是为了您的缘故。幸好这里没有别人,您能让我握一下手,我死了也不遗憾了。”青凤远远地隔着门说:“您对我情深意挚,我岂能不知道!只是叔父管束得很严,我不敢答应您的要求。”耿生苦苦哀求说:“我现在也不敢再有和您握手的奢望了,只想见您一面就满足了。”青凤好似同意了,开门出来,抓着耿生的胳膊拉他起来。

耿生喜出望外,两个携手到了楼下。耿生把青凤抱起来放在自己的膝上。青凤说道:“幸好有缘分,过了今夜,就是相思也没有用了。”耿生问:“为什么?”青凤回答说:“阿叔畏惧您太狂,所以变成厉鬼来吓唬您,您却纹丝不动。现在他已另找好了别的住处,全家人都搬东西到新居去了。我留下看守,明天就走了。”说完就想离去,说:“恐怕叔叔回来。”耿生硬不让她走,想和她亲热。正在相持不下的时候,胡叟不声不响地进来了。青凤又羞又怕,无地自容,低着头倚在床上,手拈衣带不说话。胡叟愤怒地说:“贱丫头辱没了我的门户,再不快走,就用鞭子抽你了!”青凤低着头急忙走了,胡叟也跟了出去。耿生尾随在后面,听见胡叟不住地怒骂,又听见青凤嘤嘤的小声抽泣。耿生心如刀割,大声说:“罪在小生身上,于青凤有什么关系?倘若饶了青凤,任你刀砍斧剁,小生甘愿自身承受!”过了很长时间,一点动静也没有了,耿生这才回去睡觉。

从此以后,宅院里再也没出现过怪异的声息。耿生的叔叔听说后认为耿生不同寻常,愿意把房子卖给他住,也不计较价钱多少。耿生很乐意,便把家口搬了过来。住了一年多,耿生觉得非常舒适,但一刻也没忘记青凤。

正巧清明节上坟回来,耿生见到两只小狐狸被大狗追逼。一只钻进荒草丛中逃窜了;另一只惊慌失措,沿路奔跑,看见耿生,便依依不舍地哀啼着,很温顺地伏首垂耳,好似求他援救。耿生很可怜它,便解开衣襟,把它提起来抱回了家。关上门,把它放在床上,一看竟是青凤。耿生大喜,赶忙慰问她。青凤说:“刚才和丫鬟在外面游玩,遭此大难。如果不是郎君相救,我必定葬身狗腹无疑。希望您不要因为我不是人类而厌恶我。”耿生说:“我天天都思念你,真是魂绕梦想。现在见到你,如获至宝,怎会厌恶呢!”青凤说:“这也是天数,不是因为遭此大难,怎么能够跟随您呢?而且这真是太幸运了!丫鬟一定以为我已经死了,这样正好可以和您终生在一起了。”耿生很高兴,便整理好另一间屋让青凤住下。

过了二年多,一天夜里耿生正在读书,孝儿忽然进来了。耿生放下书卷,惊讶地问他来干什么。孝儿跪在地上,悲伤地说:“家父将遭横难,非您不能拯救。他本想亲自来求您,又恐怕您不愿见他,所以只好让我来了。”耿生问:“什么事?”孝儿说:“您认识莫三郎吗?”耿生说:“他是我同窗学友的儿子。”孝儿说:“明日他将经过您的门前。倘若他携带着猎来的狐狸,希望您能把它要过来留下。”耿生说:“那一年楼下的羞辱,我至今耿耿于怀,他的事我不想过问。若非要我效微劳的话,非让青凤来求不可!”孝儿落泪说:“凤妹已死于荒野三年了!”耿生气愤地用袖子一拂衣服,说:“既然如此,那怨恨就更深了!”说完拿起书卷高声朗读起来,再也不去理他。孝儿从地上爬起来,失声痛哭,用衣袖遮掩着脸走了。耿生到了青凤那里,把事情告诉了她。青凤大惊失色说:“你究竟救不救他?”耿生说道:“救是肯定救他;刚才之所以没答应,是想报复一下他以前的蛮横罢了。”青凤这才高兴地说:“我小时候就失去了父母,依靠叔叔才长起来。过去虽然受到他的责罚,按照家规也是应该那样的。”耿生说:“的确是这样,只是使人不能不耿耿于怀罢了。假若你那次真死了,我决不会救他。”青凤笑着说:“你的心可真狠啊!”

第二天,莫三郎果然来到,他骑着胸带饰金的骏马,佩带着绣有猛虎的弓套,侍从众多,很有声势。耿生出门迎接他,见他猎获的禽兽非常多。其中有一只黑狐狸,伤口流出的血把皮毛都染红了;用手摸了摸它,身上还温和。耿生便假说自己的皮衣破了,请求要这只狐狸的皮来补缀。莫三郎很慷慨地解下它相赠。耿生把狐狸交给了青凤,这才去与客人欢饮。客人走了以后,青凤把狐狸抱在怀里,过了三天它才苏醒,一转身又变成了胡叟。胡叟一抬眼看见了青凤,怀疑这不是在人间。青凤把事情的前后经过说给他听。胡叟于是向耿生下拜,面色羞惭,对以前的过失表示歉意,又很高兴地看着青凤说:“我本来就说你不曾死,今天果真证实了。”青凤对耿生说:“您若爱怜我的话,还求您把楼房借给我家住,好让我能够对老人尽点孝心。”耿生答应了她的要求。胡叟面带愧色道谢告别而去。

到了夜里,胡叟全家果然搬来了。从此两家亲如家人父子,不再互相猜忌。耿生在书房居住,孝儿经常来与他交谈。耿生的正妻生的儿子渐渐长大了,就让孝儿作他的老师;孝儿循循善诱,很有老师的风范。


\subsection{1.1.40   画 皮}
\label{\detokenize{p00_u5176_u5b83/_u767d_u8bdd_u804a_u658b_u5fd7_u5f02:id42}}
太原的王生,清晨早起赶路,遇到一个女子,怀里抱着个包袱,独自在路上奔跑,露出很吃力的样子。王生急忙赶上一看,是一个十几岁的漂亮女子。王生心中很爱慕她,问道:“你怎么天不亮就独自一人赶路?”女子说:“你一个走路的人,又不能解除别人的愁闷,问我干什么?”王生说:“你有什么忧愁?如果我能效力,决不推辞!”女子很悲伤地说:“父母贪财,把我卖给一家有钱人家做小老婆。那家的大老婆非常妒恨我。每天早上骂,晚上打,折磨得我实在受不了了,想逃到远处去。”王生问:“你要到哪里去?”女子说:“逃亡的人,哪有一定的去处?”王生说:“我家离这里不远,就委屈你到我家去吧。”女子听了很高兴,答应了。王生替她背着包袱,领着她一块回家。

女子进了门,看到屋里没人,问:“先生怎么没有家口?”王生回答说:“这是我的书房。”女子说:“这地方很好。你如果可怜我,想救我,就要保守秘密,别让别人知道。”王生答应了,于是二人便睡在了一处。女子藏在书房里,过了许多天也没人知道。王生把这事稍微向妻子陈氏露了点风,妻子怀疑这女子是大户人家的陪嫁女,劝王生打发她走,王生不听。

有一天,王生偶然到集市上,遇见一位道士。道士看见王生,露出很惊愕的样子,问道:“你遇到什么了?”王生回答说;“没遇到什么。”道士说:“你周身邪气围绕,怎么说没有?”王生又竭力辩白,道士只好走了,说:“真蠢啊!世上竟有死到临头还不醒悟的人。”王生听了道士的话很诧异,不禁怀疑起那个女子。转念一想,明明是个美妙女郎,怎么会是妖怪?肯定是道士要假借镇邪祛灾骗饭吃。不一会儿,来到书房门口,发现门从里面关着,进不去,王生心中疑虑,便从墙缺处跳进院子;见房门也紧紧关着,他就悄悄地靠近窗口往屋里瞧,只见一个狰狞的恶鬼,面色青绿,吡着锯齿般的尖牙,拿着彩笔,正在往一张铺在床上的人皮上绘画。画完后,恶鬼扔掉彩笔,举起人皮,像抖衣服那样抖了抖,披在了身上,就立即变成了个女子。王生见此情景,恐惧万分,像狗一样悄悄地爬了出来,急忙去追赶道士,可道士已经不知哪里去了。王生到处寻找,最后在野外碰见道士。王生直挺挺地跪在地上,求道士搭救。道士说:“让我替你赶走它吧。这东西也费了不少苦心,才找到个替身,我也不忍心伤害它的性命。”说完,把一柄拂尘交给王生,叫王生挂在卧室门上。临别时,道士约他第二天在青帝庙会面。

王生回到家,不敢进书房,就睡到妻子屋里,把拂尘挂到门上。到一更时,王生听到门外有动静,自己不敢去看,叫妻子从门缝里瞧瞧。只见一个女子走过来,女子看见房门上的拂尘,不敢进来,站在门外气得咬牙切齿,过了很久才离去。不一会儿,女子又回来了,骂着说:“道士吓唬我!总不能把吃到嘴里的东西再吐出来吧!”说着,摘下拂尘,弄得粉碎,打破房门来到屋里,径直登上王生的床,撕裂开王生的肚腹,抓出心来捧着走了。王生的妻子大声哭叫,女仆听到声音进来,用灯一照,王生已经死了,到处溅满了污血。陈氏吓得不敢哭出声,只淌眼泪。

第二天,陈氏让弟弟二郎跑去告诉道士,道士发怒地说:“我本来可怜它,鬼东西竟敢这样!”就跟着二郎来到家,那女子已不知到哪里去了。道士抬头四下里看了看,说:“幸亏没逃远,”问:“南院是谁家?”二郎说:“是我的住处。”道士说:“那鬼现在你家。”二郎吃了一惊,认为不在他家。道士问他说:“你家可曾有一个不认识的人来?”二郎回答说:“我一早就到青帝庙去了,实在不知道。等我回家问问。”去了不多时又返回来,说:“果然有这事。早晨有一个老妇人来过,她想给我们家当仆人,操持家务,我妻子留下了她,现在还在家中。”道士说:“就是这个东西。”于是同二郎一块去了南院。进了院子,道士手握一把木剑,站在院当中,大喝道:“孽障!赔我的拂尘来!”那老妇人在屋里,吓得惊慌失措,面无血色,窜出门想逃。道士追赶上一剑砍去,老妇人倒在地上,身上的人皮哗的一声脱落下来,变成了一个恶鬼,躺在那里像猪一样嗥叫着。道士用木剑砍下恶鬼的头,鬼的身子化成一股浓烟,在地上旋成一堆。道士取出一个葫芦,拔下塞子,放在烟中,只听嗖嗖地像吸气一样,眨眼间浓烟便都被吸进葫芦里去了。道士把葫芦口塞严,装进口袋里。大家看那张人皮,眉眼手脚,一样不缺。道士卷起人皮,发出像卷画轴一样的声音,也装在口袋里,便告辞要走。陈氏迎门跪拜着,哭求道士救活王生。道士推辞无能为力,陈氏更悲伤了,趴在地上不起来。道士沉思了一会,说:“我法术浅薄,确实不能起死回生。我指给你一人,他或许能救活你丈夫,你去求他,肯定会有办法。”陈氏问:“是什么人?”道士说:“集市上有个疯子,时常躺在粪堆里。你去求他试试,他若侮辱你,你也不要生气。”二郎也听说过这个疯子,于是告别了道士,同陈氏一块去了。

到了集市上,见一个疯乞丐在路上颠颠倒倒地唱着歌,拖着三尺长的鼻涕,脏得让人不敢靠近。陈氏跪着爬到他跟前,疯子笑着说:“美人喜欢我吗?”陈氏讲了缘故,疯子又大笑着说: “人人都可以作丈夫,何必非得救活他?”陈氏苦苦哀求,疯子叫道:“怪哉!人死了,求我救活他,我是阎王爷吗?”生气地用木棒打陈氏。陈氏忍痛挨打,集市上的人渐渐围拢过来,像堵墙一样围着他们。疯子咳了口痰,吐了满满一把,举到陈氏嘴前说:“吃了它!”陈氏脸涨得通红,面有难色。继而又想到道士的嘱咐,只得硬着头皮吃了。咽到喉中,觉得像团棉絮,叽哩咕噜咽下去,最后堵在了胸口间。疯子大声笑着说:“美人喜欢我哟!”接着站起身,头也不回地走了。陈氏在后面跟着,见他走进庙里。陈氏进去一看,不知到哪里去了;前前后后仔细搜寻,竟没一点踪影。陈氏又惭恨又羞愧地回去了。

回家后,陈氏既痛心丈夫死得惨,又悔恨吞痰的羞辱,哭得前仰后台,只求一死。她想给丈夫擦洗血污,收尸入棺,家里人都远远地站着看,没有敢靠近的。陈氏抱着丈夫的尸体收拾肠子,一边收拾一边哭,哭得声嘶力竭。忽然想呕吐,觉得胸中那块堵着的东西,猛劲冲出来,来不及回头,已经掉进丈夫的腹腔中。陈氏吃惊地一看,原来是颗人心,在腹腔中突突地跳动,热气蒸腾像冒烟一样。陈氏大为惊异,急忙用两手合起丈夫的腹腔,用尽力气挤抱着;稍一松劲,就有热气从缝中冒出来。于是她便撕了幅绸子捆扎起来,用手抚摸着尸体,觉得渐渐温暖起来。又盖上被子,半夜里打开被子一看,鼻中有了气息。天亮后,王生竟然活了,自己说:“恍恍惚惚地像做了场梦,只觉得肚子隐隐约约有点痛。”看看原来的伤口,结了个铜钱大的痂,不久就全好了。


\subsection{1.1.41   贾 儿}
\label{\detokenize{p00_u5176_u5b83/_u767d_u8bdd_u804a_u658b_u5fd7_u5f02:id43}}
湖北有个老翁,在外地经商,只剩妻子一人在家。一次,他妻子梦见与别人睡觉,醒了后一摸,是一个又矮又小的男人,看样子不像是人,她心知是狐狸。不一会儿,狐狸下床,门没开,就消失不见了。

到了第二天晚上,妇人叫来给自己做饭的仆妇做伴。妇人有个儿子才十岁,平时在别的床上睡觉,这时也把他叫了来。夜深后,仆妇和孩子都睡着了,狐狸又来了。妇人梦中喃喃地说起梦话来,仆妇惊醒,大声喊叫,狐狸才走了。

从此后,妇人神智恍惚,整天像丢了东西一样。到了夜晚,她不敢熄灯睡觉,告诉儿子不要睡得太死。夜深后,孩子和仆妇都靠着墙壁打盹。一觉醒来,不见了妇人,还以为她去厕所了。等了很久也没回来,才开始怀疑起来。仆妇害怕,不敢出去寻找,孩子独自一人端着灯到院子里到处照了一遍。来到另一间屋子,只见母亲赤裸着身体躺在里面。孩子上前扶起她来,妇人也不知害羞退缩。从此后妇人便疯了,整天又哭又唱,连喊带骂。一到夜晚,就讨厌和别人住在一起,让儿子去别的床上睡,仆妇也被她赶走了。孩子每晚听到母亲笑语,就起来端着灯察看,母亲反愤怒地痛骂他,孩子也不介意。大家因此都夸孩子胆大。

此后,孩子忽然变得无节制地戏耍,天天模仿泥瓦匠,用砖头石块堵窗户,劝阻他也不听。有人如从窗上拿下一块石头,他就在地上打滚,撒娇地啼哭,人们没有敢惹他的。几天后,两个窗子都被他堵死了,没一点光亮。然后又和泥堵墙壁上的洞。整天忙忙碌碌,也不嫌累。墙洞堵完了,没事可干,他又把菜刀拿来霍霍地磨个不停。看见的人都厌恶他太顽皮,没人愿意理他。

一天半夜,孩子把菜刀揣在怀里,用个瓢扣着灯。等到母亲又说起梦话来,他急忙把瓢拿开,用灯照着明,把身子堵住门口,大声叫喊起来。过了很久,没有动静。便离开门口,扬言要搜,还做出要搜的样子。忽然,有个像野猫般的东西倏地窜向门口,孩子急忙挥刀砍去,只砍掉了它的尾巴。约二寸来长,还滴着鲜血。起初,孩子一端灯起来,他母亲便骂个不停,孩子充耳不闻。既而没砍死狐狸,孩子非常懊恨,只得去睡下了。自己想虽然没宰了那东西,但庆幸它从此后不会再来了。

天明后,孩子见狐狸滴下的血迹越墙而去,便一路追踪,见血迹一直通向何家园子。到了夜晚,狐狸果然没来,孩子暗暗喜欢。只是母亲依旧痴痴地躺着,像死了一般。不久,老翁回来。到床前询问妻子的病情。妇人对他谩骂不止,像是对待仇人一般。儿子把经过一说,老翁大惊,请来医生用药治疗。妇人把药泼了,还是大骂。老翁便把药掺和在汤水里让她喝下,几天后,渐渐安定下来。父子二人都很高兴。一夜,父子睡醒后,不见了妇人,二人重又在另一间屋子里找到了她。从此妇人又发疯了,不愿跟丈夫住在一起,一到天黑,就自己跑到别的屋子。想拉住她,她骂得更厉害。老翁无计可施,便把别的屋子的门全部锁死。但妇人一跑了去,门就自己打开了。老翁很忧虑。请来法师作法驱赶狐狸,一点效验也没有。

一天,孩子在天快黑的时候,偷偷地进入何家园子里,藏在乱树丛中,要探查狐狸的踪迹。月亮刚升上来,突然听到有人说话。孩子拨开树枝往外一瞧,见两个人正坐在地上喝酒,有个长胡子的奴仆捧着酒壶在一边伺候。他们穿着深棕色的衣服,谈话声很低很细,隐隐约约,听不太清楚。一会儿,听见一人说:“明天可去取瓶白酒来!”接着,二人都走了。只剩下长胡子奴仆,脱下衣服,睡在庭院石头上。孩子仔细端详了一下,见那奴仆四肢都跟人一样,只是有条尾巴垂在后面。孩子想回去,又恐怕仆人发觉,便在乱树丛里蹲了一夜。天还没明,又听见前次那二人相继走来,叽叽咕咕地说着话,进入竹丛中走了。孩子于是回了家,老翁问他晚上去哪了,他回答说:“睡在阿伯家。”

一次,孩子跟着父亲到街市上去。见帽店里挂着狐狸尾巴出售,便恳求父亲买一条。老翁不同意,孩子拉着父亲的衣服撒娇,吵闹着非要买。老翁不忍过于违了孩子,便买了一条。父亲在市场上做着买卖,孩子在一边玩耍,乘父亲没看见,偷了钱跑了。他先去买了瓶白酒,寄存在酒店的廊下。他有个舅舅在城里住,以打猎为生。孩子跑到舅舅家,正好舅舅不在。舅母询问他母亲的病情,孩子回答说:“这几天稍好一些。但又因为老鼠啃破了衣服,惹得她恼怒地啼哭不止,所以让我来讨猎药。”舅母便打开箱子,取了一钱猎药,包起来交给了他。孩子觉得太少。舅母要包水饺给他吃,孩子乘她出去,屋里没人,自己打开药包,偷了满满一捧藏在怀里。然后急忙跑去告诉舅母,让她不要做饭了,说:“父亲正在街市上等着我,来不及吃了。”说完便走了。去到酒店,把偷的猎药全都暗暗地掺在买来的酒里。又在街上东游西逛了一阵子,直到天晚了才回家。父亲问他去哪里,他假说是在舅舅家。

孩子从此后天天在街上店铺里转来转去。一天,他见那个长胡子仆人也杂在人群里。孩子认准了是他,悄悄地跟着,渐渐和他搭上了话。孩子便询问他住在哪里,仆人回答说:“北村,”又询问孩子,孩子假称:“住山洞。”仆人奇怪他住在洞里,孩子笑着说:“我祖祖辈辈都住在洞里,您难道不是吗?”那人越发吃惊,又询问孩子的姓名。孩子说: “我是胡家的儿子。好像曾在哪里见过你跟着两个年轻人,你忘了吗?”仆人仔细看了看孩子,半信半疑。孩子微微拉开下衣,稍露出一截假尾巴,说:“我们混迹在人群中,只是这东西去不掉,真是可恨啊!”仆人便问:“你在市上干什么?”孩子回答说:“父亲让我来买酒。”仆人告诉他自己也是出来买酒。孩子问:“买到了吗?”仆人回答:“我们大都很贫穷,所以偷的时候多。”孩子同情地说:“这差使也太苦了,耽惊受怕的。”仆人也说:“受主人支使,不得不干。”孩子乘机问他主人是谁,仆人回答说:“就是过去你曾见过的那两个年轻兄弟。一个迷上了北城王家的媳妇,另一个睡在东村某老翁家。老翁家的孩子太可恶,我的那个主人被他砍掉了尾巴,十天后伤才好。现在主人又去他家了。”说完,便要告辞,说:“不要耽误了我的事!”孩子说:“偷酒难,不如买酒容易。我已先买了一瓶,寄存在酒店的廊下,就把这瓶酒送给你吧。我口袋里还有点钱,不愁再买一瓶。”仆人惭愧没东西报答,孩子说:“我们本是同类,吝惜这么点东西干吗?空闲时,我还要请你痛饮一场呢!”仆人跟着孩子去到酒店,孩子取出那瓶酒来交给他,自己便回来了。

到了夜晚,孩子的母亲竟睡得很安稳,不再往外跑。孩子心知定有缘故,告诉父亲,一同去何家园子里察看,只见有两只狐狸死在亭子里,另一只死在草丛中,嘴里还在嘀嘀嗒嗒地淌着血。酒瓶子还在一边,拿起来摇了摇,里面还有剩酒。父亲惊讶地问道:“你怎么不早告诉我呢?”孩子说:“狐狸最有灵性,一旦泄露,它就知道了。”老翁高兴地说:“我儿真是讨伐狐狸的陈平啊!”于是父子二人扛着狐狸回了家,见其中一只尾巴是秃的,刀痕还很明显。

从此以后,老翁家终于太平下来。妇人病得非常瘠瘦,心里渐渐明白。但接着又咳嗽,痰一吐就是几升,不久就死了。北城王家媳妇,过去一直被狐狸迷住,现在又去问了问,狐狸绝迹了,她的病也渐渐好了。老翁由此很珍奇儿子,教他骑马射箭。后来,孩子长大做官,一直做到总兵。


\subsection{1.1.42   蛇 癖}
\label{\detokenize{p00_u5176_u5b83/_u767d_u8bdd_u804a_u658b_u5fd7_u5f02:id44}}
我的同乡王蒲令的仆人吕奉宁,有吃蛇的嗜好。他每次得到小蛇,总是整个吞下,就像吃葱一样。遇见大蛇,就用刀切成一寸一寸的,然后用手捧着吃,嚼得清脆有声,血水沾满两腮。他的嗅觉非常敏锐,曾有一次,他隔墙闻到蛇的香味,急忙奔到墙外,果然抓了条一尺多长的蛇。当时恰好没带刀,他就先吃蛇头,蛇的尾巴还在口边蜿蜒扭动。


\section{1.2   卷 二}
\label{\detokenize{p00_u5176_u5b83/_u767d_u8bdd_u804a_u658b_u5fd7_u5f02:id45}}

\subsection{1.2.1   金 世 成}
\label{\detokenize{p00_u5176_u5b83/_u767d_u8bdd_u804a_u658b_u5fd7_u5f02:id46}}
金世成,是长山县人。平时行为不检点,忽然出家做了个行脚和尚,样子疯疯颠颠的,专爱吃脏东西,吃起来像吃美味佳肴一样。有狗、羊在前面屙了屎,他就跑过去趴在地上津津有味地吃掉。还自称是“佛”,那些愚蠢的百姓妇人,惊异他的行为与众不同,自愿拜他为师的人成千上万。金世成呵斥她们让她们吃屎,没有一个敢违抗的。他给自己盖了座宫殿,花了数不清的钱,都是人们自愿捐献的。县令南公憎恶金世成行为怪诞,将他逮到县衙,打了顿板子,让他出钱去修文庙。金世成的徒弟们奔走相告,说:“佛遭难了!”都争着募钱搭救他。结果文庙没出一个月就修好了。费用的筹集,远比酷吏追逼还快。


\subsection{1.2.2   董 生}
\label{\detokenize{p00_u5176_u5b83/_u767d_u8bdd_u804a_u658b_u5fd7_u5f02:id47}}
董生,字遐思,青州西郊人。一个冬天的傍晚,董生铺好被褥,点上炉火,刚要掌灯时,有朋友来请他喝酒,董就关好门去了。到了朋友家里,在座的有个医生,擅长以诊脉来辨人贵贱吉凶。他给大家挨个诊评了一遍,最后对董生和一个名叫王九思的书生说:“我诊看的人不计其数,但脉象的奇特没人和你俩相同:要说富贵脉吧,又伴有低贱的征兆;要说长寿脉吧,又杂有短命的征状,这都不是我所能知道的。但董君的这种脉象确实很明显。”众人听罢很吃惊,一齐问为什么。医生回答说:“我诊评到这程度也没有办法了,别的不敢随便下结论。愿二位各自慎重行事。”起初,两人听后很害怕,继而一想,又觉得医生的话模棱含糊,也就没放在心上。

半夜时,董生回到家,见房门虚掩着,大为疑惑。醉意朦胧中想了想,一定是走时慌忙急促忘了上锁。进屋后,没顾上点灯,便先伸手摸被窝暖和没有。一下触摸到一个赤身的人躺在里面,董生大吃一惊。抽回手来急忙点灯一看,竟是个红颜少女,美如天仙。董生狂喜万分,上前戏摸她的下身,却意外地摸到条毛茸茸的长尾巴。董生害怕起来,转身想跑。女子已醒过来,一把抓住董生的胳膊,问:“你往哪里跑?”董生越发害怕,战战兢兢地哀求仙人可怜饶恕。女子笑着说:“你见到什么把我当仙人?”董生说:“我不畏首而畏尾!”女子又笑着说:“你搞错了,哪里有尾?”说完就拉过董生的手,硬要他再摸。董生只觉她大腿滑嫩、尾部光秃。女子仍然笑着说:“怎么样?你醉意朦胧,不知看见了什么,这样胡说八道诬赖人!”董生本来就喜欢女子的美貌,这时越发被她迷住了,反自责刚才不该错怪她;然而还是怀疑女子来路不明。女子说:“你不记得东邻的黄毛丫头了吗?算来我家搬走十年了。那时我未成人,你也是个孩子。”董生一下想起来了,说:“你是周家的小阿锁吗?”女子说:“是啊。”董生说:“经你提醒,我这才想起来了。十年不见,你竟出落得这样苗条漂亮。可是你为什么突然来这里呢?”女子说:“我出嫁才四五年,公婆就相继去世,又不幸成了寡妇,孤苦伶仃,没有依靠。想起小时认识的人中只有你了,因此才来投奔你。进门时天已黑了,碰巧有人来请你去喝酒,我就躲在一边等你回来,时间一长,浑身寒冷,就钻到你的被窝里取暖。希望你不要见怪。”董生很高兴,就解衣共枕,尽情欢乐,且十分庆幸自得。

一个月后,董生渐渐形容枯瘦,家人觉得奇怪,就问他原因,他总推说不知道。时间长了,他面目瘦得吓人,才感到害怕,忙又去找原来那位医生,恳请治疗。医生说:“这是妖脉,上次你脉象上的死兆现在已经出现。这病不能治了。”董生大哭,不肯走。医生不得已,只好给他针手灸脐,并送他一包药,嘱咐说:“若再碰到女人,必须坚决拒绝她。”董生也知道自己危险了。回到家里,女子嬉笑着又来勾引他。董生满脸不高兴地说:“不要再来纠缠我,我快要死了!”说完拂袖而去。女子恼羞成怒,生气地说:“你还想活?”晚上,董生服药后独自躺在床上,刚要合眼,就梦见与女子交欢,醒后就遗精了。董生越发惊慌害怕,便搬到内室去睡,让妻子亮着灯守着他,但是仍旧梦遗,看那女子已不知去向了。过了几天,董生就吐了一大盆血死了。

另一个书生王九思一天在书房里读书,忽见一个女子进来。王恋其美貌就和她私通。问她从哪里来,女子说:“我是董遐思的邻居,过去他与我很要好,不料被狐精迷住丧了命。这些狐类的妖气很可怕,读书人应该小心提防。”王听后越发钦佩她,于是两相欢好。日子不长,王便觉得精神恍惚,如染重病。忽然梦见董生对他说:“和你相好的那个女子是个狐精,她害死了我,又要来害你!我已向阴曹地府告了她,以报仇雪恨。七天之内,你必须每天晚上点好香插在室外,千万不要忘了!”王九思醒后觉得这事很奇怪,便对女子说:“我病得很重,恐怕要弃尸于山沟荒涧中。有人劝我不要再行房事了。”女子说:“命里注定你长寿,行房事也活着;没有寿限,就是不行房事也得死。”说着便勾引挑逗。王九思心旌摇动,不能克制,又与她苟合。事后又很悔恨,但总不能摆脱她。到了晚上,王把香插在门上,女子来到后就把香拔下扔了,夜间,王九思又梦见董生来,指责他不该不听话。第二天晚上,王九思暗中嘱咐家人,等他睡后,偷着将香点着插在门上。女子在床上,忽然吃惊地说:“又插上香了!”王推说:“不知道。”女子急忙起身,找到香把它掐灭了,回来说:“谁教你这么干的?”王九思说:“可能是内人担心我的病,听信巫婆的话,给我祛病消灾吧。”女子彷徨不定,闷闷不乐。家人在暗处见香熄灭,又点上插好。那女子叹了口气说:“你福大命好。我不该误害了董遐思又再来害你,的确是我的错。我将和他到阴曹地府对质。你若不忘咱俩过去的感情,就别弄坏我的皮毛。”说完,挣扎下床,扑倒地上死了。王九思点灯一看原来是只狐狸。怕它再复活害人,便招呼家人,剥下它的皮悬挂起来。王九思病得很重,见狐来说“我已向地府提出申诉,地府判决董生见色动心,罪当该死;但又追究我不该诱惑人,没收了我的金丹,命我还生。我的皮毛在哪里”?王九思说:“家人不知有用,已把它剥下扔了。”狐神色凄惨地说:“我害死的人太多了,现在死已经很晚了。然而你也太狠心了!”说完,怀恨而去。王九思这场病几乎送命,半年后才康复。


\subsection{1.2.3   龁 石}
\label{\detokenize{p00_u5176_u5b83/_u767d_u8bdd_u804a_u658b_u5fd7_u5f02:id48}}
新城王钦文老先生家有个姓王的马夫,幼年时入崂山学道。日子一长,就不再食人间烟火,只拣松子和白石头充饥,浑身长满了毛。

过了几年,这个马夫因挂念母亲年老,就返回故里。渐渐又恢复了吃熟食的习惯,但仍然爱吃白石头。他只要把石头对着太阳看看,就能知道石头的酸甜苦辣,吃起石头来就像吃芋头那样津津有味。母亲去世后,他又回到崂山,至今大约又过了十七八年了。


\subsection{1.2.4   庙 鬼}
\label{\detokenize{p00_u5176_u5b83/_u767d_u8bdd_u804a_u658b_u5fd7_u5f02:id49}}
新城秀才王启后,是布政使王象坤的曾孙。有一天,王秀才看见一个又胖又黑,其貌不扬的妇人走进屋里,嬉笑着靠近他坐到床上,样子很放荡。王秀才忙往外赶她走,妇人却赖着不走。从此,王秀才不论坐着躺着,总看见那妇人在跟前。他拿定主意,决不动心。那妇人恼羞成怒,抬手将王秀才的脸打得劈叭作响,王也没觉得怎么痛。妇人又将带子系在粱上,揪住王秀才的头发,逼他与自己一起上吊。王秀才身不由己地跟到梁下,将头伸进吊扣,做出上吊的姿势。有人目睹王秀才脚不沾地,直挺挺地立在半空,却吊不死。

从此,王秀才就患了疯颠病。一天,他忽然说:“她要和我跳河了!”说完就朝河边猛窜,幸亏有人发现才把他拖回来。天天如此,百般折腾,一天发作数次。家中人请巫抓药,都不见效。一天,忽见有个武士拿着铁锁链,怒气冲冲地进来,对那个妇人呵斥道:“你怎敢欺扰这样朴实忠厚的人!”随后就用铁链套住妇人的脖子,硬把她从窗棂中拉了出去。才拖到院子里,妇人就变成一个目如闪电、血盆大口的怪物。有人忽然想起城隍庙里的四个泥鬼中,有一个很像这个怪物。从此王秀才的病便好了。


\subsection{1.2.5   陆 判}
\label{\detokenize{p00_u5176_u5b83/_u767d_u8bdd_u804a_u658b_u5fd7_u5f02:id50}}
陵阳人朱尔旦,字小明,性情豪放。但他生性迟钝,读书虽然很勤苦,却一直没有成名。

一天,朱尔旦跟几个文友一块喝酒。有人跟他开玩笑说:“你以豪放闻名,如能在深夜去十王殿,把左廊下那个判官背了来,我们大家就做东请你喝酒。”原来,陵阳有座十王殿,殿里供奉着的鬼神像都是木头雕成的,妆饰得栩栩如生。在大殿东廊里有个站着的判官,绿色脸膛,红色胡须,相貌尤其狰狞凶恶。有人曾听见夜间两廊里传出审讯拷打声。凡进过殿的人,无不毛骨悚然。所以大家提出这个要求来为难朱尔旦。朱听了,一笑而起,径自离席而去。过了不久,只听门外大叫:“我把大胡子宗师请来了!” 大家刚站起来,朱尔旦背着判官走了进来。他把判官放在桌子上,端起酒杯来连敬了三杯。众人看见判官的模样,一个个在座上惊恐不安,忙请朱尔旦再背回去。朱又举起酒杯,把酒祭奠在地上,祷告说:“学生粗鲁无礼,谅大宗师不会见怪!我的家距此不远,请您什么时候有兴致了去喝两杯,千万不要拘于人神有别而见外!”说完,仍将判官背了回去。

第二天,大家果然请朱尔旦喝酒。一直喝到天黑,朱尔旦喝得醉醺醺地回到家中。酒瘾没过,他又掌上灯,一个人自斟自饮。忽然,有个人一掀门帘走了进来。朱尔旦抬头一看,竟是那个判官!他忙站起身说:“咦!看来我要死了!昨晚冒犯了您,今晚是来要我命的吧?”判官大胡子一动一动的,微笑着说:“不是的。昨晚承蒙你慷慨相邀,今晚正好有空,所以特来赴你这位通达之人的约会。”朱尔旦大喜,拉着判官的衣服请他快坐下,自己起来刷洗酒具,又烧上火要温酒。判官说:“天气暖和,我们凉喝吧。”朱尔旦听从了,把酒瓶放在桌子上,跑了去告诉家人置办菜肴、水果。他妻子知道后,大吃一惊,劝阻他躲在屋里别出去了。朱尔旦不听,立等她准备好菜肴,然后端了过去,又换了酒杯,两个人便对饮起来。朱尔旦询问判官的姓名。判官说:“我姓陆,没有名字。”朱尔旦跟他谈论起古典学问,判官对答如流。朱尔旦又问他:“懂得现时的八股文吗?”判官说:“好坏还能分得出来。阴间里读书作文跟人世差不多。”陆判官酒量极大,一连喝了十大杯。朱尔旦因为已喝了一整天,不觉大醉,趴在桌子上沉沉睡去。等到一觉醒来,只见残烛昏黄,鬼客已经走了。

从此后,陆判官两三天就来一次,两人更加融洽,经常同床而眠。朱尔旦把自己的文章习作呈给陆判官看,陆判官拿起红笔批改一番,都说不好。一夜,两人喝过酒后。朱尔旦醉了,自己先去睡下了,陆判官还在自饮。朱尔旦睡梦中,忽觉脏腑有点疼痛,醒了一看,只见陆判官端坐床前,已经给他剖开肚子,掏出肠子来,正在一根一根地理着。朱尔旦惊愕地说:“我们并无仇怨,为什么要杀我呢?”陆判官笑着说:“你别害怕,我要为你换颗聪明的心。”说完,不紧不慢地把肠子理好,放进朱尔旦的肚子里,把刀口合上,最后用裹脚布把腰缠起来。一切完毕,见床上一点血迹也没有,朱尔旦只觉得肚子上稍微有些发麻。又见陆判官把一团肉块放到桌子上,朱尔旦问是什么东西,陆判官说:“这就是你原来的那颗心。你文思不敏捷,我知道是因为你心窍被堵塞的缘故。刚才我在阴间里,从千万颗心中选了最好的一颗,替你换上了,留下这个补足缺数吧。”说完,便起身掩上房门走了。

天明后,朱尔旦解开带子一看,伤口已好了,只在肚子上留下了一条红线。从此后,他文思大进,文章过目不忘。过了几天,他再拿自己的文章给陆判官看,陆判官说:“可以了。不过你福气薄,不能做大官,顶多中个举人而已。”朱尔旦问:“什么时候考中?”“今年必考第一!”陆判官回答。不久,朱尔旦以头名考中秀才,秋天科考时又中了头名举人。他的同窗好友一向瞧不起他,等见了他的考试文章,不禁面面相觑,大为惊讶。仔细询问朱尔旦,才知道是陆判官给他换了慧心的结果。众人便请朱尔旦把陆判官给大家介绍介绍,都想结交他。陆判官痛快地答应了。众人便大摆酒席。等着招待陆判官。

到了一更时分,陆判宫来了。只见他红色的大胡子飘动着,炯炯的目光像闪电一样,直透人心。众人脸上茫然失色,牙齿不禁格格作响。过了不久便一个跟着一个地离席逃走了。朱尔旦便请陆判官到自己家去喝。二人喝得醉醺醺的时候,朱尔旦说:“你替我洗肠换心,我受你的恩惠也不少了!我还有件事想麻烦你,不知可以吗?”陆判官请他说。朱尔旦说:“心肠既能换,想来面目也可以换了。我的结发妻子身子倒还不坏,只是眉眼不太漂亮,还想麻烦你动动刀斧,怎么样?”陆判官笑着说:“好吧,让我慢慢想办法。”

过了几天,陆判官半夜来敲门。朱尔旦急忙起床请他进来。点上蜡烛一照,见陆判官用衣襟包着个东西,朱尔旦问是什么。陆判官说:“你上次嘱咐我的事,一直不好物色。刚才恰巧得到一个美人头,特来履行诺言来了!”朱尔旦拨开他的衣襟一看,见那脑袋脖子上的血还是湿的。陆判官催促快去卧室,不要惊动鸡犬。朱尔旦担心妻子卧室的门晚上闩上了。陆判官一到,伸出一只手一推,门就开了。进了卧室,见朱尔旦的妻子侧身熟睡在床上。陆判官把那颗脑袋交给朱尔旦抱着,自己从靴子中摸出把匕首,一手按住朱妻的脖子,另一只手像切豆腐一样用力一割,朱妻的脑袋就滚落在枕头一边了。陆判官急忙从朱尔旦怀中取过那颗美人头,安在朱妻脖子上,又仔细看了看是否周正,用力按了按,然后移过枕头,塞到朱妻脑袋下面。一切完毕,命朱尔旦把割下的脑袋埋到一处无人的地方,自己才离去了。

朱妻第二天醒来,觉得脖子上微微发麻,脸上干巴巴的。用手一搓,有些血片,大吃一惊,忙喊丫鬟取水洗脸。丫鬟端水进来,见她一脸血污,惊骇万分。朱妻洗了脸,一盆水全变成了红色。她一抬头,丫鬟猛然见她面目全非,更加吃惊。朱妻自己取过镜子来照了照,惊愕万分,百思不得其解。朱尔旦进来后,告诉了妻子陆判官给换头的经过,又反复打量妻子,见她秀眉弯弯,腮两边一对酒窝,真像是画上的美人。解开衣领一看,脖子上只留下了一圈红线,红线上下的皮肤颜色截然不同。

在此以前,吴侍御有个女儿,非常漂亮。先后两次订亲,但都没过门丈夫就死了,所以十九岁了还没嫁人。上元节时,吴女去逛十王殿,当时游人又多又杂,内中有个无赖窥视到她容貌艳丽,便暗暗访查到她的家,夜晚用梯子翻墙进院,从她卧室的门上打个洞钻进去,先把一个丫鬟杀死在床下,然后威逼要奸淫吴女。吴女奋力抗拒,大声呼救,无赖发怒,一刀把她脑袋砍了下来。吴夫人隐约听见女儿卧室里有动静,喊丫鬟去察看,丫鬟一见房间里的尸体,差点吓死过去。全家人都起来了,把尸体停放在堂屋里,把吴女的头放在她的脖子一侧。一家人号啕大哭,乱了一整夜。第二天黎明,吴夫人掀开女儿尸体上的被子一看,身子在,头却不见了。气得她将看守尸体的侍女挨个痛打了一顿,还以为是她们看守不严,被狗叼去吃了。吴侍御立即把女儿被杀的事告诉了郡府。郡守严令限期缉捕凶手,可三个月过去了,凶手仍没抓到。

不久,朱尔旦的妻子换了脑袋的奇异消息,渐渐传入吴侍御的耳朵里。他起了疑团,派了一个老妈子借故去朱家探看。老妈子一见朱夫人的模样,立刻惊骇地跑回来告诉了吴公。吴公见女儿尸体还在,心中惊疑不定,猜测可能是朱尔旦用邪术杀了女儿,便亲自去盘问朱尔旦。朱说:“我妻子在睡梦中被换了脑袋,实在不知是怎么回事!说我杀了你女儿,真是冤枉!”吴公不信,告了郡府。郡守又把朱尔旦的家人抓了去审讯,结果和朱说的一样,郡守也判断不清。朱尔旦回家后,向陆判官求计。陆判官说:“这不难,我让他女儿自己说清楚。”到了夜晚,吴侍御梦见女儿跟自己说:“女儿是被苏溪的杨大年杀害的,与朱举人没有关系。朱举人嫌妻子长得丑,所以陆判官把女儿的头给朱妻换上了。现在女儿虽然死了,但脑袋还活着,愿我们家不要跟朱举人为仇。”吴侍御醒后,忙把梦告诉了夫人,夫人也做了个同样的梦。于是又告诉了郡府,郡守一问,果然有个杨大年。立即抓了来一拷问,杨大年供认了罪行。吴侍御便去拜访朱尔旦,请求见一见朱夫人。又认了朱夫人为女儿,和朱尔旦结成了翁婿。于是把朱夫人的脑袋安在吴女尸体上埋葬了。

后来,朱尔旦又三次进京考进士,都因为违犯了考场规矩而被黜名。他由此灰心丧气,不再想做官。过了三十年,有一晚,陆判官告诉朱尔旦说:“你的寿命快到头了。”朱尔旦询问死的日期,陆判官回答说五天后。“能挽救吗?”陆判官说:“生死全由天定,人怎能改变呢?况且在通达人看来,生和死是一样的,何必活着就认为是快乐,而死了就觉得悲哀呢?”朱尔旦听了,觉得很对,便置办起寿衣棺材。五天后,他穿着盛装去世了。

第二天,朱夫人正在扶着灵柩痛哭,朱尔旦忽然飘飘忽忽地从外面走来了。朱夫人害怕,朱尔旦说:“我确实是鬼,但和活着时没什么两样。我挂念着你们孤儿寡母,实在是恋恋不舍啊!” 夫人听了,号啕大哭,泪水一直流到胸前。朱尔旦爱抚地劝慰着妻子,夫人说:“古时有还魂的说法,你既然有灵,为什么不再托生呢?”朱尔旦说:“天数怎能违背呢?”妻子又问:“你在阴间干些什么?”朱尔旦回答说:“陆判官推荐我掌管文书,还封了官爵,也没什么苦处。”妻子还想再问,朱尔旦说:“陆公跟我一块来了,快点准备酒菜吧。”说完便出去了。朱夫人立即按丈夫吩咐的去准备。一会儿,便听见陆判官和朱尔旦二人在室内饮酒欢笑,高腔大嗓,宛如生前。到了半夜,再往屋里一看,二人已都不见了。

从此后,朱尔旦几天就来一次,有时就在家里和妻子同宿,顺便料理料理家务事。当时,他的儿子朱玮才五岁。朱尔旦来了后,就抱着他。朱玮长到七八岁,朱尔旦又在灯下教他读书。儿子很聪明,九岁能写文章,十五岁考进了县学,还依然不知道自己的父亲早已去世多年。但此后,朱尔旦来的次数渐渐少了,有时个把月才来一次。

又一天晚上,朱尔旦来了,跟妻子说:“现在要和你永别了!”妻子问:“你要去哪里?”朱回答说:“承蒙上帝任命我为太华卿,马上就要去远方赴任。公务繁忙,路途又遥远,所以不能再来了。”妻子和儿子听了,抱着他痛哭。朱尔旦安慰说:“不要这样!儿子已长大成人,家境也还过得去,世上哪有百年不散的夫妻?”又看着儿子嘱咐说: “好好做人,不要荒废了父亲教给的学业。十年后还能见面。”说完,径直出门走了。从此再没来过。

后来,朱玮二十五岁时考中了进士,做了行人官,奉皇帝令去祭祀西岳华山。路过华阴的时候,忽然有支打着仪仗的人马,急速冲来,也不回避朱玮的队伍。朱玮十分惊异,细看对方车中坐着的人,竟是父亲!朱玮忙跳下马来,跪在路边痛哭。父亲停下车子,说:“你做官的声誉很好,我可以闭目了。”朱玮哭着跪在地上不起来。朱尔旦不顾,催促车辆飞速驰去。刚走了不几步,又回头望了望,解下身上的佩刀,派个人回来送给朱玮,远远地喊道:“佩上这把刀,可以富贵!”朱玮要追着跟去,只见父亲的车马从人,飘飘忽忽地像风一样,瞬间便消失不见了。朱玮怅痛了很久,无可奈何。抽出父亲送给的刀看了看,制作极其精细,刀上刻着一行字:“胆欲大而心欲小,智欲圆而行欲方。”

后来,朱玮做官一直做到司马。生了五个儿子,依次是:朱沉、朱潜、朱沕、朱浑、朱深。有一晚,朱玮梦见父亲告诉自己说:“佩刀应赠给朱浑。”朱玮听从了。后来朱浑官至总宪,很有政声。


\subsection{1.2.6   婴 宁}
\label{\detokenize{p00_u5176_u5b83/_u767d_u8bdd_u804a_u658b_u5fd7_u5f02:id51}}
莒县罗店的王子服,很早就死了父亲。他非常聪明,十四岁时考中了秀才。母亲十分疼爱他,平时不让他到野外去玩。王子服先是聘了萧家的女儿为妻,但萧女还没过门就死了,所以他一直还没娶亲。

一次,正赶上上元节,王子服一个舅舅家的儿子吴生,来邀请他出去游玩。二人刚走到村外,舅家来了一个仆人,把吴生叫走了。王子服见四处游玩的女子很多,便乘兴独自游逛。只见一个女郎带着个丫鬟,手里拈着一枝梅花走过来。那女郎生得艳丽无比,脸上笑容可掬。王子服呆呆地注视着她,眼睛一眨不眨,竟忘了顾忌。女郎走过去几步后,回头看着丫鬟说:“这小伙子目光灼灼,像贼一样!”便把花扔到地上,说笑着迳自走了。王子服捡起花来,惆怅了很久,像丢了魂一样,怏怏不乐地走回来。回到家中,他把花藏到枕头底下,垂着头,一声不响地睡下了,饭也不吃。他母亲十分忧虑,以为他着魔了,请来和尚道士驱邪,王子服却病得更厉害,不久就消瘦下来。母亲又请来医生,开方吃药,还是不管用,整天迷迷糊糊。母亲抚摸着问他得病的缘由,他默默不语。正好吴生来了,王母便嘱咐他暗中询问儿子。吴生来到床前,王子服见了他,流下泪来。吴生近前,说了些安慰的话,渐渐盘问起他的病由。王子服全部实说了,并请他替自己想想办法。吴生笑着说:“你也太痴了!这有什么难办的,我替你查访查访那女子。她既然徒步在野外走,必定不是大家闺秀。如果她还没订亲,事情当然好办;就是订了亲,咱们豁出去多花点彩礼,也会办成。只要你病好了,这事包在我身上!”王子服听了,脸上露出了笑容。吴生出来告诉王母经过,便开始四处探访那女郎的下落。但虽多方查找,仍没一点头绪。王母大为忧虑,一筹莫展。

王子服自吴生走后,心情舒畅,也肯稍稍吃点饭了。过了几天,吴生又来了,王子服便问他事情怎样了。吴生哄他说:“已打听明白了!我以为是谁呢,原来是我姑姑家的女儿,还是你的姨表妹呢!还没订亲,虽说是内亲不宜通婚,但实话告诉他们,没有不成的!”王子服喜笑颜开,问:“她家住在哪里?”吴生骗他说:“住在西南山中,离这里有三十多里路。”王子服又再三嘱咐,吴生大包大揽地应承着走了。从此后,王子服饭量日增,身体一天天好起来。摸摸枕头底下的那枝梅花,虽然枯萎了,但并没有凋落。王子服凝神摆弄着花枝,如同那女郎就在面前。

又过了很久,王子服奇怪吴生再不来了,便写了封请柬,让人去请。吴生借故推托,不肯来。王子服十分生气,郁郁不欢。母亲担心他又要犯病,急急忙忙地给他提亲。但每次和他商量,他都摇头不愿,只是天天盼着吴生来。吴生一直没有音讯,王子服更加怨恨。转而一想,那女子的家离这里只三十里路,何必仰仗他人呢?于是把那枝梅花掖到袖子里,也不告诉家人,自己一人负气去了。

王子服孤孤单单地走着,也无处问路,只是望着南山走去。大约走了三十多里,已进入山中。只见乱山重叠,满目葱绿,令人神清气爽。山中静悄悄的,没有一个行人,只有弯弯曲曲的山路无声地伸向山深处。远远望见谷底,在丛花乱树中,隐隐约约有个小村庄。王子服便走下山,进入村中。村里房屋不多,都是茅屋,但非常干净整洁。朝北的一家,大门掩映在丝丝垂柳中,墙内的桃花杏花开得繁杂茂盛,中间夹杂着几棵修竹,野鸟在花丛中欢快地鸣唱着。王子服以为是谁家的花园,不敢冒然进去。回头见对门有块巨石,非常光滑洁净,他便走过去坐在上面歇息。一会儿,听见墙内有个女子拉长着声音叫“小荣——”,声音娇媚清细。王子服正在凝神谛听,只见一个女子手拿一枝杏花,自东往西走来,边走边低着头,正在往头上插花。一抬头看见王子服,便不再插,含着笑走进院里去了。王子服仔细一看,正是上元节遇到的那个女郎!他心中大喜,想进去又没个理由,想称呼姨母,担心从没来往,怕弄错了。门口也没个人可以问问,急得他坐立不安,徘徊犹豫,从早晨一直挨到太阳西斜,真是望眼欲穿,连饥渴都忘记了。不时见一个女子从院内露出半张脸来窥探,似乎惊讶他还不走。

忽然,一个老太太扶着拐杖走了出来,看着王子服说:“哪里来的小伙子,听说从早晨就在这里,一直呆到现在,要干什么?莫不是饿了吗?”王子服急忙起身作揖,回答说:“我是来探亲的。”老太太耳朵聋,没听清,王子服又大声说了一遍,老太太才问:“你的亲戚姓什么?”王子服答不上来。老太太笑着说:“真稀奇啊!姓名都不知道,还探什么亲?我看你这小伙子,也是个书呆子。不如跟我回家,吃点粗茶淡饭,家中有床,住上一晚,等明早回家问清姓氏,再来探亲也不迟。”王子服正好肚子饿了,想吃点东西,而且进去又能接近那美人,所以十分高兴,于是跟着老太太走进院子。只见院内白石砌路,路两边都是红花,花片乱纷纷地布满了路面、石阶。顺白石路曲曲折折地往西走,又开了一个门,院子里满是豆棚瓜架。老太太将客人请进室内,只见粉白的墙壁,光明如镜;窗外有棵茂盛的海棠花,花枝从窗子里伸进屋里。室内桌椅床褥,都非常洁净。刚坐下,便隐约见有个人在窗外窥视。老太太喊道:“小荣,快去做饭!”外面有个丫鬟高声答应。坐下后,王子服详细讲了自己的家世。老太太问:“你的外祖父莫非姓吴吗?”王子服回答说:“是的。”老太太惊讶地说:“你原来是我的外甥!你母亲是我妹妹。这些年来,因为我们家很穷,又没个男子撑家,所以和你家很少来往,渐渐就断了音讯。外甥长这么大了,我还不认识。”王子服说:“我这次来就是探望姨母,急匆匆地忘了姓氏。”老太太说: “我家姓秦。我一辈子没有生育,只有个女儿,也是侍妾生的。她母亲改嫁走了,把她留给我抚养。她人倒不笨,只是缺少教训,整天嘻嘻哈哈的,也不知愁。过会儿,我让她来见见你,你们认识认识。”过了不久,丫鬟端上饭来,还有只熟鸡。老太太一个劲让王子服多吃。吃完,丫鬟收拾起餐具,老太太吩咐说:“去叫宁姑来!”丫鬟答应着去了。

过了很久,听见门外隐隐约约有笑声。老太太喊道:“婴宁,你姨表兄在这里!”门外仍是嗤嗤地笑。丫鬟将她推进屋来,她还捂着嘴,笑个不停。老太太嗔怪地说:“有客人在,你嘻嘻哈哈的,像什么样子!”婴宁强忍住笑站着,王子服朝她作了一揖。老太太说:“这位王郎,是你姨家的孩子。一家人还不认识,也太可笑了。”王子服问道:“妹子多大了?”老太太没听明白他的问话。王子服又问了一遍,婴宁忍不住又笑得前仰后合。老太太对王子服说:“我说她少教训,你也看见了。十六岁了,又傻又痴,还像个小孩。”王子服说:“妹子小我一岁。”老太太说:“外甥已十七岁了?莫不是庚午年生属马的吗?”王子服点头答应。老太太又问:“外甥媳妇是哪家的?”回答说:“还没有。”老太太说:“像外甥这样的才貌,怎么十七岁了还没娶亲?婴宁也没婆家,你们俩倒挺般配,可惜是内亲。”王子服默默不语,只管盯着婴宁看。丫鬟小声跟婴宁说:“目光灼灼,贼腔没改!”婴宁听了又大笑起来,回头看着丫鬟说:“去看看碧桃开了没有?”便急忙起身,用袖子捂着嘴,迈着碎步匆匆地出去了。刚到门外,就纵声大笑。老太太也站起身,唤丫鬟抱了被褥来,替王子服整理床铺。又对他说:“外甥来一趟不容易,就住三五天吧,慢慢再送你回去。如嫌幽闷,屋后有个小花园,可以去消遣消遣,还有书读。”

第二天,王子服来到屋后,果然有个半亩大的小花园。地上细草如毡,鲜艳的杨花点缀在草地里。有三间草房,四周全是花草树木。王子服穿过花丛,信步走着,忽听树上传来簌簌的声音,仰头一看,原来是婴宁在树上。她看见王子服,哈哈大笑起来,像要从树上掉下来。王子服急忙喊道:“别这样,当心掉下来!”婴宁边笑边往下爬,快到地的时候,一失手摔了下来,才住了笑声。王子服扶起她来,暗暗地捏了一下她的手腕,婴宁笑声又作,倚在树上笑得不能走路了,过了很久才住了声。王子服等她笑够了,从袖子里拿出那枝梅花给她看,婴宁接过去说:“都枯干了,还留着干吗?”王子服说:“这是上元节时妹子扔下的,所以保存着。”婴宁问:“保存它有什么意思?”王子服说:“以表示相爱不忘之意。自从上元节遇见你,我天天思念,得了重病,自以为活不成了。没想到今天竟见到了你,求你可怜可怜我!”婴宁说: “这算什么大事。我们是至亲,吝惜什么?等你回去时,我让老仆把园里的花折一大捆,给你背去。”王子服说:“妹子傻吗?”“怎么是傻呢?”“我不是爱花,是爱拿花的人!”“我们这样疏远的亲戚,谈什么爱?”王子服说:“我所谓的爱,不是亲戚之间的爱,是夫妻之间的爱。”婴宁不解地问:“有什么不同吗?”王子服说:“夜里同床共枕啊。”婴宁低头想了半天,说:“我不习惯和生人睡一起。”还没说完,丫鬟悄悄地走了过来,王子服惶急地逃走了。

过了会儿,王子服和婴宁同到老太太处。老太太问:“你们去哪儿了?”婴宁回答说在园里一起说话来着。老太太说:“饭熟了这么久了,有什么说不完的话,说了这么长时间!”婴宁说: “大哥想和我一块睡觉。”话没完,王子服大窘,急忙拿眼瞪她。婴宁微微一笑,不说了。幸亏老太太耳朵聋,没听见,还在絮絮叨叨地追问,王子服忙用别的话掩饰。过了会儿,王子服小声责备婴宁。婴宁说:“刚才的话不该说吗?”王子服说:“这是背人的话。”婴宁说:“背别人,怎能背老母呢?况且睡觉也是常事,有什么可忌讳的?”王子服恨她不开窍,又没办法让她醒悟。刚吃完饭,家里有人牵了两头驴来找他。

原来,王子服的母亲见他出去后,过了很久没回来,才开始怀疑。村里搜了好几遍,竟没有踪影,因此去问吴生。吴生想起自己过去说过的话,便让王母派人去西南山村中寻找。一连找了好几个村子,才找到这里。王子服走出大门,正巧碰上。王子服便回去告诉老太太,而且请求带着婴宁一块回家。老太太喜欢地说:“我早就有去看妹的心愿,但我老了,走不得远路。你能带你表妹去,认识认识阿姨,这很好。”于是呼唤婴宁,婴宁笑着来了。老太太说:“有什么喜事,总是笑不够?如果不笑,就是完美的人了!”说着生气地瞪了她一眼。又说:“你大哥要带你去姨家,快去收拾收拾。”招待王家的来人吃过酒饭,老太太才送他们出门,嘱咐婴宁说:“你姨家田产很多,能养活闲人。去后不忙回来,学点诗文礼节,将来也好伺候公婆。就便麻烦你姨,替你找个好女婿。”王子服和婴宁一块上了路;直到山坳,回头一望,还依稀看见老太太倚着门朝这边眺望。

回到家中,王子服的母亲见儿子领来个美丽的姑娘,惊讶地问是谁。王子服回答说是姨家的女儿。母亲说:“过去吴生告诉你的话,都是骗你的。我并没有妹妹,哪来的外甥女儿?”又询问婴宁。婴宁说:“我不是现在的母亲生的。我父亲姓秦,他死时,我还在怀抱中,不记事。”母亲说:“我有个姐姐嫁给了姓秦的,倒是真的。但她已死了很久了,哪能还在人世上呢?”又问婴宁她现在母亲的模样、身上的标记,都一一符合。母亲怀疑说:“是我姐姐的模样。但她已死了多年了,怎么可能还活着?”正疑虑间,吴生来了,婴宁忙避入内室。吴生问知缘故,茫然不解。过了很久,他忽然问:“这个女子是不是叫婴宁?”王子服说是。吴生连称怪事。问他怎么了,吴生说:“我嫁给秦家的那个姑姑去世后,姑丈单身被狐狸迷住,得病死去。狐狸生了个女儿,名字就叫婴宁,当时睡在床上,家里人都见过。姑丈去世后,狐狸还经常来。后来求天师在墙壁上贴上符,狐狸才带着女儿走了。这女子莫非就是那个狐狸生的女孩吗?”三人都在猜疑。只听屋里一片嘻嘻哈哈,全是婴宁的笑声。母亲说:“这姑娘也太憨了!”吴生要求看看她。母亲走进屋,婴宁还在大笑不顾。母亲催促她出去见客,她才极力憋住笑声,又面对着墙忍了好一会儿,才走出屋子。刚一施礼,返身就跑进屋内,放声大笑,一屋子的人都被逗得笑了起来。吴生便自报奋勇,到西南山中看个究竟,就便作媒提亲。寻到那个小村庄所在的地方,只见房屋全没有了,只有山花零落而已。吴生想起秦家姑姑下葬的地方,好像就在这一带,但坟墓湮没,辨认不出来,只得又惊奇、又叹息地返了回来。王母怀疑婴宁也是鬼,便进去将吴生的寻访结果告诉婴宁,婴宁一点也不害怕;王母又怜惜她没有家,婴宁却一点也不悲伤,整天只是憨笑,众人都猜不透她。王母叫她和自己的小女儿一块住。婴宁每天早晨都来请安,做的针线活,精巧无比。只是好笑,谁也禁不住。她的笑,虽然狂放,但不损美,众人都爱看她笑。邻居的姑娘媳妇,争着结交她。

王母选了个好日子,要为儿子和婴宁成亲,但终究还是怕婴宁是鬼。一次,王母偷偷地在太阳底下观察婴宁,见她的影子和正常人的一样。到了吉日,王母便让婴宁穿上华丽的服装,行新妇礼。婴宁笑得弯着腰直不起来,只得作罢。王子服因为她憨痴,生恐她泄露了房中隐事,但婴宁却十分保密,不肯对外人多说一句话。每当王母生气或忧愁时,婴宁来到,一笑就化解了。有时奴婢们犯了过错,恐怕遭到鞭打,也总是求婴宁先到母亲房里说话,然后奴婢再去见王母,总是免了处罚。

婴宁爱花成癖,寻遍了亲戚家,到处物色佳种,还偷偷地典当金钗首饰买花。不几个月院里院外到处是鲜花。院后有棵木香树,紧挨着西邻家。婴宁常常爬到树上,摘花插到头上玩。婆母每次碰见,总要斥责她一番,婴宁还是不改。一天,婴宁又爬树时,被西邻家的儿子看见。西邻子见到她的美貌,不禁神魂颠倒。婴宁也不回避,还笑了笑。西邻子以为她看上了自己,样子更加狂荡。婴宁指了指墙根,笑着走了。西邻子以为是指给他约会的地方,大喜过望。到了傍晚,西邻子到婴宁指给的地方,果然见婴宁在那儿,便扑上去抱在怀里。忽觉下身像被锥子刺了一下,痛彻心肺,他大声号叫着跌倒在地。仔细一看,哪里是婴宁,原来是一根枯木桩子躺倒在墙边,刚才他交接的地方是桩子上一个被水淋烂的孔洞。他父亲听到叫声,急忙跑过来询问。儿子只是呻吟着,也不言语。妻子来了,才讲了实情。点上灯往孔洞里照了照,见里面有个巨大的蝎子,像小螃蟹一样。老头劈碎了木桩,捉住蝎子杀了,把儿子背回家中,半夜就死了。老头向官府告了王子服,揭发婴宁是妖异。县令素来仰慕王子服的才华,深知他是个老实厚道的书生,认为老头是诬告,要打他棍子。多亏王子服求情,县令才免了责打,将老头赶出了大堂。婆母对婴宁说:“你平时那样痴狂,我早知会乐极生悲的,幸亏县令神明,没有牵累我们。如果碰上那种糊涂官,一定会逮了媳妇去公堂对质,那时,我儿还有什么脸面见亲戚邻居啊!”婴宁听了严肃地发誓:今后决不再笑了!母亲说:“人哪有不笑的,只是要看时候。”但婴宁从此后竟不再笑,有时故意逗她,她也不笑,但脸上也没忧愁的样子。

一晚,婴宁忽然对着王子服哭泣起来。王子服很诧异,婴宁哽咽着说:“过去我因为跟你的日子还少,说了怕让你惊骇奇怪;现在婆母和你对待我都十分爱怜,没有二心,我就实说了,谅不会有碍吧?我本是狐狸生的,母亲临走时,把我托付给鬼母,相依十多年,才有今天。我又没有兄弟,能依靠的只有你。我的鬼母孤寂地住在山中地下,没人怜惜她,让她和我父亲合葬,她在九泉之下也是遗恨的。你若不嫌麻烦和破费,让地下的人消除了悲痛,那么天下养女孩儿的人,也许不再忍心将女孩溺死或丢弃了!” 王子服答应下来,但担心坟墓迷失在荒草里,不好寻找。婴宁说不必担心。

到了商定的那天,王子服和婴宁用车载着棺材去了。婴宁在一片乱草丛里,指了指坟墓的地方,发掘后,果然找到了那老太太的尸体,还没腐烂。婴宁抚着尸体,悲哀地痛哭起来。王子服把尸体拉回来,寻到秦某的坟墓,把他们合葬了。这天夜晚,王子服梦见老太太来向他致谢,醒后,跟婴宁讲了这事。婴宁说:“我昨夜见到她了,嘱咐她不要惊吓了你。”王子服后悔没有挽留住她。婴宁说:“她是鬼,这里活人多,阳气盛,她怎能久住呢?”王子服又问起小荣,婴宁说:“她也是狐,最聪明,是我狐母留下她照顾我的,常摄来食物喂养我,所以我总是在想念着她。昨晚问我鬼母,说是她已嫁人了。”

从此后,每年的寒食,王子服夫妻二人都要到秦家墓地祭扫,从不间断。婴宁过了一年,生了个儿子,还在怀抱中时,就不怕生人,见人就笑,真像他的母亲啊。


\subsection{1.2.7   聂 小 倩}
\label{\detokenize{p00_u5176_u5b83/_u767d_u8bdd_u804a_u658b_u5fd7_u5f02:id52}}
宁采臣,是浙江人,性情慷慨豪爽,品行端正。常对人说:“我终生不找第二个女人。”有一次,他去金华,来到北郊的一个庙中,解下行装休息。寺中殿塔壮丽,但是蓬蒿长得比人还高,好像很长时间没有人来过。东西两边的僧舍,门都虚掩着,只有南面一个小房子,门锁像是新的。再看看殿堂的东面角落,长着丛丛满把粗的竹子,台阶下一个大水池,池中开满了野荷花。宁生很喜欢这里清幽寂静。当时正赶上学使举行考试,城里房价昂贵,宁生想住在这里,于是就散步等僧人回来。

太阳落山的时候,来了一个书生,开了南边房子的门。宁采臣上前行礼,并告诉他自己想借住这里的意思。那书生说:“这些屋子没有房主,我也是暂住这里的。你如愿意住在这荒凉的地方,我也可早晚请教,太好了。”宁采臣很高兴,弄来草秸铺在地上当床,支上木板当桌子,打算长期住在这里。这天夜里,月明高洁,清光似水。宁生和那书生在殿廊下促膝交谈,各自通报姓名。书生说:“我姓燕,字赤霞。”宁生以为他也是赶考的书生,但听他的声音不像浙江人,就问他是哪里人,书生说:“陕西人。” 语气诚恳朴实。过了一会儿,两人无话可谈了,就拱手告别,回房睡觉。

宁生因为住到一个新地方,很久不能入睡。忽听屋子北面有低声说话的声音,好像有家口。宁生起来伏在北墙的石头窗下,偷偷察看。见短墙外面有个小院落,有位四十多岁的妇人,还有一个老妈妈,穿着暗红色衣服,头上插着银质梳形首饰,驼背弯腰,老态龙钟,两人正在月光下对话。只听妇人说:“小倩怎么这么久不来了?”老妈妈说:“差不多快来了!”妇人说:“是不是对姥姥有怨言?”老妈妈说:“没听说。但看样有点不舒畅。”妇人说:“那丫头不是好相处的!”话没说完,来了一个十七八岁的女子,好像很漂亮。老妈妈笑着说:“背地不说人。我们两个正说着,小妖精就不声不响悄悄地来了,幸亏没说你的短处。”又说:“小娘子真是漂亮得像画上的人,老身若是男子,也被你把魂勾去了。”女子说:“姥姥不夸奖我,还有谁说我好呢?”妇人同女子不知又说些什么。宁生以为她们是邻人的家眷,就躺下睡觉不再听了。又过了一会儿,院外才寂静无声了。宁生刚要睡着,觉得有人进了屋子,急忙起身查看,原来是北院的那个女子。宁生惊奇地问她干什么,女子说:“月夜睡不着,愿与你共享夫妇之乐。”宁生严肃地说:“你应提防别人议论,我也怕人说闲话。只要稍一失足,就会丧失道德,丢尽脸面。”女子说:“夜里没有人知道。”宁生又斥责她。女子犹豫着像还有话说,宁生大声呵斥:“快走!不然,我就喊南屋的书生!”女子害怕,才走了。走出门又返回来,把一锭黄金放在褥子上。宁生拿起来扔到庭外的台阶上,说:“不义之财,脏了我的口袋!”女子羞惭地退了出去,拾起金子,自言自语说:“这个汉子真是铁石心肠!”

第二天早晨,有一个兰溪的书生带着仆人来准备考试,住在庙中东厢房里,夜里突然死了。脚心有一小孔,像锥子刺的,血细细地流出来。众人都不知道是什么缘故。第二天夜里,仆人也死了,症状同那书生一样。到了晚上,燕生回来,宁生问他这事,燕生认为是鬼干的。宁生平素刚直不阿,没有放在心上。到了半夜,那女子又来了,对宁生说:“我见的人多了,没见过像你这样刚直心肠的。你实在是圣贤,我不敢欺负你。我叫小倩,姓聂,十八岁就死了,葬在寺庙旁边,常被妖物胁迫干些下贱的事,厚着脸皮伺候人家,实在不是我乐意干的。如今寺中没有可杀的人,恐怕夜叉要来害你了!”宁生害怕,求她给想个办法。女子说:“你与燕生住在一起,就可以免祸。”宁生问:“你为什么不迷惑燕生呢?”小倩说:“他是一个奇人,我不敢靠近。”宁生问:“你用什么办法迷惑人?”小倩说:“和我亲热的人,我就偷偷用锥子刺他的脚。等他昏迷过去不知人事,我就摄取他的血,供妖物饮用;或者用黄金引诱,但那不是金子,是罗刹鬼骨,人如留下它,就被截取出心肝。这两种办法,都是投人们之所好。”宁生感谢她,问她戒备的日期。小倩回答说明天晚上。临别时她流着泪说:“我陷进苦海,找不着岸边。郎君义气冲天,一定能救苦救难。你如肯把我的朽骨装殓起来,回去葬在安静的墓地,你的大恩大德就如同再给我一次生命一样!”宁生毅然答应,问她葬在什么地方。小倩说:“只要记住,白杨树上有乌鸦巢的地方就是。”说完走出门去,一下子消失了。

第二天,宁生怕燕生外出,早早把他请来。辰时后就备下酒菜,留意观察燕生的举止,并约他在一个屋里睡觉。燕生推辞说自己性情孤癖,爱清静。宁生不听,硬把他的行李搬过来。燕生没办法,只得把床搬过来,并嘱咐说:“我知道你是个大丈夫,很仰慕你。有些隐衷,很难一下子说清楚。希望你不要翻看我的箱子包袱,否则,对我们两人都不利!”宁生恭敬地答应。说完两人都躺下,燕生把箱子放在窗台上,往枕头上一躺,不多时鼾声如雷。宁生睡不着,将近一更时,窗子外边隐隐约约有人影。一会儿,那影子靠近窗子向里偷看,目光闪闪。宁生害怕,正想呼喊燕生,忽然有个东西冲破箱子,直飞出去,像一匹耀眼的白练,撞断了窗上的石棂,倏然一射又马上返回箱中,像闪电似地熄灭了。燕生警觉地起来,宁生装睡偷偷地看着。燕生搬过箱子查看了一遍,拿出一件东西,对着月光闻闻看看。宁生见那东西白光晶莹,有二寸来长,宽如一韭菜叶。燕生看完了,又结结实实地包了好几层,仍然放进箱子里,自言自语说:“什么老妖魔,竟有这么大的胆子,敢来弄坏箱子!”接着又躺下了。宁生大为惊奇,起来问燕生,并把刚才见到的情景告诉他。燕生说:“既然我们交情已深,不能再隐瞒,我是个剑客。刚才要不是窗户上的石棂,那妖魔当时就死了。虽然没死,也受伤了。”宁生问:“你藏的是什么东西?”燕生说:“是剑。刚才闻了闻它,有妖魔的气味。”宁生想看一看,燕生慷慨地拿出来给他看,原来是把莹莹闪光的小剑。宁生于是更加敬重燕生。天亮后,发现窗户外边有血迹。宁生出寺往北,见一座座荒坟中,果然有棵白杨树,树上有个乌鸦巢。等迁坟的事情安排妥当,宁生收拾行装准备回去。燕生为他饯行送别,情谊深厚。又把一个破皮囊赠送给宁生,说:“这是剑袋,好好珍藏,可以避邪驱鬼。”宁生想跟他学剑术,燕生说:“像你这样有信义、又刚直的人,可以作剑客;但你是富贵中人,不是这条道上的人。”宁生托词有个妹妹葬在这里,挖掘出那女子的尸骨,收敛起来,用衣、被包好,租船回家了。

宁生的书房靠着荒野,他就在那儿营造坟墓,把小倩葬在了书房外面。祭奠的时候,他祈祷说:“怜你是个孤魂,把你葬在书房边,相互听得见歌声和哭声,不再受雄鬼的欺凌。请你饮一杯浆水,算不得清洁甘美,愿你不要嫌弃。”祷告完了就要回去。这时后边有人喊他:“请你慢点,等我一起走!”宁生回头一看,原来是小倩。小倩欢喜地谢他说: “你这样讲信义,我就是死十次,也不能报答你!请让我跟你回去,拜见公婆,给你做婢妾都不后悔。”宁生细细地看她,白里透红的肌肤,如同细笋的一双脚,白天一看,更加艳丽娇嫩。于是,宁生就同她一块来到书房,嘱咐她坐着稍等一会儿,自己先进去禀告母亲。母亲听了很惊愕。这时宁生的妻子已病了很久,母亲告诫他不要走漏风声,怕吓坏了他的妻子。倒说完,小倩已经轻盈地走进来,跪拜在地上。宁生说:“这就是小倩。”母亲惊恐地看着她,不知如何是好。小倩对母亲说,“女儿飘然一身,远离父母兄弟,承蒙公子照顾,恩泽深厚。愿意作婢妾,来报答公子的恩情。”母亲见她温柔秀美,十分可爱,才敢同她讲话,说:“小娘子看得起我儿,老身十分喜欢。但我这一生就这一个儿子,还指望他传宗接代,不敢让他娶个鬼媳妇。”小倩说:“女儿确实没有二心,我是九泉下的人,既然不能得到母亲的信任,请让我把公子当兄长侍奉。跟着老母亲,早晚伺候您,怎么样?”母亲怜惜她的诚意,答应了。小倩便想拜见嫂子,母亲托词她有病,小倩便没有去;又立即进了厨房,代替母亲料理饮食,出来进去,像早就住熟了似的。天黑了,母亲害怕她,让她回去睡觉,不给她安排床褥。小倩知道母亲的用意,就马上走了。路过宁生的书房,想进去,又退了回来,在门外徘徊,好像害怕什么。宁生叫她,小倩说:“屋里剑气吓人,以前在路上没有见你,就是这个缘故。”宁生明白是那个皮囊,就取来挂到别的房里,小倩才进去。她靠近烛光坐下,坐了一会儿,没说一句话。过了好长时间,小倩才问:“你夜里读书吗?我小时候读过《楞严经》,如今大半都忘了。求你给我一卷,夜里没事,请兄长指正。”宁生答应了。小倩又坐了一会儿,还是不说话;二更快过去了,也不说走。宁生催促她,小倩凄惨地说:“我一个外地来的孤魂,特别害怕荒墓。”宁生说:“书房中没有别的床可睡,况且我们是兄妹,也应避嫌。”小倩起身,愁眉苦脸的像要哭出来,脚步迟疑,慢慢走出房门,踏过台阶不见了。宁生暗暗可怜她,想留她在别的床上住下,又怕母亲责备。小倩清晨就来给母亲问安,捧着脸盆侍奉洗漱。操劳家务,没有不合母亲心意的。到了黄昏就告退辞去,常到书房,就着烛光读经书。发觉宁生想睡了,才惨然离去。

先前,宁生的妻子病了,不能做家务,母亲累得疲惫不堪。自从小倩来了,母亲非常安逸,心中十分感激。待她一天比一天亲热,就像自己的女儿,竟忘记她是鬼了,不忍心晚上再赶她走,就留她同睡同起。小倩刚来时,从不吃东西、喝水,半年后渐渐喝点稀饭汤。宁生和母亲都很溺爱她,避讳说她是鬼,别人也就不知道。没多久,宁生的妻子死了。母亲私下有娶小倩作媳妇的意思,又怕对儿子不利。小倩多少知道母亲的心思,就乘机告诉母亲说:“在这里住了一年多,母亲应当知道儿的心肠了。我为了不祸害行人,才跟郎君来到这里。我没有别的意思,只因公子光明磊落,为天下人所敬重,实在是想依靠他帮助三几年,借以博得皇帝封诰,在九泉之下也觉光彩。”母亲也知道她没有恶意,只是怕她不能生儿育女。小倩说:“子女是天给的。郎君命中注定有福,会有三个光宗耀祖的儿子,不会因为是鬼妻就没子孙。”母亲相信了她,便同儿子商议。宁生很高兴,就摆下酒宴,告诉了亲戚朋友。有人要求见见新媳妇,小倩穿着漂亮衣服,坦然地出来拜客。满屋的人都惊诧地看着她,不仅不疑心她是鬼,反而怀疑她是仙女。于是宁生五服之内的亲属,都带着礼物向小倩祝贺,争着与她交往。小倩善于画兰花和梅花,总是以画酬答。凡得到她画的人都把画珍藏着,感到很荣耀。

一天,小倩低头俯在窗前,心情惆怅,像掉了魂。她忽然问:“皮囊在什么地方?”宁生说:“因为你害怕它,所以放到别的房里了。”小倩说:“我接受活人的气息已很长时间了,不再害怕了。应该拿来挂在床头!”宁生问她怎么了,小倩说:“三天来,我心中恐惧不安。想是金华的妖物,恨我远远地藏起来,怕早晚会找到这里。”宁生就把皮囊拿来,小倩反复看着,说:“这是剑仙装人头用的。破旧到这种程度,不知道杀了多少人!我今天见了它,身上还起鸡皮疙瘩。”说完便把剑袋挂在床头。第二天,小倩又让移挂在门上。夜晚对着蜡烛坐着,叫宁生也不要睡。忽然,有一个东西像飞鸟一样落下来,小倩惊慌地藏进帷幕中。宁生一看,这东西形状像夜叉,电目血舌,两只爪子抓挠着伸过来。到了门口又停住,徘徊了很久,渐渐靠近皮囊,用爪子摘取,好像要把它抓裂。皮囊内忽然格的一响,变得有两个竹筐那么大,恍惚有一个鬼怪,突出半个身子,把夜叉一把揪进去,接着就寂静无声了,皮囊也顿时缩回原来的大小。宁生既害怕又惊诧。小倩出来,非常高兴地说:“没事了!”他们一块往皮囊里看看,见只有几斗清水而已。几年以后,宁生果然考取了进士,小倩生了个男孩。宁生又纳了个妾,她们又各自生了一个男孩。三个孩子后来都做了官,而且官声很好。


\subsection{1.2.8   义 鼠}
\label{\detokenize{p00_u5176_u5b83/_u767d_u8bdd_u804a_u658b_u5fd7_u5f02:id53}}
杨天一说:曾看见两只老鼠出洞,一只被蛇吞下,另一只瞪着眼睛如同花椒粒,非常怒恨,但它只是远远地盯着不敢向前。蛇吃饱了肚子,就蜿蜒地向洞内爬去;刚爬进一半,那只老鼠猛地扑来,狠狠地死咬住蛇的尾部。蛇怒,急忙退出洞来。老鼠本来就非常机灵敏捷,便飞快地跑了。蛇追不上,又入洞。老鼠又跑回来和上次一样咬住不放。就这样蛇入鼠咬,蛇出鼠跑,反复了好多次。最后,蛇爬出洞来把吞下的死鼠吐在地上,那只老鼠才作罢。它用鼻子嗅着自己的同伴,吱吱叫着悲鸣痛悼。继而,用嘴衔着死鼠去了。我的朋友张历友为此写了一篇《义鼠行》。


\subsection{1.2.9   地 震}
\label{\detokenize{p00_u5176_u5b83/_u767d_u8bdd_u804a_u658b_u5fd7_u5f02:id54}}
康熙七年六月十七日戍刻,发生了大地震。当时,我在稷下做客,正和表兄李笃之在灯下喝酒。忽然听见有种像打雷一样的声音,从东南方向过来,向西北方向滚去。大家都很惊骇诧异,不知是什么缘故。不一会儿,只见桌子摇晃起来,酒杯翻倒;屋梁房柱,发出一片咔咔的断裂声。众人大惊失色,面面相觑。过了好久,才醒悟过来是地震,急忙冲出屋子。只见外面的楼阁房屋,一会儿斜倒在地上,一会儿又直立起来;墙倒屋塌的声音,混合着孩子号哭的声音,一片鼎沸,震耳欲聋。人头晕得站不住,只能坐在地上,随着地面颠簸。河水翻腾出岸边一丈多远;鸡叫狗吠,全城大乱。过了一个时辰,才稍微安定下来。再看大街上,男男女女,都光着身子聚在一起,争相讲着刚才的事情,都忘了没穿衣服。

后来,听说这次地震时,某处有口水井井筒倾斜了,不能再打水;某家楼台南北掉了个方向;栖霞山裂了道缝;沂水陷下了一个有几亩大的地穴。这真是少有的奇异灾变啊!


\subsection{1.2.10   海 公 子}
\label{\detokenize{p00_u5176_u5b83/_u767d_u8bdd_u804a_u658b_u5fd7_u5f02:id55}}
东海的古迹岛上,生长着一种五色的耐冬花,一年四季鲜花盛开。岛上自古以来无人居住,是个人迹罕至的地方。

登州人张生,好探奇寻幽,喜爱游猎。听说这里风景优美,就准备好酒饭,独驾扁舟前往。到时正繁花似锦,香飘数里。最粗大的树干,需十多人才围得过来。宜人的景色,令张生留连忘返,十分惬意。于是便开瓶自饮,后悔没带个伴来。忽然,从花丛中走出个身着红色衣裙、光彩照人的漂亮女子,见张生一个人喝酒,就嘻笑着说:“我自以为兴致不凡,没想还有比我兴致更高且捷足先登的人呢!”张生吃惊地问她是什么人,女子回答说:“我是胶东的娼妓,刚跟海公子来。他到别处游玩揽胜去了,我走不动,所以留在这里等他。”张生正苦于寂寞,来了个美人作伴,非常高兴,连忙招呼她坐下一起喝酒。那女子言谈温婉,荡人心神。张生很喜欢她,怕海公子来后,不能尽情欢乐,就抱住她亲热起来,女子欣然俯就。两人正在亲热,忽听狂风大作,草木折断发出响声。女子急忙推开张生站起来说:“海公子来了!”张生慌忙扎好腰带,吃惊地回头看时,女子已不知去向。接着,见一条比水桶还粗的大蛇,自树丛中窜出。张生惧怕,急忙躲到大树后面,希望蛇没看见他。那蛇窜近前来,用身子连人带树结结实实地缠了数匝。张生的两条胳膊被缠在两胯中间,一点也不能弯曲。这时,那蛇昂起头,用舌头刺破张生的鼻子,鼻血不断往下滴着,淌到地上形成个小洼,那蛇就俯首饮血。张生自料必死。忽然想起腰间系着的荷包袋中,装着毒狐的药。就用两个指头把药夹出,弄破堆在掌心;又转过头来眼看着手掌,让血滴到药上,转眼间滴满了一把血。那蛇果然就掌中饮血,还没喝完,突然伸直了身子,尾巴猛烈摆动起来,发出霹雳一般的响声,碰着的树都被拦腰扫断。不一会儿,便像一架屋梁那样倒在地上死了。张生被吓得魂飞魄散,倒在地上站不起来,过了一阵才醒过来,便将蛇用船载回去。

到家后,他生了一场大病,一个月后才康复。他怀疑那女子也是个蛇精。


\subsection{1.2.11   丁 前 溪}
\label{\detokenize{p00_u5176_u5b83/_u767d_u8bdd_u804a_u658b_u5fd7_u5f02:id56}}
丁前溪,诸城人。家中富有钱粮,好仗义疏财,抱打不平,最钦佩古侠客郭解的为人。御史行台听说后,要拜访他,丁前溪逃跑了。到安丘,遇上下雨,他就到一家旅舍暂避。一直到中午,雨仍下个不停。这时,有个少年过来,用丰盛的饭菜招待他。转眼天黑了,雨仍下得很大,丁前溪只好去少年家过夜。那少年既照顾他的食宿,又照料他的马,处处细心周到。问那少年的姓名,回答说:“我家主人姓杨,我是他的内侄。主人喜好交往,刚才有事出去了,现只有她的妻子在家。家中贫穷,拿不出更好的东西款待你,请多多包涵。”丁前溪又问主人的职业,得知杨某并无资产,唯有靠开设赌场养家糊口。

第二天,仍旧阴雨连绵,但主人供给丁前溪的饭食照样热情周到,无丝毫怠慢。傍晚铡草喂马时,丁前溪见饲料长短不齐,且一把干一把湿,觉得奇怪,就问少年。少年说:“实不相瞒,我家穷得无草喂马,这还是娘子让我从屋顶上撤下来的茅草呢!”丁前溪越发奇怪,以为这是主人借此向他要钱。天亮后,见雨已停,他便收拾好行李,拿出银子给少年,少年不要。丁前溪硬塞给他,少年无可奈何,拿着银子进屋请示女主人。一会儿出来把银子还给丁前溪,并说:“娘子说:我们不是靠这个来赚钱吃饭的。主人在外,常常好几天不捎回一文钱来;你是客人,怎么能向你索要报酬呢?”丁前溪听了很受感动,连声赞扬,叹服女主人的为人。临走再三嘱咐说:“我是诸城的丁前溪。主人回来后,请你转告他,让他闲暇时到我家一聚。”

丁前溪走后数年没有音信。这年碰上闹饥荒,杨家穷困到极点。没有办法,杨妻就劝丈夫去找丁前溪请求接济,杨某答应了。到了诸城,找到丁前溪的家,让看门人通报了姓名,可丁前溪怎么也想不起这个人来。杨某就将当年的事对仆人说了一遍,丁前溪听后,慌得趿拉着鞋就跑出来迎客。见杨某衣着破烂,鞋子露着脚后跟,他就请杨某到暖和屋里,设宴盛情款待,礼仪隆重,非同寻常。次日,丁又为杨某赶制衣帽鞋袜,杨被打扮得表里一新,心里热乎乎的,很激动,觉得丁前溪很讲义气够朋友。但一想到家中断炊的情形,便增添了忧愁,只盼望主人能快点接济点钱粮赶回家去。又住了几天,见主人还没有送别的意思,杨某急得忍不住对丁前溪说:“我考虑再三,不能再瞒你了。我来时,家中米不满升。如今我受到你的盛情款待,当然很高兴,可家里的妻子怎么过呢?”丁前溪笑着说:“这些事你不用惦念,我已全部替你办妥了。请放心再住几天,让我给你凑点路费。”丁前溪就派人去召集众赌徒来聚赌,让杨某向赢方抽头渔利,一夜间就得到百两银子。丁前溪这才送杨某回家。

杨某进家门一看,合家衣着焕然一新,妻子身边还有个丫鬟伺候。杨某吃惊地问妻子,妻子说:“你走后,第二天就有人赶着马车送来了布匹粮食,堆了满满一屋,说是丁客人让送的;还带来个丫鬟,供我使唤。”杨某激动不已。从此,杨家过上了小康生活,再也用不着设赌场赚钱度日了。


\subsection{1.2.12   海 大 鱼}
\label{\detokenize{p00_u5176_u5b83/_u767d_u8bdd_u804a_u658b_u5fd7_u5f02:id57}}
东海的海边上,本来没有什么山。一天忽见海中峻岭重叠,连绵数里,大家都很惊讶奇怪。

又一天,群山突然迁徙到别的地方去了,海边上又恢复了原来的样子,山全不见了。

人们传说海中有大鱼,到清明时节,它们就携家带口,去祖坟祭墓,因此,在寒食时节经常见到。


\subsection{1.2.13   张 老 相 公}
\label{\detokenize{p00_u5176_u5b83/_u767d_u8bdd_u804a_u658b_u5fd7_u5f02:id58}}
有个张老相公,是山西人。因女儿将要出嫁,就携带家眷到江南去,亲自为女儿置办嫁妆。船到金山时,张老相公欲先过江,嘱咐家人在船上切莫煎炒有腥膻气味的鱼肉。因为江中有只大鼋作怪,它闻到香味就要出来毁船吞人,在这里已经为害很久了。

张老相公走了以后,家人忘记了他的嘱咐,在船上烤肉。忽然,江中巨浪滔天把船打翻,张老相公的妻女等人全落水沉没。张老相公乘小船回到大船停靠的地方,不见妻子女儿,痛不欲生,恨不得立刻报仇。他登上金山,拜见了金山寺的和尚,打听鼋怪为害的情况,想除鼋报仇。僧人听了非常害怕,惊讶地说:“我们整年整月住在它的近处,怕遭到祸害,只好将它当神仙供奉,祈祷它不要发怒。经常屠猪宰羊,半只半只地投入江中,鼋即跃出吞食而去。谁敢与它作对啊!”

张老相公听后,立刻想出一个报仇的计谋。他找来几个铁匠,在金山的半腰处安起炉灶,炼成一块百余斤重的大铁块。问清了大鼋常出没的地方,叫几个身强力壮的男子汉,用大铁钳举起铁块投向江中。鼋跃出,疾吞而下。一会儿,江上波涌如山,顷刻又浪息波平,那大鼋的尸体已浮上水面。过往的商客和金山寺的僧人都为之欢快,修建了张老相公祠,在祠内悬挂了张老相公的像,并把他当做水神供奉。人们向他祈祷,都很灵验。


\subsection{1.2.14   水 莽 草}
\label{\detokenize{p00_u5176_u5b83/_u767d_u8bdd_u804a_u658b_u5fd7_u5f02:id59}}
水莽是毒草,像葛类一样蔓生,花是紫色的,像扁豆。人如误吃了这种毒草,就会立即死去,变成“水莽鬼”。民间传说,这种鬼不能轮回,一定得再有被毒死的代替,才能去投生。因此,楚中桃花江一带,这种水莽鬼特别多。

楚中人称呼同岁的人为“同年”。往来拜访时,互称庚兄庚弟,子侄辈们则称他们为庚伯,这是本地的习俗。

有个姓祝的书生,一次去拜访他的一个同年。途中非常干渴,很想喝水。忽然看见路旁有个凉棚,一个老婆婆在里面施舍茶水,祝生就跑了过去。老婆婆将他迎入棚内,端上茶来,十分殷勤。祝生一闻,有股怪味,不像是茶水,便放下不喝,起身要走。老婆婆忙拦住他,回头向棚里喊道:“三娘,端杯好茶来!”一会儿,便有个少女捧着杯茶从棚后出来,大约十四五岁年纪,容貌艳丽绝伦。指上的戒指、腕上的镯子,光亮得能照见人影。祝生见了少女,立即被吸引住。接过茶水一闻,只觉芳香无比,一饮而尽,还想再喝一杯。乘老婆婆出去,祝生一下抓住少女的纤纤手腕,从她手指上脱下一枚戒指。少女红着脸微微一笑,祝生更加着迷,便询问她的家世。少女说: “你晚上再来吧,我还在这里。”祝生要了她一撮茶叶,连同那枚戒指,一块藏在身上走了。

祝生赶到同年家,忽觉心头不适,怀疑是喝了那杯茶水的缘故,便将经过告诉了同年。那同年惊骇地说:“坏了!这是水莽鬼,我父亲就是被这样害死的。无药可救,这可怎么办呢?”祝生恐惧万分,忙拿出藏在身上的茶叶一看,果然是水莽草。又拿出那枚戒指,向同年描述了那少女的模样。同年冥想了一会,说:“那人必定是寇三娘!”祝生听他说的名字相符,问他是怎么知道的,同年回答说:“南村富户寇家的女儿,叫三娘,以艳丽闻名。几年前误吃了水莽草死去,肯定是她在作怪害人!”有人说,碰到水莽鬼的人,如知道鬼的姓名,只要求到他生前穿过的裤子,煎水服用,就可以痊愈。祝生的同年急忙赶到寇家,讲明了实情,长跪在地,苦苦哀求帮忙。寇家却因为有人做女儿的替身,女儿从此可以超生,坚决不给。同年无可奈何,忿忿回去,告诉了祝生。祝生咬牙切齿地说:“我死后,绝不让他家女儿投生!”这时,祝生已走不动了。同年将他背回家,刚到家门就死了。祝生的母亲号啕大哭,只得把他埋葬了。祝生死后,留下一子,刚刚周岁。妻子不能守节,过了半年就改嫁走了。母亲一人抚养着小孙子,劳累不堪,天天哭泣。

一天,祝生母亲正抱着孙子在屋里啼哭,祝生忽然无声无息地进来了。祝母大惊,抹着眼泪问他情况。祝生回答说:“儿在地下听到母亲哭泣,心里很感悲伤,所以来早晚伺候您。儿虽然死了,但已成家,媳妇也马上同来替母亲操劳,母亲不要难过了!”母亲惊疑地问:“儿媳妇是谁?”祝生回答说:“寇家坐视儿死不救,儿非常恨他们!死后,一心要去找寇三娘,但不知她住在什么地方。最近遇到一个庚伯,承蒙他告诉我寇三娘的去向。儿去了后,三娘已投生到任侍郎家。儿急忙又赶到任家,将她强捉了回来。现在她已成为儿的媳妇,跟儿相处得很融洽,没什么苦恼。”过了会儿,一个女子从门外进来,打扮得非常漂亮,见了祝母,跪到地上拜见。祝生告诉母亲: “她就是寇三娘。”虽然儿媳不是活人,但祝母也觉安慰。祝生便吩咐三娘干活,三娘对家务事很不习惯,但性情柔顺,让人爱怜。二人就这样住下,不走了。三娘请婆母告诉自己娘家一声,祝生不同意。但母亲顺从了三娘的心愿,还是告诉了寇家。寇老夫妇听了大惊,急忙备车赶来,看那女子果然是女儿三娘,不禁失声痛哭。三娘忙劝住了。寇老太太见祝生家非常贫困,心里很是忧伤。三娘安慰她说:“女儿已成了鬼,还嫌什么贫穷呢?祝郎母子待我情义深厚,女儿已决意在这里安居了。”寇老太太又问:“当初和你一块施茶的那老婆婆是谁?”三娘回答说:“她姓倪。因她年老,自惭不能迷惑路人,所以求女儿帮助她。现在她已投生到郡城一个卖酒的人家。”三娘说完,又看着祝生说:“既然已成了我家的女婿,却不拜见岳父母,让我心里怎好过啊?”祝生忙向寇老夫妇拜下去。三娘便进了厨房,代婆母做饭款待自己的父母。寇老太太见了,不禁伤心。回去后,派了两个奴婢来供女儿使唤,又送了一百斤银子,几十匹布。此后还不时送些酒肉等物,祝母的生活因此稍稍富裕些了。寇家也时常让三娘回去省亲,住不几天,三娘就说:“家里没人,应早送女儿回去。”有时故意留住她不让走,三娘则总是飘然自回。寇老翁便替祝生盖了座大房子,很华丽宽敞。但祝生始终没到寇家去过。

一天,村里有个中了水莽毒的人,忽然死而复生了。大家争相传说,都认为是怪事。祝生说:“是我让他又活过来的。他被水莽鬼李九所害,我替他将李九赶走了,才救了他。”母亲说: “你怎么不找个人替自己呢?”祝生说:“儿最恨这些找人替死的水莽鬼,正想将他们全部赶走,自己又怎肯做这种害人的勾当!况且,儿侍奉母亲最快乐,不想再投生。”从此后,凡中了水莽毒的人,都备下丰盛的宴席,到祝家祈祷,无不灵验。

又过了十几年,祝母死了。祝生夫妇非常悲痛,但不接待来吊丧的客人,只命儿子穿着丧服,代为尽礼。埋葬母亲后,又过了两年,祝生为儿子娶了媳妇。新媳妇就是任侍郎的孙女。起初,任侍郎的爱妾生了个女孩,仅几个月就死了。后来任侍郎听说了三娘投生自己家被祝生捉回这件奇异的事,便驱车赶到祝家,认祝生为女婿。到现在,任侍郎又将孙女嫁给了祝生的儿子,两家更加来往不断。

一天,祝生对儿子说: “上帝因为我有功于人世,任命我做‘四渎牧龙君’,现在就要走了。”一会儿,便见院子里有四匹马,驾着一辆黄帷车,马的四肢上布满了麟甲。祝生夫妻盛装而出,一同上了车。儿子和儿媳都哭着拜倒在地。瞬间,车马便无影无踪了。同一天,寇家也见女儿来到,拜别父母,说的也和祝生说的一样。母亲哭着挽留她,三娘说:“祝郎已先走了!”出门后一下子就不见了。

祝生的儿子名叫祝鹗,字离尘。他请求寇家同意后,将三娘的骸骨与祝生合葬了。


\subsection{1.2.15   造 畜}
\label{\detokenize{p00_u5176_u5b83/_u767d_u8bdd_u804a_u658b_u5fd7_u5f02:id60}}
装神弄鬼欺骗人的巫术,可以说五花八门,不止一种。有的巫术,是以美味作诱饵,引诱你吃下去,便会神志不清,身不由己地跟着他走,这俗称“打絮巴”,江南一带叫“扯絮”。小孩无知,常常受骗上当,深受其害。还有一种巫术能把人变成牲畜,称为“造畜”。这种巫术江北一带很少见,黄河以南常有。

一天,扬州某旅店中,进来一个人,牵着五头驴,顺手拴在马厩下,嘱咐店伙计说:“我一会儿就回来,”并嘱咐:“不要给它们水喝。”说完就出去了。那些驴被太阳晒得暴躁不安,又踢又叫。店主人就把它们牵到阴凉处。驴一见水,挣扎着奔过去,店主就让驴饮足。转眼工夫,见驴在地上打滚,尘土飞扬中,立即变成了妇人。店主非常惊异,问那妇人是怎么回事,妇人舌根发硬,说不出话来。店主忙将妇人藏到屋里。一会儿,驴的主人回来了,把牵来的五只羊又拴到院子里。发现驴不见了,便惊慌地询问店主。店主忙上前拉他坐下,又命人端上饭菜,宽慰说:“你先吃饭,驴马上就来了。”店主出去,让羊饮足水后,一打滚,又全都变成了小孩。于是将此事偷偷地告到郡里。官府立即派人捉拿住那巫士,一顿乱棒便将他打死了。


\subsection{1.2.16   凤 阳 士 人}
\label{\detokenize{p00_u5176_u5b83/_u767d_u8bdd_u804a_u658b_u5fd7_u5f02:id61}}
凤阳县有个读书人,要出远门求学,临行前对妻子说:“我半年后就回来。”没想到他一去十多个月,竟音讯全无。妻子天天思念着他。

一夜,妻子上床躺下后,见皎洁的月光透过窗纱,照射进屋内,不禁勾起了她对丈夫的思念之情。正在辗转床头,难以入睡时,忽然有个头插珠花、身穿绛色裙子的美丽女郎,掀开门帘走了进来,笑着问她道:“姐姐,你莫不是想见见你丈夫吗?”妻子急忙答应着起床,女郎便请她跟自己走。妻子怕路远难走,女郎请她不要担忧。出门后,女郎挽着妻子的手,踏着月色,一起往前走去。约走了几十步,妻子见女郎走得非常快,自己又走不动,便喊她等等,要回去换双鞋子。女郎扶她坐在路边,从自己脚上脱下鞋子,借给她穿上。妻子大喜,穿在脚上竟很合适。于是,又起身跟着女郎,大步如飞地继续赶路。

不一会儿,就见丈夫骑着一头白骡子迎面走来。他看见妻子,大吃一惊,急忙下骡问道:“你要去哪里?”妻子说:“正要去找你呢。”丈夫又问那女郎是谁,妻子还没来得及回答,女郎已捂着嘴笑着说:“先不要问了吧。娘子奔波了一夜,很不容易;郎君又星夜赶路,想必人和牲口都很累了。我家距这里不远,请你们前去住宿,明天一早再走不迟。”夫妻四下一看,见几步外就有一个村庄,三人便同行,进入村内一个院子里。女郎喊起已经睡下的奴婢,做饭招待客人。又说:“今晚月色明亮,我们不必点灯了,就坐在花台上的石凳上吧。”丈夫将骡子拴在房檐前的一根大柱子上,就坐下了。女郎对妻子说:“我的鞋子大,可能不大合脚,路上穿着很不舒服吧?你回去时有骡子骑了,请把鞋还给我吧。”妻子连声道谢,把鞋还给了她。一会儿,酒菜便摆了上来。女郎斟上酒说:“你们夫妻久别重逢,今晚团聚,我借这杯薄酒,表示庆贺!”丈夫也端起酒来回敬。主客欢声笑语,杯盏交错,十分投机。渐渐地,丈夫一双眼老盯着女郎,频频用话挑逗她;对久别重逢的妻子,却一句亲热的话也没有。女郎也秋波送情,娇笑着说些情意脉脉隐诲的话。妻子只好默默地坐着,装出听不懂的样子。

又过了很久,丈夫和女郎酒意更浓,说的话越发轻薄起来。女郎又拿起个大杯劝客,丈夫推辞说喝醉了,女郎更加苦劝。丈夫便笑着说:“你如肯为我唱支曲子,我就喝了!”女郎答应,随即以象牙板拨弄琵琶,唱道:“黄昏卸得残妆罢,窗外西风冷透纱。听蕉声,一阵一阵细雨下。何处与人闲磕牙?望穿秋水,不见还家,潸潸泪似麻。又是想他,又是恨他,手拿着红绣鞋儿占鬼卦。”唱完,笑着说:“这是市井里巷中的俗调,不配让你听。因现今流行这种调子,所以姑且模仿模仿。”话声娇滴滴的,神态更加风骚。丈夫神魂颠倒,再也控制不住自己。一会儿,女郎假装醉了,离席走开;丈夫也站起身,尾随着她去了。过了很久,二人还没回来。伺候三人的奴婢又累又困,都趴在廊下睡着了。只剩下妻子一人坐着,孤孤单单的,心里又羞又气,愤懑不堪。想要逃走,但夜色茫茫,又不记得来时的路。心中踌躇着,不知如何办好,便站起身,偷偷去看二人在干什么。刚走近窗子,便从屋里隐约传来男女狂荡的声音。又细听了听,只听见丈夫平时和自己在床上时说的亲热话,这时正向那女郎尽情倾吐,不禁气得手直哆嗦,心里发颤。可又没办法阻止他们,自己真恨不得跑出门去,跳到沟里死了算了!

妻子气愤地刚走出门外,忽然看见弟弟三郎骑着马走来。三郎看见姐姐,急忙跳下马来询问怎么了。妻子向弟弟诉说了事情的经过。三郎大怒,立即跟姐姐回去,径直冲进那女郎家里。只见屋门紧闭,里面仍在传出咕咕哝哝的情话。三郎举起块斗大的石头,往窗子上猛力砸去,窗格子被砸碎,石头直飞了进去。只听屋里大声喊道:“郎君脑袋破了,怎么办!”妻子听了,惊愕万分,大哭起来,埋怨弟弟说:“我没有要你杀死我丈夫。现在可怎么办?”三郎瞪着眼说:“你呜呜哭着求我来,刚替你出了心中这口恶气,你就又护着丈夫,怨怪弟弟!我不愿供你支使!”说完返身就想走。妻子急忙拉住他的衣服,说:“你不带我一起走,要到哪里去?”三郎一挥手,把姐姐推倒在地上,自己脱身而去。妻子一下惊醒,才知道是做了个梦。

第二天,丈夫果然回来了,骑着匹白骡子。妻子见了非常惊疑,心里嘀咕着嘴上没说。过后谈起来时,丈夫那一夜也做了个梦,梦中的所见所遇,和妻子做的梦完全一样,二人大感惊骇。不久,三郎听说姐夫从远方回来,也来探问。说话间,三郎对姐夫说:“昨夜我梦见你回来,今天果然就回来了,太奇怪了!”姐夫笑着说:“幸亏没被石头砸死!”三郎惊愕地询问这话是什么意思,姐夫便将自己和妻子做的梦告诉了他。三郎大吃一惊,原来这晚三郎也梦见姐姐向他哭诉,自已往窗子上抛石头。三个梦完全相符,只是不知那女郎是什么人。


\subsection{1.2.17   耿 十 八}
\label{\detokenize{p00_u5176_u5b83/_u767d_u8bdd_u804a_u658b_u5fd7_u5f02:id62}}
新城人耿十八,病势垂危,自知将不久于人世。弥留之际对妻子说:“早晚之间就要永别了,我死后,改嫁、守寡由你选择,请说明你的打算。”妻子听了默不作声。耿十八坚持要她表态,说:“守寡当然好,再嫁也是人之常情。趁我还活着把事情挑明,有什么妨碍!马上与你诀别,你守寡,我感到安慰;你决意嫁人,我也就不再牵肠挂肚,了结了这份心事!”妻子神色凄然地说:“咱家穷得叮当响,你活着都吃不上饭,死后,我指望什么守寡啊?”耿十八听了,猛地抓住妻子的胳膊,恨恨地说:“你的心真狠啊!”随后便咽了气,可那死死抓住妻子胳膊的手却不松开,吓得她大喊大叫。家里人闻声赶来,连忙让两个有力气的人使劲将耿十八的手掰开,才将他妻子的胳膊抽出来。

耿十八不知自己已经死了,信步走出家门。见门前有十几辆小车,每辆车上坐着十个人,每个人的名字都写在方纸上,贴在车上。一个押车的人看到耿十八,督促他快上车。耿十八上车后,见已经坐着九个人,加上自己正好十人。又见名单上自己的名字写在最后。听到车子吱吱咯咯地很响,声音刺耳。自己也不知要去什么地方。

转眼来到一个场所,听见有人说:“这里是思乡地。”听到这名字,耿十八疑惑不解。又听见押车人互相窃窃私语说:“今天铡了三个人。”耿十八越发骇怕。再仔细听听他们说的,都是些关于阴曹地府的事情,他这才恍然大悟,自言自语地说:“我这不是变成鬼了吗?”立刻想到家中倒没有值得挂念的事,唯独老母年事已高,妻子嫁人后,撇下她无人侍奉。想到这里,不由难过得泪水涟涟。

走着走着,忽看见前面有座数丈高台,游人很多。他们蓬头垢面,身带枷锁,哭着叫着,上去又下来,听人说这就是“望乡台”。众人来到这里,纷纷从车上跳下来,你争我抢地往台子上爬。押车人用鞭子抽打他们,禁止他们往台子上爬,唯独轮到耿十八时,催他上去看看。耿十八一气登了几十级台阶,才到台子的最顶端。抬头一看,自家的庭院、房屋如在眼前。但室内却看不清楚,好像是烟笼雾绕似的。耿十八触景生情,心里顿感凄恻难受,不能自制。回头看时,一个短衣打扮的人站在身边,询问耿的姓名。耿如实相告。那人自称是东海的匠人。他见耿十八伤心的样子,就问:“你有什么放心不下的事吗?”耿十八就把事情的始末,告诉了他。匠人与耿十八商量,想跳台逃跑。耿十八胆小,怕小鬼来追拿他。匠人再三说没事。耿又怕跳台时跌着,匠人就让他学自己的样子,便率先纵身跳下去。耿十八果然也随着跳下,竟安然无恙地着了地,更庆幸无人察觉。看见来时乘坐的车仍停在台下,两人急忙拼命奔逃。刚跑出几步,耿十八忽然想起自己的名字还贴在车上,怕被人发现按名捉回,连忙返回车旁,用手指沾上唾液把自己的名字擦去,这才放心地猛跑。

两人跑得张口气喘,也不敢歇一歇。时间不长,就跑到了家。匠人把耿十八送到屋里,耿十八猛然看到自己的尸体,一下就苏醒过来,顿时感到精疲力竭,口干舌燥,急呼要喝水。家人大吃一惊,连忙给他端水来。耿十八一气喝了足足一大桶;随后就猛地站了起来,先是叩首作拜状,接着又到门外拱手作揖,回屋后就直挺挺地躺到床上不再动弹。家人被他怪异的行为弄懵了,怀疑他不是真活。然而再仔细观察一下,并没发现什么异样的地方。再靠到他身边询问,他才清清楚楚地说出事情的始末。问他:“你出门干什么?”回答说:“去和匠人告别。”又问他:“你怎么喝那么多水?”他回答说:“先是我喝,后是匠人喝。”家人喂他汤饭,不几天他就恢复了健康。经过这事,耿十八很讨厌、鄙视他的妻子,再也不与她同床共枕了。


\subsection{1.2.18   珠 儿}
\label{\detokenize{p00_u5176_u5b83/_u767d_u8bdd_u804a_u658b_u5fd7_u5f02:id63}}
江苏常州有一富翁,名叫李化,田产很多,但五十多岁还没有儿子,只有个女儿名叫小惠,长得如花似玉,老两口十分疼爱她。不料小惠才十四岁就得急病死了,家中更显得空空荡荡,冷冷清清。李化便纳婢为妾,一年多后生了一子,李化待如掌上明珠,给他起名叫珠儿。珠儿渐渐长大,出落得结实英俊,一表人材。然而生性痴呆,五六岁了还五谷不分,说话也结结巴巴不清楚。即便如此,李化也不介意,反而越加疼爱。

有一年,城里来了个化缘的瞎和尚,他能测知人家闺阁中的隐私,于是人们都惊讶地以为他是神仙。和尚还扬言能给人以生死祸福。他化缘时点名向人要成百上千的钱,没有一个敢违抗的。一天,和尚向李化索要一百串钱。李化很为难,给他十串,和尚嫌少不要;李化渐渐加到三十串,和尚声色俱厉地说:“必须给我一百串钱,少一文也不行!”李化也很生气,收起钱来就走了。和尚忿恨地说:“不要后悔,不要后悔!”不一会,珠儿突然心口剧疼,在床上滚来滚去,手抓脚蹬,面如灰土。李化害怕,忙带上八十串钱去拜求和尚,恳请救珠儿一命。和尚冷笑着说:“你拿出这么多钱,太不容易了!我一个瞎和尚能有什么法子呢?”李化无奈,回到家里一看,珠儿已经死了。李化很悲痛,写了状子告到县官那里。县里派人将和尚拘捕审讯。和尚极力抵赖狡辩,县官就命衙役像擂鼓似地揍了他一顿。又命人搜身,从他身上搜出了两个木人,一口小棺材,五面小旗子。县官大怒,出示和尚的罪证。和尚这才害怕,连连磕头求饶。县官不听,命令手下人一顿乱棒将他打死了。李化叩首拜谢了县官,回了家。

李化到家,时已黄昏,正与妻子坐在床上说话。忽然一个小孩急急忙忙地走进屋里,对他说:“阿翁,你为什么走得那么快?我拼命追也追不上。”细看他的长相,大约有七八岁。李化一惊,刚要问他,就见那小孩若隐若现、如烟似雾,转眼间已爬到床上。李化连忙将他推下床去,落地时一点声音也没有。小孩说:“阿翁,你这是干什么?”转眼间又爬到了床上。李化很害怕,拉着妻子就往外跑。小孩紧跟在他俩的后面,“阿翁”“阿婆”不停地叫喊。李化跑到他小妾的屋里,急忙关好门。回头看时,小孩已站在跟前。李化战战兢兢地问小孩要干什么,小孩回答说:“我是苏州人,姓詹。六岁那年父母双亡,哥嫂不容我,撵我到外婆家去住。一次在门外玩耍,被和尚施妖术迷住,把我带到桑树下杀害了。后来就强迫我供他驱使。从此,我冤沉九泉,不能超生。幸亏阿翁为我昭雪,我心甘情愿给您做儿子。”李化说:“人与鬼不是一路人,怎么能共同生活呢?”小孩说:“只要给我一间小屋,放上床及被褥,每天浇上一碗冷粥,其它就没事了。”李化答应了他的请求。小孩很高兴,独自在屋里住下来。早晨起床后,出入各屋,如同李化的亲生儿子一样。

一天,小孩听到李化的妾哭孩子的声音,就问:“珠儿死了几天了?”回答说七天。小孩说:“天气寒冷,尸体应该不会腐烂。派人去扒开坟看看,如果没损坏,我就可以借尸还魂,再活过来。”李化听后很高兴,拉着小孩到了珠儿的坟地。掘开坟,开棺一看,尸体完好。李化正在悲伤时,回头一看,那小孩不见了。李化心中很奇怪,就让人抬回珠儿的尸体。到家后,刚把珠儿的尸体放到床上,就见珠儿的眼珠能转动了;不一会便叫着要水喝;喝完水后出了一身汗,汗尽后竟站了起来,全家人都为珠儿的复活高兴。又加上他变得非常聪明,与以前大不一样。只是到了夜间,珠儿又直挺挺地躺在床上,毫无气息。翻转他的身子时,也闭着眼睛和死人一样。众人大吃一惊,以为他又死了。天将明,珠儿才做梦似地清醒过来。大家问他是怎么回事,他回答:“以前我跟着妖和尚时,有我们两个人,那一个叫哥子。昨天追阿父没追上,是因我在后边与哥子告别呢!他现在阴间,给姜员外作义子,也很自在快活。昨夜来邀我玩耍,刚用白鼻子黄马把我送回来。”李母问:“在阴间见到珠儿了吗?”回答说:“珠儿已转生了。他与阿翁没父子缘分,不过是替金陵的严子方来讨回百十吊钱的债罢了。”当初,李化曾到金陵跑买卖,欠了严子方一笔债未还,严就死了,这事无人知道。李化听后心中很震惊。李母问:“孩子,见到你惠姐没有?”回答说:“不知道。再去时,一定打听打听。”

过了两三天,小孩对李母说:“惠姐在阴间很好,嫁给了楚江王的小公子,打扮得珠翠满头,一出门就有百十人前呼后拥地开路。”李母说:“为什么不回娘家来看看呢?”回答说:“人一死,就与生前的父母没有关系了。若是有人细细地讲述生前的事,才能使她猛地想起来。昨天我托了姜员外,经他介绍见到了惠姐。姐姐叫我坐到她的珊瑚床上。我就把父母想念的话说给她,可她像睡着了一样。我又说:‘姐活着时,喜欢绣并蒂花。剪刀刺破了手,血滴在绫子上,姐姐就把它绣成红色的云霞。至今母亲仍将它挂在床头的墙上,看见那绫子,就想念你。姐姐您难道忘了吗?’说到这里,姐姐才想起生前的事,凄惨地说:‘见了丈夫我一定告诉他,我要回家探望母亲。’” 李母问什么时候来,回答说不知道。

一天,小孩对李母说: “姐姐就要到了!随从很多,要多准备些酒饭。”一会儿,他又跑回屋里说:“姐姐来了!”将坐椅搬到堂屋,并说:“姐姐先坐下休息一会,不要太伤心。”可是别人却什么都看不见。小孩领着家人在门外焚纸祭酒,回来说:“随从车马先暂时回去了。姐姐说:‘以前我盖的绿绵被,曾被烛花烧了个豆粒大小的洞,还在吗?’”李母回答:“在。”便开开箱子找了出来。小孩说:“姐姐让我把它放在以前她住的闺房中。她现在累了,要休息一会。明天再与母亲说话。”

东邻赵家的女儿,是先前与小惠在一起绣花的好朋友。这一夜,忽然梦见小惠戴着头巾,身着紫色披肩来看她,音容笑貌与生前一样。对赵女说:“我现在已不在人间了,与父母见面不亚于河山相逢。我想借你的形体去与家人聚谈,你别害怕。”天刚亮,赵女正和母亲说话,忽然扑到地上,昏了过去。过了一会才慢慢醒过来。对赵母说:“小惠与婶婶才分别几年,你竟满头白发了!”赵母惊骇道:“你得了疯病吗?”女儿拜别赵母走了出去。赵母猜知有别的缘故,就尾随着她。一直走进李家,赵女进屋就抱住李母大哭。李母惊慌失措,不知缘由。赵女说:“我昨天回来,很疲劳,没顾上与母亲说句话。女儿不孝,半路上离弃二老,让你们相念,我怎么才能赎罪呢?”李母马上明白了,于是痛哭,随即问道:“听说孩儿如今享受荣华富贵,我心里很高兴。但你是王爷家中的人,怎么能来呢?”赵女说:“郎君与我感情很好,公婆也很疼爱,都不嫌我长得丑陋。”小惠生前习惯用手托下颌,赵女边说边做这种姿势,神情动作酷似小惠生前的样子。过了不长时间,珠儿跑进来说:“接姐姐的人到了!”赵女起身拜别李母,哭着说:“孩儿走了。”说完就扑倒在地上昏了过去。过了一会赵女才苏醒过来。

又过了几个月,李化生了病,病情日益加重,求医吃药都不见效。小孩说:“这看来是早晚的事了,恐怕没有救了。两个鬼坐在床头,一个手执铁杖,一个拿着条四五尺长的麻绳。我白天黑夜地哀求他们,他们也不走。”李母哭着给李化准备寿衣。黄昏时,小孩跑进来说:“家中的闲杂人和妇女都暂回避一下。姐夫来看望阿翁了!”待了一会,小孩拍掌大笑。李母问他笑什么,他说:“我笑那两个小鬼。听说姐夫来了,吓得躲到床下像个缩头乌龟似的。”又过了不多时,见小孩对着空中寒暄了一番,并向姐姐问好。接着拍手笑道:“我苦苦哀求两个鬼离开,他们不走,现在真大快人心!”说完走出门去,又折身回来说:“姐夫走了,两个鬼被锁在马脖子的皮带上带走了。阿父的病马上就会好。姐夫说:他回去就求大王,为父母乞求百年寿限!”全家人欢天喜地。夜间,李化的病果然有了好转,几天后就康复了。

李化病好以后,请了个教师教小孩读书。他很聪明,十八岁就考进县学,还能说些阴间的事情。见到乡亲中有生病的人,都能指出鬼在什么地方为害。用火烤,往往病人就好了。后来他突然得了场急病,肌肤青紫。自已说:因为鬼神怪他泄露了秘密,以示惩罚。从此,他再也不说阴间的事了。


\subsection{1.2.19   小 官 人}
\label{\detokenize{p00_u5176_u5b83/_u767d_u8bdd_u804a_u658b_u5fd7_u5f02:id64}}
某太史,忘了他的姓名。一天,他白天躺在书房里,忽然,一个小仪仗队,从屋子一角走出。马大如青蛙,人细如手指。小小的仪仗队由数十人组成。一个官,头戴乌纱帽,身穿绣花袍,坐在二人抬的轿上,纷纷出门而去。太史心中觉得奇怪,怀疑自己睡眼朦胧看花了眼。可接着又见一小人,返回屋来,手里携着个拳头大小的包,一直走到床下,自己介绍说:“我家主人备有薄礼,敬献太史。”说完,对着太史站着,却不去打开包拿出东西。稍待了一会,又自己笑着说:“小小礼物,想太史也没什么用,不如送给小人。”太史点了点头,小人高兴地携着包走了。以后再没见过他。可惜太史当时心里有点害怕,也不曾问小人是从哪里来的。


\subsection{1.2.20   胡 四 姐}
\label{\detokenize{p00_u5176_u5b83/_u767d_u8bdd_u804a_u658b_u5fd7_u5f02:id65}}
尚生,是泰山人,平日他独自一人在书房里读书。一个秋高气爽的夜晚,天上繁星闪烁,明月当空高照。他一个人徘徊在月影花阴之下,一时之间,心头遐想万千。忽然一个女子跳墙进来,对尚生说:“秀才,何以这样沉思呀?”尚生抬头一看,见这女子容貌美丽,犹如仙女。他十分惊喜,急忙拥抱着就进了屋。二人亲密温存之至,女子自我介绍说:“姓胡,叫胡三姐。”尚生问她的住处,女子笑而不答。尚生也就不再追问,只希望永远欢好罢了。自此以后,胡三姐夜夜来会,从不间断。

一夜,尚生与胡三姐对坐灯前。尚生目不转睛地看着她,越看越觉三姐美。三姐笑着说:“你眈眈地看着我做什么?”尚生回答说:“我看你长得像红叶碧桃,就是看一夜也看不够。”三姐说:“我长得这样丑你都看不够,若见到我四妹,还不知神魂颠倒到什么样子?”尚生听了更加动心。恨不能马上见到四姐。接着就下跪请求三姐介绍四姐来相见。

第二天晚上,三姐果然领着四姐来了。四姐年纪十五六岁,长得既像露水下的荷花,又像雾润下的杏花,嫣然含笑,妩媚动人。尚生一见,欣喜若狂,急忙请她坐下。这时三姐与尚生一起说话,四姐只是低头含羞,用手摆弄身上的绣带,一语不发。一会儿,三姐起身告辞,四姐也要一同走。尚生一手拉住四姐不放,眼睛看着三姐说:“请帮着说句话吧!”三姐说:“狂郎性急了,就请妹妹稍待一会儿吧!”四姐没说话,三姐便一人走了。

尚生与四姐极为欢好,接着就互相倾吐自己的生平,越说越知己。四姐自己说是狐女,尚生贪恋她美貌,也不觉怪异。四姐说:“三姐也是狐女,但很狠毒,她已经杀了三个人了,只要被她引诱上了钩,没有不死的。承蒙你这样爱我,不忍心看着你死去,劝你早日与三姐断绝。”尚生听了很害怕,请求四姐想个办法。四姐说:“我虽是狐,但得到了仙人的正法。我写一张符贴在你寝室的门上,就可以使她退去。”随即写了一张符交给尚生贴在门上。天色将明时,三姐又来了,见了门上的符,恨恨地说:“小妮子真是负心人!倾爱新郎,就不想着引线搭桥的人了。你二人本来就有缘分,我又不嫉恨你们,何必这样对待我呢?”说罢就走了。

过了几天,四姐因为有事要到别处去,与尚生约好隔夜再来。这天尚生无事,就到野外闲逛。山下原来就有一个桷树林子,朦胧中忽然从林子里走出一个少妇。这少妇长得也很有风韵,走近尚生说:“你为什么每天恋着胡家姊妹?她们又不能给你一文钱。”说着拿出一贯钱交给尚生,说:“你先拿回去,买点好酒,我回去拿点菜肴来,今晚和你好好快乐快乐。”

尚生拿回钱来,按妇人的吩咐买了酒。稍呆一会儿,少妇果然来了,把烧鸡、火腿放在桌子上,抽出自带的小刀割成小块,就与尚生饮酒说笑,欢乐非常。酒后二人就息灯上床,这妇人非常淫荡。直到天明起床,她正穿鞋时,忽听有人声,侧耳一听,人已走入帐幕内,一看,是胡家姊妹。妇人一见仓惶逃窜,慌忙中掉了一只鞋在床上。胡家二女追着骂道:“骚狐!怎敢与人睡觉!”说着追了出去,过了一会儿才回来。四姐抱怨尚生说:“你真没出息,竟与骚狐匹配!我不能再亲近你了!”怒气冲冲地就要走。尚生既羞愧又惶恐,连连认错,态度恳切。三姐又从中和解,四姐才渐渐消了气,又和以前一样相好了。

一天,一个陕西人骑着头驴来拜访尚家。尚生的父亲开门后,那人就说:“我是来寻找妖精的,已经找了很长时间了,近来才知道在你这里。”尚父因听这人说话奇怪,就问详情。那人回答说:“小人天天在外,周游四方,一年中八九个月不在家。我弟弟在家被妖精蛊惑而死。我回来听说非常气愤,决心找到妖精除掉它!我已奔走了几千里路了,一直没找到踪影。现在发现在你家,不除掉它,你家也将有人继我弟弟之后被害死!”

这一段时间里,尚生与狐女来往密切的事,尚父也略有耳闻。听陕人这一说,心里很害怕,就请陕人进屋,要求快作法除妖。陕人拿出两个瓶子,摆在地上,念了很长时间的咒语,就见有四团黑烟分别钻入两个瓶子里。陕人说:“全家都到齐了!”接着拿出猪膀胱蒙住瓶口,捆封得严严实实。尚父很高兴,执意留陕人吃饭。尚生知道这一切,心里很觉可怜。他走近瓶子偷看,就听四姐在瓶中说:“坐视不救,真是负心人!”尚生更加动心,急忙去开瓶封,但结子很牢固,解不开。四姐又说:“不用费劲了,只要放倒坛上的大旗,用旗上的钺头刺破猪膀胱,我就能出去了。”尚生照办了,就见一丝白气从瓶中冒出,升往天空中去了。陕人饭后出来,见坛上大旗放倒在地,大惊说:“逃走了!这一定是公子干的。”他拿起瓶子摇了摇,又贴近耳朵听了听说:“幸好只跑了一个!那一个活该不死,可以赦免她!”说罢,带上瓶就告辞了。

后来,尚生在地里看着佣人割麦子,远远看见四姐坐在树下,尚生走过去拉住她的手安慰她。四姐说:“别后十多年了,今大丹已炼成,但还是没有忘记郎君,因此再来看看你。”尚生要求她一起回家,四姐说:“我已今非昔比了,不能再染红尘,以后再见吧!”说完,就不见了。

又过了二十多年,尚生正独自一人在家,见四姐从外面来了,便高兴地迎接她。四姐对尚生说:“我现在已列入仙人籍了,本来不应再下凡尘,但感念你的恩情,特来告诉你的去世之期。你可早作准备,也不要悲伤,我一定想法度你为鬼仙,也不会受苦。”说罢即辞别而去。

到了四姐说尚生将死的那一天,尚生果然去世了。尚生是我朋友李文玉的亲戚,这事李文玉亲自见过。


\subsection{1.2.21   祝 翁}
\label{\detokenize{p00_u5176_u5b83/_u767d_u8bdd_u804a_u658b_u5fd7_u5f02:id66}}
济阳祝家村,有位姓祝的老翁,五十多岁这年得病去世,家里人进屋准备丧服时,忽然听到祝翁急切的呼喊声。众人都跑到停棺的地方,见他已经复活,便都高兴地向他问长问短。但他只对老妻说:“我刚去的时候,决心不再回来。走了几里路,又一想,撇下你这把老骨头在孩子们手里,冷热吃穿都要依靠他们,也没什么活下去的乐趣,不如跟我一起走。因此才又回来,想叫着你一起走。”外人都以为他刚苏醒过来在说胡话,都不相信。老翁又把这话重复了一遍,他妻子说:“这样办倒也很好。但我正活着,怎么就能死了呢?”祝翁一挥手说:“这不难,家中的日常俗务,可赶快去办理完。”他妻子只笑不走。祝翁又催她,她才走出门去。拖延了几刻钟,回来哄他说:“一切都料理好了。”祝翁又命她快去打扮一下。老妻不肯去,他催促越急。她不忍心违背了他的意愿,便穿上裙子打扮好出来。闺女媳妇们见她这副打扮,都偷偷地笑。祝翁把头往枕边移了移,用手拍着枕头另一端,示意老妻躺下。老妻说:“孩子们都在这里,咱俩直挺挺地躺着,是什么样子?”老翁用手捶打着床说:“一块死有什么可笑的!”儿女们见祝翁急得不行,就劝老妪暂照他的意愿办。于是老妪就与祝翁一个枕头躺下了,家人又都笑了起来。接着一看,见老妪忽然收敛了笑容,又渐渐合上了双眼,好久没有动静,像熟睡的样子。众人这才走近察看,见她肌肤已经冰凉,鼻子也没有气息。再试祝翁也是一样。大家这才震惊悲痛起来。

康熙二十一年,祝翁的弟媳在毕刺史的家里当佣人,这事她说得很清楚。


\subsection{1.2.22   猪 婆 龙}
\label{\detokenize{p00_u5176_u5b83/_u767d_u8bdd_u804a_u658b_u5fd7_u5f02:id67}}
猪婆龙产于江西,形状像龙,但比龙身子短,能横着飞,常飞出水面沿江岸捕捉鹅鸭吃。有时有人捉住一头,就把它杀掉,把肉卖给陈、柯两家。这两姓人家都是陈友谅的后裔,从祖辈传下来就吃猪婆龙肉,别姓人家不敢吃。

一天,一个客人从江的西边来,捉到一头猪婆龙,把它绑在船上。这艘船停在钱塘江边,因为没把猪婆龙绑结实,被它跑掉,一头扎进江里。一转眼的工夫,江里波浪涛天,船立刻翻了。


\subsection{1.2.23   某 公}
\label{\detokenize{p00_u5176_u5b83/_u767d_u8bdd_u804a_u658b_u5fd7_u5f02:id68}}
陕西某公,是辛丑年间的进士,能记住前辈子的事。常对人说,他前生是个读书人,中年就死了。死后见阎王审判案子,大殿前有沸开的油锅,和世上传说的一样。大殿东边,扎着好几个架子,架子上搭着猪、羊、狗、马等牲畜的毛皮。掌管生死簿的官吏念着人名,念到某人罚作马,或者是罚作猪,小鬼就给他脱光了身子,从架上拿下这种皮来给他披上。

当簿吏念到某公时,阎王爷说:“应罚他为羊。”于是小鬼拿一个白羊皮来给他披在身上。簿吏这时说:“这人曾救过一个人的命。”阎王听了,再复查一下记录,说:“免了吧!他作恶虽多,但救人一事可以赎罪。”小鬼又给他脱去羊皮,可羊皮这时已经粘在身上了,脱不下来。于是两个鬼拉着他的两臂,按住他的胸膛,硬是向下脱,使他疼痛难忍。那羊皮一块一块地扯下来,到底也没有脱干净,肩膀处仍留下一小片羊皮,有巴掌大小。某公出生后,肩膀处仍长出一丛羊毛,剪了,也还长。


\subsection{1.2.24   快 刀}
\label{\detokenize{p00_u5176_u5b83/_u767d_u8bdd_u804a_u658b_u5fd7_u5f02:id69}}
明代末年,济南府属下各地有很多强盗,每个县都设置军队,捕到强盗就杀掉。章丘县的强盗尤其多。这个县的官军中有一个士兵,佩带的刀特别锋利,杀人不用费劲。一天,官军捕获了十几个强盗,全部押赴法场斩首。其中一个强盗认得这个士兵,便犹豫地凑上前去说:“听说您的刀最快,砍头不用砍第二次。求您杀我吧!”士兵说:“好吧。你跟在我身边,不要离开我。”强盗跟着士兵来到刑场,士兵一刀砍去,强盗的脑袋骨碌一声掉下来,滚到数步之外,一边在地上打着转,嘴里还大声称赞道:“好快的刀!”


\subsection{1.2.25   侠 女}
\label{\detokenize{p00_u5176_u5b83/_u767d_u8bdd_u804a_u658b_u5fd7_u5f02:id70}}
南京有一个姓顾的书生,博学多才,能写会画,他家里却非常贫穷。因为母亲年老,顾生不忍离开老母膝下,只好天天给人家画画,得点钱维持生计。顾生已经二十五岁了,还没娶妻。他家对门是一所空房子,一个老太太和一个少女借住在里边。因为她们家里没有男人,所以也从没问过她们是什么人。

一天,顾生偶然从外头回家,见对门的女子从他母亲房里出来,约有十八九岁,长得秀丽风雅、世间无比。女子迎头碰见顾生,也不太回避,但神情冷峻威严。这女子走后,顾生走到母亲房里询问母亲,母亲说:“刚才来的是对门的女郎,来向我借剪刀尺子。她说她家也只有一个母亲,别无他人。我看这个女孩子不像穷家人,问她为什么不嫁人,她说母亲年老无人侍奉。明天我过去看看她母亲,顺便暗示一下。如果她们要求条件不高,我们结亲,你可以代养她的母亲。”

第二天,顾母去女郎家里拜访,见她母亲是一个聋老太太,看她家里穷得没有隔夜粮。问她们靠什么维持生活,回答说靠女儿做针线活。顾母慢慢试探着说了一起过日子的想法,老太太似乎同意,但女儿默默不语,意思不太愿意。顾母没再说什么,就回家了。回到家里仔细想了一下当时的情况,对顾生说:“莫不是女子嫌我们家穷?这孩子为人严肃,不说也不笑,长得虽艳如桃李,性情却冷如冰霜,真是个怪人!”母子俩猜疑了一会儿,也就罢了。

一天,顾生坐在房子一端作画,有个少年来求他画幅画。这少年长得很漂亮,但样子很轻薄。顾生问他从那里来,他说是“邻村”的。自此以后,每三两天就来一次。稍稍熟了点,就渐渐与顾生说笑。顾生拥抱他,他也不太拒绝,随即关系暖昧起来。从此,两人来往更加亲热。一次,那少年见女郎走过,用眼盯着她走远,问顾生是谁。顾生说是“对门的女子”,少年说:“这女子长得这么漂亮,神情却非常可怕!”一会儿,顾生到母亲屋里。他母亲说:“刚才女子来借米,说是断炊已经一天了。这个女孩子很孝顺,家里穷得可怜,应该多少周济她一些。”顾生听了母亲的话,就背一斗米去她家,并转告了母亲的话。女子留下来,也不说感谢的话。自此,女子也常到顾生家,见到顾母做衣服或鞋子,她就拿过来替顾母做,出出进进,帮着操持家务,就像顾家的儿媳妇一样。顾生看到这样,越发感激女子。后来顾家每次得到别人送来的礼物,总是分一半给女子的母亲。而女子也仍不说感谢一类的客气话。

一次,顾母的阴处生病,疼得日夜喊叫,女子天天来看望她,给她擦洗换药,一天来三四次。顾母很是不安,但女子却从不嫌脏。顾母说:“唉!我们家到哪里娶到个像你这样的媳妇,早晚伺侯老身到死!”说罢就哭起来。女子安慰她说:“你儿子很孝顺,胜过我们孤女寡母几百倍呢!”顾母又说:“在我床头来来去去的,这岂是孝子能办得到的?况且老身已是暮年之人,早晚难保不入土,我所不放心的就是没有后代根苗。”她俩正说着,顾生进屋来。顾母哭着对儿子说:“我们欠姑娘的太多了,你不要忘记,要报答她的大恩大德呀!”顾生听了,向女子施礼感谢。女子说:“你照顾我母亲,我都没有谢你,你何必谢我呢?”于是顾生更加敬爱她。然而女子的举止一直很严肃,顾生一点也不敢轻易接近她。

一天,女子出门回来,顾生注视着她。她忽然一回头,向顾生嫣然一笑。顾生喜出望外,就跟在女子后面进了她家的门。顾生想亲近她,女子也不拒绝,欣然同意。事后,女子告诫顾生说: “这事只可一而不可再!”顾生没表示同意,就回了家。到了次日,顾生又约女子相会,女子非常严肃地连理也不理他就走了。此后,女子仍天天来顾家,天天相见,并不给顾生好话听、好脸色看。有时顾生说句笑话逗她,她就冷语拒绝。有一次,女子忽然在没有人的地方问顾生:“前几天常来的那个少年是谁?”顾生告诉了她。女子接着说:“那人举止行动,几次对我无礼!因为是你的朋友,没有理他。请转告他:要再对我无礼,他是不想活了!”到了当天晚上,顾生把女子的话告诉了那少年,并且告诫他说:“你要小心,她可不是好惹的!”少年说:“既然她不可侵犯,你怎么侵犯了她呢?”顾生表白并无其事。少年又说:“若是没有事,怎么男女之间不好说的话她都说给你呢?”顾生一时回答不上来。少年又说:“请你也转告她:不要装模作样!不然的话,我就到处给你们宣扬!”顾生听了很生气,怒形于色,那少年就走了。

一天晚上,顾生正一个人坐在屋里,女子忽然来了,笑着说:“我与你情缘未断,这岂不是天意!”顾生高兴得不得了,急忙把女子抱在怀里。忽然听到有脚步声,两人惊慌地起来,就见少年推门进来。顾生惊问:“你要干什么?”少年笑着说:“我来看贞洁的人呀!”又望着女子说:“今天不怪我了!”女子柳眉倒竖、脸面发红,一句话不说,急忙翻开上衣,露出一个皮囊,随手抽出一把匕首,闪闪发光。少年一见,吓得拔腿就跑,女子追出门外,四下一看,不见踪影。她把匕首往空中一抛,嘎嘎有声,一道亮光像长虹一样,接着就有一件东西“扑”地落在地上。顾生急忙用蜡烛一照,见是一只白狐,已经身首两处了,他大惊失色。女子说:“这就是你恋着的好朋友!我本来想饶了它,谁知他偏偏不想活!”便收了匕首放在鞘里。顾生又拉女子进屋,女子说:“刚才让妖精来扫了兴,请等明晚吧!”说罢出门就走了。

第二天晚上,女子果然又来了,二人便共同欢好。顾生问她有什么法术,女子说:“这不是你应该知道的,需要保密。泄露了,恐怕对你不利。”顾生又与女子商量嫁娶的事,女子说:“我们已经同床共枕,我又帮助你干家务,已经成了夫妻,还谈什么嫁娶呢?”顾生又说:“你是不是嫌我家穷?”女子说:“你家固然穷,难道我家富有?今晚相会正是可怜你穷呀!”临走时又对顾生说:“这种见不得人的事,不能次数太多。该来的时候我自然就来;不该来的时候,你强求也没有用。”以后两人遇到一起,顾生每每想引她单独说句话,女子每次都避开了。但是她来顾家缝衣做饭,料理家务依然如故,不亚于真正的媳妇!

又过了几个月,女子的母亲去世了,顾生竭尽全力帮助女子料理丧事。女子从此就一个人过日子。顾生想:女子一个人住在家里,可以随便去找她了。于是他就跳墙进了女子的院子,隔着窗子叫她。但喊了好几声,没有人答应。他看看门,门关得好好的,人却没在屋里。顾生暗想:女子可能与别人约会。第二夜又去探看,仍和昨晚一样,他便将一个玉佩放在窗台上回家了。隔了一天,顾生与女子在顾母房里相遇,顾生走出房来,女子也追了出来,对顾生说:“你怀疑我了?人都有自己的心事,有的不能随便告诉别人。今天不叫你怀疑,也不可能。但是有一件急事你得赶快帮我想法子。”顾生问是什么事,女子说:“我已怀孕八个月了,恐怕不久就要生了。我的身份还不清楚,只能给你生下来,但不能帮你抚养。你可秘密告诉你母亲,找一个奶妈,假说是抱了个小孩,不能说是我生的。”顾生答应后,告诉了母亲。他母亲笑着说: “这个姑娘真奇怪,娶她不愿意,却与我儿私自相好。”高兴地同意了他们的要求,只等生下孩子再说。

又过了一个多月,一连几天女子没有来顾家。顾母心里有疑虑,就去女子家探看。一看大门关得严严的,院里寂静无声。叫门很长时间,女子才蓬头垢面从屋里出来,请顾母进了屋,又把门关上。顾母进屋一看,一个小婴儿已在床上了。顾母惊讶地问:“生了多少时候了?”回答说:“三天了。”顾母解开小褥子一看,是一个小男孩,宽宽的脑门,非常可爱。顾母高兴地说:“我的儿,你已经为老身生孙子了。今后你孤单一人,哪里去安身?”女子说:“我还有一件心事未了,不能告诉母亲。等夜里没有人时,可把小儿抱去。”

顾母回到家里,与儿子说了这一切,心里都暗暗觉得奇怪。到了夜里,便把婴儿抱回去了。又过了几天,夜半三更时,女子忽然推开顾生的门进来,手里提着一个皮口袋,笑着对顾生说: “我的大事已办完了,从此咱们就分别了。”顾生急着问是什么原因,女子说:“你帮我奉养母亲的恩德,我一时一刻不会忘记。以前我曾对你说过‘可一而不可二’,是说报答你的恩情不在于与你同居,而是因你家贫不能娶妻,想给你留下后代根苗。本来希望一次就能怀孕,谁知又来了月经,所以破戒又与你同房了一次。今日既已报答了你的大恩大德,我的心事也已了却了,没有什么遗憾了!”顾生问:“皮袋中是什么东西?”回答说:“仇人的头。”打开一看,血肉模糊。顾生非常惊慌,细问原因。女子说:“过去一直没有与你说,就是因为事情机密,怕走漏了风声。今天大事已经成功,不妨告诉你。我本是浙江人,父亲官居司马,为仇人陷害,被抄斩满门。我背着母亲逃了出来,隐姓埋名三年了。之所以没有立即报仇,就是因为有老母在世。后来老母去世,却又有一婴儿在肚内,因此又推迟了一些时间。那些夜晚我没在家,是去探探仇人家的道路和门户,怕不熟,出了失误。”说罢,就出了顾生房门,回头又嘱咐说:“我所生的孩子,你要好好的养着。你福薄且没有多少寿限,这个孩子可以给你光宗耀祖。夜已深了,不要惊动老母亲,我走了。”顾生心甚凄凉,正想问她到哪里去,女子身子一闪,像电光一亮,就不见了。顾生呆呆地站在那里像木头一样,一直过了很久很久。

到了天亮,顾生告诉了他母亲,母子两人只有感叹而已。三年后,顾生果然死了。他的儿子十八岁就中了进士,奉养祖母,直到送终。


\subsection{1.2.26   酒 友}
\label{\detokenize{p00_u5176_u5b83/_u767d_u8bdd_u804a_u658b_u5fd7_u5f02:id71}}
有个车生,家产还算不上中等人家。可是整天饮酒,晚上如不饮上三大杯便不能睡觉。因此,床头的酒瓮经常不空。

一天夜里,车生睡醒,一转身,觉得好像有个人同他睡在一块。起初他以为是盖在身上的衣服掉了,用手一摸,毛茸茸的一件东西,像猫但比猫大。用灯一照,原来是只狐狸,像犬一样卧着,醉得呼呼大睡。他一看自己的酒瓶,全空了,就笑着说:“这是我的酒友啊!”不忍心惊醒它,给它盖上衣服,用胳膊搂着它一块睡下,又留着烛光看它的变化。半夜里,狐狸欠身伸腰,睡醒了。车生笑着说:“睡得真美啊!”打开盖着的衣服一看,是一位俊俏书生。书生起身在床前跪拜,叩谢车生的不杀之恩。车生说:“我嗜酒成癖,但人们都认为我痴。你才是我的知己,如果你不疑心,我们就结为酒友。”说着又拉他上床睡下,说:“你可以常来,我们不要互相猜疑。”狐狸答应了。车生早晨醒了以后狐狸已经走了,他就准备了一些美酒,专门等候着狐狸来。

到了晚上,狐狸果然来了,二人促膝畅饮。狐狸酒量很大,说话诙谐,两人相见恨晚。狐狸说:“几次来饮你的美酒,怎么报答你呢?”车生说:“斗酒之欢,何必挂在嘴上?”狐狸说: “虽然这样,你并不富裕,弄点酒钱很不容易。我应当为你筹划点酒资。”第二天晚上,狐狸来告诉车生说:“从这里往东南七里路,路边有遗失的金钱,你可早点去捡来。”第二天早上车生去了,果然拾到二两银子。于是就买了佳肴,以备夜里饮酒用。狐狸又告拆车生说:“院后的地窖里藏着银子,你应当挖出来!”车生按它说的做了,果然挖出成百上千的银钱。车生高兴地说:“我口袋里有钱了,不用再为买酒犯愁了。”狐狸说:“不对。车辄中的那点水怎够长时间捧用呢?我要再为你想个法子。”又过了一天,告诉车生说:“集市上的荞麦价钱很便宜,这是奇货,你可以屯积。”车生听从了狐狸的话,收买了四十多石荞麦,人们都笑话他。没过多长时间,天大旱,地里的谷子、豆子都枯死了,只有荞麦还可以种。车生卖荞麦种,赚了比原来多十倍的钱,从此就个很富裕了。他又买了二百亩肥沃的良田,只要问狐狸,说多种麦子,麦子就丰收;多种高梁,高梁就丰收。种植的早与晚,都让狐狸决定。车生同狐狸的关系越来越好,狐狸称呼车生的妻子为嫂,把车生的孩子看作自己的儿子。后来车生死了,狐狸就不再来了。


\subsection{1.2.27   莲 香}
\label{\detokenize{p00_u5176_u5b83/_u767d_u8bdd_u804a_u658b_u5fd7_u5f02:id72}}
沂州有个书生,姓桑,名晓,字子明,少年时成了孤儿,一个人在红花埠居住。桑生性情文静,不好交往。除了每天去东邻吃两顿饭外,其余时间都在住所。东邻的书生与他开玩笑说:“你独自一人住在这院子里,不怕有鬼狐吗?”桑生笑着说:“大丈夫还怕鬼狐?雄的来了,我有利剑;雌的来了,我还要开门收留她呢!”东邻的书生回去后,与朋友们谋划好了,到了晚上用梯子越墙把一个妓女送进桑生住的院子里。那妓女走到桑生的房子前,轻叩房门。桑生瞧了瞧,问她是谁,那妓女自称是鬼。桑生非常恐惧,牙齿格格地响。妓女在门外徘徊了一会才去了。

第二天凌晨,东邻的书生来到桑生的书斋,桑生把夜间遇鬼的事诉说了一遍,并说要回家。东邻的书生拍手大笑,讥笑他说:“怎么不开门留她呢?”桑生一下明白是假鬼,随即安心照常住下来。

过了半年,夜里又有个女子叩门。桑生以为又是朋友与他开玩笑,便开门请她进来。一看,原来是个倾国倾城的美人。桑生吃惊地问她是从哪里来的,女子说:“我叫莲香,是西邻的妓女。”因为红花埠一带妓院很多,桑生也便信而不疑。随后,两人灭烛登床,亲热欢好。从此,每隔三五夜莲香就来一次。

一天晚上,桑生独自坐在书斋里,对着灯凝想,一个女子轻轻推门进来。桑生以为是莲香来了,忙起身与她说话。一照面,并不认识这女子。这女子约十五六岁,还没束发,两臂下垂,长袖拖地,十分风流美丽,走起路来,飘然若仙。桑生十分惊奇,怀疑她是狐精。女子说:“我是良家女子,姓李。爱慕你高雅风流,希望你能见爱。”桑生一听欣喜异常,急忙去拉她的手,却凉如冰块,他吃惊地问:“怎么这样凉啊?”女子回答说:“我自幼身单体弱,今晚来时又蒙了一身霜露,怎么能不凉呢?”说罢宽衣上床,竟是处女。女子说:“我为情缘,把贞操交给了你。若不嫌我丑陋,愿常来陪伴。这里还有别人来吧?”桑生说:“没有别人,只是西邻有个妓女,但不常来。”李女说:“应当避开她,我不同于妓院里的人,请您一定保密。可以她来我去,她去我来。”不一会,雄鸡报晓,李女便起身告辞。临走,将一只绣鞋赠给桑生,说:“这是我脚上穿的。常摆弄它可寄托你的思念之情。但是有外人在场时,千万别摆弄它。”桑生接过绣鞋一看,尖尖的像锥子,很喜欢。第二天晚上没人在屋,桑生就把鞋拿出来摆弄。李女忽然轻飘飘地来了,两人又亲热一番。此后,只要拿出绣鞋,李女便随即来到。桑生奇怪地询问原因,李女笑着说:“是碰巧了。”

一天夜间,莲香来到书房,吃惊地问道:“桑郎,你的气色怎么这样不好啊?”桑生说:“我自己不觉得。”莲香便起身告辞,约好十天后再相会。莲香走后,李女每夜都来,从没间断。李女问桑生:“你的情人怎么这么长时间不来?”桑生便把两人十天之约告诉了她。李女笑着说:“你看我比得上莲香美吗?”桑生说:“你两人可称双绝。但相比之下,莲香的体肤要比你温暖些。”李女闻言变色说:“你说双美,是对我说。她必定是月宫嫦娥,我一定比不上她。”因此很不高兴。算计起来,十天的约期已到。嘱咐桑生不要说出去,到时她要偷偷地看一看莲香。

次夜,莲香果然来了。与桑生嬉笑言谈,非常融洽。睡觉时,莲香大为惊骇地说:“坏了!才十天不见,你怎么劳损疲困到这个程度啊?你保证没别的女人来过吗?”桑生问她为什么这样说,莲香说:“我观察你的精神气色,脉像虚乱如丝,是被鬼缠身的症状。”

次夜,李女进门,桑生就问:“你偷看莲香长得怎样?”李女答:“确实很美。我原来便认为人间没有如此美貌的人,果然是个狐!她走后,我一直跟着,原来她住在南山一个山洞里。”桑生怀疑李女忌妒,也没理会她的话。

隔了一夜,桑生对莲香戏言:“我是绝对不信,可偏有人说你是狐精。”莲香慌忙问:“是谁说的?”桑生笑着说:“是我自己和你闹着玩的。”莲香说:“狐狸哪些地方与人不一样?”桑生说:“被狐狸迷住的人都会得病,严重的还会丧命,因此很可怕。”莲香说:“不是这样。像你这般年龄,行房三天后,精气便可复原。纵然是狐狸,也没什么害处。假若天天纵情淫乐,人比狐狸更厉害。世间死了那么多淫徒、色鬼,难道都是被狐狸迷惑死的吗?虽是如此,必定有人在背后说我的坏话。”桑生竭力表白没有,莲香追问得更急。迫不得已,就实说了。莲香说:“我本来就奇怪你为什么这样衰弱,为什么弱得这么快,难道李女不是人吗?你先不要声张,明晚,我也像她那样,偷偷看看她。”

到了夜间,李女来到,与桑生才说了几句话,便听到窗外有人咳嗽,慌忙离去。莲香进屋对桑生说:“你太危险了!李女真是鬼!若还贪恋她的美色,不与她一刀两断的话,你的死期近了!”桑生以为莲香嫉妒李女,也没吭声。莲香说:“我知道你割不断与她的感情。可是我也不忍心看你死去。明天,我会带药来医治你的病毒。幸亏中毒不深,十天就可治好。请允许我看护着你康复。”

次夜,莲香果然带了一小包药来,给桑生服药不大工夫,就泻了二三次。桑生只觉得内脏清爽,精神倍增。心中虽然感激莲香,但始终不信自己患的是鬼病。莲香夜夜同床陪伴着桑生;他几次求欢,都被莲香拒绝了。几天后,桑生的身体又健壮起来。莲香临走,殷切嘱咐桑生,一定要断绝与李女的关系,桑生假意答应了。

待到桑生夜间闭门后,又在灯下将绣鞋拿出把玩。李女忽然来了,几天不见,她一脸不高兴。桑生说:“她天天为我煎药治病,请不要怨她。对你好不好在我。”李女这才稍稍高兴些。桑生在枕边小声说:“我最爱你了,但有人说你是鬼。”李女张口结舌了很久,才骂道:“这一定是那个骚狐狸精乱说一气来迷惑你!你若不与她断绝往来,我就不再来了。”说完就呜呜地哭,桑生说了无数劝慰的好话,她才罢休。

隔了一夜,莲香来了,知道李女又来过,生气地说:“你是一定想死了!”桑生笑着说:“你怎么这样妒忌她呢?”莲香更气恼地说:“你得了绝症,我为你治好了,不妒忌的人又怎样做呢?”桑生仍假托玩笑说:“李女说,前几天我的病是狐狸作祟造成的。”莲香叹了口气说:“真像你说的这样,你就太执迷不悟了!万一不好,我纵有一百张嘴也解释不清了。请从此分别,一百天后,我再来看躺在病床上的你。”桑生挽留她,莲香不听,气愤地去了。

从此,李女无夜不来与桑生欢会。大约过了两个月,桑生便觉得浑身乏力,委靡不振。起初还自我安慰,后来,一天天变得枯瘦如柴,每顿饭只能喝一碗粥。本想回家调养,但还是恋着李女不忍离去。挨了几天,终于病倒床上,再也起不来了。邻生见他病重,天天派书童来送饭送水。直到这时,桑生才怀疑李女,对她说:“我悔不该不听莲香的话,弄到这步田地!”说完便昏死过去。过了好久才苏醒过来,睁眼四下看了看,李女早没了踪影。从此关系断绝。

桑生一个人躺在空房里,盼望莲香盼得望眼欲穿。一天,他正在想念莲香时,忽然有人掀帘进来。睁眼一看,果然是莲香。莲香走到床前,嘲笑着说:“乡巴佬,我是瞎说吗?”桑生泣不成声,过了一阵,自己说知道错了,求莲香快救命。莲香说:“你已病入膏肓,实在无法救治了。我这是来向你诀别的,以证明我并不是出于嫉妒。”桑生非常难过地说:“我枕头底下有件东西,烦你替我把它弄坏!”莲香找出,见是只绣鞋,便拿到灯下,反复细看。李女忽然进来,一见莲香,转身想逃。莲香用身体挡住了门。李女很窘,急得不知从哪里走。桑生数落着指责李女,李女无言以对。莲香笑着说:“我今天才有机会与你当面对质。以前你说桑郎的病不是你造成的,今天看你怎样说?”李女低头谢罪。莲香说:“这么漂亮的美人,怎么会为了爱结仇呢?”李女跪在地上哭得很悲痛,恳请莲香救救桑生。

莲香便把李女扶起来,详细询问她的生平。李女说:“我是李通判的女儿。少年夭亡,埋在院外。我好比是死了的春蚕,情丝未断,与桑郎交好,是我的心愿。致他于死地,确实不是出于本心。”莲香说:“听说鬼都愿致人于死地,以图死后在阴间可以常在一起,是吗?”李女说:“不是。两个鬼在一块,没什么乐趣。如有乐趣,阴间的少年郎难道少吗?”莲香说:“傻呀!夜夜交欢,人都受不了,何况是鬼呢?”李女也问:“听说狐能迷人致死,你有什么法术能不致如此呢?”莲香说:“你说的是那些采人精血补养自身的狐。我不是那一类的。因此,世间有不害人的狐,而决没有不害人的鬼,这是因为鬼的阴气太盛了!”桑生听了她们的对话,才知道鬼狐都是真的。幸亏相处已久,根本没觉得害怕。但一想到自己已是奄奄一息的人,不由得失声痛哭起来。莲香问李女:“你有救桑郎的办法吗?”李女红着脸摇头,说无能为力。莲香笑说:“恐怕桑郎身体健壮后,醋娘子又要吃杨梅了。”李女拜了拜说:“如有高明医生救得桑郎,使我不负罪郎君,我一定在阴间老老实实,哪敢有脸再到人间来!”莲香解开药袋,取出药来说:“我早就知道有今天。分别后,我跑遍了三山五岳,采集草药,历时三个多月,才配齐了药方。损劳过度待死的人,服用后没有不康复的。但是,病因谁得,还须由谁出药引子,这就不得不转求你全力协助。”李女问:“需要什么?”莲香说:“樱桃小口中的一点唾液罢了。我将药丸放进他口中,烦你口对口用唾液把它送下去。”李女听罢羞得面红耳赤,低着头直瞅着绣鞋犯难。莲香取笑说:“妹妹最得意的就只有绣鞋!”李更感羞惭,无地自容。莲香又说:“这不是你往常最熟练的技巧吗?今天怎么这样吝啬?”说罢将药丸放入桑生的口中,转身催促李女。李女不得已,只好口对口地输送唾液。莲香说:“再唾。”李女唾了一口,一连三四次,药丸才被送下去。不一会,就听到桑生的肚子雷鸣般地响起来。莲香又给他服下一丸后,亲自为他接唇布气。桑生觉得丹田发热,精神焕发。莲香说:“病好了。”这时雄鸡报晓,李女彷徨地告别走了。

莲香因桑生初愈,还需调养。特别是吃喝没有着落,便将院门反锁,让人误认桑生已回家,借以断绝外界来往,自己日夜护理他。李女也每夜必来,殷勤伺候。侍奉莲香也像亲姐姐一般。莲香也很疼爱她,过了三个月,桑生完全恢复了健康,此后,李女一连好几夜没来。有时来了,也只是看一看便走。对坐时,也总是闷闷不乐。莲香曾多次留她与桑生共寝,她都坚决不肯。有一次桑生追上她,硬把她抱回来,觉得她身子轻如草人。李女走不成,回来便和衣而卧,身子蜷曲起来不到二尺长。莲香越发爱怜她,示意桑生拥抱她,但无论怎样,也摇不醒她。桑生无奈只好自己睡下。及至醒来找她时,又不知去向了。此后十几天,李女再也没来过。桑生非常想念她,经常拿出绣鞋来与莲香共同把玩。莲香说:“如此美貌女子,我见了都很喜欢她,何况你们男人呢?”桑生说:“以前,一动绣鞋,她立刻就到,心里很怀疑,但是始终没想到她是鬼。现在见鞋思人,实在太令人难过了。”说着说着,泪流满面。

这以前,有个姓张的财主,他的女儿名叫燕儿,十五岁时死了。过了一夜又苏醒过来,睁眼一看,起身就向外跑。张财主急忙关上门,她出不去,便自己说:“我是李通判女儿的灵魂,感谢桑郎的关照,我送给他的绣鞋还在他那里。我真的是鬼啊,关起我来有什么好处呀!”张翁听她说的有些缘故,就问她为何来到这里。燕儿低头看了一下,自已也解释不清楚。旁边有人说桑生已回家养病了,燕儿执意分辩说没有,家人非常怀疑。东邻的书生听说这事,就从墙头上偷偷观察桑生住处,见桑生正与一个美女说话,他就突然闯了进去,仓促之间,已不见女于的踪影。邻生很惊疑,再三追问桑生,桑生笑着说:“我过去与你说过,雌的来了我就留下她!”邻生将燕儿刚才的话,向桑生说了一遍。桑生马上开锁出门,想去打听一下。但转念一想没有去的理由,十分苦恼。

张母听说桑生果然没有回家,越发觉得奇怪,就派佣女到桑生那里要绣鞋。桑生将鞋交给她。燕儿见到绣鞋十分高兴,急忙试穿,绣鞋却比脚小了一寸多。她大吃一惊,拿过镜子一照,模模糊糊像是明白自己是借尸还魂了。于是便把以前发生的事细细说了一遍,张母才相信了。燕儿对镜哭着说:“我对那时的容貌很有自信,但是每当见了莲香姐,还自愧不如。而今成了这个样子,做人还不如做鬼呢!”拿着绣鞋放声大哭,谁也劝说不住。哭完后,蒙上被子就躺在床上,饭也不吃。不久,全身浮肿起来,七天不吃东西,也没死,而浮肿却渐渐消了。此后,她便饥饿难忍,开始吃饭。过了几天,浑身发痒,脱了一层皮。早晨起床时,睡鞋掉下来,抬起来再穿时,鞋子又肥又大。试穿以前的绣鞋,肥瘦正合适。她很是喜欢,再照镜子,眉眼已和过去一样,更为高兴。梳洗打扮好了,去见母亲,凡是见她的人都非常高兴。

莲香听说这一奇闻,就劝桑生向张家提亲。桑生觉得两家贫富悬殊,没敢唐突去提。不久,逢张母寿辰,桑生就随着张家的子婿们前去祝寿。张母见帖上有桑生的名字,就让燕儿躲在帘子后偷偷辨认。桑生最后一个到,燕儿急忙跑上去,拉住桑生的袖子,要跟他一块回家。张母训斥她一顿,燕儿才害羞地回到屋里。桑生仔细辨认燕儿,确是李女再生,不觉流泪,拜倒在张母面前不起来。张母忙上前把他扶了起来,并不轻视他。桑生出来后,就托燕儿的舅舅前去提亲。张母议定下良辰吉日,招桑生为养老女婿。桑生回去,把这事告诉莲香,并商量怎么办。莲香难过了好一阵子,才决定要和桑生分别。桑生大吃一惊,泪如雨下。莲香说:“你被人家招赘成婚,我跟着去,有什么脸面?”桑生再三考虑,还是先把莲香送回家去,再回来迎娶燕儿,莲香应允。桑生把实情告诉了张家,张家听说他已有了妻子,便怒气冲冲地训斥他。燕儿在一旁极力为桑生辩解,张家才同意了桑生的请求。

婚期到了,桑生亲自去迎娶燕儿。他家的摆设本来很不像样,可是等迎亲回来时,从大门到新房,全是花毡铺地;千百只灯笼蜡烛,照耀得如同白昼。莲香扶新娘入了洞房,蒙头绸一揭下,她们就高兴得像以前那样。莲香陪伴他俩喝合婚酒,细细询问了燕儿还魂的事。燕儿说:“那天离开后,心中闷闷不乐,觉得自己是鬼,没脸见你们,决定再也不回坟里去了,便随风漂游。每每见到世上的人,就非常羡慕。白天藏在草丛中,夜里便由着自己的脚信步走。偶然到了张家,见一个少女病死在床上,魂就附到她身上,没想到真的活了。”莲香听了,沉默了好久,像是在思索什么。

过了两个月,莲香生下一个儿子。产后得病,日渐沉重。她握住燕儿的手说:“我只好把孩子托付给你了,希望你能把他当作亲生儿子来抚养。”燕儿流下了眼泪,并千方百计地劝慰她。几次要给她请医生,都被莲香拒绝了。眼看着莲香生命垂危,只有一丝气息,桑生和燕儿都难过得哭泣。忽然她又睁开眼说道:“不要这样,你们愿我活,我却愿意死。若有缘分,十年之后还能再见面。”说完就断了气。掀开被要给她穿寿衣时,她已化为狐。桑生不忍心另眼相待,仍以隆重的葬礼安葬了她。

莲香生的孩子,取名狐儿。燕儿抚养他如同亲生。每逢清明节,都抱着他到莲香的坟上哭祭。后来,桑生考中了举人,家境渐渐富裕起来。而燕儿一直愁着没有生育。狐儿聪明伶俐,只是体弱多病。燕儿就经常劝桑生再娶一妾。

一天,丫鬟忽来禀报: “门外有个老婆子,领着个女孩要卖。”燕儿就让领进来看看。乍见面,便吃惊地说:“莲香姐转世了!”桑生细看那女孩,酷似莲香,也觉惊异。便问:“多大了?”回答说:“十四岁。”又问:“聘金要多少?”老太婆答:“我这孤老婆子,只有这么个闺女,但愿能给找个好人家,我也有个吃饭的地方,日后老骨头不至于丢在荒山野谷中,也就满足了。”桑生多付了些银两,买下姑娘。

燕儿握住姑娘的手,来到内屋,托起她的下颌笑问:“你认识我吗?”姑娘回答:“不认识。”细问她的身世,姑娘说:“我姓韦,父亲在徐城卖酒,已死了三年了。”燕儿数着指头细算,莲香已死了整十四年。再仔细观察姑娘的容貌神态,无处不像莲香。于是拍拍她的头大声叫道:“莲姐!莲姐!你说十年后再见面,当真没骗我。”姑娘像大梦初醒似地“咦”了一声,盯着燕儿细看。桑生见状高兴得笑着说:“这真是‘似曾相识燕归来’啊!”姑娘流着泪说道:“是了!听母亲说,我一出生就会说话,家中人以为是不祥之兆,让我喝了狗血,就忘记了前世因果,今天才如梦初醒。娘子,你就是那个不愿做鬼的李妹妹吗?”三人共同回忆前生的事,百感交集。

寒食节那天,燕儿说, “今天是我与桑郎每年哭祭姐姐的日子。”便与姑娘同到莲香墓前,见墓地野草丛生,树也长高了。姑娘也触景伤情地叹息。燕儿对桑生说:“我与莲香姐两世都是好友,不忍分离,应该把前世的尸骨同葬一墓。”桑生听从她的意见,就挖开李女的坟,取出尸骨,运回来与莲香的合葬在一起。亲友们知道这桩怪事后,都穿着吉庆的服装赶来观看葬礼,不约而来的达几百人。

我庚戌年南游到了沂州,下雨天走不了,住在旅店里。有个叫刘子敬的,是桑生家的一个表亲,拿出同乡王子章写的《桑生传》约万余字,我得以细看了一下。这里只是故事的大概情况。


\subsection{1.2.28   阿 宝}
\label{\detokenize{p00_u5176_u5b83/_u767d_u8bdd_u804a_u658b_u5fd7_u5f02:id73}}
广东西边有个叫孙子楚的人,是个名士。他生来有六个手指,性格憨厚,口齿迟钝。别人骗他,他都信以为真,有时遇到座席上有歌妓在,他远远地看见转身便走了。别人知道他有这种脾气,就把他骗来,让妓女挑逗逼迫他,他就羞得脸一直红到脖子,汗珠直往下淌。大家因此而笑话他,形容他呆痴的样子,到处传说,并给他起了一个难听的外号“孙痴”。

本县有个大商人某翁,可与王侯比富,他的亲戚都是有钱有势的人家。大富商有个女儿叫阿宝,长得非常漂亮,她父母正忙着为她挑选佳婿。许多名门贵族的子弟争着来求亲,商人都没看中。正巧孙子楚的妻子死了,有人捉弄他,劝他到大富商家提亲。孙子楚也没想想自己的情况,竟托媒人去了。商人也知道孙子楚的名字,但嫌他太穷没答应。媒人要离开的时候,正好遇上阿宝。阿宝问什么事,媒人讲了来意,阿宝便开玩笑地说:“他如能把那个多余的指头砍了,我就嫁给他。”媒婆把这话告诉了孙子楚。孙自言道:“这不难。”媒婆走后,孙子楚便用斧头把第六个指头剁去了,血流如注,痛得他几乎死了过去。过了几天,刚能起床,便到媒婆家把已剁掉六指的手给她看。媒婆吓了一跳,忙去富商家告诉了阿宝。阿宝也很惊奇,又开了个玩笑说孙子楚还得去掉那个痴劲才行。孙子楚听后大声辩解,说自己并不痴呆,然而却没有机会当面向阿宝表白。转念一想,阿宝也未必美如天仙,何必把自己的身价抬那么高。因此求亲的念头也就凉了下来。

正好清明节到了,按风俗这一天是妇女们到郊外踏青的日子。一些轻薄少年也结伴同行,对妇女们评头论足,随意调笑。有几个文朋诗友也硬把孙子楚拉去了,有的嘲笑他说:“不想看看你那可意的美人吗?”孙子楚知道大家在戏弄自己,但因为受了阿宝的嘲弄,也想见见阿宝是个什么样的人,所以便欣然随大家边走边看。远远地见一个女子在树下歇息,一群恶少把她围得像一堵墙似的。朋友们说:“这一定是阿宝了。”跑过去一看,果然是阿宝。仔细看真是美丽无比,十分动人。一会儿,围观的人越来越多,阿宝急忙起身走了。众人神魂颠倒,评头论足,简直要发狂了,唯有孙子楚默默无语。大家都走开了,可回头一看,孙子楚仍然呆呆地站在那儿,喊他也不答应。大家来拉他说:“魂随阿宝去了吗?”孙子楚还是不说话。大家因为他平时就呆板少语,也不奇怪。有人推着他,有人拉着他,回家去了。

孙子楚到家后,一头倒在床上。整天昏睡不起,像醉了一样,喊也喊不醒。家里人以为他丢了魂,到野外给他叫魂也不见效。用劲拍打着问他,他才含糊朦胧地说:“我在阿宝家。”再细问他时,又默默无语了。家里人心里害怕,也不知是怎么回事。

当初,孙子楚见阿宝走了,心里非常难舍,觉得自己的身子也跟她走了,渐渐地靠在了她的衣带上,也没有人呵斥她。于是跟阿宝回了家。坐着躺着都和阿宝在一起,到了夜里便与阿宝交欢,很是亲热欢治。可就是觉得肚子里特别饿,想回家一趟,却不认得回家的路。阿宝每每梦到与人交合,问他的名字,都是说:“我是孙子楚。”阿宝心里很奇怪,可也不便告诉别人。

孙子楚躺了三天,眼看就要断气了,家里人又急又怕,就托人到商人家里恳求给孙子楚招魂。商人笑着说:“平时素无往来,怎么会把魂丢在我家呢?”孙家哀求不已,商人才答应了。巫婆拿着孙子楚的旧衣服、草席子来到商人家。阿宝知道了来意后,害怕极了,她不让巫婆到别处去,直接把巫婆带到自已房中,任凭巫婆招呼一番后离去了。巫婆刚回到孙家门口,屋内床上的孙子楚已经呻吟起来。醒后,他还能清清楚楚地说出阿宝室内摆设用具的名字和颜色,一点不错。阿宝听说后,更加害怕了,但心里也感到了孙子楚的情义之深。

孙子楚能起床后,不论坐着,还是站着,总是若有所思,精神恍惚,好像丢了什么。常打听阿宝的消息,盼望能再见到阿宝。浴佛节那天,听说阿宝将要到水月寺烧香,孙子楚一早就去在路旁等候,直看得头晕眼花,到中午阿宝才来。阿宝从车里看到了孙子楚,用纤手撩起帘子目不转睛地注视着他。孙子楚更加动了情,就在后面跟着走。阿宝忽然让丫鬟来问他的姓名。孙子楚很殷勘地说了,更加魂魄摇荡,直到车走了以后,他才回家。

孙子楚回家后,旧病复发,不吃不喝,昏睡中喊着阿宝的名字,直恨自己的灵魂不能再像上次那样到阿宝的家里去。他家中养了一只鹦鹉,这时突然死了,小孩子拿着在床边玩。孙子楚看见了心想,我如果能变成鹦鹉,展翅就可飞到阿宝的房里了。心里正想着,身子果然已变成了一只鹦鹉,翩然飞了出去,一直飞到了阿宝的房中。阿宝高兴地捉住它,将它的腿用绳子绑住,用麻籽喂它。鹦鹉忽然口吐人声说:“姐姐别绑我,我是孙子楚啊!”阿宝吓了一跳,忙解开绳子,鹦鹉也不飞走。阿宝对着鹦鹉祷告说:“你的深情已铭刻在我心中。现在我们人禽不同类,良姻怎么能复圆呢?”鹦鹉说:“能在你的身边,我的心愿已经满足了。”其它人喂它它不吃,只有阿宝喂它,它才肯吃。阿宝坐下,鹦鹉就蹲在她的膝上;阿宝躺下,鹦鹉就偎在她的身边。这样过了三天,阿宝很可怜它,就悄悄派人去察看孙子楚。见孙子楚僵卧在床上已断气三天了,只是心口窝还有点温暖。阿宝又祷告说:“你要是能重新变成人,我就是死也要与你相伴。”鹦鹉说:“你骗我!”阿宝立刻发起誓来。鹦鹉斜着眼睛,好像在思索什么。一会儿,阿宝裹脚,把鞋脱到床下,鹦鹉猛地冲下,用嘴叼起鞋来飞走了。阿宝急忙呼叫它,鹦鹉已经飞远了。阿宝派个老妈妈前去探望,孙子楚果然已经醒过来了。

孙家的人见鹦鹉叼来一只绣鞋,便落地死了,正感到奇怪,孙子楚已醒来。刚醒就开始找鞋,大家都不知道是怎么回事。正好阿宝家的老妈妈赶到。进房看到孙子楚,就问他鞋在哪里。孙子楚说:“鞋是阿宝发誓的信物,请你代我转告,我孙子楚不会忘记她金子般的诺言。”老妈妈回来把话照榉说了一遍,阿宝更是奇怪,便让丫鬟把这些事故意泄露给自己的母亲。母亲又详细询问明白后才说:“孙子楚这个人还是有才学的,名声也不错,但却像司马相如一样贫寒。选了几年的女婿,最后得到的竟是这样的,恐怕会被显贵们笑话。”阿宝因为鞋的缘故,不肯他嫁,父母只好听了她的。有人跑去告诉孙子楚这个消息,孙子楚非常高兴,病立刻全好了。

阿宝的父母想把孙子楚招赘过来。阿宝说:“女婿不应该长住在岳父家里。何况他家又贫穷,住长了会让人家瞧不起。女儿既然已经答应了他,住草屋、吃粗饭决无怨言。”孙子楚于是去迎娶新娘成婚,两人相见如隔世的夫妻又团圆了一样高兴。

自此以后,孙子楚家有了阿宝的嫁妆,增加了财物家产,生活有了好转。孙子楚只是迷在书里,不知道治家理财。阿宝是个善于当家过日子的人,家中诸事都不用孙子楚操心。过了三年,家中更富了。可孙子楚忽然得热病死了。阿宝哭得非常悲痛,眼里的泪水始终没有干过,觉也不睡,饭也不吃,谁劝也不听,乘夜里上吊了。丫鬟发现后,急忙救护,才醒了过来,可总也不吃东西。三天后,家人召集亲友,准备给孙子楚送葬。听到棺材里传出呻吟之声,打开一看,孙子楚复活了。自己说是:“见到了阎王,阎王说我这个人生平朴实诚恳,命令我当了部曹。忽然有人说:‘孙部曹的妻子将要到了。’阎王查了生死簿,说:‘她还不到死的时候。’又有人说:‘已经三天不吃饭了。’阎王回身说:‘你妻子的节义行为感动了我,让你再生阳世吧。’于是就派马夫牵着马把我送了回来。”

从此以后,孙子楚身体渐渐恢复了正常。这年适逢乡试,进考场之前,一群少年戏弄孙子楚,一起拟了七个偏怪的试题,把孙子楚骗到僻静之处,说:“这是有人通过关系得到的试题,因咱们友好才偷偷地告诉你。”孙子楚信以为真,黑天白日地琢磨,准备好了七篇八股文。大家都暗暗地笑他。当时主考官考虑到用过去的题目会有抄袭之弊,有意地一反常规,试题发下来,竟与孙子楚准备的七题一模一样,孙子楚中了乡试头名。第二年又中了进士,进了翰林院。皇上听说了孙子楚这些怪事,把他召来询问,孙子楚全都如实地奏明了。皇上非常称赞和高兴。后又召见阿宝,赏赐了很多东西。


\subsection{1.2.29   九 山 王}
\label{\detokenize{p00_u5176_u5b83/_u767d_u8bdd_u804a_u658b_u5fd7_u5f02:id74}}
曹州府有一个姓李的书生,家里很富有,但住宅不宽敞。宅子后面有一个几亩地的园子,一直荒废着。

一天,一个老头来租他的房子住,愿出一百两银子作租金。李生以没有多余的房子为由谢绝他。老头对李生说:“请你放心收下租金,不要顾虑。”李生也不知道他的意思,就暂且收下租金,看看是怎么回事。

过了一天,村里的人见有车马家眷进了李家的大门,纷纷扬扬好像有很多人。大家都怀疑李家宅子并不大,怎么住得下这么多人?有的来问李公子,李却一点也不知道这回事。回家看了看,并没任何迹象和动静。

又过了几天,租房子的老头忽然来拜访,对李生说:“搬来贵府已经好几天了,事事都得重新安排,又得支锅做饭,又得打铺睡觉,一直没来得及来拜访主人。今天叫小女做了顿便饭,请你一定赏光过去坐坐。”李公子当即跟着老头去赴宴。

一走进他家后面的园子,忽见房舍一片,非常华丽,都是新盖的。进入正房,房里陈设也很漂亮。酒鼎正在廊下沸着,茶炉的烟也从厨房里袅袅冒向天空。刚落坐一会儿,就端上了酒菜,尽是山珍海味。时常看到门外有少年人来来往往,又听到男女青年聒聒说话,欢声笑语不绝于耳,家人奴婢像有一百多人。李公子心里已经明白,这家人家是狐。

李生喝完了酒回到自己房里,心里暗起杀机。他每次去赶集,就买下一些硫磺、芒硝,积攒了几百斤,暗暗布满后园。等他布置好了,就骤然点燃,顿时满园烈火冲天,浓烟滚滚,烧得臭不可闻。群狐乱叫之声惊天动地,嘈杂一片。烧了一阵子,大火才灭了。进园子一看,满园都是烧死的狐狸,焦头烂额的,不计其数。李生正检看间,老头自外面进来,满脸悲惨,责怪李公子说:“我与你远日无仇,近日无恨,租你的荒园出银百两作租金,也算对得起你。你怎么忍心烧灭我的全家!这个奇仇大恨,哪有不报的道理!”说完,愤然而去。李生怕它们来报复,加强了防范。可是一年多的时间,没有任何动静。

顺治初年,山中盗贼群起,约聚集了一万多人,官兵也不能剿灭他们。李生因为家中人多财丰,天天发愁,怕贼下山来抢劫。正在为难之际,村中来了一个算命先生,自称“南山翁”,算人的生死命运,祸福吉凶,了然如他亲见。一时名声大振。李公子也请他来家算卦,算命先生一进屋就肃然起敬,惊呼:“足下乃真主也!”李公子听了大吃一惊,以为这是无稽之谈。算命先生却郑重其事地坚持这样说。李公子半信半疑,对算命先生说:“哪有白手起家而成了帝王的?”算命先生说:“不然!自古帝王君主有很多是出身匹夫的,有谁生下来就是天子的呢?”李公子仍持怀疑态度,但对算命先生却尊敬起来,请他上坐。算命先生竟以“卧龙”自居。提议先准备胄甲几千套,弓箭几千副。李公子顾虑招不起人马来,算命先生说:“臣愿为大王联合诸山人马,订立盟约,并宣扬大王为真龙天子,山中将领、士卒必然前来响应。”李公子很高兴,便让先生去按计划行事。他把家藏的银子全部拿出来,制造胄甲,购买弓箭,准备起事。隔了几天,算命人来说:“凭借大王的福威,加上我三寸不烂之舌,各山头领没有不愿归你指挥的。”果然,没出十天,就有数千人马来归顺。于是李公子便拜算命先生为军师,树起大旗,设置五色彩旗,占据山头,建造围墙,一时声势大振。县官带兵来剿,算命人指挥兵马,打得官兵大败而归。县官害怕,报告了兖州知州。兖州兵远来讨伐,算命人又指挥人马埋伏起来,一举将兖州兵打得大败,伤亡惨重。从此,李公子声势更大,人马到了一万多。李公子便自立为“九山王”。算命人愁马少,又谋划派一支兵抢劫了京城解往江南的军马。于是“九山王”威震天下,加封算命人为“护国大将军”。

从此,李公子在山上高枕无忧,非常自负,以为黄袍加身称王称帝的日子为期不远了。不料,东抚因为夺马一事,已经准备进剿他们;后又得到兖州兵败的报告,便会集了六路兵马,精兵数千,四面包剿“九山王”。这时人喊马叫,遍布山谷。“九山王”大为震惊,呼唤算命人来商议对策,却已不见了。“九山王”束手无策,他登上山顶一望,长叹道:“我今日才知朝廷的势力之大了!”

不久,官兵攻破山寨,李公子被擒,妻子老小全家被杀。他这才明白,算命先生就是当年的老狐狸,原来是以杀害李公子满门来报他当年的灭族之仇的。


\subsection{1.2.30   遵 化 署 狐}
\label{\detokenize{p00_u5176_u5b83/_u767d_u8bdd_u804a_u658b_u5fd7_u5f02:id75}}
诸城县丘公,在遵化当官时,官署中原就有很多狐。署中最后面的一座楼上,雄狐成群地住在上面,成了狐的老窝,还经常出来祸害人,越是撵它们走,闹得就越厉害。以前凡在这里当宫的人,都是摆上供品,对它们恭敬地祷告,没有敢得罪这些狐的。

丘公来这里上任后,听说有这样的事,很生气。这些狐也害怕丘公性情刚烈,就变成一个老婆婆,告诉丘公的家人说:“请转告大人,我们不要为仇。给三天时间,我们带领全家老小搬走。”丘公听了,也不言语。

到了第二天,丘公阅兵已毕,告诉大家不要解散,叫众兵把各营的大炮都抬来,突然包围了署后的楼。丘公命千门大炮齐发,倾刻之间,几丈高的大楼,摧为平地。群狐的毛皮、血肉,像下雨一样从天而降。只见滚滚浓烟中,有一缕白气冲天而去,大家都望着天空说:“逃了一只狐!”然而自此后,署中却太平无事了。

后二年,丘公打发得力仆人送银子若干去京都,打算托人办理升迁,事还没有着落,就暂时把银子藏在班役的家里。忽然有一个老头到宫殿喊冤叫屈,说他妻子被人杀害,还揭发丘公克扣军饷,行贿高官,银子现藏在某人家里,可以去查证。皇帝下旨押着老头去班役家检查,但怎么搜也搜不到银子。老头就用一只脚点地,差人明白他的意思,挖开地一看,果然挖出银子来,银子上还刻着“某郡解”的字样。接着找老头,已经不见了。官府拿了地方上的户口名册想找这个老头,竟没有其人。丘公因此案被处死了。人们才知这个老头就是当年逃走的狐狸。


\subsection{1.2.31   张 诚}
\label{\detokenize{p00_u5176_u5b83/_u767d_u8bdd_u804a_u658b_u5fd7_u5f02:id76}}
河南有个姓张的人,祖籍是山东人。明朝末年山东大乱,他的妻子被清兵抢走了。张某常年客居河南,后来就在河南安了家,娶了妻子,生了个儿子取名张讷。不久,妻子死了,张某又娶了一个妻子,也生了个儿子,取名张诚。继室牛氏性情凶悍,常嫉恨张讷。把他当作牛马使唤,给他吃粗劣的饭食。又让张讷上山砍柴,责令他每天必须砍够一担。如果砍不够,就鞭打辱骂,张讷几乎无法忍受。牛氏对亲生的儿子张诚,则像宝贝一样,偷偷给他吃甜美的食物,让他到学堂去读书。

后来,张诚渐渐长大了,性情忠厚、孝顺。他不忍心哥哥那样劳累,暗暗地劝说母亲,牛氏不听。一天,张讷进山砍柴,还没砍完,忽然下起暴雨,他只好躲避在岩石下。雨停了,天也黑了,他肚子太饿,只好背着柴回家。牛氏看到砍的柴不够数,就发怒不给饭吃。张讷饥饿难忍,只得进屋躺下。张诚从学堂回来,看见哥哥沮丧的样子,就问:“病了?”张讷说:“饿的。”张诚问他原因,张讷把实情说了。张诚悲伤地出去了。过了一会儿,张诚怀里揣着饼来送给哥哥吃。哥哥问他饼是哪来的,他说:“我从家中偷了面,让邻居做的。你只管吃,不要说出去。”张讷吃了饼,嘱咐弟弟说:“以后不要这样了!事情泄漏了会连累弟弟的。况且一天吃一顿饭虽然饿,也不会饿死。”张诚说:“哥哥本来身体就弱,怎么能多打柴呢!”

第二天,吃过饭后,张诚偷偷上山,来到哥哥砍柴的地方。哥哥见到他,惊奇地问:“你来干什么?”张诚回答说:“帮哥哥砍柴。”张讷问:“谁叫你来的!”他说:“是我自己来的。” 哥说:“别说弟弟不能砍柴,就是能砍,也不行。”催促他快回去。张诚不听,手脚并用扯着柴禾,还说:“明天要带斧头来。”哥哥过去制止他,见他的手指出血,鞋也磨破了,心痛地说:“你不快回去,我就用斧头割颈自杀!”张诚才回去了。哥哥送了他一半路,才又回去。张讷砍完柴回家,又到学堂去,嘱咐弟弟的老师说:“我弟弟年龄小,要严加看管,不要让他出去,山里虎狼很多。”老师说:“上午不知他到什么地方去了,我已经责打了他。”张讷回到家,对张诚说:“不听我的话,挨打了吧?”张诚笑着说:“没有。”第二天,张诚怀里揣着斧头又上山了。哥哥惊骇地说:“我再三告诉你不要来,你怎么又来了?”张诚不说话,急忙砍起柴来,累得汗流满面,一刻不停。约摸砍得够一捆了,也不向哥哥告辞,便回去了。老师又责打了他。张诚就把实情告诉老师,老师赞叹张诚的品行,也就不禁止他了。哥哥屡次劝阻他,他始终不听。

一天,张讷兄弟俩同其他一些人到山中砍柴。突然来了一只老虎,众人都害怕地伏在地上,老虎径直把张诚叼走了。老虎叼着人走得慢,被张讷追上。他使劲用斧头砍去,正中虎胯。老虎疼得狂奔起来,张讷再也追不上了,痛哭着返回来。众人都安慰他,他哭得更悲痛了,说:“我弟弟不同于别人家的弟弟,况且是为我死的,我还活着干什么!”接着就用斧头朝自己的脖颈砍去。众人急忙救时,斧头已经砍入肉中一寸多,血如泉涌,昏死过去。众人害怕极了,撕了衣衫给张讷裹住伤口,一起扶他回家。后母牛氏哭着骂道:“你杀了我儿子,想在脖子上浅浅割一下来搪塞吗?”张讷呻吟着说:“母亲不要烦恼!弟弟死了,我绝不会活着!”众人把他放到床上,伤口疼得睡不着,只是白天黑夜靠着墙壁坐着哭泣。父亲害怕他也死了,时常到床前喂他点饭,牛氏见了总是大骂一顿。张讷于是不再吃东西,三天之后就死了。

村里有一个巫师“走无常”,张讷的魂魄在路上遇见他,诉说了自己的苦难,又询问弟弟在什么地方。巫师说没看见,但反身带着张讷走了。来到一个都市,看见一个穿黑衣衫的人,从城中出来。巫师截住他,替张讷打听张诚。黑衣人从佩囊中拿出文牒查看,上面有一百多男女的姓名,但没有姓张的。巫师怀疑在别的文牒上,黑衣人说:“这条路属我管,怎么会有差错?”张讷不信张诚没死,一定要巫师同他进城。城中新鬼旧鬼来来往往,也有老相识,问他们,没有人知道张诚的下落。忽然众鬼一齐叫:“菩萨来了!”张讷抬头看去,见云中有一个高大的人,浑身上下散放光芒,顿时世界一片光明。巫师向张讷贺喜说:“大郎真有福气!菩萨几十年才到阴司一次,给众冤鬼拔苦救难,今天你正好就碰上了!”于是拉张讷一起跪倒。众鬼纷纷嚷嚷,合掌一齐诵慈悲救苦的祷词,欢腾之声震天动地。菩萨用杨柳枝遍酒甘露,水珠细如尘雾。不一会儿,云霞、光明都不见了,菩萨也不知哪里去了。张讷觉得脖子上沾有甘露,斧头砍的伤口不再疼痛了。巫师仍领着他一同回家,看见村里的门了,才告辞去了。张讷死了两天,突然又苏醒过来,把自己见到和遇到的事讲了一遍,说张诚没有死。后母认为他这是编造骗人的鬼话,反而辱骂他。张讷满肚子委屈无法申辩。摸摸创痕已全好了,便支撑着起来,叩拜父亲说:“我将穿云入海去寻找弟弟。如果见不到弟弟,我一辈子也不会回来了。愿父亲仍然以为儿已死了。”张老汉领他到没人的地方,相对哭泣了一阵,也没敢留他。

张讷离家出走后,大街小巷到处寻访弟弟的下落。路上盘缠用光了,就要着饭走。过了一年,来到金陵。一天,张讷衣衫褴褛佝偻着身子,正在路上走着,偶然看见十几个骑马的过来,他赶紧到路旁躲避。其中有一人像个官长,年纪有四十来岁,健壮的兵卒,高大的骏马,前呼后拥。随行的一个少年骑一匹小马,不住地看张讷。张讷因为他是富贵人家的公子,不敢抬头看。少年勒住马,忽然跳下马来,大叫:“这不是我哥哥吗!”张讷抬头仔细一看,原来是张诚!他握着弟弟的手放声大哭。张诚也哭着说:“哥哥怎么流落到这个地步?”张讷说了事情的缘由,张诚更伤心了。骑马的人都下来问了缘故,并告知了官长。官长命腾出一匹马给张讷骑,一同回到他的家里。张讷这才详细地问了张诚后来的经过。

原来,老虎叼了张诚去,不知什么时候把他扔在了路旁,张诚在路旁躺了一宿。正好张别驾从京都来,路过这里。见张诚相貌文雅,爱怜地抚摸他。张诚渐渐苏醒过来,说了自己的家乡住处,可是已经相距很远了。张别驾将他带回家中,又用药给他敷伤口,过了几天才好了。张别驾没有儿子,就认他作儿子。刚才张诚是跟随张别驾去游玩回来。张诚把经过全部告诉哥哥,刚说完,张别驾进来了,张讷对他拜谢不已。张诚到里面,捧出新衣服,给哥哥换上;又置办了酒菜叙谈离后经过。张别驾问:“贵家族在河南有多少人口?”张讷说:“没有。父亲小时候是山东人,流落到河南。”张别驾说:“我也是山东人。你家乡归哪里管辖?”张讷回答说:“曾听父亲说过,属东昌府管辖。”张别驾惊喜地说:“我们是同乡!为什么流落到河南?”张讷说:“明末清兵入境,抢走了我的前母。父亲遭遇战祸,家产被扫荡一空。先是在西边做生意,往来熟悉了,就在那儿定居了。”张别驾惊奇地问:“你父亲叫什么名字?”张讷告诟了他,张别驾瞠目结舌。又低头想着什么,急步走进内室。不一会儿,太夫人出来了,张讷兄弟两人一同叩拜。拜毕,太夫人问张讷说:“你是张炳之的孙子吗?”张讷说:“是。”太夫人哭着对张别驾说:“这是你弟弟啊!”张讷兄弟俩不知是怎么回事。太夫人说:“我跟你父亲三年,流落到北边去,跟了黑固山半年,生了你的这个哥哥。又过了半年,黑固山死了,你哥哥补官在旗下,做了别驾。如今已解任了,常常思念家乡,就脱离了旗籍,恢复了原来的宗族。多次派人到山东打听你父亲的下落,没有一点消息。怎么会知道你父亲西迁了呢!”于是又对别驾说:“你把弟弟当儿子,真是罪过!”张别驾说:“以前我问过张诚,张诚没有说过是山东人。想必是他年幼不记得了。”就按年龄排次序:别驾四十一岁,为兄长;张诚十六岁最小;张讷二十二岁为老二。别驾得了两个弟弟,非常欢喜,同他们住在一间屋里,尽述离散的端由,商量着回归故里的事情。太夫人怕不被容纳。张别驾说:“能在一起过就在一起,不能在一起就分开过。天下哪有没有父亲的人呢?”于是就卖了房子,置办行装,定好日子起程。

回到家乡,张讷和张诚先到家中给父亲报信。父亲自从张讷走后,妻子牛氏也死了,孤苦伶仃,对影自叹。忽然见张讷回来,惊喜交加,恍恍惚惚;又看到了张诚,高兴得说不出话来,只是流泪。兄弟俩又告诉说别驾母子来了,张老汉惊呆了,也不会哭,也不会笑了,只呆呆地站着。不多会儿,别驾进来,拜见父亲。太夫人抱住张老汉相对大哭。看见婢女仆人屋里屋外都站满了,张老汉不知如何是好。张诚不见母亲,一问,才知已经死了,哭得昏了过去,有一顿饭功夫才苏醒过来。张别驾拿出钱来,建造楼阁。请了老师教两个弟弟读书。槽中马群欢腾,室内人声喧闹,居然成了大户人家。


\subsection{1.2.32   汾 州 狐}
\label{\detokenize{p00_u5176_u5b83/_u767d_u8bdd_u804a_u658b_u5fd7_u5f02:id77}}
汾州有个州判,姓朱,他住的宅子里有很多狐。一天,朱公正在房里静坐,忽然有一个女子在灯下往来。起初朱公以为是家里仆人的妻子来干事,没有在意。过了一会儿抬头一看,竟然不认识。又见她容貌很美,心里知道这一定是个狐女,不过还是很欢喜她,就急忙叫她过来。女子停住脚笑着说:“你声音这么严厉,哪个是你的丫鬟使女?”朱公笑着起身拉她坐下,向她道歉,便与她共同欢好。时间一长,就像是一对夫妻一样恩爱。

一天,狐女忽对朱公说: “你的官职将要有调动,我们马上就要分别了。”朱公问她:“什么时候?”回答说:“就在眼前。但是祝贺的人在你门前的时候,吊丧的人却在你的老家,你不能当官了!”三天后,朱公调迁的命令果然到了;再一日,老家就来报丧,说太夫人去世了。朱公只得马上辞职,回家奔丧。他要求狐女一同回家,狐女不同意,送朱公到了河边。朱公再次要她一块上船,狐女说:“你不知道,我们狐不能过河。”朱公不忍别离,在河边恋恋不舍。这时狐女忽然出去,说是去拜谒一个老朋友。过了一霎她又回来。接着就有人来回拜,狐女就到别的地方与来人说话。客人去了以后,狐女来与朱公说:“请上船吧!我送你过河。”朱公说:“你刚才还说不能过河,怎么现在又能过了呢?”女子说:“刚才我去拜谒的不是别人,是河神。我是为了你,特去拜见他,他限我十天回来,所以暂时依了你。”于是他们同船一起回了家。到了十天期限,狐女果然辞别朱公回去了。


\subsection{1.2.33   巧 娘}
\label{\detokenize{p00_u5176_u5b83/_u767d_u8bdd_u804a_u658b_u5fd7_u5f02:id78}}
广东有一个绅士姓傅,六十岁那年,生了一个儿子,名叫傅廉。长得很聪明,但却是天生的阉人,十七岁了,生殖器才像个蚕那么大。远近的人们都知道,所以没有人家把女儿嫁给他。傅公自料想宗嗣已绝,因此,日夜忧愁担心,也没有办法。

傅廉跟着家庭老师读书。一天老师出了门,街上来了个耍猴的,傅生出去观看,耽误了当天的功课。他约摸老师快回来了,怕挨体罚,就逃学跑了。

傅生一气跑到离家几里远的地方,见一个穿白衣的女郎带着一个丫鬟走在他的前面。女郎一回头,傅生见她美丽无比,迈着小步走得很慢,他就紧走几步,赶上了女郎。女郎回头对丫鬟说: “问问郎君可是往琼州去的?”丫鬟奉命来问傅生,傅生问她们有什么事。女子说:“你若是去琼州,有一封信,烦你顺道捎到我家去。我母亲在家里,还可以招待招待你。”傅生本来就没有一定去向,心里想,坐船到海上玩玩也可以,就答应了女子的拜托。女子把信交给丫鬟,丫鬟又交给傅生。傅生问她的姓名居处,女子回答:“姓华,住秦女村,距城北三四里路。”傅生到了海边,上了船就去琼州。

傅生按女子指点的路线到了城北郊,太阳已落山了。打听秦女村,却没人知道。又向北走了四五里路,天上已繁星点点,月亮也挂在天边了。眼前一片荒草野坡,不见一个走路的人,又没有人家。这时他心里又害怕,又为难。忽见路旁有座坟,心想暂且在坟旁坐一夜吧。又怕有虎狼,就爬到坟边一棵树上过夜。他蹲在树杈上,耳边只听得风声呼呼,草虫哀叫,心里忐忑不安,一时懊悔万分。

傅生正在树上,忽听树下像有人声。他低头一看,一座庭院清清楚楚就在下面。有一个美女坐在石头上,两个丫鬟打着灯笼伺候在两边。美女向左右看了看说:“今夜月明星稀,华姑送来的团茶可泡一杯来赏月。”傅生在树上想:这些一定是鬼!吓得毛发倒立,不敢大声喘气。忽然一个丫鬟说:“树上有人!”女子惊起说:“哪里来的大胆小子,敢偷看人!”傅生十分害怕,又没处逃藏,只好从树上滑下来,跪在地上求饶。那女子走近一看,马上变怒为喜,伸手拉起傅生,并肩坐下。傅生斜眼一看,这女子大约十六七岁,容貌体态十分艳丽,听口音很像当地人。女子问傅说:“你为何来这里?”傅生说:“给人家送信。”女子又说:“野外经常有强盗,露宿这里不安全。你若不嫌我家简陋,就将就着住几天。”便请傅生进了屋。这屋里只有一张床,女子命丫鬟铺两条被子在上面。傅生自惭残废,愿在地上睡。女子笑着说:“贵客光临,我女元龙哪敢一人高卧床上?”傅生不得已,只得和她睡在床上。但心里恐慌不安,一动不敢动。没多时,傅生觉女子伸过手来摸他,并轻轻捏了一下他的大腿。傅生佯装睡着了,好像没有觉得。又一会,女子钻到傅生被筒里,用手摇他。傅生仍然一动不动。女子便伸手去摸傅生阴处,刚一摸,手马上就停住了,大失所望,悄悄爬出了傅生的被筒,偷偷地哭了起来。这时,傅生又害怕,又羞惭,真是无地自容,只怨恨老天爷使他有缺陷。

女子起来,命丫鬟点上灯。丫鬟一看主人脸上有泪痕,惊问怎么了。女子摇了摇头说:“叹我命不好!”丫鬟站在床边,只看主人的脸,等主人吩咐。女子说:“可叫醒郎君,放他走吧!”傅生听了更加惭愧,而且担忧半夜三更,茫茫无去处。正思索的功夫,一个妇人推门进屋。丫鬟说:“华姑来了!”傅生偷眼一看,见这妇人五十开外的年纪,很有风度。这妇人见女子未睡,便问原因,女子没有回答。她又见床上躺着一个人,便问:“同床的是什么人?”丫鬟替女子回答:“夜里来了个少年郎借宿在这里。”妇人笑着说:“不知是巧娘的花烛之夜。”抬头又见女子珠泪未干,吃惊地问:“新婚之夜,不该悲泣,莫不是新郎有粗暴之处?”女子仍不回答,而且越发伤心。妇人掀开被子想看个究竟,不料一掀被子,却发现一封信掉在地上。她拿起来一看,惊奇地说:“这是我女儿的笔迹。”马上拆开信一着,非常诧叹。女子问妇人,她说:“这是三姐的家信。信中说吴郎已死,三姐一人无依无靠,日子不好过。”女子说:“这个少年曾说过替人送信,幸亏没打发他走。”

女人叫起傅生,问傅生信是从哪里来的。傅生把经过说了一遍。妇人说:“这么远麻烦你送信,我怎么报答你呢?”又看着傅生笑着说:“你怎么得罪了巧娘?”傅生胆怯她说:“我不知什么罪。”妇人又问巧娘,巧娘叹口气说:“可怜我自己活着的时候嫁了一个阉人,谁知死后又遇到一个阉人,所以悲伤。”妇人又看了看傅生说:“这么聪明漂亮的孩子,竟是阉人吗?这是我的客人,不能长时打扰别人。”于是领着傅生到了东厢房,伸手去傅生阴处检查,笑着说:“无怪巧娘哭泣!幸好还有根蒂,有办法治!”说着就点上灯,翻箱倒柜,找到一粒黑药丸,叫傅生吃下去,小声告诉他不要动,然后关门出去。

傅生独自一人躺在屋里,心想不知这药是治什么病的。将到五更天时,才一觉醒来,觉得肚脐下边一股热气直冲阴处,好像有什么东西垂在股下。用手一摸,自己已成了真正的男子汉。他心里又惊又喜,像是一下子封了公爵那样高兴。

第二天早上,窗户上刚看清窗棂的时候,妇人就进了屋,拿了烧饼给傅生吃,嘱咐他耐心坐着。她反锁上门,出来对巧娘说:“傅郎送信有功,得叫三娘来与他拜为干姊妹。暂且藏他几天,免得大家厌恶他。”说完就出门去了。

傅生被关在屋里,走来走去,觉得无聊,不时从门缝里向外瞧,像个关在笼子里的鸟。看见巧娘在院子里,想叫她过来说说自己的变化,又觉得惭愧,不好开口。挨到晚上,妇人才带了女儿回来。妇人把门打开就说:“闷煞郎君了吧?三娘快来谢过傅郎。”三娘犹犹豫豫走过来向傅生行了个礼。妇人叫傅生与三娘互称兄妹。巧娘笑着说:“叫姐妹也行。”说罢,就摆下酒一起坐饮。喝了几杯,巧娘就戏弄傅生说:“阉人,你也为美女动心吗?”傅生说:“瘸子忘不了穿鞋,瞎子忘不了看东西!”大家都一起笑了起来。

巧娘因为三娘一路辛苦,命人另安排房子,请三娘休息。妇人看了看三娘说:“叫他们兄妹俩在一屋里睡吧!”三娘羞答答的不好意思。妇人又说:“这个人看上去是个男子汉,实际是个女孩子,你怕什么?”催促他们早休息。偷着嘱咐傅生:“你可以明着算是我的干儿子,实则是我的女婿。”傅生非常高兴,拉着三娘上了床。这一夜他才初次接触女子,欢快无比。接着就在枕边问三娘:“巧娘是什么人?”三娘回答:“是个鬼。她才貌无人可比,但命运不好,找了个郎君姓毛,因生阉病,十八岁还不能过性生活。所以巧娘闷闷不乐,以至死去。”傅生怕三娘也是鬼,三娘就说:“实话告诉你,我不是鬼,是狐。因为巧娘一人住在这里没人作伴,我与母亲又没有家,就借住在这里。”傅生大为害怕。三娘又说:“你不必怕,我们虽然是鬼狐,但都不害人。”

从此,傅生与三娘天天住在一起,虽然知道巧娘是鬼,但心里却爱她娟娟美丽,恨没有机会表明自己的变化。傅生风雅温存,又非常诙谐,好说好笑,也很得巧娘喜欢。

一天,华氏母女要到别处走亲戚,临走又把傅生锁在屋里。他觉得闷得慌,就在屋里转来转去,隔着窗子喊巧娘。巧娘命丫鬟拿钥匙来试着开锁,试遍了所有钥匙,才碰巧开了锁。傅生附耳对巧娘说,要求单独在一起,巧娘就把丫鬟支走了。傅生挽巧娘上床拥抱。巧娘用手探傅生脐下,开玩笑说:“可惜可意的人这里少生了点东西。”活未说完,竟抓了满满一把,不禁惊奇地问:“为什么上次这东西小小的,而现在如此大了?”傅生笑着说:“上次害羞,所以见了你就缩回去了;这次因被毁谤很难堪,所以就像蛙怒一样鼓起来了。”两人欢好之后,巧娘生气地说:“今天我才知道华姑整日锁着你的原因!她们母女俩到处流浪无地容身,我借房子给她们住;三娘向我学刺绣,我毫无保留地教她,谁知她们竟如此忌恨!”傅生安慰劝解巧娘一番,巧娘始终耿耿于怀。傅生说:“这事一定不要说出去,华姑叫我不要让别人知道。”话还没有说完,华姑就推门而入。两人慌忙穿衣起床,华姑怒目圆睁,问:“谁开的门?”巧娘笑着坦然说是自己开的。华姑更怒气不息地唠叨没完,巧娘反唇相讥: “阿姥也太可笑了!他不是明为男子实为女子的吗?能干什么呢?”三娘见母亲与巧娘顶嘴,觉得不安,从中调解,才各自转怒为喜。巧娘虽然言词激烈,但事后仍屈意对待三娘。而华姑却日夜防范,巧娘与傅生不能接近,只是眉目传情而已。

一天,华姑对傅生说: “我女儿与巧娘姊妹俩都奉事了你,长此下去也不是办法,你应该回家去告诉父母,早订婚约。”便整理行装催傅生上路。二女相送,恋恋不舍,巧娘更是忧伤,双泪交流,如断珠滚落,哭个不止。华姑止住她们,拉傅生出了门。傅生回头一看,房子全没有了,只有一座荒凉的大坟。华姑送他上船,说:“你走后,我就带两个女儿去你县租房居住。若是不忘旧好,我们在李氏废园里等你迎亲。”傅生便回家了。

当时傅生逃学出走后,傅家到处寻找,他父母焦急万分。忽然见傅生回来,一家人都高兴得不得了,傅生大略说了他的经历。并提出与华氏订亲的事。他父亲说:“妖精说的话怎么能信?你能活着回来,就是因为你身体有缺陷,不然早死在外边了!”傅生说:“她们虽不是人类,但感情和人一样;也很漂亮聪明,娶进门来也不至于叫亲友笑话。”父亲没说什么,只是嗤笑而已。

傅生此后经常性欲发作,不安分守己。常与丫鬟私交,竟至白日淫乱,故意想叫他父母知道。一天被一个小丫鬟看见了,禀告了老夫人。夫人不信,就偷着去看。觉得十分奇怪,就叫了与儿子私交的丫鬟来问,她们都招认了。夫人心里非常喜欢,逢人就宣传儿子病好了,并要找世家大族给儿子说亲。傅生知道后,私下告诉母亲:“非华家姊妹不娶。” 他母亲说:“世上不缺少美女,为什么非要娶个鬼物呢?”傅生说:“儿若不是华姑,不能治好病。背弃了人家是不吉利的。”他的父亲同意了,于是派了一个男仆,一个女仆去打听。家人出城东四五里,找到李氏废园,果然见残墙竹树中,有缕缕炊烟。女仆一直进了屋,见华氏母女正擦拭桌椅,好像正准备迎接客人。女仆说了主人的意思,见到三娘,惊叹说:“这就是我家小主妇吗?我见了都喜欢,无怪我家公子整天神魂颠倒呢!”又问她的姐姐在哪里,华姑叹道:“她是我的义女,三天前忽然去世了。”随即备了酒菜招待来人。

傅家家人回来详细向主人说了情况,并说了三娘的相貌言谈,傅氏夫妇非常高兴。后又说巧娘死了,傅生听了伤心得想哭。到了迎亲的日子,傅生亲自问华姑,华姑说:“巧娘已在北方投生为人了。”傅生听了,抽抽搭搭哭了很久。

傅生虽然娶了三娘为妻,但仍不忘巧娘,凡是从琼州来的人,都请来向他们打听。有人说:“秦女坟夜间有哭声。”傅生觉得奇怪,就告诉了三娘。三娘沉吟半日,哭着说:“我辜负巧娘姐了。”傅生再三追问,三娘才说:“我与母亲来时,实没有告诉巧娘。现在悲伤啼哭的,莫非是巧娘姐姐?一直想告诉你,又怕母亲斥责。”傅生听了先是伤心而后转为欢喜,马上命人备了车,日夜兼程去找巧娘。到了坟上,进入坟内敲打着巧娘的棺木说:“巧娘!巧娘!我在这里!”一霎时,见巧娘抱着孩子从墓中出来。见到傅生,伤心凄楚,埋怨不止。傅生也哭了起来,探怀中问这孩子是谁的。巧娘说:“是你的小孽种,已生下三个月了。”傅生叹息说:“错听了华姑的话,使你们母子埋在地下,受苦担忧,我的罪过是不可推却的。”随即一起乘车、坐船回了家。傅生与巧娘抱着孩子见了父母,他父母一见,孩子身体健壮,一点也不像个鬼物,心里好生喜欢。姐妹俩相处和谐,孝敬公婆。后带傅父生病,请医生来治。巧娘说:“病已不能治了,魂已离开躯体了。”督促准备后事,备妥后傅父便去世了。

傅生的儿子长大后,很像傅生,而且更为聪明,十四岁就中了秀才。淄川高珩曾在广东听说过这件事,详细地名遗忘了,以后的事也不知道有什么结果。


\subsection{1.2.34   吴 令}
\label{\detokenize{p00_u5176_u5b83/_u767d_u8bdd_u804a_u658b_u5fd7_u5f02:id79}}
吴县有位县令,忘了他叫什么名字,为人刚毅梗直。吴县民俗最敬重城隍神,当地人用木头雕成神像,再披上锦制衣服,把神像打扮得栩栩如生。每到城隍神的诞辰,居民们都要敛资做神会,用华丽的车子拉着神像,在大街游行;打着五颜六色的旗帜和各种各样的仪仗,排着整整齐齐的队伍,一路吹吹打打,呜呜哇哇,跟着看热闹盼人挤满了大街小巷。时间长了,做神会成了风俗习惯,每年都不敢稍有懈怠。

有一次,这位县令外出,正好碰上做神会。便命游行的队伍停下,询问究竟,人们告诉了他。县令又得知做神会要花费大量的人力物力,不禁大怒,指着神像斥责说:“你是主管一个县的城隍神,如果冥顽不灵,就是糊涂昏庸的鬼,不值得人们供奉你;如你有灵,就应该知道爱惜民力,怎么拿这些无益的花费,来耗费民脂民膏呢!”骂完,命人把神像拉倒在地,打了二十板子。从此才破除了这个旧习。

县令为官清正无私,只是年纪轻轻,很贪玩。一年多后,有一次他偶然在官衙中上梯子掏屋檐下的鸟窝,失足掉了下来,摔断了大腿,不久就死了。人们听到城隍庙中传出县令愤怒的吵嚷声,似乎在和城隍神争执,连续几天也没停止。吴县的人不忘记县令的恩德,聚集到城隍庙里为他们调解。又另建了一个祠堂,供奉县令,争执声才消失了。县令的祠堂也称城隍庙,春秋按时祭祀,较原来的城隍神更加灵验。吴县至今还有两个城隍。


\subsection{1.2.35   口 技}
\label{\detokenize{p00_u5176_u5b83/_u767d_u8bdd_u804a_u658b_u5fd7_u5f02:id80}}
村里来了一个年轻的女人,大约有二十四五岁。她带着一个盛药的皮囊,到这里来行医看病。有的人去找她看病,她自己不能开药方子,要等到晚间问一问各位神仙。晚上,她把一间小房子打扫得干干净净,把自己关在里面。大伙儿围绕在门窗口,斜着头侧着耳朵静静地听,只听里面在小声私语,谁也不敢咳嗽一声。屋里屋外,黑洞洞的一片,没有一点动静。

大约到半夜的时候,忽然听到门帘微动的声音。女子在屋里说:“九姑来了吗?”一女子回答说:“来了。”又问:“腊梅也跟着九姑来了?”好似一个丫头的声音,说:“来了。”三个人话语间杂,唠叨起来没个完。过了一会儿,又听到帘钩馓动的响声,女子说:“六姑来了?”接着听到几个女子杂乱的说话声:“春梅也抱小郎君来了吗?”一个女子说:“这个顽皮的小家伙,怎么哄也不睡,定要跟来。身子有百十斤重,背着真累死人。”马上又听到女子殷勤的接待声,九姑的问讯声,六姑与姊妹们的寒暄客套声,两个丫头的互相慰劳声,小孩儿的嘻闹声,一齐嘈嘈杂杂地传出来。就听女子笑着说:“小郎君倒很好玩耍,老远的抱了个猫儿来。”接着说话的声音渐渐稀疏下来。门帘又响了一声,满屋里都喧哗起来,说:“四姑来得怎么这样晚?”听到一个女孩子细微的声音,说:“路足有一千多里,我同阿姑走了这么长时间才到。阿姑走得太慢了。”于是各人问寒问暖的声音,移动座位的声音,招呼着加座的声音,各种声音并作,喧闹满屋,有一顿饭的工夫才静下来。接着就听到女子问病求药的声音。九姑说当用人参,六姑认为当用黄芪,四姑说该用白术。协商一会儿,听到九姑叫人拿笔墨砚台来。不久,听到折纸的刷刷声,拔下笔帽扔到桌子上的丁丁声,隆隆的研墨声。接着就听到把笔投到桌几上的碰撞声,抓药包纸的苏苏声。过了一会。女子掀开门帘,招呼着病人的名字,把药包和药方一起递了出来。她转身入室后,立刻听到三位姑娘作别的声音,三个丫头的道别声,小儿哑哑的叫声,小猫儿的呜呜声,又一时并发起来。九姑的声音清晰悠扬,六姑的声音和缓苍老,四姑的声音娇滴宛转;以及三个丫头的声音,各有自己的特点,听着完全可以辨别得清楚。大家感到很惊讶,认为真是神来了。回家试试药方,也并不灵验。这就是民间流传的口技,特意借这种方法卖药罢了。但她的口技水平,也真够高超的了。

以前,朋友王心逸曾讲过:他在京城时,偶尔从集市上经过,听到一阵管弦音乐的声音,围着看的人好像一堵墙。他到跟前一看,是一位少年,用优美的声音在演唱。他手中并没有乐器,只用一个指头按着脸颊,一边按一边唱,听起来铿锵有声,与弦乐没什么差别。也是口技者的后代啊。


\subsection{1.2.36   狐 联}
\label{\detokenize{p00_u5176_u5b83/_u767d_u8bdd_u804a_u658b_u5fd7_u5f02:id81}}
有个姓焦的书生,是章丘县虹先生的叔伯弟弟,在一个园子里读书。一天夜里,有两个美人来到园子里,长得都非常俊美。一个是七八岁;一个十四五岁,走进焦生屋,就扶着桌子对焦生笑。焦生一看就知道是两个狐女,便非常严肃地叫她们走。大的女子说:“你胡子这么长,为什么没有一点大丈夫气?”焦生说:“我焦某生平不好二色。”女子笑着说:“真是个书呆子,你还遵守着那些陈腐规矩吗?下界的鬼神,凡事都拿黑的当白的,何况这床上的小事呢?”焦生再次怒斥她们出去。女子知道打动不了焦生,就说:“你是读书名士,我有一副对联,请你作下联,能对上下联我就走:‘戊戍同体,腹中只欠一点’。”焦生听罢,想了半天没有想出下联来。女子笑着说:“名士就是这样吗?还是我代你对上吧:‘己巳连踪,足下何不双挑?’”说罢,一笑而去。这件事是长山李司寇说的。


\subsection{1.2.37   潍 水 狐}
\label{\detokenize{p00_u5176_u5b83/_u767d_u8bdd_u804a_u658b_u5fd7_u5f02:id82}}
潍县李某有一所闲房要出租。一天,忽然来了一个老翁想租赁这座房子,每年愿出租金五十两银子。李某答应了。老翁走了后,却再没有音讯,李某便嘱咐家人把房子再租给别人。第二天,老翁来了,说:“已讲妥把房子租给我,为什么又要租给别人?”李某告诉他说,自己怀疑他不再来了。老翁说:“我马上就要搬来长住,之所以迟迟没搬过来,是因为我选择的乔迁吉日在十天之后。”老翁又先付给李某一年的租金,说:“这座房子就是空上一年,你也不要再过问了。”李某送他出去,询问他搬家的确切日期,老翁说了。后来又过了那日期好几天,老翁还是没有踪影。李某就去察看动静,只见大门从里边闩着,院里炊烟袅袅升起,人声嘈杂。李某大为惊讶,投进名帖拜访。老翁急忙迎了出来,将他请进屋内,满脸笑容,言谈和蔼可亲。

李某回来后,派人赠给老翁家一些东西,老翁盛情款待了派去的人,也回送了很多礼物。又过了几天,李某摆下酒席,请老翁聚会,二人谈得十分投机、欢快。李某问起老翁的家乡,回答是陕西。李某惊讶陕西距这里太远,老翁说:“你们这里是福地。陕西不能再住了,那里将要发生大灾难。”当时正天下太平,李某听了老翁的话也没在意,没有深问。又隔了一天,老翁下帖子回请李某。酒宴上的菜肴、摆设都非常奢侈华丽。李某更加惊异,怀疑老翁是贵官。老翁因为和他交往深了,便自称是狐仙。李某惊骇万分,从此后逢人便说。本县的官绅听说后,天天有人骑着马去拜访老翁,都想和他结交,老翁无不恭敬地接待,渐渐地和郡官也来往起来。但是,唯独本县县令要求见他,老翁总是借故推辞。县令又托李某先给介绍介绍,老翁仍旧不愿见。李某询问缘故,老翁离席凑近李某,悄悄地说:“您不知道,郡县令前世是头驴。现在虽然人模狗样的统治着老百姓,但却是一个见了钱什么都干得出来的无耻之徒!我虽然不是人类,也羞于和他交往!”李某便找托词告诉县令,说狐翁畏惧县令的神明,所以不敢见。县令信以为真,也就不再勉强了。这是康熙十一年的事。不久,陕西果然遭了兵乱。狐能先知先觉,看来是真的了。


\subsection{1.2.38   红 玉}
\label{\detokenize{p00_u5176_u5b83/_u767d_u8bdd_u804a_u658b_u5fd7_u5f02:id83}}
广平县的冯老头有个儿子,字相如,父子都是秀才。老头年近六十,性格耿直,但家中一贫如洗。几年间,老太太和儿媳相继死去,一切家务都得冯老头自己操劳。

一天夜里,相如坐在月光下,忽见东邻的女子在墙上向这边偷看。相如仔细看她,很漂亮;相如走近她,女子向他微笑;相如向她招手,女子不过来也不走开。再三请求,女子才从墙上爬梯子过来。于是,两人睡在了一起。相如问她的姓名,女子说:“我是邻家女儿,叫红玉。”冯生很喜欢她,和她约定永远相好,红玉答应了。从此,两人便夜夜往来。

大约过了半年多,一夜冯老头半夜起来,听到儿子房里有女子的说笑声。偷偷一看,见一个女子在里面。冯老头大怒,把儿子叫出来,骂道:“你这畜牲干了些什么事!咱家如此穷苦,你不刻苦攻读,反而学做淫荡之事。被人知道,丧你的品德;别人不知道,也损你的阳寿!”冯生跪下认错,流着泪说一定悔改。冯老头又呵叱红玉说:“女子不守闺戒,既玷污了自己,又玷污了别人!倘若这事被人发觉,丢丑的该不只是我们一家!”骂完了,气愤地回去睡觉了。红玉流着泪说:“父亲的训诲,实在让人羞愧。我们两人的缘份尽了!”冯生说:“父亲在,我不能自作主张。你如果有情,还应当忍辱为好。”女子坚决绝交,冯生就哭了起来。女子对他说:“我与你没有媒灼之言、父母之命,私相结合,怎么能白头偕老?此地有一个佳偶,你可以聘娶她。”冯生说家中贫穷,女子说:“明天晚上等着我,我为你想个办法。”第二天夜里,红玉果然来了,拿出四十两银子送给冯生,说:“离这儿六十里,有个吴村,村中卫家的姑娘,十八岁了,因为要的彩礼很高,所以还没有许配人家。你以重金满足他家的要求,一定会答应你的。”说完就告别走了。

冯生找机会告诉父亲,想到吴村相亲,但隐瞒了红玉赠送银子的事。冯老头担心家穷没钱,不让儿子去。冯生婉转地说:“只是去试探一下,看怎么样。”冯老头点头答应了。冯生就借了仆人和车马,到了卫家。姓卫的老头是个庄户人,冯生招呼他出来,和他说要向他提亲。卫老头知道冯生家是有声望的家族,又见他仪表堂堂,性情豁达,心里应允了,可担心他家不舍得花钱。冯生听他说话吞吞吐吐,明白他的意思,就把银子都拿出来放在桌上。卫老头才高兴了,请邻居的书生做中人,用红纸写了婚约。冯生进屋拜见岳母,见他们住的房子十分狭窄。卫女正依偎在母亲身后,冯生稍微斜眼看了她一眼,见卫女虽然是贫家妆束,但光彩艳丽,心中暗暗高兴。卫老头借房子款待女婿,又对冯生说:“公子不必亲自迎娶,等我为女儿多少准备些衣服嫁妆,用花轿送去。”冯生同他订下成亲的日期,就回去了。回家后,冯生骗父亲说卫家喜爱清寒门第,不要彩礼,冯老头也很高兴。到了日子,卫家果然送女儿来了。卫女过门后,勤俭孝顺、夫妻感情深厚。过了二年,生了一个男孩,取名福儿。

一次,赶上清明节,冯生夫妇两人抱着孩子去扫墓,遇到县里一位姓宋的绅士。姓宋的当过御史,因行贿罪被免职,回家隐居,但仍然大施淫威。这天,他也上坟回来,看见卫女很漂亮,问村里的人,得知是冯生的媳妇。姓宋的以为冯生是个穷秀才,用重金贿赂他,就可以打动他的心,便派家人去透口风。冯生乍听到这消息,顿时满脸怒气;转念一想。敌不过宋家的势力,便收敛怒容,换上笑脸,进去告诉父亲。冯老头一听大怒,跑出去对着宋家的家人,指天画地,臭骂了一通,宋家的人像老鼠一样逃跑了。姓宋的也生了气,竟派了好多人闯人冯生家,殴打冯老头和冯生,吵闹得像开了锅。卫女听到,把孩子扔在床上,披散着头发大声呼救。那帮家伙一涌而上,将她抬起,哄然离去。冯老头父子两人受了重伤,倒在地上呻吟;小孩子在屋里呱呱啼哭。邻居们都可怜他们,把父子两人扶到床上。过了一天,冯生能拄着拐杖起来了;老头却气得吃不下饭,吐血死了。冯生大哭,抱着儿子去告状,一直告到省督抚,不知告了多少遍,还是申不了冤。后来冯生听说妻子不屈从那姓宋的,死了,他更加悲痛。满肚子的冤恨,无处申诉。多次想去路上刺杀姓宋的,又怕他仆人多,儿子又没处寄托。日夜哀思,觉也睡不着。

一天,忽然有一个大汉来到冯家慰问。那人长着蜷曲的络腮胡子,四方脸,跟冯家从无交往。冯生拉他坐下,刚想问他的家乡姓名,客人突然问道:“你有杀父之仇,夺妻之恨,难道忘了报仇吗?”冯生怀疑他是宋家的侦探,只是用假话应酬着。客人气得眼眶像要裂开,怒睁双目,猛然起身,边往外走边说:“我以为你是一个君子,现在才知道是个不足挂齿的庸俗之辈!”冯生见他果然是个异人,忙跪下挽留他,说:“我实在是怕宋家的人来试探我。现在把心里话全部告诉你:我卧薪尝胆,伺机报仇,已经很长时间了。只是可怜我这襁褓中的婴儿,怕断了冯家香火。你是位义士,能否为我抚养孩子?”客人说:“这是妇人们的事,我做不到!你想托付别人的事,请你自己去做;而你想自己去做的事,我愿意替你去办!”冯生听了,跪在地上直磕响头。客人看也不看,就出去了。冯生追出去问他姓氏,客人说:“如不成功,不受人埋怨;成功了也不受人报答!”说完就走了。冯生害怕招来灾祸,抱着儿子逃走了。

到了夜里,宋家所有的人都睡了,有个人越过几道墙进去,杀了姓宋的父子三人和一个媳妇、一个奴婢。宋家拿了状纸告到官府,官府大惊。宋家咬定是冯生干的,官府便派衙役捉拿冯生。冯生逃得不知去向,官府更加相信是冯生杀的人。宋家仆人同官府衙役到处搜捕,夜里来到南山,听到小孩啼哭,跟踪过去,将冯生抓住,捆起来带回去。小孩哭得更厉害了,那帮人夺过孩子扔掉了,冯生怨恨得要死过去。见到县令,县令问冯生:“你为什么杀人?”冯生说:“冤枉啊!他是夜里死的,我在白天就出门了,而且抱着呱呱啼哭的孩子,怎么能越墙杀人?”县令说:“没杀人,你为什么逃走?”冯生哑口无言,无法辩解,被关进狱中。冯生哭着说:“我死了不可惜,孤儿有什么罪?”县令说:你杀人家的人多了,杀你的儿子,有什么可怨的!”冯生被革除功名,屡次受到酷刑,始终没有招供。

这天夜里,县令刚睡下,听到有东西打在床上,震震有声。县令吓得大喊大叫,全家都被他惊醒了。围过来用蜡烛一照,原来是一把锋利如霜的短刀,扎入床内一寸多,牢牢地拔不出来。县令看了,丧魂落魄,派人拿着刀枪到处搜索,没有一点踪迹。县令心中很胆怯,又认为姓宋的已经死了,没什么可怕的,就把案件呈报上级衙门,替冯生辩解开脱,把冯生释放了。

冯生回到家,瓮里没有一粒粮食,孤身一人对着空荡荡的屋子。幸亏邻居怜悯他,送点吃的来,才勉强度日。一想到大仇已报,冯生便露出笑容;又想到遭受这次惨酷的灾祸,几乎全家被害,又不断地落泪;接着又想到半辈子穷透了,又失去了儿子,断了香火,不禁在没人的地方失声痛哭,不能抑制。如此过了半年,官府对犯人的追捕也松懈了,冯生就去哀求县令,要求把卫氏的尸骨判给他。等把妻子的尸骨埋葬好回到家里,冯生悲痛欲绝,在空床上翻来复去,觉得没法再活了。

忽然听到有敲门的,冯生定神细听,听见门外有人正低声和小孩说话。冯生急忙起身,从门缝里往外看了看,好像是个女子。刚打开门,那女子便问:“大冤已经昭雪,庆幸你安然无恙!” 声音很熟悉,但仓猝之间想不起是谁。用烛光一照,原来是红玉,挽着一个小孩,在她腿边嬉笑。冯生来不及询问,就抱着红玉呜呜哭开了。红玉也惨然泪下,接着把小孩推到他面前说:“你忘了父亲了吗?”小孩牵住红玉的衣服,目光灼灼地看着冯生。冯生仔细一看,原来是福儿,非常吃惊,哭着说:“儿子是从哪里找到的?”红玉说:“实话告诉你,往日我说是邻居的女儿,是假的,我实际是狐仙。那天刚巧夜间走路,看见孩子在谷口啼哭,就抱到陕西抚养。听说大难已经过去,就带他来与你团聚了。”冯生挥泪拜谢。小孩在红玉怀中,像依偎在母亲怀里,竟然不再认得父亲了。

天还没亮,红玉就急忙起床,冯生问她干什么。她回答说:“我想回去。”冯生光着身子跪在床头,哭得抬不起头来。红玉笑着说:“我骗你的。如今家道新创,非早起晚睡不可。”接着就剪除杂草,清扫庭院,像男人一样操作。冯生忧虑家中贫穷,不能维持生活。红玉说:“只管闭门苦读,不要问家中盈亏,还不致于饿死人吧。”就拿出钱来买了纺织工具,租下几十亩田地,雇了佣人耕作。她自己扛着锄头除草,拉来藤萝修补房屋,天天如此。村里人听到冯生的媳妇如此贤慧,都愿意帮助她。大约过了半年,人丁兴旺,家里富裕了,冯生说:“已经是劫后余生了,多亏你白手起家。只有一件事没有安排妥当,怎么办?”问他什么事,他说:“考试的日期已经临近,秀才的资格还没恢复。”红玉笑着说:“我前几天已把四两银子寄给了学官,你的名字已重新登记上了。如果等你说,早就误事了!”冯生更觉得神奇。这次考试冯生中了举人,这年三十六岁,家中肥田连片,房屋宽阔深广。红玉轻盈柔美,好像随风可以飘去,但操作胜过农家妇女。虽然是严冬,又很劳苦,但双手还是细腻如脂,自己说二十八岁了,别人看上去就像二十才出头的人。


\subsection{1.2.39   龙}
\label{\detokenize{p00_u5176_u5b83/_u767d_u8bdd_u804a_u658b_u5fd7_u5f02:id84}}
河北省界有条龙坠落到村里,龙笨重迟缓地爬进一绅士家中,门口刚能容过它的身躯。龙挤塞进去,家里的人都吓跑了,登到楼上大声乱叫,轰轰隆隆地放着火枪土炮,龙才从他家里爬出去。门外积存着一洼泥水,水浅得不到一尺。龙爬进去,在水里翻滚,浑身涂满了泥污,用尽力气腾跃,只有一尺多就坠落下来。龙在泥水中盘曲了三天,苍蝇爬满了它的鳞甲。忽然下起大雨,龙才随着一声霹雳凌空而去。

房生同朋友一块登牛山,进入一座寺庙游览,忽然椽闯落下一块黄砖,上面盘着一条小蛇,像蚯蚓那样细小。蛇忽然旋转了一周,变得同指头一样粗;又转一周,已经同带子一样了。人们都很惊讶,知道是龙,忙一块跑下山。刚走到半山腰,听到寺庙中一声霹雳,震动山谷,天上的黑云像一顶盖子,一条大龙在云中矫健地翻腾着,一会儿就不见了。

章丘县的小相公庄,有一个农妇到田野去,遇到大风,风沙扑面,觉得一只眼被眯了,像是有根麦芒。又揉又吹,眼始终不好。翻开眼皮仔细看看,眼睛倒没有毛病,只在肉上有一条弯弯曲曲的红线,有人说:“这是一条蛰居的龙。”农妇听说忧愁害怕得要死。过了有三个多月,天忽然下起暴雨,忽然一声霹雳,蛰龙挣开眼皮飞走了,农妇没受一点伤。

袁宣四说:“在苏州时,一天正遇阴雨天气。忽然霹雳大作,众人看见一条龙从云间垂下,鳞甲张动着,龙爪上抓着一个人头,眉毛胡须看得清清楚楚,不一会儿穿入云端不见了,也没听说有谁掉了头的。”


\subsection{1.2.40   林 四 娘}
\label{\detokenize{p00_u5176_u5b83/_u767d_u8bdd_u804a_u658b_u5fd7_u5f02:id85}}
青州道陈宝钥公,是福建人。一天夜晚,他独自一人在书房里看书。有一个女子掀帘进来,陈公抬头一看,不认得这女子。但她长得艳丽绝世,身上穿着宫里的服装。女子笑着对陈公说:“冷冷清清的深夜里,独自一人坐着,不觉得寂寞吗?”陈公惊奇地问她是什么人,女子说:“妾家住不远,就在你的西邻。”陈公想她可能是个鬼,但心里却非常喜欢她,于是走过来挽着她的手请她一起坐下。女子说话言词风雅,陈公很是赏识,坐在女子身边拥抱地,女子也不太拒绝。她四下看看,说: “屋里没有别人吗?”陈公忙关上门说:“没有别人。”接着就催女子脱衣上床,但她却非常羞涩害怕。陈公替她脱下衣服,女子说:“妾虽然二十岁,但还是处女,粗暴了可不堪忍受。”于是二人欢好,床席上沾了点点血迹。既而在枕边谈心,女子自己说叫林四娘。陈公又问她的身世,她说:“我一生坚贞,现在已经被你轻薄够了。你有心爱我,就图个永远相好,何必再多问?”过了一段时间,鸡叫天明,她就起身走了。

从此,女子夜夜必来。每次来,二人都关上门对饮,说话很投机。谈到音乐韵律,林四娘都很精通。陈公说:“你一定会唱歌曲。”四娘说:“小的时候学过一些。”陈公请求她唱一曲听听,她说:“很长时间不唱了,音阶节奏多半都忘记了,唱了恐怕叫内行笑话。”陈公一再要求,她才低下头来敲打节奏,唱伊凉之曲,声调哀怨婉转。唱完后,便哭了起来。陈公也被她打动,心酸悲伤,上前抱着四娘安慰说:“你不要唱亡国曲调,令人抑郁。”四娘说:“音乐是表达人的感情的,悲哀的人不能叫他欢乐,就如欢乐的人不能叫他悲伤一样。”

陈公与四娘非常亲密,如同夫妇。时间长了,家人们都知道了,也都来房外偷听唱歌,凡听过她唱歌的人,没有不流泪的。陈夫人偷偷见到过四娘,怀疑人世间不会有这样妖丽的女子,认为不是鬼,就是狐,怕陈公被妖魅缠身,就劝说陈公与女子绝缘。陈公没有听,还想向四娘问个明白,便又一次问四娘的身世。四娘不愉快地说:“妾是当年衡王府的一名宫女,遭难而死,已有十七年了。因为你高雅义气,才与你相好,实在不敢害你。倘若你怀疑或者是害怕我,咱们就从此分手。”陈公马上说:“我不是怀疑,也不是害怕,既然我们相好到这个地步,不可不知道你的实情。”陈公又问起当时宫中的事。四娘回忆详述,有条有理,讲得很动人。说到王府衰落时,就哽咽哭泣,不能成声。

四娘夜里不大睡觉,每夜都起来念准提经和金刚经等。陈公问她说:“九泉之下能自己超度吗?”她回答:“能!妾想自己一生沦落,愿超度来生好好为人。”四娘还常常与陈公评论诗词,对不好的句子就提出批评,对好的句子就细声吟咏,情意风流,使人忘了疲倦。陈公问她:“你写过诗吗?”四娘回答说:“在世时也偶而写过。”陈公要求她赠诗一首,她笑着说:“儿女的诗句,哪能拿出来叫高手笑话?”

又住了三年,一天夜里,四娘来向陈公告别,面色凄惨。陈公一惊,忙问:“怎么了?”四娘说:“阎王因为我生前无罪,死后又没忘记念经,所以叫我投生到王家。今天就要离别,永远不能再相见了。”说罢,形容哀楚。陈公也掉了泪,马上摆酒为四娘送行。四娘一边饮酒,一边慷慨歌唱,歌词曲调哀伤凄凉,一字百转,到了悲伤处,哽咽不能成声。最后好歹总算唱完了一曲。四娘情绪不好,饮酒的兴趣也不高,起身徘徊,想要告别。陈公不忍离别,又强拉住她坐了一会,直到鸡唱天明,四娘才对陈公说: “一定不能再留了!你每每怪我不肯作诗,今日将要永别,应当写一首诗送给你,作临别纪念。”于是拿起笔来,一挥而就,并说:“心酸意乱,不能推敲,音节错乱,不要拿出去叫别人看。”说完,掩袖低头走去。

陈公送四娘出门,一转眼就不见了。陈公怅然了多时。看四娘写的诗,字体端正,书写整齐,便十分珍贵地收藏了起来。其诗是:“静锁深宫十七年,谁将故国问青天?闲看殿宇封乔木,泣望君王化杜鹃。海国波涛斜夕照,汉家箫鼓静烽烟。红颜力薄难为厉,惠质心悲只问禅。日诵善提千百句,闲看贝叶两三篇。高唱梨园歌代哭,请君独听亦潸然。” 诗中重复脱节,怀疑似乎有记错的地方。


\section{1.3   卷 三}
\label{\detokenize{p00_u5176_u5b83/_u767d_u8bdd_u804a_u658b_u5fd7_u5f02:id86}}

\subsection{1.3.1   江 中}
\label{\detokenize{p00_u5176_u5b83/_u767d_u8bdd_u804a_u658b_u5fd7_u5f02:id87}}
王圣俞南游,一夜船停在江心。睡下后,见江中明月如练,他睡不着,便让童仆为他按摩。忽听船顶上芦席发出声响,像小孩走路的声音,从船尾过来,渐渐接近船舱门口。王圣俞怀疑是盗贼,急忙起来询问童仆,童仆也听见有动静。二人一问一答间,见一个人伏在船顶上,垂下头来往舱里窥视。王圣俞很惊愕,拔剑呼叫仆人们,一船人都醒了。王圣俞讲了刚才看见的情形,有人怀疑他看花了眼。一会儿,脚步声又响了起来,众人四下里寻视,渺无人影,只有疏星皎月、漫漫江波而已。

众人正坐在船上,忽见一朵灯笼状的青色火苗冒出水面,随波飘游。渐渐靠近船时,火一下子熄灭了,却有一个黑人骤然冒出,屹立在江面上,用手攀着船走着。众人鼓噪呐喊,说:“一定是这个东西了!”想用箭射它。刚要开弓,黑人忽然钻进水中,看不见了。众人询问船家,船家说:“这里是古战场,鬼时常出没,没什么奇怪的。”


\subsection{1.3.2   鲁 公 女}
\label{\detokenize{p00_u5176_u5b83/_u767d_u8bdd_u804a_u658b_u5fd7_u5f02:id88}}
招远县有一个书生叫张于旦,性情放荡不羁,在一座荒庙里读书。当时,招远县的县官是鲁公,三韩人氏。他有一个女儿专好打猎。有一次,张生在野外遇到鲁公女,见她长得风韵娟美,恣态秀丽;身穿锦缎貂皮袄,骑着一匹小马驹,像画上的人一样。回到庙中,每每想起这女子的美貌,心里总是念念不忘。后来听说这女子忽然死了,张生悲伤得不得了。鲁公因为距老家很远,便把女儿的灵柩暂时寄存在张生读书的荒庙里。

张生因为和鲁公女有一面之缘,对她非常崇敬,犹如对神明一般。他每早都到鲁公女灵前烧香,吃饭时必定祭奠。每每举着酒杯对着鲁公女灵柩祝告说:“我才见了你一面,就常在梦里想到你,没想到你这玉一样的人竟然死了。现在你虽近在我的身边,但却如远距万里河山,何等遗憾。我活着要受礼法约束,你死了的人该无禁忌了吧!你在九泉之下有灵的话,应当珊珊走来,以安慰我的倾慕之情。”

张生日日祷告,将近半个月。一天晚\_上,他正在灯下读书,忽一抬头,见鲁公女含笑站在灯下。张生惊讶地起来询问,女子说:“感念你对我的一片真情,不能忘怀,所以不避私奔的嫌疑来与你相会。”张生大喜过望,二人于是共相欢好。此后,鲁公女没有一晚不来。她对张生说:“我生前好骑马射箭,以射獐杀鹿为快事,罪孽很大,死了以后无处可去。若是你真的爱我,烦你替我念金刚经五千零四十八卷,我生生世世永远不忘你。”张生恭恭敬敬地答应她的嘱托,从此常常夜里起来到鲁公女柩前捻着佛珠诵经。一次,偶然碰上节日,张生想带鲁公女一起回家过节。女子担忧自己腿脚没劲,走不动。张生要背着她走,女子笑着同意了。张生像背个小孩一样,一点不觉得重。此后,背着她走路就成了常事。张生考试时,也背她一块去,但必须夜里走。

有一年,省里开科考试,张生要去赴考,女子说:“你福气薄,去也是徒劳往返。”张生听了她的话就没去参加考试。又过了四五年,鲁公罢了官,穷得没有钱雇车把女儿的棺材运走,就打算就地埋了,但苦于没有坟地。这事张生知道后,就对鲁公说:“我有块薄地在庙旁,愿埋下你家女公子。”鲁公大喜。张生又张罗着帮助料理葬事。鲁公对张生非常感激,但也不知道张生是为了什么。

鲁公罢官回家去了,张生与鲁公女仍然欢好如初。一天夜里,女子依在张生怀里,哭得泪如雨下,对张生说:“我们相好五年,现在要分别了!我受你的恩义,几世都不足以相报。”张生惊讶地问她,她说:“承蒙你给我这九泉之下的人施加恩惠。现在你已为我念满了经数,所以我得以托生到河北卢户部家。若是你不忘今天,再过十五年的八月十六日,请你去卢户部家相会。”张生也伤心地哭着说:“我现在已三十多岁了,再过十五年,我就快入棺材了,相会又能怎样呢?”女子说;“到时愿给你当奴婢作为报答。”一会儿,她又说:“你可送我六七里路。这半路上有很多荆棘,我穿着长裙子难以走路。”说罢,抱着张生的脖子,张生便送她上了大路。

到了大路上,见路旁有许多车马,马上有骑着一人的,有骑着两人的;车已有的坐三人、四人的,甚至坐十几个人的不等。唯有一辆以金花为装饰挂着朱红绣帘的车子,只有一个老婆子坐在里面。老婆子见鲁公女来了,就叫着:“来了?”女子答应:“来了。”女子回过头来对张生说:“就送到这里,你回去吧!不要忘了我刚才说的话。”张生答应着。女子就走到车前,老婆子伸手拉她上了车,铃铛一响,车马就向遥远的地方走去了。

张生无精打采地回到庙里,将十五年后相会的日期记在墙上。想到念经还有这样大的作用,就更加诚心念经。他夜里做梦,梦见神人告诉他:“你志气很好,但须要到南海去。”问神: “南海多远?”神人说:“远在天边,近在眼前方寸之地。”醒后,他领悟了神人的意思,就念起菩提经来,修行更加诚心。

三年后,张生的大儿张政、二儿张明相继高中。张生虽一下显贵起来,可他仍然坚持修行。一次梦见一个青衣人请他,到了一座宫殿,见殿中坐着一个神,像是菩萨,迎接他说:“你行善可喜,可惜不能长寿,幸好请示了上帝,可以延长你的寿命。”张生跪下叩头,菩萨叫他起来坐下,请他喝茶,茶香犹如兰花。又叫童子领他到一个池子里去洗澡。池水很清,里边的鱼都看得很清楚。进入池中,水很温热,捧起来闻一闻,有荷叶香味。一会儿,他渐渐到了深处,失足陷入水底,水深没了头顶,一下子就惊醒了,大为惊异。从此,张生身体更加健壮,眼更明了,自己捋了一下胡子,白胡子都落了。又过一些时候,黑胡子也落了,脸上也没有了皱纹;又数月后,面目像儿童,跟十五六岁一样。还好游戏,也像个孩子,很不注意衣服饰物,礼仪小节。玩出了事,两个儿子就去救他。不久他夫人老病去世了,张生的儿子们要给他娶大户人家的女儿为继室。他说:“等我到河北去一趟回来再说。张生屈指一算,已经到了与鲁公女约定相会的时候了,便命人备马率仆人到了河北。一打听,果然有个卢户部。

早先,卢公生一女儿,生下来就会说话,长大了更加聪明漂亮,父母最喜爱她。一些富贵人家来求婚,女儿都不愿意。父母觉得奇怪,就问她,女儿详细说了生前的姻缘。大家给她算了算时间,大笑着说:“傻丫头!张郎现在已年过半百了,人事变迁,怕他尸骨都烂了;就是还活着,也老掉牙了。”女儿不听,还是等着。母亲见她决心不动摇,与卢公计谋,叫看门的不要通报客人,等过了约期,她就会绝望了。

果然不长时间张生就来访问,看门的不给他通报。张生不得已回到旅店,心里又不痛快又没有办法,就去郊外散心,也借此机会暗暗打听女子的消息。

托生后的鲁公女以为张生负约,终日哭泣,也不吃东西。母亲对她说:“张生不来,一定是去世了。就是活着,违背了盟约,错也不在你。”女子也不说话,终日躺在床上。卢公很忧心,也想知道张生到底是个什么样的人,于是托词郊游,正好遇到张生。一见是个少年,十分惊讶,互相谈了几句话,见张生风流潇洒,卢公很喜欢,便邀他到家里去。张生刚想问话,卢公忽然立起,叫客人等一下,自己匆匆进内房告诉了女儿。女儿很高兴,自己奋力起床,出来一看,见长得不大像张生,就哭着回房了,埋怨父亲诳她。卢公极力表明这个人就是张生,女儿也不说话,只是哭。卢公出来,情绪懊丧,对张生也不热情接待了。张生问:“贵府有当户部的吗?”卢公随便应了一声 “有”,眼睛向别处看,似乎不觉得有客人在。张生感到有些慢待自己,就告辞走了。

鲁公女只是哭,几天就哭死了。张生夜里做梦,见鲁女来对他说:“来找我的果然是你吗?你年纪相貌都变了,见了面竟没有认出。现在我已忧愁而死,烦你赶快到土地祠招回我的魂,还能复活,晚了就来不及了。”张生醒来,急忙去叫卢户部的门,果然他女儿已经死了两天了。张生悲恸欲绝,进屋吊唁一番,把梦中的事告诉了卢公。卢公听从了他的话,急忙去土地祠招回了女儿的魂。又掀开被子,抚摸着女儿的尸体,一面叫女儿的名字,一面祷告。不多时,便听到女儿喉咙里咯咯地响,见她朱唇一张,吐出一口冰块样的痰,渐渐呻吟起来。卢公高兴得不得了,敬请张生客厅就坐,命人摆上酒宴,细问张生门第,才知道他家是巨族大户,越发高兴。于是选择良辰吉日,命女儿与张生成了亲。

张生在卢公府住了半个月,便带着妻子回家,卢公亲自护送女儿,并在张府住了半年才回家。

张生夫妇住在一起,真像小两口一样。很多人认为鲁女的儿媳是她婆婆,因为她儿媳都近四十的人了。

卢公回家后,过了一年就死了。儿子很小,被豪强人家欺侮,家产几乎都被人霸占了。张生夫妇就把他接了来养着,成了一家人。


\subsection{1.3.3   道 士}
\label{\detokenize{p00_u5176_u5b83/_u767d_u8bdd_u804a_u658b_u5fd7_u5f02:id89}}
韩生,是大户人家的子弟,为人好客。同村有一个姓徐的,经常在他家喝酒。

一次,韩生和徐某又在家里宴饮,门外忽然来了个道士,手托着饭钵化缘。仆人们给他钱和粮食却不要,也不走。仆人生气地走开了,不再理他。韩生听见门口击钵的声音响了很久,叫来仆人询问,仆人向他禀报了事情经过。话还没说完,道士已径直走了进来。韩生让他入座,道士举手向主客略一致意,便坐下了。韩生简略地问了一下他的来历,得知他住在村东破庙中,便说:“道长什么时候来到村东庙里住下的?我竟一点也不知道,太缺主人之礼了!”道士回答说:“小道刚来此地不久,跟人没什么交往。听说您慷慨好客,所以来求杯酒喝。”韩生听说,便斟上酒,让道士举杯畅饮。徐某见道士穿得又脏又破,很瞧不起,傲慢地不大理睬他。韩生也把道士当作一般的江湖食客对待。道士一连喝了二十多杯,告辞离去。从此后,韩生每次宴会,道士总是不请自到,见到饭就吃,见到酒就喝。次数多了,韩生也多少有些厌烦起来。一次在酒席上,徐某嘲笑道士说:“道长天天当客人,自己难道一次东道主也不做吗?”道士笑着说:“我和你一样,都是双肩托着一张嘴罢了!”徐某大为羞惭,无言可对。道士又说:“话虽然这样说,但小道很早就诚意想邀请了。小道定当尽力准备几杯水酒,聊以报答。”喝完后,道士嘱咐说:“明天中午,敬请光临”。

第二天,韩生和徐某一起去村东庙中,怀疑道士什么也没准备。一路走去,见道士已在途中等候。边谈边走,已到庙门。进门一看,只见房舍院落,焕然一新,楼台亭阁,绵延一片。韩、徐二人大吃一惊,说:“很久没来这里,这是什么时候建造的?”道士回答说:“刚竣工不久。”等走进屋子,又见陈设富丽堂皇,连富贵大家都没这般气派。二人不禁肃然起敬。入席坐下后,往来上菜斟酒的都是些十几岁的聪明小童,穿着锦衣红鞋。酒香菜美,极为丰盛。饭后,又上了些水果,都很珍奇,叫不上名来,盛在用水晶、玉石制作的盘里,光华晶莹,照亮了桌几、床榻。又用大玻璃杯盛酒,杯子周长一尺多。这时,道士命小童说:“去叫石家姐妹来!”小童去了不一会儿,便见有两个美人进来。一个细高,犹如风摆弱柳;另一个身材稍矮,年龄也小。二人都妩媚多姿,俊俏无比。道士命她们唱歌劝酒。年小的那个击节而歌,高个的吹着洞箫伴奏,声音清细嘹亮。一首歌唱完,道士举杯劝酒,喝完后,命小童都斟上,回头看着二女说:“美人很久没有跳舞了,还能跳吗?”话刚说完,便有童仆在地上铺下了毛毡,两个美人在毡上翩翩对舞起来,只见长袖飞舞,香气四散。舞完,娇媚地斜倚在画屏上喘息。韩、徐二人看得神魂颠倒,不知不觉喝得大醉。道士也不管他们,自己举起杯来一饮而尽,站起身对两个客人说;“请你们自斟自饮吧。我去稍休息一会,马上就来。”说完便走了。南屋墙下摆着一张精美的螺钿床,两个女子铺上锦褥,扶着道士躺下。道士拉着高个的那个同床共枕,命年小的在一边给他挠痒。韩、徐二人见此情景,十分不平。徐某大叫:“道士不得无礼!”跑了过去,要扰乱他们,道士急忙起来逃走了。徐某见年小的美女还站在床下,乘着酒意把她拉到北边一张床上,公然拥抱着她躺下了;见道士床上的美人还睡在被窝里,便对韩生说:“你怎么这样傻啊!”韩生听了,径直上了道士的床,想跟那美女亲热,却见她沉沉睡去,扳也扳不动,便搂抱着她睡着了。

天亮后,韩生一下子从醉酒和睡梦中醒过来,觉得怀中有个东西非常冰冷,一看,自己原来是抱着块长条石躺在石阶下;急忙看看徐某,见他还没醒过来,头枕着块茅坑里的臭石头,呼呼大睡在一个破厕所里。韩生忙踢醒他,二人都非常惊异,四下一看,只有一院荒草、两间破房而已。


\subsection{1.3.4   胡 氏}
\label{\detokenize{p00_u5176_u5b83/_u767d_u8bdd_u804a_u658b_u5fd7_u5f02:id90}}
河北省有一个大户人家,想请一名教书先生。忽然来了一个秀才,找上门来推荐自己。主人就请他进来谈。此人说话开朗直爽,主客谈得很投机。秀才自我介绍姓胡。主人便聘请他来家教书。

胡氏教书很勤苦,学识也很渊博,比一般教书先生好得多。就是好出馆游玩,并且常常深夜才回来。大门关着,不听见敲门,人已进屋了。于是家人都怀疑他是狐。但仔细观察,又看不出他有什么恶意,所以主人仍然按常礼对待他,不因他是狐而怠慢。

胡氏知道主人有一个女儿,想向主人求婚,几次向主人示意,主人都佯装不懂。有一天,胡氏向主人说要出去办点事,主人同意后他便走了。第二天,有个客人来拜访主人,拴一头黑驴在门外。主人请他进屋,这人年约五十多岁,衣服鞋袜光鲜洁净,谈吐风雅。宾主落坐后,来人才说明是给胡氏提亲的。主人听了沉默很久才说:“我与胡先生已是莫逆之交,何必非成为亲戚不可呢?况且小女已许配人家了,请代我转告先生婉谢他的好意。”客人说:“我知道女公子并没许亲,何必这样坚决推辞呢?”客人再三恳求,主人执意不肯。客人有些不高兴地说:“胡先生也是世家大族,怎么就配不上你家呢?”主人就直截了当地说:“实话实说吧!因为我们不是同类。”客人听了大怒,主人也生了气,两人争吵起来。客人立起用手抓主人,主人就命家人用棍子把他打了出去。客人驴子也没骑,就跑了。众人见这驴毛是黑的,长着大耳朵,长尾巴,个头很大,可是牵它不动;一赶它,驴就随手倒下了,却是个正在鸣叫的草虫。

主人因为客人走时很气愤,估计肯定回来报复,所以叫家人作了戒备。第二天果然有大批狐兵来侵犯。有骑兵,还有步兵;有持戈的,有拿弓箭的。人喊马叫,声势浩大。主人不敢出去。狐兵扬言要用火烧屋,主人越发害怕。这时,有个大胆的家人带领大伙叫喊着冲了出去,两相撕打,飞石放箭,各有伤亡。狐兵渐渐败退,纷纷逃走,丢弃了一些刀剑在地上,亮如霜,走近拾起一看,都是些高粱叶子。众人笑着说:“就是这么大本事吗?”但仍怕它们再来,加强了戒备。

第三天,家人正聚集在一起议论,忽见一个巨人从天而降,高一丈多,身粗好几尺,挥舞着一把像门扇一样的大刀,追着众人砍杀。众人一见便拿石块打他,放箭射他,一打那巨人就倒下死了。走近一看,原来是一个用草扎的哀杖。众人更加不怕狐兵了。

这一仗后,狐兵三天没再来,家人们也稍有懈怠。一天主人正上厕所,忽见狐兵朝他乱箭射来,都射到他的腚上。主人大叫,命家人来反击,狐兵才退去。主人拔出腚上的箭一看,竟是些黄蒿杆子。以后月余,小规模的经常打来打去,虽无有什么大害,却也日夜不宁,需要天天防范,主人很是苦恼。

一天,胡氏亲自带狐兵来犯。主人也亲自出面。胡氏见主人出来了,有点不好意思,就躲在众狐后面。主人叫他,他才出来相见。主人对他说:“我自认为没有对你失礼的地方,为什么三番五次兴兵动众来扰乱我?”众狐正要朝主人放箭,胡氏立即制止住。主人便走向前去握住胡氏的手,请他进屋,并设宴款待。主人从容地说:“先生是明白人,一定能理解。以我们之间的友情,我能不愿与你结亲吗?可是先生的房子,车马,都不与我们人类一样,小女嫁过去,先生也会认为不合适。况俗话说:‘强扭的瓜不甜’。先生看怎么好呢?”胡氏觉得很惭愧。主人又说:“没有关系,咱们交情仍在。你若不嫌我们是尘俗之辈,我有个小儿子,今年才十五六岁,愿与你们结亲,不知有合适的女孩子没有?”胡氏高兴地说:“我有个小妹妹,年纪比小公子小一岁,长得很不丑,愿嫁给小公子,不知同意吗?”主人起身拜谢,胡氏也答拜。于是饮酒谈心,以前的不快顿时消除。主人又命家人摆酒招待同来的狐兵。上下人等皆大欢喜。接着又问胡住在哪里,准备去纳聘礼。胡氏谢绝了,说日后自会送来。一直喝到黄昏,胡氏才大醉而归。从此后主人家才安安静静地过日子。

一年的工夫过去了,胡氏一直没有来。主人怀疑他忘记了婚约,但还是坚持等着。又过了半年,胡氏忽然来了,互道寒暄以后,胡氏说:“小妹已长大成人,请你选个吉日过门成亲,好来侍奉公婆。”主人大喜,随即一同定了日子准备成亲。

到了那天夜里,果然有车马人等来送新人,新娘的嫁妆非常丰盛,摆了满满一新房。新娘美丽异常,见了公婆温顺有礼。主人夫妇极为高兴。胡氏与一个弟弟来送亲,弟弟的言谈举止也很风雅,饮酒海量,兄弟二人一直喝到天明才走。

新娘进门后,能预知年成丰歉,平时公婆都听她的主意居家过日子。胡氏兄弟及亲家婆,还时常来走亲戚,人人都见过他们。


\subsection{1.3.5   戏 术}
\label{\detokenize{p00_u5176_u5b83/_u767d_u8bdd_u804a_u658b_u5fd7_u5f02:id91}}
有一种用桶耍的把戏,桶的大小可放进一个升,没有底,中间是空的,跟通常耍把戏用的桶一样。耍把戏的人把两张席子铺在街上,把一个空的升放进桶里。一会儿取出来,就有满满一升米,再把米倒在席子上。如此不断地用升取米、倒下,顷刻间,两张席上都满了。然后再用升把席上的米一一量进桶里,完了后一举桶,仍然是空的。这个把戏奇就奇在米取得多。

利津县人李见田,在颜镇一处陶瓷场里闲逛,想买一个大瓮。跟卖陶人讲了会价钱,买卖没成便走了。到了夜晚,卖陶人窑中本来还有没出窑的六十多个瓮,可等打开窑一看,瓮全都不见了。卖陶人大惊,怀疑是李见田干的事,便到他门上哀恳,李见田推辞说不知。主人再三哀求,李见田才说:“是我替你出了窑,一个瓮也没损坏。魁星楼下的那些不是吗?”主人依言去看了看,果然瓮都在。魁星楼在颜镇的南山,离陶场有三里多路。卖陶人雇了人把这些瓮运回去,连运了三天才运完。


\subsection{1.3.6   丐 僧}
\label{\detokenize{p00_u5176_u5b83/_u767d_u8bdd_u804a_u658b_u5fd7_u5f02:id92}}
济南有一个和尚,不知叫什么名字。他赤着脚,穿着百衲衣,每天都到芙蓉街、大明湖各酒店念经化缘。人们给他酒饭、钱粮、米面,他都不要。大家问他要什么,他也不回答。终日没见他吃过一口饭。有人劝他说:“师傅既然不吃荤酒,应到乡下去化缘,为什么天天在这腥膻的地方呢?”和尚仍闭眼念经,耷拉着一指多长的睫毛,好像什么也没听见。过了一会儿,人们又这样劝他。和尚瞪着眼睛厉声说:“我就要这样化缘!”说罢又念经不止。他念的时间长了就自己走去。有些好奇的人跟在他后面,要问个究竟,为什么必定这样化缘,可和尚始终不应声;再三问下去,他又厉声说:“你们不懂,老僧就是要这样化!”

又过了好几天,和尚忽然出了南门,躺在路旁像僵死了一样。一躺三天,一动也不动。当地人怕他饿死,把他抬到城墙边,都劝他到别处去,若要钱就给钱,若要饭就给饭。但和尚一直闭着眼,一句话也不说。大家一齐摇着他对他说,和尚大怒,从百衲衣中抽出一把短刀,一刀剖开自己的肚子,用手伸到肚子里掏出肠子理一理放在路上,于是气绝身亡。大家都害怕了,赶快报告了官府。官府来草草埋葬了他。

后来,包和尚尸体的席子被狗扒了出来。人们用脚踏踏,好像是空的。打开一看,死尸没有了,席子原样捆着,像个空茧壳一般。


\subsection{1.3.7   伏 狐}
\label{\detokenize{p00_u5176_u5b83/_u767d_u8bdd_u804a_u658b_u5fd7_u5f02:id93}}
有个太史,遭了狐祟,生了重病。求神、画符,办法都用尽了,仍然不见效。于是就请假回家,想逃避一下。可是太史前头走,狐就在后面跟着,太史更加害怕,但又无计可施。

一天,他走到涿县城门外,停下来休息。忽听有个医生摇着铃走来,自己喊着能伏狐。太史命人请他来治狐。这个医生就给了他药,实则是房中之术。催着他吃了药,让他去与狐性交。太史此时性欲旺盛,狐忍受不了,要逃又逃不走,哀求作罢。太史不听,反而越发猛烈,狐设法脱身,苦无办法。过了会儿,听不到狐的声音了,一看,已经现原形死了。

早先,我们乡里某书生,素来被看作是秦之嫪毒,自己说生平没得到过一次满足。一天,夜宿孤馆,四面没有邻舍。忽然来了一个逃女,没有开门就进屋来了。书生心想一定是个狐女,就欣然同她就寝。上床之后,衣裤未脱,就直接交欢。狐女惊喊疼痛,吱吱乱叫,忽地像老鹰脱钩一样从窗子里逃走了。书生还向窗外哀求她再回来,却早已无影无踪了。这真是伏狐猛将,应该张榜为业。


\subsection{1.3.8   蛰 龙}
\label{\detokenize{p00_u5176_u5b83/_u767d_u8bdd_u804a_u658b_u5fd7_u5f02:id94}}
於陵有一个掌管收天下奏状的银台,姓曲,他经常在楼上读书。一天正当阴雨天气,见一个小东西,身上发着像萤火虫一样的光,蠕蠕地爬动。它经过的地方,留下一道黑黑的痕迹,渐渐又盘在他的书上,书也焦了。曲公想可能是条龙,就双手捧着书送到外面去。

到了门外,曲公端着书等了很长时间,可小东西盘在书上一动不动。曲公说:“难道你认为我不恭敬吗?”于是端着书又回到屋里,仍旧放在书桌上,整了整衣帽,恭恭敬敬地作了个揖,再端起书来送出去。刚刚到屋檐下,就见那小东西昂首伸尾,离开书飞去:嗤嗤有声,带着一缕白光;几步远以后,回过头来朝着曲公,就已头大如瓮,身子数十围了。接着又一翻身,霹雳一声,腾云驾雾飞上天空。曲公回到屋里查看它爬出的地方,原来是曲曲弯弯从书箱里爬出来的。


\subsection{1.3.9   苏 仙}
\label{\detokenize{p00_u5176_u5b83/_u767d_u8bdd_u804a_u658b_u5fd7_u5f02:id95}}
高明图任彬州知州时,发生了这样一件事。有一个姓苏的民女在河边洗衣服,河中有一块大石头,女子蹲在石头上。有一缕青苔,碧绿柔滑,非常可爱,在水面上荡漾,围着石头飘动了三圈。民女看了后心里一动,回家以后就怀了孕,肚子一天天大了起来。她母亲私下问她,女子把实情告诉了母亲,母亲一时也弄不明白。几个月后,竟生下了一个男孩。家人想偷着把他扔掉,但女子不忍心,藏在柜子里养着他。女子也决心不出嫁,以表明好女不嫁二夫。然而没有丈夫就生孩子,总归是不光彩的事。孩子已长到七岁了,还从来没让他出来见外人。

一天,儿子忽然对母亲说:“儿已渐渐长大了,怎么能长久关在家里呢?我要走了,不能连累母亲一辈子。”问他到哪里去,他说:“我不是人种,我要腾云上天。”母亲哭着问他什么时候回来,他说:“等到母亲归天时,儿才来。我走了以后,你若需要什么,就打开藏我的柜子要,要什么有什么。”说罢,拜别母亲就走。母亲出门看时,已无影无踪了。女子回去告诉她的老母亲,老母也觉得很奇怪。

此后,女子坚守旧志,一直没有嫁人,与母亲相依为命。但是家境却越来越困难了,有时吃了上顿没下顿。女子忽然想起儿子临走时的话,打开柜子,果然有米有面,于是烧火做饭叫母亲吃。后来缺什么就要什么,有求必应。

又过了三年,女子的母亲因病死了。一切丧葬用品,都是取自柜中。葬了母亲后,女子独自一人过日子,一直过了三十年,从未接近过男人。

一天,邻居一个妇人去女子家借火,见她一个人坐在空房里,与她说了一会话就走了。过了一会,忽见一团彩云围着女子的房子,清清楚楚像盖子一样。云中立着一个人,穿着华丽的衣服,仔细一看,就是苏家的女子。转了很长时间,就渐渐升高看不见了。邻人都非常疑惑,到她屋里一看,见她打扮得非常漂亮,端端正正坐在那里,已经没有气了。大家因为她孤苦一人,正议论怎么给她出殡,忽然一个少年进来。这少年长得英俊魁伟,向着众人一一道谢。邻居们也听说过这女子曾有个孩子,所以也不怀疑。少年拿出钱来埋葬了母亲,并在墓旁栽上两棵桃树,就告辞而去,走了几步就脚下生云,然后就不见了。

后来,这两棵桃树结的桃,甘甜味美,当地人都叫它“苏仙桃树”。年年枝叶繁茂,硕果累累。在这里做官的,每每拿着这桃馈赠亲友。


\subsection{1.3.10   李 伯 言}
\label{\detokenize{p00_u5176_u5b83/_u767d_u8bdd_u804a_u658b_u5fd7_u5f02:id96}}
书生李伯言,是沂水人,为人刚正不阿,很有胆气。一天,他忽然生了重病,家人要给他吃药,李伯言阻止说:“我的病不是药能治好的!阴间里因阎王一职空缺,要让我暂时去代理。我死后不要埋葬,等着我复生。”这天,他果然死了。

李伯言死后,他的阴魂被一队骑马的侍从领着,进入一座宫殿。有人向他献上王服。皂隶书吏们都肃穆地站在两边。李伯言见桌子上积攒了厚厚一叠卷宗,便立即开始审案。第一件案子,被告是江南某人,经查这人一生共奸淫良家妇女八十二人。把他提来一审问,证据确凿。按阴间法律,应受炮烙刑罚。只见大堂下竖着一根铜柱子,有八九尺高,一抱粗。柱子中间是空的,里面烧着炭,里外烧得通红。一群鬼卒们用铁蒺藜抽打着那人,逼他往铜柱上爬。那人手抱脚盘,顺着柱子往上爬。刚爬到顶,铜柱内烟气飞腾,轰的一声,像放了个爆竹,那人从顶上一下子摔下来,蜷曲着趴在地下。过了一会儿,他才苏醒过来。鬼卒又打他,逼他再爬,爬到顶又摔下来。如此三次,那人渐渐被烧成了一团黑烟,慢慢散去,再也聚不成人形了。

另一件案子,被告竟是李伯言同县的亲家王某,奴婢的父亲告他强夺亲生女儿。原来,有一个人要卖奴婢,王某知道那奴婢来路不明,但贪图价格便宜,还是买下了。不久,王某暴病而死。隔了一天,王某的朋友周生忽然在路上遇到他,知道是鬼,吓得忙跑回自己的书斋,王某竟也跟着进去了。周生害怕地祷祝着,问他要干什么。王某说:“想麻烦你到阴间里给我作证!”周生惊恐地问:“什么事?”王某说:“我家那个奴婢,明明是我出钱从别人手里买的,现在被奴婢的父亲诬告是强夺的。这件事你亲眼见过,所以请你去给我说句话,没有别的事。”周生坚决不去。王某走了出去,说:“这事恐由不得你!”不久,周生果然死了,一同去阎王殿受审。李伯言一见被告是亲家王某,心里产生了袒护的念头。这个念头刚一出现,忽见大殿上冒出火苗,火焰汹汹地烧着屋梁。李伯言大惊,急忙站了起来。一个书吏连忙告诉他说:“阴间和人世不同,容不下一点私念。您赶快打消别的念头,火就自己熄灭了!”李伯言忙聚精会神,收回私念,火光一下子没有了,便接着审案。王某与奴婢的父亲争执不休,李伯言便审问周生,周生如实说了。李伯言判王某明知故犯,应受笞刑。打完,派人送他们返阳。周生与王某都在三天后醒了过来。

李伯言审完案子,坐着车返回来。半路上见一群缺头断足的鬼,足有好几百,迎面跪在地上哭泣。李伯言停下车子询问缘故,原来都是些死在异乡的鬼,想回故土,又怕沿途关隘阻挡,所以乞求阎王给个路条。李伯言说:“我只代理三天职务,现在已经卸任了,怎么帮你们呢?”众鬼说:“南村的胡生,将要建道场,您替我们嘱托他,这事就能办到。”李伯言答应了。到家后,随从们都回去了,李伯言就醒了过来。

胡生,字水心,跟李伯言关系很好。他听说李伯言又活了过来,便来探望。李伯言突然问他:“什么时候建道场?”胡生惊讶地说:“战乱之后,我妻子儿女侥幸得以保全。过去我跟妻子谈起过这个心愿,但并没跟任何人说。你怎么知道了?”李伯言详细告诉了他众鬼的请求。胡生叹息说:“没想到卧室里的一句话,竟传到阴司里去,真是可怕啊!” 便恭敬地答应下走了。第二天,李伯言去王某家。王某还在疲惫地躺着,看见李伯言来了,肃然起敬,再三感谢他庇护了自已。李伯言说:“阴司里不能徇情。你的伤好些了吗?”王某说:“没什么了,只是挨打的地方化了脓。”又过了二十多天,王某才好了,屁股上的烂肉都掉了下来,只留下一片像是棍伤的疤痕。


\subsection{1.3.11   黄 九 郎}
\label{\detokenize{p00_u5176_u5b83/_u767d_u8bdd_u804a_u658b_u5fd7_u5f02:id97}}
何师参,字子萧,他的书斋在苕溪东边,门口对着一望无际的原野。有一天傍晚,他出门去散步,看见一个妇人骑着驴走过来,一个少年跟在后面。妇人年纪大约五十多岁,意态不俗。再看少年,年约十五六岁,长得非常俊雅,胜过美丽的女孩子。何子萧素有同性恋的癖好,看到这个少年不禁出了神,直着眼,翘着脚,一直目送他走了老远才回了书斋。

第二天,何子萧一早就出门等那个少年。直到夜幕降临时,少年才又从他门前经过。何生忙上前热情相迎,面带笑容问少年从哪里来。少年回答说:“从外祖父家来。”何生又殷勤地请少年到屋里休息一下,少年推辞说没有时间。何生一定坚持要他坐一会,扯住不放。那少年才勉强进屋。但只坐一会儿,定要告辞,不能再留。何生只好拉着少年的手邀他出门,还殷切地嘱咐再来玩。少年只是唯唯答应着,就走了。

从此后,何生如饥似渴地想念那少年,天天来来去去,心神不定地在门口眺望,脚不停步。一天,太阳刚落了一半的时候,少年忽然来了。何生大喜,赶快向前迎进书斋,急忙命童子摆酒共饮。询问少年姓名,回答说:“姓黄,排行第九,因为年纪小还没有名字。”何又问:“为什么从这里来来去去这样频繁?”少年回答:“母亲在外祖父家,常生病,所以得经常去看她。”酒过几巡,九郎就想走。何生拉住他的手,挡住他的路,又去上了门锁。九郎无可奈何,红着脸只好又坐下。两人点上灯共同说话,九郎温柔得就像个女孩子。何生言词中有戏语时,他便羞答答地脸朝着墙。不多时,何生就拉他一同睡觉,九郎不同意,坚持说两人在一起睡不着。何生勉强再三,九郎解开衣服穿着裤子躺下了。何生吹了灯,过一会就过去与九郎同在一个枕头上,又拥抱他,要求与他私交。九郎生气地说:“我以为你是风雅之士,才住了下来。你这种行为,真是禽兽之爱了!”一会儿,天上晨星闪闪,九郎便起身走了。

何生唯恐九郎绝情不来,还是天天等他,无目的地走来走去,望穿北斗。又过了几天,九郎才又来了。何生高兴地迎接他,并向他道了歉意。强拉入斋,共坐笑谈,偷偷庆幸他不念旧恶。过了一会,上床睡觉,何生又苦苦哀求纠缠九郎。九郎说:“缠绵之意,我已铭记在心。但是互相亲爱,何必一定要这样呢?”何生仍甜言蜜语纠缠他,并且说只要求亲近亲近。九郎无奈,只好同意。可等九郎睡着了,何生就偷偷去轻薄。九郎醒来,十分气愤,拿起衣服趁夜走了。何生郁郁不乐像失去了什么似的,整日废寝忘食,一天天消瘦、憔悴起来。唯有叫童子天天到处去找九郎。

一天,九郎又从何生门外经过,想直接走掉。童子向前扯住衣服拉他进屋。见何生那副消瘦的样子,九郎大为吃惊,忙问是什么原因。何生以实相告,哭得泪如雨下。九郎小声说:“我的意思实在是因为这样的相爱,既无益于弟,也有害于兄,所以不愿那样做。既然你非要那样不可,我还有什么顾惜的呢?”何生非常高兴。九郎走以后,病马上就好了许多,几天后就完全康复了。九郎果然又来了,于是二人交好。九郎说:“今晚勉强顺从了你的意思,但绝不能当作常事。”接着又说:“我向你提个要求,能办到吗?”何问他有何事,九郎说:“我母亲患心疼病,只有太医齐野王的先天丹能治,你与太医关系很好,我想你一定能求得到。”何生马上答应了。九郎临走又嘱咐再三。

何生入城求了药来,到晚上给了九郎。九郎非常高兴,上去握着何生的手表示感谢。何生又趁机要求九郎交欢,九郎说:“不要再纠缠了!我想给你找一个美人,比小弟强一万倍。”何生问是谁,九郎说:“是我的一个表妹,美丽无比。你若同意,我就给你作媒。”何生只是微笑,没有回答。九郎拿了药就走了。

过了三天,九郎又来求药。何生嫌他隔这么长时间才来,话里带刺。九郎说:“本来我不忍心害你,所以故意疏远你。既然你不谅解我,请你以后不要懊悔!”自此以后,九郎天天来与何生相会,但三天必求一次药。齐太医嫌何生拿药太频繁,说:“我的药吃三副就好,为什么吃了这么多还不好?”一下给了他三副药。齐太医又看着何生说:“你神色不好,生病了吗?”何生回答说:“没有。”齐太医给他试试脉像,惊惧地说:“你有鬼脉,病在少阴。你自己不保重,命就难保了!”何生回来把太医的话告诉了九郎,九郎叹道:“真是神医!我是狐。我们交往久了,恐怕不是你的福气。”何生还怀疑九郎是诳他,没把三副药都给九郎,怕他不再来了。

不久,何生果然病倒了,请齐太医来看病,太医说:“那天你不说实话,现在魂已出壳了,再有名的医生也无能为力了。”九郎天天来看望何生,说:“不听我的忠告,果然有今天!”不久,何生就死了,九郎痛哭而去。

在这以前,本县某太史,少年时与何生同学,十七岁就选入翰林。当时陕西藩台贪污暴虐,因他买通了朝中大官,所以没有敢揭发他的。而这个太史却告发了他的罪行,但却被以越职言事的罪名罢了官。藩台还升了这个省的中丞,天天找太史的把柄。太史少年时小有名气,曾求一个叛王重用自己,中丞买到了他们当年的来往信件,以此威胁太史。太史害怕,就自杀了。他夫人也上吊而死。

太史死了一夜,忽然醒来,自己说:“我是何子萧。”别人问他,说的都是何家的事。大家才明白这是何子萧借尸还魂了。留他住下,他不愿意,出门就跑到何家去了。

抚台怀疑其中有诈,一定要陷害太史,派人向他索取一千两银子。何生只好应着,但却没有银子。正发愁时,忽报九郎来了,何生高兴地和九郎说话,悲喜交集。接着又要求欢爱。九郎说: “你有三条命吗?”何说:“我懊悔活着辛苦,还不如死了安逸。”于是对九郎诉说冤苦。九郎想了半天后说:“幸好我们再次相聚。你现在已是孤身无伴,我以前说过的表妹,聪明有智谋,人又漂亮,必然能替你分忧。”何生想看看她。九郎说:“不难,明天她就陪老母从这里走。你装作我的兄长,到时我来找水喝,你说 ‘驴子跑了’,便是同意了。”他们谋划好了便分别了。

第二天中午,九郎果然同女郎从何生门前经过。何生拱手相迎,唠唠叨叨与九郎说话,斜眼看了一下女郎,见女郎长得蛾眉秀眼,像仙人一般。九郎要求喝茶,何生请他进屋,九郎对女郎说:“三妹不要怕,这是我的盟兄,不妨稍休息一下再走。”九郎扶女郎下驴,把驴子拴在门外。何生趁倒茶之际,看着九郎说:“你上次说的话如不能做到,我今天就到了死期了。”女郎似乎听出了他们的话是算计自己,便起身想走,细声说:“走吧!”何生赶忙大声喊:“驴子跑了!”九郎一听忙去追赶驴子。何生抱住女郎就要求欢。女郎吓得脸色发紫,窘得像被囚禁一样,直喊九兄。九郎也不答应。女郎说:“你有妻子,为什么糟踏别人?”何生说没有家室。女郎又说:“你能对山河起誓,不抛弃我,才能听从你。”何生便对天盟誓,女郎才不拒绝了。

事后,九郎也就回来了。女郎显出很生气的样子,不拿好脸色给他看。九郎说:“这个何子萧,以前是名士,现在是太史,与我最好,可以信赖。就是把这事告诉妗子,她也不会怪罪。”一直到了晚上,何生留女郎住下,女郎怕姑母责怪,坚决要走。九郎愿一人承担,便一人上驴走了。

何生与女郎住了几天,有个妇人带着丫鬟从门前过。妇人年约四十岁,长相、神情与三娘很像。何生叫出三娘偷看,果然是自己的母亲。母亲也看见了三娘,便奇怪地问:“你怎么在这里?”女儿非常羞惭,无话对答。于是何生把母亲请到房里,施礼以后,告知详情。母亲笑着说:“九郎孩子气,为什么不与我商量?”女儿亲自下厨房做饭给母亲吃。饭后母亲便走了。

何生得到佳人三娘,很是高兴。但因愁那千两银子的事,脸上总有忧色。三娘问他原因,他就讲述了经过。三娘笑着说:“这事九郎一人便可以解决,你愁什么?”何生问有什么办法,三娘说:“听说抚台大人爱听歌曲、喜欢男孩子,这都是九兄所长。投其所好,把九郎献给他,旧冤可消,新仇可报。”何生怕九郎不肯去。三娘说:“只管苦苦哀求他。”隔了一天,何生见九郎来,跪下相迎。九郎惊问:“咱们两代世交,凡要我效力的事,从头到脚都不会吝惜,何必做出这种样子?”何生把计谋说了一遍,九郎听了面带难色。三娘说:“我已失身于郎君,这都是谁造成的?假设他中途被害死抛我而去,我可怎么办?”九郎不得已,只好答应。

何生与九郎谋划好后,就写信给原来与他要好的王太史,并介绍九郎前去。王太史领会了信中的意思,设盛宴请抚台前来饮酒,叫九郎扮成美女跳天魔舞,宛然如女郎一般。抚台越看越着迷,于是极力向王太史要求,出重金买九郎,惟恐不成功。王太史假装沉思,像有难处,考虑了很长时间,才表示为了抚台而割爱。抚台高兴得不得了,以前的成见都消了。

抚台得到九郎,便形影相随,片刻不离。原有的妻妾、侍女十几个,全都视如粪土。九郎的一切饮食、用具均与王侯一样,还赐给九郎银子万两。半年的工夫,抚台就病了。九郎知道抚台死期不远了,就载上金银财宝,假装送回抚台原籍去。很快抚台就死了。

九郎拿出银两,盖房子、置家具、雇了仆人、丫鬟,母亲和妗子都来一块住。九郎出出进进,车马随从很多,人们都不知道他是狐。


\subsection{1.3.12   金 陵 女 子}
\label{\detokenize{p00_u5176_u5b83/_u767d_u8bdd_u804a_u658b_u5fd7_u5f02:id98}}
沂水县人赵某,进城办事,在回来的路上,见一个白衣女子在路边哭,哭得十分哀恸。他斜眼一看,见女子长得很俊俏,心里非常喜欢,站在那里盯了很长时间。女子掉着泪说:“你一个大丈夫不走路,只看人家干什么?”赵某说:“因为野外无人,你又哭得很伤心,我实在不忍心走了。”女子又说:“我丈夫死了,无路可走,所以伤心。”赵某劝她再找一个好男人。女子说:“我一个孤身女子,能去找谁?若能找个存身的地方,给人家做妾也行!”赵某欣然自荐,女子也愿意,就跟着他一起往家走来。赵某因为距家还很远,想雇一匹马或驴叫女子骑,女子说:“不用。”说罢,就走在前面。走起来轻飘飘的像仙女一般。

这女子到了赵家,推磨担水,干活非常勤快。两年多后,忽有一天对赵某说:“感谢夫君恩爱,我跟你已快三年了,现在也应当走了。”赵某说:“以前你说没有家,现在你到哪里去?”女子回答说:“我那是随便说罢了,其实我哪能没有家?我父亲在金陵卖药。你要想再见到我,可载着药去金陵找我,我还可给你一些钱作资本。”赵某打算给她雇车马,女子谢绝了,一出门就飞快走去,追都追不上,一转眼就不见了。

过了很长一段时间,赵某非常想念那个女子。于是就载上药去金陵找她。到了金陵,把药寄存在旅店里,沿街到处打听这女子。忽然一间药店里一个老头看见他,说,“贤婿来了!”就请赵某进了院子。那女子正在院中洗衣服。女子看了看他,不说也不笑,照常洗衣。赵某心里很生气,回头就想走,老头拉他回来,女子仍然不看他一眼。老头命女子做饭摆酒招待客人,还打算厚厚地赠给他些东西。女子制止说:“他福份薄,多给他东西他享受不了,少给他点慰劳辛苦就行。再给他十几个药方,就够他吃用一辈子的了。”老头又问赵某载来的药在哪里,女子说:“已经给他卖完了,钱在这里!”老头便把钱交给赵某,又给了他十几个药方子,就打发赵某回家了。

赵某回家后,试验带来的药方子,个个都有特效。沂水至今还有知道这些方子的人。据说用蒜臼子接屋檐水洗疣赘,就是其中的一方,疗效很好。


\subsection{1.3.13   汤 公}
\label{\detokenize{p00_u5176_u5b83/_u767d_u8bdd_u804a_u658b_u5fd7_u5f02:id99}}
汤聘是辛丑年的进士。他生病快要死去的时候,忽然觉得下部有一股热气,渐渐向上升,到了腿部,脚就死去,没有了知觉;到了肚子,腿就死了;到了心部,心最难死。这时,汤公觉得凡是小时候的一些事情和早已经忘了的琐事,现在都潮水般在心头一一浮现。如果是一件好事,心中就觉得清静;如果是做了一件坏事,心中就觉得懊恼烦躁,像油烧开了锅,难受得无法形容。还回忆起七八岁时,曾因掏鸟窝而打死过小麻雀,这件事使他心头热血翻滚,一顿饭的工夫才过去。这样直到把平生所作所为翻腾完了,才觉得那股热气一缕一缕穿过喉咙进入脑子,自头顶穿出,腾空而起,像炊烟一般袅袅升向天空。过了几个时辰,魂才脱离躯体而去,自己忘了自己的身子,只感到渺渺茫茫无有归宿,一直飘到郊外的路上。忽然来了一个巨人,高几十丈,低头把他拾起来,放进了袖筒里。汤公进了袖筒,直觉里边人挤人,烦热闷气,难受极了。忽然他想起佛能解除危难,便在袖里呼叫佛号,才叫了三四声,一下就飘出袖外。巨人就又把他拾进袖里。如此拾了三次,巨人便不再拾他了。

汤公独自一人彷徨路边,一时不知向哪里去。又一想,佛在西天,还是向西吧!走了不多时,见路边有一个和尚坐在那里,便向前施礼问路。和尚对他说:“凡是文官的生死册,都由文昌、孔圣人管着,你必须到两处销了名,才能到别处去。”汤公又问他们的住处,和尚指了路,汤公就顺路走去。

不一会儿走到圣庙,见孔圣人朝南坐着,汤公赶快上前跪拜,说明来意。孔圣人说:“你要销名,还得去找帝君。”告诉他去路。汤公就又走。见前面有一宫殿,像是君王住的地方,便俯下身子进去。宫殿上坐着一个神人,像世上传说的帝君模样。汤公向前跪下祈祷。帝君详细查看名册,对汤公说:“你有一颗诚恳正直的心,还可以再活几年。但你尸骨已经腐烂,找菩萨才能使你还魂。”于是叫他赶快去找菩萨。

汤公又按帝君指的路往前走。走到一个地方,见有茂盛的树林,碧绿的修竹,华丽的殿堂。汤公走进大殿,但见正面坐着菩萨,高髻端庄,金光满面。玉瓶里插着杨柳枝。依依低垂,葱翠如烟。汤公肃然叩头,禀告帝君之意。菩萨听了,面带难色。汤公又不断叩头,苦苦哀求。菩萨旁边一位尊者建议说:“请菩萨施大法力,撮土作肉,折柳为骨。”菩萨同意,随即折了一柳枝,又从瓶中倒出一点净水,用净水和成泥,把泥拍附在汤公身上,令仙童把他送回,推着与他的尸体合为一体。于是就听到汤公的棺材中有呻吟声,家人惊讶地聚过来,把汤公搀扶出来,他病已痊愈。计算了一下时间,汤公死去已经七天了。


\subsection{1.3.14   阎 罗}
\label{\detokenize{p00_u5176_u5b83/_u767d_u8bdd_u804a_u658b_u5fd7_u5f02:id100}}
莱芜县有一个秀才,叫李中之,性情刚直不阿。每几天就昏死一次,僵卧如尸体,三四天就又苏醒过来。有人问他看见些什么,他总是严守秘密不说。

这时本县有个姓张的书生,也是几天昏死一次。他告诉别人说:“李中之是阎王爷,我到了阴间,也是给他当差。”阎罗殿的对联,张生都能背诵下来。有人问:“李中之昨天去阴间干什么?”张生说:“不能一一细说,但是提审了曹操,打了二十板子。”


\subsection{1.3.15   连 琐}
\label{\detokenize{p00_u5176_u5b83/_u767d_u8bdd_u804a_u658b_u5fd7_u5f02:id101}}
杨于畏,搬家居住在泗水岸边。他的书房临近旷野,墙外有很多古墓。每到夜晚,墓地里的白杨被风刮得哗哗作响,声音如同波涛汹涌。一天深夜,杨于畏一个人在灯下,正感到凄凉,忽听墙外有人吟诗:“玄夜凄风却倒吹,流萤惹草复沾帷。”反复吟诵了好几遍,声音悲哀凄楚。仔细一听,柔弱婉转像是个女子,杨于畏心中大疑。第二天一早,出去看看墙外,并没有人迹,只有一条紫带子遗弃在荆棘丛中。杨于畏捡了回来,顺手放在窗台上。到了夜晚,二更天时,又传来吟诗声,和昨夜一样。杨于畏悄悄地搬了个凳子到墙边,登上去往外一望,吟诗声顿时没有了。杨于畏醒悟是女鬼,但心里却很倾慕她。第二夜,他早早地藏在墙头上等着。一更天快完的时候,只见一个年轻的女子,从荒草中姗姗而出,手扶小树,低着头悲伤地念起那两句诗。杨于畏轻轻咳嗽了一声,女子倏忽一下,隐入荒草中不见了。杨于畏继续在墙下等着,等那女子又出来吟完诗,他隔墙续道:“幽情苦绪何人见,翠袖单寒月上时。”过了很久,墙外寂静无声。

杨于畏回到书房中,刚坐下,忽见一个美丽的女子从外面走进来,向他施礼说:“您原来是位风雅之士,我却过分害怕而躲避开了。”杨于畏大喜,拉她坐下。那女子又瘦又弱,似乎连衣服的重量也承担不起。杨于畏问道:“你的家乡是哪里?怎么长久地住在这地方?”女子回答说:“我是陇西人,随父亲流落到这里居住。十七岁时得暴病死去,到现在二十多年了。住在荒野地下,十分孤单寂寞。那两句诗是我自己作的,以寄托幽恨之情。想了很久,也没想出下句,承蒙你代续上了,我九泉之下也感到欢快!” 杨于畏想和她交欢,女子皱着眉头说:“阴间的鬼魂,不比活人,如果幽欢,会折人阳寿。我不忍祸害君子。”杨于畏只好作罢,却又用手摸女子的胸,见仍是处女的样子。又要看看她裙下的一双脚。女子低头笑道:“你这狂生太罗嗦了!”杨于畏摸着女子的脚,见月白色的锦袜上系着一缕彩线,再看另一只脚上却系着一条紫带子,便问:“怎么不都用带子系住?”女子回答说:“昨夜因害怕你躲避时,紫带不知丢到了什么地方。”杨于畏说:“我替你换上。”便去窗台上取来那条紫带递给女子。女子惊讶地问哪来的,杨于畏如实说了。女子解下彩线,仍用带子系住。收拾完,女子翻阅起桌上的书,忽见元稹作的《连昌宫》词,感慨地说:“我活着时最爱读这些词。现在看到,真如在梦中。”杨于畏和她谈论起诗文,觉得她聪慧博学,令人喜爱。杨于畏和她在窗下剪着灯花夜读,如同得到了一个知心朋友。

从此后,只要一听到杨于畏低声吟诗,一会儿女子就来了。常嘱咐杨于畏说:“咱们交往的事你一定要保密,不能泄露。我自幼胆小,恐怕有坏人来欺负我。”杨于畏答应了。两人如鱼得水,亲热非常。虽然未曾同寝,但双方的感情却胜过了夫妻。女子常在灯下替杨于畏抄书,写的字端正柔媚。又自己选了一百首宫词,抄录下吟诵。还让杨于畏准备下棋具,买来琵琶,每夜教杨于畏下棋。有时女子自己弹起琵琶,奏起《蕉窗零雨》的曲子,让人心酸。杨于畏不忍心听完,女子便又奏起《晓苑莺声》,杨于畏顿觉心旷神怡。两人灯下玩乐,往往忘了天明。直到看见窗上有了亮色,女子才慌慌张张地走掉。

一天,薛生来访,正碰上杨于畏白天睡觉。见屋子里琵琶、棋具都有,知道这些东西不是杨于畏擅长的。又翻阅他的书时,发现了一些抄录的宫词,字迹端正秀丽,心中越发怀疑。杨于畏醒来后,薛生问道:“这些游戏用具是哪来的?”杨于畏回答说:“想学学。”又问诗卷是哪来的,杨于畏假称是从朋友处借的。薛生反复赏玩,见诗卷最后一行小字写的是“某月日连琐书”,便笑着说:“这是女子的小名,你怎么如此欺骗我?”杨于畏窘迫不安,不知怎么回答好。薛生苦苦追问,杨于畏闭口不答。薛生便卷起诗卷,以拿走相要挟。杨更加窘困,只得实说了。薛生要求见见这个女子,杨于畏告诉他女子的嘱咐,薛生却更加仰慕。杨于畏迫不得已答应了。到了夜晚,女子来了。杨于畏便转述了薛生要见见她的意思。女子发怒地说:“我怎么嘱咐你的?你竟喋喋不休地跟人说了!”杨于畏解释说明当时的情况。女子说:“我和你缘分尽了!”杨于畏百般安慰解释,女子终究还是不高兴,起身告别说:“我暂时躲避躲避。”

第二天,薛生来了,杨于畏告诉他女子不愿见。薛生怀疑他在推托,晚上又带了两个同学来,赖着不走,故意扰乱杨于畏,吵吵嚷嚷闹个通宵。气得杨于畏直翻白眼,但是无可奈何。众人一连几夜,也没见那女子的影子,便都有了回去的心思,不再吵闹了。忽听外面传来吟诗声,大家静静一听,只觉那声音非常凄惋。薛生正在凝神倾听,同学中有一个武生王某,搬起块大石头投了过去,大喝道:“拿架子不见客人,什么好诗,呜呜咽咽的,让人烦闷!”吟诗声顿时消失了。大家都埋怨王生,杨于畏更是恼怒,脸色不好看。说话也难听了。第二天,同学们都走了。杨于畏独宿空房,心中盼望着女子再来,却一直渺无人影。

又过了两天,女子忽然来了,哭泣着说:“你招了些恶客,差点吓死我!”杨于畏连连道歉。女子匆匆地走了出去,说:“我早说过和你缘分尽了,从此永别了!”杨于畏正想挽留,女子已消失不见了。此后过了一个多月,女子一次没来。杨于畏天天思念,人瘦得皮包骨头,但却没法挽回了。

一晚,杨于畏正一个人喝着酒,女子忽然掀帘进来了。杨于畏高兴地说:“你原谅我了?”女子流着泪,默默不语。杨于畏忙问怎么了,女子欲言又止,只说:“我赌气走了,现在有急事又来求人,实在羞愧!”杨于畏再三询问,女子才说:“不知哪里来的个肮脏鬼役,逼我当他的小妾。我自想是清白人家的后代,怎能屈身于鄙贱的鬼差呢?可我这个弱小的女子,又怎能和他抗拒?您如认为我们感情深厚,如同夫妻,不会听任不管吧?”杨于畏大怒,恨恨地要打死那鬼差。可又顾虑阴问阳世不同路,怕无能为力。女子说:“来夜你早点睡觉,我在你梦中请你去。”于是两人重新和好,一直谈到天亮。女子临去又嘱咐杨于畏白天不要睡觉,等到夜晚相会,杨于畏答应了。

第二天午后,杨于畏喝了点酒,乘着酒意上了床,蒙衣躺下。忽见女子来了,给他一把佩刀,拉着他的手走去。来到一个院子,两人关上门正在说话,忽听有人用石头砸门。女子吃惊地说: “仇人来了!”杨于畏打开门,猛地窜了出去。见一个人红帽青衣,满脸刺猬般的胡须。杨于畏愤怒地斥责他,鬼役横眉怒目,凶悍地漫骂不止。杨于畏大怒,持刀冲了过去。鬼役捡起石块,雨点般地砸过来,其中一块正中杨于畏的手腕,再也握不住刀。正在危急时候,远远望见一人,腰里挂着弓箭正在打猎。杨于畏仔细一看,却是王生,急忙大声呼救。王生弯弓搭箭,急忙跑过来朝鬼役一箭射去,正中大腿;再一箭,结果了性命。杨于畏喜欢地道谢。王生询问缘故,杨于畏都说了。王生高兴自己上次得罪了女子,这次可以赎罪了,于是和杨于畏一块进了女子的住室。女子战战兢兢的,羞怯不安,远远地站着一句话不说。王生见桌子上放着把小刀,有一尺多长,用金玉装饰。他把刀从匣中抽出来一看,冷光四射,能照见人影。王生赞叹不绝,爱不释手。跟杨于畏说了几句话,见女子羞愧害怕得可怜,王生便走出屋子,告辞走了。杨于畏也独自返回,翻过墙后,一下子跌倒在地,于是从梦中惊醒,只听树中的雄鸡已高一声低一声地叫开了。杨于畏觉得手腕很疼,天明后看了看,手腕上皮肉都肿了。

到了中午,王生来了,说起夜晚做了个奇怪的梦。杨于畏说:“没梦见射箭吗?”王生奇怪他预先知道。杨于畏伸出手腕,讲了缘故。王生回忆着梦中见到的那个女子,只恨不是真正见面。自觉对女子有功,又请杨于畏给通融通融。到了夜晚,女子来拜谢。杨于畏归功于王生,就便讲了王生想见一面的诚恳心情。女子说:“他的帮助,我不敢忘记。但他是个纠纠武夫,我真的害怕!”过了会儿又说:“他喜欢我的佩刀。那把刀是我父亲出使粤中时,用一百两银子买来的。我很喜欢,就要了过来,缠上金丝,并镶上了明珠。父亲可怜我年幼死去,用刀殉莽。现在我愿割爱,把刀赠给他,见了刀就像见了我本人一样。”第二天,杨于畏跟王生说了女子的意思,王生大喜。到夜晚,女子果然带着刀来了,对杨于畏说:“告诉他珍重,这把刀不是中华出产的!”从此后,杨于畏和女子来往如初。

过了几个月,女子忽然在灯下边笑边看着杨于畏,像要说什么,可又脸色一红,不说了,如此好多次。杨于畏便抱着她询问,女子说:“长久以来承蒙你眷爱,我接受了活人的气息,天天食人间烟火,白骨竟有了活意。现在只须人的一点精血,我就可以复生。”杨于畏笑着说:“是你不肯,哪是我吝惜呢?”女子说:“我们结合后,你定会大病二十多天,但吃药可以治好。”于是两人恩爱起来。过了会儿,女子穿上衣服起来,说:“还需一点生血,你能够拚上疼痛爱惜我吗?”杨于畏取过利刃,刺破手臂,女子仰卧在床上,让血滴进肚脐中,起来说:“我不再来了。你记住一百天后,看我的坟前有青鸟在树梢上鸣叫,就赶快挖坟。”杨于畏答应。女子临出门又嘱咐说: “千万记住,不要忘了。早了晚了都不行!”说完便走了。

过了十多天,杨于畏果然大病,肚子胀得要死。请来医生抓了药服下,排泻出很多稀泥样的浊物。又过了十多天,病才好了。计算着到了一百天,杨于畏让家人拿着工具在女子的坟前等着。到了傍晚,果然见两只青鸟在树枝上鸣叫。杨于畏高兴地说:“可以了!”于是刨去荆棘,挖开坟墓,只见棺木已经腐烂,但女子的面貌仍像活的一样。杨于畏用手一摸,女子身上有温气,便盖上衣服,把她背回家中,放到温暖的地方。觉得女子口里有了一丝气息,又喂了些汤粥,到半夜女子醒了过来。从此后,女子常对杨于畏说:“死了二十多年,就像做了一场梦一样!”


\subsection{1.3.16   单 道 士}
\label{\detokenize{p00_u5176_u5b83/_u767d_u8bdd_u804a_u658b_u5fd7_u5f02:id102}}
韩公子,是淄川县官宦人家的子弟。有个姓单的道士,精通变戏法。韩公子很喜欢他的法术,把他待为座上宾。单道士跟人走路或坐在一起时,常常忽然不见了。韩公子想跟他学这种隐身法,道士不肯。公子再三恳求,单道士说:“我并不是吝啬我的法术,是恐怕传出去后坏了我的名声。如果我教给的是君子倒还罢了,传给小人就不行了,会有人借此隐身法去行窃。公子当然不会去行窃,但你出去后,如发现谁家的姑娘媳妇漂亮,一喜欢上,就用隐身术偷进闺房,我岂不是助纣为虐,成了淫徒的帮凶了吗?所以不敢从命!”韩公子不能强迫道士,可怀恨在心,暗地里和仆人们商量痛打道士一顿,羞辱他一番。恐怕打他时他又使隐身法跑了,就用细灰洒在麦场上,心想,他即使用隐形术,但走过的地方必定在灰上留下痕迹,这样就可以追着他的足迹痛打了。一切布置停当,韩公子便把单道士骗到场上,命仆人手持牛鞭快打。单道士忽然不见了,但灰上果然有鞋子走过的痕迹。仆人们四下里一顿乱打,刹那间灰土飞扬,再也找不到道士的踪影了,韩公子只得悻悻地回家了。

过了会儿,单道士也回来了,跟伺候自已的仆人说:“我不能在这里住了!一向有劳你们,现在要分别了,我要报答你们!”说完,从衣袖中掏出一壶美酒,又拿出一盘佳肴,都放在桌子上。摆完,又掏,共掏了十几次,桌上的菜肴已摆满了。于是。单道士邀请大家入座喝酒。众人都开怀痛饮。吃喝完,单道士仍把酒壶、菜盘一一放回袖子里。

韩公子听说这件奇异的事后,便让道士再变个戏法看看。单道士便在墙壁上画了座城,画完,用手推推城门,门竟一下子开了。单道士将衣服行李全都扔进城门里,又向韩公子拱拱手说:“告辞了!”说完,纵身跳入城内,门立即又关上了,单道士便消失不见了。

后来,听说有人在青州的街市上又见到单道士,见他教儿童在手掌上画墨圈,然后逢人把手一扬,墨圈就会抛落下来,印到行人的脸上或衣服上。又听说单道士善房中术,能让下部吸一壶烧酒。这件事韩公子曾当面检验过。


\subsection{1.3.17   白 于 玉}
\label{\detokenize{p00_u5176_u5b83/_u767d_u8bdd_u804a_u658b_u5fd7_u5f02:id103}}
有一个书生叫吴筠,字青庵,少年时就很有名气。当地葛太史曾看过他的文章,给以好评。因喜欢他的文才,就托与吴筠要好的人请他来交谈,以观察他的言谈与文采,并说:“哪里有像吴筠这样的才学还长期过穷日子的呢?”并叫邻居们传话给吴筠:“要是能奋发上进,考取功名,我就把女儿嫁给他。”

葛太史有一个女儿,长得很漂亮。这话传到吴筠耳朵里,他非常高兴,也很有信心。可是第一次考试就落了榜。他就托人转告太史:“我能富贵那是命中注定,只不过不知道是早是晚。请等我三年,我实在不能成功,他的女儿再另嫁。”于是他更加刻苦学习。

一天夜里,明月之下,有一个秀才来拜访他。这人长得白净脸,短头发,细细的腰,长长的手。吴生有礼貌地问这人从哪里来,有什么事。那人说:“我姓白,字于玉。”两人又稍稍说了几句话,吴生觉得此人心胸开阔,心里很是赏识,就留白生同宿一处。白生也不推辞,睡到天明才走。吴生再三嘱咐,顺便时再来叙谈。白生也觉得吴生诚实热情,就提出要在吴生家借住。吴生非常同意,约好搬家的日子,就分手了。

到了搬家的那天,先是一个老头送炊具及其它用具来,随后白生才到。他骑一匹白龙马,吴生迎接进来,忙命家人打扫房间安排住下。白生也打发跟来的人牵马回去。

从此以后,两人朝夕相伴,互相研讨学问,各有收益。吴生见白生读的书不是常见的书,也没有八股文一类的文章,便奇怪地问白生。白生回答说:“人各有志,我不是求功名的人。”晚上还经常请吴生到他屋里喝酒,拿出一卷书来给吴生看,书中都是些气功方面的事,吴生看不懂,便信手放在一边。又过几天,白生对吴生说:“那天晚上给你看的书,书中讲的都是些《黄庭经》的要术,是羽化登仙的入门教材呀。”吴生笑着说:“我对成仙不感兴趣。成仙得断绝情缘,没有杂念,这我是做不到的。”白生问他:“为什么?”吴生回答是为传宗接代。白生又问:“为何这么大年纪还不娶亲呢?”吴生笑道:“‘寡人有疾,寡人好色’。”白生说:“‘王请无好小色’。你想娶个什么样的意中人?”吴生才把等葛太史女儿的事告诉了白生。白生怀疑葛家女子未必真美。吴生说:“这女子美是远近都知道的,不是我自己眼光浅。”白生一笑了之。

第二天,白生忽然整理行装,对吴生说是要走。吴生依依不舍,难过地与白生絮絮话别,不忍分离。白生就叫童子背了行李先走,自己与吴继续说话。忽然见一个青蝉叫着落在桌子上,白生告辞说:“车子已经来了,我告辞了。以后你着想我,就扫一扫我睡的床,躺在上面。”吴生听了刚想再问什么,转眼间,白生就缩小得像指头一样大,一翻身骑在青蝉背上,吱地一声飞走了。渐渐消失在彩云中。吴生这才知道白生不是平常人。惊愕了很久,才怅然若失地回房。

过了几天,天上忽然下起蒙蒙细雨来。吴生很想念白生,就走到白生住的房间。一看白生住的床上布满了老鼠的爪迹,叹了口气,用条帚扫了一下,铺上一张席子躺下休息。不多时,就见白生的书童来请他,吴生非常高兴,跟了童子就走。一霎时,见一群小鸟飞来,童子捉住一个对吴生说:“黑路难走,可骑小鸟飞行。”吴生顾虑这么小的鸟能担负得动吗?童子说:“可以骑上试试。”吴生就试着骑在上面,见鸟背非常宽绰,童子也骑在他身后,只听嘎的一声就飞上了天空。

不多时,眼前出现一座红门。鸟落了地,童子先下,扶吴生也下来,吴生问:“这是哪里?”童子回答说:“这是天门。”门两边有一对大老虎蹲在那里,吴生很害怕,童子护着他领着进去。只见处处风景与世间大不相同。童子领他到了广寒宫,宫内都是水晶台阶,走路像走在镜子上一般。两棵大桂花树,高大参天,荫翳天日,花气随风飘来,香气袭人。房屋、亭子都是一色红窗红门,时常有美女出出进进,个个端庄秀美,人间无比。童子说:“王母宫的宫女更漂亮。”因怕白生等久了,没能多留,童子匆忙领他走出广寒宫。

又走了一段路,就看见白生在门口等他。白生一见到吴生,忙上前来握住他的手,领他进了院子。吴生见屋檐下清水白沙、涓涓流淌,玉石雕砌的栏杆,好像月宫一样。刚进屋坐下,就有妙龄女子前来献香茶,接着就摆上酒宴。四个美女,金佩玉环、叮当作响,侍立两边。吴生刚觉背上有点痒痒,就有美人伸入细手用长指甲轻轻搔痒。吴生直觉心神摇曳,一时平静不下来。不一会儿,就喝得有点醉意,渐渐掌握不住自已,笑着看看美人,殷勤地与美女说话,美女每每笑着避开他。白生又命美女唱歌佐酒。一红纱女子端着酒杯献酒,一面唱动听的歌曲,众美女也都随着一起演奏起来。奏完,一个绿衣女子一面唱歌,一面献酒;一个穿紫衣的和一个穿白纱衣的女子嗤嗤笑着,暗中互相推让,不敢向前。白生又命她们一人唱歌,一人敬酒。于是紫衣女便来敬酒。吴生一面接杯,一面用手挠女子的手腕。女子一笑失了手,把酒杯掉在地上打碎了。白生责备她,这女子含笑捡杯,低下头细声说:“冷如鬼手馨,强来捉人臂。”白生大笑,罚紫衣女自唱自舞。紫衣女舞完后,白衣女又来敬一大杯,吴生谢绝;白衣女捧酒不快,吴生只得又勉强喝了。吴生用醉眼细看这四个女子,都风度翩翩,没有一个不是绝世佳人。吴生忽然对白生说:“人间的美女,我求一个都很难,你这里这么多漂亮的美人,能不能让我真正快乐快乐?”白生笑着说:“足下不是有意中人吗?这些你还能看上眼?”吴生惭愧地说:“我今天才知道我见识得太少。”于是白生就召集起所有美女让吴生选择。吴生看看哪个也好,一时拿不定主意。白生因为紫衣人曾和他有过捉臂之交,便吩咐她抱了被子去侍奉吴生。

吴生与紫衣女同床睡觉,尽情欢乐,恩爱无比。事后,吴生向紫衣女索取信物,她就摘下金手镯赠给他。忽然童子来说:“仙人凡人有别,请吴先生马上回家。”女子急忙起床出门去了。吴生问童子白生哪里去了,童子说:“早去上朝了,他吩咐我去送你。”吴生闷闷不乐,只好跟童子按原路返回。到了天门,一回头,童子不知何时已不见了,门边的两个老虎张着大嘴一起向他扑来。吴生急忙快跑,眼前却是一条无底的山谷,想住脚已来不及了,一头扎进了山谷,吃了一惊,出了一身冷汗。一睁眼,原来是做了个梦。太阳已红彤彤的了。拿起衣服一抖,觉得有件东西掉在床上,一看,正是那金镯子,吴生心里好生奇怪。

从此,吴生想升官发财、娶美女的心思,全部没有了,心灰意冷。对人间不感兴趣,一心向往名山大川,拜寻赤松子,得道成仙。然而他还一直没有忘记传宗接代。

又过了十几个月,有一天,吴生白日睡觉正浓,忽然梦见紫衣女子自外边进来,怀里抱着一个婴儿,对吴生说:“这是你的骨肉。天上难留这个孩子,所以抱来送还你。”说罢,把孩子放在床上,又用衣服盖好,匆匆就走。吴生一把拉住她,要她再住一夜。紫衣女说:“上次同床为新婚,这一次同床为永别,百年夫妻就到这里。若郎君有志,或者还能相见。”吴生醒来,见婴儿睡在身边被褥之中。赶快抱着去见母亲。他母亲高兴得不得了。于是雇了奶娘喂养这个婴儿,起了个名字叫梦仙。

吴生有了孩子,就托入转告葛太史,说自己要去隐居,请他女儿另嫁。太史不肯,吴生固辞,太史便告诉了他女儿。女儿说:“远近没有不知道我已许配吴生了,今又改嫁别人,这不是嫁了二夫吗?”于是葛太史又把这话转告了吴生。吴生说:“我不但已经不图功名,而且也绝情于男女了。我所以没有马上进山,只是因为尚有老母健在。”太史又把吴生的话告诉女儿。女儿说:“吴郎穷,我甘心跟他吃糠咽菜;吴郎要去,我就在家侍奉婆母,定然不另嫁他人。”如是再三,商量不妥,葛太史最后还是择了日子,用车马把女儿送到了吴家。吴生感念妻子的贤惠,特别敬爱她。女子侍奉婆母非常孝顺,也不嫌家里贫穷。

过了两年,吴母死了,葛女卖了嫁妆,安葬了婆母,尽到了礼节。吴生对妻子说:“我有像你这样的贤妻,还有什么忧愁!只是听说一人得道,拔宅飞升,所以想离家出走,家中一切就拜托给你了。”葛女也坦然答应,一点也不挽留。于是吴生就辞别妻子出走了。

吴生走后,葛女外理生活,内训娇儿,治家井井有条。梦仙也渐渐长大,学习聪明过人,十四岁中了秀才,人称神童;十五岁又入翰林。每次皇上褒封,不知他的生母是谁,只封葛氏一人。每次有祭礼,梦仙总是问父亲在哪里?他的养母就实话告诉了他。梦仙想辞官不做,去找父亲。养母说:“你父亲已走了十几年了,想来也已成仙了了你哪里去找?”

后来,梦仙奉旨去祭南岳,路上碰到一伙强盗,正在危急之时,来了一个持剑的道士,强盗被吓跑了,为他解了围。梦仙很感激他,赠给道士银子,道士不要,只拿出一封信托梦仙捎回,嘱咐说:“我有个朋友与大人是同乡,托你代问个好。”梦仙问:“你朋友叫什么?”回答说:“王林。”梦仙想来想去村中没有这个人。道士说:“他是个老百姓,大人可能不认识他。”道士临走拿出一只金镯子说:“这是闺阁之物,我拾了来没有用,就送给你作为捎信的报答吧!”梦仙拿着手镯细看,做工精细,镶嵌精美,就拿回家去给了他夫人。夫人很珍爱,叫能工巧匠照样再造一只配成对,怎么也造不了这么好。

梦仙遍问村中百姓,并没有王林这个人。实在无法找到,就打开信看,信中写着:“三年鸾凤,分拆各天。葬母教子,端赖卿贤。无以报德,奉药一丸。剖而食之,可以成仙。”后面写着: “琳娘夫人妆次。”念完了仍不知是什么人,就拿着去问他养母。养母一看便哭了,说:“这是你父亲的家书,琳是我的小字。”梦仙才恍然大悟,王林是琳字的拆白,悔恨得不得了。又拿出镯子请母亲看,母亲说:“这是你生母的遗物。你父在家时,常拿出来给我看。”又看药丸,有豆子那样大。梦仙高兴地说:“我父亲是仙人,吃了这丸子一定长生不老。”他母亲没有立刻吃,暂时藏了起来。等葛太史来看外孙时,便给他念了吴生的信,并奉上丹丸给他添寿。太史一分两半,与女儿分吃了,顿时精神焕发。太史已七十多岁老态龙钟,吃了丹丸忽然筋骨强壮,不坐车马,步行走得很快,家人跑路才跟上他。

又过了一年,城里发生了火灾,大火终日不灭,全城人都不敢睡觉。梦仙家的人都在院子里看,见大火渐渐漫延过来,一家人无计可施。忽然夫人手上的金镯子嘎然作响,自行脱手飞上天空,逐渐扩大,圆圆地盖在宅子上,镯子口朝东南。众人都惊呆了。一霎时,火自西来,烧到镯子边就转向了东。等火势烧远了,众人认为镯子不会再回来时,忽见红光一下收敛起来,镯子当地一声掉在夫人足下。这次城中大火烧了民房几万间,前后左右都成灰烬,只有吴宅安然无恙。只有东南角一小阁被烧,正是镯子口处没盖住的地方。

葛女年五十多岁时,有人看见,还像二十多岁人一样。


\subsection{1.3.18   夜 叉 国}
\label{\detokenize{p00_u5176_u5b83/_u767d_u8bdd_u804a_u658b_u5fd7_u5f02:id104}}
交州有一个姓徐的,驾船渡海去远方做买卖,在海上遭遇大风,船被吹到不知什么地方。风停后,徐某睁眼一看,见来到一处,山峰绵延,树木苍苍。徐某希望有人居住,便将船拴好,背着粮食、干肉,下船登上了海岸。

刚进山,见两边悬崖上,密密麻麻地排列着很多洞口,像蜂房一样,洞内隐约有人声。徐某来到一个洞外,停下脚步往里一瞅,里面有两个夜叉,吡着两排白森森的剑戟般的利齿,双眼瞪得像灯笼一样,正用爪子撕生鹿肉吃。徐某吓得魂飞魄散,急忙返身要逃,夜叉已看见他,扔下死鹿,爪子一伸,把他抓进洞里。两个夜叉互相说着话,像鸟兽的叫声,争着撕扯徐某的衣服,似乎想吃了他。徐某恐惧万分,忙取出背在身上的干粮和熟牛肉干,送给夜叉。夜叉分吃完了,觉得味道很美,又去翻徐某的袋子。徐某摇摇手,表示没有了。夜叉大怒,又把他抓了起来。徐某哀求说:“放开我!我船上有锅子,可以再做给你们吃!”夜叉不明白他的话,仍然发怒。徐某打着手势又说了一遍,夜叉像是稍微有点明白了,便跟着他来到船上,把锅子拿到洞中。徐某抱来柴禾,点上火,将夜叉吃剩下的生鹿肉煮了献给他们,两个夜叉吃得非常高兴。到了夜晚,夜叉用石头堵住洞口,像是怕徐某逃跑。徐某蜷曲着身体,远远地躲着夜叉躺下,整夜战战兢兢的,生怕最终不免一死。

天明后,两个夜叉出去了,临走前又堵上洞口。不一会儿,取来一头死鹿交给徐某。徐某便剥了鹿皮,到洞深处打了水,煮了好几锅。又过了一会,来了好几个夜叉,聚到一起,吞吃着锅里的熟鹿肉。吃完了,一齐用手指着锅子,似乎嫌太小。过了三四天,一个叉背来一口大锅,像是人常用的那种。于是,夜叉们纷纷拿来死狼、死鹿等动物,放在锅里煮。煮熟后,招呼徐某也一块吃。这样过了几天,夜叉们渐渐和徐某熟悉起来,出去时也不再堵洞口了,待他像一家人一样。徐某也渐渐能根据夜叉发出的声音,揣摩出他们的意思,还常常学着他们的腔调,说些“夜叉话”。夜叉们更加高兴,又带来一个母夜叉,给徐某当老婆。起初徐某很害怕,在母夜叉面前不敢动弹。后来母夜叉主动亲热他,徐某才和她成了夫妻。母夜叉大为喜悦,此后便常常留下熟肉给徐某吃,真像是恩爱夫妻一样。

一天,夜叉们早早起来,每个夜叉脖子上都挂着一串明珠,轮番走出洞外,像是在迎候什么贵客。又让徐某多煮些肉。徐某问母夜叉,母夜叉说:“今天是天寿节。”又走出去跟别的夜叉说:“徐郎没有骨突子!”众夜叉听说,各摘下五颗珠子,一块交给母夜叉。母夜叉又从自己脖子上摘下十颗,共凑了五十颗,用野麻皮搓了根绳子串起来,挂在徐某脖子上。徐某看了看这些明珠,一颗足值百十两银子。一会儿,夜叉都走了出去。徐某煮完肉,母夜叉来叫他说:“去接天王!”

徐某跟随夜叉们来到一个大洞。这个洞足有好凡亩地大,中间有一块巨石,上面又平又滑,像桌几一样。巨石周围摆着些石座,最上首一个石座上蒙着豹皮,其余蒙的都是鹿皮,共坐了约二三十个夜叉。不一会儿,只听大风呼呼,飞沙走石。夜叉们慌忙出迎。徐某见走来一个巨大的怪物,样子也像是夜叉。那怪物径直奔进洞中,高高地蹲坐在豹皮座上往下俯视着。众夜叉们跟着一块进洞,分东西两列站好,都昂起头,双臂交叉成十字状,向大夜叉行礼。大夜叉点了点人头,问道:“卧眉山上的,就是这些吗?” 众夜叉乱哄哄地答应。大夜叉看见了徐某,问:“这个是从哪来的?”母夜叉回答说:“他是我丈夫。”大家对大夜叉夸起徐某的烹调来。随即有两三个夜叉跑去取了些熟肉来,献到石桌上。大夜叉双爪撕着,饱吃一顿,极力夸赞味道美,并且命令此后要按时供应他熟肉吃。又看着徐某说:“你的骨突子怎么这样短?”众夜叉回答说:“他刚来,还没准备好。”大夜叉便从自己脖子上摘下明珠串,脱下十颗明珠赏给徐某。这些珠子都比手指尖大,圆圆的像弹丸一样。母夜叉急忙接了过来,替徐某穿好挂在他脖子上。徐某也学夜叉的样子,双臂交叉,说着“夜叉话”表示感谢。大夜叉便走了,驾着狂风,快得像飞一样,片刻便消失不见了。众夜叉吃了他剩下的熟肉,便散了。

又过了四年多,母夜叉忽然生产了。一胎生下两个男孩,一个女孩,都是人样,不像他们的母亲。夜叉们都很喜欢这三个孩子。常常一块逗弄他们。

一天,夜叉们都出去打食了,只剩下徐某一个人在洞里坐着。忽然从别的洞来了一个母夜叉,想跟徐某私通。徐不肯。母夜叉发怒,将他一下子扑翻在地。正好徐某的妻子从外面进来,见此情景,暴怒地冲上前去,撕打起来,一口把她的耳朵咬了下来。过了一会,那母夜叉的丈夫也来了,徐妻才放了她,让她走了。从此后,徐妻天天守着丈夫,一刻也不离开。三年后,孩子们已能走路了。徐某教他们说人的语言,渐渐地咿咿哑哑会说话,大有点人气了。虽然还是儿童,但登山如走平地一般;跟徐某依依恋恋,很有父子情意。

一天,母夜叉跟一个儿子和女儿外出,半天没回来。正好北风大作,徐某凄伤地想起故乡。便领着另一个儿子来到海岸边,见原来的船还在,便和儿子商量着返回老家。儿子想告诉母亲,徐某劝阻住了。父子二人登上船,顺风行驶,只用了一天一夜,便到达交州。到家后,徐某得知妻子已经改嫁走了。他拿出两颗明珠,卖了几万两银子,家境因而非常富裕。儿子取名叫徐彪,十四五岁时,就能举起几百斤重的东西,粗直刚猛,生性好斗。交州的驻军主帅见了他后很惊奇,便让他做了千总。正赶上边疆叛乱。徐彪在作战中所向披靡,立了很多功劳,十八岁就提升成了副将。

这时,有一个商人乘船渡海,也遭遇大风,被刮到卧眉山。刚上岸,见走来一个少年人。少年见了商人大惊,知道他是中原人,便询问他的家乡,商人说了。少年把他拉进一条深谷中的一个山洞里,洞外布满了荆棘丛,嘱咐他不要出去。少年离去了不一会,拿来鹿肉让商人吃,自己说:“我父亲也是交州人。”商人询问姓名,知道姓徐,自己认识他,便说:“你父亲是我的老朋友。现在他儿子已做了副将。”少年不知“副将”是什么意思,商人说:“这是中国的官名。”少年又问:“什么叫官?”商人回答说: “官就是出去乘漂亮车马,回家住高堂大屋;在上轻轻一呼,百人应声雷动;别的人不敢正眼看,只能侧身立,这就是官!”少年听得欢欣鼓舞。商人又问他:“你父亲既然在交州,你为什么长久留在这地方?”少年详细讲了以前的事情。商人便劝他返回故土,少年人说:“我也常常这样想。但母亲不是中国人,语言相貌都跟那里不同。况且,一旦走不成,同类知觉必被残害。因此踌躇不决,拿不定主意。”说完少年便走了,临出洞时跟商人说:“等起了北风,我来送你回去,麻烦你给我父亲,哥哥带个信去。”

商人在洞里一直藏了将近半年。他不时从洞口荆棘丛中往外窥视,见山中总有夜叉来来往往,吓得他一动也不敢动。一天,北风忽起,山中一片风吹树叶的唰唰声。少年忽然来了,领着他急急地逃窜。边逃边嘱咐他说:“我嘱托你的事不要忘了!”商人答应。于是,在少年的帮助下,商人终于逃了回来。一到交州,商人立即去副将府,跟徐彪详细讲了自己的见闻。徐彪听了又悲又喜,便要去寻找母亲、弟弟和妹妹。父亲担忧大海滔滔,又是去夜叉国,一路险恶,极力劝阻他不要去。徐彪捶胸痛哭,非去不可。父亲劝阻不住,只得由他。

徐彪便告诉了交州总帅,挑了两名健勇的士兵,乘船下了海。正赶上逆风,船行得十分艰难。在大海上颠簸了半个月,四周一望,只见海水茫茫,无边无际,再也分辨不出东西南北。忽然,一阵暴风吹来,波浪滔天,船被一下子打翻。徐彪落入水中,随着海浪漂流了很久,被一个怪物拖上了岸。怪物带着他来到一个地方,这里竟有房舍。徐彪醒了后,四下一看,一个像夜叉的怪物站在自己身边,便用“夜叉话”询问。夜叉惊讶地反问他,徐彪告诉他自己要去的地方。夜叉高兴地说:“卧眉山是我的故乡。刚才太冒犯你了。你离开去卧眉山沟路已八千里了,这条路是去毒龙国的,不去卧眉山。”于是找了条船送徐彪去卧眉山。夜叉在海水里推船疾行,像箭一样快,瞬间已跑了一千多里。过了一夜,来到卧眉山北岸。徐彪见岸上有个少年,正在眺望着茫茫无际的海水。徐彪知道深山里没有人类,怀疑那少年就是弟弟。走近一看,果然不错,兄弟俩手拉手痛哭起来。徐彪问起母亲和妹妹,少年回答说都很平安康健。徐彪便想和弟弟一起去寻她们,弟弟阻止了他,自己一人急急忙忙地走了。徐彪转身想感谢送自己来的夜叉,却见那夜叉不知什么时候已经走了。不一会儿,母亲和妹妹来了,看见徐彪都哭了起来。徐彪告诉母亲想接她们回去,母亲说:“恐怕去了后会被人家欺负!”徐彪说:“儿在中国非常荣华富贵,别人不敢欺负母亲。”于是,母子三人决意返回。但苦于正值逆风,难以行船。正在徘徊犹豫时,忽见船上的布帆向南飘动,起了瑟瑟北风。徐彪大喜。说:“天助我也!”四人一个跟一个上了船。北风很急,只用了三天,便抵达交州岸边。四人一上岸,看见他们的人以为是妖怪,吓得四处逃窜。徐彪便脱下自己的衣服,让他们三人分着穿上了。回到家中,母夜叉见了徐某,怒骂不止,恨他当初回来不跟自己商量。徐某连忙谢罪道歉。家里的人都来拜见主母,无不吓得浑身颤抖。徐彪便劝母亲学说中国话,又让她穿锦衣,吃肥肉,母夜叉才高兴起来。

母夜叉和女儿都喜欢穿男人服装,像满族人的打扮。几个月后,渐渐会说中国话了。弟弟妹妹的皮肤也逐渐变得白皙。弟弟叫徐豹,妹妹叫夜儿,二人都很勇猛有力。徐彪耻于自己不会读书写字,便让弟弟读书。徐豹很聪慧,经史书籍,一过目就明白了。但他不想做一个只会读书的文人,徐彪便仍然让他练习拉硬弓、骑烈马,结果考取了武进士,娶了阿游击官的女儿为妻子。夜儿因为相貌奇异,没人敢向她提亲。正好徐彪部下有个姓袁的守备死了妻子,徐彪便将妹妹硬嫁给了他。夜儿能开百石弓,百余步之外,用箭射小鸟,百发百中。袁守备每次出征,总是带着妻子。后来他一直升到同知将军,立下的功劳多半出自妻子之手。徐豹到三十四岁时,做了一个省的提督。母亲曾经跟着他南征,每次跟强敌对阵,母亲总是脱去盔甲,赤膊上阵,手持利刃为儿子接应。凡跟她接战的人,无不败得落花流水。后来,皇帝要诏封她为“男爵”,徐豹急忙上疏推辞,说明她是自己的母亲,皇帝才改封了她一个“夫人”的称号。


\subsection{1.3.19   小 髻}
\label{\detokenize{p00_u5176_u5b83/_u767d_u8bdd_u804a_u658b_u5fd7_u5f02:id105}}
长山县有个居民,在家闲居,常有个矮个子人来,与他长时间闲聊,他一直不知道这个人是哪里人,干什么的,颇为怀疑。

一天,客人说:“三几天我就搬来住,咱们就成邻居了。”过了四五天,又说:“现在咱们已经同庄住了,早晚可以来讨教。”主人问他:“迁住在什么地方?”那人也不细说,只是用手向北指了指。从此,每天总来一次,时常向邻居借器具用。有的人吝啬不借给他,器具就不翼而飞。众人都怀疑他是狐。

村北有一个古墓,非常深,看不见底,众人怀疑他可能住在里边。大家拿着兵器、木棒去围剿他。有人趴在墓口听了听,很久没有动静。一更天将尽的时候,听到墓穴中好像有几百人对着耳朵小声说话。大家都一动不动地等着。一会儿,一尺多长的小人爬了出来,络绎不绝,数也数不过来。大伙一声喊叫,共同出击,每打到他们,杖杖都打出火来。转眼之间,小人四散奔逃。只留下一个小髻,像核桃那样大,上面还扎着纱,镶着金线。用鼻子嗅一嗅,骚臭不可闻。


\subsection{1.3.20   西 僧}
\label{\detokenize{p00_u5176_u5b83/_u767d_u8bdd_u804a_u658b_u5fd7_u5f02:id106}}
有两个从西域来的和尚,一个去了五台山,另一个要去泰山。他们的衣服颜色、语言相貌,跟中国都不一样。自己说:“曾经过火焰山,峰峦重叠,烟气蒸腾,热得就跟炉灶一样。凡要翻过这座山,必须在雨后才能走。走时要聚精会神,双眼凝视着地面,轻轻地抬脚,慢慢地走;一不小心误踏到山石上,就会立即冒出烈焰,把人烤伤。还经过流沙河,河里有座水晶山,陡峭的悬崖绝壁直插天际。山峰四面都晶莹清澈,像透明一般。还有一座关隘,宽窄仅能容一辆车子通过。有两条龙,口角相交,把守着这里。凡过关的人须先拜龙,龙如同意过,口角就会自己张开。龙的颜色是白色的,身上的鳞鬣都像水晶的一样。”

和尚又说:“我们共在路上走了十八年。刚离开西方时,有十二人,等来到中国,只剩下了我们两个。西方都传说中国有四座名山,一个泰山,一个华山,一个五台山,还有一个洛伽山。相传这些山上遍地都是黄金,观音菩萨、文殊菩萨就在这些山上住,凡能去这些地方的人,都会变成佛身,可以长生不老。”听这西域和尚说话的口气,就跟我们这里的羡慕西方乐土是一样的。倘若有去西方游历的人,和来东方求佛的人中途相遇,双方分别说说本地的实际情况,一定都会相视失笑,免除万里跋涉之苦了。


\subsection{1.3.21   老 饕}
\label{\detokenize{p00_u5176_u5b83/_u767d_u8bdd_u804a_u658b_u5fd7_u5f02:id107}}
山西泽州有一绿林豪杰,名叫邢德。他善拉强弓,射连珠箭,被称为一时绝技。但此人一生潦倒失意,运气不佳,不善于经营谋利,出门做买卖总是亏本。南北两京的大商人却总是喜欢和他结伴,路上有了他就不用害怕了。正值初冬时节,有两三个商人借给邢德一点钱,邀他一同去贩运;邢德也拿出自己所有的钱,准备做件大买卖。他有一个朋友很会算卦,就去问问吉凶。友人算了一卦说:“这一卦是个‘悔’字,你这次的生意不但赚不了钱,怕是还要亏本。”邢德听了很不高兴,打算不干了,可那几个商人强拉着他匆匆上了路。到了京都,果然像卦里算的赔了老本。腊月中旬,他单人匹马出了城门,自己想到来年身无分文,更加忧闷。

这时,晨雾迷蒙,邢德暂时走进路旁一家酒店,解下行装寻酒喝。看见一白发老翁和两个少年在北窗下同桌喝酒,一个蓬松着满头黄发的童仆在旁边侍候。邢德在南边,面对老头坐了下来。那童仆给白发老头三人斟酒时,不小心弄翻了菜盘,沾污了老头的衣裳,少年生了气,立刻狠揪童仆的耳朵,又拿起手帕给老头擦拭。这时,邢德看见童仆的拇指上套着半寸来厚的铁箭环,每一个铁环大约有二两多重。吃过饭,老头让少年从皮口袋中拿出银钱放在桌上,称称算算,约有喝数杯酒的功夫,才将银钱包裹起来。少年从牲口棚里牵出一头瘸腿的黑骡子来,扶老头骑上;童仆也骑上一匹瘦马跟着出了门。两少年各自腰佩弓箭,牵着马出了店门。

邢德看见老头有那么多银钱,眼馋得像要冒出火来。酒也不喝了,急忙尾随而去。他看见老头与童仆还在前面慢慢地走着,就离开大路抄小路斜插到老头前面,气势汹汹地面对老头,带住马,张弓待射。老头俯身脱掉左脚靴子,微笑着说:“你不认识我老饕吗?”邢德拉满弓一箭射去,老头仰卧在马鞍上伸出脚来,张开两个脚趾像钳子一样夹住了飞箭,笑着说:“就这么点本事,还用得着老子用手来对付吗?”邢德火了,使出他的绝招,前箭刚发后箭又到。老头用手抓住一支箭,似乎没有防备他的连珠箭,后一支箭直射进他的嘴里。老头从马上跌落下来,嘴里含着箭直挺挺躺在那儿,童仆也跳下马来。邢德很高兴,以为老头已经死了,刚走到近前,老头吐出箭跳了起来,拍着巴掌说:“初次见面,怎么这样恶作剧?”邢德大吃一惊,马也惊得狂奔起来。这才知道老头是个奇人,不敢再找老头的麻烦了。

走了三四十里路,邢德正碰上往京城押送财物的官差,便拦路抢劫了钱财,大约有千两左右。邢德这才觉得得意起来。正在策马疾驰,听到后面传来一阵马蹄声。回头一看,原来是那个童仆换乘了老头的瘸骡飞驰而来,大喊:“那男子别走!你夺取的东西应少分一点给我。”邢德说:“你认识‘连珠箭’邢某吗?”童仆说:“刚才已经领教过了。”邢德以为童仆其貌不扬,又无弓箭,容易对付,一连发了三箭,连续不断如同群鹰飞冲。童仆却不慌不忙,手接住两支,嘴衔住一支,笑着说:“这样的技艺真丢人死了!你老子来得匆忙,没空找得弓来,这箭也无用处,还给你吧。”说着从大拇指上脱下铁环,将箭穿了进去,用手使劲一扔,呜鸣风响。邢德急忙用弓去拨箭,弓弦碰到铁环上,嘣的一声断了,弓也震裂了。邢德吓坏了,来不及躲避,箭已穿过耳朵,不觉翻身掉下马来。童仆跳下马来就要搜刮银两。邢德躺在地上举弓向童仆打去。童仆夺过弓,一折两段,又一折成了四段,扔到一边。接着就一只手握着邢德的胳膊,一只脚踩着邢德的两腿。邢德觉得两只胳膊好像被捆住了,两条腿好像被压住了,用尽力气也不能动一动。邢德腰中缠着两层三指宽的带子,童仆用手一捏,那带子随手断如灰烬。童仆搜取了银子,跳上瘸骡子,把手一举,说了声: “莽撞了。”疾速而去。

邢德回到家里,成了一个安分守己的善人。他常常给人讲过去的这些事,毫不隐讳。这和刘东山的故事大致差不多。


\subsection{1.3.22   连 城}
\label{\detokenize{p00_u5176_u5b83/_u767d_u8bdd_u804a_u658b_u5fd7_u5f02:id108}}
晋宁县有一个姓乔的书生,少年时就很有才气,但二十多岁了,依旧穷困潦倒。乔生为人正直,他有一个好朋友,姓顾,早年就死了,乔生经常接济他的妻子儿女。本县县令因为乔生的文章写得好,对他很器重。后来,县令死在任上,家口滞留晋宁,无法返回故乡。乔生变卖了自己的家产,买了棺枢,往返两千多里,把县令的遗体连同他的家人一起送回了家乡。因为这件事,当时的文人们更加尊重乔生,但乔生却因此更加贫穷了。

当时,一个姓史的举人有个女儿叫连城,精于刺绣,又知书达礼,史举人非常宠爱她。一次,史举人拿出一幅女儿绣的“倦绣图”,征求年轻书生就图题诗,意思是要借此选个有才学的好女婿。乔生也作了一首诗献上,这首诗说:“慵鬟高髻绿婆娑,早向兰窗绣碧荷。刺到鸳鸯魂欲断,暗停针线蹙双蛾。”又题了一首诗,专赞这幅图绣得精妙:“绣线挑来似写生,幅中花鸟自天成。当年织锦非长技,幸把回文感圣明。”连城见到这两首诗,非常喜欢,便对着父亲夸奖乔生的才华。但父亲嫌乔生太贫穷,不愿找这么个女婿。此后,连城逢人就夸乔生,又派了个老妈子,假借父亲的名义赠给乔生一些银两,作为他读书的费用。乔生感叹地说:“连城真是我的知己啊!”对她一往情深,如饥似渴地想念她。

不久,连城跟一个名叫王化成的盐商的儿子订了亲,乔生才开始绝望起来。但仍然梦魂萦绕,无时无刻不想着连城。不长时间,连城便生了重病,卧床不起。有个从西域来的和尚,自称能治好她的病,但必需一钱男子胸上的肉捣碎了配药。史举人派人去告诉王化成。王化成笑着说:“傻老头!想叫我剜心头肉吗?”把派去的人又打发回来。史举人便对人说:“谁愿从自己身上割下肉救我女儿,我便把女儿嫁给他!”乔生听说,立即赶到史家,自己掏出把刀子,从胸上一刀割下片肉,交给了和尚。鲜血染红了乔生的衣服,和尚忙给他敷上刀伤药才止住了血。和尚用乔生的肉和了三个药丸,给连城分三天服下,病果然好了。史举人便想履行诺言,把连城嫁给乔生。先去通知王化成,王化成大怒,要告状打官司。史举人害怕,便摆下宴席,将乔生请来,然后取出一千两银子,放在桌子上,说:“辜负了您的大恩大德,就用这些银子报答您吧!”并对乔生讲了毁约的缘由。乔生生气地说:“我所以不吝惜心头肉,不过是为了报答知已罢了,难道我是卖肉的吗?”说完,拂袖而去。连城听说后,心里很不忍,托老妈子去劝慰他。并说:“以他的才华,不会久处人下的,何愁天下没有美女?我近来做的梦都不吉利,三年内必死,不必跟别人争我这个泉下之鬼了!” 乔生告诉老妈子说:“古人说:‘士为知己者死’。我报答她不是为了她生得漂亮。我真怕连城未必真知我的心,如果真知,就是做不成夫妻又有何妨呢?”老妈子忙替连城表白了她的一片真情。乔生说:“果然这样,我们相逢时,她若为我笑一笑,我就死而无憾了!”

老妈子回去不几天,乔生偶然出去,正好遇上连城从叔家回来。乔生看着她,连城也看见了他。只见她秋波送情,微微地启齿一笑。乔生大喜。说:“连城真是我的知心人!”过了不久,王盐商家来到史家商议连城的婚期。连城听说后又病了,几个月便死了。乔生前去吊唁,痛哭一场,也死了过去,史家把他抬回家中。

乔生知道自己已经死了,也没感到有什么难过。一个人出了村,还想着再见见连城。远远望见有条南北大路,路上的行人像蚂蚁一样拥挤。他也走了过去,混杂在人群里。一会儿,进入一座衙门,正碰上他过去的好朋友顾生。顾生看见他,惊讶地问:“你怎么来了!”说着,就拉着乔生的手,要送他回去。乔生长长地叹息了一声,说:“我的心事还没了!”顾生便说:“我在这里掌管典籍,很受上司信任。有用得着我的地方,我一定尽力!”乔生便向他打听连城在哪儿。顾生领着他串了很多地方,最后才发现连城和一个穿白衣服的女郎,眼泪婆娑地坐在一条走廊的一角。连城看见乔生,急忙起身,像是喜出望外,略问了问他是怎么来的。乔生说: “你死了,我怎敢偷生世上!”连城听了,哭着说:“我这样一个忘恩负义的人,你还不唾弃我,又以身殉死干什么!我今生今世不能跟你了,来生我一定嫁给你!”乔生回头告诉顾生说:“你有事就忙去吧,我觉得死了很快乐,不想再活了。只想麻烦你代为访查一下连城托生到什么地方,我要和她一起去!”顾生答应着走了。

这时,那白衣女郎问连城乔生是什么人。连城便向她讲述了往事。女郎听了像压抑不住心中的悲伤。连城告诉乔生说:“这姑娘与我同姓,小名叫宾娘,是长沙史太守的女儿。我们一路同来,处得很亲密。”乔生打量了打量宾娘,见她哀伤凄惋的样子,十分惹人怜爱。正想再问什么,顾生已返了回来,向乔生庆贺说:“我给你办妥了,就让小娘子跟你一起还阳复生,好不好?”两人听了,都很喜欢。正想拜别顾生,宾娘大哭着说:“姐姐走了,我去哪里?恳求您可怜可怜,救救我,我就是给您当仆人也愿意!”连城心里难过,想不出办法,就和乔生商量,乔生转而哀求顾生帮忙。顾生很为难,一口咬定说不好办。乔生执意恳求,顾生才无可奈何地说:“我去胡乱试试看吧!”去了有一顿饭的工夫,便回来了,连连摆手说:“怎么样!我实在无能为力了!”宾娘听说,哀哀地啼哭着,依在连城的胳膊下恋恋不舍,恐怕她马上就走了。三人相对无语,一筹莫展。再看看宾娘那愁苦凄伤的样子,真让人心里发酸。顾生奋然而起,说:“你们带宾娘一起走吧。真有罪责,我豁上这条命一人承担了!”宾娘听了才高兴起来,跟着乔生一块出去。乔生担心她一人去长沙路太远,又没有伴。宾娘说:“我想跟你们走,不愿回去了!”乔生说:“你太傻了!不回去,见不着你的尸身,怎么能还阳呢?以后我们到了湖南,你不躲着我们,我们就很荣幸了!”正好有两个老婆婆拿着勾牒要去长沙勾人,乔生便把宾娘托付给她们,然后洒泪而别。

路上,连城走不动,走一里多路就得歇息歇息。共歇了十多次,才看见本村的庄门。连城说:“还阳后恐怕我们的事又有反复。请你先去我家,索要我的遗体,然后我在你家重生,我父亲应当不会再反悔了!”乔生认为很对,两人便先去乔生家。连城战战兢兢地像走不动了,乔生站住,等着她。连城说:“我走到这里,禁不住浑身发抖,六神无主,真担心我们的心愿实现不了!我们还得再好好商量商量,不然,我们活了后,可就又身不由己了!”两人相互搀扶着,进入一间厢房中,过了很久,谁也没说话。连城忽然笑着说:“你厌恶我吗?”乔生惊讶地询问这是什么意思。连城害羞地说:“恐怕我们的事不成,那时就太辜负你了!请让我先以鬼身报答你吧!”乔生大喜,两人极尽欢爱。因为不敢急忙还生,两人徘徊不决,在厢房中一直呆了三天。连城说:“俗话说:‘丑媳妇终得见公婆’。老是在这里忧愁担心,终究不是长久之计!”催促乔生快去还阳。乔生一走到灵堂,猛然苏醒过来。家人非常惊异,给他喝了些汤水。乔生便派人去请史举人来,请求得到连城的尸身,说自己能让她复活。史举人大喜,听从了他的话。刚把连城抬进乔生家,一看,连城果然也已活了。连城告诉父亲说:“女儿已把自己许给乔郎了,再没回去的道理。父亲如不允,我只有再死!”

史举人回了家,便派了奴婢去乔家供女儿使唤。王化成听说后立即写了状子告到官府。官府受了王家的贿赂,将连城又判给了王化成。乔生愤懑不堪,直想死去,但终究还是无可奈何。连城到了王家,气愤愤地不吃饭,只求快死。看屋里没人,便把带子悬到房梁上上了吊,被人救下后。隔了一天,病得越重,眼看就要死了。王化成害怕,把她送回了娘家。史举人又把她抬到乔生家。王化成听说后,也没有办法,只得作罢了。

连城病好后,常常思念宾娘。想派个人捎信去,就便探望她。因为路太远,很难前去。一天,家人忽然进来禀报说:“门外来了好些车马。”乔生夫妇迎出屋门一看,见宾娘已在院子里了。三人相见,悲喜交集。史太守亲自把女儿送来了。乔生将他请进屋里,史太守说:“我女儿多亏你才能复生,她立誓不嫁别人,现在我听从了她的意愿!”乔生忙叩头拜谢。史举人也来了,还跟史太守叙上了同宗。

乔生名年,字大年。


\subsection{1.3.23   霍 生}
\label{\detokenize{p00_u5176_u5b83/_u767d_u8bdd_u804a_u658b_u5fd7_u5f02:id109}}
文登有个姓霍的书生,与一个姓严的书生小时很要好,长大了经常开开玩笑,顶起嘴来,谁也不让谁。

霍生邻居有个老婆子,曾给严生妻子接过生。有一次,婆子偶然与霍生妻子说起严生妻子阴部有两个肉瘤。霍妻又告诉了霍生。霍生与同学们谋划好了,听到严生将要来时,故意小声说: “某某人妻子曾与我私通。”众人敢意不信,霍生便捏造了始末情节,并且说:“你们不信,我知道她的阴部有两个肉瘤。”严生正走到窗外,听得明明白白,返身便走,回家拷打他的妻子。妻子说没有这回事,严生打得更厉害。妻子不堪其苦,就上吊自杀了。

霍生这才懊悔莫及,但又不敢向严生说明情况。后来,严妻冤魂夜夜哭闹,全家人不得安宁。没有多长时间,严生也突然死了,鬼也就不哭了。

此后,霍生妻子夜夜梦见一个女子披头散发朝她大喊:“我死得好苦啊!哪能叫你们夫妻欢乐?”霍妻醒来就得了病,几天也死了。霍生也梦见一个女子来指点着骂他,并用手打他的嘴。醒了以后,感觉嘴唇隐隐作痛,用手一摸,高高肿起,三日后成了两个小肉瘤,成为顽固的症状,不敢大声说笑,一开口就疼痛难忍。

又:本县有一个姓王的书生,与同学某生要好。某生的妻子要走娘家,王生知道某妻骑的驴怕惊,他就预先藏在路边草丛里。等某妻骑着驴走过来时,他猛地出来吓驴一跳,某妻就摔倒在地上了。赶驴的是个小憧,无力扶她上驴,王就殷勤周到地把某妻抱扶上驴。某妻也不认识王生是谁。

王生洋洋得意,对人说:小僮追驴去了,自己曾与某妻在路边草丛里私通,某妻穿的是什么袄,什么裤。说得清清楚楚。某生听后,非常羞愧地离开了。

过了一会,王生从窗缝中看见某生一手持刀,一手扯着他妻子朝自己走来,满面愤怒凶恶之色。王生吓得跳墙就跑,某生在后紧紧追赶不放。约追了二三里路,某生看看追不上王生,才回去。王生却因极力快跑,肺叶扩张,得了哮喘病,几年都治不好。


\subsection{1.3.24   汪 士 秀}
\label{\detokenize{p00_u5176_u5b83/_u767d_u8bdd_u804a_u658b_u5fd7_u5f02:id110}}
汪士秀,是庐州人,刚强勇猛,力气大得能举起几百斤重的石臼。他和他父亲都善于踢球。他父亲四十多岁过钱塘江时淹死了。又过了八九年,汪士秀有事去湖南,晚上停泊在洞庭湖。当时,圆月东升,澄江如练。正眺望时,忽见有五个人从湖中冒出来,带着一张足有半亩地大的席子,平铺在水面上。接着又纷纷摆出酒肴,盛酒肴的器皿发出一片温厚的摩擦碰动的声响,不像是陶瓷器皿。不一会儿,有三个人在席上坐下,另外两个人在一边伺候。坐着的三人中,一个穿黄衣服,两个穿白衣服,头上都戴着皂色的头巾,头巾高高的,后幅拖下来一直搭到肩背上,样式非常古老。月色迷茫,远远望去,看不清楚他们的面貌。伺候的两人,都穿褐色衣服,一个像是童仆,另一个像是老翁。只听黄衣人说:“今晚月色极好,很值得我们痛饮一场!”一个穿白衣的说:“今晚的风景,大有广利王在梨花岛摆宴时的样子呢!”三人互相劝酒,痛饮起来。说话的声音越来越小,汪士秀再也听不到了。给他撑船的船家吓得趴在那里,大气不敢出。汪士秀又仔细看了看那老翁,相貌非常像已经死去的父亲,但听他说话的声音又不是。

二更将尽时,三人中忽有一人说:“趁月光明亮,我们应该踢球为乐!”就见那童仆从水中取出一个圆球,有一抱大小,球中像是贮满了水银,表里透明。坐着的人都站起身来,黄衣人招呼老翁一块踢。那球被他们踢起有一丈多高,光芒四射,直刺人眼。一会儿,只见那球腾空飞起,远远地飞过来落在了汪士秀的船上。汪士秀不觉脚痒,飞起一脚,想把球踢回去。只觉那球异常轻软,这一下猛踢,似乎把它给踢破了,球飞起有几丈高,从破口处泻下一道银光,犹如彩虹,又如划过天空的彗星,一下子扎进了水里。接着水面冒出一阵气泡,球不见了。席上的三人都发怒说:“哪里来的生人,败坏我们的清兴!”老翁却笑着说:“不错不错。刚才那一脚正是我们家的‘流星拐’踢法。”白衣人怪他多嘴,嗔怒地说:“我们都在烦恼,老奴怎敢讲笑话?快和小崽子去把那狂人抓来!不然,我就用锤子砸断你的腿!”汪士秀见无路可逃,索性横下心,提刀立在船头上。一会儿,见童仆和老翁手持兵器冲了过来。汪士秀仔细一看,那老翁果然是父亲,急忙大叫:“阿爹,儿子在此!”老翁大吃一惊,父子相对悲伤。童仆见状,立即返了回去。老翁说:“儿子快藏起来,不然我们爷俩都要死了!”话还没说完,那三人突然出现在船上,面都如黑漆,眼睛比石榴还大,一把就把老翁抓了过去。汪士秀急忙奋力争夺,船被挣得摇晃不止,缆绳一下子断了。汪士秀挥刀向黄衣人砍去,把他的胳膊砍了下来,黄衣人负痛逃窜。另一个穿白衣的向汪士秀冲来,汪士秀又挥刀剁中他的头颅,扑通一声掉进水里。剩下一人也看不见了。汪士秀正和父亲商量着连夜乘船返回,忽然水面上冒出一张像井一样深的大嘴,四周的湖水哗哗地往里灌注着,砰砰地响,一会儿,那大嘴又把水往外一喷,波滔汹涌,高接星斗,湖里所有的船都颠簸起来,船上的人恐惧万分。汪士秀见自己的船上有两个石鼓,都有一百斤重,他便举起一个往那大嘴里投下去,激起雷鸣般的波涛。不一会,湖面渐渐平静,他又把另一个石鼓投了下去,才风平浪静。

汪士秀怀疑父亲是鬼。老翁说:“我本来就没死。在江上落水的十九人,都被妖怪吃了。我因为会踢球,才保住了命。那些妖怪得罪了钱塘江龙君,所以来洞庭湖避难。三人都是鱼精,刚才踢的球就是鱼胞。”父子二人都为了团聚而高兴,连夜划着船走了。天明后,见船上有片鱼翅,有四五尺长,才醒悟这就是夜晚被汪士秀砍断的黄衣人的那条胳膊。


\subsection{1.3.25   商 三 官}
\label{\detokenize{p00_u5176_u5b83/_u767d_u8bdd_u804a_u658b_u5fd7_u5f02:id111}}
诸葛城有一个叫商士禹的读书人,因酒醉后开玩笑,触犯了本县一个富豪,被富豪指使家奴殴打了一顿,刚抬回家中就死了。商士禹有两个儿子,大儿子叫商臣,二儿子叫商礼,还有一个女儿叫商三官,才十六岁。本来三官马上就要出嫁了,现在因父亲去世,把婚事给耽搁了。她的两个哥哥去告状打官司,打了一年也没打出个结果来。三官的婆家便派人拜见她母亲,商量着请女家迁就一下,将三官尽快从简嫁过去,母亲也准备答应。三官知道后,就去见母亲说:“哪里有父亲尸骨未寒就办喜事的道理?难道他家就没有父母吗?”婆家听了这话,很惭愧,就打消了原来的念头。

不久,三官的两个哥哥没打赢官司,含冤负屈地返回家来,全家人悲愤不堪。商臣、商礼还打算保留住父亲的尸体,以便作为日后再上告的证据。三官劝阻说:“人被杀死了,官府却不受理,可知这是什么世道了!难道老天会专为你们俩生一个阎罗包公吗?让父亲的尸骨长久暴露在外,于心何忍呢?”两个哥哥觉得妹妹的话有理,只得将父亲埋葬了。

丧事办完,三官突然在一夜失踪了,谁也不知她去了哪里。她母亲又着急、又害怕,唯恐她婆家知道,也不敢告诉亲戚邻居,只是嘱咐两个儿子暗暗访查她的踪迹。将近半年,三官仍然不见人影。

一次,打死商士禹的那个富豪正赶上寿辰,叫了几个戏子来演戏庆寿。戏子领头的叫孙淳,带着两个徒弟。一个叫王成,姿色平平,但唱得清脆动听,大家纷纷叫好。另一个叫李玉,相貌秀丽温雅,像个漂亮的女子。有客人叫他唱歌,他推辞说不会;再三要他唱,他才唱了些掺杂着本地歌谣的土腔土调,客人们哄堂大笑,乱糟糟地鼓起掌来。孙淳非常羞惭,禀告主人说:“我这弟子跟我学艺不久,还不能唱,只能做些斟斟酒之类的事,请不要见怪!”便命李玉斟酒。李玉往来伺候,很会看主人的意思给客人斟酒,富豪大为高兴。等酒席撤下、客人散去后,便留住李玉,要和他同床共枕。李玉替富豪扫了床,又替他脱了鞋子,殷勤侍奉。富豪大醉中,不断说些挑逗的话,他也只是微微地笑着。富豪更加神魂颠倒,把仆人们全部赶走,只留下李玉。李玉见仆人们都走了,便关上门,插上门闩。仆人们也都到别的屋子里喝酒去了。

不一会儿,有个仆人听见富豪卧室内传出一阵奇怪的格格声,忙过去往屋里偷偷地看了看,见屋内漆黑一团,无声无息。心想没什么事,刚转过身来要走开,忽听屋星“呯”的一声大响,像是悬挂着的重东西断了绳子掉到地上发出的声音。仆人急忙向屋里问了两声,静静地没一点回答。仆人急叫众人撞开门冲进去,只见主人的脑袋已和身子分了家,李玉也自已吊死了。因吊着他的绳子断了,所以尸体掉到了地上,房梁上还残留着一截绳子头。众人大惊失色,急忙通知富豪家里人。大家都聚集到一起,谁也猜不透是怎么一回事。众人把李玉的尸体搬到院子里,一抬起来后,觉得他鞋袜内空空的,像没有脚一样。脱下鞋一看,只见一弯小脚,才知李玉原来是个女子!大家更加惊骇,叫过孙淳来,仔细盘问。孙淳早已吓得魂不附体,不知说什么才好,只是说:“李玉一个月前才投奔我做弟子,这次他自愿跟来给主人庆寿,我实在不知他是从哪来的!”众人见李玉身穿丧服,怀疑她是商家的刺客,便命两个人暂且看守住尸体,好去官府上告。女子虽然死了,面貌仍然栩栩如生,用手一摸,身上还是温暖的。这两个看守的人动了邪念,商量着要奸尸。其中一人抱起尸体,正转动着想解开她的衣服,忽然脑后像被什么东西猛砸了一下,嘴一张,鲜血狂喷,片刻就一命呜呼了!另一人大惊,急忙告诉了众人。众人听了又惊又惧,不由得把李玉的尸体看作是神明一般。

富豪家告到郡里后,郡守便将商臣、商礼拘了去审讯。二人只是说:“我们不知道这事。妹妹逃走后,到现在已半年不见人了!”郡守便带了他们二人去验尸,死者果然是商三官!郡守很感惊奇,便判决商臣、商礼将妹妹的尸体领回埋葬,并敕令富豪家此后不得跟商家为仇。


\subsection{1.3.26   于 江}
\label{\detokenize{p00_u5176_u5b83/_u767d_u8bdd_u804a_u658b_u5fd7_u5f02:id112}}
乡里有个叫于江的,他父亲夜里睡在地头上,被狼吃了。于江当时只有十六岁,拾到父亲遗留下的鞋,痛恨得要死。夜里等到母亲睡着了,他偷偷地拿着铁锤,来到父亲睡觉的地头上,希望能为父亲报仇。

不一会儿,一只狼来了。狼迟疑徘徊地嗅着于江,于江一动也不动。不多时,狼摇着尾巴扫于江的额头,渐渐又低头舔于江的大腿,于江仍然一动不动。狼欢跳着直扑上前,要咬于江的脖子。于江急用铁锤猛击狼的脑袋,狼立刻被打死了。于江起身把狼放在草丛中。不多时,又来了一只狼,同前面那只狼一样,又被于江打死了。于江一直躺到半夜,再没有狼来,就迷迷糊糊睡着了,梦见他父亲告诉他说:“你杀了这两只狼,足以解我的恨了!但领头杀我的狼,鼻子是白的,死了的这两只都不是。”于江醒了,继续躺在原地等着,天亮了,没有狼再来。于江想把那两只狼拖回家,又恐怕吓着母亲,就把狼扔到了枯井里,自己回去了。

到了夜里,于江又来到田间,还是没有狼来。就这样过了三四夜,于江正睡着,忽然来了一只狼,咬住他的脚,拉着他走。走了几步,棘针刺进于江肉中,石头磨伤了于江的皮肤,于江就同死了一样。狼就把于江放在地上,想要咬他的肚子。于江猛然挥起铁锤朝狼打去,狼被打倒了;又接连打了几锤,狼才死了。于江仔细一看,真是只白鼻子狼。于江非常欢喜,背着死狼回了家。这才把报仇的事告诉母亲,母亲哭泣着跟于江到田间,果然从枯井中找到两只死狼。


\subsection{1.3.27   小 二}
\label{\detokenize{p00_u5176_u5b83/_u767d_u8bdd_u804a_u658b_u5fd7_u5f02:id113}}
滕县有个叫赵旺的人,夫妻二人都信佛,不吃荤,被村中的人看做“善人”,家中过着小康生活。他们有一个女儿叫小二,长得聪明美丽,赵旺夫妻爱如掌上明珠。小二六岁时,就让她与哥哥赵长春一起跟老师读书,五年的工夫熟读了五经。同学中有个姓丁的学生,字紫阳,比小二大三岁,长得风流潇洒,文采也很好,他们二人互相爱慕。丁生私下告诉母亲,向赵家提亲。而赵旺想让女儿找个有钱的大户人家,所以没有答应这门亲事。

过了不多时,赵旺参加了白莲教。徐鸿儒造反后,一家人都成了贼寇。小二知书善解,对剪纸作马,撒豆成兵的法术,都能一见就通。有六个小女孩跟徐鸿儒学艺,唯有小二学得最好,因而很快学到了徐的法术。赵旺也因为女儿学的武艺好而得到了重用。

这时,丁生已十八岁了,在县里中了秀才,一直没有成亲,因他心里忘不了小二。一天,他忽然从家里逃了出来,投到徐鸿儒部下。小二见了很高兴,对丁生特别好。小二是徐的高徒,在徐部主持军务,日夜忙碌,连自已的父母都不常见,可他与丁生每晚都在一起谈话,并且谈话时将仆役都打发走,每每谈到夜里三更多天。有一次,丁生私下对她说: “我来这里,你知道是为什么吗?”小二回答说:“不知道!”丁生说:“我不是为了想出人头地。我所以来,实在是为了你。白莲教本是左道旁门,无济于事,只能是自取灭亡。你是聪明人,难道不明这个道理吗?你若能跟我走,就不辜负我找你这份心意了。”小二听了,黯然地思索了一会,心里如梦初醒。她对丁生说: “咱们背着我父母走了,是为不义,咱们去告诉他们!”于是二人到了赵旺夫妇处,向他们说明利害。可赵旺不觉悟,还说:“我师傅是神人,绝不会错!”小二知道不能再劝了,就把辫子梳成小髻,拿出两个纸鸢,与丁生每人骑一个。纸鸢慢慢展开双翅,像比翼鸟一样双双飞走了。

天明,来到莱芜地界,小二用手捻一下鸢脖子,二人就双双着了地。他们收了鸢,换骑两匹驴,一路小跑奔驰到山里,假装是来避难的,赁了房子住下了。

二人逃走时,因为比较匆忙,带的衣服不多,柴米也没有。丁生很是犯愁,向邻居家借,也没有人肯借给。然而小二却面无愁容,只是卖簪子、耳环等首饰度日。二人闭门静坐,互相猜灯谜,背诵过去学过的书,以赌输赢、论高低。谁输了,谁就被对方用手指打板子。

他们住的西邻有个姓翁的人,是个绿林好汉。一天打猎回来,被小二看见了,对丁生说:“这个人很富,我们愁什么?暂借他一千两银子用用,不知肯借不肯借?”丁生认为不好办。小二说:“我要让他自愿拿出银子来!”她就剪了个纸判官,放在地上,盖上个鸡笼子,然后拉着丁生上了床,摆上存下的一点洒,拿出《礼记》来行酒令。随便说书上第几册、第几页、第几行,然后翻书检阅。如果这一行是“食”字旁,“水”字旁或“酉”字旁,就喝一杯酒;若是“酒”字部,就加倍喝。小二正好翻到“酒人”,丁生就以大杯斟满给小二喝。小二祝祷说:“我若是能借来银子,你就得‘饮部’。丁生一翻书,得“鳖人”。小二大笑着说:“事情成了!”斟上酒拿给丁生。丁生不服。小二说:“你是水族,应该和鳖一样喝酒。”正在互相喧闹间,忽听鸡笼里嘎嘎有声。小二说:“来了!”打开鸡笼一看,下面满满一袋银子。丁生又惊又喜。

后来,翁家一个妇女抱着孩子来串门,偷着说:‘我家主人刚从外边回来,点上灯才坐下,就见地上忽然裂了一道缝,深不见底。一个判官从缝里出来说:“我是地府的官吏。泰山帝君召集阴曹官吏造恶人名录,需要银灯一千架,每架用银子十两。你施舍一百架,就能消除你的恶行。’我家主人害怕已极,烧香叩头,捐上一千两银子,判官才回去了,地上的缝也合起来了。”丁氏夫妻听了,故意装得非常诧异。

自此以后,丁氏夫妻渐渐购买牛马,雇用丫鬟、仆人,自己新盖了房子。本村的一帮无赖之徒,见他们一下子富起来,就纠集一伙坏人,跳墙进了丁家抢劫。丁氏夫妇从梦中醒来,点着苫子一照,贼已满了屋子。两个贼捉住丁生,一个贼伸手向小二怀中乱摸。小二赤着身子起来,用手一指说:“别动!别动!”就见贼寇十三人都吐着舌头,呆若木鸡,一动也不能动。小二这才穿上衣服下床,招呼众家人来,把盗贼一个一个都绑起来,逼他们招供了罪行。小二于是责备盗贼说:“我们是从远处来这里避难的,希望大家互相帮助,为什么你们竟不仁不义到这种地步!人都有一时富裕贫穷的时候,日子困难的不妨明说,我岂是那种视财如命的守财奴?按你们的这种豺狼行为,本应都杀掉,可我心里不忍。暂时先放了你们,以后要是再犯,定杀不饶!”盗贼们叩头谢恩而去。

小二与丁生在这里住了不长时间,徐鸿儒就被官府擒住了,赵旺夫妇也诛连被杀。丁生帮助小二带了银子去官府赎回哥哥赵长春的小孩。这孩子当时才三岁,丁生把他当自己的儿子来抚养,改姓丁,叫丁承祧。于是这村中的人,渐渐知道丁氏一家是白莲教的遗属。这年正遭蝗灾,小二剪了几百只纸鸢放在自己的地里,吓得蝗虫都飞不进她的田,免了一场灾害。村中的人都嫉恨他们,向官府告发他们是徐鸿儒的余党。官府见丁家很富有,想敲诈他们,就把丁生抓起来。丁生拿钱重重贿赂县官,才免了灾。小二说: “咱们的钱来得不太明白,可以散散财。但这里的人心如蛇蝎,不能久住。”因此,他们就贱价变卖了家产,搬到益都西边去住。

小二为人心灵手巧,会过日子,经营家业比男人还强。他们开了个琉璃厂,雇了工人,小二亲自教他们制作技术。他们生产的玻璃灯具,样式奇巧,色彩缤纷,其它厂子都比不上。因此,他们生产的货虽然价钱高,可还是卖得很快。几年后,丁家就更豪富了。小二管理工人很严格,几百人干活,没有敢偷懒的闲人。小二工作之余,经常与丁生品茶、下棋,或者以看史书为乐。家里的财务收支及奴婢、仆人的工作,小二都是每五天检查一次。检查时,她手里拿着计工作数量的筹子,丁生拿着名册点名。对勤快的进行奖赏,多少不等;对懒惰的当众打板子,或者罚跪。检查的这天,全体放假休息,晚间不干活。小二与丁生招呼奴婢唱俚曲饮酒作乐。小二明察秋毫,没有人敢欺骗她。奖赏时又超过工人的劳动,所以事事顺利;村中二百多户人家中,有个别穷的,小二就酌情帮助他们些资本谋生,所以,这村里没有无业游民。

有一年大旱,小二命人在野外设坛,夜里坐车到坛上,作起法术,就下了大雨,五里以内雨水充足。人们更感到她的神奇。小二出门从不遮面孔,村里人都认得她。有的少年聚起来议论她长得漂亮,但见到她时,都肃然起敬,没有敢仰头直看她的。每年到了秋天,村中的童子不能干重活的,小二都给孩子钱,叫他们去采野菜,二十年积了一楼阁。村里的人都笑她。可是后来山东发生了灾荒,饿得人吃人。这时,小二拿出野菜来掺上粮食给人吃,邻近村的人都得了救,没有到外地去逃荒的。


\subsection{1.3.28   庚 娘}
\label{\detokenize{p00_u5176_u5b83/_u767d_u8bdd_u804a_u658b_u5fd7_u5f02:id114}}
金大用是中州旧官宦人家的子弟,娶的是尤太守的女儿,名叫庚娘,长得既美丽又贤惠,夫妻俩感情很深。那时正是兵荒马乱的年头,金大用一家远离故乡,到南方逃难。路上遇到一位少年也带着妻子逃难,自称是扬州人,名叫王十八,愿意在前面引路。金大用很高兴,两家人便同行同住。

这天,到了一条河边,庚娘偷偷告诉金大用说:“不要和那少年同乘一条船。他总是盯着我看,眼珠乱转,神色不正常,好像心术不正!”金大用答应了。王十八殷勤地雇了条大船,帮着金家搬运行李,忙忙碌碌,非常周到。金大用不忍拒绝他的好意,又想到他还带着少妇,不该有什么问题。少妇与庚娘住在一起,看上去也很温顺和气。王十八坐在船头上,同船家亲近地说着话,好像是早就认识的亲朋好友。不多时,太阳落山了,辽阔的水面一望无际,分不清东西南北。金大用看到四周荒凉险恶,心中很是疑惑奇怪。船行了一会儿,月亮升起来了,只见到处是芦苇。船停下后,王十八邀金大用父子到船头望望风景,乘机将金大用挤下水去。金大用的父亲看见刚要呼喊,船家用篙一下把他打落水中。金母听到声音出来察看,又被打下船去,王十八这才喊救人。刚才金母出来时,庚娘在后边,已察觉刚才发生的事。听到一家人都掉进河里,也不惊慌,只是哭着说:“公婆都淹死了,我到哪里去呢!”王十八进来劝她:“娘子不要忧虑,请跟我到南京去吧。我家有房子有地,很富裕,保你吃穿不愁。”庚娘止住泪说:“要能这样我就满足了。”王十八非常喜欢,一路殷勤地伺候庚娘。到了晚上,王十八拉住庚娘求欢,庚娘假托来了月经,王十八就到少妇那里睡了。天将初更,只听王十八夫妇吵了起来,也不知什么原因,只听到女的说:“你办这种事,怕雷霆会劈碎你的头!”王十八就打那女人,女的喊起来:“死了算了!实在不愿给杀人贼当老婆!”王十八吼叫着把女人拖出船舱,只听到咕咚一声,接着就听到喊妇人落水了。

过了几天,到了南京,王十八领庚娘回到家,上堂拜见母亲。王母惊讶不是原来的媳妇了。王十八说:“原先的媳妇掉到水里淹死了,这个是新娶的。”回到房里,又要亲近庚娘,庚娘笑着说:“三十多岁的男人了,还不懂这人情世事吗?普通人家成亲,还得喝一杯薄酒呢;你家中这么富裕,当然不难办到。如没有几分酒意,草率行事,成什么样子?”王十八很高兴,置办了酒席,两人对坐饮酒。庚娘拿着酒壶殷勤地劝酒,王十八慢慢有些醉了,推辞不喝了。庚娘换了大碗,媚笑着强要他喝,王十八不忍拒绝,又喝了下去,不禁酣然大醉,脱了衣服睡到床上,催促庚娘快睡。庚娘撤了灯烛,借口小解,走出房门,拿了把刀进来;摸黑来到床前,伸手摸王十八的脖子,王十八还抓着庚娘的胳膊,说着亲热的话。庚娘用力一刀砍下去,没把他砍死,王十八叫着要爬起来;庚娘又砍了一刀,王十八这才死了。王母好像听到响声,过来问出了什么事,庚娘也把她杀死了。王十八的弟弟王十九发觉了,庚娘知道不免一死,立即挥刀自杀。可刀刃卷了,砍不进去,她便打开门跑了出去。等王十九追出来,她已跳进池塘里了。十九急忙呼告邻居,把庚娘捞上来,见已经死了,但面色端庄艳丽,依然同活着一样。大伙一同检验了王十八的尸首,看见窗上有一封信,打开一看,原来是庚娘写的,信里详细讲述了她全家的冤情。众人都认为庚娘是个烈女子,商量好敛钱给她出殡。天亮后,来看的人有好几千,见了庚娘,个个敬佩,人人朝拜。一天的时间,就敛得了上百两银子。好心的人们为她买了珠冠袍服、金银首饰,上等棺材和很多随葬东西,把她葬在了南郊墓地。

当初,金生被挤入水中后,幸亏浮在一片木板上,才大难没死。天亮时,漂到淮河上,被一条小船救上来。这条小船是富户尹老汉专门为搭救落水遇难人设置的。金大用清醒后,去登门拜谢,尹老汉优厚地待承他,要留下他教自己的儿子读书。金大用因为不知道亲人的消息,想前往探访,所以拿不定主意是走是留。这时听说:“捞上来了淹死的老头和老妈妈。”金大用疑心是自己的父母,急忙跑去看,果然不错。尹老汉代他买了棺木,金大用正在哀伤痛哭,又听说:“救了一个落水的女人,自称金大用是她丈夫。”金大用擦干泪惊疑地跑出去,那女子已经来了。并不是庚娘,而是王十八的妻子,向着金大用大哭起来,请求收留她。金大用说:“我心绪已乱,哪有心思替你打算!”女子哭得更厉害了。尹老汉问明缘故,说这是老天的报应,劝金大用收留这女子为妻。金大用借口服丧,况且还打算报仇雪恨,怕有家是累赘。那女人说:“如果像你说的,要是庚娘还活着,你也会为了报仇而抛弃她吗?”尹老汉觉得这女子说话在理,就提出暂时代金大用收留这女人,他勉强应允了。大用埋葬父母时,那女人披麻戴孝,哭得非常悲痛,如同死的是自己的公婆。办完葬事,金大用怀揣利刃手托饭钵,要去扬州报仇。女人劝他说:“我姓唐,祖籍是南京,和那个豺子是同乡。以前他说是扬州人,都是骗人的;况且江湖上的水寇多半是他的同党,你这样去怕是报不了仇,还会惹祸。”金大用听她一说,犹豫不定。这时忽然传来烈女子杀人报仇的事,这事在沿河一带流传很广,姓甚名谁非常详细。金大用听了很痛快,但知道庚娘死了也更加悲痛。就辞谢唐氏说:“幸亏我没做有辱你的事。我家有这样的烈女子,怎能忍心负她另娶呢?”唐氏以他们先前已有夫妻之约,不肯中途离开,愿意做妾。

正巧有个姓袁的副将军,同尹老汉交情很深,路过这里西去,前来看望尹老汉,见到金大用,非常喜爱,请他当了军中的书记官。过了一阵子,流寇造反,袁将军立了大功。金大用因为参赞军务有功,被授游击官职回来,这时他才和唐氏成了亲。过了几天,金大用带上唐氏去南京,准备去给庚娘扫墓。刚过镇江,要登金山。船到江心,忽然有一条小船过来。船中有一老妈妈和一个少妇,金大用惊疑那少妇很像庚娘。小船疾驶而过时,那少妇从窗中窥看金大用,神情更像庚娘。金大用惊疑又不敢追问,急忙呼叫说:“看那鸭子飞上天去了!”少妇听了也呼喊说:“馋狗想吃猫腥吗!”这是当年闺房内夫妻俩开玩笑的话。金大用大惊,回船追近仔细一看,真是庚娘。丫头扶庚娘到这边船上,两人相抱大哭,同船的人也跟着伤感不已。唐氏以嫡妻礼拜见庚娘,庚娘惊奇地询问,金大用才仔细地述说了缘由。庚娘拉着唐氏的手说:“同船时一席话,心中常常忘不了,想不到成了一家人。多亏你代我葬了公婆,我应当首先谢你,哪能以这种礼节相见呢?”于是以年龄论,唐氏小庚娘一岁,二人便以姐妹相称。

原来,庚娘被埋葬以后,自己不知道过了多长时间,忽然听见一人喊她说:“庚娘,你丈夫没死,还应当重新团圆。”接着就如同从梦中醒来,用手摸摸四面全是墙壁,这才醒悟自己是被埋葬了。只觉得闷得慌,也没有什么痛苦。有几个恶少发现庚娘的陪葬物丰富,便挖坟破棺,正要搜括,见庚娘仍然活着,双方都既惊又怕。庚娘害怕他们害自己,哀求说:“幸亏你们来,才使我又见天日。头上的首饰,你们全都拿去,请你们把我卖到庵里当尼姑,也可以得几个钱,我不会把这事告诉别人。”盗墓的磕头说: “娘子是贞烈女子,神人都敬佩。小人们不过是贫困没有办法,才干这见不得人的事。只要你不说,我们便感恩了,怎么敢卖你为尼呢?”庚娘说:“这是我自己愿意的事。”另一盗墓的说:“镇江有个耿夫人,一人守寡没有子女,如果见到娘子一定会很高兴。”庚娘谢过他们,自己摘下珠宝首饰,全都给了他们。盗墓人不敢收,庚娘再三给他们,才拜谢收下来。接着雇了车船,把庚娘送到了耿夫人家,假说是乘船遇风迷路。耿夫人是个大户,守寡一人过日子,见了庚娘非常喜欢,把庚娘当作亲生女儿。刚才是母子二人从金山回来。庚娘把自己的经历讲述了一遍,金大用就过船去拜见耿夫人。耿夫人像对亲女婿一样款待他,邀金大用到家中,留住了好几天才走。从此两家来往不断。


\subsection{1.3.29   宫 梦 弼}
\label{\detokenize{p00_u5176_u5b83/_u767d_u8bdd_u804a_u658b_u5fd7_u5f02:id115}}
柳芳华是河北保定人。家中财产,在乡里数第一。他为人慷慨好客,家中常有百十客人。他常急人之所急,为朋友解救困难,往往千金不惜。朋友们向他借钱,也很少有归还的。唯有一个宾客名宫梦弼,是陕西人,从来没提出过什么请求。但他每次来到柳芳华家,一住就是一年。这人性格潇洒,谈吐文雅,柳芳华和他相处的时间最多。

柳芳华有个儿子,叫柳和,当时年纪很小,称宫梦弼为叔叔。宫梦弼也很喜欢与这孩子一起玩。每逢柳和自私塾回来,他们就揭开地上铺的砖,把石子埋进去,假装埋金子以为游戏,家中的五所房子,几乎全都埋遍了。众人都笑宫梦弼像孩子一样的稚气,唯独柳和喜欢他,和他亲近。

过了十多年,柳家的财产慢慢地用空了,供不起这众多食客朋友的需求,于是客人们逐渐地离去。然而在柳家,十余人的宴会,通宵达旦,还是常有的事。柳芳华到了晚年,家境越来越难以支持,只好出卖土地得几个钱,以备饭菜招待客人。柳和也挥霍,学着他父亲结交小朋友,柳芳华看到也不禁止他。

不久,柳芳华病死了,家里穷得连买棺材的钱都没了。宫梦弼从自己的腰包里拿出钱来,为柳芳华办理了丧事。柳和更加感激宫的恩德,家中无论事情大小,都委托给他。宫梦弼每次从外边回来,袖子里必定带些碎瓦片,进了屋,就扔到阴暗的屋边角落里,别人更不理解他的用意是什么。柳和经常与宫梦弼谈起家中的贫苦,宫梦弼听了对他说:“孩子,你现在还不知道真正受苦的滋味!不要说没有钱,就是给你一千两金子,你也会马上花光的。男子汉所愁的是不能自立,愁什么贫穷?”

一天,宫梦弼告辞回家,柳和流着眼泪,嘱咐他早些回来,宫梦弼答应了就去了。柳和家逐渐穷得不能自给,家里的东西也卖完了,天天盼望着宫梦弼回来,替他料理一下家事。但宫梦弼一走,毫无音信。

从前,柳芳华在世的时候,为柳和结亲于无极县黄氏,也是一个大户人家。后来,黄氏听说柳家如今一贫如洗,暗地里就有悔婚的念头。柳芳华去世,给黄家送去讣告,黄家也没来吊唁;而柳家只认为是路远,就原谅了他。柳和守孝三年期满,母亲就让他自己到黄家订下完婚的日期,希望得到黄家的同情与照顾。及到了黄家,他的岳父听说柳和穿着破衣烂衫,鞋子有了洞,就告诉门人,不要放柳和进来,并让门人转告他说:“回去筹划一百两银子,可以再来;不然的话,就从此断绝这门亲事。”柳和听了这话,痛哭流涕。黄家对门的一位刘老妈妈看了很可怜他,就留他在自己家里吃饭,送了他三百个铜钱,劝慰着让他回去。

柳和回到家后,母亲很气愤,但也没有别的法子。她想起过去交往的宾客中,十个里有八九个借过他们家的钱,都没有归还,就想让柳和去找几家富裕的人家,向他们求助。柳和说:“过去和我们交好的人,都是为了我家的钱财,假若儿子乘坐驷马的高车,去借一千金也不难;眼下,穷到这样子,谁还去想过去待他的好处?而且父亲当初借钱给人的时候,也从没立过字据或找过中间保人,去讨债也没有凭据啊!”母亲坚持一定让他去,柳和只好去试试。结果,讨了二十多天,一文钱也没讨到;只有演戏为生的李四,从前受过柳家的恩恤,听到他们眼下的情况,赠送他一两银子。母子两人大哭一场,从此也就绝了讨债的念头。

黄家的女儿已经到了出嫁的年龄,听说父亲拒绝了柳和的婚事,心里很不以为然。父亲要把她改嫁给别人,女儿哭着说:“柳郎并不是天生就的穷命。假若他现在比以前还富几倍,谁又能把他从我们手中夺去?现在因为他穷了,就抛弃了他,是不仁义的。”黄老头子心里很不愉快,婉言劝解训导,女儿终不改变自己的主意。黄老头子与老婆子都生气了,一天到晚责骂女儿,女儿也安然不放在心上。

不久,黄家夜间遭到强盗的抢劫,夫妇两人被炮烙得几乎死去,家中的财产也被抢得荡然一空。时间慢慢地又过了三年,黄家家境更加零落下来。有一位西方来的商人,听说黄家的女儿很漂亮,愿意出五十两银的聘礼。黄氏贪图这笔钱财,就答应了,想强迫女儿嫁给他。

女儿得知他们的阴谋,就毁坏了衣裳,涂抹了丽孔,乘着黑夜逃出家门,沿途乞讨。经过两个月,到了保定。她打听到柳和家的住址,就直接到了柳和家。柳和的母亲以为她是讨饭的人,就大声呵叱她。女儿哭着说明了自己的身份。柳和的母亲拉着她的手,流着泪说:“孩子,你怎么这副样子?”女儿又凄惨地告诉了所以这样的原因。讲罢,母女两人大哭。接着就给她盥洗沐浴,那娇秀的面容,眉宇间的神采,焕然一新。柳和与他母亲都很高兴。然而,一家三口人,一天只能吃一顿饭。母亲流着泪说:“我母子二人本应如此,可怜的是你这贤德的媳妇,也跟着我们受苦。”媳妇笑着安慰她说:“媳妇沿途讨过饭,很知道讨饭人的境况和滋味,现在回过头去看看,已经觉得有天堂与地狱的区别了。”柳和的母亲听了这话,也就笑了。

一天,媳妇走进一间空闲的房子,地上杂草丛生,几乎无插脚之地。她慢慢走进内间,只见里面积满了灰尘,在黑暗的偏屋角堆积着东西,用脚踢一踢,硬硬的抬起一看,全是银子。她惊喜地告诉柳和,柳和同她去一看,就是宫梦弼原先抛的碎瓦砾,现在都变成了银子。柳和因而想起,孩童时与宫叔叔在屋里埋的石子,是否都是银子呢?可是,那屋子现已典给别人,他急忙赎了回来。在断砖残缺处所埋的石子都明显地露出来,很觉失望。挖开别的砖一看,光灿灿的银子都摆在那里。转眼间,柳和就成了百万富翁。从此,柳和赎回自己的田产,购买了奴仆,门庭的繁华,超过了往日。因而自己发奋说:“我若不能自立,就辜负了宫叔叔的期望。”于是严格刻苦要求自己,苦读三年,考中举人。他就带着银子,到无极县感谢刘老太太。

柳和穿着鲜艳华丽的衣服,光彩夺目,跟从着健仆十余人,各自骑着膘壮的马。刘老太太只有一间狭窄的屋子,柳和就坐在床上。人马喧腾,充满了狭小的巷子。黄老头自女儿逃走后,那个商人就逼着他退还聘礼,可是那五十两银子已经用去了一半,他只好卖掉了屋子,偿还债务,所以穷困潦倒像柳和当年一样。听到过去的女婿很显赫,只有闭门叹气。刘老太太买酒备肴款待柳和,顺便说起黄氏之女很贤惠,并且惋惜她现己逃走。又问柳和娶了妻子没有?柳和说:“娶了。”吃罢饭,他定要刘老太太到自己家看看新娘,便用车子载着一同回去。到了柳家,黄女穿戴着华服盛装,出来迎接,侍女们前后簇拥着,活像一位天仙。见面后,刘老太太大吃一惊,相互叙谈了往事,黄氏女询问了父母的情况。一连数日,主人热情款待刘老太太,并给她做了好的衣服,上下一新,才让人把她送回家。

刘老太太到家后,就到黄家报告他女儿的消息,并转达了他女儿的问候。黄氏夫妇大吃一惊。刘老太太劝他们去投靠女儿,他们很觉难为情。由于家益败落,冻饿难忍,不得已才到保定。到了女婿门前,只见门楼高耸,很有气势。守门的人瞪着眼睛看着他,整整一天也不给他通报。后来,看到一位妇人从里面走出来,黄老头陪着笑脸,用谦卑的语言,说明了自己的姓名,请她偷偷地告诉女儿。妇人一会儿出来,把他引到一间耳房里,说:“娘子很想拜见您,但又怕郎君知道,请您稍候,等待机会。你老人家什么时候来到此地?是否有点饥饿?”黄老头说明自己的苦楚。妇人送来一壶酒,两盘菜,放在桌上。又赠给五两银子,说:“郎君正在房中请客,娘子恐怕来不了。明天早晨你应当早早离开这里,不要让郎君得知风声。”黄老头点头称是。

第二天早晨,他早起打点行李准备出去,可是,大门上的锁还未开,他只好在大门洞中,坐在行李上等待开门。忽然听到有人喧哗,说主人出来了。黄老头急急收拾行装,准备回避,可是已经来不及了。柳和看到他,责问他是什么人?家人都没法回答。柳和生气地说:“这一定是个坏人,把他捆起来,押送到衙门去审办。”众仆从一涌而上,把他用一根绠绳捆到树上。黄老头惭愧畏惧,不知如何说才好。过了一会儿,昨晚那位妇人出来,双膝跪在柳和面前说:“他是我的舅舅,昨天来得很晚,所以没来得及告诉主人。”柳和叫人给他解绳子,那妇人送黄出门,说:“昨天忘了嘱咐守门的人,致使造成今天这样的差错。娘子说,想念时,可以让老夫人扮装成卖花的人,和刘老太太同来。”黄老头答应了。回到家里,把这事告诉了黄老太太。黄老太太想念自己的女儿,如饥似渴,把心思告诉了刘老太太。刘老太太就按黄氏女儿说的办法,到了柳和的家。她们走过十几道门,才到了女儿的绣房。女儿身穿彩帔,头梳高髻;头戴珍珠翡翠,身着绫罗,满身散发着扑鼻的香气。只要小声一喊,大小丫鬟仆妇就围在身边,搬来金饰交椅,安放好一对夹膝,有聪慧的丫鬟来沏茶倒水。母女见面各自用暗语问寒问暖,相视泪水荧荧。晚间,打扫一间房子,安排两位老太太,铺盖的被褥,温暖而柔软,连当年富庶时都不曾有过。

住了三五天,女儿待母亲心意很恳切。母亲把女儿引到无人之处,哭泣着说明以前的过错。女儿说:“我们母女间,有什么忘不了的过错?但柳郎气愤没有消除,是提防他知道。”每当柳和来时,黄老太太便躲开。一天,她们母女刚促膝谈话,柳和突然进来,看到这种情景,生气地说:“哪来的村妇?敢大胆和娘子靠在一起坐着,应该叫人把你的鬓毛都薅干净。”刘老太太急向前解释说:“这是我的亲戚,卖花的王大嫂,希望你不要责怪。”柳和让刘老太太坐上首谢了罪。接着他就坐下来,说:“姥姥来了好几天了,我只是忙,未能坐下来与您拉拉家常。黄家那老畜生,现在还活着?”刘老太太笑着说:“都好。只是穷日子难过。官人现在富贵了,为什么不思念翁婿间的情谊?”柳和拍着桌子说:“想当初,不是姥姥您可怜我,送我一碗粥,我怎么能活着回乡!现在,我恨不得吃他的肉,剥他的皮,有何可想念的啊!”说到气愤时,竟跺脚骂起来。黄氏女忿恚地说:“他们做得不对,也是我的生身父母啊!我千里迢迢来投你,手都裂了口子,脚趾也都磨穿了。我自己想,没有辜负郎君的地方,怎么能对子骂父,使人不堪忍受!”柳和才收敛怒容,走了。

黄老太太感到很惭愧,心中也很懊丧,面无血色。辞别女儿,要回家去。女儿偷偷交给她二十两银子。回到家后,再也没有听到音信。黄氏女很想念他们,柳和就派人把黄氏夫妇接来。老夫妻到柳家,羞愧得无地自容。柳和道歉说:“去年你来到我家,又不明自告诉我,使我冒犯得罪的地方很多。”黄老头子只是唯唯地应付。柳和为他们更换了衣服和帽子,留在家里。住了一个多月,黄老头子心里总是不踏实,告辞回家。临走时,柳和赠送他一百两银子,说:“那个西方商人给你五十两,我今天给你加倍。”黄老头子满脸羞惭,接过银子。柳和用车马把他们送回去。到了晚年,他家也成了小康人家。


\subsection{1.3.30   鸲 鹆}
\label{\detokenize{p00_u5176_u5b83/_u767d_u8bdd_u804a_u658b_u5fd7_u5f02:id116}}
王汾滨讲过这样一个故事:他的家乡有个养八哥的人,教八哥说话,八哥学得很熟练。出门游玩总把八哥带在身边,相伴已经有好几年了。

一天,这个人将要经过山西绛州的时候,盘费已经用光了。这人很愁闷,没有办法。八哥说:“为什么不把我卖掉?送我到王府去,你会得到好价钱,就不用愁回去无路费啦。”这人说: “我怎么忍心呢?”八哥说:“不妨!你得到钱就赶快走,在城西二十里地的大树下等我。”这人听从了八哥的话。带着八哥到城里,与八哥相互问答对话,围着看的人越来越多。有个太监见到,告诉了王爷。王爷把这人召入王府,想买这只八哥。这人说:“小人与八哥相依为命,不愿意把它卖掉。”王爷问八哥说:“你愿意住下吗?”八哥说:“愿意。”王爷很喜欢。八哥又说:“给十两银子,不要多给。”王爷更加喜欢,立刻给了十两银子。这人故意作出很懊悔的样子走了。

王爷与八哥对话,八哥对答如流。八哥喊着要吃肉,吃完,八哥说:“臣要洗澡。”王爷命府人用金盆盛上水,打开笼子叫它洗。洗完后,八哥飞到屋檐间,梳理着羽毛,还和王爷喋喋不休地说着话。一会儿,羽毛干了,便轻捷地飞起来,操着山西口音说:“臣去了!”王爷左顾右盼间,八哥已飞得无影无踪了。王爷和府上的侍从们,只是仰天叹息。急忙去找那卖八哥的人,这人也早已渺无踪影了。

后来,有到陕西的人,见那养八哥的人带着那只八哥在西安市上闲逛。这个故事是毕载积先生记下来的。


\subsection{1.3.31   刘 海 石}
\label{\detokenize{p00_u5176_u5b83/_u767d_u8bdd_u804a_u658b_u5fd7_u5f02:id117}}
刘海石是蒲台人,十四岁时,随家人到滨州躲避战乱,与滨州书生刘沧客同拜一个老师学习,两人关系很好,便结拜为兄弟。没过多久,刘海石父母双亡,奉丧回了原籍,此后一直杳无音信。

刘沧客家境富裕,四十岁生了两个儿子。长子刘吉,十七岁了,是县里的名士。次子也很聪明伶俐。后来,刘沧客又娶了本县倪家的姑娘为妾,对她非常宠爱。过了半年,长子患头痛病去世,夫妻大为悲伤。不久刘沧客的妻子又病故;过了数月,大儿媳也死了,家中的奴婢佣人也一个接一个地去世。刘沧客接二连三屡遭不幸,几乎不能忍受。

一天,他正在独自闷坐,忽然看门人进来禀告:刘海石来了。沧客很高兴,急忙出门恭迎入坐。刚要问候寒暄,刘海石忽然吃惊地望着他说:“老兄,你有灭门之祸,不知道吗?”刘沧客目瞪口呆,不明白是怎么回事。刘海石又说:“很久不通音信,我估计你近来的状况就未必很好!”刘沧客听后,忍不住掉下泪来,就将家中近来发生的灾难,如实相告。刘海石也难过得落了泪,既而又转悲为喜,笑着说:“这场灾难还没完,所以我先是为你悲伤;但幸亏遇到我,又该为你庆贺。”刘沧客说:“久不见面,难道你精通了给人看病的越人术吗?”刘海石回答说:“这不是我的专长。看看宅子风水、或给人相相面我到是比较在行。”刘沧客很欢喜,便求他相看住宅。刘海石里里外外察看了一遍后,又要求再看看家中所有成员。

刘沧客按他的吩咐,把全家人都集合到堂屋,挨次一一地指给刘海石。轮到倪氏时,刘海石忽然仰天大笑不止。众人正惊奇时,就见倪女吓得浑身打颤,面无人色;整个身体骤然缩短到二尺多长。刘海石用界尺敲敲倪女的头顶,发出一种像敲石罐的声音。他又上前揪住倪女的头发,仔细检查她的脑后,见有几根白毛,伸手就要拔去。倪女缩着脖子,跪在地上哭着说马上就走,求他不要拔了。刘海石怒斥道:“你还想害人吗?”硬将白毛拔去了。那女子随即变成了一只黑色像狸一样的动物。众人都异常惊惧。刘海石把那动物抓来放到袖子里,看着刘沧客的儿媳说:“她受毒很深,背上肯定有异样的变化,请让我检查一下。”媳妇害羞,不肯脱衣服。刘的儿子执意让她脱下,见她背上长着白毛,有四指多长。刘海石用针给她挑出,说:“这毛已老,再过七天就没救了。”刘海石又看沧客的儿子,见背上也长着二指多长的白毛,便说: “这些毛若再长一个多月,你也没命了。”他又逐个察看了刘沧客及家人,一一挑去了白毛,对众人说:“我若不及时赶来,全家人没有再活的了!”有人问:“这是个什么东西?”刘海石回答:“也属狐类,专靠吸取人的精气为生,最能置人于死地。”刘沧客说:“好久不见,你怎么能这样料事如神,莫非是神仙吗?”刘海石笑着说:“我这不过是跟师傅学到的一点雕虫小技罢了,怎敢称神仙呢?”沧客问谁是他师傅,刘海石说:“山石道人。剐才这东西,我还治不死它,要献给师傅,让他处置。”说完就告别要走,一抬手觉得袖子空空的,惊骇地说:“跑了!尾巴上还有大毛没拔去,竟然逃了。”众人骇然。刘海石忙安慰说:“他脖子上的毛已拔了,不能再变成人,只能化成兽类,不会跑远。”说着就进屋看看猫,又出门看看狗,都说不是。打开猪圈门看时,刘海石笑着说:“在这里呀!”刘沧客过去一看,果然多了一头猪。那猪听到刘海石的笑声,立时伏在地上一动也不动。刘海石提着耳朵把它抓出来,见尾巴上果然有一根白毛,坚硬如针。才要拔掉,那猪翻转哀鸣不让拨。刘海石气愤地说:“你作孽这么多,还想一毛不拔吗?”边说边强行拔掉,那猪随手又化为狸。刘海石将它收到袖中要走,刘沧客苦苦挽留,才在一起吃了顿饭。临行前,刘沧客问他什么时候再见,刘海石说:“这事难以预定。我师傅立下宏图大志,常派我们邀游世上,搭救众生,以后未必没有见面的机会。”

分别后,刘沧客细想刘海石师傅的名字,才恍然大悟地说:“海石大概已变成仙人了!‘山石’合起来是‘岩’字,正是仙人吕洞宾的名字。”


\subsection{1.3.32   谕 鬼}
\label{\detokenize{p00_u5176_u5b83/_u767d_u8bdd_u804a_u658b_u5fd7_u5f02:id118}}
尚书石茂华是青州人。他还是秀才时,郡城门外有个大水湾,即使久旱不下雨,湾里的水也不干涸。一次,官府将捕获的几十名大盗在水塘边上杀了。这些鬼聚众为害,凡是从塘边经过的行人常被拖入水中。

一天,某甲正遭众鬼围困,忽然众鬼惊散逃窜,说:“石尚书来了!”不多时,石公果然来到,某甲向他讲述了刚才的事。石公听完,便用石灰粉在墙壁上写道:“石某为禁止鬼害特告:察得你们素来无善良之心,才招致上天雷霆之怒;图谋不轨,方导致刀斧加颈。只应收起害人的心肠,争相悔过,或许能洗去你们骨头上的污血,脱离苦海。你们生前已受极刑,死后竟仍聚集作恶。跳来跳去,披发成群;徘徊前后,一味害人。用黄泥塞住耳朵,常逞阴鬼之凶;大白天兴妖作怪,断了行人之路!那丘陵三尺外的地方,还由人管辖;岂能偌大天下,任你们胡作非为?见此告示后,你们都应消声匿迹,不要继续作恶。无定河边的尸骨,静待投生轮回;金闺梦里的鬼魂,愿早日返回故土。如重蹈前辙,不思悔改,必定后悔!”

从此,鬼害绝迹,水湾也随即干涸了。


\subsection{1.3.33   泥 鬼}
\label{\detokenize{p00_u5176_u5b83/_u767d_u8bdd_u804a_u658b_u5fd7_u5f02:id119}}
我家乡的唐济武太史,几岁时,有个表亲带他到寺院玩耍。太史童年胆子很大,见廊中的泥鬼,怒目圆睁,琉璃眼球闪闪发光,非常喜欢,便偷偷地挖出琉璃眼球,藏到怀里回了家。

到家后,那位表亲突然得病,不能说话。过了一会儿,他忽然站起来厉声说:“为什么挖去我的眼睛?”叫嚷不休。众人都不知是怎么回事,太史才讲了他挖眼睛的事。家中人听后赶快祷告说:“孩子年幼无知,伤害了您尊贵的眼睛,我们马上就去奉还给你。”话音刚落,那位表亲便大声说:“这样,我该走了。”说完就仆倒在地,昏了过去。过了很久,他才慢慢苏醒过来。问他刚才说了些什么,他茫然不知。于是家人连忙将琉璃眼球送回寺院,安到泥鬼的眼眶中。


\subsection{1.3.34   梦 别}
\label{\detokenize{p00_u5176_u5b83/_u767d_u8bdd_u804a_u658b_u5fd7_u5f02:id120}}
李王春先生的祖父,与我已故的叔祖玉田公,交情最好。一天夜里,李的祖父梦见玉田公到他家里,神色凄伤地和他说话。李祖问:“你来有什么事吗?”他回答说: “我要出远门,特来向你告别。”又问:“到哪里去?”回答说:“很远啊。”说完转身走了。李祖把他送到山谷中,见石壁有条裂缝,玉田公便拱手告别,然后背对石缝慢慢倒退着隐入石缝中,喊他也不应声。李祖一下从梦中惊醒了,挨到天亮,将梦中的事告诉了太公敬一,并让家人准备好吊丧的物品,说:“玉田公已经去世了。”太公叫派人先探听一下,如果是真的,再去吊唁。李祖不听,竟换上素服去了。到了玉田公的门前,果然见门上已挂着白幡了。

唉!古人对朋友,连生死都深信不疑;古人张元伯的灵车要等范巨卿来到,才肯前行,岂是荒诞之谈?


\subsection{1.3.35   犬 灯}
\label{\detokenize{p00_u5176_u5b83/_u767d_u8bdd_u804a_u658b_u5fd7_u5f02:id121}}
光禄寺署丞韩大千的仆人,一夜,在前厅睡觉,见楼上有盏灯,像明亮的星星。不一会儿,那灯像萤火一样,轻飘飘地落到地上,变成了一只狗。仆人斜眼一看,见狗转到屋后去了。他急忙起来,尾随在它后面。到了后园,那狗又变成了一个女子。仆人心中知道它是个狐精,仍旧回来躺下。忽然女子从后面走过来,仆人假装睡着了,看她要干什么。女子俯在仆人身上摇晃他,仆人装做刚醒来的样子,问她是谁。女子不回答。仆人说:“楼上的灯光,莫非就是你吗?”女子说:“既然知道,何必问呢?”于是两人一起睡了。从此女子白天走晚上来,习以为常。

主人知道这件事后,就叫两个仆人一边一个把这个仆人夹在当中睡觉。但当二人醒来时,却都睡在床下,也不知是什么时候掉下来的。主人越发生气,对这个仆人说:“她再来时,你就把她捉来;不然,你当心挨鞭子!”仆人没敢言语,答应着退下,心中捉摸:捉她很难,不捉吧,少不了挨打。想来想去束手无策。忽然想起,女子睡觉时,贴身穿着一件小红衫,从来没脱下过,一定是她的要害,拿到这件东西就可以要挟她。

到了夜里,女子来了,问仆人:“主人叫你捉我了吗?”仆人说:“是有这回事。但咱俩感情这么好,我怎能这样办?”睡下后,仆人偷偷抓她的小红衫,女子急叫一声,使劲挣脱跑了。从此没有再来。

后来,仆人从外地回来的路上,远远看见女子坐在路边。他走到近前,女子就用袖子挡住面孔。仆人下马喊道:“何必这样作态?”女子便站起来,握住他的手说:“我以为你已经忘了咱们的旧情呢!既然你还恋着没忘,还可原谅你。以前的事是因为主人吩咐,我也不怪你了。但你我的缘分已尽,今天我准备了点小菜,请到我家小饮,算是告别吧。” 这时正是初秋,高粱长得正茂盛。女子上前拉着仆人走进高粱地,一进去,见里面有座宽敞的大宅子。仆人把马拴好,见厅堂里已经摆好了酒肴。刚坐下,一群丫鬟就忙着给斟酒。太阳快下山时,仆人因有事,要回去禀报主人,便起身告辞。出来一看,仍是一片高粱地。


\subsection{1.3.36   番 僧}
\label{\detokenize{p00_u5176_u5b83/_u767d_u8bdd_u804a_u658b_u5fd7_u5f02:id122}}
体空和尚说:“在青州,曾见两个外国和尚,相貌长得很古怪;耳朵上戴着双环,披着黄布,长着卷曲的头发和胡须。自己说是从西域来的,听说青州府的太守很敬佛,特来拜见。太守派了两个差役送他们到和尚住处。有个叫灵辔的和尚,对他们不怎么礼貌;但管事的见他俩不同寻常,就自己设宴款待他们,并留他们住下。

有人问他俩:‘西方有很多能人,师傅您是否也有奇妙的法术?’其中一个西域和尚笑了笑,从袖中伸出手来,掌中托着一个小塔,高不过一尺,玲珑可爱。这房子墙壁上最高处,有个小龛,这和尚顺手一扔,小塔就稳稳当当地落在小龛的正中间。小塔上还有舍利子放着光芒,照耀满屋。稍过一会,和尚又抬手招塔,塔仍落在他的掌中。另一个西域和尚露着臂膀,一伸左臂,延长达六七尺,而右臂就缩得不见了;再伸右臂,也与刚才伸左臂一样。”


\subsection{1.3.37   狐 妾}
\label{\detokenize{p00_u5176_u5b83/_u767d_u8bdd_u804a_u658b_u5fd7_u5f02:id123}}
山东莱芜的刘洞九,在山西汾州做官。一天,他独自坐在府中,听到庭院有说笑声越来越近。他抬头一看,进来四个女子,一个四十来岁,一个年约三十,另一个约二十四五岁,最后是个没有束发的少女。她们都站到桌前,相视而笑。刘公早知官府里有很多狐,因此就不理她们。过了一会,少女拿出一条红纱巾,开玩笑般地扔到刘公的脸上。刘公拾起来扔到窗下,仍不答理她们。四个女子一笑走了。

一天,年纪最大的那个女子来到刘公的房中,对他说:“我妹妹与您有缘分,请不要嫌弃她。”刘公随便答应了一声,女子就走了。转眼工夫,她领着一个丫鬟拥着少女走来,让少女与刘公并肩坐下,说:“你俩真是一对好伴侣,今夜就成亲吧。好好侍候刘郎,我走了。”刘公仔细端详那少女,艳丽无比,便与她欢好了。又问女子的来历,少女说: “我不是人,可实际是人。我是前任州官的女儿,因被狐迷惑,受害而死,埋葬在园子里。众狐用法术救活了我,所以我就飘然像狐。”刘公听后,就用手摸摸她的后身。女子察觉了,笑着说:“你莫不是以为我有尾巴吧?”转过身去说:“请摸吧!”从此后,少女就住下不走了,一举一动都和那个小丫鬟在一起。家中人都以小夫人之礼对待她。丫鬟婆子们来拜,她都给很多赏赐。

一次刘洞九过生日,前来祝寿的人很多,共三十多桌宴席,需雇好多厨师,但事先约定的厨师才来了一两个。刘公很生气。女子知道后,对刘公说:“不用愁!厨子既然不够用,不如连来的两个也打发走。我虽然本事不大,但办三十多桌席并不难。”刘公听后转忧为喜,忙派人将鱼肉蔬菜调料等物品都搬到内院。家里人只听见里边刀案炒勺声叮当响,不绝于耳。门内放一张桌子,端菜仆人将托盘放在上面,转眼间,菜肴已盛满。十几个仆人来去不停,仍取之不尽。最后,仆人来要汤饼,只听里边女子说:“主人事先没要汤饼,急切之间怎能立即拿出来?”接着又说:“不要紧,先借借!”不大工夫,女子就喊仆人来取汤饼。众人一看,三十多碗汤饼热气腾腾地摆在桌上。客人走后,女子对刘公说:“拿出钱来,偿还某家的汤饼钱。”刘公忙派人将汤饼钱送去。那家失了汤饼,正在奇怪时,送钱的人到了,这才解开疑团。

一天晚上,刘公在喝酒,一阵想起来要喝家乡的苦醁酒。女子就说她去取,随即出门走了。不一会就回来说:“门外有一坛够你喝几天的。”刘公出门一看,果然有一坛酒,真是家乡的“瓮头春”。

过了几天,刘公的夫人派了两个仆人来汾州。路上,一个仆人说:“听说狐夫人赏钱很多,这一回去得了赏钱,可买件皮衣穿。”女子在汾州官署中已知道了这话,便对刘公说:“家中派来的人快到了。可恨这个奴才无礼,我一定要惩治他一下。”到了明天,那两个仆人刚进城,突然一个头痛起来。到了州衙,痛得抱头大叫。众人要给他服药,刘公笑着说:“不用治疗,到时候自然会好。”大家都猜疑是得罪了小夫人。那仆人暗想:我刚来还没放下东西,哪里来的罪呀?无处诉说,只好跪下求饶。只听到帘子里面有人说:“你称夫人就叫夫人罢了,为什么还加上‘狐’字呢?”仆人这才恍然大悟,再三叩头谢罪。又听里面说:“既然想要皮衣,怎么能无礼呢?”接着又说:“你的头痛好了!”话音刚落,那仆人的头立刻不痛了,他连忙拜谢要走,忽见从帘内扔出一个包裹来,里面说:“这是件羊羔皮衣,你可拿去。”仆人解开一看,包里是五两银子。刘公问起家里的情况,仆人回说家里一切平安,只是某日少了一坛藏酒。计算一下丢失的日期,正是女子取酒的那天晚上。大家都俱怕小夫人的神力,称她为“圣仙”。刘公还为她画了一幅肖像。

当时,张道一为汾州的提学使,听说这些怪事。便以老乡的名义去拜见刘洞九,并要求见小夫人一面。女子拒而不见。刘公就拿出她的画像让张看,张强拿着就走了。张回府后,将画像挂起来,天天对着祈祷说:“以你的天姿和气质,跟谁不行?偏要跟一个白发老头子!我哪一点比刘洞九差,为什么不来见我一面呢?”女子在州府里,忽然对刘公说: “张公对我无礼,得稍给他点惩罚!”一天,张正对像祈祷时,觉着像有人用戒尺打了一下他的前额,头痛得像要裂开一样,心中异常恐惧,忙派人将画像送还。刘公故意询问原因,来人隐瞒实情不说真话。刘公笑着说:“你主人的额头没痛吗?”来人见瞒不过去,只好说了实话。

没过多长时间,刘公的女婿亓生来,要见小夫人。女子推辞不见。亓生一再求见,刘公就对女子说:“女婿又不是外人,怎么就一定不见他呢?”女子回答:“女婿来见我,必定得赠送他东西。但他的心愿太高,我估计不能满足他,所以才不愿见他。”后来,女婿非见不可,才允许等十天以后相见。到了约定的日期。亓生进屋,隔着帘子施了礼,稍问候一下。只见小夫人的相貌隐隐约约,他不敢仔细看,就告退出来;走出数步之后,忍不住回头看看。只听女子说:“女婿回头了。”说完大笑不止,声音像猫头鹰叫一样。亓生听了,吓得腿都软了,摇摇晃晃地像丢了魂似的。出门后,坐了好久,才稍定下心来,说:“刚才听到笑声,就如霹雳震耳,竟不觉得身子是自己的了。”不一会,一个丫鬟奉女子的命,赠给亓生二十两银子。亓生收下后,对丫鬟说:“圣仙与岳父大人住在一起,难道不知我向来挥霍成性,不习惯花小钱吗?” 女子听到这话说:“我早知道他是这种人!上次钱袋空了,我与同伴去开封,正好城被水淹没,仓库藏的银子都淹在水中。我们各自打捞了一点点银子,怎能满足他贪得无厌的要求呢?况且我就是能多给他些,他福分太薄也担当不起。”

女子凡事都能预先知道,刘公每碰到疑难问题,总和她商议,都能解决。一天她正与刘公并坐,忽然仰面观天大惊地说:“大难临头了,怎么办呢!”刘公吃惊地问家人吉凶,女子说:“别人都没事,只是二公子令人担忧。此处不久将成为战场,你应当请求一个差事到远方去,才能免遭灾难。”刘公听从了她的建议,请求上司准许他押粮饷去云南贵州一带。此行路途遥远,别人听说后,对他表示担心,唯独女子表示祝贺。不久,姜瓖叛变,汾州被贼寇占据。刘公的次子从山东来,正赶上这个变故,被杀害。汾州城沦陷后,大大小小的官员全部遇难。唯有刘公,因出差在外得以幸免。平息叛乱后,刘公才回来。后来他被一场大案牵连受到处分,穷得吃不上饭。当权者又多方敲诈勒索,因此刘公就想一死了之。女子劝他说:“不要犯愁,床下还有三千两银子,可以用来过日子。”刘公高兴地问:“你从哪里偷来的?”女子回答说:“天下无主的东西取之不尽,还用得着偷吗?”刘公倚仗女子的计谋,才回了原籍,女子也跟着去了。几年后,女子忽然离去,留下了个纸包,包着几样东西。其中有出丧时挂在门上的小幡,约有二寸长,大家都以为是不祥之兆。果然,不久刘洞九就病故了。


\subsection{1.3.38   雷 曹}
\label{\detokenize{p00_u5176_u5b83/_u767d_u8bdd_u804a_u658b_u5fd7_u5f02:id124}}
乐云鹤、夏平子二人,小的时候是同乡,大了又是同学,他们是莫逆之交。夏平子自幼聪明,十岁时就有文名。乐云鹤虚心向他学习,夏平子也认真地帮助他;乐文思日见长进,终于和夏齐名。但二人科考不得志,总是名落孙山。不久,夏平子染上疫病死去,家里穷得无力下葬,乐云鹤独力承担了丧事。夏平子撇下了寡妻和还在怀抱中的孩子,乐云鹤按时接济她们。每得到一升半斗粮食,必定分一半给夏家。夏平子的家属因此得以生存下去。士大夫和文人们也因此更加敬重乐云鹤。

乐云鹤家产本来不多,又常常周济夏家,因此生活日渐困难,他叹息道:“像夏平子那样的文才,都碌碌无为地死了,何况我呢!人生不能及时争取富贵,年年忧愁,恐怕早于狗马填了沟壑,辜负了这一辈子。不如早点自作打算!”于是放弃读书,改做买卖。经营了半年,家境逐渐富裕起来。

一天,乐云鹤到金陵,住在客店里。见一人长得很高大,身上筋骨隆起,在饭桌座位旁徘徊犹豫,脸色黯淡,面带悲伤。乐云鹤问他:“想吃点东西吗?”那人也不说话。乐云鹤把自己的饭推过去,那人双手抓着,一眨眼就吃了个净光。乐云鹤又给他买了两个人的饭,一会儿又吃完了。乐云鹤便让店主人割来一只猪腿,堆上一大摞蒸饼。那人吃了几个人的饭才吃饱,道谢说:“三年了,没吃这么饱过。”乐云鹤说:“你是一个壮士,怎么如此漂泊潦倒呢?”那人回答说:“罪犯天条,不能说啊!”又问他的家乡,回答说:“我地上没屋,水上没船;早上住在乡村,夜晚睡在城市。”乐云鹤收拾行装,准备上路。那人跟在后面,恋恋不舍。乐云鹤向他告辞,那人说:“您有大难,我不忍忘记这顿饭的恩德。”乐云鹤很惊异,便带着他同行。路上拉他吃饭,那人推辞说:“我一年只吃几顿饭。”乐云鹤更加惊奇。

第二天,乐云鹤乘船渡江时,忽然狂风大作,波浪滔天,江上的商船全部倾覆,乐云鹤和那人都掉进江里。一会儿,风平浪静,那人背着乐云鹤,踏着波浪钻出水面,把乐云鹤送到一只客船上,自己又破浪游去。一会儿,拉来一只小船,扶乐云鹤上去,嘱咐他躺着等着;自己又跳进江中,两个胳膊夹着货物出水,扔在船上。然后又潜进江中,出入几次,捞出的货物摆满了小船。乐云鹤感激地说:“你救了我的命,我已很知足了,哪敢指望连货物都能保全呢?”检查货物钱财,一点也没丢失。乐云鹤更加高兴,惊异地认为他是神人。放开船要走时,那人告辞。乐云鹤苦苦挽留,才一块渡江。乐云鹤笑着说:“这一场大灾难,只丢失了一枚金簪。”要再入江寻找,乐云鹤正要劝阻,那人已跳进江中不见了。乐云鹤惊愕了很久,忽见那人含笑而出,把一枚金簪交给乐说:“侥幸不辱使命!”江上的人见了,无不惊骇诧异。

乐云鹤带着那人返回家乡,吃住都在一起。那人十几天才吃一顿饭,一顿饭吃得不计其数。一天,又说要告别,乐云鹤执意挽留。正好天阴了下来,要下雨,远处传来雷声。乐云鹤说: “云里头不知是什么样子?雷又是什么东西?如果能到天上看看,才能解开这个疑惑。”那人笑着说:“您想到云中游览游览吗?”过了一会儿,乐云鹤觉得非常困倦,伏在床上打瞌睡。猛然醒来,觉得身子摇摇晃晃,不像在床上。睁眼一看,自己已在云海中,四周全是棉絮般的云朵。乐云鹤惊讶地站起来,头晕得像在船上。用脚一踏,软软的不是地面。仰头看看星辰,就在眼前。于是怀疑是在做梦。仔细一看,星星都镶嵌在天上,就像莲子嵌在莲蓬上一样。大的像瓮,小点的像小瓦罐,最小的像钵盂。用手扳扳,大星星牢不可动;小星星活动,像能摘下来。乐云鹤便摘下一颗,藏在袖子里。拨开云层往下看看,云海茫茫,地面上的城市只有豆粒那样大小。乐云鹤惊愕地想:如果一失足掉下去,这条性命还用问吗?一会儿见两条龙蜿蜒矫健地驾着一辆车飞来。龙尾一甩,像鸣牛鞭。车上有个容器,好几丈粗,里面贮满了水。有几十个人,用家什从容器中舀水遍洒云间。忽然看见乐云鹤,都感到奇怪。乐云鹤仔细看了看,那个壮士也在这些人里面。那人告诉众人说: “这是我的朋友。”说着,拿过一个舀水的家什,让乐云鹤洒水。当时正好大旱,乐云鹤接过家什,拨开云朵,望着大约是故乡的地方,尽情地舀水倾洒。过了会儿,那人对乐云鹤说:“我本是雷曹,因为误了行雨,被罚到人间三年。现在期限已满,我们从此分别了。”于是拿过驾车的绳子,有一万尺长,堆在一边,让乐云鹤缀着绳子一头下去。乐云鹤害怕,那人笑着说:“没事。”乐云鹤按他说的,缀着绳子嗖嗖地往下落,瞬间便到了地面。一着,正好落在自己村外。绳子渐渐收回云中,看不见了。当时,天旱了很久,十里外的地方,仅下了一指雨;唯独乐云鹤的村里下得沟渠都满了。

乐云鹤回到家中,摸摸袖子里,摘下的那颗星还在。拿出来放到桌上,只见黑黝黝的像块石头。到了夜晚,星星一片光明,照亮了屋子。乐云鹤更加当作宝贝,珍重地收藏起来。每次来了知己客人,他才拿出来照着明喝酒。正眼注视这颗星,光芒刺目。一夜,乐云鹤的妻子面对着星星坐着梳头发,忽见星光渐渐缩小,像个萤火虫一样,在空中流动横飞。妻子正在诧异,星星已钻进她嘴里,吐也吐不出来,竟咽到肚子里去了。妻子惊愕地跑去告诉乐云鹤,乐也感到奇怪。睡下后,乐云鹤梦见夏平子对他说:“我是少微星。你对我的恩惠,我一直没有忘记。又承蒙你从天上把我带下来,我们算是有缘。现在我做你的子嗣,以报大德。”乐云鹤三十岁了,还没有儿子,得了这个梦非常高兴。后来,妻子果然怀孕了。等到生产时,光明满室,就像那颗星在桌上放着时一样。因此生下的儿子就取名叫“星儿”。星儿非常聪明,十二岁就考中了进士。


\subsection{1.3.39   赌 符}
\label{\detokenize{p00_u5176_u5b83/_u767d_u8bdd_u804a_u658b_u5fd7_u5f02:id125}}
有个韩道士,住在县城里的天齐庙。他会幻术,人们都称他仙人。先父和韩道士很要好,每次进城都去看望他。有一天,先父和我已故的叔父进城,准备去拜访韩道士,恰好在路上碰见了他。韩道士把钥匙交给他们二人说:“请你们先去开门坐一会儿,我马上就回去。”他们按道士说的来到庙里,开锁进门一看,韩道士已经坐在屋里了。这样的奇事真是太多了。

原先,我有个本家,嗜好赌博。经先父介绍,认识了韩道士。当时大佛寺来了一个和尚,专事赌博,而且赌注很大。我那个本家听后非常高兴,带上家里所有的钱去赌,却输了个干干净净。本家不甘心,典当了房子田产又去了,一夜间又输了个净光。那本家心情忧闷,路过天齐庙,顺便去访韩道士。韩道士见他神情惨淡,语无伦次,就问他怎么了,本家把输钱的事如实告诉了韩道士。韩道士笑着说:“经常赌博,哪有不输的道理!你如果能戒赌,我有办法让你把输掉的钱全部赢回来。”本家说:“如果能把输的钱赢回来,我就用铁杵把骰子砸碎。”韩道士用纸画了一道符,让他扎在腰里,嘱咐说:“只要赢回你输掉的钱就住手,千万不可贪得无厌。”又给了他一千文钱作本,约定赢钱后偿还。

本家非常高兴地去了。和尚看了他的钱,嫌太少,不屑与他赌。本家非赌不可,说只赌一次,和尚笑着答应了。本家把一千文钱一下全押上,孤注一掷了。和尚掷了骰子,没有胜负;本家却一投就赢了。和尚又押上两千钱为注,结果又输了。渐渐地和尚把赌注增到十几千。本家掷的本来是输点,一吆喝,却都变成了赢点。就这样很快就把以前输掉的钱全部赢了回来。他暗想,如果再赢几千就更好了。于是又赌起来,但手气越来越坏。本家觉得奇怪,起来看看腰带上,原来纸符已经没有了。他大吃一惊,立刻作罢,拿着赢回来的钱回到庙里。除偿还和尚那一千文钱外,细细计算,减去最后输掉的,正好和他原来输掉的钱一样多。本家向韩道士道歉,说是丢了纸符。韩道士笑着说:“符已在我这里,一再嘱咐你不要贪得无厌,而你不听我的话,所以我把纸符拿回来了。”


\subsection{1.3.40   阿 霞}
\label{\detokenize{p00_u5176_u5b83/_u767d_u8bdd_u804a_u658b_u5fd7_u5f02:id126}}
文登的景星,小时候就很有名。他与陈生住近邻,两家的书房仅隔一堵短墙。

一天黄昏,陈生路过一处荒凉的废墟,听到松林里传来女子的啼哭声。走近一看,见树的横枝上挂着一条带子,一个女子像要上吊。陈生问她怎么了,女子抹了一下眼泪对陈生说:“母亲出远门,把我托给了一个外姓哥哥照管。没想到他狼子野心,对我不怀好意。我一人孤单到这地步,还不如死了!”说完又哭起来。陈生急忙为她解下带子,劝她嫁人。女子怕没有可靠的人,陈生就请她暂时住在自己家里。女子同意了。

回到家中,陈生点上灯对着女子一看,见她十分美丽,喜出望外,要与她同寝。女子厉声抗拒,两人吵闹的声音传到隔壁。景生听到后,跳过墙来看。陈生见景生来了,才放了女子。女子一见景生,眼睛一眨不眨地盯着他,看了很长时间才转身跑了。二人追了一阵,女子竟不知去向。景生回到家里,关上门刚要上床睡觉,那女子笑盈盈地从里屋出来。景生吃惊地问她,女子回答说:“陈生命薄福浅,不可将终身寄托于他。”景生听后很高兴,问她姓名,女子说:“我老家在齐国,姓齐,小名叫阿霞。”景生与女子说笑,那女子也不拒绝,随后同床共枕。

景生的书斋里常有朋友来往,阿霞总是躲在里间屋里。过了几天,阿霞说:“我暂时离开几天。你这里人太多,我觉得受约束,心中烦躁。从今后我只夜里来。”景生问:“你家在哪里?” 回答说:“不远就是!”说完便早早走了,到了夜里果然又来,两人情意深长。又过了几天,阿霞对景生说:“你我虽然恩爱,但总归是苟合之事。我父亲在西疆做官,明天我要跟随母亲去,找机会禀告他们,咱俩就可以在一起过一辈子了。”景生问:“我们要分别多久?”女子说:“大约十来天。”

女子走后,景生想,光住书房不是长法,搬回家里,又怕妻子妒忌阿霞。盘算不如将妻子休了。主意已定,从此看见妻子就辱骂,妻子不能忍受他的欺侮,哭得直想死去。景生说:“你死了还连累我呢!请快滚!”就赶她走。妻子哭着说:“我跟你十年,从来没有过不好的行为,你为什么对我这样绝情!”景生不理,越发急着撵她走。妻子一看没法了,就出门走了。

从此后,景生就把屋子粉刷一新,里里外外打扫干净,翘首盼望阿霞回来。没想到一直没有阿霞的音信,犹如石沉大海。他妻子回到娘家后,多次托景生的亲友为她说情,想破镜重圆,但景生就是不答应。于是她就改嫁了一个姓夏侯的人。夏侯的住所与景生挨着。两家因地界问题,世代有仇。景生听说妻子嫁给了夏侯,越发怨恨。然而仍希望阿霞快点回来,才可自慰。又过了一年多,仍没见到阿霞的踪影。

一次,正逢海神祝寿大会,祠堂内外善男信女云集。景生也来赶会,远远望见一个女子很像阿霞。景生跟上去,那女子就混入人群中;跟随她走出门外,再继续追她,竟飘然而去。景生追不上那女子,心中又恨又恼地回了家。

后来,过了半年,景生在路上见一位女郎,身穿红色的衣裙,后面跟着老仆,牵着一头黑驴走过来,景生一看是阿霞。因怕认错,就先问仆人:“这娘子是谁?”回答说:“她是南村郑公子的继室。”景生又问:“娶了多长时间了?”“半个月了。”景生想莫不是认错了人?女子闻言,回头一看,景生看清楚了,正是阿霞。知她已嫁他人,气愤填胸,大声喊道:“霞娘!为什么忘了旧约?”仆人听到有人喊叫主妇,很生气,便想打景生。女子急忙制止住,揭开帏幔对景生说:“你这负心人,有何脸面来见我?” 景生辩解说:“是你自己负我,我哪里负你?”女子说:“你负了你的夫人比负我还厉害!你对结发夫妻都那样,何况别人呢?过去我因为你祖上积了德,你将榜上有名,才以身相许;如今因你抛弃了妻子,阴间已削了你的官职。今年开科的亚魁王昌就是替你名位的人。我已是郑公子的人,你不要再有什么念头了。”景生俯首帖耳,嘴里一句话也说不出来。抬头看那女子,已扬鞭飞驰而去,心中只有悔恨而已。

这年开科,景生落榜,亚魁果然名叫王昌。郑公子也考中了。景生因此得了薄倖的名声,四十岁仍没妻子,家境也败落下来,常向亲友讨饭吃。一次偶然去拜访郑公子,郑热情款待他,并留他住宿。阿霞窥见,觉得非常可怜,就问郑公子:“前厅的客人,莫非是景庆云吗?”郑公子反问她是怎么认识的,阿霞回答:“我未嫁给你时,曾在他家避过难,也得到他的照顾。他行为虽贱,而祖德还未断,并且和你过去也是朋友,你应该帮助他。”郑公子认为很对。就让景生脱下破衣,给他换上新衣,留他住了好几天。一天晚上,景生将要上床睡觉,有个丫鬟拿着二十多两银子来赠给他。听到阿霞在窗外说:“这是我的私房钱,略酬谢一下你过去对我的情义。拿回去,找个好女子为伴。幸亏你祖上积德厚重,还可保佑到子孙后代,你不要再办缺德事,缩短你的寿限。”景生表示感谢。

景生回家后,拿出十两银子,买了个乡绅人家的丫鬟。这女子长得丑陋凶悍。后来给他生了个儿子,长大后中了进士。郑公子官做到吏部郎,他死后阿霞给他送葬回来,人们打开车门帘,里面竟空空无人,才知道阿霞不是人类。唉!人没有德行。喜新厌旧,到头来鸡飞蛋打一场空,老天给人的报应也太惨了!


\subsection{1.3.41   李 司 鉴}
\label{\detokenize{p00_u5176_u5b83/_u767d_u8bdd_u804a_u658b_u5fd7_u5f02:id127}}
李司鉴,是河北永年县的举人。他在康熙四年九月二十八日,打死了自己的妻子李氏。地方上就将此案上报广平府。广平府命令把他拘捕,到永年县查审。李司鉴来到县府门前,忽然从卖肉架下夺过一把屠力,飞快地跑进城隍庙。他爬到戏台上面,对着神像跪下,自己说:“神责怪我不该听信奸人的话,在乡村邻里间颠倒是非,叫我割耳朵。”便把左耳割下来,抛到台下。又说:“神责怪我不该骗人银钱,令我剁手指。”遂将左指剁去。又说:“神责怪我不该奸淫妇女,让我自行阉割。” 随后就自阉,顿时昏死在地上。

当时,总督朱云门写呈文奏请朝廷革除李的功名并追究治罪,得到皇上的批准。而这时,李司鉴已经被阴司刑法诛杀。此事抄自邮报。


\subsection{1.3.42   五 羖 大 夫}
\label{\detokenize{p00_u5176_u5b83/_u767d_u8bdd_u804a_u658b_u5fd7_u5f02:id128}}
山西河津县人畅体元,字汝玉。作秀才时,梦见有人称他为“五羖大夫”。他很高兴,认为这是自己仕途显达的一个好兆头。

有一年,碰上流寇之乱。他被流寇逮住,剥光了衣服,关在一间空屋子里。正值隆冬季节,天气寒冷。他在黑暗中摸索,摸到几张羊皮子,用来裹护着身体,才不至于冻死。等到天明一看,恰巧正是五张羊皮。他哑然失笑,知道这是神灵在戏弄自己。

后来,因他有明经科头衔,朝廷授他为雒南知县。这是毕载积先生记载的。


\subsection{1.3.43   毛 狐}
\label{\detokenize{p00_u5176_u5b83/_u767d_u8bdd_u804a_u658b_u5fd7_u5f02:id129}}
农民马天荣,二十多岁时死了妻子,因家穷没有再娶。一天,他在田间干活,见一个少妇浓妆艳抹,踏着庄稼从田埂上走过来。脸面彤红,标致风流。马天荣怀疑她迷路了,环顾四野无人,就调戏她,少妇也微微迎合。马天荣便要求与她野合。少妇笑着说:“青天白日的,干那事合适吗?你回去,掩上门等我,晚上我就来。”马天荣不信,妇人发誓一定去。马天荣就告诉了自己住家的方向,少妇才走了。

夜间,少妇果然来了,两人便成了好事。马天荣觉得少妇的肌肤滑嫩异常,点灯一照,皮肤又红又薄像婴儿,浑身长着细毛。他觉得奇怪,又怀疑她来路不明,自己想这个少妇莫非是狐?就开玩笑般地追问她。那少妇也不隐讳,自认是狐。马天荣对她说:“你既然是仙人,当然会要什么有什么。蒙你对我这么相好,能否送些银子救济我呢?”少妇答应了。第二夜来到,马天荣就向她要银子。少妇故作惊愕地说:“我忘记了。”天明少妇临走时,马天荣又嘱咐了一遍。到了夜晚,少妇来后,马天荣就问:“我要的东西大概没有忘记吧?”妇人笑着说请再等一天。过了几天,马天荣向妇人要银子。妇人笑着从袖中拿出二锭银子给他,约五六两,翅着边有细花纹,非常好看。马天荣很喜欢,包好后珍藏在柜子里。

待了半年,马天荣有事需要钱用,就拿出藏的银子让人看。人们看了后说:“这是锡。”用牙一咬就掉下来。马天荣大为惊骇,收好回了家。到了夜间,妇人来到,马就对她生气地说风凉话。妇人笑着说:“你命薄,担不得真金呀。”一笑了之。马天荣说:“听说狐仙都是国色天香,可你却不然。”妇人说:“我们都是随着人变。你连一金之福都没有,落雁沉鱼的美人,你如何能享受?就我这个丑样子,当然配不上侍奉上流人物;然而比起大脚驼背的女人,也算是天姿国色了。”过了几个月,妇人忽然将三两银子赠给马天荣,并说:“你屡次向我要钱,我因你命薄不应藏有银子,所以没有给你。现在即将有媒人来提亲,我给你够买个媳妇的钱,也借以表示赠别。”马天荣自己表白并没有打算娶妻,妇人说:“一两天之内,自然会有媒人来。”马天荣问:“你没听说那妇人长得怎样?”少妇说:“你想要漂亮的,当然就是漂亮的。”马天荣说:“这我不敢奢望。可是三两银子怎么能买个媳妇呢?”妇人说:“这是月老安排好的,不是人力可以改变的。”马天荣问:“你为什么忽然说咱们要分别?”妇人回答说:“像我们这样戴月披星地偷情,终不是个长法。你自然会有妻子,干吗要这样搪塞下去呢?”天一亮少妇就走了,走时将一包黄药面送给马天荣,说:“分别后恐怕你会得病,服这药可以治好。”

第二天,果然有媒人来马家。马天荣先问女方的相貌,媒人说:“不美,但也不丑。”马天荣问:“多少聘金?”答说:“约四五两银子。”马天荣不愁这个价钱,但必须要亲眼见见那个女子。媒人怕良家女子不肯抛头露面,就约马天荣一起去相看,见机行事。到了女方村边,媒人先进村,让马天荣在村外等着,过了很长时间,才回来说:“巧了!我的表亲与她同住在一个院落,刚才我去看见那女子正在她屋中坐着。请你假装着拜访表亲的,从她身边走过,相距很近,你可偷着看看。”马天荣跟着媒人进去,果然见一个女子坐在屋里,身子伏在床上,正让侍女给她搔背。马天荣从她身边走过,看了一眼,女子长得果然和媒人说的一样。到了商定聘金时,女方并不计较,只要一二两银子,打发女子出嫁。马天荣以为得了便宜,就按数交付了银子,又酬谢了媒人及写婚书的人,三两银子恰好用完,也没多费一文钱。

选了个良辰吉日,将女子娶进门来。一看,原来是个鸡胸弯腰驼背的女人,脖子很短像乌龟;看裙下,露着两只尺把长的大脚。这才明白狐仙说的话是有原因的。


\subsection{1.3.44   翩 翩}
\label{\detokenize{p00_u5176_u5b83/_u767d_u8bdd_u804a_u658b_u5fd7_u5f02:id130}}
罗子浮,邠州人,父母很早就去世了。八九岁时,被叔叔罗大业收养。罗大业任国子监祭酒,富有家产,但没儿子,他疼爱罗子浮就像疼爱亲生的一样。罗子浮十四岁时,被坏人引诱去嫖妓宿娼。当时有个从金陵来的妓女,侨居本郡,罗子浮很喜欢她,被她迷住了。这妓女返回金陵,罗子浮也偷偷地跟着她逃离了家乡。在妓院住了半年,他钱财都花光了。妓女们都讥笑他,但还没有立即赶他走。不久,罗子浮身上长满了梅毒疮,溃烂发臭,沾染床席,被妓院赶了出来。他只得在街市上讨饭,街上的人们见了他都远远地躲着。罗子浮害怕死在异地它乡,便一路讨着饭往西走。每天走三四十里,渐渐到了邠州地界。又想到自己衣衫破烂,脓疮污秽,没脸回家,依旧在临近县里徘徊。

一天傍晚,罗子浮想去山中寺庙投宿。路上遇到一个女子,容貌美丽得跟天仙一样。女子走近他问:“去哪里?”罗子浮实说了。女子说:“我是出家人,住在山洞里,你可以去留宿,还能躲避虎狼。”罗子浮很高兴,跟着女子走了。进入深山中,见有一座洞府,门前横淌着一条小溪,溪上架着根长条石作桥。过桥几步,有两间石室。室内一片光明,不需点灯。女子让罗子浮脱下破衣到溪水中洗个澡,说:“洗洗,疮就好了。”又拉开帷帐,扫扫被褥,催促罗子浮去睡,说:“快睡吧,我要给你做件衣服。”取过一些像芭蕉的大叶子,裁剪好后缝制起来。罗子浮躺在床上看着,见女子做了不一会儿,衣服便缝好了。折叠整齐,放到床头上,说:“明早穿上吧!”说完,便在对面床上睡了。罗子浮洗了澡后,觉得身上的疮不疼了。醒过来一摸,已结了厚厚的疮痂。到第二天早晨,罗子浮要起床,心里怀疑芭蕉叶衣服没法穿。取过来一看,却是绿色的锦缎,光滑异常。过了会儿,女子准备早饭,只见她取过一些山叶来,说是饼,一吃,果然是饼。又把叶子剪成鸡、鱼,烹调好后都和真的一样。室内角落里有个小瓮,盛着好酒。女子一次次取来饮;少了,就再用溪水灌满。过了几天,罗子浮身上的疮痂都脱落了,就到女子床上要求同宿。女子说:“轻薄东西!刚能安身,就要妄想!”罗子浮说:“聊以报答您的大德!”于是二人一起睡了,欢爱非常。

一天,有个少妇笑着进来,说:“翩翩小鬼头快活死了!薛姑子的好梦,几时做成的?”翩翩迎上去笑着说:“原来是花城娘子!你贵足很久不踏贱地了,今天西南风紧,把你吹送了来了。抱了儿子没有?”少妇回答说:“又是个丫头!”翩翩笑着说:“花娘子真是个瓦窑啊!孩子带来了吗?”少妇说:“刚才哄好了,已睡下了!”于是一齐落坐,翩翩设宴款待。少妇又看着罗子浮说:“小郎君烧了好香了!”罗子浮见她有二十三四岁年纪,容貌依旧很漂亮,心里很喜欢她。剥果子时误落到桌子底下,罗子浮俯身假装捡拾,暗地里捏她的脚。花城看着别处笑笑,像不知道。罗子浮正在神魂颠倒,忽觉身上的衣服顿时不暖和了,低头一看,衣服全变成了秋叶。吓得他差点闭过气去,急忙收回邪念,端坐了一会儿,衣服才又渐渐变成了原来的样子。他心里暗自庆幸两个女子都没看见。过了会儿,罗子浮给花城劝酒时,又用手指搔她的掌心。花城坦然地说笑着,一点也没知觉。罗子浮心神不安时,衣服又变成了叶子,过了一阵子才变回来。他只得羞愧地打消了杂念,再不敢妄想。花城笑着说:“你家小郎君太不正经,如不是醋葫芦娘子,恐怕他早跳到云间里去了!”翩翩也讥笑说:“轻薄东西!就该活活冻死!”两人拍掌大笑起来。花城离席说: “小丫头醒来,恐怕把肠子都哭断了。”翩翩也起身说:“贪图勾引人家的男人,就忘了小江城哭死了。”

花城离去后,罗子浮害怕被翩翩讥笑谴责,但翩翩仍和平常一样对待他。住了不久,节令已到深秋,寒风阵阵,霜叶降落。翩翩捡拾落叶,储藏起来准备过冬。见罗子浮冻得瑟缩发抖,她便拿个包袱,到洞口抓白云,絮成棉衣。罗子浮一穿上,觉温暖得像真棉衣一样,而且非常轻快。过了一年,翩翩生了个儿子,非常聪明漂亮。罗子浮天天在洞里逗弄婴儿取乐。但他常常想起家乡,便恳求翩翩一同回去。翩翩说:“我不能跟你去;要不,你自己走吧。”拖延了两三年,儿子渐渐长大,于是就和花城结成了亲家。罗子浮担心叔叔已经老了,没人照顾。翩翩说:“叔叔固然已经高龄,但庆幸比较强健,用不着你挂念。等保儿结婚后,是走是留,全凭你。”翩翩在洞中,总是拿树叶写上字教儿子读书,儿子一看就明白了。翩翩说:“这孩子生就福相,让他到人世上去,不愁做不到高官。”不久,儿子已十四岁,花城亲自把女儿送了来。翩翩见那江城姑娘衣着华美,容光照人,与罗子浮都非常高兴。合家团聚,设宴庆贺。翩翩敲着头钗,唱道:“我有佳儿,不羡贵官。我有佳妇,不羡绮纨。今夕聚首,皆当喜欢。为君行酒,劝君加餐。”酒后,花城离去。翩翩夫妇让儿子、媳妇住对屋。新媳妇很孝敬,依恋在翩翩膝下,就像亲生女儿一样。罗子浮又说要回去。翩翩说:“你有俗骨,终究不是成仙的料。儿子也是富贵中人,你可以带了去,我不耽误他的前程。”新媳妇正想回家跟母亲告别,花城已经来了。儿女恋恋不舍,热泪盈眶。翩翩和花城都安慰说:“暂时离去,以后还可以再回来。”翩翩便把树叶剪成毛驴,三人骑上往回走来。

罗大业此时已告老还乡,以为侄子早已死了。忽见罗子浮带着漂亮的儿子和儿媳回来,罗大业欢喜地像得到了宝贝。罗子浮三人进入家门,分别看看自己的衣服。都变成了芭蕉叶。扯破一看,里面的棉絮像蒸汽一样四散了。于是三人重薪换了衣服。

后来,罗子浮想念翩翩,带着儿子回去探望,只见黄叶满路,白云迷失洞口,再找不到踪迹,只得流着泪返了回来。


\subsection{1.3.45   黑 兽}
\label{\detokenize{p00_u5176_u5b83/_u767d_u8bdd_u804a_u658b_u5fd7_u5f02:id131}}
听太公李敬一说:“有一个人在沈阳,和朋友在山顶上开宴会。低头向山下看时,见有只老虎口里衔着东西走过来,用爪子在地上挖了一个坑,将衔来的东西放进去埋好后就离开了。这人便派了个人去察看埋的是什么,结果,挖出一只死鹿。下人便把死鹿取走,将坑重新埋好。一会儿,那只虎领着一只黑兽走来,黑兽的毛有好几寸长。虎为黑兽带路,好像邀请了一位尊贵的客人。到了埋鹿的坑前,黑兽瞪着眼蹲在一旁等候着。虎挖开坑,鹿不见了,吓得战战兢兢地趴在地上,一动也不敢动。黑兽恼怒老虎欺骗自己,用爪子猛击老虎的额头,老虎立刻就死了。黑兽也悻悻地离去了。”


\section{1.4   卷 四}
\label{\detokenize{p00_u5176_u5b83/_u767d_u8bdd_u804a_u658b_u5fd7_u5f02:id132}}

\subsection{1.4.1   余 德}
\label{\detokenize{p00_u5176_u5b83/_u767d_u8bdd_u804a_u658b_u5fd7_u5f02:id133}}
武昌府的尹图南,有一座空闲着的宅子,租给了一个秀才居住。半年多,尹图南再也没过问这件事。

一天,尹图南在这座宅子门口遇见那秀才。见他年龄很小,但容貌俊雅,风姿翩翩,衣着华丽,便上前和他交谈起来。秀才谈吐文雅含蓄,令人喜爱。尹图南很感惊异,回家后便告诉了妻子。妻子派了个丫鬟以赠送礼物为名,去暗地里察看秀才的家室情况。见他家有个天仙般的美艳女子,家里的花草山石、衣服器具,都是从来没有见过的。尹图南听说后,揣测不出秀才到底是什么人,便去他家登门拜访,正赶上秀才外出了。第二天,秀才就来回拜。尹图南打开他的名帖一看,才知他姓余名德。两人交谈之间,尹图南又详细打听他的家族门第,秀才的回答却十分含糊。尹图南反复地追问,秀才就说:“您如想和我交往,我不敢拒绝。要知道我并不是逃亡在外的盗匪,何必苦苦地逼问来历呢?”尹图南连忙道歉。命家人摆下酒宴,二人吃喝谈笑。一直喝到天黑,才有两个健壮的奴仆,挑着灯,牵着马,把秀才接了回去。

第二天,秀才回请尹图南。尹图南来到他家中,见室内墙壁都用明光纸裱得和镜子一样,光滑洁净。狻猊形状的金香炉里燃着奇异的香料。一只碧玉瓶里插着两支凤尾和两支孔雀翎,都长二尺多。还有一只水晶瓶里浸着一棵开粉色花的花树,叫不出什么名字,也是二尺来高。这花树长长的枝条倒垂着,覆盖在花儿之外,叶疏花密,含苞待放。湿润的花瓣就像收敛着翅膀的蝴蝶,而花蕊就像是蝴蝶的须。酒席上不过摆了八个盘,但每样菜都异常丰美。秀才命童子击鼓催花行酒令。鼓声一响,只见花瓶中的花颤颤地抖动起来。像要折断一样。一会儿,蝴蝶的翅膀渐渐张开,鼓声一停,一声轻响,花蒂和花须立即飘落,变成一只蝴蝶,飞落到尹图南的衣服上。秀才笑着起身,拿个大杯斟上酒让尹图南喝了。酒刚斟满的时候,蝴蝶便飞走了。过了一会儿,鼓声又作,有两只蝴蝶飞到余德的帽子上。余德笑着说:“这可是自作自受了!”也喝了两大杯。第三次鼓声响过,蝴蝶乱纷纷落下,又翩翩地飞到二人的袖子和衣襟上。击鼓的童子笑着过来,用手指点着,数每人身上的花朵:尹图南应喝九杯,余德喝四杯。这时尹图南已微有醉意,不敢多喝,勉强喝了三杯,便离席告辞了。

从此后,尹图南更加感到余德是个奇人。但余德很少和人交往,总是关着门自家过日子。村人们有喜事、丧事,他也从不去庆贺或吊唁。尹图南逢人就宣扬余德,听到他的奇事的人,都争着结交他,常常是贵客盈门,十分热闹。余德很不耐烦,忽然辞别尹图南搬走了。余德走后,尹图南来到他家,见庭院空空,地上洒扫得一尘不染。燃剩的蜡烛堆放在石阶下,窗子上只剩些残帛断线,上面还留着清清楚楚的指痕。只在屋后遗留下一个小白石水缸,能盛一石水左右。尹图南把缸拿回家去,贮上水养了几尾红鱼。过了一年,缸里的水仍然清澈如初。后来,这缸被仆人们搬动石块时失手打碎了。奇怪的是缸里的水像凝固了一样,也不流泻出来。再一看,缸好像仍在那里,用手一摸却空空软软的。手一伸进去,水就随着手流出来;拿出手,水又合拢起来。到了寒冬,水也不结冰。一夜,缸水忽然结成水晶状,但红鱼依然在里面自由自在地游动。尹图南恐怕别人知道这件奇珍,总是把它藏在密室里,除了儿子、女婿这样的亲人,从不拿出给人看。但时间长了,还是传了出去,要求观看的人纷纷登门,络绎不绝。

在腊月的一夜,水晶忽然又分解为水,流了一地,红鱼也不见了。原来碎缸的残片还在。忽然来了个道士,登门索要碎缸片。尹图南拿出一片让他看,道士说:“这是龙宫中盛水的器具。” 尹图南又描述了缸破后水不流泻的情景,道士说:“贮水的是缸的魂魄。”说完,很殷切地恳求给一小块碎缸片。尹图南问他有什么用,道士说:“把它捣为碎末入药,能使人长生不老。”尹图南给了他一片,道士非常感谢,欢欢喜喜地走了。


\subsection{1.4.2   杨 千 总}
\label{\detokenize{p00_u5176_u5b83/_u767d_u8bdd_u804a_u658b_u5fd7_u5f02:id134}}
毕自严公就任陕西洮岷备兵官职时,千总杨化麟前来迎接。仪仗行进途中,偶见有人蹲在路旁大便。杨千总想用弓箭射他,毕公急忙呵止。杨千总说:“这个奴才太无礼了,应该让他受点小惊吓。”于是远远地喊道:“哎!大便的!送给你一股会稽藤簪绾髻子用。”随即一箭射去,正中遗便者的发髻。这个人吓得急起奔跑,便液撒了一地。


\subsection{1.4.3   瓜 异}
\label{\detokenize{p00_u5176_u5b83/_u767d_u8bdd_u804a_u658b_u5fd7_u5f02:id135}}
康熙二十六年六月,城西一个村民的菜园里,黄瓜上又生出另一枝蔓来,结了一个像碗一样大的西瓜。


\subsection{1.4.4   青 梅}
\label{\detokenize{p00_u5176_u5b83/_u767d_u8bdd_u804a_u658b_u5fd7_u5f02:id136}}
南京有个姓程的书生,性情磊落,不受礼俗的约束。一天,他从外面回来,宽解衣带时,觉得衣带末端很沉重,像有东西往下堕。看了看,并无任何东西。转身之间,有个女子从衣服后面出来,手理秀发向他微笑,真是美丽极了。程生怀疑她是个鬼。女子说:“妾不是鬼,是狐。”程生说:“倘若能得到美人,就是鬼也不可怕,更何况是狐呢!”于是和她亲热起来。过了二年,生了个女儿,取小名叫青梅。狐女常对程生说:“你不要再娶妻子了,我会为你生个儿子的。”程生相信了狐女的话,就不再娶妻。但是,亲戚朋友们都讽刺讥笑他。程生动摇了,终于改变了主意,聘了湖东的王氏为妻。狐女听说后,非常恼怒,抱起女儿喂完奶,抛给程生说: “这是你家的赔钱货,愿意养她或杀她,全由你;我何必代人作奶妈呢!”说着出门而去。

青梅长大了,非常聪明,相貌美好秀丽,酷似她的母亲。不久,程生病死,王氏改嫁出走,把青梅寄养在堂叔家里。她的堂叔品行恶劣,行为放纵,竟想把青梅卖掉得钱自用。恰好有个正在家候选官职的王进士,听说青梅很聪明,便出大价钱把她买来,让她给自己的女儿阿喜当侍女。阿喜十四岁年纪,容貌美丽绝顶。她见了青梅非常高兴,就和她同住在一起。而青梅也善于侍奉人,聪明伶俐,会看眼眉行事,因此王家人全都喜爱她。

城里有个姓张的书生,字介受,家境贫穷,没有财产,租赁了王进士的房子居住。张生非常孝顺,遵守礼仪,品行端正,又勤奋好学。青梅偶然有事到张家,看见张生坐在石头上吃米糠粥;她进屋和张母说话时,却见桌子上摆着味美的猪蹄。当时张翁正卧病在床,张生进屋抱着父亲小便。便液沾脏了张生的衣服,父亲觉察了非常恨自己,而张生却掩盖着脏处,急忙出屋自己洗净,唯恐让父亲知道。青梅看了大为惊奇,回来后就对阿喜讲述在张家见到的情形,并说:“咱家的房客,是个不同寻常的人。您若不想得好夫君便罢;想得好夫君,张生就是理想的人。”阿喜恐怕父亲嫌张生贫贱。青梅说:“不见得,这事全在您自己。假如您认为合适的话,我可以偷偷地告诉张生,让他家请媒人来提亲。到时候老夫人一定要召您去商量这事,只要您应着‘同意’,事情就好办了。”阿喜怕跟了张生穷一辈子让人耻笑。青梅说:“我自以为能为天下士人看相,绝不会出错的。”

第二天,青梅把意思告诉了张生的母亲,张母大惊,说她说的话不是好兆头。青梅说:“我家小姐听说公子人品好,赞美他有道德有才能,我是因为摸透了她的心意才来这样说的。您请媒人去提亲,我和小姐两人从中帮助,估计王家能够应允。即使王家不同意,对公子来说还有什么辱没吗?”张母说:“行。”于是便托卖花的侯氏前去做媒。王夫人听说就笑了,并把这事告诉了丈夫。王进士也大笑起来。便把女儿叫到面前,说明了侯氏的来意。阿喜还没来得及回答,青梅急忙夸赞张生贤能,并断言他日后必定富贵。夫人又问女儿:“这可是你的百年大事。假如你愿意吃糠咽菜,就为你答应这门亲事。”阿喜低头沉思了好一会,看着墙壁回答说:“贫富是个命。倘若命厚,就是贫也贫不了几天;而命中注定不贫,那就更不会有多少穷日子了。假如命薄,就是那些富贵子弟,后来穷得无立锥之地的难道还少吗?这事全在父母作主。”最初,王进士叫女儿来商量,是想拿这事来博一笑;听到女儿的话,心里很不高兴,说:“你真想嫁给张家吗?”女儿没回答;再问,还是不回答。王进士非常气忿地说:“贱骨头全不长进!想提着讨饭筐当叫花子媳妇,岂不羞死!”女儿被骂得涨红着脸透不过气来,含着眼泪退去。媒人见事不妙也跑了。

青梅见为小姐办不成,便想着替自己来谋求。过了几天,她趁夜间到张生家里去。张生正在读书,见她来,非常震惊,问她来干什么,她说话吞吞吐吐。张生很严肃地让她离去。青梅哭着说:“我是好人家的女儿,并不是来私奔;只是因为你贤德,所以我才自愿以身相托。”张生说:“您爱我,说我贤德。然而昏天黑夜里来往,连洁身自爱的人都不愿做,而所谓贤德的人能去做吗?就是起初不正当而最终能成就的事,君子还说不可;更何况不会成就的事!以后你我怎么做人?”青梅说:“万一能成的话,你愿意收留我吗?”张生说:“能得到您这样的人就非常满足了,还要求什么呢?只是眼下有三件难事,因此不敢轻易答应。”青梅问:“什么难事?”张生回答:“您不能自己作主,是一难;即使您能自己作主,若我父母不乐意,是二难;就算我父母乐意,而您的身价必定很高,我家贫拿不出应付的钱,是尤其难。您赶紧走吧,瓜田李下的嫌疑是令人可畏的!”青梅只好回去,临走又嘱咐道:“您若有意,求您和我共同想办法来促成。”张生答应了她。

青梅回来,阿喜追问她到哪里去了,她就跪下主动承认去过张家。阿喜非常生气,以为青梅私奔,要用家法责打。青梅哭着说自己没干见不得人的事,于是把实情告诉了她。阿喜赞叹道: “不私自结合,是礼;一定禀告父母,是孝;不轻易许诺,是信。有这三德,老天必定会保佑他的,张生不用再担忧自己贫困了。”随后又说:“你打算怎么办?” 青梅回答说:“要嫁给他。”阿喜笑着说:“傻丫头,你能自己作得了主吗?”青梅说:“若不成,就去死!”阿喜说:“我一定满足你的愿望。”青梅便叩头感谢她。

又过了好几天,青梅对阿喜说:“以前您说的是玩笑话呢,还是真想发慈悲呢?若当真的话,我还有些难言的隐情,再求您同情帮助。”阿喜问是什么事。青梅回答道:“张生拿不出订婚的聘礼,我又没有能力自己赎身,如非要原来身价的话,同意把我嫁给他实际上还是不同意。”阿喜沉吟着说:“这不是我能办到的事。我说把你嫁给他,还怕不太合适。再说一定不要你的身价,这是父母绝不会应允的,也是我不敢说的。”青梅听了,难过地流下眼泪,只是求阿喜能同情帮助她。阿喜沉思了好一阵,说:“实在没有办法,我自己积攒了一些钱,全部给你帮忙吧。”青梅拜谢了阿喜,并把这事偷偷地告诉了张生。张母知道了非常高兴,多处求借,凑齐了身价钱,收藏起来等着听好消息。

正巧王进士被选任山西曲沃知县,阿喜趁机对母亲说:“青梅年龄也不小了,咱们又要随父亲上任,不如送她走了吧。”母亲本来就认为青梅太伶俐,怕她引导阿喜不走正路,多次想把她嫁出去,就怕女儿不乐意。现在听女儿这么说,心里非常高兴。过了两天,有个佣人的妻子来说了张家想娶青梅的意思。王进士笑着说:“这家人也只配找个丫鬟作媳妇,他们前次的做法简直也太荒唐了!不过要把她卖给富贵人家做妾的话,价钱还能比过去高一倍。”阿喜急忙进屋说:“青梅侍奉我这么长时间,把她卖给人家做妾,太不忍心了。”王进士于是传话给张家,仍然按原来的身价付钱,还了卖身契,把青梅嫁给了张生。

青梅嫁到张家后,孝敬公婆,尽心周到,胜过了张生。而操持家务更是勤快,糠秕当饭也不觉得苦,因此全家人都非常敬重她。青梅又以刺绣为业,她绣出的东西卖得很快,商贩们等候在张家门前抢购,惟恐得不到手。用刺绣换来的钱多少可以应付穷日子。她还劝张生不要光顾家耽误了读书,家里的事情全由她自己承担起来。因为主人就要上任了,青梅便去与阿喜道别。阿喜见到她,哭着说:“你得到了好的归宿,我实在不如你。”青梅说:“我知道这是谁赐给我的,怎敢忘了呢?不过您认为不如我,恐怕要折我的寿了。”于是两人哭着惜别。

王进士一家到了山西任上,仅半年,夫人就死了,灵柩停在寺庙中。又过了两年,他这个知县因为行贿罪被免职,罚交赎罪的银两数以万计,因而家道渐渐贫困不能自给,随从们也都四下逃散。这时,瘟疫流行,王进士感染疾病也死了,仅有一个年老的女佣人跟随着阿喜。没过多久,女佣人又死去,只剩下阿喜自已孤苦伶仃,日子越加难过。有个邻居老太婆来劝阿喜出嫁。阿喜说:“谁能为我埋葬父母,我就嫁给谁。”老太婆很同情她,送给她一斗米就走了。半月后老太婆又来说:“我为你费了很大劲,事情很难办。贫的不能为你葬双亲,富的又嫌你家道败落,怎么办!还有一个主意,只是怕你不会同意。”阿喜问:“什么主意?”老太婆回答:“这地方有个李郎,想讨个二房,若见到你的容貌,即使让他多花钱来厚葬你的父母,他必定在所不惜。”阿喜大哭道:“要我这官宦人家的女儿去做妾啊!”老太婆没再说话,就走了。阿喜自此每日只吃一顿饭,勉强维持着等待有人出钱买她。这样过了半年,日子越来越难维持。有一天,老太婆又来了。阿喜哭着对她说:“困难到这种地步,常想自杀;所以还能苟活着,仅仅是因为还存双亲的灵柩停在这里。我自己死了填沟壑不要紧,谁来收我父母的尸骨呢?因此想还不如按照你说的主意办吧。”

老太婆于是领李郎来,他一见到阿喜,心中大喜,立即出钱为阿喜父母办理安葬。等一切处理完了,就用车把阿喜拉回家,去见他的大老婆。因为这大老婆既厉害又嫉妒,所以李郎起初不敢说阿喜是妾,只是假说买了个侍女。等到见了阿喜,大老婆暴跳怒骂,拿木棍把她打了出去,不让再进门。阿喜披头散发痛哭流涕,进退两难。正好有个老尼姑经过这里,见状动了恻隐之心,便邀她一同居住。阿喜转悲为喜,就跟老尼姑走了。

到了庵堂中,阿喜拜求削发为尼。老尼不同意。说:“我看你并不是久落风尘的人。庵中的粗碗糙米大体上可以自足,你暂且先寄居在这里等待着。只要时机到来,你就会自己走的。”这样住了不长时间,城市中的一些无赖之辈见阿喜长得美,经常来敲门并说脏话调戏她,老尼也无法制止他们,逼得阿喜又是哭叫又是寻死的。为此,老尼前去请求吏部的某官专门贴了告示严厉禁止,这些恶少们才开始稍微有些收敛。后来又有人乘黑夜在庵墙上挖洞,幸被尼姑们发现惊呼才离去。因而再次告到吏部某官那里,捉住了首恶,送郡城中拷打,才渐渐安稳了。又过了一年多,有个贵公子经过庵中,被阿喜的美貌惊呆了,硬求老尼替他通殷勤,又重礼厚赂老尼。但老尼婉言对他说: “她是官宦世家的后人,不会甘心给人家作侍妾的。公子暂且回去,推迟几天再去给您报信。”贵公子走后,阿喜想服毒药求死,夜里梦见父亲来,很痛心地说: “以前我没有依从你的心愿,才使你至于此,现在后悔已经晚了!但只要你暂缓片刻不死,夙愿还可以再实现。”阿喜感到非常奇怪。天亮了,阿喜梳洗过后,老尼见了惊讶地说:“看您的脸上,浊气已经全消了,一切艰难和不顺心的事都不用再愁了。您的福气就要来了,不要忘了老身啊。”话未说完,就听到了敲门声。阿喜惊慌失色,知道必定是贵公子的家奴,老尼开门一看果真是他。家奴急问事情的结果,老尼好话应承,再请宽限三日。家奴转达主子的话,事若不成,让老尼亲自向公子回话。老尼毕恭毕敬满口答应,说着感谢话打发家奴走了。阿喜大为伤心,又想自尽。老尼急忙劝止。阿喜担心贵公子过三天再来催,无话可对。老尼说:“有我在,要砍要杀我自己承当。”

第二天下午,下起了倾盆大雨。忽然听到有好几个人用力敲门,并大声喊叫。阿喜以为发生了什么变故,吓得手足无措。老尼冒着大雨开开门,看见门前停放着一抬轿子;有几名丫鬟从里面扶出一位美人来,随从簇拥,声势显赫,车轿非常漂亮。老尼惊奇地问他们有什么事,回答说:“是司理大人的家眷,想在这里暂时避避风雨。”老尼引导美人进了大殿,移过坐榻恭敬地请她坐下。家人和女佣们全都跑向禅房,各人寻找休息的地方。女佣进屋见到了阿喜,见她很美,连忙跑去告诉了夫人。不多时,雨停了,夫人起身要去禅房看看。老尼领她进屋,夫人见到阿喜惊呆了,两眼盯着一眨也不眨,阿喜也把她端详了好一阵子。这位夫人不是别人,竟是青梅。两人相认都失声痛哭,于是谈起了分别后的经历。原来张翁病故后,张生服丧期满复出做官,连连升迁,被授予司理官职。他先同母亲一起赴任,随后这才来搬家眷。阿喜叹息着说: “今日看来,你我二人可以说是有天壤之别呀!”青梅笑着说:“幸亏您遭受磨难未嫁夫君,老天爷匹想叫我们两人团聚呢。假如不是遇到这场大雨,怎么会有今天的相逢呢?这其中全有鬼神相助,并非是人力能办到的。”于是拿过珍珠蔻和锦缎绣衣,催促阿喜换装。阿喜低头徘徊不接,老尼从中极力夸赞并劝说她。阿喜担心到张府同居名不正言不顺。青梅说:“咱俩的名位以前早有定分,婢子我哪敢忘了您的大恩大德!试想那张郎岂是忘恩负义的人?”说完硬为阿喜换上装,辞别老尼而去。

到了司理官邸,张氏母子见了都很欢喜。阿喜拜见老夫人说:“我今天真没有脸面来见母亲。”张母笑着安慰她。随后商量选择吉日举行婚礼。阿喜对青梅说:“尼庵中只要有一线生路,我也不愿意跟随夫人到这里来。若念往日的友情,能得到一间房子,只要容得下一个能坐的蒲团就很满足了。”青梅笑笑没有答话。到了婚礼那天,她把华丽的礼服抱了过来,阿喜左右为难,不知如何是好。忽然听见鼓乐声响了起来,她也身不由己了。青梅带领丫鬟女佣硬给她换上礼服,簇拥着走出来。见张郎身穿朝服在拜,于是自己也不觉盈盈而拜。青梅把她拉入洞房,说:“空着这个位子等待您已经很久了。”又回头对张郎说:“今夜是您报恩的机会,可要好自为之。”说完返身要走,被阿喜捉住了衣襟。青梅笑着说:“不要留我,这事可不能代替。”掰开阿喜的指头脱身而去。

自此,青梅小心谨慎地侍奉阿喜,从不冒犯。而阿喜始终惭愧心中不安。于是张母便叫对她两人都称夫人。但是青梅仍以原来的名分对阿喜行婢妾礼,而且从不懈怠。过了三年,张生由司理职选调进京,经过尼庵,送上五百两银子酬谢老尼。老尼不收。再三强留,于是收下二百两,用来修建了大士祠,立起了王夫人碑。后来张生官职做到侍郎。程夫人青梅生了两个儿子一个女儿,王夫人阿喜生了四个儿子一个女儿。张侍郎又上书皇帝陈述了事情的始末,青梅和阿喜都被封为夫人。


\subsection{1.4.5   罗 刹 海 市}
\label{\detokenize{p00_u5176_u5b83/_u767d_u8bdd_u804a_u658b_u5fd7_u5f02:id137}}
马骥,字龙媒,是商人的儿子。他风度翩翩,一表人材,从小就洒脱大方,喜欢唱歌跳舞。经常跟着戏班子演出,用锦帕缠着头,就像一个美丽的少女,因此又有“俊人”的美称。他十四岁考中秀才,很有名气。父亲年老体衰,放弃了经商,回家闲住,对马骥说:“几卷书,饿了不能煮着吃,冷了不能当衣穿,我儿应该继承父业去经商。”马骥从此就慢慢做起买卖来。

一次,马骥跟别人去海外经商,被飓风刮走了。漂了几天几夜,来到一个都市。这里的人个个都非常丑陋,看见马骥来,以为是妖怪,都惊叫着逃走了。马骥刚见到这情景时,还很害怕;等知道那些人是惧怕自己时,就反而去欺负他们。遇到吃饭的,他就跑过去,人家吓跑了,他就把剩余的饭菜吃掉。这样过了很久,进入一个山村。山村中的人相貌也有像人的,但是都破衣烂衫,像讨饭的。马骥在树下休息,村里人都不敢过来,只是远远地看着他。时间长了,觉出马骥并不是吃人的妖怪,才开始慢慢接近他。马骥笑着同他们攀谈,他们的语言虽然不同,但大半能听懂。马骥就告诉他们自己的来历。村里人很高兴,遍告乡邻:来客不吃人。但是那些长得丑陋的,看看他就跑了,始终不敢到跟前来。那些来的人,五官的位置都与中国人大体相同。他们摆上酒菜共同招待马骥。马骥问他们怕他的原因,回答说:“曾经听祖父说;往西走二万六千里,有个中国。那里的人形象都很诡秘奇异。原来只是听说过,现在才相信了。”问他们为什么这样穷,村人回答说:“我国所看重的不在学问才能,而在相貌。长得最美的做大官,稍差一点的做小官,再差一点的也能受到贵人的宠爱,得到赏赐的食物,养活妻儿。像我们这样的,刚出生时,父母就以为不吉利,常常都被抛弃了。父母不忍心丢弃的,也都是为了传宗接代罢了。”马骥问:“这叫什么国?”回答说:“叫大罗刹国,往北三十里是都城。”马骥请他们领着到都城看看。于是,第二天鸡一叫村人就起身,领马骥一块去了。

天亮后,才到达都城。都城的城墙是用黑石头砌的,颜色像墨一样黑。楼阁高近百尺,但很少用瓦,都用红色石头盖顶。抬一块碎石在指甲上磨磨,和红色的朱砂没有两样。这时正好退朝,朝中有一顶大轿子出来,村人指着说:“这是宰相。”马骥一看,那人两只耳朵朝后长着,三个鼻孔,睫毛像帘子一样盖住了眼睛。又出来几个骑马的,村人说: “这是大夫。”挨着指出各人的官职,大都是披头散发、相貌狰狞的丑八怪。官职越低的,丑相也渐减。一会儿,马骥往回走,街市上的人看见他,吓得大声嚷叫着,跌跌撞撞地跑了,就像碰上了怪物。村人再三说明,街市上的人才敢远远地站着看。

回去以后,罗刹国里老老小小都知道了山村有一个奇怪的人。于是大小官员都想见识见识,就叫村里的人把马骥送去。可是每到一家,看门人总是把门关死,男女老少偷偷地从门缝里往外瞅着议论着。整整一天,没有一个敢开门让马骥进去的。村人说:“这里有一个执戟郎,曾为先王出使外国。他见得多,可能不会害怕你。”领着马骥去登门拜访。那位执戟郎果然很高兴,把马骥奉为上宾。马骥看他的相貌,像有八九十岁,眼睛突出,胡须卷曲得像刺猬。执戟郎说:“我年轻时,曾奉国王的命令,出使过许多国家,唯独没有去过中国。如今我一百二十多岁了,能有幸看到上国的人物,这可不能不报告天子。但是我已经退职,十多年不去朝廷了。明天早上,就为你去一趟。”说完,备了酒菜,招待马骥。酒过数巡,出来十多名歌女,轮流歌舞。都长得像夜叉样,全用白锦缠着头,红色的衣服拖在地上。不知扮的什么角色,唱的什么歌词,腔调节奏都很特别。主人看着很满意,问:“中国也有这样好的歌舞吗?”马骥说:“有。”主人请马骥模仿几句。马骥就用手敲着桌子唱了一曲,主人高兴地说:“真妙啊!你的歌声就像凤鸣龙啸,我从没听到过。”

第二天,执戟郎上朝,把马骥推荐给国王。国王高兴地要下诏书召见。有两三个大夫说,马骥样子怪异,怕惊吓了皇上龙体,国王才没有召见他。执戟郎出来告诉马骥,深表惋惜。马骥和他一同居住了好多天,同主人一起饮酒,喝醉了,拔剑起舞,用煤粉抹在脸上扮成张飞。主人认为很美,说:“请你扮成张飞去见宰相,宰相一定乐意用你,高官厚禄不难到手。”马骥说:“嘻,闹着玩玩还行,怎么能换个假脸去谋取荣华富贵呢?”主人再三强求,马骥才应了。主人马上备了酒筵,请那些大官们来喝酒,叫马骥画了脸等着。不久客人来了,主人喊马骥出来见客。客人惊讶地说:“奇怪,怎么前几天那样丑陋,今天又这样漂亮!”于是就同马骥一起喝酒,非常快活。马骥跳着舞,唱了一首“弋阳曲”,满座的客人无不倾倒。

第二天,大官们纷纷上奏国王,推荐马骥。国王高兴,派使者持旌节以礼召见他。见面后,国王问马骥中国治国安邦的办法,马骥原原本本地陈述了一番。国王大加赞赏,在别宫赐宴款待。喝到畅快的时候,国王说:“听说你善唱优雅的乐曲,能不能叫寡人欣赏欣赏?”马骥便起身舞起来,也仿效罗刹舞女的样子用白锦缠头,唱些靡靡之音。国王高兴极了,当天就封他为下大夫。并经常请马骥参加家宴,特别恩宠他。时间长了,那些官僚们都知道了马骥的面目是假的。他无论走到哪里,总是看见人们小声耳语,不愿意同他接近。马骥感到很孤立,心里很不安,就上书国王要求辞职,国王不准。他又要求休假,国王便给了他三个月的假期。于是马骥坐官车载着金宝又回到了山村。村人跪在路上迎接他,马骥把金钱分给过去与他结交的那些朋友,村里欢声雷动。村人说:“我们这些小人受到大夫的恩赐,明天去海市,寻求些珍贵玩物,来报答大夫。”马骥问:“海市在什么地方?”村人说:“海市是四海蛟人聚集在那里卖珠宝的地方。到时四方十二国,都去做买卖。集市中还有许多神人来游玩。云霞遮天,波涛汹涌。那些贵人们都珍惜自己,不敢去冒险,只是把银钱交给我们,替他们买奇珍异宝。现在离海市的日子不远了。”马骥问他们怎么知道日期,村人说:“如果看见海上有红色的鸟飞来飞去,七天以后就是海市。”马骥问他们动身的日期,想一起去看看。村人劝他自己珍重。马骥说:“我本来就是海上客,还怕什么风涛浪涌。”

不几天,果然有人登门送钱托他们买东西。马骥就和村人把钱装上船,一起去了。船能容几十个人,船底是平的,栏杆高高的,有十个人摇橹,船像飞箭一样行进。走了三天,远远看见水云荡漾之中,楼阁层层叠叠,各处来做买卖的船,像蚂蚁一样纷纷聚集。不多会儿,来到城下,见墙上的砖,都和人一样长,城楼高得接天。他们系好船进城,见集市上摆放的货物,全是奇珍异宝,光彩夺目,都是人世间没有的。有一位少年骑着骏马走过来,集市上的人都急忙躲开,说是“东洋三世子”来了。世子过来,看见马骥,说:“这不是偏远小国来的人。”接着就有个在马前开路的人问马骥乡籍是哪里,马骥站在路旁行了礼,详细讲了自己的籍贯和姓氏。世子高兴地说:“你既然能屈尊来到这里,说明我们的缘分不浅。”于是就给他一匹马,请他同行。

二人出了西城,刚走到岸边,骑的马嘶叫着跃进水中,马骥吓得失声喊叫。却见海水从中间分开,两边的水像墙壁一样屹立着。一会儿,看见一座宫殿,玳瑁装饰的梁,鱼鳞片做的瓦,四壁亮如水晶,夺目耀眼,能照出人影。马骥下马,世子拱手将他请入,抬头看见龙王坐在殿上。世子启奏道:“臣游览海市,遇见这位中华贤士,领他来参见大王。” 马骥上前跪拜行礼。龙王说:“先生既然是位有文才的学士,一定能够胜过屈原、宋玉。我想烦劳你的大手笔,写一篇描写海市的文章,希望你不要吝惜你的妙词。”马骥叩头答应了。龙王给他一方水晶砚台,一枝龙须笔,光滑如雪的纸张,香气如兰的墨。马骥立时写出了篇千余言的文章,呈献给龙王。龙王赞赏说:“先生真是高才,给水国添了光彩!”接着召集龙族,在采霞宫举行盛宴。酒过几巡,龙王举杯向马骥说:“寡人有个爱女,还没有许配人家,愿意把她许给先生,先生意思如何?”马骥忙离席站起,惭愧地表示感激,连连答应。龙王便对左右说了。不一会儿,有几个宫女扶着一个女郎出来,佩环声声,鼓乐齐奏。拜完天地,马骥偷眼一看,那女郎真是一位天仙。龙女拜完天地就走了。不多会,宴席散了,两个丫鬟挑着宫灯,领着马骥进了旁宫。龙女正浓妆坐等。珊瑚做的床上,装饰着各种珠宝;帐外流苏,缀着斗大的明珠;床上的被褥又香又软。天刚亮,便有许多年轻美貌的丫鬟使女前来侍候。马骥起床后,上朝拜谢。龙王封他为驸马都尉,并把他写的《海市赋》传送四海龙宫。四海龙王都派专员来祝贺,争着下请柬请驸马赴宴。马骥身穿锦绣衣衫,坐着青龙拉的车子,前呼后拥,外出赴宴。几十名骑马的武士都身佩雕弓,扛着白色的棍杖,威风凛凛。骑马的弹筝,坐车的奏玉,三天里,游遍各海。从此“龙媒”的名字,传遍四海。

龙宫里有一棵玉树,一人多粗,树干晶莹透澈,像白琉璃;中间有一淡黄色的心。比胳膊稍细一点;叶子类似碧玉,有铜钱那么厚;树荫细碎浓密。马骥常同龙女在树下吟诗唱歌。树上开的花形状类似枙子花,花瓣落在地上,发出锵的一声。拾起来看看,像用红色玛瑙雕成的,光明可爱。常有一种奇异的鸟儿飞来啼叫,金绿色的羽毛,尾巴比身体还长,叫声像玉笛奏出的哀婉乐曲,使人忧伤。马骥一听这鸟的叫声就思念家乡,对龙女说:“我流浪在外三年了,远离父母,每当想起他们,便伤心流泪。你能跟我回家乡吗?”龙女说:“仙境同尘世隔绝,不能跟随你去。我也不忍心以夫妻之爱,夺走你父子之情。容我慢慢想个办法。”马骥听了,忍不住又流下眼泪。龙女也叹息说:“这实在是不能两全齐美的事啊!”

第二天,马骥从外边回来,龙王说:“听说驸马思念故乡,明天早晨收拾行装送你上路,可以吗?”马骥连忙拜谢说:“我一个孤身旅居在外的臣子,受到过分的优待宠爱,感恩图报之情,牢记在心中。容许我暂时回家探望一下父母,以后还要回来团聚。”到了晚上,龙女摆酒话别。马骥同她约好以后见面的日子,龙女说:“我们的情缘已经到头了。”马骥非常悲痛。龙女说:“回家奉养双亲,可见你有孝心。人生聚散,百年如同旦夕,何必像多情儿女般哭泣?今后我一定为你坚守贞节,你也要为我不再另娶,两地同心,就是美满夫妻。何必一定要早晚守在一起,才叫白头偕老呢?要是违背了盟誓,再婚嫁也不会吉利。如果顾虑无人主持家务,你可以收一个婢女为妾。还有一件事要嘱咐你,成亲后,我好像怀孕了,请给孩子取个名。”马骥说:“如果是女的,就叫龙宫,男的就叫福海。”龙女要一件东西作凭证,马骥把在罗刹国得到的一对赤玉莲花拿出来给她。龙女说:“三年后的四月八日,你要划船去南岛,那时送还你的儿女。”龙女用鱼皮做了个口袋,装满珠宝,送给马骥说: “你好好珍藏,几辈子也吃不完用不尽。”天刚放亮,龙王设宴饯别,赠送马骥许多礼物。马骥拜别出了龙宫,龙女乘白羊车送他到海边。马骥上岸下了马,龙女说声“珍重”,掉转车头回去了。不一会,就走远了,海水又合到一块,再也看不见了。马骥便往回走来。

自从马骥被海水漂走,人们都以为他已经死了。他一到家,家里人无不惊疑。幸亏父母都健在,只有妻子已经改嫁了。马骥这才明白龙女“守义”的话,原来已经先知道自己的妻子改嫁了。父亲想为马骥再娶一房妻子,马骥不答应,只收了一个婢女做妾。他牢记龙女叮嘱的三年期限。到日子后乘船来到岛中,看见两个小孩坐浮在水面上,拍打着水嬉笑,不动也不下沉。马骥到跟前用手一拉,一个小孩笑着抓住马骥的手臂,跳入他怀里;另一个大声哭起来,似乎怪马骥不拉自己,马骥就把他也拉上来。仔细看去,一男一女,相貌都很俊秀。头上的花帽子各点缀着一块玉,便是那赤玉莲花。背上有个锦囊,拆开一看,里边有一封书信,上写:“公婆想必都安康吧!转眼已过三年,红尘永远隔离了我们,盈盈一带之水,书信难通。朝思暮想,只有梦中才能相见;殷切地盼望,盼得脖子发酸。面对茫茫大海,有恨又有什么办法呢?又想奔月的嫦娥,尚且独守月宫;投梭的织女,也在天河一边惆怅。我是什么人,哪能永远和爱人相聚?每每想到这里,便又破涕为笑。我们分别两个月后,竟生了一对儿女。如今已经在怀抱中咿呀学语,能懂笑语,摸枣抓梨,没有母亲也可以活下去了。现在把他们送还给你。你赠送的赤玉莲花,装饰在孩子们的帽子上作为凭证。你把孩子抱在膝头时,就像我在你身边一样。知道你履行了过去的盟誓,心里很安慰。我这一生不会有二心,到死不会再嫁别人。梳妆匣里不再放兰膏;对镜梳妆,久已不涂抹脂粉。你就好比久出远门的游子,我就是游子之妇,虽然远隔两地,但我们仍是恩爱夫妻。只是想公婆虽然已经抱上孙子,却从没见过儿媳,按情理说,也是个缺陷。一年后婆婆安葬时,我一定亲临墓穴,尽儿媳孝道。从此以后,则‘龙宫’平安,还有见面之期;‘福海’长寿,或许还能来往。希望你多多珍重,想要说的话是说不完的。”马骥反复读着书信,泪流不止。两个孩子抱着他的脖子说:“回家吧。”马骥更加悲痛,抚摸着他们说:“我儿知道家在什么地方?”孩子更加哭闹,伊伊呀呀地喊着要回家。马骥望着茫茫大海,无边无际,看不见龙女的影子;波浪翻腾,没有去龙宫的道路。只好抱着孩子掉转船头,满腹惆怅地回去了。

马骥知道母亲的寿命不长了,把衣服棺木都准备好了,在墓地上种植了一百多棵松树。过了一年,母亲果然死了。灵车刚到墓地,就有一个穿孝服的女子走近墓穴哭吊。众人正吃惊地看她时,忽然风激雷轰,接着下起了急雨,转眼间那女子已经不见了。新种的松树本来大都枯萎了,这时又全活了。福海稍长大一点,常常思念母亲,忽然自己投入大海,几天后才回来。龙宫因为是女孩不能去,常常关上门独自哭泣。一天,大白天忽然乌云遮天,龙女走进房内,劝女儿说:“儿自己能长大成家,为什么哭泣?” 说着赐她一棵八尺高的珊瑚树,一帖龙脑香,一百颗明珠,一对八宝嵌金盒子,作为嫁妆。马骥听说龙女来了,急忙跑进来,拉着手就哭。顷刻间,一声疾雷震破房顶,龙女已经不见了。


\subsection{1.4.6   田 七 郎}
\label{\detokenize{p00_u5176_u5b83/_u767d_u8bdd_u804a_u658b_u5fd7_u5f02:id138}}
武承休,是辽宁辽阳县人。他喜欢结交朋友,所交往的都是些知名人物。一天夜里,梦见一个人告诉他说:“您的朋友遍天下,都是滥交。惟有一人可以和您共患难,怎么反而不去结识呢?”武承休问道:“他是谁呀?”那人说:“不就是田七郎吗?”武承休醒来感到很奇怪。第二天早晨,他见到朋友们,就打听谁是田七郎。朋友中有人认得田七郎是东村一个打猎的。武承休便寻访到田家,用马鞭子敲门。不多时,有个人出来,年纪二十多岁,生得虎目蜂腰,戴着一顶满是油污的便帽,穿着黑色的犊鼻裤,上面有很多白补丁。他拱手齐眉问客人从哪里来。武承休说出自已的姓名;并假托路上不舒服,要借间房子暂时休息一下。他打听谁是田七郎,七郎回答说:“我就是。”于是引着武承休进了家门。

武承休见院内有几间破屋,用木岔支着墙壁。进了一间小屋,看到一些虎皮、狼皮悬挂在柱子上,也没有板凳椅子可坐。七郎就地铺虎皮代替座位。武承休和他谈起话来,听他的言语很朴实,非常喜欢他。立即送给他一些银子,让他过日子用。七郎不接受,武承休硬是给他。七郎接过银子去告诉母亲。不一会儿又拿回来还给了武承休,坚决推辞不收。武承休强让了好多次,他还是不收。这时田母老态龙钟地来到,很严厉地说:“老身只有这一个儿子,不想叫他侍奉贵客!”武承休很羞惭地退了出来。

在回家的路上,武承休反复地想来思去,不明白其中的意思。恰好随从的仆人在屋后听到了田母说的话,于是便告诉了他。起初,七郎拿着银子去告知母亲,田母说:“我刚才看见公子,脸上带有晦气纹理,必定要遭奇祸。岂不闻:受人知遇的要分人忧,受人恩惠的要急人难。富人报答人用财,贫人报答人用义。无故得到别人厚赠,不吉利,恐怕是要让你以死相报啊。”武承休听到这些话,深深赞叹田母的贤能,然而也越加倾慕七郎。

第二天,武承休设筵邀请田七郎,七郎推辞不来。武承休便到七郎家,坐在屋里要酒喝。七郎亲自为他斟酒,端上鹿肉干,很尽情礼。过了一天,武承休又邀请答谢他,七郎这才来了。两人亲密融洽,非常高兴。武承休又赠送他银子,七郎就是不收。武承休借口购买他的虎皮,七郎才收下了。七郎回家看了看所存的虎皮,计算了一下,抵不上武承休的银子数,想再猎到虎皮而后献给他。可是进山三天,毫无猎获。又遇上妻子有病,需要看护熬药,也来不及再去打猎。过了十天,妻子忽然病重死去。为了料理祭祀和丧葬,拿回来的银子逐渐花光了。武承休亲自来吊唁送殡,拿来的礼仪很丰厚。葬事处理完了,七郎带上弓箭进了山林,更想猎到虎以报答武承休,然而最终还是一无所获。武承休知道后,就劝他不用急,恳切地希望七郎能来看望他;而七郎始终认为欠武承休的债,感到遗憾,不肯来。武承休于是先向他索要家存的虎皮为借口,好让七郎快点来。七郎查看原先所存的虎皮,已被蠹虫蛀坏,上面的毛也都脱落了,心情愈加懊丧。武承休知道了,骑马来到七郎家里,极力安慰劝解他。又看了看坏了的皮革,说:“这样更好,我所想要的皮,本来就不用毛。”于是卷起皮革拿出门,并邀请他一同前去。七郎不同意,武承体只得自己回家。

七郎想,这样终归不足以报答武承休,便带上干粮进了山。过了几夜猎获了一只虎,把它完整地送给了武承休。武承休大喜,治办了酒筵,请七郎留住三天。七郎推辞得很坚决。武承休锁上了院子的大门,使他无法出去。宾客们见七郎衣着质朴简陋,暗地里都说武公子乱交朋友。而武承休应酬照顾七郎,比对其他的宾客都周到得多。他为七郎换新衣,七郎不接受;只好乘七郎睡觉时偷偷地把衣服换了,七郎没办法只好穿上了。七郎回家以后,他的儿子遵照祖母的吩咐,给武家送回了新衣,并索要父亲的破衣服。武承休笑着说:“回去告诉你祖母,旧衣已拆作鞋衬了。”从此以后,七郎每天都把猎获的兔、鹿赠送给武承休,但武承休请他时,却再也不去了。武承休有一天到七郎家里去,正遇七郎外出打猎还没回来。田母出来,倚着门对他说:“请你不要再来招引我的儿子了,大不怀好意!”武承休恭恭敬敬地向田母行了个礼,很羞惭地走了。

过了半年多,家人忽然告诉武承休说:“田七郎因为与人争夺一只猎豹,殴死人命,被抓进官府里去了。”他听了大惊,骑上马疾驰官府探望,七郎已被带上镣铐收押在狱中了。七郎见到他没有话,只是说:“从此以后麻烦您多周济我的老母。”武承休很凄惨地出来,急忙拿出很多的银子奉送给县令;又拿一百两银子赠送死者的家庭。过了一个多月没有什么事了,七郎才被释放回家。田母感慨地对七郎说:“你的生命是武公子给的了,再不是我所能吝惜得了的。但愿公子能一生平平安安,不遇上灾难,就是儿的福气。”七郎要去感谢武承休,田母说:“去就去罢,见了武公子不要感谢他。要知道小恩可谢,而大恩不可谢。”七郎到了武家,武承休用温暖的话语安慰他,七郎只是恭顺地答应着,家人都怪七郎粗疏,而武承休却喜欢他诚实,愈加厚待他。自这以后,七郎常常在武家一住好几天。赠送他东西就接受,不再推辞,也不说报答。

适逢武承休过生日,这一天宾客仆从非常多,夜间房舍里全住满了人。武承休同七郎睡在一间小屋子里,三个仆人就在床下铺稻草躺卧。二更天将尽的时候,仆人们都已睡着了。他们两人还在不停地谈话。七郎的佩刀原先挂在墙壁上,这时忽然间自己跳出刀鞘好几寸,发出铮铮的响声,光亮闪烁如电。武承休惊起。七郎也起来,问道:“床下躺的都是些什么人?”武承休回答说:“都是些仆人。”七郎说:“其中必定有坏人。”武承休问他是什么缘故。七郎说:“这刀是从外国买回来的,杀人不见血痕,至今已有三代人佩带过它。用它砍了上千个脑袋,仍像新磨过的一样。只要碰见坏人它就鸣叫着跳出刀鞘,此时就离杀人不远了。公子应当亲近君子,疏远小人,也许万一能避免灾祸。”武承休点头同意。七郎始终闷闷不乐,在床席上翻来复去不能入睡。武承休说:“人的祸福是命运罢了,何必这样担忧?”七郎说:“我什么都不怕,只是因为有老母在堂。”武承休说:“怎么竟会到了这种地步!”七郎说:“不出事就好。”原来床下睡着的三个人:一个叫林儿,是个一直受宠的仆人,很得武承休的欢心;一个是僮仆,十二三岁,是武承休平日常使唤的;一个叫李应,最不顺从,好因为小事与公子瞪着眼争执,武承休常生他的气。当夜武承休心里揣摸,怀疑这“坏人”必定是李应。到了早晨,便把李应叫到跟前,好言好语把他辞退了。

武承休的长子武绅,娶了王氏为妻。有一天,武承休外出,留下林儿在家看门。当时武的住处菊花正好开得很鲜艳,新媳妇认为公爹出了门,他的院子里一定不会有人,便自己过去采摘菊花。林儿突然从屋里出来勾引调戏她。王氏想逃避,被林儿强行挟进了屋里。她大声喊叫着抗拒,脸色急变,声音嘶哑。武绅听见跑进来,林儿才撒手逃去。武承休回来听说此事,愤怒地寻找林儿,竟已不知逃到何处。过了两三天,才知道他投奔到某御史家里去了。

这位御史在京城任职,家里的事务都托付他弟弟处理。武承休因为与他有邻里情谊,送书信去索还林儿,而他居然置之不理。武承休愈加愤恨,便告到了县令那里,捕人的公文虽然下了,然而衙役却不去逮捕,县令也不过问。武承休正在愤怒之际,恰好七郎来了。武承休说:“您说的话应验了。”于是把事情的经过告诉了他。七郎听说脸色惨变,始终没说话,径直走了。

武承休嘱咐干练的仆人寻察林儿的行踪。林儿夜里回家的时候,被寻察的仆人抓获,带到了主人面前。武承休拷打了他,他竟出言不逊辱骂主人。武承休的叔叔武恒,本来就是位很厚道的长者,恐怕侄子暴怒会招致祸患,就劝他不如用官法来治办林儿。武承休听从叔叔的吩咐,把林儿绑赴公堂。但是御史家的名帖信函也送到了县衙。县令释放了林儿,交给御史弟弟的管家带走了。这样一来,林儿更加放肆,竟然在人群中扬言,捏造说武家的儿媳和他私通。武承休拿他没有办法,忿恨填胸,气得要死。便骑马奔到御史家门前,指天划地地叫骂。邻人们好歹慰劝着让他回了家。

过了一夜,忽然有家人来报告说:“林儿被人碎割成肉块,扔到野外了。”武承休听了又惊又喜,心情稍微得以舒展。不一会儿又听说御史家告了他和叔叔杀人,于是便和叔叔同赴公堂对质。县令不容他俩辩解,要对武恒动杖刑。武承休高声说:“说我们杀人纯是诬陷!至于说辱骂官宦世家,我确实干过,但与叔叔无关。”县令对他说的话置之不理。武承休怒目圆睁想冲上前去,众差役围上去揪住了他。拿棍杖行刑的差役都是官宦人家的走狗,武恒又年老,签数还没打到一半,就已气绝。县令见武恒已死,也不再追究。武承休一边号哭一边怒骂,县令好像没听见。武承休于是把叔叔抬回了家。他悲愤欲绝,一点办法也没有。想和七郎商议一下,而七郎却一直不来吊唁慰问。他暗自想:对待七郎又不薄,怎么竟如同不相识的路人呢?进而也怀疑杀林儿的人必定是田七郎。但转念一想,果真是这样的话,他为什么事先不来和我商量?于是派人到田家探寻。去了一看,田家锁门闭户寂静无人,邻居们也不知道他们到哪里去了。

有一天,御史的弟弟正在县衙内宅,与县令通融说情。当时正是早晨县衙进柴草和用水的时候,忽然有个打柴的人来到了跟前,放下柴担抽出一把快刀,直奔他俩而来。御史的弟弟惊慌急迫,忙用手去挡刀,被砍断了手腕,接着又被一刀砍掉了脑袋。县令见状大惊,抱头鼠窜而去。打柴人还在那里四顾寻找。差役吏员们急忙关上县衙的大门,拿起木棍大声疾呼。打柴人于是用刀自刎而死。役吏们纷纷凑过来辨认,有认识的知道这打柴人就是田七郎。县令受惊以后镇定下来,这才出来复验现场。见田七郎僵卧在血泊之中,手里仍然握着那把快刀。县令正要停下来仔细察看一下,七郎的僵尸忽地一下跃起,竟然砍下了县令的头,随后才又倒在地上。县衙的官吏派人去抓田七郎的母亲和儿子,但祖孙二人早已逃走好几天了。

武承休听说七郎死了,急忙赶去痛哭,表达哀伤之情。仇人们都说是他指使田七郎杀人。武承休变卖家产贿赂当权的人,才得以幸免。

田七郎的尸体被扔在荒野中过了三十多天,有许多飞禽和狗环围守护着他。武承休把七郎的尸体取走,并且厚葬了他。

田七郎的儿子当时流落到登州一带,改姓了佟。后来当了兵,因为立功升到同知将军。他回到辽阳时,武承休已经八十多岁了,这才领着他找到父亲的坟墓。


\subsection{1.4.7   产 龙}
\label{\detokenize{p00_u5176_u5b83/_u767d_u8bdd_u804a_u658b_u5fd7_u5f02:id139}}
壬戍年间,淄川县邢村李家的媳妇,丈夫死了,她还怀着身孕。孕妇的腹部时常有变化,忽然胀得像瓮一样粗大,突然又缩成了细细的一束。分娩时,过了一天一夜也生不下来。仔细看去,看见个龙头,一见人就又缩了回去。家里人都非常害怕,没有敢靠近的。

有个王老太太,烧上香,迈着作法的步子走来,用手在产妇身上一边往下按一边念着咒语。不多时,胞衣掉下来,没再见到龙;只有几片鳞。都和酒杯一样大。随后生下一个女孩,皮肉透亮得像水晶一样,连脏腑都能看得很清楚。


\subsection{1.4.8   保 住}
\label{\detokenize{p00_u5176_u5b83/_u767d_u8bdd_u804a_u658b_u5fd7_u5f02:id140}}
藩王吴三桂还没有反叛的时候,曾经谕令将士:谁能独自擒获一只老虎,可以享受优等俸禄,并赠送他“打虎将”的称号。将士中有一个人,名叫保住,身体健捷得像猴子那样灵巧。官邸中建高楼,梁木刚刚架起来,他能沿着楼角往上攀登,顷刻之间登到顶颠,站在脊檩上,快速行走,来回三四趟;走完就从上面跳下来,挺直站立。

藩王有个爱姬善弹琵琶,她弹的琵琶是用暖玉做的牙柱,抱着它,整个房子里都会温暖,爱姬当宝贝藏着它,没有藩王的手谕,从不拿出来让人看。一天晚上举行宴会,客人提出想观赏琵琶的奇异,藩王正好懒得走动,答应明天再看。这时候保住在旁边,说:“若不奉大王命令,臣也能将琵琶取来。”藩王先让人速告府中,内外戒备森严,然后才派保住去取。

保住越过十几重院墙,才到达王姬住的院子里。只见室内灯光明亮,而门却紧闭着,无法进入。廊檐下有只鹦鹉栖宿在架子上。保住于是学作猫叫,随后再学作鹦鹉鸣,急呼“猫来了”,作出扑飞的声音并且很急迫。听见王姬说:“绿奴,快去看看,鹦鹉被猫扑杀了!”保住隐藏到暗处。一会儿一个女子挑灯出屋,她的身子刚刚离开门,保住已侧身进入屋内。见琵琶放在桌上,王姬在旁守护着,便直往桌前提起来快步出屋。王姬惊呼:“贼来了!”警卫们听到呼喊声全都冲出来,看见保住抱着琵琶走了,追他已经赶不上了,向他射去的箭像雨点那样密集。保住一跃登到树上。墙下原有大槐树三十多棵,他穿行于树梢上,像鸟飞移于树枝间那样轻巧,树尽登屋,屋尽登楼,飞一样奔向殿阁,不亚于鸟类,转眼间就不知去向了。

客人们正在饮酒。保住抱着琵琶飞落在筵席前,门还像原先那样紧闭着,并未惊动鸡犬鸣叫。


\subsection{1.4.9   公 孙 九 娘}
\label{\detokenize{p00_u5176_u5b83/_u767d_u8bdd_u804a_u658b_u5fd7_u5f02:id141}}
于七失败后,因这桩案件受牵连而被杀的人,以莱阳、栖霞两县为最多。有时,每天搜捕几百人,都被杀在演武场上。鲜血满地,尸骨纵横。有的官员发慈悲,给被杀者捐出一笔钱买棺材。于是,省城棺材铺里的棺材都被购买一空。那些被杀者大都埋葬在城南郊。

康熙十三年,有个莱阳的书生来到济南。他的亲友中,有两三个人也在这里被杀。他买了些纸香祭品之类,来到城南郊累累荒坟之中,祭奠那些死者的魂灵。晚间,就在荒坟旁的一座寺院中。租赁一间房子住下。

第二天,莱阳生因有事进城去了,天很晚还没回来。忽然有一位少年来访,见莱阳生不在寓所,摘下帽子,鞋子也没有脱,就仰躺在床上。仆人问他是谁,那少年闭着眼也不回答。当莱阳生回到寺院时,天已经很晚,夜色朦胧,什么也看不分明。他亲自到床边去问,那少年直瞪着两眼说:“我在等你的主人,你在一边絮絮叨叨追问什么?难道我是盗贼不成!”莱阳生笑着说:“主人就在这里。”少年听了,急忙起身,戴上帽子整整衣服,向莱阳生作揖礼拜,坐下与莱阳生殷勤地道寒暄。听他的口音,好似曾经相识。急喊仆人拿来灯火,一看,原来是同乡好友朱生,他也因于七一案被杀了。莱阳生大吃一惊,不禁向后倒退,转身欲走。朱生向前拉住他,说:“我与你有文字之交,你怎么这样薄情?我虽然做了鬼,但朋友的情分,还是念念不忘的。如今对你有所冒犯,望你不要认为我是鬼就猜疑。”莱阳生坐下,问他有什么话要说。朱生说:“你的外甥女孤身独居,还没有婚配。我很想找个夫人,几次托人去求婚,她总以无长者作主而推辞了。希望能得到你的帮助,把这件事办成。”

原来,莱阳生确有一个外甥女,年幼时就失去了母亲,寄养在莱阳生家。十五岁那年她才回到自己父亲身边,后被官兵捕到济南。她听到父亲惨死的消息,又惊吓又哀痛,不久就死了。

莱阳生听了朱生的请求说:“她有自已的父亲作主,求我干什么?”朱生说:“她父亲的灵柩,被侄儿迁走了,已不在这里。”莱阳生又问:“她过去都依靠谁呢?”朱生说:“与邻居的一位老太太住在一起。”莱阳生私下思虑,活人怎能给鬼做媒?朱生说:“如果蒙您应允,还得请您走一趟。”说完站起来,拉住莱阳生的手。莱阳生坚决推辞说: “到哪里去?”朱生说:“你尽管跟我走就是。”莱阳生只好勉强跟他走了。

向北大约走了一里多路,有一个很大的村庄,全村约有几百户人家。走到一座宅院前,朱生停下叩门。立刻有位老太太出来,敞开两扇门,问朱生有什么事。朱生说:“请您告诉姑娘,她舅舅来了。”老太太进去,不一会又返身出来,邀莱阳生进去,回头对朱生说:“两间屋子太狭窄,有烦公子在门外稍候片刻。”莱阳生跟随老太太进去,见半亩荒院中,有两间小屋。外甥女迎在门口哭泣,莱阳生也哭了。

走进屋里,灯光微弱。只见外甥女容光秀丽,白皙如同生时。她眼泪汪汪地望着舅舅,问家中舅母与姑姑都好?莱阳生说:“大家都好,只是你舅母已去世了。”外甥女听了,又哭起来,说:“孩儿从小受舅舅与舅母的抚养,恩情未能报答一点,没想到自己先被埋葬在沟里,让人感到愤恨。去年,大伯家的哥哥把父亲迁走,把我弃置在这里,毫不挂念。我一人在这几百里外的异乡,孤苦伶仃,像深秋的燕子。舅舅不以我孤苦之魂可弃,又赐我金钱和锦帛,孩儿都收到了。”莱阳生把朱生求婚的事告诉她,外甥女只是低头不语。老太太在一旁说:“朱公子以前曾托杨老太太来过三五次,我也认为这是一门好亲事,可是姑娘自己总是不肯马马虎虎地应下来。今天有舅舅作主,也就满意了。”

说话间,有位十七八岁的姑娘推门进来,后边跟着一个丫鬟。姑娘一眼瞥见莱阳生,转身要走,外甥女拉住她的衣襟说:“不必这佯,是我的舅舅,不是外人。”莱阳生作揖行礼,姑娘也整整衣服还礼。外甥女介绍说:“她叫九娘,姓公孙,栖霞县人。她的爹爹也是世家子弟,后来败落了,眼下也变成了这般穷愁。孤孤单单,事事不称心。我俩很要好,经常往来。”说话间,莱阳生偷眼看九娘,只见她笑时两眉像秋天新月一勾;羞怯时,脸颊像泛起红晕的朝霞,实在是天上的仙人。莱阳生说:“可见是大家闺秀!小户人家的姑娘,哪有这般的仪表风度?”外甥女说:“而且是个女学士,诗词造诣都很高,昨天还给我些指教。”九娘微笑说:“小丫头,无缘无故败坏别人的名声,叫阿舅听了笑话。”外甥女又笑着说:“舅母死了,舅舅还未续娶,这个小娘子,你能满意吗?”九娘笑着跑出去,说:“这丫头犯了疯颠了。”虽然这话是开玩笑、而莱阳生心里对九娘颇有好感。外甥女好像也觉察到了,便说:“九娘的才貌天下无双,舅舅若不以她是地下之鬼为忌讳,我就与她母亲说说。”莱阳生很高兴,但心中老是疑虑人鬼难以婚配。外甥女解释说:“这倒不妨,舅舅与九娘是有缘分的。”莱阳生告辞时,外甥女说:“五天后,月明人静时,我就派人去接你。”

莱阳生出门后,不见朱生。举目四望,下弦的月亮挂在西方天际,在昏暗的月光下,还能辨清来时的道路。只见一座向南的宅子,朱生正坐在台阶上等候。见莱阳生,起身说:“静候你好久了,这就是我的家,请里边稍坐。”于是便拉着莱阳生的手,把他请到屋里,殷切地向他表示谢意。取出一只金杯,一百粒向皇宫进贡的珍珠,说:“没有其它值钱的东西,就以这些作为我的聘礼吧!”又说:“家有薄酒,这是阴间的东西,不足款待嘉宾,很是抱歉。”莱阳生说了几句客气的话,就告辞了。朱生送到半路,两人才分手。

莱阳生回到住所,寺院中的和尚、仆人都来问他。莱阳生隐蹒真情说:“说是鬼,那是胡说,我是到朋友家喝酒去了。”五天后,朱生果然来了。他穿着整齐,手里摇着扇子,像是很满意。走进院子,老远就向莱阳生行礼。片刻,朱生笑着说:“您的婚事已经谈妥了,吉期定在今晚。那就烦您大驾了。”莱阳生说:“因没听到回信,聘礼还未送去,怎么能匆匆举行婚礼呢?”朱生说:“我已代您送过了。”莱阳生很感激,就跟他走了。

两人径直来到朱生住处,外甥女穿着华丽的衣服,含笑迎出门来。莱阳生问:“什么时候过门的?”朱生回答说:“三天了。”莱阳生把朱生所赠送的珍珠,给外甥女作为嫁妆,外甥女再三推辞才收下。外甥女对莱阳生说:“孩儿把舅舅的意思转告了公孙老夫人,她很高兴。但她又说:她已老了,家中没有其他儿女,不愿将九娘远嫁,今晚让你到她家入赘。她家无男子,朱郎陪同你去。”于是朱生领着莱阳生就走了。快到村的尽头,有一家门开着,朱、莱二人进入堂上。片刻,有人传话说:“老夫人到!”但见两个丫鬟搀扶着一位老太太拾阶而上。莱阳生上前欲行叩头大礼,公孙夫人说:“我已老态龙钟,还礼也不便当,这套礼节就免了吧!”她指派着仆人,摆下丰盛的宴席。朱生又叫仆人专给莱阳生另备些酒菜。宴席上所陈列的菜肴,无异于人世间。只是主人自斟自饮,从不劝让客人。一会儿,宴席散了,朱生告辞回去。一小丫鬟为莱阳生引路。进入洞房,只见红烛高照,九娘身着华丽服装,凝神在等待着。两人相逢,情谊深长,极尽人世间亲昵之情。

当初,九娘母子被俘,原准备押送到京城。至济南,其母难忍虐待之苦,就死了。九娘在悲愤中也自杀身亡。九娘与莱阳生在枕席上谈起往事,哭泣得不能入睡,便吟成两首绝句:“昔日罗裳化作尘,空将业果恨前身。十年露冷枫林月,此夜初逢画阁春。”白杨风雨绕孤坟,谁想阳台更作云?忽启缕金箱里看,血腥犹染旧罗裙。”天将亮,九娘敦促莱阳生说:“你应离开这里了,注意不要惊动仆人。”自这以后,莱阳生天未黑就来,天刚放亮就走,两人恩爱情深。

一天夜里,莱阳生问九娘:“这个村庄叫什么名字?”九娘说:“叫莱霞里。因这里多是刚埋葬的莱阳、栖霞两县的新鬼,就起了这个名字。”莱阳生听后,感叹欷歔。九娘悲哀地说: “我这千里之外的一缕幽魂,漂零于蓬蒿无底的深渊,母子二人孤苦伶仃,说起来叫人伤心。望你能念夫妻之恩,收拾我的尸骨,迁葬回你祖上的坟地,使我百年之后也有个依托,那我就死而无恨了。”莱阳生应允了。九娘说:“人与鬼不是一条路,你不宜于长久在这里滞留。”她取出一双罗袜赠给莱阳生,挥泪催促他离开。莱阳生恋恋地凄然地走出来,心中忧伤,失魂落魄,惆怅不安,不忍归去。路经朱生门前,就敲朱生的门,朱生赤脚出来,迎着莱阳生。外甥女也起来了,头发蓬松,吃惊地问是怎么回事。莱阳生惆怅一会儿,把九娘的话说了一遍。听罢,外甥女说:“就是舅母不说这话,我也日夜在思虑这件事。这里并非人世间,久居的确是不妥当的。”于是,大家相对哭泣,莱阳生含泪而别。

回到寓所,莱阳生翻来复去,直到天亮也未能睡着。欲去找九娘的坟墓。但走时又忘记问墓的标记。到天黑再去时,只见荒坟累累,蓬蒿满目,竟迷失了去莱霞里的路,只得哀叹返回。打开九娘所赠的罗袜,罗袜见风便粉碎了,像烧过的纸灰一样。于是,莱阳生就整装东归。

半年后,莱阳生心中始终不能忘怀这件事,又来到济南,希望能再有遇到九娘的机会。当他到了南郊,天色已晚。他把马车停放在寺院的树下,就急忙到丛丛坟地中去。只见荒坟累累,千百相连,荆棘荒草迷目,闪闪的鬼火与阴森可怖的狐鸣,使人惊心失魄。莱阳生怀着惊恐的心情回到寓所。

这次济南的游兴完全消失了,他马上返程东归。行至一里许,远远见一女郎,独自在高高低低的坟墓间行走。从体态神情上看,很像是九娘。莱阳生挥鞭赶上去,一看,果然是九娘。莱阳生跳下马想与她说话,女郎竟然走开了,好像从来就不相识。莱阳生再赶上去,女郎面有怒色,举袖遮住自己的脸。莱阳生连呼:“九娘!九娘!”女郎竟如轻烟,飘飘然消失了。


\subsection{1.4.10   促 织}
\label{\detokenize{p00_u5176_u5b83/_u767d_u8bdd_u804a_u658b_u5fd7_u5f02:id142}}
明宣德年间,皇宫中流行斗蟋蟀的蝣戏,每年都要向民间征收大量蟋蟀。蟋蟀本不是陕西特产,有个华阴县令,为了讨好上官,奉上一只蟋蟀。让它试斗了一番,却非常厉害,于是上官就责令华阴县每年供奉。县令又把这差事交给了里正。集市上那些游手好闲的人,每得到一只好的蟋蟀,便用笼子养着,抬高价格,当作奇货高价出售。乡里的公差狡猾奸诈,常借此按人口摊派费用;每征一头蟋蟀,常要好几户人家倾家荡产。

县里有个叫成名的,是个童生,好久考不中秀才。成名为人老实憨厚,不善谈吐,因此被刁滑的小吏报到县里,让他担任里正,他想尽了办法也推脱不掉。不到一年,家中那点微薄的家产就折腾光了,这一年,正遇上皇宫征收蟋蟀,成名不敢勒索百姓,自已又没钱赔偿,忧愁烦闷得要死。妻子说:“死了有什么益处?不如自己去捉捉看,说不定还有希望得到一只。”成名认为很对,于是早出晚归,提着竹筒、丝笼,在破墙下草丛中,搬石挖穴,什么办法都用了,始终没有捉到一只可以进贡的。即使捕到两三头,也是又弱又小,不够规格。县令限期追逼,只十多天,成名就挨了一百大板,两条腿被打得脓血淋漓,连蟋蟀也不能去捉了;天天躺在床上,翻来复去,只想自尽。

这时,村中来了一个驼背巫婆,能假借鬼神算卦,非常灵验。成名的妻子带着钱去问卦,见红妆少女和白发婆婆挤满了门口。走进巫婆的屋里,有间密室,挂着帘子,帘子外摆放着几案。问卦的人,先在香炉中燃上香,连拜两拜。巫婆在一边望着天空代她们祈祷,嘴唇一张一合,不知说些什么。求卦的人恭恭敬敬地站在那里听着,不多时,帘里扔出一张纸,上面写着求卦人想问的事情,没有丝毫差错。成名的妻子把钱放在香案上,像前面的人那样点香跪拜。有一顿饭功夫,帘子动了一下,一张纸片抛落出来。她忙拾起来一看,纸上不是字而是画。上面画着殿堂楼阁,像是座佛寺;寺后面的小山下,到处是奇形怪状的石头和一丛丛的荆棘,一只青麻头蟋蟀藏在那里,旁边有只蛤蟆,像要跳起来的样子。成名的妻子反复观看,不懂是什么意思。但见画上有蟋蟀,隐隐说中心事,便将纸片摺藏起来,带回家给成名看。

成名看着画反复思索,莫不是指给我捉蟋蟀的地方吗?仔细察看了画上的景物,与村东的大佛寺很相似。于是他勉强起身,拄着拐杖,拿着图画来到村东大佛寺的后面。见在茂密的草丛中有一座古坟,成名沿着坟往前走,只见层层乱石,跟鱼鳞一样,和画中的很相像。成名便在蓬蒿野草中,一边侧身细听,一边慢慢走着,像在寻找细小的针,芥。直找到眼花耳聋,还是没一点蟋蟀的踪迹。他正在凝神搜寻着,突然一只癞蛤蟆跳了出来。成名很惊愕,急忙追赶过去,蛤蟆已钻进草丛中。他拨开草丛,仔细寻找,见一只蟋蟀趴在棘根旁,急忙用手一扑,蟋蟀钻进石洞中。成名用草尖拨弄,拨不出来;又用竹筒里的水灌它,蟋蟀才出来。见这只蟋蟀身躯健壮,体态俊美。成名捉住它仔细审看,个头很大,尾巴修长,青脖子金翅膀。成名非常高兴,忙装进笼子提回家中,全家人欢庆祝贺,把它看得比价值连城的宝玉还要珍贵。用盆子养起来,喂它好东西,爱护备至,只等到了期限,送到县里去交差。

成名有个儿子,才九岁,看到父亲不在家,偷偷打开盆盖去看。蟋蟀一下从盆里蹦了出来,快得没法捕捉。等把它扑到手中,蟋蟀腿掉了,肚子也裂开了,一会儿便死了。孩子害怕了,哭着告诉了母亲。母亲一听,吓得面如死灰,大骂道:“祸根!你的死期到了!等你父亲回来,会同你算帐的!”孩子大哭着走了出去。

不一会儿,成名回来,听了妻子的诉说,像被冰雪浇透了,怒气冲冲地寻找儿子,可儿子不知到哪里去了。后来,从井里打捞上来了孩子的尸体,成名夫妻顿时转怒为悲,呼天喊地,哭得要死。夫妻两人相对发呆,饭也不做,只是默默地坐着,不再感到有一点活着的乐趣。天快黑了,才拿上草席想把孩子葬了。近前抚摸儿子的身体,发现有微弱的气息,夫妻二人欢喜地把儿子放到床上。到了半夜,儿子苏醒了,夫妻二人心中稍感到宽慰。但一看到蟋蟀的笼子空空的,又气得说不出话来;又不敢再去追究儿子,从黄昏到天亮,连眼睛也没合一下。

东方的太阳已经升起来,成名仍直挺挺地躺在床上发愁。忽然听到门外有蟋蟀的叫声,成名惊讶地起来察看,见那只蟋蟀仿佛还活着。成名高兴地捕捉它,蟋蟀一叫便跳开了,跳得还非常快。成名用手掌盖住它,感到掌心里空空的没什么东西;刚一抬手,蟋蟀又远远地跳开了。成名急忙追赶,转过墙角。蟋蟀不知钻到哪里去了。成名来回四下寻找,见蟋蟀趴在墙壁上。仔细一看,身躯短小,黑红色,不是先前那只。成名嫌它小,不捉,只是来回察看,寻找刚才追的那只。墙壁上的小蟋蟀忽然跳到了成名的衣襟上,成名再细一看,形状像蝼蛄,长着梅花样翅膀,方头长脖子,像是好品种,这才欢喜地把它捉起来。将要献给官府时,又惴惴不安,恐怕不中官府意,便想让它试斗一番看看。

村中有个好事的少年,驯养了一只蟋蟀,自己给它起名叫“蟹壳青”,天天同一些少年角斗,没有一次不取胜的。他想靠这只蟋蟀发财,便抬高价钱,却没有买的。这天,这少年登门找成名,看到成名养的那只小蟋蟀,忍不住捂着嘴笑起来,便拿出自己的蟋蟀,放进笼子里较量。成名见他养的蟋蟀,个头大,身子修长,心中很羞愧,不敢和他的较量。少年再三强求,成名想:养一只劣等蟋蟀也没什么用,不如让它拚一次,博众人一笑。便把两只蟋蟀都放到一个盆里,让它们角斗。成名的那只小蟋蟀趴在那里一动不动,蠢若木鸡,少年又大笑起来。他用猪鬃撩拨小蟋蟀的须子,一次又一次,小蟋蟀突然发怒了,直冲过去,接着就互相搏斗起来,跳跃腾击,振翅有声。一会儿,小蟋蟀一跃而起,直扑对手去咬它的脖子,少年大吃一惊,急忙把它们分开,停止了搏斗。小蟋蟀振起双翅,骄傲地鸣叫着,好像是报告主人知道。成名高兴极了,正在赏玩,突然过来一只鸡,径直去啄那只蟋蟀。成名惊骇地站在那里呼喊,幸好没被啄中,小蟋蟀跳出去有一两尺远。鸡又大步追上去,小蟋蟀已经落在鸡爪下了。成名惊慌失措,不知怎么救它,急得直跺脚,脸色都变了。转眼间,见鸡伸着脖子扑楞着,走近一看,原来小蟋蟀趴在鸡冠子上用力叮着不放松。成名更加惊喜,忙把小蟋蟀捧放到笼子里。

第二天,成名把小蟋蟀献到县官那儿。县令见蟋蟀太小,愤怒地呵斥成名。成名讲述了它的奇异,县令不相信,就试着让它同别的蟋蟀斗了斗,结果所有的蟋蟀都被斗败了;又让它同鸡斗,果然同成名说的一样。县令赏了成名,把这只蟋蟀献给巡抚。巡抚非常高兴,用金笼盛着进献给皇上,并在奏章中详细讲述了蟋蟀的本领。小蟋蟀入宫后,将天下进贡的蝴蝶,螳螂、油利达、青丛额等各种稀奇的蟋蟀都斗了一遍,没有超过它的。小蟋蟀每当听到琴瑟的声音,就按着节拍舞蹈,人们越发觉得它奇特。皇上非常高兴,下诏赏赐巡抚名马和衣缎。巡抚没有忘记这荣幸是从哪来的,没过多久,县令就因政绩优异被擢升。县令也高兴了,就免去了成名的差役,又嘱咐学使,让成名进了县学。

后来过了一年多,成名的儿子精神复原了,自己说身子变成了蟋蟀,轻捷善斗,现在才苏醒过来。巡抚也重赏了成名。不几年,成名便有田百顷,楼阁无数,牛羊满圈。一出门便穿着裘皮衣服,骑高头大马,富贵赛过了官宦世家。


\subsection{1.4.11   柳 秀 才}
\label{\detokenize{p00_u5176_u5b83/_u767d_u8bdd_u804a_u658b_u5fd7_u5f02:id143}}
明朝末年,青、兖二州发生蝗灾,并渐渐地蔓延到沂县。沂县的县令对此很担忧。退堂后睡卧在邸舍中,梦见一位秀才来拜见。秀才头戴高冠,身穿绿衣,长得非常魁梧,自称有抵挡蝗灾的办法。问他有什么办法,秀才回答说:“明日在西南道上,有个妇人骑着一头大肚子母驴,她就是蝗神。哀求她,可以免灾。”

县令感到这个梦很奇怪,就操办好酒食带到了城南。等了很长时间,果然有个梳着高高的发髻、身披褐色斗蓬的妇女,独自一人骑着一头老驴,缓慢地往北走着。县令立即点燃香,捧着酒杯,迎上去拜见,并捉住驴子不让走。妇人问:“您想干什么?”县令便哀求道:“区区小县,希望能得到您的怜悯,逃脱蝗口!”妇人说:“可恨柳秀才多嘴,泄露我的机密!立即让他身受蝗害,不损害庄稼就是了。”于是饮酒三杯,转眼间不见了。

过后蝗虫飞来,遮天蔽日,但是不落在庄稼地,只是集中在杨柳树上,蝗虫经过的地方,柳叶全被吃光了。县令这才明白梦中的秀才就是柳神。有人说:“这是县官忧民所感动的。”确实如此。


\subsection{1.4.12   水 灾}
\label{\detokenize{p00_u5176_u5b83/_u767d_u8bdd_u804a_u658b_u5fd7_u5f02:id144}}
康熙二十一年,山东大旱。自春至夏,光秃秃的地上不长青草。六月十三日下了一场小雨,才开始有种谷子的。到十八日,大雨充足,于是种豆。

有一天,石门庄有位老人,傍晚看见两头牛在山上相斗,就对村里的人说:“大水将要到了!”随即携带家人迁走了。村里的人都嘲笑他。不久,暴雨如注。彻夜不停,平地水深好几尺,房子全都淹没了。一个农人舍弃两个儿子不顾,先和妻子搀扶着老母亲,跑到一个高岗上躲避。再往下一看,整个村子一片汪洋,已成水国,也就无法顾及自己的两个儿子了。等到大水退落回到家里,见全村都成了废墟和坟墓。进自己家门一看,竟然还有一间屋留存下来,两个儿子并排坐在床头上,正玩耍嬉笑,安然无恙。人们都说这是他们夫妻二人行孝的好报。这是六月二十二日发生的事。

康熙二十四年,山西平阳大地震,死了的人数占十分之七八。整个城市内外都成了废墟,仅剩下的一间屋,原来是某孝子的家。茫茫大劫中,惟独孝子的后代安然无恙,谁说老天爷不分青红皂白呢?


\subsection{1.4.13   诸 城 某 甲}
\label{\detokenize{p00_u5176_u5b83/_u767d_u8bdd_u804a_u658b_u5fd7_u5f02:id145}}
淄川县教谕孙景夏先生曾说:他们县的某甲,遇上流寇作乱,被杀,头坠在胸前。流寇退去,家里的人得到了他的尸体,将要抬去埋葬。忽然听见他有微弱的喘气声音。仔细一看,他的咽喉处竟还有一指多宽没断下来。于是扶着他的头,把他扛回家。过了一天一夜他开始呻吟,用勺子和筷子稍微喂他点饮食,半年后竟然痊愈了。

又过了十几年,某甲和两三个人聚会交谈,其中有个人说了句笑话,引得哄堂大笑。某甲也兴奋地鼓掌。不料想他一俯仰之间,原来的刀痕突然破裂,头掉了下来,鲜血直流。大家看他时,已经气绝身死了。某甲的父亲告了那个说笑话的人。众人敛钱安抚他,又安葬了某甲,于是才和解了。


\subsection{1.4.14   库 官}
\label{\detokenize{p00_u5176_u5b83/_u767d_u8bdd_u804a_u658b_u5fd7_u5f02:id146}}
山东邹平的张华东公,奉皇帝之命去祭祀南岳衡山。路经江淮地区,需要在这里的驿站住宿。前驱官禀报道:“这个驿站中有妖异作怪,在里面住宿一定会出乱子。”张公不听。

到了半夜,张公穿戴齐整佩剑而坐。一会儿,听到有靴子走路的声音进来了,原来是一个须发花白的老头,戴着黑帽,扎着黑带。张公很奇怪,便问他的来历。老头叩拜说:“我是库官,为您管理库存财物已经很长时间了。幸遇钦差大人远道来临,下官也好卸去这个沉重的负担了。”张公问:“库存多少?”老头回答说:“二万三千五百两银子。”张公怕这么多钱带着路上累赘,便约好回来时再与他查点验收。老头答应着退下。

张公到了南中地带,得到的馈赠非常丰厚。等到归来时,还是住宿在原来的驿站,老头又来拜见他。当问到库存钱财时,老头答道:“已经拨充辽东兵饷了。”张公对他前后不一致的说法深感惊讶。老头说:“人生命中注定的收入,都有定数,分毫不能增减。大人这次出行,应得的钱财都已如数得到了,还求什么呢?”说完,就走了。

张公于是计算他这次所获得的钱财,竟与老头所说的库存数字正相符合。他这才慨叹一餐一饭皆有命定,不可任意强求啊。


\subsection{1.4.15   酆 都 御 史}
\label{\detokenize{p00_u5176_u5b83/_u767d_u8bdd_u804a_u658b_u5fd7_u5f02:id147}}
四川酆都县城外有个山洞,深不可测,相传是阎罗天子的衙门。它里面的一切刑具,都是借助人来制造的。脚镣和手铐坏了,就扔在洞口,县令马上用新的替换,过一夜就不知去向了。有关洞内的供应开支,已经载入官府的报销制度中。

明代有个御史行台华公,巡视酆都时,听到这个传说,不相信是真的,想进洞去破除这个疑惑。人们都说不行,但华公不听。他手持蜡烛进入洞中,让两个衙役在后随从。深入洞内一里多路后,蜡烛突然灭了。华公看了看,台阶宽阔明朗,有大殿十余间,里面并排坐着尊官,身穿袍服手执笏板很庄重,惟独东头空着一个座位。官员们见华公到了,都走下台阶来迎接,笑着问道:“来了吗?分别以后可好啊?。”华公问:“这是什么地方?”尊官说:“这是阴曹地府。”华公惊讶地告退。尊官指着空座位说:“这是您的座位,哪能再回去?”华公更加害怕,一再请求宽容。尊官说:“定数怎么可以逃脱呀!”于是检出一卷簿子给他看,上面记载着:“某月某日,某以肉身归阴。”华公看了,吓得浑身颤抖,像掉入冰水中。念及母老子幼,流下了眼泪。不一会有个穿着金甲的神人,手捧黄色帛书来到。群官拜舞,打开帛书读后,祝贺华公说:“您有回阳间的机会了。”华公惊喜地问原因。尊官说:“刚才接到大帝御诏,要大赦幽冥,可以为您设法折免罪过。”于是为华公指示道路让他出来。几步以外,幽黑如漆,辨认不出道路。华公非常为难。

忽然一位神将气宇轩昂地走来,红脸长须,光芒射出数尺以外。华公迎拜并哀求他,神人说:“背诵佛经可以出去。”说毕去了。华公心想,经咒大多不能记忆,只有金刚经还曾稍微学习过,于是合掌背诵。立刻觉得有一线光明,映照着眼前的路。忽然有遗忘了的句子,眼前立即黑暗;镇静下来思考一会儿,再背诵再显光明,这才出得洞来。而那两个随从的衙役,就不必再问了。


\subsection{1.4.16   龙 无 目}
\label{\detokenize{p00_u5176_u5b83/_u767d_u8bdd_u804a_u658b_u5fd7_u5f02:id148}}
山东沂水县下大雨时,忽然从天上掉下一条龙来,两只眼睛全没了,还有微弱的气息。县令大人用了八十张芦席来盖它,都没能盖严整个龙身。又为它摆设野祭。龙仍然反复地用尾巴击打地面,发出非常大的声响。


\subsection{1.4.17   狐 谐}
\label{\detokenize{p00_u5176_u5b83/_u767d_u8bdd_u804a_u658b_u5fd7_u5f02:id149}}
万福,字子祥,是博兴县人,少年时就喜读诗书。家里很有些财产,但命运不好,二十多岁了,还考不上个秀才。他家乡有种旧习,官府派下公差徭役,往往都摊给那些富裕人家,忠厚老实的人常常为此倾家荡产。万福正好被报上充劳役,他害怕,就逃走了。

万福跑到济南,在旅店里租了间房子住下。夜晚,有个女子私奔了来,十分美丽。万福很喜欢,就留住了她。问她的姓名,女子说:“我是狐女,但不会祸害你!”万福因喜欢她而丝毫不怀疑。女子嘱咐他不要跟别的客人一起住,于是每天都来与万共寝。凡日用东西,无不仰仗狐女供给。时间不长,万福的几个朋友常来找他聚会,往往一坐就是一通宵。万福很厌烦,但又不好意思拒绝,只得跟客人讲了实话。客人听说,便要见见狐女。万福对狐女说了。狐女对客人说:“见我干什么?我也不过是个人罢了!” 听狐女的声音,像在眼前,四下一看,却不见人影。

客人中有个叫孙得言的,爱开玩笑,非要见见狐女,还说:“听见这娇滴滴的声音,叫我神魂颠倒!为什么要吝惜你的花容月貌,让人光听声音害相思呢?”狐女笑着骂道:“好个贤孙!想为你老祖母画一幅行乐图吗?”客人听了都笑起来。狐女又说:“我是狐,就为客人们说一个狐的典故。你们愿听吗?”大家忙表示愿听。狐女讲道:“从前,某村有个旅店,有很多狐狸,经常出来迷惑旅客。客人们知道后,都互相告戒不要在这家旅店住宿。半年来,旅店门前冷落,店主人非常担忧,十分忌讳说‘狐狸’。一天,忽然有个远方来客,自称是外国人,看见旅店,便进去要住宿。店主人大为高兴。来客刚进门,便有个路人暗暗告诉他:‘这家有狐狸!’来客害怕,忙告诉主人要搬走。主人极力辩白店里没狐,来客才住下来。进入房间刚刚躺下,见一群老鼠从床下钻了出来,来客大吃一惊,急忙跑出屋子,高声大叫:‘有狐!’店主人惊问,来客说:‘狐狸的老窝在这里,你怎么骗我说没有?’主人又问:‘你刚才看见的狐狸是什么样子?’来客说:‘我刚才看见的,又细又小,不是狐狸儿子,就是狐狸孙子!’”讲完,满座人都哈哈大笑。孙得言说:“既然不愿意让我们见见仙容,我们今晚就住在这里,不走了,你们俩也别想睡觉!”狐女笑着说:“在这里借住不要紧,倘若我小有冒犯之处,请不要放在心上!”众人恐怕她恶作剧,只得一起走了。但此后,几天就来一次,来了就找狐女互相笑骂。狐女十分诙谐,每说一句话,无不使客人笑得前仰后台,再滑稽的人也难不倒她。大家戏称她“狐娘子”。

一天,朋友们在一起宴会。万福坐在主人位上,孙得言和另外两位客人分坐左右,上边摆一坐榻,让狐女坐。狐女推辞说不会喝酒,大家异口同声地请她坐下说话,狐女答应了。酒过数巡,众人掷骰子,行“瓜蔓”酒令。其中一个客人犯令受罚,应该喝酒,便开玩笑地将酒杯推到上坐说:“狐娘子还很清醒,请代喝一杯!”狐女笑着说:“我不会喝!愿意讲一个故事,给大家下酒!”孙得言忙捂起耳朵,连说不听。客人都说:“谁骂人,就罚谁喝酒!”狐女笑说:“我骂狐,可以吗?”大家说:“行!”于是都竖起耳朵,听她讲。狐女讲道:“从前,有个大臣,出使红毛国。这个大臣戴一顶狐皮帽子去见国王。国王见了帽子很惊奇,问:‘这是什么皮?皮毛这样厚实温暖。’大臣告诉他是狐皮。国王说:‘这种东西,我生平从没听说过。那狐字怎么写?’大臣在空中用手比划着说:‘右边是一大瓜,左边是一小犬!’”在座的人哄堂大笑。客人中有弟兄两个,一个叫陈所见,一个叫陈所闻,此时见孙得言十分窘迫,便说:“那雄狐哪里去了?任雌狐在这里放毒!”狐女接着说:“刚才的故事还没讲完,就让群狗的乱叫声给打断了。请让我讲完它。国王见大臣骑着骡子,非常奇怪。大臣告诉他说:‘这是马生的。’国王更加惊奇。大臣说:‘在中国,马生骡子,骡生驹驹。’国王又详细询问。大臣说:‘马生骡,是臣所见;骡生驹驹,是臣所闻。’”全座的人又大笑起来。大家知道开玩笑敌不过她,便约定:谁再开玩笑骂人,罚做东道主,请大家喝酒。又过了一会儿,大家酒兴更浓。孙得言又戏弄万福说:“我有一联,请你对下联。”万福问:“什么联?”孙得言说:“这一联是:妓女出门访情人,来时‘万福’,去时‘万福’。”一座的人都冥思苦想,对不上。狐女忽然笑着说:“我对上了!”大家忙都听着。狐女念道: “龙王下诏求直谏,鳖也‘得言’,龟也‘得言’。”众人拍手叫绝。孙得言大为恼怒,说:“刚才已和你约好,为什么又犯戒?”狐女笑道:“真是我错了!但除了这一句对不上你的上联。明天我一定设宴请大家,以赎我的罪过!”众人一笑作罢。狐女的诙谐,如此这般,一时也说不完。

连住了几个月,狐女便跟万福一同返回。到了博兴县界,狐女告诉万福说:“这里有我的一家远亲,很长时间没来往了。这次路过,不可不去看看。天要黑了,我们正好去借住一晚,明天一早走吧。”万福问在哪里,狐女往前一指,说:“不远。”万福怀疑前面本来没有村庄,姑且跟着她走。走了二里多路,果然看见一处村落,以前从没见过。狐女敲敲门,一个老仆人答应着出来开了门。进入院子,只见楼阁重重,一派富贵大家的气象。不一会儿,主人迎出来,一个老翁、一个老太太,见过礼请万福坐下。摆上丰盛的酒宴,把万福当作新女婿般款待。饭后,二人住了一晚。狐女第二天早早起来,对万福说:“我匆匆忙忙地跟你回家,恐怕你家里人会感到意外和惊怪。你先回去说一声,我随后就到。”万福答应,先回了家,告诉了家人。不久,狐女果然来了。跟万福谈笑时,家里的人光听见声音,看不见人在哪里。

过了一年,万福又有事到济南去,狐女也跟随着。忽然来了几个人,狐女跟他们打招呼,问寒道暖,十分亲热。又对万福说:“我本是陕西人,因为和你有缘分,所以跟了你这么长时间。现在我的兄弟们来了,我要跟他们回去,不能再伺候你了!”万福百般挽留,狐女竟自走了。


\subsection{1.4.18   雨 钱}
\label{\detokenize{p00_u5176_u5b83/_u767d_u8bdd_u804a_u658b_u5fd7_u5f02:id150}}
滨州有一个秀才,在书房读书。听到有人敲门,开门一看,原来是一个须发全白的老翁,相貌穿着很古怪。秀才将老翁请进房,问他的姓名。老翁说:“我叫胡养真,是个狐仙,因爱慕你的高雅品行,愿与你朝夕相处。”秀才胸怀宽广,也不当作怪事,就和他评论起古往今来的事。老翁知识渊博,话语生动,谈吐不凡。有时谈论经书的涵意,他所说的道理极为深奥,尤其使人觉得出乎意外。秀才十分敬服,留他住了很长时间。

一天,秀才偷偷乞求老翁说:“你对我的感情这样深,你看我这样贫穷,只要你一举手,金钱马上就能得来,能不能稍微周济我一点呢?”老翁沉默了一会,似乎不同意。过了一会儿,老翁笑着说:“这太容易了,但要有十几个钱作母才行。”秀才就按他说的办了。二人一起进入密室中,老翁迈着巫师道士的步子,念起咒语。顷刻之间,只见有百余万的铜钱,从梁上锵锵落了下来,像下暴雨一样,一瞬间便没了膝盖。拔出脚来,又没了踝骨,丈多宽的房间里,铜钱巳深约三四尺了。老翁这才看着秀才说:“能满足你的愿望了吧?”秀才说:“满足了!”老翁一挥手,铜钱立刻不掉了。两人出来锁好门,秀才暗自高兴,以为发大财了。过一会儿,秀才进屋取钱用,却见刚才满满一屋钱全没了,只有他那十几枚铜钱还在。秀才很失望,就对老翁发火,埋怨老翁欺骗他。老翁生气地说:“我和你只作文字朋友,不打算替你作贼!如要满你的意,你就该去找盗贼交朋友,老夫不能从命!”接着就一甩袖子走了。


\subsection{1.4.19   妾 击 贼}
\label{\detokenize{p00_u5176_u5b83/_u767d_u8bdd_u804a_u658b_u5fd7_u5f02:id151}}
益都西部边境的某人出自富贵人家,家里有很多钱。他纳了一个妾,很美。大老婆凌辱折磨她,横加鞭挞,但妾侍奉大老婆仍然十分小心周到。这人对她很同情,往往在背地里用好话安慰她,她也未曾有过什么怨言。

有天夜里,几十个贼人越墙进院,用力冲撞屋门,几乎要撞坏了。这人和妻子吓得丧魂落魄,浑身颤抖,不知如何是好。妾听到动静起来,默不作声,暗中在屋内摸索,抓到一根挑水用的担杖,拨开门栓冲出。贼人慌乱如麻,妾挥舞担杖,风鸣钩响,打得四五个人趴在地上;贼人全都溃败,惊恐逃窜,急得爬不上墙,跌下来咿呀乱叫,一个个丧魂失魄狼狈不堪。妾手拄担杖,看着他们笑着说:“你们这群东西,真不值得我下手打!竟然也还学着作贼!我不杀你们的,杀了还嫌辱没了我呢!”说完全放他们逃去。丈夫大惊,问道:“你怎么会有这么大的本事?”原来妾的父亲过去是枪棒教师,她得到父亲传授的全部武艺,不止能抵挡百人。

大妻尤其害怕,非常后悔从前没能看清妾的本领,从此便好好地看待她。而妾始终也没有丝毫失礼的地方。邻家妇女有的对妾说:“嫂子击贼好像打猪狗那样容易,你为什么还甘心低头受棍棒鞭打的痛苦呢?”妾说:“这是我分内应该的,还敢说别的吗。”听的人更加佩服她的贤良。


\subsection{1.4.20   驱 怪}
\label{\detokenize{p00_u5176_u5b83/_u767d_u8bdd_u804a_u658b_u5fd7_u5f02:id152}}
长山县的徐远公,是明朝的秀才。明朝覆灭后,他放弃了考取功名的志向,一心访道求仙,慢慢学会了一些驱怪的法术,远远近近的人大都听说过他的名字。某县有一个大富翁,这天写了一封诚恳邀请他的书信,派人带着钱牵着马去接他。徐远公问:“你家主人召我去有什么事吗?”仆人推辞不知,说:“主人只是嘱咐小人务必请您屈驾光临。”徐远公就跟着他走了。

徐远公来到主人家中,主人已在院子里摆好了宴席,非常恭敬地招待他;但是始终不说为什么请他来。徐远公忍耐不住,就问主人说:“你到底让我来干什么?早点告诉我,解除我心中的疑团。”主人总说没什么事,只是劝他喝酒,说话吞吞吐吐,让人没法理解。说话之间,天不知不觉黑下来,主人便邀请徐远公到花园中饮酒。这座花园构造非常精巧,但被竹、树遮蔽,显得阴森森的,丛丛的杂花,大半隐没在杂草中。来到一座亭阁,只见阁顶盖板上蛛网密布,大大小小,上上下下,杂乱得数不过来。酒又过数巡,天色慢慢黑了,主人让掌上灯再饮。徐远公推辞不能再喝了,主人便命撤酒上茶。仆人们匆匆忙忙地撤掉酒具、菜盘,全部堆放在左边一间屋子的桌案上。茶喝了不到一半,主人借故竟自走了。仆人便端着蜡烛引着徐远公去左边屋子里住宿。他一进屋,仆人把蜡烛放到桌几上,急忙返身走了,显得慌慌张张。徐远公以为是去拿被褥来同他作伴,可等了很久,一点动静都没有了。他只好自己起来关上门睡了。窗外皎洁的月光,透入室内照到床上,夜鸟秋虫啾啾唧唧地叫着,让他心中忧闷,睡不着觉。

过了一会,徐远公听到阁板上发出橐橐的声音,好像是脚步声。声音很响,一会儿到了楼梯,一会儿又靠近他睡觉的房门。徐远公害怕万分,毛发倒竖,急忙用被子盖上头。这时屋门豁然开了,徐远公偷偷掀开被角一看,见是一个怪物。兽头人身,浑身长满像马鬃一样的毛,呈深黑色;尖尖的牙齿白森森的,两眼像灯笼一样闪闪发光。到案桌前,低头舔吃盘中的剩菜。舌头一舔,一连几个盘子便被舔得干干净净。接着走近床前,嗅徐远公的被子。徐远公猛然起身,用被子蒙住怪兽的头,按住它狂喊起来。怪兽出乎意外,吃惊地挣脱开,开了外边的门逃窜了。徐远公披上衣服逃了出来,只见园门从外边锁上了,出不去。只好沿墙奔逃,从一处低矮地方爬出去,正好是主人的马厩。喂马的人吃了一惊,徐远公告诉他事情的经过,请求在马厩里留宿。

天快亮时,主人叫人去看徐远公。一看徐远公不在,大吃一惊。后来在马厩里找到他。徐远公从马厩里出来,非常生气,怒冲冲地说:“我本不熟悉驱怪的法术,你叫我来,又不说一句实话;我口袋里装有一支如意钩,又不给我送来,这是要置我于死地!”主人谢罪说:“本打算把实情告诉你,怕你为难。我们也不知你口袋里藏着如意钩,请免我死罪。”徐远公始终闷闷不乐,要了一匹马骑着回家了。从此怪兽也绝迹了。主人每在园中设宴时,总是笑着向客人说:“我忘不了徐先生的功劳啊。”


\subsection{1.4.21   姊 妹 易 嫁}
\label{\detokenize{p00_u5176_u5b83/_u767d_u8bdd_u804a_u658b_u5fd7_u5f02:id153}}
掖县有个当宰相的毛公,原先家中门第低微,生活贫寒,他的父亲常常给别人放牛。当时,县城有个世代为官的姓张的人家,在东山南面有块新坟地。有人从旁边经过,听到墓中有怒骂声:“你们赶快躲开,不要总在这里玷污贵人的宅地。”姓张的听说这事,不太相信。接着又连连在梦中得到警告,说:“你家的新坟地,本是毛公的墓地,你为什么长久占据在这里?”从此,张家时常有不吉利的事发生。别人劝他还是把坟迁走好,姓张的听从劝告,把坟迁走了。

一天,毛公的父亲出去放牛,走到张家原先的坟地,天突然下起大雨,就跑到废弃的墓穴里避雨。雨越下越大,滔滔雨水,冲进墓穴,把墓灌满了,毛公的父亲被淹死在里面。当时毛公还是个孩童。母亲独自去见张姓的,乞求给一小块地方掩埋毛公的父亲。姓张的问明白他们的姓氏,十分惊异,就到毛父淹死的地方察看,发现毛父正好死在该放棺材的地方。姓张的更加惊异,就让毛父葬在这个墓穴里了,还嘱咐毛母带着儿子来一趟。办完丧事,毛母同儿子一块来张家致谢。姓张的见了毛家孩子,非常喜欢,就把他留在家里,教他读书,把他当作自家的孩子看待。又提出要把大女儿许给他作妻子。毛母大惊,不敢答应。张的妻子说:“既然说了,就不会中途变卦。”毛母只好答应了。

但张家大女儿对毛家极为看不起,言词、神色间常常流露出怨恨、羞愧的情绪,偶尔有人提起这件婚事,她就捂住耳朵。还常对别人说:“我就是死了也不会嫁给放牛人的儿子。”到了迎亲的那天,新郎坐入酒席,花轿停在门外,这女子还捂着脸面对墙壁哭泣。催她梳妆,她不肯,也不听劝解。不多时,新郎起身请行,鼓乐齐奏,她还是蓬头散发地哭个不停。父亲让女婿稍等,自己亲自去劝女儿,女儿哭着像没听见一样。父亲大怒,逼她上轿,女儿更加号哭起来,父亲无可奈何。仆人又来传话:“新郎要走了!”父亲急忙出来说:“还没打扮好,请新郎再稍等等。”就又跑进屋去看女儿,出来进去不住脚。又拖延了一会儿,事情更加紧急,大女儿终究不回心转意。父亲没有办法,急得要寻死。

二女儿在一旁很不满意姐姐的态度,苦苦相劝。姐姐生气地说:“小妮子,你也学着多嘴多舌,你为什么不嫁给他?”妹妹说:“咱爹当初并没有把我许给毛郎;若把我许配毛郎,何须姐姐劝驾!”父亲听到二女儿说活爽快,就与她母亲暗地商量,用二女儿代替大女儿。母亲就问二女儿:“那个不孝顺的丫头不听话,如今想叫你代替姐姐出嫁,儿愿意吗?”二女儿痛快地说:“父母既然叫儿去,就是逃荒要饭也不推辞。况且,怎么知道毛郎就会穷一辈子,最后饿死呢?”父母听了她的话十分高兴,就用姐姐的嫁妆给妹妹妆扮起来,匆匆忙忙地打发她上轿走了。过了门,两口子和睦融洽,相敬如宾。只是二女儿素来头发稀少,稍微叫毛公不满意。后来,毛公渐渐听说了姐妹易嫁的事,从此更加感激她,把她看作贴心知己。

过了不久,毛公中了秀才,去参加乡试,路上经过王舍人店。店主人在前一天夜里梦见神仙对他说:“明天有个毛解元来,日后他会从危难中解救你。”于是店主人从早晨起来,就专门留心察看东边来的客人。等见到毛公,店主人大喜,备了一桌丰盛的酒菜,也不要钱,特地把梦里吉兆告诉他。毛公也很自负,暗想着:如果得中第一名举人,自己妻子的稀秃头发,恐怕被贵人讥笑,富贵之后应当换一个妻子。然而录取榜文公布之后,毛却名落孙山。他精神不振,步履沉重,觉得十分丧气。心中羞愧,没脸再见店主人,只好绕道回家。

三年以后,毛公又去赴试,那家店主人仍像上次那样热情招待。毛公说:“你的话那次没应验,实在对不起你那一番诚意。”店主人说:“秀才是因为暗想要换妻子,所以被阴间除名落榜了,并不是我的梦不灵验。”毛公惊愕地问他是怎么知道的,店主告诉他,那次分别后,又做了一个梦才知道的。毛公听了,又心惊又后悔,呆若木偶。店主人说: “秀才应当自爱,终究会作解元的。”不久,毛公果然考中第一名举人。妻子的头发也长起来了,乌黑油亮的发髻,更增添了她的美丽。

张家大女儿嫁给了同村的一个富户,非常趾高气扬。可是,她丈夫是个懒惰的浪荡公子,家境渐渐衰败,连家产也卖光了,穷得连饭都吃不上。听说妹妹做了举人的夫人,越发感到惭愧。有时和妹妹在路上相遇,就赶紧躲开。又过不久,张家大女儿丈夫死了,家里更加破落。不久,毛公又考中进士。大女儿听说,刻骨般恨自己,气恼地削发当了尼姑。到毛公当上宰相回家乡时,她强打发女尼到毛府去拜问,盼望着能得到点什么。女尼来到毛府,毛夫人赠给许多绫罗绸缎,将银子裹在里面。女尼并不知道,拿回去交给师傅,师傅大失所望,生气地说:“给我点金钱,还可买点柴米,这些东西给我有什么用?”于是又让女尼送了回去。毛公和夫人很疑惑,打开一看银子还在里面,才明白退回来的意思。毛公拿出银子笑着说:“你师傅连一百两银子都承受不起,哪有福份嫁给我这个老尚书啊!”随即拿了五十两银子给女尼说:“带回去作你师傅的生活费。多了,怕她福份薄,承受不起。”女尼回去,如实汇报,师傅默默不语,不停地叹息。想想自己的一生作为,常常正反颠倒,美的恶的,追求什么和躲避什么,哪里由得了自己呢!

后来那家店主人因人命案子被捕入狱,毛公竭力解说,他才被免罪释放。


\subsection{1.4.22   续 黄 粱}
\label{\detokenize{p00_u5176_u5b83/_u767d_u8bdd_u804a_u658b_u5fd7_u5f02:id154}}
福建有一位姓曾的举人,考中进士时,与二三位同科考取的进士到京城郊区游逛。偶然听别人说,在佛寺里住了一位算命的先生,便一块去请算命先生给算一卦。进了屋子,行礼坐下。算命先生见他那副得意的样子,就顺便奉承了他几句。曾某摇着扇子微笑,问算命先生:“我有没有身穿蟒袍、腰系玉带的福分啊?”算命先生一本正经地说:“你可做二十年太平宰相。”曾某听了,很高兴,神气更足。这时,外边下起小雨,于是就和同游的人在和尚的住房里避雨。屋里有一位年老的和尚,眼睛深深地凹下去,高高的鼻梁,端端正正地坐在蒲团上,神情淡淡地不主动见礼,几个人略一打招呼,便一起坐在床榻上,说起话来。都以宰相称呼曾某,向他表示庆贺。这时,曾某心高气盛,指着一位同游者说:“曾某当了宰相时,推荐张年丈做南京的巡抚;家中的中表亲戚,可以作参将、游击;家中的老仆人,也要作个小千总或者小把总,我的心愿也就满足了。”在坐的人都大笑起来。

一会儿,门外的雨下得更大。曾某感到很疲倦,就在床上躺下。忽然间,见到两位皇宫的使者送来皇帝的亲笔诏书,召曾太师入宫商讨国事。曾某很得意,很快地跟随来使朝见皇帝。皇帝把座位向前挪了挪,用温和的话语与他谈了很久;并说,三品以下的官员都要听从他的任免、提升,不必向皇上奏准;赐给他蟒袍、玉带和名贵的马匹。曾某披戴整齐,跪下向皇帝叩头谢恩,下朝而去。回到家里,发现不是以前那些旧房舍,而是雕梁画栋,极为壮丽,自己也说不清楚为什么一下子变成这样。但是,捻着胡须一呼唤,家中的仆人,就前呼后应的,如同雷鸣。过了一会,就有公卿大臣给他献上山珍海味,躬着身子毕恭毕敬的人,接二连三地出入他的门。六部尚书来了,他鞋子还没穿好,就迎上去;侍郎们来了,他便只作个揖,陪着说几句话;比这更低一级的官员来,只是点一点头罢了。山西的巡抚,赠给他乐女十人,都是秀美的女子。其中特别俊美的袅袅和仙仙,尤其得到他的宠爱。每当他在家休息的时候,就整天沉溺于歌舞声色中。

有一天,他忽然想起在未发迹时,曾经受到本县士绅王子良的周济,今天自己置身青云之上,那王子良还在仕途上很不得志,为什么不拉他一把呢?第二天早起,就给皇帝写了一道奏疏,荐举王作谏议大夫。得到皇帝的许可,就立刻把王子良提升到朝中。又想到,郭太仆曾经对自己有小怨隙,马上把吕给谏和侍御陈昌等叫来,把自己的意图告诉他们。过了一天,弹劾郭太仆的奏章,纷纷投到皇帝面前,得到皇帝的圣旨,把郭撤职赶出朝中。曾某报恩报怨,办得分明,颇快心意。

有一次,他偶尔来到京郊的大道上,一个喝醉酒的人,冲撞了他的仪仗队,就命下人把他捆起来,交给京官,立刻被打死在木棍之下。那些与他接近的近邻和田地相连的富人家,也都畏惧他的权势,把自己的好房子与肥沃的土地献给他。自这以后,他家的财富可与一个国王相比。不久,袅袅和仙仙先后死去了,他日夜思念她们。忽然想起,往年见他的东邻有一个少女特别美丽,每每想把她买来作妾,只因当时家势财力单薄,未能如愿,今天,可以满足自己的意愿了。于是派去几个干练的奴仆,硬把钱财送到她的家中。一会儿,用藤轿把她抬来一看,女子出落得比以前看见时更加美丽。自己回忆平生,各种意愿都达到了。

又过了一年,曾某常听到朝中有人在背后窃窃议论他,但他认为这只不过是像朝廷门口那些摆样子的仪仗马而已。他仍然盛气凌人不可一世,不把别人的议论放在心上。谁知竟有一位龙图阁大学士包公,大胆上疏,弹劾曾某。奏疏中说:“臣认为曾某,原是一个饮酒赌博的无赖,市井里的小人。只不过偶然一句话的投合,而得到圣上的眷顾。父亲穿上了紫色朝服,儿子也穿上了红色的朝服。皇上的恩宠,已经达到极点。曾某不思献出自己的躯体,不思肝胆涂地以报皇上之万一;反而在朝中任意而为,擅自作威作福。他可以处死的罪,像头发那样难以数清;朝廷中的重要官职,被曾某据为奇货,衡量官位的轻重,为收价的高低。因而朝中的公卿将士,都奔走在他的门下,估计官职买卖的价钱,寻找机会偷空钻营,简直如同商贩。仰仗他的鼻息,望尘而拜的人物,无法计算。即使有杰出之士与贤能的良臣,不肯依附于他,对他阿谀奉承,轻的就被他放置在情闲无实权的位置,重的就被他削职为民。更有甚者,只要不偏袒他的,动辄就触犯了他这指鹿为马的权奸;只要片言触犯了他,便被流放到豺狼出没的荒远之地。朝中有志之士为之心寒,朝廷因而孤立。又有那平民百姓的膏血,任意被他们蚕食;良家的女子,依势强娶。凶恶的气焰,受害百姓的冤愤,暗无天日。只要他家的奴仆一到,太守、县令都要看颜色行事;他的书信一到,连按察司、都察院也要为之徇情枉法。甚至连他那些奴才的儿子,或者稍有瓜葛的亲戚,出门则乘坐驿站的公车,气势浩大。地方上所供给的东西稍为迟缓,在马上的鞭子立刻就会抽打你。残害人民,奴役地方官府,他随从所到之处,田野中的青草都为之一光。而曾某现在却正是声势煊赫,炙手可热,依仗朝廷对他的宠信,毫无悔改。每当皇帝召见他到宫阙之中,他就乘机进陷别人;曾某刚从官府退回,他家中后花园中已响起歌声。好声色,玩狗马,白天黑夜荒淫无度,国计民生,他从来不去考虑。世界上难道有这样的宰相吗?内外惊恐,人情汹动,若不马上把他诛除,势必要酿成曹操与王莽那样的夺权之祸。臣日夜忧虑,不敢安居,我冒杀头之罪,列举曾某的罪状,上报圣上得知。俯伏请求割断奸佞之头,没收他贪污的财产。上可以挽回上天的震怒,下可以大快人心,顺通民情。如果臣言是虚假捏造,请以刀、锯、鼎、镬处置臣子”。

曾某听到消息后,吓得胆颤魂飞,如同饮下一杯凉冰的水,浑身上下凉透了。幸而圣上优待宽容,扣下此疏不作处理。但是,继之各科各道、三司六部的公卿大臣,不断上奏章弹劾;就连往日那些拜倒在他门下的,称他为干爸爸的,也翻了脸向他攻击。圣上下令抄没他家中的财产,充军到云南。他的儿子在山西平阳任太守,也已经派遣公差去把他提到京师审问。曾某刚刚听到圣旨,惊恐万分,接着就有几十名武士,带着剑拿着枪,径到曾某的内房,扒掉他的官服,摘下他的帽子,把他同他妻子一块捆绑起来。一会儿,看到许多差役,从他家中向外搬运财物,金银钱钞有数百万,珍珠翡翠、玛瑙宝玉有数百斛。幄幕、帐帘、床榻之属,有数千件;至于小儿的襁褓,女人的鞋子,掉得满台阶都是。曾某一一看得很清楚,感到心酸伤目。不一会,一个人拖着曾的美妾出来,她披头散发娇声啼喊,美丽的面容六神无主。曾某在一边,悲伤的心如同火烧,含着愤怒而不敢说。不一会,楼阁仓库,全被查封。差役立即呵叱曾某出去,监管他的人就用绳子套着他的脖颈,把他拉出去。

曾某同他妻子忍声含泪地上路。要求能有一匹老马拉的破车代步,差役也不答应。走了十里,曾某妻子脚小无力,快要跌倒,曾某用手搀扶着她走。又走了十里,自己也疲惫不堪。突然见前边有一座高山,直插云霄,自己发愁无法攀登过去,时时挽扶着妻子相对哭泣。而监管的人面目狰狞地过来催促,不容许他们稍微停歇。又看到太阳西斜,晚间无处可以投宿,不得已,就弯着腰,深一步,浅一步地走着。快到半山腰时,妻子实在无力了,坐在路旁哭泣。曾某也坐下来稍微休息,任凭监迭的差役叱骂。

忽然间听到多人一齐叫喊,有一群强盗各自拿着锋利的刀枪,跳着跑着追过来。监送的差役大惊而逃。曾某直挺挺地跪在地上说:“我孤身被贬谪边疆,行李中也无值钱的东西。”哀求他们宽恕。这些强盗个个瞪大了眼睛,忿怒地说:“我们这群人是被害的冤枉百姓,只要你这贼的头,别的什么也不要!”曾某愤怒叱责说:“我虽然有罪,可我仍然是朝廷的命官,你们这群乱贼,怎敢胡为!”群贼也怒极,挥动巨大的斧头,就朝曾某的脖颈砍去,只听得自己的头落地有声。惊魂未定,立刻见到两个小鬼,把他的双手捆起来,赶着他走。大约走了几个时辰,到了一个大的都市。不多时,看到一座宫殿,大殿之上坐着一位相貌很丑陋的阎王,靠在一个长长的几案上,在决断鬼魂的祸福。曾某急忙向前,匍匐跪在地上,请求阎王饶恕。阎王翻看着卷宗,才看了几行,就勃然大怒说:“这是犯了欺君误国的罪,应当放到油锅里炸!”殿下无数的鬼在应和着,声如雷霆。马上有一个巨鬼,把曾某抓起,摔到台阶之下。见有一只大油锅,约有七尺多高,四周围烧着火炭,油锅的腿都烧红了。曾某浑身发抖,哀哀啼哭,逃窜又无去路。巨鬼用左手抓住他的头发,右手握着他的脚脖,把他扔到油锅中。觉得孤零零的身子随油花上下翻滚,皮与肉都焦糊,疼痛彻心钻骨;沸着的油灌到口里,把他的肺腑都烹熟了。心想快死算了,而想遍了法子也不能马上死去。约一顿饭的时间,巨鬼才用大铁叉把曾某从油锅里取出来,又让他跪到大堂下。阎王又查检了簿籍,生气地说:“生时依仗权势,欺凌别人,应当上刀山之狱。”鬼又把他揪去,见到一座山,不很大,而峻峰峭拔,锋利的刀刃纵横交错、像密密的竹笋。已经有几个人的肚肠挂在上边,呼喊号叫的声音,惨忍难听。巨鬼督促曾某上去,曾大哭着向后退缩。臣鬼用毒锥刺他的头,曾某忍痛乞求可怜。巨鬼大怒,抓起曾某,向空中掷去。曾某觉得自己身在云霄间,昏昏然地向下掉,锋利的刀交刺在他的胸膛上,痛苦之情难以言状。过了一会,由于他的身体太重,向下压去,被刺入的刀口渐渐大了,忽然他从刀上脱落下来,四肢蜷曲着。巨鬼又撵着他去见阎王。阎王让计算一下他生平卖官鬻爵、贪脏枉法所霸占的田产,所得的金银财宝有多少。立刻有一个胡须卷曲的人数着筹码,屈着指头算计说:“三百二十一万。”阎王说:“他既然能搜括来,就让他都喝下去。”不多会,把金钱取来堆集到台阶上,像小山丘。慢慢地放到铁锅里,用烈火熔化。巨鬼让几个小鬼,更替着用勺子灌到他的口中,流到面颊上皮肤都臭裂;灌到喉咙,五脏六腑像开锅一样。曾某活着时,恨自己搜括得太少,眼下又以此物太多为患。半天才灌尽。阎王下令,把曾某押解到甘肃甘州托生个女的。走了几步,见到架子上有一铁梁,粗有好几尺,上边穿着一个火轮,大也不知有几百里,发出五彩般的火焰,光亮照耀到云霄间。巨鬼鞭挞着曾某上去蹬火轮子。他刚一闭眼,就跃登上去,火轮随着他的脚转动,似觉身子向下倾坠,遍身冰凉。

他睁开眼一看,自身已变成婴儿,还是个女的。看看生他的父母,都穿着破烂的棉衣。土房中,放着破瓢和讨饭的棍子。知道自己已变成了讨饭人的女儿。从此,每天跟随讨饭人沿街乞讨,肚子里常常饿得直叫,不得一饱。穿着破烂的衣服,被风吹得刺骨疼。十四岁那年,被卖给一个姓顾的秀才当小妾,衣食才算自给。而家中的大老婆很凶狠,每天不是用鞭子抽就是用板子打,还用烧红的烙铁烙乳房。幸好丈夫还可怜她,稍稍有些安慰。墙东邻有个很不正经的恶少年,忽然越过墙来,逼着与她私通。心想,自己前所行的罪孽,已受到鬼的惩罚,现在哪里能再犯呢!于是大声呼救。丈夫与大老婆都起来,恶少年才逃去。过了不久,秀才刚到她的房间中睡觉,在枕上喋喋地诉说自己的冤苦。忽然一声巨响,房门大开,有两个贼持刀闯进来,竟然砍掉秀才的头,抢光衣物就走了。她团团地爬在被子底下,大气不敢出。等到贼去了,才哭喊着跑到大老婆的房中。大老婆大惊,哭着与她一块去验看秀才的尸体。怀疑是她招引奸夫杀死自己的丈夫。因而写状告到州官刺史。刺史严加拷问,以酷刑毒打,使她招认定案,依照法律,判凌迟处死,把她绑着到行刑的地方。她胸中冤枉之气堵塞,大跳着喊冤屈,觉得比十八层地狱还黑暗。

正在悲痛呼号的时候,听得同游的朋友说:“老兄你作恶梦了吗?”曾某忽然醒悟过来。见到老和尚还盘着腿坐在那里。同游的人都问他:“天晚了,肚子都饿了,为什么睡了这么久?”曾某这才面色惨淡地坐起来。老和尚微笑着说:“占卦说你作宰相,是否灵验?”曾某越发惊异,行礼向老和尚请教。老和尚说:“要修自己的德行,要行仁道,就是在火坑中,也能生长出青莲花来。我这个山野中的和尚,哪里能参透其中的玄妙!”曾某满腹胜气地来了,垂头丧气地回去,追求升官享受荣华富贵的想法,由此慢慢地淡薄了。后来,他隐遁到深山之中,不知所终。


\subsection{1.4.23   龙 取 水}
\label{\detokenize{p00_u5176_u5b83/_u767d_u8bdd_u804a_u658b_u5fd7_u5f02:id155}}
民间传说,龙取江河里的水而成雨,这是个令人半信半疑的说法。徐东痴南游时,停船江岸,看见一条苍龙从云彩中垂下来,用尾巴搅动江水,立时波浪涌起,随着龙身往上升腾。远远看去,水光一闪一闪的,比三匹白练还要宽。一会儿,龙尾收回去,水也即刻平息了。刹那间大雨倾盆而注,沟满濠平。


\subsection{1.4.24   小 猎 犬}
\label{\detokenize{p00_u5176_u5b83/_u767d_u8bdd_u804a_u658b_u5fd7_u5f02:id156}}
山西省的卫中堂,当年做秀才的时候,厌烦家中杂务的干扰,就搬到一所寺院里读书。可寺院的臭虫、蚊子、跳蚤非常多,竟使他终夜睡不着觉。

一天,吃过饭后,他躺在床上休息。忽然看见一个小武士,头插雉翎,身高约二寸,骑着一匹只有蚂蚱那么大小的马,胳博上架着一只苍蝇大的措鹰,从外边进来,在屋里盘旋,走走跑跑。卫中堂正看得出神,忽然又进来一个小人,穿戴和前一个武士一样,腰中扎着小弓箭,牵着一只蚂蚁大小的猎犬。又过了一会儿,步行的、骑马的,又有数百人纷纷而来,共架着数百只鹰、牵着几百头猎犬,只要有蚊蝇飞起来,小武士们就放鹰腾空扑击,全都杀死。小猎犬则跳到床上,爬到墙壁上,搜吃跳蚤、臭虫。凡是藏在被褥和墙隙里的臭虫和跳蚤,没有小猎犬嗅不出来的,顷刻之间,全部扑杀死了,卫中堂假装睡觉,眯着眼偷偷地看着,鹰和猎犬都在他身上窜来跑去。接着一个穿黄衣服的人,头戴平天冠,好像是大王,登上另外一张床,把马拴在席子上。随从的人都下了马,小武士们有的献上蚊蝇,有的献上臭虫、跳蚤,纷纷嚷嚷也不知说的什么话。时间不长,大王登上一辆小车,卫士们匆忙上马,万马奔驰,纷纷扬扬像撤菽粒子,烟飞雾腾,不一会儿就散尽了。

卫中堂看得清清楚楚,又惊骇又诧异,不知它们是从哪里来的,急忙穿上鞋子偷偷往外看,已经无影无踪。他返回身四面看看,都没有看到什么,只有墙壁的砖上遗留下一只小猎犬。卫中堂急忙捉住它,小猎犬很温驯,卫中堂把它放在砚台的匣子里,反复瞻玩,见它的毛极细而且柔软,脖子上有个小环。喂它饭粒,它一嗅就走开。跳到床上,寻找衣缝,咬杀虮子虱子,吃饱了再回到匣子里趴着。过了一夜,卫中堂疑心它已经走了;一看,仍然蜷曲着趴在那里。卫中堂躺下,它就跳到床席上,遇到臭虫就咬死,蚊蝇没有敢落下来的。卫中堂非常喜爱它,比宝贝还珍贵。

一天,卫中堂白天躺着睡了,小猎犬偷偷地趴在他身旁。卫中堂醒了翻身,把它压在腰底下。卫中堂感觉身下有什么东西,怀疑是小猎犬,急忙起身一看,已经被压扁死了。但是从此墙壁上再没有活着的蚊虫了。


\subsection{1.4.25   棋 鬼}
\label{\detokenize{p00_u5176_u5b83/_u767d_u8bdd_u804a_u658b_u5fd7_u5f02:id157}}
扬州的督同将军梁公,辞官回乡居住,每天携带着棋酒,游玩在青山绿林之间。正好九月九日重阳节登高,梁公和客人们下棋取乐。忽然有一个人来到,在棋局旁边徘徊,过了很长时间也不离去。看他的样子,很清贫,衣服破败不堪。但是他的仪态却温文尔雅,有文人的风度。粱公礼让他,他才非常谦逊地坐下。梁公指着棋对他说:“先生一定有很好的棋艺,为什么不和客人对阵呢?”他非常有礼貌地推让了一会,才开始和客人对局。第一局下完他败了,神情懊丧焦急,像是不能控制自己的样子。再下再败,他更湘恼怒。请他喝酒,也不喝,只是拉客人继续下棋。从早晨到太阳偏西,他都没来得及大小便。正在因为一个棋子争路,双方争执不休的时候,忽然书生离开座位很恐惧地站在那里,神色凄惨沮丧。不一会,他屈膝向梁公跪下,叩头请求救护。梁公很惊异,起来扶他说:“本来是游戏,何至于这样?” 书生说:“求您嘱咐养马人,不要捆绑我的脖颈。”梁公更觉奇怪,问道:“养马人是谁?”他答道:“马成。”

原来梁公的养马仆役马成,充当阴间的鬼吏,经常十几天一次入阴曹地府,拿着冥府的文书作勾魂捕役。梁公认为书生的话很奇特,便派人去看马成,果然他僵死卧床已两天了。梁公于是叱责马成不得对书生无礼。一转眼,书生就地倒下不见了。梁公叹息了好久,这才明白书生原来是个鬼。

过了一天,马成醒过来,梁公召他来问这件事。马成说:“书生是湖襄人,爱好下棋成癖,家产都弄光了。父亲为他的事发愁,把他关在书房中,但他总是越墙出去,偷偷躲避到无人的地方,和爱好下棋的人继续来往。父亲听说后责骂他,终究也没能制止住,父亲为此气愤愁闷怀恨而死。阎王因书生无德,减了他的寿命,罚入了饿鬼狱,至今已经七年了。后遇东岳风楼落成,下文通知各个地府,征召文人撰写碑记。阎王把书生放出牢狱,让他前去应召自我赎罪。不料想他途中拖延,衍误了期限。东岳大帝派值日的官吏问罪于阎王。阎王大怒,派我们搜捕他。前天接受您的吩咐,没敢用绳索捆绑他。”梁公问:“今日他的状况怎么样?”马成说:“仍然被交付狱吏,永远没有生还期限了。”梁公叹息道:“癖好误人竟到了这样的地步啊!”


\subsection{1.4.26   辛 十 四 娘}
\label{\detokenize{p00_u5176_u5b83/_u767d_u8bdd_u804a_u658b_u5fd7_u5f02:id158}}
广平县的冯生,是明代正德年间的人。他年轻时轻佻放荡,酗酒无度。一天早晨,他偶然外出,遇到个少女,披着红斗篷,容貌秀丽。身后跟着个小仆人,正踏着早晨的露水赶路,鞋袜都沾湿了。冯生心里暗喑喜爱她。傍晚,冯生喝得醉醺醺地回来,走到路边一座荒废很久的寺庙前时,见一个女子从里面走出来;一看,正是早晨遇到的那个少女。少女看见他,转身又走了进去。冯生暗想,美人怎么会在寺庙里?把驴拴在门前,想进去看个究竟。

进入庙门,只见断壁残垣,石阶上铺着层绿毯一样的细草。冯生正在犹豫,一个衣帽整洁的白发老翁走了出来,问道:“客人从哪里来?”冯生说:“偶然经过这座古刹,想瞻仰瞻仰。老丈怎么到了这里?”老翁说:“老夫流落到此地,没有住所,暂时借这里安顿家小。既然承蒙光临,有山茶可以当酒。”说完,请冯生进庙。冯生见殿后有个院子,石子路非常干净,再没有杂树乱草。进入屋内,帷幔床帐,都香气袭人。坐下后,老翁自我介绍说:“老夫姓辛。”冯生乘醉唐突地问道:“听说您有个女公子,还没找到好女婿;我不自量力,愿意礼聘女公子。”辛老翁笑了笑,说:“容我和老妻商量商量。”冯生要来笔,写下一首诗:“千金觅玉杵,殷勤手自将。云英如有意,亲为捣玄霜。”主人看了后,笑着把诗交给了仆人。一会儿,有个丫鬟出来和老翁耳语了几句,老翁起身,请客人耐心坐会儿。自己掀起门帘进了里屋。隐约听得里面讲了两三句话,老翁又走出来。冯生以为定有好消息,但老翁坐下后,只是谈笑,再不提婚事。冯生忍不住,问道:“我还不知您的意思,请说明以消除疑惑。”老翁说:“您是卓越不凡的人,我仰慕已久。但我有点隐衷,不便直言。”冯生再三请求。老翁说:“我有十九个女儿,已嫁出去了十二个。女儿的婚姻大事由老妻作主,老夫不参与。”冯生说:“我只要今天早晨带着小仆人,踏着露水赶路的那位。”辛老翁没说话,两人相对无语。这时里屋传来女子的娇声细语,冯生乘着醉意,掀起门帘说:“既然做不成夫妻,就看看容貌,以消除我的遗憾!”屋里的人听见门帘响,都惊愕地站了起来看着他。冯生见果然有那红衣少女,打扮华美,手捻着腰带,亭亭玉立。看见冯生闯进来,屋里的人都惊慌不安。辛老翁大怒,命几个人将冯生揪了出去,冯生酒涌上来,跌倒在乱草丛里,瓦块石头雨点般地落下来,幸亏没砸在身上。

躺了一会儿,听见驴子在路边吃草,冯生爬起来骑上去,踉踉跄跄地上了路。夜色迷茫,冯生误进了山谷,狼奔鸱叫,吓得他寒毛直竖。犹豫着四下看了看,并不知这是什么地方。远远望见一片黑树林中隐约有灯光,冯生以为必定是村庄,赶着毛驴跑了过去。抬头一看,是一座高门,便用鞭子敲了敲。门内有人问道:“哪里来的年轻人,半夜跑到这里来?”冯生回答说:“迷了路。”那人说:“等我禀告主人。”冯生伸着脖子,呆呆地等着。忽听抽门栓开门声,一个壮健的仆人走出来,替他牵驴。冯生进去,见房屋都非常华美,大堂上灯火通明。略坐了会,有个妇人出来,询问客人的姓名。冯生告诉了她。过了一会儿,几个丫鬟扶着一位老太太走出来,说:“郡君来了!”冯生站起身,恭恭敬敬地想行礼,老太太止住他,让他坐下。说:“你是不是冯云子的孙子啊?”冯生回答说:“是的。”老太太说:“你是我的外甥。我老态龙钟,风烛残年,骨肉亲戚之间,久没来往了。”冯生说:“我小时候就死了父亲,跟我祖父交往的人,十个里也不认得一个。我从没拜见过您,请指示明白该怎样称呼您?”老太太说:“你自己会知道的!”冯生不敢再问,坐在那里冥思苦想。老太太说:“外甥深夜怎么到了这里?”冯生平素常以胆大自夸,便把自己的遭遇一一叙述了一遍。老太太笑着说:“这是大好事。况且外甥是名士,也不玷污她家,野狐精怎么就这么自大?外甥不要担心,我能给你办成。”冯生连连称谢。老太太看着两边伺候的人说:“我不知辛家的女儿,竟是这样端庄漂亮。”一个丫鬟说:“他家有十九个女儿,都生得姿态翩翩。不知官人要聘的那个排行第几?”冯生说:“她大约十五岁左右。”丫鬟说:“这是十四娘。三月里,曾跟她母亲来给郡君庆寿,郡君怎么忘了呢?”老太太笑着说:“是高底鞋上刻着莲花瓣、里面填上香屑,用纱巾蒙面走路的那个吧?”丫鬟说:“是的。”老太太说:“这个婢子倒很会出花样,弄媚态。但也真是俊俏,外甥的眼光不错。”便对丫鬟说:“可派个小丫头去叫她来。”了鬟答应着去了。过了会儿,丫鬟进来禀报:“辛家十四娘叫来了!”接着便见红衣女子,望着老太太施礼。老太太拉她起来说:“以后成了我外甥媳妇了,就不要行女孩儿礼了。”女子起来,亭亭玉立,低垂着红袖。老太太理理她的头发,又捻捻她的耳环,说:“十四娘最近在闺中做些什么?”女子低声说:“闲着没事,绣些花。”说着,一回头看见冯生,立即羞缩不安起来。老太太说:“这是我外甥。他一心一意要和你结为夫妻,你怎么就让他迷了路,在山谷里窜了一夜?”女子低着头,默默不语。老太太说:“我叫你来,没别的事,想给我外甥做媒人。”女子仍一言不发。老太太便命丫鬟去扫床铺被,让他们二人完婚。女子红着脸说:“我得回去告诉父母。”老太太说:“我给你做媒,有什么差错?”女子说:“郡君之命,我的父母不敢违抗。但如此草草从事,我就是死,也不敢从命!”老太太笑着说:“小女子志气倒高,不屈从威势,真是我的外甥媳妇。”于是,便从女子头上拔下一朵金花交给冯生,让他回去查查历书,定个良辰吉日;又让丫鬟送十四娘回去。这时,雄鸡高唱,老太太派人牵着毛驴送冯生出去。

冯生出来走了几步,回头一看,只见房屋村落全消失了,只有一片茂密的松林和蓬草掩盖着的几座坟墓而已。冯生定神想了会儿,醒悟这里是薛尚书的坟墓,薛尚书是冯生祖母的弟弟,所以老太太称他为外甥。冯生心中明白遇上了鬼,但也不知十四娘是什么人。一路感叹着回了家,漫不经心地查了个日子等着,心里恐怕鬼约靠不住。再去那座寺庙看看,一片荒凉,寂无人迹。询问当地的人,说是庙里常见狐出没。冯生暗想:只要得到美人,狐也是好的。

到了选定的那天,冯生整理房间,打扫道路,让仆人轮番在门外眺望。一直等到半夜,还没动静,冯生已经绝望了。一会儿,忽听门外人声喧哗,冯生趿拉着鞋跑出去一看,花轿已停在院子里了,两个丫鬟扶着十四娘坐在轿里。嫁妆也没多余的东西,只有两个长胡子仆人扛着个瓮大的储钱罐,从肩上卸下放在屋子一角。冯生高兴娶了个美丽妻子,并不疑虑她是异类。他问十四娘:“一个死鬼,你们家怎么那样服贴她?”十四娘说:“薛尚书现在已做了五都巡环使,数百里内的鬼狐都供他役使。他不常回家。”冯生不忘老太太给做媒,第二天,到她的墓上祭祀了一番。同去时,有两个丫鬟来赠送带有贝纹的锦帛作贺礼,放到桌子上走了。冯生告诉十四娘,十四娘看了看,说:“这是郡君的东西!”

同县有个楚银台的公子,从小就和冯生同学,两人十分亲匿。他听说冯生娶了个狐夫人,便在冯生结婚三日那天,送来礼物,并亲自上门举杯庆贺。过了几天,楚公子又写来请柬,请冯生赴宴。十四娘得知,对冯生说:“上次公子来,我从墙缝里见他猿眼鹰鼻,这人不可长久交往,不去为好。”冯生答应了。第二天,楚公子登门责问冯生负约,就便献上自已的新作诗篇。冯生评论这些诗篇时,说了些嘲笑话,楚公子很羞惭,两人不欢而散。冯生回屋,笑着跟十四娘讲了一遍。十四娘凄然地说:“楚公子是匹豺狼,不能跟他开玩笑!你不听我的话,将遭大难!”冯生笑着认了错。此后,冯生和楚公子经常来往调笑,原来的过节渐渐消除了。正好提学驾下临,主持科考,楚公子考了第一,冯生考了第二。楚公子沾沾自喜,派仆人来邀请冯生去喝酒。冯生推辞不去,连叫了几次,才去了。到后来才知道是楚公子的生日,客人坐满了屋子,酒宴十分丰盛。楚公子拿出自己的试卷给冯生看,亲友争相围拢来观赏,边看边赞叹着。酒过数巡,有乐队在下面奏起音乐,一片喧杂,宾主都非常高兴。楚公子忽然对冯生说:“俗话说‘场中莫论文’,现在才知道这句话的错误。我之所以名次排在你前面,不过因为我的文章开头几句略高一筹罢了。”公子说完,一座人都赞扬起来。冯生乘着醉意,再忍耐不住,大笑着说:“你到现在还以为你是凭文章考第一的吗?”冯生话音刚落,一座人脸上失色。楚公子羞惭忿怒,无言答对。客人们见状渐渐都走了,冯生也悄悄地溜了回来。酒醒后,冯生很后悔,把这事告诉了十四娘。十四娘不高兴地说:“你真是乡下的轻薄子弟!拿轻薄之态对待君子,就会丧失品德;对待小人,就会惹杀身之祸。你大难不远了!我不忍心见你败落,我们分手吧!”冯生害怕,哭泣着说自己已很后悔。十四娘说:“如想要我留下来,我和你约定,从今后你闭门不出,断绝交游,不要再酗酒!”冯生恭敬地答应下来。

十四娘为人勤俭利落,天天纺线织布。经常自己回娘家,但从不在娘家过夜。还常拿出些金银布帛作买卖,每有赢余,就把钱投进储钱罐里。天天关门闭户,有人来访,就让仆人谢绝。一天,楚公子又送来信请冯生,十四娘把信烧了,不让冯生知道。第二天,冯生出门去城里吊丧,在丧家遇到楚公子。楚公子拉着他的胳膊,苦苦邀请。冯生借故推辞,楚公子让马夫拉着马,拥着冯生就走。到了家,楚公子立即命家人设宴。冯生又告辞,说有事要早点回去。楚公子再三挽留,吩咐家姬弹筝奏乐。冯生本来就放荡不羁,前些日子又一直关在家里,很觉烦闷。忽然遇上今天这个痛饮的机会,酒兴大发,再也不管不顾,喝得酩酊大醉,昏沉沉地趴在桌上睡着了。楚公子的妻子阮氏,非常凶悍嫉妒,婢妾们都不敢施脂抹粉。前天有个丫鬟到楚公子的书房中,被阮氏抓住,用木杖猛击丫鬟的头部,丫鬟脑袋破裂,立即死了。楚公子因为上次冯生当众羞辱自己,怀恨在心,天天想着报复,于是图谋借这个事先把冯生灌醉,诬告他杀人。乘冯生正在昏睡,楚公子把丫鬟的尸体扛到床上,闭上房门走了。冯生五更天时酒醒过来,发现自己趴在桌子上。起来寻找枕头床铺,觉得有个滑腻腻的东西绊了脚,用手一摸,是个人。冯生还以为是主人派了童仆陪伴自己睡觉,便又用脚踢踢,那人一动不动,像具僵尸。冯生恐惧万分,跑出房门大声怪叫起来。楚家的仆役们都起来了,点上灯一照,发现一具尸体,便抓住冯生愤怒地吵闹起来。楚公子出来察看了一番,诬说冯生逼奸不遂,杀了丫鬟,将他捆起来,送到了广平县衙。

隔了,一天,十四娘才知道这件事,不禁潸然泪下,说:“早知道会有今天了。”于是每天都送钱给冯生花费。冯生见了府尹,无理可伸,被天天严刑拷问,打得皮开肉绽。十四娘亲自去询问他经过,冯生见了她,悲愤填膺,说不出话来。十四娘知道这次陷井已深,便劝冯生先屈认了,以免再挨打,冯生哭着答应了。十四娘来来往往时,别的人在眼前也看不见她。十四娘回家又感慨又叹息,忽然,她把自己的丫鬟打发走了。一个人住了几天,十四娘又托媒婆买了个良家女子,名叫禄儿,十五岁,容貌颇为艳丽。十四娘跟禄儿,同吃住,看待她不同于一般丫鬟。冯生招认误杀人命后,被官府判了绞刑。仆人得知这个消息,泣不成声地告诉了十四娘。十四娘听说,面色坦然,像毫不介意。不久,快到了秋后处决犯人的日子,十四娘才惶惶不安,经常白天出去,晚上才回来,脚不停歇。常在没人的地方,悲伤哀痛,以至于寝食都废。

一天下午,十四娘派出的那个狐丫鬟忽然回来了。十四娘急忙起身,将丫鬟叫到无人处,二人小声交谈起来。十四娘再出来时,笑容满面,和平常一样料理家务。第二天,仆人到监狱,冯生托他带回话来,要十四娘去见一面,以便永诀。十四娘漫不经心地答应了一声,也不悲伤,没当回事,家人私下里议论她太忍心。忽然路人到处流传,楚银台已被革职,平阳观察奉皇帝特旨,重审冯生一案。仆人听说大喜,急忙告诉了十四娘。十四娘也很高兴,便派他到官衙中探听。去了后,冯生已经出狱,与仆人见面,悲喜交集。一会儿,楚公子逮到,平阳观察一审问,明白了其中的全部实情,便立即释放了冯生,让他回家。冯生回家见了十四娘,不禁泪珠滚滚;十四娘也看着他心酸不已。悲伤过后,才又喜欢起来,但冯生终究不知自己的案子皇帝是怎么知道的。十四娘指着丫鬟说:“这是你的功臣啊!”冯生惊愕地询问缘故。

原来,十四娘派丫鬟进京,想到皇宫告状,为冯生申冤。丫鬟来到京城,见宫中有神灵守护,便在御沟外徘徊犹豫,一连几个月进不去。丫鬟怕误了事,正想再回来商量个办法,忽听说皇帝要去大同,丫鬟便预先赶到大同,装作妓女。皇帝到妓院游逛,特别宠爱她;又怀疑她不是一般的风尘女子,丫鬟便哭起来。皇帝问:“有什么冤屈吗?”丫鬟回答说:“我原籍广平县,是生员冯某的女儿。父亲因冤案将被处死,于是把我卖到了妓院里。”皇帝听说,很惨然,赐给她一百两银子。临走前,又详细问了事情经过,用纸笔记了姓名;还说要和她共享荣华富贵。丫鬟说:“但愿我和父亲能团聚,不想过富贵生活。”皇帝点头答应,便走了。丫鬟讲了经过,冯生急忙下拜,热泪盈眶。

不久,十四娘忽然对冯生说:“我如不是为了情缘,哪里会有这些烦恼?你被下狱时,我奔走于亲戚之间,却没一个人肯为我想个办法。那时的酸楚,真让人没法说。现在我越感到这尘俗世界令人厌烦苦恼。我已替你找了个女子,我们从此分别吧!”冯生听说,哭着跪在地上不起来,十四娘才作罢。到夜晚,十四娘让禄儿去跟冯生睡,冯生拒而不纳。第二天早晨看看十四娘,容光顿减。又过了一个多月,十四娘渐渐衰老。半年后,便又黑又丑,像个村妇。但冯生仍恭恭敬敬地对待她,始终不变。十四娘忽然又说要告别,还说:“你自有美丽的妻子,要我这丑老婆子干什么?”冯生像上次那样哭着哀求。又过了一个月,十四娘暴病,不吃不喝,疲惫地躺在床上。冯生端汤喂药,像侍奉父母。请来巫婆、医生,都不灵验,十四娘终于不治,去世了。冯生悲痛欲绝,就用皇帝赐给丫鬟的那一百两银子,埋葬了十四娘。过了几天,狐丫鬟也走了。冯生便娶了禄儿为继室,过了一年便生了个儿子。可是连年歉收,家境日渐萧条,夫妻二人一筹莫展,相对忧愁。冯生忽然想起屋角里的储钱罐,常见十四娘往里投钱,不知钱罐还在不在。过去一看,只见豆豉盆子、盐罐子摆了满满一地。一件件挪开,见储钱罐还在,用筷子往罐里捅了捅,坚硬得插不下去。把罐子摔碎,金钱哗哗地淌了出来。从此,冯生一下子富裕起来。

后来,冯生的仆人到太华山,遇见十四娘,骑着匹青骡子,丫鬟骑着驴跟在后面。十四娘见了仆人,问:“冯郎平安吗?”还说,“回去告诉你主人,我已名列仙籍了。”说完,便消失不见了。


\subsection{1.4.27   白 莲 教}
\label{\detokenize{p00_u5176_u5b83/_u767d_u8bdd_u804a_u658b_u5fd7_u5f02:id159}}
白莲教中的某人,是山西人,忘了他的姓名,大概是徐鸿儒的门徒。他用法术迷惑众人,羡慕他法术的人多拜他为师。

有一天某人要外出,他在堂屋中放置了一个盆,又用另一个盆盖住它,嘱咐门徒坐着看守,并告戒他不能掀开看。某人走后,门徒把上盆掀开,见下面盆里盛放着清水,水上浮着一只草编的小船,船上风帆桅杆俱全。他感到奇异,便用手指拨了一下,小船随手翻倒;他急忙把船扶成原来的样子,仍旧用盆盖好。一会儿某人回来,愤怒地斥责说:“为什么违背我的吩咐?”门徒立即表白说没有。某人说:“刚才海中船翻,怎么能欺骗得了我呢?”又一天傍晚,某人点燃大蜡烛放置堂上,告戒门徒要严加看守,不能让风吹灭。天到二更,某人仍没回来,门徒疲倦,便松懈了,躺到床上小睡;等到醒来,蜡烛已经灭了,急忙起来点燃。蜡烛刚点着,某人就进来了,又责备他。门徒说:“我本来就不曾睡,蜡烛怎么能熄灭呢?”某人愤怒地说:“刚才让我摸黑走了十几里路,你还在胡说什么?”门徒大惊。像这样奇怪的事情多得很,数不胜数。

后来某人的爱妾与门徒私通。他觉察后,隐忍不说。他派这个门徒去喂猪,门徒进圈后,立刻变成了一头猪。某人便叫屠户把这猪杀了,把肉卖掉,人们都不知道。这门徒的父亲因为儿子没回家,就来询问,某人告诉说他已经很久不来了。门徒的家人到处打听寻找,始终也没有消息。有个和这个门徒同师学艺的人,暗中知道此事,把消息泄露给了门徒的父亲。门徒的父亲告到了县令那里。县令恐怕某人逃走,没敢逮捕他;而把这事报告了上一级官员,请求派了披甲的兵士一千人,包围了某人的家,把某人和他的妻儿全都捉住,紧闭在木笼囚车里,要把他们押解到京城去。

途中经过太行山时,山中突然出来一个巨人,和大树一样高,它的眼睛像坛子,嘴像盆那样大,牙有一尺多长。兵士们都惊讶地站住脚不敢再往前走。某人说:“这是个妖怪,我的妻子可以打退它。”于是兵士们按照他的说法,解开了他妻子身上的枷锁。妻子持戈追上前去,巨人发了怒,张开大口把她吸吞到肚里。众人更加害怕。某人说:“既然妖怪杀了我的妻子,必须让我儿子来对付它。”于是再放出他的儿子前去,又被巨人吞了。众人都面面相觑,不知怎么办才好。某人哭着并且发怒地说:“既然杀了我的妻子,又杀了我的儿子,这我怎能甘心!看来非我亲自去收拾它不可了。”众人果然把他放出木笼,并给他武器让他前去。巨人非常愤怒地迎上来,格斗了不多时,巨人便抓起某人放入口中,伸了伸脖子咽了下去,不慌不忙地走了。


\subsection{1.4.28   双 灯}
\label{\detokenize{p00_u5176_u5b83/_u767d_u8bdd_u804a_u658b_u5fd7_u5f02:id160}}
魏运旺,是益都县盆泉人,他家是原来的世族大家。后来家势败落,不能再供他读书,二十来岁时,就荒废了学业,跟着他岳父家卖酒。

一天晚上,魏生独自躺在酒楼上,忽然听见楼下有脚步声。他吃惊而起,很害怕地听着。声音渐渐近了,随即上了楼梯,一步比一步响。一会儿,有两个丫鬟挑着灯,已经到了床边。后边有一少年书生,引导着一名女郎,微笑着走近床前。魏生大为惊愕。转念一想知道是狐,因而毛发直竖,低着头不敢再看。书生笑着说:“魏君请勿猜疑,舍妹与您有夙缘,就应当来侍奉您。”魏生见少年身穿绸缎貂皮,耀人眼目,相比之下自惭不如,羞愧得不知怎样对答。书生带领丫鬟,留下灯就走了。魏生仔细端详女郎,衣服鲜明,身材美好,像仙女一般,心里非常喜欢她。但是由于羞愧而说不出亲密的调笑语。女郎笑着对他说:“您又不是靠啃书本生活的人,怎么会有迂腐的书生气?”她便走近床边,把手伸进他的怀中取暖。魏生这才有了笑脸,拉扯说笑,于是两人亲热起来。天还没亮的时候,两个丫鬟就来接女郎走了。还订好夜里再相会。

到了晚上,女郎果然来了,笑着说:“痴郎是何福气?不费一文钱,得到这么好的媳妇,能夜夜自己来相会。”魏生窃喜没有别人在,就摆上酒和她对饮,并玩赌藏枚的游戏。女郎十有九赢,便笑着说:“不如让我来掌握枚子,郎君自己猜,猜中就胜,猜不中就败。若是还让我猜的话,郎君便没有赢的时候了。”于是按她说的那样,二人玩得很痛快。将要睡觉的时候,女郎说:“昨天晚上的被褥既不光滑又冷,让人不能忍受。”就叫丫鬟抱了被褥来,展开放到床上,带素花纹的丝绸料子又香又软。一会儿,解衣相偎,脂香浓烈,像这样的艳福真不亚于帝王的温柔乡。从此以后,便成了平常事了。

后半年,魏生回了家。一个月夜,他正和妻子在窗下说话的时候,忽然看见女郎穿着华丽的衣服坐在墙头上,用手招呼他。魏生走到她的身边。女郎拉他,一同越墙而出,手把手地告别说: “今天要和您分别了。请送我几步,以表示半年来的恩爱情义吧。”魏生惊问她是什么缘故,女郎说:“姻缘自有定数,还有什么可说的呢?”说着,到了村外,原来的丫鬟挑着双灯在等候着。走到了南山,登到高处以后,向魏生告辞言别。魏生留不住她,只得让她走了。魏生久久地站在那里不知怎样才好,遥远看见双灯一闪一闪的,渐渐远去看不见了,才闷闷不乐地返回家。这一夜山头上的灯光,村里的人都看见了。


\subsection{1.4.29   捉 鬼 射 狐}
\label{\detokenize{p00_u5176_u5b83/_u767d_u8bdd_u804a_u658b_u5fd7_u5f02:id161}}
李著明,是雎宁县令李襟卓先生的儿子,为人豪爽勇敢,从不知胆怯。他是新城王季良先生的内弟。王先生家有很多楼阁,经常有人看到楼阁里出现一些怪异的事情。

李著明常常夏日在王家寄宿。一次,他喜欢阁楼上晚风凉爽,要去阁楼上睡。有人告诉他阁楼上的怪异,李著明笑了笑,不听,执意要求设床在上面睡。主人只得照办了,吩咐仆人和他作伴。李著明推辞说:“我喜欢一个人睡,平生不知道什么叫害怕!”主人便在香炉里烧上香,又铺好床,问明头朝何方,服侍他睡下,然后灭了灯,掩上房门走了。

李著明刚躺下一会儿,在月光下,忽见桌几上的一只茶叶罐倾斜着飞快地旋转起来,既掉不下来,也不停止。李著明呵斥了一声,茶叶罐立时止住。一会儿,又见像有人拔出了香炉里的香,在空中上下左右地摇晃,织出了一片纵横交错的花线。李著明起身斥责说:“什么鬼物,胆敢这样!”光着身子下床要去捉住它。伸下脚去找鞋子,只找到了一只。他来不及再找另一只,赤着脚过去朝香头摇晃的地方扇了一掌,香立即又插回香炉中,静悄悄的一点动静也没有。李著明俯下身子摸遍了暗处角落,忽然有个东西飞过来正打在脸上,觉得像是鞋子,再找却又找不到。李著明便开了门下楼,喊来仆人,点上灯搜寻了一遍,什么也没有。他便又躺下睡了。天明后,李著明让几个人帮着找那只鞋,翻席倒床地找遍了,仍然找不到,主人便替他换了双鞋子。过了一天,有人偶然一抬头,见一只鞋夹在屋顶上粱椽之间,挑下来一看,正是李著明那只鞋。

李著明是益都人,在淄川县的孙家借住。孙家的房子很多,都闲置在那里,李著明只住了其中的一半。南院紧挨着一座高阁,中间只隔一堵墙。有人经常看到高阁上的门自动开了、又自动关上。李著明听说后,也不以为意。一次,他偶然和家人在院子里聊天,见高阁上的门忽然自已开了,有个小人走了出来,面朝北坐下。身高不满三尺,穿着绿色的袍褂,白色的袜子。大家一起指着他看,那小人一动不动。李著明说:“这是狐精!”急忙取过弓箭想射它。小人见了,嘴里咿呀呀发出嘲笑的声音,立即消失不见了。李著明提着刀登上楼阁,一边骂着一边搜寻,却什么也没有,只得又返回来。后来,李著明又在这里住了好几年,一直安安稳稳的,也没再发生怪异。

李著明的长子李友三,是我的亲家,这些事都是他亲眼看见的。


\subsection{1.4.30   蹇 偿 债}
\label{\detokenize{p00_u5176_u5b83/_u767d_u8bdd_u804a_u658b_u5fd7_u5f02:id162}}
李著明先生,是个慷慨乐施的人。同乡某人,当佣工住在李公家里。这个人从小游手好闲,不能干农活,家里很贫穷。不过他也有些小技能,常为李家做些杂务,每次都得到很丰厚的报酬。有时吃不上早饭,向李公哀求乞讨,李公就给他一升半斗粮食。有一天,他对李公说:“小人天天得到您丰厚的救济,一家三四口才不致饿死。然而怎可长久这样下去呢。请求主人借给我一石绿豆做经商的资本吧。”李公很高兴,立即让家里人如数给了他。某人把绿豆背走,过了一年多,也没偿还。问起他,才知道绿豆钱早已花光了。李公可怜他的贫困,也就放置一旁不再索要了。

后来李公到佛寺读书。过了三年多时间,忽然梦见某人来,说:“小人欠您的绿豆钱,今天来偿还。”李公安慰他说:“假若还要你偿还的话,那么平日所借欠的东西,怎么算得清呢?”某人忧伤地说:“的确是这样。不过若为人做了事,即使得到千金也可以不偿还;假如毫无缘故的受人资助,就是一升半斗都不容许昧下,何况更多的呢!”说完,就走了。李公更加生疑。不久家人对李公说:“夜里母驴生了一个驴驹,而且很高大。”李公忽然明白过来说:“难道这驴驹就是某人吗?”过了几天李公回家,见到驴驹,便戏呼某人的名字。驴驹听到呼唤便跑过来,就像知道是在叫它。从此以后便把这驴驹叫做某人的名字。

李公骑着驴驹去青州,衡王府的内监看见了很喜欢这驴驹,愿出高价购买,但价钱还没说定。正好李公遇到家中有急事不能等待,就回来了。又过了一年,驴驹和一匹雄马同槽吃食时,被马咬折了胫骨,不能治疗。有个牛医来到李公家里,看见了,对李公说:“请您把驴驹交给我,每天精心治疗养护,需要用些日子。万一能把它治好,卖得的钱和您平分。”李公同意按他的请求办。过了几个月,牛医卖驴驹得了一千八百钱,拿出一半给了李公。李公接受了这些钱,顿时醒悟,原来钱数恰好符合某人所借的绿豆价钱。噫!阳世欠下的债,而经阴司转生来偿还,这事足以劝人为善的了。


\subsection{1.4.31   头 滚}
\label{\detokenize{p00_u5176_u5b83/_u767d_u8bdd_u804a_u658b_u5fd7_u5f02:id163}}
举人苏贞下的祖父白天卧床时,看见一个人头从地里冒出来,像能盛五斗米的斛那样大,在床下旋转不停。他因此受惊吓而得病,终于死了。后来苏举人的叔祖因为和放荡的女人同宿,遭到杀身之祸。头滚大约便是先兆吧?


\subsection{1.4.32   鬼 作 筵}
\label{\detokenize{p00_u5176_u5b83/_u767d_u8bdd_u804a_u658b_u5fd7_u5f02:id164}}
秀才杜九畹,妻子有病。遇到九月九日重阳节,杜秀才被朋友邀请登山赴茱萸酒会。这天他早早起来,梳洗过后,告诉妻子他要去的地方,穿戴整齐就要出门。忽然看见妻子神智不清,嘴里不住地唧唧咕咕,像是在和人说话。杜秀才感到奇怪,便靠近床问她。妻子就把他当儿子来呼叫。家人心里都知道事出有因。当时杜母的棺材还未入葬,都怀疑是死人的灵魂依附到秀才妻子身上了。杜秀才祝祷说:“难道您是我的母亲吗?”妻子骂道:“畜生怎么不认识你父亲了?”杜秀才说:“既然是我父亲,为什么还要来家作祟您的儿媳呢?”妻子叫着他的乳名说:“我是专为儿媳的事来的,为什么反要怨恨我呢?儿媳本应立即死去,有四个人来勾她的魂,为首的是张怀玉。我说了无数好话哀求他放人,这才得到了允许。我许愿送点小礼,应该快给他们。”杜秀才按照吩咐,到门外烧了纸钱。妻子又说:“那四个人已经走了,他们不忍心驳我的老面子;三天后,一定要置办酒筵酬谢他们。你母亲老了,那么大年纪了不能料理饮食,到时候,还要麻烦儿媳去一趟。”杜秀才说:“阴阳不同路,怎么能代为料理?还希望父亲宽恕。”妻子说:“儿子不要怕,去去就回来。再说这也是为她自己的事情,一定不能怕劳累。”说完就昏迷了,好久才苏醒过来。杜秀才问她说过的话,她茫然不能记忆。只是说:“刚才看见四个人来,要捉我去。幸亏阿翁哀求请免,并解囊贿赂他们,才走了。我见阿翁盛钱的包袱里还剩下了两锭银子,想偷拿一锭来,用在咱家的吃饭花销上。阿翁看见了,责骂说:‘你想干什么!这东西岂是你可以用的吗?’我才抽回手没敢动。”杜秀才以为妻子病危,对她的话半信半疑。

过了三天,杜妻正在说笑的时候,忽然两眼直瞪着杜秀才看了很长时间,说:“你的媳妇太贪婪了,前次看见我的银子,就生非分之想,这大概是因为太贫困的缘故,也不足为怪。现在将要让儿媳去,为我照管厨房事务,不用担忧。”话刚刚说完,她就昏死过去,约摸半天功夫,才苏醒。她告诉杜秀才说:“刚才阿翁叫我去,对我说:‘不用你亲自下手,我这儿烹调自有人干,只要你坚持坐在那里指挥就行了。阴间喜欢东西丰满,所有的食物都要盛得满溢到器皿以外,一定要记住这些。’我答应了。到了厨房里,看见有两个妇女在用刀切东西,穿的帔衣都是黑红色,镶着绿边。她们称呼我嫂子。每当把菜肴盛到器皿中时,她们都要请我看过。先前要拘捕我的那四个人都在筵席中。菜肴端了上去,酒也全部准备好了,阿翁才让我回来。”杜秀才听后大为惊异,经常说给同人们听。


\subsection{1.4.33   胡 四 相 公}
\label{\detokenize{p00_u5176_u5b83/_u767d_u8bdd_u804a_u658b_u5fd7_u5f02:id165}}
山东莱芜的张虚一,是学政张道一的二兄。他性情豪放不受约束。听说城里某家的宅院被狐仙居住着,就郑重其事地带着名帖前往拜访,希望能见上狐仙一面。他把名帖投入大门的缝隙中,不多时,门扇自开。跟随的仆人大惊,赶紧后退。张生整理衣帽恭恭敬敬地进了门。看见堂屋里摆设着桌椅,但却寂静无人。于是望空拱手作揖说:“小生斋戒诚意拜访,仙人既然不拒我于门外,为什么不让我见一面呢?”忽然听到空屋里有人说:“有劳您大驾降临,让人十分高兴。请坐赐教。”随即见两个座位自行移动并相对摆好。张生刚刚坐下,就有一个雕花的红漆茶盘,盛着两杯香茶,悬空来到跟前。各取茶杯相对饮,虽然能听见喝茶的吸沥声,然而始终看不见那位喝茶人。饮完茶,接着摆上酒。张生细问对方的家族姓氏,回答说:“小弟姓胡氏,排行第四,随从的人称呼我为相公。”于是双方相互敬酒交谈议论,意气相投。桌上的菜肴尽是些海味山珍,非常丰盛。送酒端菜的,似乎都是些年轻的晚辈,并且人数很多。酒后张生很想饮茶,这念头刚一产生,香茶早已放置在桌子上。凡是有想要的东西,没有不应念而到的。张生非常高兴,便尽情开怀痛饮,大醉而归。自此以后他每隔三几天便去拜访胡四相公,胡四相公也经常到张家来,互相依照主客往来礼节招待。

有一天,张生问胡四相公说:“南城中的巫婆,天天托借狐仙的神术从病人家里索要好处。不知她家的狐仙,您认识不认识?”胡四相公说:“她是在说谎骗人,实际上她家并没有狐。”一会儿,张生起身去小便,听到有人小声说:“刚才您说的南城狐巫,不知是什么人?小人想跟随先生去看看,麻烦您能为我说句话,请求主人允许。”张生知道这是个小狐仆,便答应说:“行。”就在席间请求胡四相公说:“我想得到足下一两个仆人的帮助,去探视巫婆,敬请您同意。”胡四相公坚持说没有必要。张生再三要求,才被允许,随后张生出门,马自己走了过来,像有人牵引着。张生走过去骑上前行,狐仆在路上与他边走边说话。狐仆对张生说:“以后先生走在道上,如发觉有细沙散落在衣襟上时,便是我辈跟从着。”说着进了城,到了巫婆家。

巫婆见张生来,笑着迎上前去说:“贵人怎么忽然降临?”张生说:“听说你家的狐子很有灵验,是这样吗?”巫婆收起笑容严肃地说:“像这样的轻薄话,不宜贵人说!怎么随便就说狐子?恐怕我家花姊听见不高兴!”话没说完,从空中扔下半块砖来,打中了她的手臂,她晃了几下差点跌倒。便吃惊地对张生说:“官人怎么扔砖头打老身呢!”张生笑着说:“婆子眼瞎!哪曾见过自己的额头破了,却冤枉诬赖袖手人的事?”巫婆非常惊讶,不知砖头是从哪里打来的。正在疑惑不定的时候,又有一个石子落下来,打中了她,随即跌倒在地上。接着污泥纷纷往下落,把巫婆涂抹成了鬼脸,她只有哀号请求饶命。张生请狐仆饶了她,污泥才不再落。巫婆急忙爬起来逃奔到屋里,关上门不敢出来。张生高声对她说:“你的狐能比得上我的狐吗?”巫婆只得认错。张生仰起头望着空中,告诉狐仆不要再伤害巫婆了,她才提心吊胆地走出屋来。张生笑着告诫她一番,才回了家。从此张生每逢独行在路上,只要发觉尘沙落在身上,便招呼小狐仆说话,两狐仆总是应答无误。就是面对虎狼歹徒,张生也觉得有了依靠而不胆怯。

这样过了一年多,张生和胡四相公的交情更加深厚。张生曾问胡四相公的年龄,他早已记不清了,只说:“见黄巢造反,还像是昨天的事。”有天晚上两人在一起说话,忽然听见墙头上有动静,声音很猛烈。张生很奇怪。胡四相公说:“这一定是我哥哥。”张生说:“为什么不邀他来一块坐坐?”胡四相公说:“他的道业很浅,只要能抓只鸡吃便很满足了。”张生说:“交友情深,像咱两人,可以说是毫无遗憾了;但始终没能见到您的颜面,实在是令人遗恨。”胡四相公说:“只要交情深厚就足了,何必见面?”一天,胡四相公置办酒席邀请张生,并且告别。张生问道:“您要往哪里去?”胡四相公回答说:“小弟生于陕中,要回那里去。您每次都因看不到我的脸面而不满意,今天就请您见一见几年来的朋友,以后再见面时好相认。”张生四面寻找都没见到。胡四相公说:“您试开寝室的门,我就在里面。”张生按他的说法,推门一看,只见里面有一个美少年,相对而笑。他的衣裳华丽,眉眼如画,转眼之间,就再也看不到了。张生刚转身行走,就有脚步声跟随在他的后面,说:“今天算是解除了您的遗憾了。”张生依恋不忍心分别。胡四相公说:“离合自有定数,何用放在心上。”于是用大酒杯劝饮。一直喝到半夜,才用灯笼送张生回家。等到天明再去探望时,胡宅早已成了冷落的空房子。

后来张道一先生官任西川学使,而张虚一却还像原先那样清贫。因此张虚一前往西川去看他弟弟,抱的希望很大。可是只过了一个月就返回去了,很不如当初的心愿,边走边在马上叹息,垂头丧气就像个木头人。忽然有一个少年骑着黑色的马驹,跟随在他的身后。张生回头看了看,见少年衣着非常华丽,风度潇洒文雅,便和他闲谈起来。少年见张生不痛快,就问他。张生于是叹息着把原因告诉了他。少年听说后也用好话安慰他。二人同行了一里多路,到了岔道口,少年这才拱手道别说:“前边路上有一个人,将把您的老友送给您的礼物转交给您,请能收下。”再想问时,他已赶马径直奔驰而去。张生解不开这个谜,又走了二三里地,看见一个老家人,手持一个小竹箱子,把它献到了马前,说:“这是胡四相公敬送给先生的。”张生这才恍然大悟。接过来打开一看,原来是满满的一箱白银。等到再看老家人时,却早已不知去向了。


\subsection{1.4.34   念 秧}
\label{\detokenize{p00_u5176_u5b83/_u767d_u8bdd_u804a_u658b_u5fd7_u5f02:id166}}
异史氏说:“人世间暗中害人的伎俩,到处都有;而南北交通要道上,此害尤其严重。像那些手持武器乘着快马,在郊外抢掠行人财物的,人人都知道;还有的割裂口袋刺破行李,在城里夺取财物,行人回头,而钱财货物已空,这不是害人伎俩中最厉害的行径吗?又有萍水相逢,甘言如美酒的人,他来得既不突然,和人也特别亲近,可一旦误认作好朋友,马上就遭受丧失资财之害。他们随机应变设置陷阱,变化多端。因为这种人专用甜言蜜语令人上当而行骗,民间起名叫做‘念秧’。如今北面路上这样的人不少,遭受他们祸害的人也特别多。”

我的同乡王子巽,是县里的秀才。因有个同族长辈在京城作旗籍太史,他要前去探望。整理好行装北上,出了济南,走了几里路,有一个骑着黑驴的人赶上来和他同行。这人不时地说些闲话引他,王生便和他搭上了话茬。这人自己说:“我姓张,是栖霞县的衙役,受县令大人派遣去京城出差。”他对王生称呼很谦逊,恭恭敬敬地非常殷勤。两人同行几十里,并约好了一起住宿。一路上若王生走得快了,张某就加鞭赶驴追上;若王生落在了后面,张某就在前边停下来等他。王生的仆人很怀疑张某,就非常严厉地赶他走开,不让他前后跟从。张某自觉得很羞愧,于是挥鞭走了。到了傍晚,王生住进一家旅店,偶然经过门前,见张某在外舍饮酒。正在惊疑的时候,张某也看见他,便起身垂手拱立,谦虚得像奴仆一样,并略作问讯。王生也很随便地和他应酬,没有怀疑他,然而仆人却整夜防备着他。鸡叫的时候,张某来招呼王生一起走,仆人呵斥拒绝,于是他便自己走了。

太阳已经出来了,王生才上路。走了半天时间,见前边有一个人骑着头白驴,年纪约四十开外,衣帽整洁;他的头眼看就要低垂到驴身上,瞌睡得像要掉下驴来。他一会儿走在王生的前头,一会儿走在王生的后头,始终不离地走了十几里地。王生很奇怪地问他道:“你夜里干什么了,竟然迷糊成这个样子?”这人听了,猛然伸了伸懒腰,说:“我是清苑人,姓许,临淄县令高檠是我表兄。我哥哥在表兄府上设帐教书,我去看他,得了一点馈赠。今夜在旅店,误同念秧的住到了一起,一夜警惕没敢合眼,困得大白天迷迷糊糊。”王生问他:“念秧是怎么一回事?”许某回答说:“您出门在外少,不知人的险诈。如今有些坏人,用甜言蜜语引诱行人旅客,攀附拉拢和他们一同住宿,从而乘机欺骗钱财。昨天有个远房亲戚,就因为这而丢了盘缠。咱们都得警惕防备。”王生听了点头称是。原先,临淄县令和王生有旧交,王生曾经去过他的官府,认识他家的门客,其中果然有姓许的,于是便不怀疑,和许某寒暄起来,还问了他哥哥的近况。许某相约天晚了同住一家旅店,王生答应了他。而仆人始终怀疑许某是伪装的,就暗暗地和主人商量好,慢慢落在了后边不再往前走,与许某的距离越拉越远,终于看不见了。

第二天,中午时分,王生又遇到一个年轻人,年纪约有十六七岁,骑着一匹健壮的大骡子,穿戴华丽整洁,模样长得很秀美。他们一同走了很长时间,没有互相说过话。太阳已经偏西了,年轻人忽然说:“前面离屈律店不远了。”王生轻声应着。年轻人于是唉声叹气,像是不能忍受的样子。王生略微问了一下原因,年轻人叹了口气说:“我是江南人,姓金,三年苦读,盼望能够考试得中,不料想竟然名落孙山!我哥哥在京城任部中主政,我便带着家眷来,希望能排解心中的郁闷。但我从来没有走过远路,尘沙扑面,令人烦恼。”说着便取出红手帕擦险,叹气不已。听他说话是南方口音,柔美婉转得像女子。王生心里喜欢他,慢慢用好话安慰。金某说:“刚才我先走了一步,家眷这么长时间还没跟上来,仆人们怎么也没有赶到呢?天都快黑了,怎么办!”他停留观望,走得很慢。王生于是先走,和金某越离越远。

王生晚上到客店住宿,进入房间一看,靠墙下有一张床,见先有别人的行李摆在了上面,便问行李的主人。立即有一个人,携起行李往外走,说:“请尽管安排,我这就搬到别的屋里去。” 王生看了看他,原来是许某。就让他留下同住一屋,许某便不走了。于是两人坐下交谈起来。过了一会儿,又有一个人携带行李进来,见王、许二人在屋里,返身就往外走,说:“已经有客人住了。”王生仔细一看,原来是路上遇到的年轻人金某。王生没说话,许某急忙起来拉他留下,金某也就坐了下来。许某于是问起了他的家族姓氏,金某又用在路上对王生说过的话说给许某听。过了片刻,金某解开口袋取出银子,堆了很多;称了一两多,交给店主人,嘱咐治办肴酒,作为夜里聊天用。王、许二人争相劝阻,金某不听。不久,酒肉都摆上桌来。筵席上,金某谈论诗文显得很风雅。王生问起江南考场中的试题,金某全都说给他听,并且背诵自己八股文的破题承接,以及篇章中的得意之句,说完,显得心里很不平气。王、许也都为他惋惜。金某又因家眷走失,夜里没有仆人,担心自己不懂怎样喂牲口。王生便让自己的仆人替他给骡子拌上草料,金某非常感谢。

过了不多时,金某忽然顿足生气地说:“命运不顺,出门也遇不到好事。昨天夜里住旅店,和恶人住到了一起,他们赌博掷骰子叫喊,吵得耳朵难受心里烦躁,一夜没睡着。”南方口音把 “骰”字说成“兜”,许某听不明白,问他是什么东西。金某用手比划骰子的形状。许某便笑着从口袋里拿出一枚骰子来,说:“是这种东西吗?”金某答应 “是”。许某就用骰子行酒令,三人很高兴地喝起来。酒喝得差不多了,许某提议大家都掷骰子,赢个东道主。王生推辞不懂,许某便和金某掷骰呼喊赌了起来。许某又偷偷地嘱咐王生说:“您不要走漏了话。这个南方公子很富裕,年纪又小,不一定懂得赌博的诀窍。我赢他些银子,明天一定请您的客。”许某和金某于是进了隔壁房间,不久听到里面几个人聚赌的声音很热闹。王生暗暗地过去瞅了瞅,见栖霞县的衙役张某也在其中。王生大为惊疑,便展开被子自己先躺下了。又过了一会儿,众人都来拉王生去赌博,王生坚决推辞说不会。许某愿代替王生辨认输赢,王生还是不同意,二人便硬替王生掷骰。不多时,许某走到床前向王生报告说:“你赢了若干筹码了。”王生在睡梦中答应着。

突然有几个人推门进来,叽哩咕噜地讲着外族语。领头的说是姓佟,是满族旗人专门巡逻捉拿赌徒的。当时禁赌的法令很严,人们都非常惊慌。佟某大声恐吓王生,王生也以旗籍太史的旗号来抵挡。佟某的态度缓和下来,和王生叙起了同籍,笑着让众人继续玩赌博的游戏。大家果然再次赌起来,佟某也参加了。王生对许某说:“胜负我不想知道,只愿睡觉,请不要打扰。”许某不听,仍然反复地来向王生报告。到了最后散局的时候,各人计算所得的筹码数,王生输了很多,佟某便搜王生钱袋中的银子取偿。王生愤怒地起来和他争夺,金某捉住王生的胳膊偷偷地说:“他们都是些坏人,居心叵测。咱们毕竟是文字交,没有不互相照顾的道理。恰好赌局上我赢了不少,可以相抵。这些钱本来应由许君偿还我。现在请变换一下,就让许君偿还佟,您来偿还我。这样做不过是暂时掩人耳目,等过了今晚仍再原数相还。凭着咱们的道义之交,总不会就真拿您的钱吧?”王生本来就忠厚,相信了他的话。金某出去,把相互变换的办法告诉了佟某,这才当着众人的面打开王生的钱袋,把银子如数装进了自己的腰包。佟某便转而向许、张两人讨了钱去了。

金某于是抱着铺盖来,和王生连枕睡一头,他的被褥都很精美。王生也招呼仆人睡到床上,各人都安然就枕不再说话。过了很长时间,金某故意转侧身体,把臀部靠近仆人。仆人移身躲避,金某又靠近他。当触及金某滑腻如脂的臀部时,仆人心动,便和他亲热起来;而金某更加殷勤周到。被子响动的声音,王生都听到了,虽然很惊奇,但始终也没怀疑有别的事。天刚拂晓,金某就起床,催促一同早走。并且说:“看您的驴体弱疲惫的样子,昨夜寄存的银子,等到前边再交给您吧。”王生还没有说话,金某已把行李装好登上了大骡子。王生不得已,只好跟着他上路。骡子走得很快,渐渐地走远了。王生以为金某一定会到前边等着他,最初也没在意。就以夜里听到的动静问仆人是怎么回事,仆人如实告诉了他。王生这才大惊说:“今天被念秧的骗了!哪有官宦家的名士,而自荐给养马仆人的?”又转念一想金某谈词风雅,不是念秧之人所能办到的。急追了几十里路,一点踪迹也没寻到。直到这时王生才明白:张、许、佟都是同伙,一局不行,又换一局,务必使自已进入圈套。夜里逼迫交换偿债,已经埋伏了一个企图抵赖的机会;假若天明驮银子先走的计谋不行,也必定会借口偿还赌债硬是强夺而去。为了几十两银子,曲折跟随几百里;恐怕仆人揭发这个阴谋,而又以身和他交欢,他们的手段也可说是用心良苦了。

过了几年,又出现了吴生的事情。淄川县有个姓吴的书生,字安仁。三十岁死了妻子,一人独睡空房。有个秀才常来和他交谈,于是认作知己,非常高兴。秀才的小仆人,名叫鬼头,和吴生的僮仆报儿也很要好。时间长了才知道鬼头是个狐,吴生出远门,总要带他一齐去,同在一间屋子里,别人却看不见他。吴生有次客居京城,将要回家的时候,听说王生遭了念秧的祸害,因此告戒僮仆要警惕防备。狐仆笑着说:“没有必要,此行并无不利的事情。”到了涿州,见有个人系马坐在烟店里,穿着很高贵的裘皮服装。这人看见吴生过去了,也起身跳上马跟随在后面;渐渐地和吴生说上了话,他自己说:“我是山东人,姓黄,在户部任提堂。今将东归,很高兴咱们同路,不至孤单寂寞。”于是吴生住下他也住下,每次都一起吃饭,并且总是替吴生偿还饭钱。吴生表面上感谢,背地里却怀疑他,偷偷地以此问狐仆,狐仆只是说道:“不妨。”吴生的疑心便消除了。

到了晚上,一同找到旅店,见有位美少年先坐在里面。黄某进去,和少年拱手行礼,高兴地问他:“什么时间离开京城的?”少年回答说:“昨天。”黄某便拉他住在一起,对吴生说: “这是史郎,我的中表弟,也是文人,可以陪您谈论诗文,夜里闲谈肯定不会冷落。”就取出银子,治办酒肴一起畅饮。史某风雅含蓄,谈吐不凡,和吴生互相都很喜爱。饮酒时,史某总是使眼色暗示吴生行酒令作弊,惩罚黄某,强迫他用大杯喝酒,然后鼓掌大笑。吴生更加喜欢他。不久史某和黄某商量要赌博,拉吴生参加,于是各人都拿出钱袋里的银子作抵押。狐仆嘱咐报儿暗中锁好门扇,并叮嘱吴生说:“倘若听到人声喧哗,只管睡觉不要出声。”吴生答应了。吴生每次掷骰,小赌注就输,大赌注就赢。过了一更多时辰,他计算着已赢了二百两银子。而史和黄的钱袋却都空了,商议着再拿黄的马作抵押。忽然听到敲门声非常猛烈,吴生急忙起身,把骰子投进火里,蒙上被子躺下装睡。过了好久,听见主人找不到钥匙,砸锁拔闩,有好几个人气势汹汹地进来,要搜捕赌博的人。史和黄都说没有。其中一人竟然掀开吴生的被子,指着吴说是赌博人,吴生大声喝叱他。好几个人强行检查吴生的行装,眼看无法和他们抗拒的时候,突然听到门外有大官出行鸣锣开道的吆喝声。吴生急忙跑出去呼喊,众人这才害怕,硬把吴生拉回来,只求他不要出声。吴生于是从容地包裹好行装交付店主人。听到官府的仪仗走远了,众人这才出门离去。黄和史某都作出很惊喜的样子,随后相继找地方休息。黄某让史某和吴生睡一个床铺。吴生把盛钱的袋子放在枕头下,这才打开被子躺下。不多时,史某掀开吴生的被子,裸体投入吴生的怀里,小声说:“爱慕兄长的磊落,愿意和您交好。”吴生心里知道他的诡计,但也认为这是个难得的好机会,于是互相偎抱在一起。史某殷勤地和吴生周旋,然而却受不了吴生的折磨,便呻吟哀求饶恕。吴生毫不留情,直到史某的下体鲜血崩流,才放他离去。到了天明,史某疲惫不能起床,借口说突然病了,只是请吴生和黄某先上路。吴生临走时,送给史某银子作医药调养费用。路上和狐仆说起来,才知道夜里的官府仪仗,都是狐仆假装的。

黄某在途中,更加讨好吴生。傍晚又同住一屋,这小屋很狭窄,仅能容下一张床,非常暖和洁净,而吴生却嫌床太小。黄某说道:“这床睡两人是稍窄点,您自己睡就宽松多了,有什么关系呢?”吃过饭后他就走了。吴生也很希望独睡,这样可与狐友在一起。坐了很久,狐仆没来。忽然听见墙壁上的小门外,有用手指弹敲的声音。吴生拨开门闩探望,一个妆扮艳丽的女子急速进来,自己把门闩上,向吴生露出笑容,美得如同仙女一般。吴生高兴地询问她的来历,原来是店主人的儿媳。于是和她亲热起来,非常喜爱她。女子忽然流下眼泪,吴生惊问她悲伤的原因,女子说,“不敢隐瞒,我实际上是主人派来引诱您的。原先让我引诱别人的时候,我一进屋,就会被主人关门逮住,不知今晚为啥这样久了还没来到。”随后又呜咽着说:“我是良家女子,并不甘心这样做。现在我已经倾心爱慕您,请求您能搭救我!”吴生听说,非常惊恐,别无办法,只有让她赶快离去。女子只是低头哭泣。忽然听到黄某和店主砸门吵闹,声如鼎沸。只听黄某说:“我一路上敬奉着,说你是正人君子,怎么竟引诱我的弟媳!”吴生害了怕,逼着女子离去。这时听到墙壁上的小门外也有撞击声。吴生心慌意乱汗流如雨,女子也伏身哭泣。又听见有人劝止店主,店主不听,推门越急。只听劝解的人说:“请问主人的意思想要怎么办?如果想杀人吗?有我们旅客数人,必定不会坐视逞凶。如果他们两人中有一个逃走的,让他抵罪时怎么说?如果想对质公堂吗?家庭淫乱之事,只能自己丢人。况且是你们自己宿在旅客房间的,明明是陷害诈骗,怎能保证女子不说实话?”主人瞠目结舌答不上来。吴生听见这些话,暗暗地感激佩服,然而却不知道说话的人是谁。

起初,店门将要关闭的时候,就有一个秀才和仆人,来到外房住宿。他们携带着好酒,邀请同屋的人共饮,对黄某和店主劝酒尤其殷勤。黄某与店主两人告辞想起身,秀才牵着他们的衣襟,苦苦挽留不让去。后来他俩抽个空子悄悄溜了出来,拿着棍棒奔向吴生的住房。秀才听到喧闹声,这才过来劝解。吴生伏在窗户上一看,原来是狐仆鬼头,心里暗喜。又见店主的气焰被压去了许多,于是说大话来恐吓他,便对女子说:“你为什么默不作声?”女子哭着说:“我恨自己不如人,被人逼迫干这种下贱的事情!” 店主听说,面如死灰。秀才叱骂道:“你们的禽兽行为,也已经彻底败露了,这是我们所有的旅客都愤恨的!”黄某和店主都放下刀棍,跪在地上请罪。吴生也开门出来,顿足大声怒骂。秀才又劝说吴生,双方这才和解。女子哭哭啼啼,宁死不归。后院里跑出几个老妇人和丫头来,抓着女子往里拖,女子趴在地上哭得更加哀痛。秀才劝说店主把她重价卖给吴生,店主低着头说:“我这是当老娘三十年、今日竟包反了孩子,还有什么可说呢!”就依了秀才说的话。吴生硬是不肯出大价钱;秀才为双方调和,商定卖五十两银子。人钱两相交付后,晨钟已响。于是大家都急忙整理行装,载着女子上路。

女子从未骑过马,路上奔波非常疲乏。中午时分稍微休息了一下。将要走的时候,喊童仆报儿,却不知他到哪里去了。太阳已经偏西了,还没见到踪影,感到很奇怪,就问狐仆。狐仆说: “不用担忧,他会自己回来的。”星星月亮出来时,报儿才来到。吴生问他干什么去了,报儿笑着说:“公子拿五十两银子肥了奸诈小人,我心里很不平。刚才和鬼头商订计谋,返回去索要回来了。”于是把银子放到桌子上。吴生惊问其中的缘故,原来鬼头知道女子只有一个哥哥,出了远门十几年没有回来,于是变成了她哥哥的形状,让报儿冒充弟弟,进店里向店主人要人。店主害怕,假说女子已经病死。他们二人要去告官,店主更加害怕,便用银子贿赂他们,逐渐增加到四十两,他们二人才走。报儿一五一十地讲述了事情的经过。吴生立即把银子送给了他。

吴生回到家中,和女子的感情非常好。家里也更加富有。细问女子,才知道先前的美少年史某就是她的丈夫,原来史某就是金某。她穿的一件槲绸披肩,说是从山东一个姓王的人那里得来的。他们党羽很多,那个客店主人也都是他们的同伙。怎会料到吴生所遇到的,正是王子巽连天叫苦的那些人,这不也是件让人高兴的事吗!古人说:“骑者善堕。”真是可信啊!


\subsection{1.4.35   蛙 曲}
\label{\detokenize{p00_u5176_u5b83/_u767d_u8bdd_u804a_u658b_u5fd7_u5f02:id167}}
王子巽说:“在京城时,曾见到一个人在街市上演杂耍。他带着一个木盒子,里面做了许多格子,共有十二个孔。每个孔格里都趴伏着一只青蛙。演杂耍的用细棒敲青蛙的头,青蛙就呱呱鸣叫。若有人给他钱,他就乱敲青蛙的头顶,如同打击云锣奏出的乐声,五音曲调,一一可辨,听得清清楚楚。”


\subsection{1.4.36   鼠 戏}
\label{\detokenize{p00_u5176_u5b83/_u767d_u8bdd_u804a_u658b_u5fd7_u5f02:id168}}
子巽又说:“有一个人,在长安街市上卖艺,表演鼠戏。他背上背着一个口袋,里面装着十余只小老鼠。每当到了人多的地方,就拿出一个小木架,放在肩膀上,很像一座戏楼的样子。接着就拍打着鼓板,唱起古代的杂剧。歌声刚出口,就有小鼠从口袋里出来,蒙着假面具,披挂着小戏妆,从卖艺人的背后登上戏楼,像人一样站立着舞蹈。而且表演的男女悲欢之情,和卖艺人唱的戏文情节完全吻合。”


\subsection{1.4.37   泥 书 生}
\label{\detokenize{p00_u5176_u5b83/_u767d_u8bdd_u804a_u658b_u5fd7_u5f02:id169}}
淄川罗村有个叫陈代的人,从小就愚笨丑陋。娶了个妻子某氏,却很美貌。她因为自己的丈夫不如人,郁郁寡欢,很不如愿。但是她能保持贞操清白,婆媳之间也相安无事。

一天晚上她独自一人睡在屋里,忽然听到风把门吹开了,一个书生进来,脱了衣帽,和她同床而卧。妇人害怕,苦苦用力抗拒。然而浑身顿时瘫软,听任书生轻薄而去。此后书生每晚上都来。过了一个多月,妇人面容憔悴,身体困乏。婆母感到奇怪,就问她。妇人起初羞惭不想说;再三追问,才把实情说了出来。婆母害怕地说:“这是个妖怪!”便想方设法禁止,最终也没能杜绝。

于是婆母就让陈代藏在屋里,手持木棒等候着。到了半夜,书生果然又来了,把帽子放在桌子上,又脱下袍服,搭到衣架上。才要登床时,忽然大惊道:“哎呀!有生人气!”急忙再去披衣。陈代从暗中突然跳出来,挥棒打中书生的腰胁,只听到嗒的一声,再四下一看,书生已经没了踪影。拿把柴草点火一照,看见有一片泥衣掉在地上,桌子上的泥帽仍然放在那里。


\subsection{1.4.38   土 地 夫 人}
\label{\detokenize{p00_u5176_u5b83/_u767d_u8bdd_u804a_u658b_u5fd7_u5f02:id170}}
淄川窎桥村有个叫王炳的人,出村时,看见土地庙中出来一个美女,顾盼王炳,情意殷切。王炳用猥亵的语言调戏她,她却很乐意接受。两人想亲热一番没有合适的地方,就约好夜里幽会。王炳于是把自己居住的地方告诉了她。

到了夜里,美女果然来到,两人欢爱异常。王炳问她的姓名,美女不肯说。从此往来不绝。有时王炳和妻子同床,美女也必定来找他作乐,而王妻竟然感觉不到。王炳惊奇地问美女,美女说:“我是土地的夫人。”王炳非常害怕,屡次想拒绝和她往来,但是任何办法都不能阻挡。就这样过了半年,王炳的身体病乏得不能起床了。美女来得更加频繁,家里的人都能看得见她。没有多久,王炳真的死了,美女还每天来一次。王妻叱骂她说:“你这淫鬼好不害羞!人都已经死了,还来干什么?”美女于是离去,再没来过。

土地爷虽小,也是神,岂有让妻子私奔的?就是糊涂也不至如此。不知是什么淫昏东西,竟使千古之后说这个村里有肮脏下贱不严谨的神,真是冤屈啊!


\subsection{1.4.39   寒 月 芙 蕖}
\label{\detokenize{p00_u5176_u5b83/_u767d_u8bdd_u804a_u658b_u5fd7_u5f02:id171}}
济南有一个道士,不知他是什么人,也不知他姓甚名谁。无论冬夏,总是穿件夹衣,腰上系条黄带子,此外再不穿别的衣服。常用一把半截梳子梳头,梳完,把头发挽成个发髻,用梳子别起来,像戴着个帽子一样。道士天天赤着脚在市上游逛,夜里就睡在街头,身体周围几尺以外的冰雪都融化得干干净净。

道士刚来济南的时候,常给人表演魔术,街上的人都争着送他食物。有个市井无赖,送给他一些酒,想跟他学魔术,道士不肯。一次,无赖正好碰上道士在河里洗澡,便突然抱走了他的衣服,以此要挟他。道士向他作揖说:“请你还给我衣服,我一定不吝惜自己的这点小法术。”无赖怕他骗自己,抱着衣服不肯放下。道士说:“你真不还我吗?”无赖说:“不还!”道士默默地不再说话。一会儿,忽然见那条黄带子变成了一条大蛇,有几把粗,绕着无赖的身子缠了六七圈;又昂起头,嘴里吐着红信子,怒目瞪着无赖。无赖大吃一惊,急忙跪倒在地,脸也吓青了,气也喘不过来了,嘴里连喊饶命。道士一把抓过那条黄带子,竟然不是蛇。另有一条蛇,蜿蜿蜒蜒地爬进城去了。

从此后,道士更加出名。那些官绅家听说了他的奇异本领。都把他请了去,与他交往,从此道士不断出入于富贵人家。连司、道的长官都听说了他的名气,每次宴会,也总是把他请了去。

一天,道士声称要在大明湖水面亭设宴,回请各位长官。到了那天,每一个被请的客人都在自己的桌子上得到一份请帖,但谁也不知请帖是怎么送来的。客人们如约赶到设宴的地方,道士躬着腰,恭敬地出来迎接。走进亭子一看,什么都没有,静悄悄的,连桌椅都没设。大家怀疑道士在说谎骗人。道士对几个官员说:“贫道没有仆人,想借借你们的随从,来帮帮忙。”官员们都答应了。道士便去一面墙壁上画了两扇门,然后用手敲敲,墙里面竟传出了答应声,接着是开锁声,哗啦一声,门敞开了。大家一起往里瞧去,见里面影影绰绰地有好多人正来回奔忙,屏风帐幔、床榻桌椅一应俱全。有人不断地把这些东西递出来,道士命官员的随从们接过来排列在亭子里,还嘱咐他们不要和里边的人讲话。双方传递东西时,只是互相打量着笑笑。不一会儿,亭子里便摆满了,用具都极为华丽。接着,又从门里边递出散发着阵阵香味的美酒和热气腾腾的佳肴。客人们见了,无不惊骇诧异。

水面亭本是背靠湖水的。每当盛夏六月时,几十顷湖面盛开荷花,一望无际。道士开宴时,正值隆冬,从窗户里往外望去,绿色的湖水一片茫茫,只有清波在荡漾而已。一个客人偶然叹息着说:“今天的盛会,可惜没有莲花点缀!”大家都有同感。过了会儿,一个穿青衣的仆人奔跑进来说:“荷叶长满池塘了!”满座人吃惊,推开窗子往外一望,果然满眼都是绿葱葱的荷叶,中间夹杂着数不清的荷花苞。转瞬间,千万朵荷花一齐怒放,严寒的北风吹来,送来了沁人肺腑的荷香。大家都大感惊异,便派了一个仆人荡着小船去采些莲子来。远远看见仆人进了荷花深处。过了不久,仆人返回来,空着两手回话。官员问他怎么没采到,仆人说:“小人驾着船去,见荷花总是在前面隔得很远。一直划到北岸,又见荷花远远地开在湖的南面。”道士笑着说:“这不过都是幻梦中的空花罢了。”不久,酒宴结束,荷花也凋谢了。一阵北风吹来,将一片残荷败叶全都吹倒在水中,再也看不见了。

客人中有个济东观察,很喜欢道士的法术,将他请到官衙中,天天玩乐。一天,这位观察与客人一起喝酒,他有种家传好酒,每次请客,最多一斗,不肯让客人多喝。这天,客人喝了酒后,觉得味道很美,喝完一斗,还要再喝。观察执意不许,说酒快没有了。道士便笑着对客人说:“你一定要过足酒瘾,跟我要好了!”客人请他拿酒。道士取过酒壶,塞进袖筒里;一会儿拿出来一看,满满一壶,给在座的都斟上。壶里的酒与观察家的酒味道没什么两样。于是大家尽欢而散。观察起了疑心,客人走后,忙去看看自家的酒坛子,见坛口上依旧封得很严实,抱起来一摇,空空的,一点酒也没有了。观察既羞愧又愤怒,把道士抓了起来,说他是妖怪,命人用棍子痛打。棍子刚打到道士身上,观察便觉得屁股一阵剧痛;再打,屁股上的肉像要裂开一样。道士装模作样地在台阶下声嘶力竭,观察屁股上的血却已染红了座椅。观察只得命令不要打了,将道士赶了出去。

从此道士离开了济南,不知去到哪里。后来有人在金陵遇上他,还和在济南时一个打扮。问他话,笑而不答。


\subsection{1.4.40   酒 狂}
\label{\detokenize{p00_u5176_u5b83/_u767d_u8bdd_u804a_u658b_u5fd7_u5f02:id172}}
缪永定,是江西的拔贡生。平素爱酗酒,亲戚朋友都吓得躲避他。缪生偶尔有事到族叔家里,因他为人滑稽爱开玩笑,族叔家的客人便和他谈起来,很喜欢他,于是大家一起畅饮。缪生喝醉了,使酒性辱骂同席的人,得罪了客人。客人生气,整个酒席大乱。族叔出面左右劝解,缪生说偏袒了客人,又更对族叔发起怒来。族叔没有办法,只好跑去告诉他家。家里来人,把缪生扶回家中。才放到床上,他的四肢全都凉了,摸了摸,竟然气绝了。

缪生死后,有个戴黑帽子的人把他拘捕了去。一会儿,来到一处官府,房顶都是浅青色的琉璃瓦,人世间没见有这样壮丽的。到了高台下,好像是要等候见官。缪生自想没犯什么罪,一定是因为客人告发了酒后斗殴的事。回头看黑帽人,他怒瞪着两眼像牛一样,又不敢问。然而自己认为贡生和人发生争吵,或许犯不了大罪。忽然大堂上一个官吏宣布说,让打官司的人明日早来等候。于是堂下的人纷纷扬扬像鸟兽那样散去。缪生也随着黑帽人走了出来,又没有地方去,只好缩着头站在一家店铺的屋檐下。黑帽人生气地说:“你这酒狂无赖子!天快黑了,各人都去找地方吃饭睡觉,你到哪里去?”缪生战战兢兢地说:“我至今还不知道是怎么回事,并没告诉家里的人,所以没有一文钱,难道还有地方去吗?”黑帽人说:“你这酒狂无赖!若是买酒自己吃,就有钱了!要再胡说,我用老拳砸碎你这狂骨头!”缪生低下头不敢再作声。

忽然有一个人从门内出来,看见缪生,惊奇地说:“你怎么来了?”缪生一看,原来是他的母舅。母舅贾某,早已死了好几年了。缪生见了他,才恍然大悟自己已经死了,心里更加悲痛害怕,向贾某哭着说:“阿舅救我!”贾某回头对黑帽人说:“东灵不是外人,请来寒舍说话。”儿人于是进门。贾某又给黑帽人作揖,并且叮嘱他要多加关照。不多时,摆上酒菜,围坐着喝起来。贾某问:“我的外甥发生了什么事,竟麻烦您去勾他的魂来?”黑帽人说:“大王要去和太上老君会面,遇到您的外甥在狂骂,叫我把他抓来了。”贾某问他:“见到大王没有?”他回答说:“因为太上老君正好遇上花子案,大王还没回来。”贾某又问:“我的外甥将会判什么罪?”黑帽人回答说:“还很难知道。不过大王很生这类人的气。”缪生在旁,听见两人说的话,吓得汗水流了下来,连酒杯筷子都举不起来了。过了一会儿,黑帽人站起来,感谢贾某说:“吃这么丰盛的酒宴,已经醉了。就把令甥先交付给您。等大王回来了,再容我来拜访。”说完就走了。

贾某对缪生说:“外甥别无兄弟,父母对你爱如掌上明珠,责备一次都不忍心。你十六七岁的时候,每喝上三杯后,就嘟嘟囔囔地找人家的毛病,小不合心意,就砸门谩骂。那时还可以说你年纪小,不想分别十几年了,你一点也不长进。如今将怎么办!”缪生伏在地上哭着,只是说后悔已经来不及了。贾某拉起他来说:“我在这里开酒店,很有点小名望,定当为你竭尽全力。刚才那个黑帽子是东灵使者,我常请他喝酒,和我很要好。大王每天的事情多以万计,也未必就能记着你。我婉转地和东灵使者说说,央求他看在个人的交情上放你回去,也许能够答应。”立刻又转念说:“这个责任很重,没有十万不能办成。”缪生感谢不已,表示由自己来承担费用,贾某答应了他,缪生也就在舅舅家里住下了。第二天,黑帽人早早来察看。贾某请他密商,谈了一会儿,来对缪生说:“谈妥了,等一会他就再回来。我先拿出所有的钱,用来压契约,其余不够的钱等你回去再慢慢凑足送给他。”缪生高兴地问:“一共需多少?”贾某答:“十万。”缪生说:“我到哪里弄这些钱?”贾某说:“只需要金币钱纸一百挂,就足够了。”缪生高兴地说:“这太容易办了。”

等到将近中午的时候,黑帽人还没来到。缪生想去街市上稍微走走看看。贾某叮嘱他不要走远了,缪生答应着出了门。看到街市上的商贩贸易,如同人世间一样。到了一处地方,见高高的围墙上安装着棘刺,像是一座监狱。对门有个酒馆,很多人纷纷往来进出。酒馆外是一条长溪,黑水涌动,深不见底。正要站住窥探,就听到酒馆里有人招呼道:“缪君怎么来了?”缪生急忙看去,原来是邻村的翁生,是他十年前的旧文友。翁生走出来与缪生握手,高兴得像生前那样,就约到里面喝起酒来,谈起了两人分手后的情况。缪生庆幸将要复生,又遇到了旧友,便开怀痛饮。他喝得酩酊大醉,顿时忘记自己已死,旧态复发,渐渐地絮叨挑剔起翁生的毛病来。翁生说:“几年不见,你怎么还像以前的老样子?”缪生向来讨厌别人说他酒后的毛病,听到翁生的话,更加愤怒,便砸桌子跳骂。翁生斜了他一眼,拂袖而去。缪生追到长溪的边上,伸手去抓翁生的帽子。翁生生气地说:“这真是个不讲理的人!”便把缪生推落到溪水中。溪水并不太深;然而水中尖锐的刀子多如麻杆,穿透了缪生的胁下和小腿,固定住不能动,一直疼到骨髓。黑水中拌杂着粪便等脏东西,随着呼吸灌入咽喉,更受不了。岸上笑着围观的人像堵墙,并无一人伸手救他。正在危急的时候,贾某忽然来到,看见缪生,大为吃惊,便把他扯出来拖回家去,说:“你没有治了!死了还不觉悟,不配再作人!请你仍旧跟着东灵使者去受斧刑吧。”缪生异常恐惧,哭着说:“我知罪了!”贾某这才说:“刚才东灵使者来过,等候你来立契约,可你却在外面纵饮游荡不归。而他很忙不能再等,我已经立了契约,付钱一千让他走了;其余的钱,以旬末为期限。你回去后,应当赶快想法筹办,夜里到村外旷野,叫着我的名字烧了它,许下的这个愿就可以了结了。”缪生全都答应了他。贾某于是催促缪生上路,送他到郊外,又叮嘱说:“务必不要背弃诺言连累我。”这才指示路途让他回家。

当时缪生已经僵卧了三天,家里人都说他醉死了,然而鼻子里的气息还隐隐约约的像悬丝一样。缪生这一天苏醒后,大吐一场,吐出黑汁好几斗,臭不可闻。吐完,汗水湿透了褥子,身体才觉得清爽。他把这些奇异的事情告诉了家里的人。立即觉得刺伤的地方疼痛肿胀,隔了一夜成了疮,还幸好没大溃烂,到第十天上渐渐能够拄着棍子行走了。家里人都求他偿还阴间的欠债,缪生计算了一下所用的钱,没有几两银子不能办成,心里很吝惜,说道:“过去也许是醉梦中的幻境罢了;就算是真的,东灵使者因为是私自放我,怎么敢再让冥王知道?”家里人劝他,不听。然而缪生心里很警惕,不敢再纵饮。邻里乡党都喜欢他的进步,便稍稍和他在一起同饮。

过了一年多,缪生把阴间的报应渐渐忘记了,胆子慢慢大起来,旧态也渐渐萌发。一天,缪生在同姓晚辈家里饮酒,又骂同席的主人。主人把他赶出门外,关上大门径直回去。缪生吵骂多时,他的儿子才知道,来到把他扶持回家。缪生进屋,脸朝墙壁跪在地下,自己叩头无计其数,说:“这就还您的债!这就还您的债!”说完,便倒在地上。看了看他,已经气绝了。


\section{1.5   卷 五}
\label{\detokenize{p00_u5176_u5b83/_u767d_u8bdd_u804a_u658b_u5fd7_u5f02:id173}}

\subsection{1.5.1   阳 武 侯}
\label{\detokenize{p00_u5176_u5b83/_u767d_u8bdd_u804a_u658b_u5fd7_u5f02:id174}}
阳武侯薛禄,是胶东薛家岛人。他的父亲薛公非常贫穷,为本乡官宦人家放牛。这家有块荒地,薛公在那里放牛时,常见蛇和兔子在草丛中相斗;以为是块不同寻常的风水宝地,于是向主人请求要来作墓地,并盖了间茅草房居住着。后几年,薛公的妻子临产,当时大雨突降,恰巧有两个指挥使奉命稽查海路,经过这里,就到薛家屋里避雨。看见房顶上乌鸦、喜鹊成群地聚集在上面,争着用翅膀覆盖漏雨的地方,觉得很奇怪。一会儿薛公从里屋出来,指挥问道:“刚才你在干什么?”薛公便把妻子生孩子的事告诉了他们。又问生了个什么孩子,薛公答道:“是个男孩。”指挥更加惊愕,说:“这个孩子日后必定非常显贵!不然的话,怎么会得到我们两位指挥来护守门户呢?”两人赞叹着走了。

薛侯已经长大了,但是挺脏的脸上垂着鼻涕,很不聪明。岛上的薛姓家族,本来隶属军籍。这一年应该薛公家出一口人去戌守辽阳,薛公的长子很为这事发愁。当时薛侯十八岁,人们都认为他太憨痴,没有给他提亲的。他忽然对兄长说:“大哥嘀嘀咕咕的,该不是因为愁咱家没人能去当兵吧?”兄长说:“是啊。”薛侯笑着说:“倘若你肯把丫鬟给我作妻子,我就去服役。”兄长很高兴,就把丫鬟许配给他。薛侯立即携带妻室赶赴辽阳。才走了几十里,天忽然下起了暴雨。路边上有一处高耸的石崖,夫妻二人就跑过去躲避到下面。过了一会,雨停了,他们才再上路。刚刚走了几步,崖石就崩塌了。附近村里的人远远地看见有两只老虎从石崖下跃出,逼近依附到他二人身上就不见了。薛侯从此便勇猛超人,丰采立刻异于往常。后来他因为军功显赫被朝廷封为阳武侯世袭爵位。

到了天启、崇桢年间,世袭阳武侯爵位的薛家某公死了,没有儿子,只有遗腹,于是暂由旁支来代替。当时凡是世袭爵位的人娶的妻妾,只要有了身孕就得报告给朝廷知道,官府便派遣一个老年妇女伴守着她,直到生下孩子才算完事。过了一年,这位薛夫人生了个女孩。产后,腹部还有震动,总共过了十五年,更换了几个伴守的老妇人,又生了个男孩。本来应该嫡支赐封侯爵,但是旁支都吵闹反对,认为这孩子不是薛家的血统。官府收容了原来那些伴守的老妇人,用了各种办法进行拷问,全都承认孩子真是薛家的后代无疑。这才决定把爵位赐封给了他。


\subsection{1.5.2   赵 城 虎}
\label{\detokenize{p00_u5176_u5b83/_u767d_u8bdd_u804a_u658b_u5fd7_u5f02:id175}}
赵城县有一位老妇人,七十多岁了,只有一个儿子。一天她儿子进山,被老虎吃了。老妇人痛不欲生,哭叫着到县衙门告状。县官笑着说:“老虎能用官法去制裁它吗?”老妇人更加哭闹不止。县官呵叱她,也不害怕。县官可怜她岁数大了,不忍心惩罚她,就答应为她捉虎。老妇人趴在地上不走,一定要等县官发出捉虎公文才肯回去。县官实在没有办法,就问堂上的衙役,谁能去捕虎。一个叫李能的衙役,喝得醉醺醺地走到县官面前,自告奋勇说:“我能!”李能拿着勾牒下去,老妇人才回去了。

李能醒过酒后很后悔,又一想,这可能是县官应付老妇人的骗局,以解脱她的纠缠,所以也没把这事放在心上,便拿着勾牒去交差。县官发怒地说:“你说能办到,怎能容许反悔!”李能很为难,便请求县官召集猎户进山提虎,县官答应了。李能召集了所有的猎人,日夜埋伏在山谷中,希望能捕捉到一只老虎,搪塞过去。过了一月多,一只虎也没捉到,李能为这事挨了几百板子,冤苦无处申诉,就到城东庙里跪下祈祷,失声痛哭。一会儿,一只老虎从外边进来。李能惊慌失措,害怕被老虎吃掉。老虎进来,哪里也不看,只是蹲立在门当中。李能向老虎拜祝说:“如果害了老妇人儿子的就是你,你就趴下让我捆起来。”接着就拿出绳索捆住老虎的脖子,老虎俯首贴耳让他绑了。李能牵着老虎来到衙门,县官问老虎说:“老妇人的儿子是你吃了?”老虎点点头,县官说:“杀人偿命,是自古以来的定律。况且老妇人只有这一个儿子,你杀了他,老妇人风烛残年,依靠什么生活?如果你能给她当儿子,我就赦免你。”老虎又点点头。县官于是让衙役给老虎松了绑,放它走了。

老妇人埋怨县官不杀了老虎为她儿子偿命。第二天早晨,她打开门,看见一条死鹿。老妇人卖了鹿皮鹿肉,用来度日。从此老虎经常送东西来,有时衔着金钱或布匹扔到院子里。老妇人从此富裕起来,生活比她儿子在世时还好,心中不禁暗暗感激老虎。老虎来了,时常趴在屋檐下,一整天不走,人畜相安。几年后,老妇人死了,老虎来到房中大声吼叫。老妇人平素的积蓄,足够置办葬事的,家族中的人一块来把老妇人埋葬了。刚把坟墓修好,老虎突然跑来,送葬的宾客都吓跑了。老虎一直跑到坟前,像打雷一般嗥叫了一会儿,才走了。村里人在东郊立了一块“义虎祠”,至今仍在。


\subsection{1.5.3   螳 螂 捕 蛇}
\label{\detokenize{p00_u5176_u5b83/_u767d_u8bdd_u804a_u658b_u5fd7_u5f02:id176}}
一个姓张的人,偶尔在山谷中行走,听到山崖上发出很大的响声。他找到一条小路攀上去,偷偷地看。只见一条碗口粗的大蛇,在树丛中颠倒扑打,用尾巴乱打柳树,柳枝劈劈啪啪纷纷地落下来。看那翻转跌倒的样子,好似有什么东西制住了它。但是,细细一看,并没什么东西。他感到疑惑不解。便慢慢地向前靠进几步,但见一只螳螂紧紧地伏在蛇的头顶,用它那刀似的前爪,撕抓蛇头;蛇竭力摔动着头,想把螳螂摔下来,但总也摔不掉。过了好半天,蛇竟然死了,它头顶的皮肉,早被撕裂开了。


\subsection{1.5.4   武 技}
\label{\detokenize{p00_u5176_u5b83/_u767d_u8bdd_u804a_u658b_u5fd7_u5f02:id177}}
李超,字魁吾,家住淄川县的最西边,他性情豪爽,好施舍和尚。一天,偶尔有个和尚托着钵盂到他家化缘,李超让和尚饱餐了一顿。和尚很感激,便说:“我是少林寺僧人,有点武艺在身,愿意教给你。”李超非常高兴,请和尚住在家里的客房里,供给丰盛的伙食,天天跟和尚学武。

学了三个月,李超已觉得得心应手,不禁洋洋自得起来。和尚问他:“你感到行了吗?”李超回答说:“行了!师傅的武艺,我已都学到手了!”和尚听了,笑着让他练练看。李超便脱下外衣,往手上吐了口唾沫,飞拳踢腿地练了起来。只见他一会儿像跳跃的猴子,一会几像掠过的飞鸟;练完了,很骄傲地站在那儿。和尚笑笑说:“可以了。你既然已全部学到了我的武功,就让我们来比划比划,分个高低。”李超欣然同意。二人拿好架势,便你一拳我一脚地打在了一起。李超时时想找和尚的弱点攻击。和尚忽然飞起一脚,李超还没看明白是怎么回事,已仰面朝天,跌在了一丈开外。和尚拍手大笑说:“你并没学到我的全部功夫啊!”李超既惭愧,又沮丧,跪伏在地,请师傅指教。和尚又教了他几天,才告辞离去。从此后,李超以武艺高强闻名,走遍南北,很少碰上对手。

一次,李超偶然有事来到济南。见一个少年尼姑正在摆场练武,四周挤满了围观的人。尼姑练了一会,对众人说:“我一人在这里翻来复去地练,也太冷清了。有哪位行家,请不妨下场来玩玩!”一连招呼了三遍,大家面面相觑,始终没一个下场的。李超在一边看了,手不禁痒痒起来。一时心盛,便下场了。尼姑笑了笑,合掌行礼,两人便打在了一起。才一交手,尼姑忽叫停下,说:“这是少林派的拳法。”问李超;“你师傅是谁?”李超起初不肯说,尼姑再三询问,李超只得把和尚师傅说了出来。尼姑拱手说:“憨和尚是你师傅吗?既然这样,我们不必较量了,我甘拜下风!”李超再三要求和她比试,尼姑坚决不肯。众人在一边怂恿二人,尼姑才说:“你既然是憨和尚老师的弟子,那我们都是一路上的人,不妨玩玩。但点到为是,你我明白就行了。”李超答应下,心里却轻视尼姑生得文弱;加上他年轻气盛,好胜心强,一心要打败尼姑,以博得个不败的名声。于是,两人重新打在了一起。刚一会儿,尼姑忽然住手不打了。李超不解地询问缘故,尼姑只是笑着,也不说话。李超以为她胆怯了,非要和她比到底不可,尼姑才又动手。一会儿,李超飞起一脚向尼姑踢去;尼姑并拢五指,手掌像利刃一样,往下轻削李超的小腿。李超只觉膝盖下一阵巨痛,像被刀斧砍中了一般,一下子摔倒在地,再也爬不起来。尼姑笑着谢罪说:“太冒犯您了,请不要见怪!”李超被人背了回去,养了一个多月才好。

过了一年多,师傅来看他,李超便向师傅讲述了这件往事。和尚听了大惊说:“你也太鲁莽了!惹她干什么!幸亏你先把我的名字告诉了她,不然,你的腿早就断了!”


\subsection{1.5.5   小 人}
\label{\detokenize{p00_u5176_u5b83/_u767d_u8bdd_u804a_u658b_u5fd7_u5f02:id178}}
康熙年间,有个玩魔术的人携带着一个盛酒的榼,榼中藏着一个小人,才一尺多高。人们扔钱给术人,他就让小人从榼中出来,唱个曲子再退回去。术人到了山东掖县。掖县县令派人把榼带进宫府,仔细询问小人的来历。小人起初不敢说,再三追问他,才说出了自己的家乡和姓氏。

原来小人是个读书的童子,从学堂中回家时,被术人拐骗,给他吃了一种药,身体便突然缩小了,术人于是携带着他,当成了赚钱的道具。县令听说后大怒,杀了术人,把童子留了下来,想给他医治,但还没有得到药方。


\subsection{1.5.6   秦 生}
\label{\detokenize{p00_u5176_u5b83/_u767d_u8bdd_u804a_u658b_u5fd7_u5f02:id179}}
山东莱州的秦生,自制药酒时,错放了有毒的药物,舍不得倒掉,把它封存了起来。过了一年多,有一天夜里恰好想喝酒,又没处去弄。忽然想起封存的药酒,启封一闻,浓烈的芳香气味喷溢而出,馋得他肠子发痒口水直流,没法制止。拿过酒杯想尝尝,妻子苦苦地劝说他。秦生笑着说:“痛痛快快地喝了酒死,倒比被酒馋死强得多。”一杯入肚,倒瓶再斟。妻子把酒瓶打翻,酒淌了一地。秦生趴下像牛饮水那样去喝淌了的酒。不一会儿,他肚子疼痛紧闭着嘴说不出话,半夜里就死了。妻子嚎啕大哭,为他准备好棺材,将要入硷。第二天夜里,忽然有个美女进来,身高不满三尺,径直走到灵床旁边,用手中杯子里的水灌他。秦生豁然苏醒过来,叩头追问她是谁。美女说:“我是狐仙。刚才丈夫到陈家窃酒醉死了,我去救活他回来,偶然路过您的家门;丈夫可怜您与他同病,因此让我用剩余的药水把您救活了。”说完,就不见了。

我的朋友丘行素贡士,爱饮酒。有一天夜里想喝酒,无处去买,翻来复去的无法忍耐,于是想用醋来代酒。和妻子商量,妻子嗤笑他。丘贡士再三强求,妻子就煨好醋端过来。一壶醋喝光了,这才解衣安睡。第二天,丘夫人拿出足够买一壶酒的钱,派仆人代她买酒。丘贡士的伯弟襄宸在路上遇见仆人,问知缘故,怀疑嫂子不肯为兄买酒。仆人道: “夫人说:‘家里存的醋不多,昨夜已经喝尽了一半;恐怕再喝一壶,就断了醋根了。’”听到的人都笑他。不知道酒瘾上来了,就是毒药尚且觉着甜美,更何况是醋呢?此事也可以流传。


\subsection{1.5.7   鸦 头}
\label{\detokenize{p00_u5176_u5b83/_u767d_u8bdd_u804a_u658b_u5fd7_u5f02:id180}}
东昌府秀才王文,从小就很诚实。有一年,他到湖北去,过了六河,住在一座旅舍里。偶而到街上闲逛,遇见同乡赵东楼。这人是个大商人,长年在外,几年没回家了。一见面,热烈握手,十分亲昵,邀王文到他的住处叙谈。王文一进门,见室内坐着一个美貌女子,吃了一惊,想退出来;赵一把拉住他,一面隔着窗子喊了一声:“妮子去吧!”然后拉着王文进来。赵摆上酒菜,问寒道暖地与王文叙谈起来。王文便问:“这是什么地方?”赵痛快地告诉他:“这是一座小妓院。我久客他乡,不过暂时借宿休息罢了。”谈话间,妓女妮子出出进进地照应着。王文有点局促不安,便起身告辞。赵东楼又强拉他坐下。一会儿,王文瞥见一个少女从门外走过。少女也瞥见了王文,秋波频转,含情脉脉,体态窈窕轻盈,俨然是个仙女。王文虽然平素端方正直,此时也有点神情摇荡起来,便问:“这漂亮女孩是谁?”赵东楼说:“她是妓院鸨母的二女儿,名叫鸦头,十四岁了。想送缠头礼的客人多次以重金打动鸨母,鸦头本人执意不从,惹得鸨母常鞭打她。她以自己年岁太小为由苦苦哀求,总算免了。所以到现在还在待聘中呢!”王文听着,低头默坐,呆呆地答非所问起来。赵便开玩笑说:“你如有意,我一定替你作媒!”王文长叹一声说:“我不敢有这个念头!”可日落西山也不说告辞的话,坐着不走。赵便又提起这话,王文才说:“您的好意我感激,可我囊中羞涩,怎么办?”赵明知鸦头性情刚烈,这事必定不答应,便故意答应拿十两银子帮他。王文千恩万谢,急忙回到旅馆,倾囊倒箧地又凑了五两,跑回来请赵送给鸨母。鸨母嫌少。不料鸦头对母亲说:“妈不是天天骂我不肯当摇钱树吗?这一回我想遂了妈的心愿。女儿初学作人,将来报答妈的日子有的是,何必因为这次数目少点,便把财神放跑了!”鸨母没想到鸦头一向执拗,这一回却同意了,便很欢喜地答应了,吩咐婢女去请王郎。赵东楼不便中途翻悔,只好顺水推舟,加上银子送给鸨母。

王文与鸦头非常恩爱。晚上,鸦头对王说:“我是个烟花下流女子,配不上您。既然承蒙您相爱,这份情又是重的。可郎君您倾囊换取这一夜之欢,明天怎么办呢?”王文难过得直流泪。鸦头说:“不必发愁。我沦落风尘,实在不是出于自愿。只是一直没碰见一个像您这样的诚实人可以托付终身罢了。您如果有意,我们就趁夜逃走吧!”王文高兴极了,急忙起身!鸦头也起来,侧耳听谯楼上正敲三更鼓。鸦头赶紧女扮男装,二人匆匆出走,敲开旅馆的门。王文本来带来两匹驴,借口有急事出门,命仆人立即动身。鸦头掬出两张符系在仆人背后和驴耳朵上,就放开辔头让驴子奔驰起来,快得让人睁不开眼,只听见身后风声呼呼。

天亮时候,到了汉口,他们租了一座房住下来。王文感到十分惊异。鸦头对他说:“告诉你,你不害怕吧?我不是人,而是狐。我母亲贪淫,我天天挨打受骂,我真恨她。今天总算脱出苦海了。百里以外,她便打听不到,咱们可以安然过日子了。”王文完全相信鸦头的话,对狐鬼也无疑虑,只是发愁说:“面对你这芙蓉一般的美人,可我四壁空空,实在于心不安,恐怕到头来还得被抛弃。”鸦头说:“何必为这个发愁,现在在市面上做个小买卖,养活三几口人,粗茶淡饭还是可以的。你可以卖掉驴子作本钱。” 王文于是按鸦头的话,在门前开了个小店,卖酒卖茶,由王文和仆人两人忙活应酬;鸦头便在家中缝披肩,绣荷包。这样每天赚点赢余,一家吃喝也还不错。一年之后,也能雇老妈子、婢女了,王文也不用亲自干活,只是看管着伙计们经营就可以了。

一天,鸦头忽然悲伤起来,对王文说:“今夜该当有灾难,怎么办?”王文问她是何事,鸦头说:“母亲已经打听到我的消息了。她必定来逼我回去。若是派妮子阿姐来,我还不愁应付。就怕她亲自来!”夜深人静之后,鸦头庆幸地说:“不要紧了。是阿姐来的。”过了不一会儿,妮子推门而进,鸦头笑着迎上去。妮子骂道:“丫头也不害羞,跟男人私奔!老母叫我来抓你。”说着掏出绳子就往鸦头脖子上套。鸦头生气地说:“我跟一个男人从良,有什么罪?”妮子一听,更气上加气,揪住鸦头撕打起来,把鸦头的衣襟都扯破了。家中婢女老妈子们听见吵闹,都拥上来,妮子害怕了,跑了出去。鸦头说:“妮子阿姐回去,我老母必定亲自上门,那就大祸临头了!赶紧想办法吧!”就急忙收拾行装,准备搬到更远的地方去。正在忙乱之际,老娘已经闯进来,满脸怒气,喊道:“我早就知道这丫头无礼,非得我亲自来一趟不可!”鸦头赶紧迎上去跪下哀告求饶,老婆子二话不说,揪住头发拖着就走了。王文急得团团转,顾不得吃饭睡觉,急忙赶到六河,打算把鸦头赎回来。不料到了那里,那座妓院倒是照旧开着,人却全换了。向院中人打听,都说不知她们到哪里去了。王文痛哭一场回来,打发仆人们散去,自己收拾财物,返回东昌老家。

过了几年,王文偶然因事到燕都去。经过育婴堂时,仆人看见一个小孩,七八岁的样子,长得很像王文。仆人感到惊奇,不住地打量起来。王文问仆人:“老看人家小孩干什么。”仆人笑着回说了。王文一看,也笑了。再仔细一端详,小孩生得很英俊;又一想自己还没儿子,因小孩很像自己,就喜爱上了,把他赎了出来。王文问他的姓名,小孩说叫王孜。王文觉得奇怪,又问:“你吃奶时就被爹娘丢了,怎么还知道姓名?”王孜说:“我保姆说的:拾我时,我胸前有字,写着‘山东王文之子’。”王文大吃一惊,说:“我就是王文。哪里有儿子?”又想也许是个同名同姓的人吧。心里挺高兴,很疼爱他。带回东昌老家后,看见的人不问就知道是王文的亲生儿子。

王孜逐渐长得高大健壮起来,性格勇武,力气又大,喜欢打猎,还好打架,王文也管不住他。又说能见鬼狐,别人都不相信。恰好村里真出了一个狐精作祟的人家,便请他去看看。他去了便指出狐精隐藏之处,叫几个壮汉向他指处猛砸。只听见狐嗷嗷直叫,毛血扑扑地落下来。从此这个人家就安静无事了,人们也更惊奇佩服他了。

王文有一天到集市上闲逛,忽然遇见赵东楼,衣帽不整,面容枯瘦。王文惊讶地问他从何而来,赵凄惨地请求到僻静处谈,王文便邀他到家里来,让仆人摆上酒菜,二人叙谈起来。赵说: “老婆子把鸫头抓回去后,打得好惨。又搬家到燕都去,逼她另嫁别人。鸦头坚决不从,老婆子就把她关起来。后来鸦头生了一个男孩,一生下来他们就给扔到胡同里去了。听说育婴堂拾了去,也该长大成人了。这是您的后代。”王文不禁潸然泪下,说:“苍天保佑,这孽子我已找回来了!”于是把经过说了一遍。又问赵: “您怎么落拓到这个地步?”赵长叹一声说:“今天才知道与青楼人相好,不可过分认真了。还有什么好说的呢!”

原来鸨母迁往燕都的时候,赵东楼也借做买卖跟了去。手中那些难运的货物,都在当地贱价卖掉,一路上的吃用花销,弄得他已经元气亏损。妮子又奢华讲究,开销很大,几年之间,纵有万金之富,也荡然无存了。鸨母见他没了钱,日夜白眼相加。妮子也常到富贵家去陪宿,经常一连几夜不回来。赵东楼气愤难忍,但又无可奈何。有一天,正巧鸨母外出,鸦头从窗内招呼赵说:“妓院哪有什么真情!她们所爱的,不过是钱罢了。您再恋恋不舍,就要遭祸啦!”赵害怕起来,这才如梦初醒;临行前,偷着去和鸦头告别。鸦头把一封信交给他,托他转给王文,赵就这样回了家。说着,把信掏出来交给王文。信上说:“听说孜儿已经回到您的身边了。我的苦难,东楼君自会向您详细说明。前世作孽,有何话说!我身陷幽室之中,暗无天日,终日鞭打,皮开肉绽,疼痛难忍,饥饿又如同油煎一般,挨过一天,似经一年。您如不忘在汉口时雪夜夫妻拥抱取暖的情景,希望能和孜儿商量,让他救我脱离苦海。老母、阿姐虽然残忍,总是骨肉之亲,您可嘱咐孜儿不要伤害她们的性命。这是我的愿望。”

王文读了信,禁不住失声痛哭起来。拿出些散碎银子赠给赵东楼,送他回家。

这时王孜已经十八岁了,王文把前因后果一说,又给他看了母亲的信,王孜登时气得两目圆睁,当天就启程去燕都。一到那里,就打听吴家鸨母住处,那里门前车水马龙。王孜直闯而进,妮子正陪着一个湖广商人饮酒,抬头望见是王孜,吓得立刻变了脸色。王孜扑过去,杀了她。宾客都吓坏了,以为来了强盗;一看妮子的尸首,已经变成了狐。王孜抡刀继续往里闯,吴老婆子正在厨房里催女婢作羹汤。王孜刚闯到门口,老婆子忽然不见了。王孜仰头向四处一看,立即抽弓搭箭往屋梁上射去,一箭正中老狐心窝,老狐掉了下来,王孜便砍下它的脑袋。然后找到自己母亲被困的住所,拾起一块大石头砸破门锁,母子二人痛哭失声。鸦头问老娘怎样了,王孜说:“已经杀了!” 鸦头埋怨说:“你这孩子怎么不听娘的话!”立即命他快到郊外把老娘埋葬了。王孜口头上答应着,却偷偷把老狐精的皮剥下收藏起来。又把吴老鸨屋中的箱箱匣匣检查了一遍,把里面的金银珠宝全收起来,王孜便陪母亲返回了东昌老家。

王文与鸦头夫妻重逢,悲喜交集。王文又问起吴老太太,王孜说:“在我的袋子里!”王文惊问所以,王孜拖出两张狐皮给父亲看。鸦头一见,气得大骂:“这个忤逆不孝的孩子!怎么能这么干啊!”哭得用手打自己的脸,直想寻死。王文百般劝解,斥令王孜快把狐皮埋葬了。王孜生气地说:“今天刚安稳了,就把挨打受骂的苦日子忘啦!”鸦头更气得痛哭不止。王孜去埋葬了狐皮,回来当面禀报,鸦头才平静下来。

王家自从鸦头到来,家道更加兴旺起来。王文感激赵东楼,以重金相赠。赵这才知道妓院母女都是狐精。王孜也很孝顺父母,不过偶尔触犯了他,他就恶声吼叫。鸦头对王文说:“这孩子长着拗筋,如若不给他拔掉,他到头来终会暴躁杀人,弄得倾家荡产。”于是趁夜里王孜睡熟时,把他手足捆起来。王孜醒了,说:“我没有罪!”鸦头说:“妈要给你治拗病,你别怕痛!”王孜大叫,可是绳子捆着挣不开。鸦头就用大针刺他的踝骨旁边,扎到三四分深处,把拗筋挑出来,用刀砰的一声割断;又把他的胳膊肘上、脑袋上的拗筋照样割断,然后放开他,轻轻拍抚几下,让他安心睡觉。第二天早晨,王孜跑到父母跟前问安,哭着说:“儿昨天夜里回想以前做的事,简直不像人干的!”父母高兴极了。从此,王孜就温和得像个女孩儿,村中老幼都夸奖他。


\subsection{1.5.8   酒 虫}
\label{\detokenize{p00_u5176_u5b83/_u767d_u8bdd_u804a_u658b_u5fd7_u5f02:id181}}
山东长山的刘某,身体肥胖爱好饮酒,每当独饮,总要喝尽一瓮。他有靠近城郭的三百亩好地,常常只种一半庄稼;而家里非常富足,并没因为爱喝酒使家境受影响。

一个西域来的僧人见到刘某,说他身患奇异的病症。刘回答:“没有。”僧人问他:“您饮酒是不是不曾醉过?”刘某说:“是的。”僧人说:“这是肚里有酒虫。”刘某非常惊讶,便求他医治。僧人说:“很容易。”刘某问:“需用什么药?”僧人说什么药都不需要,只是让他在太阳底下俯卧,绑住手足;离头半尺多的地方,放置一盆好酒。过了一会儿,刘某感到又热又渴,非常想饮酒。鼻子闻到酒的香味,馋火往上烧,而苦于喝不到酒。忽然觉得咽喉中猛然发痒,哇的一下吐出一个东西,直落到酒盆里。解开手足一看,一条红肉三寸多长,像游鱼一样蠕动着,嘴、眼俱全。刘某很惊骇地向僧人致谢,拿银子报答他,僧人不收,只是请求要这个酒虫。刘某问他:“作什么用?”僧人回答:“它是酒之精,瓮中盛上水,把虫子放进去搅拌,就成了好酒。”刘某让僧人试验,果然是这样。

刘某从此厌恶酒如同仇人,身体渐渐地瘦下去,家境也日渐贫困,最后竟连饭都吃不上了。


\subsection{1.5.9   木 雕 美 人}
\label{\detokenize{p00_u5176_u5b83/_u767d_u8bdd_u804a_u658b_u5fd7_u5f02:id182}}
商人白有功说:“在济南泺口河岸,见一个人扛着个竹箱子,牵着两只巨大的狗。他从箱子里拿出个木雕美女,有一尺多高,手和眼能转动,穿着艳丽的衣服,如同真人。又用锦缎做成的小马鞍垫子披在狗身上,便命令美女跨上去坐好。安置完了,呼呵大狗快跑。美女自己起身,表演各种马术,先脚踩马蹬蹲藏到狗肚子一侧;再从狗腰向狗尾滑坠,抓住狗尾飞身上狗;后在狗背上跪拜站立,变化灵巧而不失手。又扮作昭君出塞的样子;另拿出一个木雕男子,在他帽子上插野雉尾,给他披上羊皮袍子,让他跨在狗身上跟在美女后面。昭君频频回头张望,穿羊皮衣服的男子扬鞭追赶,真像活人一样。”


\subsection{1.5.10   封 三 娘}
\label{\detokenize{p00_u5176_u5b83/_u767d_u8bdd_u804a_u658b_u5fd7_u5f02:id183}}
范十一娘,是{[}田鹿{]}城祭酒的女儿,年轻貌美,有文才,父母十分钟爱她。有上门来求婚的,总是让她自己选择,但十一娘却始终没有一个中意的。适逢上元节,水月寺中的尼姑们举行“盂兰盆会”。这一天,游女如云,范十一娘也来了。正在游玩观赏的时候,有个女子一直跟在十一娘身边,不住地打量她,像有话要说。十一娘仔细看了看她,是一位十五六岁的绝代佳人。十一娘很喜欢她,转回身来盯住她细看,那女子微笑着说:“姐姐莫不是范十一娘吗?”十一娘回答:“是的。”女子说: “久闻姐姐是个才貌双全的女子,人们说的果然一点不假。”范十一娘也询问她的姓名、住处。女子笑着说:“我姓封,排行第三,就住在邻近的村子。”说着挽起十一娘的手臂。又说又笑,言语情态婉顺温柔。两人相互爱悦,依恋不舍。十一娘问:“你怎么没有人陪伴?”三娘说:“父母早就去世了,家中只有一个老妈子,留在家中看门,所以不能跟来。”十一娘要回去了,封三娘目不转睛地看着她,眼泪都快要掉下来了。十一娘也惘然若失,就邀请她到自己家里去玩。封三娘说: “姐姐是个富贵人家,我和你又不沾亲带故。怕惹人讥讽!”十一娘执意请她,三娘才说:“改天再去吧。”十一娘摘下一股金钗赠给她,封三娘也从发髻上摘下一支绿簪子回赠。十一娘回家以后,十分想念封三娘,拿出三娘赠给的绿簪子看,不是金的也不是玉的,家里人都不认识,很觉奇异。十一娘天天盼望三娘来,总是失望,就病倒了。父母知道了她生病的原因,派人到邻近村子打听,却没有一个人知道封三娘。

到九月九重阳节,十一娘已病得憔悴不堪,感到无聊,就让婢女扶着,勉强来到花园,铺了褥子在东篱下观赏菊花。忽然一个女子扒着墙头往这边看,仔细看时,原来是封三娘!只听三娘喊道:“快来扶我一把!”婢女急忙过去扶她下来。十一娘又惊又喜,站起身拉三娘一同坐在褥子上,责怪她不守信用;又问她从哪里来。三娘回答说:“我家离这里还远,但常来舅舅家玩耍。以前我说住在邻近的村子,说的是我舅舅家。分别后我苦苦想念你,但贫贱之人同富贵家交往,脚还没登门,心中先感到羞惭,恐怕被婢女仆人们瞧不起,所以没有来。刚才从墙外经过,听到有女子说话,就扒墙看看,盼望是姐姐,果真就是你!”十一娘述说了因思念而得病的经过,封三娘泪如雨下,感动地说:“我这次来你一定要保密,不然让造谣生事的人说长道短,我可受不了!”十一娘答应了。二人一同回到闺房,同吃同住,一同说心里话。十一娘的病很快好了,两人结拜为姐妹,衣服鞋袜,总是换着穿。见有人来,封三娘就藏到幕帐后边。过了五六个月,十一娘的父母终于听说了这件事。一天,两人正在下棋,范母悄悄地走了进来,仔细端详着三娘,惊喜地说:“真不愧是我女儿的好朋友!”又对十一娘说:“你有这样一位好朋友,我们两人都高兴,为什么不早告诉我?”十一娘就把封三娘的顾虑告诉了母亲。范母看看三娘说:“你和我女儿作伴,我感到很欣慰,为什么怕人知道呢?”三娘满脸羞容,只是默默地搓弄着衣带。范母一走,封三娘就要告别。十一娘苦苦挽留她,才又住下来。一天夜里,封三娘从门外急匆匆地跑进来,哭着说:“我本来就说不能再留在这里了,如今果然受到这样大的侮辱!”十一娘吃惊地问她怎么回事,三娘说:“刚才出去入厕,有一个少年男子,强来拉扯我,幸亏逃掉了。像这样,叫我怎么再见人呢?”十一娘仔细询问了那人的相貌,向三娘道歉说:“请不要见怪,那人是我傻哥哥。我会告诉母亲,用棍子打他一顿的!”封三娘执意要走,十一娘请她等到天亮,封三娘说; “舅舅家近得很,只须用一架梯子送我过墙就行了。”十一娘知道留不住了,就派两个婢女送她过墙。走了半里多路,封三娘辞谢她们自已走了。婢女回去后,见十一娘伏在床上悲伤地啼哭,像失去了最亲密的爱人。

过了几个月,婢女有事到东村去,傍晚往回走的路上,遇见封三娘跟着一位老妇人走来。婢女很高兴,迎上去问好。封三娘很感忧伤,询问十一娘的情况。婢女拉着封三娘的衣袖说:“三娘到我家去吧,我家姑姑盼你盼得要死!”封三娘说;“我也思念她,但是不愿意让她家的人知道。你回去后打开花园门,我自己会去的。”婢女回去告诉十一娘,十一娘非常高兴,按她说的做了,见封三娘已经在园中了。两人相见,各自述说分别之情。话越说越长,连觉也不睡了。见婢女们都睡熟了,三娘起身和十一娘躺在一个枕头上,悄悄地说:“我知道你还没有许配人。以你的才貌和门第,不愁找不到个尊贵的女婿。但那些浪荡子弟,不值一提。如果想得到一个好丈夫,请不要以贫富论人。”十一娘连连称是。封三娘说:“去年我们见面的地方,现在又做起了道场,明天请你再去一趟,我要让你见一个如意郎君。我小时候读过相面的书,绝对没有差错的。”天不很亮,封三娘就走了,约好在寺院等她。十一娘果然来到水月寺,封三娘已先在那里了。眺望游览了一周,十一娘便邀请三娘一同上车。两人挽着手出了寺院门,看见一个秀才,年龄有十七八岁,穿着朴素的布袍,但容貌英俊,仪表不凡。封三娘暗暗指着秀才对十一娘说:“这个人是能做翰林的人才。”十一娘稍稍斜眼瞅了一下。封三娘又说:“你先回去,我随后就到。”黄昏时侯,封三娘果然来了,说:“我刚才已经打听清楚,那个秀才就是此地人,叫孟安仁。” 十一娘知道孟安仁家里很穷,觉得不大合适。封三娘说:“你怎么也落入世俗之中去了。这人如果是长期贫贱的人,我就把眼睛剜掉,不再给天下人相面了!”十一娘说:“那么又该怎么办呢?”封三娘说:“请你给我一件东西,拿去送给他,就算订了婚约。”十一娘说:“姐姐太草率了。有父母在,如不答应怎么办?”封三娘说:“我这样做,正是怕他们不答应。如果你主意坚定,就是死也阻挡不了的。”十一娘执意不肯。封三娘说:“你的姻缘已经来了,但是魔难没有消除。我所以这样做,是报答你以前对我的好处。我现在就去,把你以前送给我的金凤钗,假托你的名义送给他。”十一娘刚想说再商量商量,封三娘已经出门走了。

当时,孟生虽然博学多才,但因家境贫穷,所以十八岁还没有定下婚事。白天在寺院,忽然看见两个美丽的女子,回家后一直苦苦思念。一更时尽,封三娘叫开门进来。孟生拿蜡烛一看,认识是白天在寺院见过的女子之一,高兴地问她是谁。三娘说:“我姓封,是范十一娘的女伴。”孟生高兴极了,顾不得细问,突然上前拥抱她。封三娘推开他说: “我不是自荐的毛遂,是来代人作媒的。范十一娘愿意和你结为夫妻,请你托媒人去提亲吧。”盂生愕然不信。封三娘拿出金钗给他看,孟生喜欢得不得了,发誓说:“承蒙她如此眷恋我,我要得不到十一娘为妻,宁肯终身不娶!”封三娘就走了。

第二天早晨,孟生托邻居老妈妈去见范夫人,给自己提亲。范夫人嫌他穷,也不同女儿商量,立即把老妈妈打发走了。十一娘知道后,心里很失望,埋怨封三娘耽误了自己。但是金钗要不回来,只好决意也不嫁别的人。又过了几天,有一个绅士来为儿子向范家求婚,怕不成,就请县令作媒。当时,那绅士很有权势,范家害怕他,就问十一娘的意见。十一娘不愿意,母亲问她为什么,她不说话,只是掉泪。十一娘叫人暗暗告诉母亲,不是孟生,死也不嫁。范公知道了十分生气,索性把女儿许给了那绅士的儿子。又怀疑十一娘和孟生有私情,就选定吉日,想尽快为她完婚。十一娘气得不吃饭,天天只是呆呆地躺着。到了迎亲的前一天晚上,十一娘忽然起来,对着镜子自己梳妆打扮起来。范夫人暗暗高兴。一会儿侍女跑来说:“小姐上吊了!”全家上下大吃一惊,痛哭流涕,后悔也来不及了,三天后只好安葬了。

孟生自从邻居老妈妈告诉他婚事不成以后。心里悲愤,气得要死,但依然转弯抹角地打听消息,梦想能挽回与十一娘的婚事。听说十一娘已经许配给人了,怒火中烧,什么念头也没有了。不久,听说十一娘死了,孟生悲愤不已,恨不得跟十一娘一起死去。傍晚走出家门,想趁黑夜去十一娘坟上哭一场。忽然有一个人走过来,近前一看是封三娘。三娘向孟生说:“恭喜你的姻缘总算能成就了!”孟生含着泪说:“你不知道十一娘已经死了?”封三娘说:“我说的能成就,正是因为她死了。你赶快叫家人挖开坟墓,我有一种奇异的药,能让她复活!”孟生听了她的话,挖开墓穴,打开棺材,把十一娘抬出来,又把坟墓重新掩埋好。孟生自己背着尸体,与封三娘一同回到家里,把十一娘放到床上,三娘给她灌了药。一会儿,十一娘慢慢苏醒过来,看着封三娘问:“这是什么地方?”封三娘指着孟生说:“这就是孟安仁。”就把事情的经过告诉了她,十一娘这才如梦初醒。

封三娘怕泄漏消息,陪送他们到五十里外的一个山村里躲藏起来。封三娘要告辞回去,十一娘哀求她留下作伴,让她住在另一个院里。又卖了殉葬的首饰,用来度日,日子还算过得去。封三娘每次遇到孟生来,总是避开。十一娘从容地说:“咱们姊妹俩的情谊,就是同胞姐妹也比不上,可哪能百年都聚在一起?我想,不如仿效女英、娥皇一起嫁给孟生。”封三娘说:“我从小就得到吐纳长生的秘决,所以不愿意嫁人。”十一娘笑着说:“世上流传的养生术书籍多得很,行而有效的哪里有啊?”封三娘说:“我得到的不是人世流传的那种。世上流传的并不是真诀,只有华佗的五禽图还差不多。凡是修练的人,无非是想让血气流通罢了;若是得了厄逆症。学作老虎的形体动作,马上就会好,不正是它灵验的地方吗?”十一娘就私下和孟生商量,让他假装出远门。到了夜里,用酒强把三娘灌醉,孟生悄悄进来和她同了床。三娘醒后说: “妹子害了我了。如果我色戒不破,道业修练成功,能升第一天。如今被你算计了,这是命该如此。”就起身告辞。十一娘告诉她自己的实心实意,哀求她不要怪罪自己。封三娘说:“实话告诉你,我是狐仙。因为看到你的美貌,忽然生了爱慕之情,今天却作茧自缚,这也是情魔劫数,不是人力造成的。若是再留下来,情魔更纠缠我,就无休止了。妹妹福分不浅,前程远大,请珍重自爱。”说完就没影了。夫妻两人惊叹了很久。

过了一年,孟生乡试、会试果然都考中了,在翰林院做了官。他拿了自己的名帖去拜见范十一娘的父亲。范父既羞愧又悔恨,不肯见他。孟生再三请求,才见了面。孟生进来,以女婿的礼节,恭恭敬敬地拜见。范公很恼怒,怀疑孟生故意轻薄羞辱自己。孟生便请他到没人的地方,把事情的经过讲了一遍。范公还是不太相信,派人去他家查看后,这才大为惊喜。又暗里告诉孟生不要宣扬,怕有祸秧。又过了二年,那绅士因行贿被查处,父子二人都被充军到辽海卫,十一娘才回到娘家。


\subsection{1.5.11   狐 梦}
\label{\detokenize{p00_u5176_u5b83/_u767d_u8bdd_u804a_u658b_u5fd7_u5f02:id184}}
我的朋友毕怡庵,卓越超群,豪放不羁。长得很胖大,胡子很多,在文人学士中很知名。他曾因有事到叔叔毕际有刺史的别墅里去,在楼上休息。人们传说这楼中过去有很多狐仙。毕友每次读《青凤传》时,心里总向往不已,恨不能也遇见一次。于是便在楼上,苦思凝想起来。随后回到自己家里,天已逐渐黑了。当时正是暑天很闷热,他便对着门躺下睡了。睡梦中觉得有人摇晃他。醒来一看,原来是一位妇人,年纪已经四十多岁,但是风韵犹存。毕友很惊奇地起身,问她是谁。妇人笑着说:“我是狐仙。承蒙您倾心想念,感激不尽。”毕友听说后很高兴,便和她说些调笑戏言。妇人笑着说:“我的年龄已经大了,即使人们不厌恶,我先自惭沮丧。我有个女儿刚刚成年,可让她在身边侍奉您。明天晚上,您不要留别人在屋里,到时候就来。”说完就走了。

到了夜里,毕友烧上香坐等。妇人果然带领女儿来到。狐女体态容貌文雅美好,绝世无双。妇人对女儿说:“毕郎和你早有缘分,今夜你便留在这里。明晨早点回去,一定不要贪睡。”毕友和狐女携手入帏,恩爱备至。过后,狐女笑着说:“肥胖郎君笨重,叫人不能忍受!”天不亮就走了。到了晚上她自己来到,说:“姊妹们要为我祝贺新郎,明天就委屈您一同去吧。”毕友问:“在什么地方?”狐女说:“大姐作筵席主人,离这里不远。”毕友果真等候着。过了很久,狐女也没来,他感到渐渐疲倦,才趴到桌子上,狐女忽然进来说:“有劳您久等了。”于是两人握手而行。很快到了一个地方,见有个大院落。他们径直进了中堂,看到里面灯烛闪烁,光亮犹如星点。不久女主人出来,年纪约近二十岁,虽是淡妆却美丽无比。她提起衣襟行礼祝贺后,将要入席,丫鬟进来说:“二娘子到了。”见一女子进来,年纪约十八九岁,笑着对狐女说:“妹子已破瓜了,新郎很如意吧?”狐女用扇子打她的背,并用自眼瞅她。二姐说:“记得小时候和妹妹打闹着玩,妹妹最怕别人戳她的肋骨,远远地呵手指,就笑得不能忍受,对我发怒,说我应当嫁给矮人国的小王子;我说丫头日后嫁个多髭郎,刺破小嘴。今天果然这样了。”大姐笑着说:“难怪三妹怨谤,新郎在旁边,竟然如此胡闹。”

一会儿,大家并肩而坐,举杯吃喝说笑,非常高兴。忽然有个少女抱着一个猫来,年纪约十一二岁,稚气未退,却艳媚已极。大姐说:“四妹妹也要来见姐夫吗?这里没有你坐的地方。”就把她提抱在膝盖上,拿菜肴水果给她吃。不一会儿,又把她转放到二姐的怀中,说:“压得我胫骨酸痛!”二姐说:“丫头才这么大,但身子却像有百斤重,我脆弱不能忍受。既然想见姐夫,姐夫本来就高大,胖膝盖耐坐。”于是把她放到毕友的怀里。少女入怀香软,轻得像无人一样。毕友抱着她用同一只杯子饮酒。大姐说: “小丫头不要喝多了,酒醉失态,恐怕姐夫笑话。”少女笑孜孜的,便用手抚弄猫,猫戛然而鸣。大姐说:“还不快扔掉,抱一身跳蚤虱子!”二姐说:“请以猫为酒令,拿筷子传递,猫叫时筷子在谁手里谁喝酒。”大家都按她说的方法来玩。筷子一到毕友手里猫就叫。毕友本来酒量大,连喝了好几大杯,才知道是少女故意弄猫让它叫的,因而哄堂大笑。二姐说:“小妹回家睡觉去吧!要压煞郎君,恐怕三姐怨人的。”少女于是抱猫走了。

大姐见毕友善饮,就摘下头上的髻子盛酒来劝。看上去髻子仅能容一升;然而喝起来,却觉得有好几斗。等到喝干了再看,原来是个荷叶盖子。二姐也要敬酒,毕友推辞不胜酒力。二姐拿出一个口脂盒子,比弹丸稍大一点,斟上酒说:“既然不胜酒力,暂且表示点意思吧。”毕友看了看,一口可以喝尽;可是连续喝了百余口,再也喝不干。狐女在旁边用小莲花杯换了盒子去,说:“不要再被奸人戏弄了。”把盒子放到桌上,原来是一个巨大的饭钵。二姐说:“关你什么事!才三天的郎君,就这样的亲爱啊!”毕友拿着莲花酒杯对着口一饮而尽。手里的酒杯变得很软;仔细一看,不是酒杯,竟是一只刺绣精美的绣花鞋。二姐夺过鞋骂道:“你这狡猾的丫头!什么时候偷了人家的鞋子去,怪不得脚冷冰冰的!”于是起身,进屋换鞋。狐女约毕友离席告别。把他送出村后,让他自己回家。毕友忽然睡醒,竟然是梦境;但是口、鼻里醺醺然,酒味仍很浓,感到非常奇怪。到了晚上,狐女来了。说:“昨夜没醉死吧?”毕友说:“刚才还在怀疑是梦呢。”狐女说:“姊妹们怕您胡来,所以假托梦境,其实不是梦。”

狐女经常和毕友下棋,毕友总是输。狐女笑着说:“您终日爱下棋,我以为必定是高手,今天看来,只不过平平罢了。”毕友求她指点。狐女说:“下棋的技艺,在于人的自悟,我怎么能帮您呢?每天早晚慢慢熏陶,或许应有长进。”过了几个月,毕友觉得稍有进步。狐女试了试,笑着说:“还不行,还不行。”毕友出门和曾经在一起下过棋的人再下,人们就觉得他棋艺大大高于以前,都感到奇怪。毕友为人坦白耿直,心里藏不住事儿,就把原因稍稍地透露一些。狐女早已知道了,责备他说:“怪不得同道们不愿和狂生来往。屡次叮嘱你要谨慎守密,怎么仍然这样!”说完很生气地要走。毕友急忙谢罪,狐女这才稍微解怒,然而从此来的次数便逐渐少了。

过了一年多,有天晚上狐女来到,面对毕友呆呆地坐着。毕友和她下棋,不下;和她睡觉,也不睡。她沉闷了很久,说:“您看我比青凤怎么样?”毕友说:“恐怕要比她强。”狐女说: “我自愧不如她。然而聊斋先生和您是文字交,请麻烦他给作个小传,未必千年以后没有像您这样爱念我的人。”毕友说:“我早就有这个愿望;只因过去一直遵照您原来的叮嘱,所以秘不告人。”狐女说:“原来是这样嘱咐您的,可今天已经到了将要分别的时候了,还再避讳什么呢?”毕友问:“到哪里去?”狐女答:“我和四妹妹被西王母征去当花鸟使,不再回来了。过去有个同辈姐姐,因为和您家的叔兄在一起,临别时已经生下了两个女孩,所以至今还没嫁出去,我和您幸亏没有这样的拖累。”毕友求她留一赠言。狐女说:“盛气平,过自寡。”于是起身,拉着毕友的手说:“您送我走吧。”两人走了一里多路,洒泪分手。狐女说:“咱们彼此有志,未必没有再见面的时候。”说完便离去了。

康熙二十一年腊月十九日,毕怡庵和我一起睡在绰然堂,详细叙述了他这段奇异的经历。我说:“有这样的狐仙,那我聊斋的笔墨也因而有光采了。”于是就记下了这个故事。


\subsection{1.5.12   布 客}
\label{\detokenize{p00_u5176_u5b83/_u767d_u8bdd_u804a_u658b_u5fd7_u5f02:id185}}
长清有个人,靠贩布为生,客住在泰安。听说有个算命的算得很准,便去询问吉凶。算命的给他算了一卦,说:“运数太坏,赶快回家吧!”布客害怕,急忙带着资财北返长清。

路上,布客遇到一个短打扮的人,像是个衙役。两人渐渐搭上话,谈得十分投机、高兴。布客每次买来酒饭,都喊短衣人一起吃,短衣人很感激。布客问他要干什么去,短衣人回答说:“要去长清勾人。”布客问勾什么人,那人拿出一份勾牒,让布客自己看。布客见上面第一个人名就是自己,惊骇地说:“为了什么事要勾我?”那人说:“我不是活人,是鬼都蒿里山东四司的衙役。想必是你寿数已尽。”布客哭着向他求救。鬼衙役说:“这不好办。但勾牒上人名很多,全部拘齐还需要好几天。你赶快回去处理后事,我最后去招呼你,这就算是对我们交好的报答了。”没多久,两人来到一条河边。因为河桥断了,行人都在艰难地涉水过河。鬼衙役对布客说:“你马上就要死了,一文钱也带不走。请你在这里建一座桥,以方便行人。虽然花费不少,但对你未必没有好处!”布客认为很对。

布客回到家中,告诉妻子给自己准备后事。自己纠合工匠,立即去建桥。过了很久,鬼衙役也没来,布客心里不禁暗暗怀疑起来。一天,鬼衙役忽然来了,说:“我已将你建桥的事上报城隍,城隍又转达给冥司,说这件事可以延长你的寿命。现在你已被从勾牒上除名,我特地来通知你。”布客欢喜地道谢。

后来,布客又来到泰安,没忘记鬼衙役的恩德,恭敬地备了香、纸,喊着他的名字祭奠了一番。布客一转身出来,只见那鬼衙役匆匆地赶了来,说:“你差点给我惹了祸!刚才正好司君在处理公务,幸亏他没听见!否则,还以为我在徇私舞弊呢!那可怎么办!”送布客走了几步,又说:“以后不要再来了。倘若我有事去北方,自会绕道去看望你的。” 说完告辞走了。


\subsection{1.5.13   农 人}
\label{\detokenize{p00_u5176_u5b83/_u767d_u8bdd_u804a_u658b_u5fd7_u5f02:id186}}
有一个农夫在山下种地,他的妻子用陶罐给他送午饭。他吃饱以后,就把陶罐放在垄边。傍晚一看,罐里的剩粥一点都没了。这种情况一连发生了好几次。他心里怀疑,于是就一边种地,一边斜着眼睛注意放饭罐的地方。不一会儿,来了一只狐狸,把头伸到陶罐中。农夫扛着锄头蹑手蹑脚地走过去,狠力砸了它一下。狐狸猛吃一惊,急忙逃窜。可陶罐套住了头,怎么也挣不脱。狐狸急得又蹦又跳,猛地跌倒碰碎了陶罐,才露出头来。它见农民追打,窜逃得更急,越过山粱就跑了。

几年以后,山南边有一富贵人家的女儿,被狐狸精迷惑上了,请法师画符念咒全都不管用。狐狸精还对女子说:“纸上的符咒,能把我怎么样!”女子哄骗狐狸精说:“你的道术非常高深,很庆幸和你永远相好。但不知你生来是不是也有惧怕的人?”狐狸精说:“我什么都不害怕。但十年前在北山的时候,曾到田垄边去偷吃剩粥,被一个头戴大苇笠,手持弯脖子兵器的人追打,差一点死在他手里,到现在想起来心里还打颤。”女儿把狐狸精的话告诉了他父亲。父亲想让狐狸害怕的这个人来制服它,但不知道姓名、住址,没法打听。恰巧他家的仆人因事到山村,偶尔向人们谈起他主人家闹狐狸的事情。旁边有一个人吃惊地说:“这和我当年遇上的事正好相符,莫非是被我打过的那只狐狸,现在能兴妖作怪了?”仆人听了觉得很奇怪,就回去告诉了主人。主人非常高兴,当即命令仆人用马把农夫接到家里来,恭恭敬敬地请求他驱赶狐狸。农夫笑着说:“从前我确实遇到过这样一件事情,但不一定就是这只狐狸。况且它既然能成了精来作怪,怎么还会再惧怕一个农夫?”富贵人家再三强求,农夫便打扮成那天追打狐狸时的样子,走进女儿的房间,把锄头往地下一顿,厉声呵叱:“我天天找你找不到,你原来躲藏在这里呀!今天又碰在我手里,一定要杀了你,绝不宽恕!”话音刚落,就听到狐狸在屋里哀叫。农夫越发装出威武盛怒的样子,狐狸精便哀求饶命。农夫叱责说:“马上离开这儿,我就放了你!”女儿见狐狸抱头鼠窜而逃,从此以后,就平安无事了。


\subsection{1.5.14   章 阿 端}
\label{\detokenize{p00_u5176_u5b83/_u767d_u8bdd_u804a_u658b_u5fd7_u5f02:id187}}
河南卫辉府的戚生,年轻含蓄大度,有胆量,敢说敢当。当时一个大户人家有巨宅,因为白天见鬼,家里人相继死去,愿意把宅子贱价卖掉。戚生贪图价廉,便买过来住了。然而宅院太大家人稀少,东院的楼亭,艾蒿长成了小树林,也只好让它暂且荒废着。家人每到夜里便惊恐不安,总是相互惊恐地说有鬼。两个多月后,死了一个丫鬟。没过多久,戚生的妻子傍晚到东院楼亭去,回来以后就得了病,过了几天即死去。家人更加害怕,劝戚生搬家到别处住,戚生不听。然而孤身一人没有伴侣,只有独自凄凉悲伤。丫鬟仆人们又不时地拿发生的怪异现象来喧扰,戚生发了怒,盛气之下抱了被褥,独自躺到荒亭中,留着蜡烛以观察会出现什么怪事。过了很久没有什么动静,也就睡着了。

忽然有人把手伸进了他的被窝,反复地摸索。戚生醒来一看,原来是一个年长的老侍婢,她耳朵蜷曲、头发蓬乱,面目臃肿得很厉害。戚生知道她是个鬼,便抓住胳膊推她,笑道:“尊容不敢领教!”老婢很惭愧,缩回手迈着小步走了。过了一会儿,一个女郎从西北角出来,神情美妙,突然闯到灯下,怒骂道:“哪里来的狂生,居然敢在这里高枕而卧!”戚生坐起来笑答:“小生是这里的房主,等候着向你讨房租呢。”于是起来,光着身子去抓她。女郎急忙逃避。戚生先跑到西北角,挡住了她的退路。女郎没办法,便索性坐到他的床上。戚生靠近她细看,在烛光的映照下竟美如天仙;便渐渐把她拥抱到自己怀里。女郎笑问:“狂生不怕鬼吗?会把你祸害死的!”戚生强解她的衣裙,她也不太抗拒。随后她自己说:“我姓章,小名阿端。因为错嫁了一个刚愎不仁、放荡邪僻的男人,横遭折磨侮辱,使我愤恨郁闷而早亡,埋在这里二十多年了。这宅子下面全是些坟墓。”戚生问:“那老婢是什么人?”女郎答:“也是一个先死的鬼,专门伺候我。上面有生人居住,鬼在下面就不安宁,刚才是我派她来驱赶您的。”戚生又问:“她为什么要摸索我?”女郎笑答:“这老婢三十年从未经历过男女间的事,这是值得怜悯的;但是她也太不自量了。总而言之:心虚胆小的人,鬼越是欺侮折磨他;刚强正直的人,鬼就不敢侵犯了。”听到邻家的钟声响过,女郎穿衣下床,说:“如不被猜疑的话,夜里我定当再来。”

到了晚上,女郎果然来到,两人情意殷切,更加喜悦。戚生说:“我的妻子不幸亡故,悼念之情一直不能忘怀。您能不能为我招她来?”女郎听说后很悲伤,说:“我死了二十年,有谁向我表示过怀念的!您真是多情,我一定竭尽全力。不过听说她已有了投生的地方了,不知道还在不在阴间。”过了一夜,女郎告诉戚生说:“您的娘子将要投生到富贵人家。因为她前生丢失了耳环,拷问鞭打侍女,侍女自缢身亡,这个案子还未完结,为此仍留在阴间。现在还寄居在药王廊下,有人监守着。我已派侍女前往行贿,或许能来。”戚生问:“您为什么能够这样闲散?”女郎答:“凡是屈死鬼不自己去投见的,阎罗王还来不及知道。”二鼓将尽的时候,老婢果然领着戚生的妻子来到。戚生抓住妻子的手大为悲伤。妻子含着眼泪说不出话来。女郎告别,说:“你们两人可以叙谈别后之情,过一夜咱再见面。”戚生问妻子侍女缢死的情况。妻子说:“不要紧,已经完结了。”两人上床拥抱,恩爱欢乐如同生前。从此欢聚成了常事。

五天后,妻子忽然哭着说:“明天将奔赴山东,要长久痛苦地别离了,有什么办法!”戚生听说后,挥泪淋漓,悲哀伤痛难以自持。阿端劝慰说:“我有一个办法,可以使你们得到暂时的团聚。”两人收住眼泪询问她。阿端请戚生拿纸钱十串,焚烧于南屋前的杏树下面,她好带着去贿赂押送戚妻投生的冥吏,以便能延缓时日。戚生按照她说的话办了。到了晚上,妻子来到说:“幸赖端娘帮助,今又得到十天团聚的时间。”戚生大喜,不再让阿端离去,留她同住在一起,每天从傍晚到天晓,惟恐欢乐失去。过了七八天,戚生因为十天期限将满,同妻子整夜痛哭,找阿端想办法。阿端说:“看来很难再有法子。不过还可以再试着办,非冥钱一百万不可。”戚生如数焚烧钱纸。阿端来,高兴地说:“我派人和押生的冥吏说情,起初很难,见到这么多钱后,他的心才开始动摇。现在已经让别的鬼去代替投生了。”从此白天也不再离去,让戚生把门窗塞严,灯烛不灭。

这样过了一年多,阿端忽然病得昏沉沉的,烦躁不安,神志不清,像是见了鬼的样子。戚妻抚摸着她说:“她这是被鬼弄病的。”戚生说:“端娘已经是鬼了,又有什么鬼能使她生病呢?” 妻子说:“不然。人死了变成鬼,鬼死了变成聻。鬼害怕聻,犹如人害怕鬼一样。”戚生想为端娘请巫医。妻子说:“鬼怎么可以让人治疗?邻居王老太太,如今在阴间当巫婆,可以前去请她来。然而离这里十几里路,我的脚柔弱,不能走远路,麻烦您焚烧个纸马。”戚生答应按她的要求去办。纸马刚刚点燃,就见丫鬟牵来一匹黑尾红马,在庭下把马缰绳递给戚妻,转眼之间就不见了。不一会儿,戚妻和一个老太太两人同骑在红马上来到,把马拴在廊柱上。老太太进屋,按着阿端的十指切脉。随后端端正正地坐在椅子上,头哆嗦作态,倒在地上一会儿,突然起来说:“我是黑山大王。娘子病得很重,幸亏遇见小神,福份不浅呀!这是恶鬼作祟,不妨,不妨!只是这病好了,必须重重地给我供养,银子百铤、钱百贯、丰盛酒筵一桌,一样也不能少。”戚妻一一高声应承。老太太又倒在地上再苏醒过来,向病人呵叱,才算完事。过一会老太太要走,戚妻送她到门外,赠送给她那匹马,她很高兴地走了。进屋见阿端,似比原先稍微清醒了些。夫妻二人非常高兴,便安慰她。阿端忽然说道:“我恐怕不能再回到人间了。一闭眼就看见冤鬼,这是命该如此!”于是落下泪来。过了一夜,阿端的病情更加严重,弯曲着身子颤抖着,好像看见了什么。她拉戚生和她卧在一起,把头放进他的怀里,似害怕被人扑捉的样子。戚生一起身,她就惊叫不宁。这样过了六七天,夫妻俩毫无办法。恰巧戚生有事外出,半天才回来,听到了妻子的哭声。惊问缘故,原来阿端已经死在床上,遗骸犹存。掀开被子,只见一堆自骨摆放在那里。戚生大为悲痛,便按生人礼仪把她葬在祖墓旁边。

一天夜里,戚妻在睡梦中呜咽起来。戚生摇醒她并问怎么了,妻子说:“刚才梦见端娘来,说她丈夫已经变成了聻鬼,对她在阴间不守贞节非常愤怒,怀恨追了她的命去,求我作道场。”戚生早起,即要按妻子的话去做。妻子阻止他说:“超度鬼魂不是您可以用上力的。”于是起来走了。过了一会儿回来说:“我已经让人邀请僧侣去了。必须先焚烧钱纸作用场。”戚生都照办了。太阳才落,许多僧人集合到这里,金铙法鼓,如同人间。戚妻虽然常说铙鼓声、诵经声喧扰得难受,戚生却一点也听不见。道场做完了以后,戚妻又梦见阿端来感谢,说:“冤仇已经化解了,将要投生作城隍的女儿。烦代为转达。”

这样过了三年,家里人起初听说都很害怕,时间长了也就渐渐习惯了。戚生不在的时候,家人就隔着窗子向他妻子请示禀报。一天夜里,妻子哭着对戚生说:“原先押生的冥吏,受贿作弊的事情现已败露,追查得很急,恐怕不能长久团聚了。”过了几天,妻子果然得病,说了我因为钟情于您,情愿长死,也不愿意去投生。现在将要永别,难道不是天意吗!”戚生非常恐慌,急忙求她想办法。妻子说:“这已经不可能了。”戚生问:“要受责罚吗?”妻子回答:“小有惩罚。然而偷生罪大,偷死罪小。”说完,就不动了。仔细看去,她的脸面体形,逐渐地消失了。戚生常常独宿在亭子里,希望能再遇到什么,但是最终也没再有什么动静,人心于是也就安定了。


\subsection{1.5.15   馎 饦 媪}
\label{\detokenize{p00_u5176_u5b83/_u767d_u8bdd_u804a_u658b_u5fd7_u5f02:id188}}
有个韩秀才,在别墅中住了半年,年底才回家。一天夜里,他的妻子正在床上躺着,忽然听见有人走路的脚步声。一看,炉子里的炭火烧得很旺,照得屋里非常明亮。见一个老太婆,年纪大约八九十岁,皮肤像鸡皮一样,还驼着背,头上稀疏的白发可以数得清。她对韩妻说:“你吃馎饦吗?”韩妻吓得不敢应声。

老太婆于是用铁筷子拨了拨炉火,把锅放到上面,又往锅里倒水。不一会儿就听见开了锅。老太婆撩起衣襟解开腰上的口袋,拿出数十个馎饦,放进锅里,历历有声。又自言自语地说:“等我找筷子来。”就出了门。

韩妻乘她出去,急忙起来端起锅把馎饦倒在竹席的后面。再蒙上被子躺下。过了一会儿,老太婆回到屋里,逼问锅里的馎饦哪里去了。韩妻吓得大声呼喊,家里的人全醒了,老太婆才离去。拿开竹席用火一照,原来是数十个土鳖虫,堆放在那里。


\subsection{1.5.16   金 永 年}
\label{\detokenize{p00_u5176_u5b83/_u767d_u8bdd_u804a_u658b_u5fd7_u5f02:id189}}
利津县的金永年,八十二岁了还没有儿子,老妻也已七十八岁,自以为绝望了。忽然梦见神人告诉他说:“本来应该断绝你的子嗣,念你做买卖公平,赐给你一个儿子。”金永年醒了就把自己做的梦告诉了老妻。老妻说:“这真是妄想。两人都快要进棺材了,怎么能再生儿子?”不久,金妻真的怀孕了。到了十个月,竟然生下了一个男孩。


\subsection{1.5.17   花 姑 子}
\label{\detokenize{p00_u5176_u5b83/_u767d_u8bdd_u804a_u658b_u5fd7_u5f02:id190}}
陕西有个贡生,名叫安幼舆,为人慷慨有义气,又好放生。如果看见猎人捉住鸟兽,往往不惜高价买下来放掉。

有一次,他舅父办丧事,他去帮忙,回来时天已晚了。路过华山,慌忙中迷了路,在一个乱山谷里打转转,走不出来,心里十分害怕。忽然瞥见一箭地之外有灯光闪烁,便快步投奔那里。正走着,又见几步之外有一个驼背老汉,拄着拐杖从斜路上匆匆赶过来。安生停住脚步,刚想向他问路,老汉却先开口问起他是谁。安生便把迷路情况说了一通,并说看见前边有灯光,一定是山村,要到那里去投宿。老汉说:“那可不是安乐窝,幸亏我来了!快跟我走吧,我家茅庐可以住。”安生十分高兴,跟着老汉走了一里之遥,看见一个小山村。老汉到一个柴门前敲门,一个老太婆出来,一边开门一边问:“郎君来啦?”老汉答应着。安生进屋一看,果然又低矮又潮湿。老汉挑亮油灯,请他坐下,便让备饭。老太婆说:“先生是咱的恩人,不是外人!老婆子腿脚不利索,叫花姑子出来烫酒吧!”

一会儿,一个姑娘端着酒菜出来,摆好后,站在老汉身旁,一双秋水般的眼睛顾盼着安生。安生一看,姑娘年轻俊俏,像个下凡的仙女。老汉又让她去烫酒。西间屋里有个煤火炉,姑娘便进去拨开炭火,烫酒去了。安生便问:“这是您的什么人?”老汉回答道:“老夫姓章,七十多岁了,就这一个女儿。庄户人家没有奴仆,因您不是外人,才敢叫妻子女儿出来,别笑话才是!”安生又问:“许了哪里的婆家?”老汉答:“还没许人!”安生便不住口地夸赞她长得漂亮聪明。老汉正谦让着,忽听花姑子惊叫了一声,急忙跑过去看,原来是酒沸出壶盖火焰腾起。老汉一面把火扑灭,一面申斥说:“这么大丫头啦,烫沸了还不知道!”一回头,看见炉台旁放着一个没编完的青草心插的紫姑神,便又申斥:“辫子这么长了,还跟小孩儿一样!”说着便拿过来给安生看,还说:“就是贪着编这玩艺儿,把酒烫沸了。您还夸奖她,岂不羞死!”安生接过来一看,那紫姑神编得有眉有眼有袍裙,手工十分精致,禁不住啧喷称赞:“别看是个玩物,可也看出慧心!”反复端详着,爱不释手。花姑子频频来斟酒,嫣然含笑,毫无羞涩之态。安生注视着她,十分动情。

恰巧老太婆在厨房里招呼人,老汉应声进去。安幼舆趁机对花姑子说:“一见姑娘的仙容,我的魂儿都丢了。我想托媒来你家说亲,恐怕不成,怎么好呢?”花姑子默默地端着酒壶在炉上温酒,似乎没听见。又问了几次,都不应声。安生就向西屋里凑近,花姑子急忙站起身躲避,厉声说:“狂生闯进来想干什么?”安生长跪地上哀求,花姑子夺门要走,安生突然起身紧紧搂住了她。花姑子尖叫一声,嗓音都颤了。老汉闻声匆匆赶来询问,安生赶紧松开手退出来,一脸羞愧,十分害怕。花姑子却从容地对父亲说:“酒又沸了,要不是安郎过来,酒壶就烧化了!”安生一听,才放下心很感谢她,更加神魂颠倒,忘了是怎样来的。于是装醉离开酒席,花姑子也就去了。老汉给他铺好被褥,也关门离开。安生睡不着,天不明就起身告别回家,立即托一位好友前来作媒说亲。等到黄昏,好友回来了,竟然连村子都没找着。安生不信,又让仆人备马,亲自寻路去找。到了华山一看,尽是高山绝壁,果然不见那个村庄;又到近处打听。山民都说很少听说有姓章的人家。这才无精打彩地回家来。

安幼舆从此昼思夜想,饭吃不下,觉睡不着,不久便患了昏瞀症,卧床不起了。家里人熬粥喂他,也都呕吐出来。他在昏迷中总是呼唤花姑子,家人们也不懂是什么意思,只好日夜守护着,眼看病危了。一天晚上,护理的人实在困倦,睡着了。安生在朦胧中觉得有人轻轻推他,他略睁开眼看,竟是花姑子站在床边,不禁精神清醒,望着她潸潸流泪。花姑子低头凑近他笑着说:“痴情儿何至到这个地步!”说着上床坐在安生的腿上,用两手替他揉搓太阳穴。安生觉得头上像是吹进一股麝香气,穿过鼻梁,一直浸润到全身骨髓里去。揉搓了一会儿,就满头冒汗,渐渐地四肢也汗浸浸了。花姑子小声说:“你屋里人多,我住下不方便。三天后我一定再来看你。”又从花袄袖里掏出几个小圆蒸饼放在床头,悄悄地走了。

到了半夜,安幼舆汗已消去,想吃东西,摸过蒸饼一尝,又甜又酥,不知包的什么馅,就吃了三个。又用衣裳把蒸饼盖住,就呼呼酣睡了。直到上午八九点钟才醒来,浑身顿觉轻松。三天过去,蒸饼吃完,便精神抖擞起来。晚上,安生打发家人们散去,又怕花姑子来了打不开门进来,便偷偷跑到庭院里把门闩都拔掉。不大工夫,花姑子果然来了,笑着说:“痴郎君!不谢谢大夫吗?”安生高兴极了,抱住她同眠,亲爱已极。花姑子说:“我冒着人说闲话的罪名前来,是为了报您的大恩。咱俩并不能百年合好,希望您早点另作打算。”安生默想了半天,便问:“素不相识,什么地方和您有过来往?实在想不起来。”花姑子也不回答,只是说:“您自己再想想。”安生又求花姑子与他正式成婚,花姑子说:“天天夜里来,固然不行;要想结为夫妻,也办不到。”安生一听,不禁一阵悲伤。花姑子说:“您一定要结为夫妇。那就明天晚上到我家来吧。”安生又转悲为喜,问花姑子:“路这么遥远,你一双纤秀的脚,怎么说来就来了呢?”花姑子说:“我本来就没回家。村东头聋老妈是我姨,我住在她家。为了你拖延到现在,说不定家里已经起疑心了。”安生与花姑子同床,只觉得她的肌肤和呼吸,无处不生香气,问道:“你熏的什么香料,以致骨肉都有香味?”花姑子说:“我从来不熏香料,是天生就这样的。”安更惊奇了。

第二天早上花姑子告别时,安生又担心迷路,花姑子便约定在路口等他。天刚擦黑,安幼舆便骑马跑去。花姑子果然在路口迎接,两人一同走进章家院子,老汉老妪高兴地迎他进去。酒菜没有什么名贵佳品,庄户饭菜吃得格外香甜。晚上安生就寝时,花姑子也没过来看看,安生很怀疑。夜深之后,花姑子才来了,说:“爹妈唠叨个没完,叫你久等了。”两人倍加亲热。花姑子对安生说;“今夜的欢会,就是百年之别。”安生惊问为什么。花姑子说:“我爹因为这小村荒凉寂寞,要搬家到远方去了。我和你的欢好,过了这一夜便到尽头了。”安生不愿分手,翻来复去,叹息不止。两人正依依难舍,天透亮了,老汉忽然闯进来骂道:“臭丫头,清白门庭,全被你玷污了!真叫人没脸见人!”花姑子大惊失色,慌忙逃了出去。老汉也退出去,边走边骂不绝口。安生又羞又怕,无地自容,赶紧偷偷溜回。

安幼舆回到家,好几天坐不下来,心神不定,光景难挨。又想夜里再去;越墙进去,见机而作。老汉既说有恩,即使发现了,总不会大加谴责吧。于是乘夜跑去,在大山中转来转去,又迷路了。这才惊恐起来。正在寻找归路,又见山谷里隐隐有所宅院,便高兴地朝那里走去。走近一看,是一座高门大院,像是大户人家,大门还没有关。安幼舆上前敲门打听章家的住处。一个丫鬟走出来问:“深更半夜的,谁打听章家呀?”安生说:“我和章家是亲戚,迷路了,没找到。”丫鬟说:“您不用打听章家啦!这里是她妗子家,花姑正在这里呢,容我去禀报她一声!”进去不大工夫,就又出来邀请安进院。安生刚登上廊下台阶,花姑子已经快步迎接出来,对丫鬟说:“安郎奔波了大半夜,一定累坏了,快侍候床铺让他歇息吧!”不一会儿,两人便携手进入罗帐。安问:“妗子家怎么没有别人呢?”花姑子说:“妗子出去了,留下我替她看家。可巧你就来了,岂不是前世的缘分吗?”可是安生一亲近这女子,一股膻腥昧直冲鼻子,心里好生猜疑。这女子却一把搂住他的脖颈,突然伸出舌尖舔他的鼻孔,安生顿时觉得像锥子扎进脑袋一样痛彻骨髓。他吓坏了,想挣扎逃跑,身子却又像被粗绳捆住,转眼间便昏迷过去,失去了知觉。

安幼舆没回家,家人们四处找遍。忽听有人说黄昏时曾遇见他在山路上走,家人又找到山里,见他已经赤身裸体地死在悬崖下面。家人感到惊异,又琢磨不出是何缘故,只好把他抬回来。全家人正围着他伤心哀哭,忽见一个年轻女子从大门外一路嚎啕大哭着进来吊丧,趴在安生的尸体上,呼天抢地地痛哭起来:“天啊,天啊!怎么糊涂到这地步啊!” 直哭到嗓音嘶哑。才收住泪,向家中人们说:“千万别急着收殓,停尸七天再说。”众人不知这是何人;正要问她,她也不答理,含泪返身出门去了。家人招呼挽留她,她连头也不回,家人紧跟出去,已经无影无踪了。大家疑心她是神仙下凡,赶紧照她的嘱咐办理。夜里她又来了,照样痛哭如昨。

到了第七夜,安幼舆忽然苏醒过来,翻了个身,呻吟起来,家中人们都吓了一跳。这时,女子又来了,安生一见,是花姑子,相对呜呜痛哭起来,安生撰撰手,让众人退出去。花姑子拿出一把青草,煎了一升药汤,就着床头给安生喝下去,一会儿,他就能说话了。他长叹一声说:“杀我的是你,救活我的也是你!”于是把那天晚上的遭遇述说了一遍。花姑子说:“这是蛇精冒充我。你前一次迷路时看见的灯光,便是这东西。”安生说:“你怎么竟能让人起死回生呢?莫非真是神仙吗?”花姑子说:“早就想告诉您,又怕吓着您。您五年前是不是曾在华山路上从猎人手中买下一匹獐子放了?”安幼舆一想:“是啊!有这回事。”花姑子说:“那就是我父亲。上次他说大恩,就是指这件事。您那天晚上已经转生到西村王主政家了。我和父亲赶到阎王面前告状,起初阎王还不受理。是我父亲提出情愿毁了自己多年修炼的道业替你去死,哀求了七天,才得到愚准。今天咱俩还能见面,实在是万幸。可是您虽然活过来了,必定瘫痪;须得蛇血兑上酒喝下去,病才会好。”安生一听,恨得咬牙切齿,又愁没办法把蛇捉住。花姑子说:“这也不难。不过多杀生命,会连累我百年不能得道升天罢了。蛇洞就在华山老崖下,可以在晌午过后堆上茅草去烧,再在洞外准备强弓提防着,一定能捉住这妖物。”说罢,也长叹一声,说:“我不能终身陪伴您,实在令人伤感。可我为了您,十分道业已经损去了七分,您就原谅我吧。这一个月来,常觉得腹中微动,想必是种下孽根了。无论是男是女,一年后一定给您送来。”说着又流下泪来,告辞而去。

安劫舆一夜醒来,果然觉得下半截身子就像死了一样,用手挠挠,毫无痛痒,就把花姑子的话告诉家人们。家人们便按照说的办法到华山老崖下蛇洞口点起火来。果然有条大白蛇冒着浓烟钻出来,家人们一齐放箭,把它射死了。火熄灭以后,他们进洞一看,大小数百条蛇也都烧焦了。家人们把死蛇运回家,煎蛇血药物给安幼舆喝下去。服了三天,两腿渐渐能够转动,半年后就能下床走路了。

后来安幼舆因思念花姑子,又独自到华山里去,在山谷中遇见了章老太太,抱着一个襁褓婴儿交给他说:“我女儿她向您致意、问候。”安幼舆刚想打听花姑子的消息,老太婆却转眼间消失了。安幼舆把小被褥打开一看,是个男孩,急忙抱回家来抚养,终生没再娶妻。


\subsection{1.5.18   武 孝 廉}
\label{\detokenize{p00_u5176_u5b83/_u767d_u8bdd_u804a_u658b_u5fd7_u5f02:id191}}
石某是个武孝廉,他带着钱去京城,准备到朝中谋求个官做。到了德州,忽然得了重病,咳血不止,病倒在船上。他的仆人偷了他的钱跑了,石某十分气愤,更加重了病情,钱粮俱断,船主也打算赶他下船。正在这时,有一个女子夜里驾船来停在一旁,听到这事后,就自愿叫石某上她的船;船主很高兴,就扶石某上了女子的船。

石某见这女子约有四十多岁,穿得很华丽,还很有神采风韵,他呻吟着向她表示了谢意。女子走到石某近前看了看他的面容,对他说:“你本来就有病根,现在魂已出了舍,游于坟墓问了。”石某听了,吓得嚎啕大哭。女子说:“我有药丸子,吃了可以起死回生。你若好了,可不能忘了我。”石某哭着对天盟誓,誓死不赢救命之恩。妇人随即拿药丸给石某服下。过了半天,石某觉得稍好了一些,女子就到床前喂石某好东西吃,侍奉得十分殷勤,胜过夫妻。石某越是感激不尽。

一个月后,石的病就全好了;他跪着爬向女子,敬她犹如敬母。女子对他说:“我孤单一人,没有依靠,你若不嫌我年纪大,我愿与你结为夫妻。”当时石某三十多岁,妻子死了一年多了,听了女子的话,喜出望外,于是两人便同床共枕,互相爱怜。女子拿出钱来给他去京求官,并且约定好,一旦有了官职,回来接她一起回家。

石某到了京城,用女子的钱贿赂朝官,得到了本省司阃的官职;剩下的钱买了华丽的车马,准备回家。这时候石某想,船上的女子年纪太大,终归不是合适的妻子。于是又用一百两银子聘了王氏女为继室。他心中有愧,怕女子知道,就绕开德州前去赴任。到任后一年多没有给女子去信。

石某有个表弟,偶然到德州办事,与女子住近邻。女子知道他和石某的关系,就问石某的情况,表弟就如实告诉了女子。女子听了大骂,并把她怎样救石某的情况也告诉了石的表弟。表弟为她不平,劝慰女子说:“我表哥可能因为公务繁忙,没有工夫来接你,请写封信由我转达他。”于是女子写了信,由石的表弟捎去。然而石某一点不放在心上。

又过了一年多,女子自己去找石某,到后住在一家旅店里。找到石某官衙门前,请看门的给通报一下,石某拒不接见。

一天,石某正在喝酒,听到大门外有喧骂声。他放下杯正听时,女子已掀帘进了屋子。石某吓了一跳,面如土色。女子指着他骂道:“无情郎,你好快乐!不想想你的富贵是哪里来的?我对你情分不算薄,你就是想娶个妾,和我商量一下何妨?”石某一句话也说不出。过了好长一会儿,石某才跪在地下自己认错,花言巧语地乞求饶恕。女子的气才稍稍平静下来。石某与王氏商量,叫王氏以妹妹的身份向女子见礼,王氏不同意;石某一再要求,王才答应了,去拜见女子。女子也回拜了王氏,并对王氏说:“妹妹不要担心,我并不是妒嫉厉害的女人。他做的事,实在不近人情,就是妹妹你也不愿意有这样的男人。”于是便向王氏讲了以前的经过,王听了也很气愤。她俩交替着骂石某,石某惭愧得无地自容,唯要求今后自己赎罪。这才安静下来。

在这之前,女子还没有来时,石某已告诉看门的,若有女人来不要通报。事已至此,石就迁怒看门人,暗中责备看门人不应给女子开门。可是看门的却坚持说大门一直锁着,没进来什么女人。石某对女子产生了怀疑,又不敢再去问。他与女子表面上有说有笑,但貌合神离。幸亏女子贤惠,从不争晚上与他在一起。一日三餐后,便关上门自已早早睡了,从不问石某睡在哪里。王氏起初对女子有些害怕,怕与自己争男人;见女子这样,就更加敬重她,早晚问候,像伺候婆婆一样。

女子对下人宽和体谅,但却明察秋毫。一天,石某失了官印,合府沸腾,都走来走去,无计可施。而女子却笑着说:“不用愁,把井里的水淘干了,就能找到。”石某照办了,果然官即找到了,问她是怎么回事,她只是笑,却不回答。看样子,她好像知道偷印人是谁,但一直不肯说出来。

又住了近一年,石某观察女子一举一动,有许多奇异的地方。便怀疑女子不是人类,常叫人偷听女子夜里说些什么。下人说只听到她终夜在床上有振衣服的声音,也不知道是为什么。

女子与王氏十分亲密。一晚,石某到上司官署去没有回来,女子就与王氏饮酒。因多喝了几杯,就醉了。伏在桌子上现了原形,变成了一只狐。王氏十分怜爱她,就给她盖上被子。过了一会,石某回来,王氏告诉他女子的情况,石某想杀了女。王氏说:“她就是狐,哪里对不起你?”石某不听,急忙找佩刀要动手,而女子已经醒来。她对石某骂道: “你真是蛇蝎行为,豺狼心肠,一定不能与你常住在一起了。以前我给你吃的药丸子,请你还给我!”说罢朝石某脸上唾去,石某觉得像冰水一样凉,顿时喉咙一阵发痒,吐出了药丸子,这丸子仍如以前一样。女子抬起丸子,气愤地走了。石与王氏追出看时,已无影无踪了。石某当天夜里旧病复发,咳血不止,半年工夫就死了。


\subsection{1.5.19   西 湖 主}
\label{\detokenize{p00_u5176_u5b83/_u767d_u8bdd_u804a_u658b_u5fd7_u5f02:id192}}
书生陈弼教,字明允,河北人。他家里很贫穷,跟着副将军贾绾当文书。一次,陈生和贾绾在洞庭湖停船,正巧一条猪婆龙浮出水面,贾绾一箭射去,正中猪婆龙的背。有条小鱼衔着龙尾巴不走开,一起被捉住了。猪婆龙被拴在船桅上,奄奄一息,嘴巴还一张一合,似乎在恳求援救。陈生很可怜它,便向贾绾请求放了猪婆龙,还把随身带的金创药试着涂在它的箭伤上。把龙放入水中,见它浮游了一会,消失不见了。

过了一年多,陈生返回北方老家,再次经过洞庭湖时,遭遇大风,船被打翻。陈生幸亏扳着一个竹箱子,漂泊了一夜,才被树挂住。刚爬上岸边,水上漂过来一具尸体,原来是他的童仆。陈生将尸体用力拉上来,童仆早已死了。陈生伤心悲哀,面对着尸体坐下歇息。看看前方,只见小山起伏,一片苍翠,青青的细柳在风中摇曳,没有一个行人,也无法问路。从早晨一直坐到太阳老高,心中迷惘,无处可去。忽见童仆四肢微微动了动,陈生高兴地给他按摩,不一会儿,童仆吐了几斗水,一下子醒了过来。两个人都把湿衣服脱下来晒到石头上,快到中午时才干了穿上。但是饥肠辘辘,饿得不能忍受,于是翻山急走,盼望能找到个村庄。

刚走到半山腰,忽听有响箭声。陈生正在惊疑地细听,有两个女郎骑着骏马飞驰而来,都用红巾包着额头,发髻上插着雉尾,穿着小袖紫衣,腰扎绿锦带。一个手持弹弓,另一个胳膊上套着架鹰的皮套。陈生和童仆越过山岭,见又有几十个人骑着马在树丛里打猎。全都是漂亮的女子,一样的打扮。陈生不敢再往前走。这时有个男子跑了过来,像是个马夫,陈生便向他打听。马夫说:“这是西湖主在首山打猎。”陈生讲了自己的来历,而且告诉他自己和童仆都很饿了。马夫解开包裹,拿出干粮给他,嘱咐说:“赶快远远地避开,犯了西湖主的驾,要被处死!”陈生害怕,急忙下山。

忽见一片茂密的树林中,隐隐约约露出殿阁,陈生以为是庙宇。走近一看,粉白的围墙环绕着,墙外是一道溪水。红漆大门半敞开着,有座石桥通向大门。陈生扒着门往里一看,楼台水榭,高耸入云,比得上皇家花园,又怀疑是富贵人家的园亭。陈生犹豫着走了进去,古藤挡路,花香扑鼻。走过几折曲栏,又是一个院子。几十株高大的垂杨,枝条轻拂着红色的屋檐。山鸟一叫,花片齐飞;深苑微风吹过,榆钱飘飘落下。陈生赏心悦目,恍如进入了仙境。穿过一个小亭,有架秋千,高入云间。秋千索静静地垂着,杳无人迹。陈生怀疑已走近闺阁,惶恐地不敢再往前走。一会儿听见从大门外传来马蹄声,似乎有女子的笑语,陈生和童仆忙藏到花丛里。过了不久,笑声渐渐走近,听一个女子说道:“今天打猎的运气不好,猎物太少了。”又一个女子说:“要不是公主射下来几只飞雁,几乎空劳人马。”不一会儿,几个红衣女子簇拥着一个女郎到亭上坐下。那女郎穿着短袖戎装,大约有十四五岁。头发犹如一团云雾,纤细的腰肢像经不起风吹,即使是玉蕊琼花也比不上她的美貌。女子们有的捧茶,有的熏香,华丽的衣服光灿灿的犹如堆锦。过了会儿,女郎起身,走下石阶。一个女子说:“公主鞍马劳累,还能打秋千吗?”公主笑着答应。女子们有的架着肩膀,有的搀胳膊,有的提裙子,有的拿鞋,把公主扶上了秋千。公主伸开雪白的手臂,脚下用力,像轻轻的飞燕一样,直入云霄。打完秋千,女子们扶公主下来,都说:“公主真是个仙人啊!”嬉笑着走了。

陈生偷看了很久,心神飞扬。等笑语声消失后,他从花丛里出来,到秋千下徘徊凝思。见篱笆下有条红巾,陈生知道是刚才的女子们丢的,喜欢地拾起来技到袖子里。登上那个小亭,见案上摆着文具,陈生便在红巾上题了首诗:“雅戏何人拟半仙?分明琼女散金莲。广寒队里应相妒,莫信凌波上九天。”写完,一边吟咏着一边走下亭子。顺原路往回走,却见一重重的门都上了锁了。陈生彷徨无计,又返回来把楼台亭阁游历了个遍。

一个女子悄悄地进来,看到陈生吃惊地问:“你怎么来到这里?”陈生作了一揖说:“我是迷路的人,请能救助我!”女子问:“拾到一条红巾了吗?”陈生说:“抬到一条,但已被弄脏,怎么办?”便拿出那条红巾。女子大惊,说:“你死无葬身之地了!这是公主常用的东西,你涂成这个样子,怎么交待!”陈生吓得脸上失色,哀求女子代为求情免罪。女子说:“你偷看宫廷里的情形,已经罪不可赦;念你是个文雅书生,本想私下周全你,现在你自己作了孽,我有什么办法?”说完慌慌张张地拿着红巾走了。陈生心惊肉跳,恨没有翅膀飞走,只有伸着脖子等死了。过了很久,那女子又来了,悄悄祝贺说:“你求生有望了!公主看了三四遍红巾,面色坦然,没有生气,或许会放你走。你应该耐心等着,不要爬树跳墙,发现了就不饶恕了!”

这时,天色已晚。是吉是凶还说不定,又饥饿难忍,陈生心中忧愁得要死。不长时间,那个女子挑着灯来了。一个丫鬟提着饭盒酒壶,让陈生吃饭。陈生急忙打听消息,女子说:“刚才我找了个机会跟公主说:‘花园里那个秀才,能饶恕就放了他吧;不然,快饿死了。’公主沉思了一会儿,说‘深夜让他到哪里去?’于是让我来给你送饭。这不是坏兆头。”陈生徘徊了一整夜,惶惶不安。第二天太阳升起很高了,女子又来送饭。陈生哀求她替自己讲情。女子说:“公主不说杀,也不说放,我们这些仆人怎敢絮絮叨叨,自讨没趣?”等到太阳西斜,陈生正殷切地盼望着,女子忽然气喘吁吁地跑了来,说:“坏事了!不知哪个多嘴的把这事泄露给了王妃。王妃展开红巾一看,扔在地上,大骂狂妄,大祸就要临头了!”陈生大惊,面如灰土,跪在地上求救。忽听人声喧哗,女子摇着手躲开了。有几个人手拿绳索,气势汹汹地闯过来。其中一个丫鬟端详着陈生说:“我以为是谁呢,是陈郎吗?”于是止住拿绳索的人,说:“先不要动手,等我去禀告王妃。”返身急急忙忙地走了。过了会儿又回来,说:“王妃请陈郎进去。”陈生战战兢兢地跟着她,绕过几十重门户,来到一座宫殿,门上挂着碧色的帘子,白银帘钩。立即有个美丽的女子掀开门帘高呼道:“陈郎到。”陈生见座上有个美丽的妇人,穿着光闪闪的袍服,急忙跪地叩头。说:“远方的孤臣,请求饶命!”王妃忙起身,亲自拉起他来,说:“我如不是你,不会有今天。丫鬟们无知,冒犯了贵客,罪不可赎!”便命摆下丰盛的酒席,让陈生用雕花的酒杯喝酒。陈生茫然不解,不知是什么缘故。王妃说:“救命之恩,常恨无以为报。我的小女儿承蒙你题巾相爱,当是天定缘份,今晚就让她侍奉你。”陈生大感意外,神情恍恍惚惚,没个着落。

天刚晚,一个丫鬟进来禀报:“公主已梳妆完了。”于是领着陈生去新房。忽然笙管齐鸣,台阶上铺着花毡,门前堂上、篱笆墙角,到处都挂着灯笼。几十个妖艳的女子,扶着公主和陈生交拜。兰麝的香气,充溢殿庭。交拜完,陈生和公主相互搀扶着进入床帐,十分恩爱。陈生说:“寄身在外的人,平素没来拜见,玷污了您的芳巾,免于被杀,已很幸运了;反而赐婚姻之好,实在没想到。”公主说:“我的母亲,是洞庭湖君的妃子,是扬子江王的女儿。去年她回娘家,偶然在湖上游着,被流箭射中。承蒙你相救,又赐刀伤药,我们全家都非常感激,一直记在心中。你不要因为我是异类而疑虑,我跟着龙君得到了长生秘诀,愿和你共享。”陈生才醒悟是神人,便问:“那个丫鬟怎么认得我?”公主说:“那天在湖中船上,曾有条小鱼衔着龙尾,就是这个丫鬟。”陈生又问:“既然你不杀我,为什么迟迟不放我走?”公主笑着说: “我实在是喜爱你的才华,但又不能自己作主。辗转了一夜,别人哪里知道。”陈生叹息说:“你真是我的知音啊!那个给我进饭的是谁?”公主回答说:“她叫阿念,也是我的心腹。”陈生问:“怎么报答她呢?”公主笑着说:“她侍候你的日子还长着呢,慢慢再报答她也不迟。”陈生又问:“大王在哪里?”公主说:“跟着关公讨伐蚩尤还没回来。”

过了几天,陈生担忧家里得不到消息,会十分挂念,便先写了封平安家信,派自己的童仆送去。家里的,人听说陈生在洞庭湖翻了船,妻子已戴了一年多的孝了。童仆回来,才知道他没死,但音讯隔绝,终究还是怕陈生难以返回。

又过了半年,陈生忽然回来了。衣服马匹都非常漂亮,口袋里装满宝玉。从此陈生家资万贯,声色豪华,那些富贵大家都比不上。在后来的七八年里,陈生生了五个儿子。天天设宴招待客人,房屋、饮食都穷极奢侈丰盛。有人问陈生的经历,陈生都详细叙述,一点也不隐瞒。

有个和陈生童年就要好的朋友梁子俊,在南方做官十几年,回家时路过洞庭湖,见一只画船,雕栏红窗,笙歌悠扬,缓缓地飘荡在烟波之中。不时有个美人推开窗子往外眺望。梁子俊往船中望了望,见一个少年男子未戴帽子盘腿坐在船上,旁边有个十五六岁的美丽女子,正给他按摩。梁子俊以为必定是这一带的大官,但随从却很少。又仔细打量了一下,却原来是陈明允。梁子俊不觉倚着船栏干大声叫他。陈生听到喊声,命停船,出来到船头上邀请梁子俊过船来。梁子俊见船内剩菜满桌,酒雾仍浓。陈生立命将残席撤去,只一会儿,便有三五个美丽丫鬟捧上酒来,泡上好茶,山珍海味纷纷摆了上来,都是没见过的。梁子俊惊讶地说:“十年不见,怎么竟富贵到如此程度?”陈生笑着说:“你小看穷书生不能发迹吗?”梁子俊问:“刚才和你一块喝酒的是谁?”陈生说:“是我的妻子。”粱子俊更感惊异,问:“你带着家眷要去哪里?”陈生回答说:“往西方去。”梁子俊还要再问,陈生急忙命奏乐劝酒。一句话刚说完,只听乐声如旱雷般震耳,一片嘈杂,再也听不见说笑声了。梁子俊见美人站满桌前,乘醉大声说:“明允公,能让我真个消魂吗?”陈生笑着说:“你醉了!但有点足够买个美妾的钱,可以赠给老朋友。”于是命丫鬟送上明珠一颗,说:“凭这个不难买个美女,以说明我不是吝惜。”说完,告辞说:“小事紧迫,来不及跟老朋友久聚。”把粱子俊送过船去,陈生的船便解开缆绳,径自走了。

梁子俊回来后,到陈生家里探望,见陈生正在和客人喝酒,心中越发惊疑。便问:“昨天还在洞庭湖,怎么这么快就回来了?”陈生回答说:“没有的事!”梁子俊便追述了当时的情景,满座人都惊骇不已。陈生笑着说:“你弄错了!我难道会有分身术吗?”大家都很惊异,但终究不解是什么缘故。

后来,陈生活到八十一岁时去世。下葬时,人们惊讶棺材太轻,打开一看,只是一具空棺而已。


\subsection{1.5.20   孝 子}
\label{\detokenize{p00_u5176_u5b83/_u767d_u8bdd_u804a_u658b_u5fd7_u5f02:id193}}
青州东香山前的村子里,右个叫周顺亭的人,侍奉母亲最孝顺。母亲大腿上生了个很大的毒疮,疼痛得难以忍受,白天黑夜都皱着眉头呻吟。周生为母亲擦洗换药,到了废寝忘食的地步。但过了几个月仍不见痊愈,周生忧心如煎毫无办法。

有天夜里,周生梦见父亲对他说:“你母亲的病全靠你的孝顺。然而这种疮口不用人膏涂抹是不能治愈的,焦急悲痛也是徒劳。”周生醒来感到很奇怪。于是起床,用很锋利的刀子割自已腰侧的肉;肉割下来了,觉得不太痛苦。急忙用布缠住腰际,血也不往外流。于是把肉烹制成膏,拿去敷在母亲疮口上,疼痛立时就止住了。母亲大喜,问:“什么药这样灵验?”周生没对母亲说实话。母亲的疮口很快痊愈了。周生时时掩盖着自己刀割的伤口,就连妻子和孩子也不知道。他的伤口愈合以后,留有巴掌大的一块疤痕。妻子再三追问他,才得知真情。


\subsection{1.5.21   狮 子}
\label{\detokenize{p00_u5176_u5b83/_u767d_u8bdd_u804a_u658b_u5fd7_u5f02:id194}}
暹罗国来进贡狮子,每住到一处,就吸引很多人来围观。这狮子的形状和世间流传的刺绣画上的大不一样,它的毛是黑黄色,长约数寸。有的人扔给狮子一只鸡,它先用爪子抟弄后再用嘴吹;一吹,那鸡毛就全都掉光,像扫尽了一样。这也是事物规律中的奇怪现象。


\subsection{1.5.22   阎 王}
\label{\detokenize{p00_u5176_u5b83/_u767d_u8bdd_u804a_u658b_u5fd7_u5f02:id195}}
李久常,是山东临朐人。有一次他带着酒食野游,见一股旋风呼呼响着旋转过来,便很恭敬地把酒洒在地上祭奠它。后来他因为有事到某个地方去,看到路边有一处很宽广的庭院,殿阁恢宏壮丽。一个青衣人从里面出来,邀请他进去。李生坚决推辞。青衣人拦住他的去路很殷切地邀他进门。李生说:“我向来不认识您,是不是认错了人呀?”青衣人说:“没认错。”便说出李生的姓名来。李生问:“这是谁家?”青衣人回答:“进去您自己就会知道的。”李生进去,过了一层门,见有个女子手足钉在门板上。走近一看,竟是他的嫂子。他有个嫂子,臂上生恶疮,已经一年多不能起床了。李生心想她怎么能到这里呢。转而怀疑青衣人招他进来没怀好意,心里害怕便停住了脚步。青衣人催促他,才又往里走。

到了大殿下,见上面坐着一个人,衣冠服饰像是帝王,样子很威严。李生跪伏在地上,不敢抬头再看。阎王命令青衣人把李生拉起来,安慰他说:“不要害怕。我因为过去吃过你的酒食,想见见面表示感谢,没有别的事。”李生这才放了心,然而终归不知道是怎么回事。阎王又说:“你不记得在田野里酹酒祭奠的时候了吗?”李生顿时醒悟过来,知道他是神。便叩头说:“刚才见我嫂子受这么严厉的刑罚,骨肉之情,心里实在悲伤。乞求大王能可怜饶恕她!”阎王说:“她太悍妒,应该得到这种惩罚。三年前,你哥哥的妾生孩子时肠子盘绕而下,你这个嫂子竟暗暗把一根针刺在妾的肠子上,使妾至今脏腑常痛。这种做法哪还有点人性!”李生再三哀求他,阎王这才说: “就看在你的面子上饶恕了她。你回去应当劝这个悍妇痛改以前的恶行。”李生谢过阎王后往外走,一看门板上钉着的嫂子已经不见了。

李生回家去看嫂子,嫂子躺在床上,疮口流出的血殷透了床席。当时正因为妾做事不合她的心意,她正在对妾诟骂。李生就劝她说:“嫂子不要再这样了!今日的痛苦,都是平日嫉妒造成的。”嫂子生气地说:“小叔这么个好男子;屋里的小娘子又贤慧得像孟光,任郎君东家眠,西家宿,而不敢出一声。就算是小叔有最好的夫纲,也用不着你来替哥哥降伏老太婆!”李生微笑着说:“嫂子不要发怒。我若说出真情,恐怕你想哭都来不及了。”嫂子说:“我既没去偷王母娘娘笸簪中的线,又没和玉皇大帝的香案吏眉来眼去,心怀坦荡,哪个地方用得着哭了!”李生小声说:“你用针刺在人家的肠子上,该当何罪?”嫂子突然变了脸色,就问这句话的来由。李生便把在阎王殿前看到的情形和阎王说过的话告诉了她。

嫂子听说后吓得不住地颤抖,泪流满面地哀告道;“我不敢了!”啼泪还没干,就觉得疮痛顿时停止,过了十天就痊愈了。从此她立改以前的恶行,于是人们都称赞她贤淑。后来妾再生孩子的时候,肠子又垂下来,针还仍然刺在上面。把针拔去以后,妾的腹痛病才好。


\subsection{1.5.23   土 偶}
\label{\detokenize{p00_u5176_u5b83/_u767d_u8bdd_u804a_u658b_u5fd7_u5f02:id196}}
沂水县有个姓马的人,娶王氏为妻,夫妻感情非常深厚。马生不幸早亡。王氏的父母想让她改嫁,她发誓不嫁。婆婆怜她年轻,也劝她,王氏就是不听。母亲说:“你的心意很好;然而年龄太小,又没生孩子,常见有些人起初勉强不嫁,可后来却留下耻辱,所以不如趁早改嫁,这也还是人之常情。”王氏脸色严肃,誓死不嫁,母亲便听任了她。王氏让泥塑匠工为丈夫塑了座土偶像,每次吃饭,都要为夫像端献酒食,像他活着时那样。

一天夜里,王氏将要睡觉,忽然看见土偶人打了个呵欠伸了伸懒腰走了下来。王氏心情紧张,很惊讶地看去,土偶已猛然长得像人一样高,还真是她的丈夫。王氏害怕了,便呼唤母亲。鬼制止她说:“不要呼喊。感谢你的深情,我在阴间心里很难过。一门中有忠贞之人,数世祖宗,都有光荣。我的父亲生前有损德的地方,应该绝嗣,以致促我年轻轻地死去。冥司念你苦守贞节,所以让我回家来,再和你生一个儿子继承宗嗣。”王氏听了也涕泪沾襟。于是两人亲好如生时。鸡叫的时候,鬼就下床去了。这样过了一个多月,王氏觉得腹中微动。鬼这才哭着说:“阴间的期限已经到了,从此永别了!”就此绝迹。

王氏起初没有对人说过;不久腹部渐渐胀大,不能再隐瞒了,就偷偷地把事情的经过告诉了母亲。母亲怀疑她说谎;然而观察王氏又没与别人来往过,因此非常疑惑不解。到了十个月上,王氏果然生了男孩。对人说起这件事,听到的人无不偷笑;王氏自己也没法辩白。有个里长过去和马家有仇,就把王氏告到了县令那里。县令传拘邻人进行审讯,并无别的说法。县令说:“听说鬼的儿子没有影子,若有影子就是假的。”把孩子抱到太阳下,见影子泼如轻烟。又刺破孩子手指把血涂到马生的土偶像上,立刻渗入进去不留痕迹;再涂到别的土偶上,一擦便擦了去。因此都相信了王氏的话。后来孩子长到几岁,他的相貌和言谈动作,没有一点不酷似马生的,众人的疑惑才解开。


\subsection{1.5.24   长 治 女 子}
\label{\detokenize{p00_u5176_u5b83/_u767d_u8bdd_u804a_u658b_u5fd7_u5f02:id197}}
陈欢乐,是山西潞安府长治县人。他有个女儿,又聪明又美丽。有个道士来募化行乞。瞥眼看了看她就走了。从此以后道士每天都拿着钵来附近转游。恰好一个盲人从陈家出来,道士追上去和他一起走,问他干什么来了。盲人说:“刚才是去陈家推八字算命。”道士说:“听说他家有个女儿,我的一个姑表弟,想和她家作亲,只是不知道她的年岁生辰。”盲人便把陈女的生辰八字告诉了他,道士于是告别而去。

过了几天,陈女在屋里刺绣,忽然感到脚有点麻木,渐渐麻到大腿,又麻到腰腹,不久便晕倒在地上。镇定了好一会,才迷迷糊糊能站起来,想去告诉母亲。一出房门,见茫茫一片黑色水波中,只有一条像线一样细的小路,吓得她直往后退。这时大门房舍和自己的住屋,已被黑水淹没。再看那条小路上,几乎没有行人,惟有一个道士缓慢地走在前面。陈女就远远地跟随在他的后边,希望能见到本村人对他们说说。走了几里路以后,忽然见有了村舍,看了看,竟是自己的家门,便非常惊讶地说:“跑了这么多路,原来还在村里。怎么先前迷糊成这个样子!”她很高兴地进了家门,见父母还没回家。又回到自己的房问,看见原来没绣完的花鞋,仍然放在床上。自己觉得跑得实在累极了,便坐在床上休息。道士忽然进来了,陈女大惊,想逃走。道士捉住她用力按住。陈女想呼喊,但嗓子哑了喊不出声来。道士急速用快刀子剖开她的心,陈女觉得灵魂飘飘忽忽离开身躯立在一旁。四面一看家舍全没了,只有崩裂的山崖覆盖着。见道士用她的心血点在一个木人身上,又叠起手指念起咒语,陈女觉得木人和自己合在了一起。道士叮嘱说:“从此以后一定要听我的差遣,不得违误!”于是就把她佩戴在身上。

陈家丢失了女儿,全家惊慌疑惑。寻找到牛头岭,才听到村人传说,岭下有一个女子被剖心而死。陈欢乐急忙奔去查看,果真是自己的女儿。他哭着向县令诉说。县令拘捕了岭下的居民,拷问了多次,终究没有头绪。便暂且把这些嫌疑犯收监,留待查问。

道士走到几里路外,坐在路旁的柳树下,忽然对陈女说:“今天派你第一件差事,去侦察县衙中审案的情况。去后一定要隐藏在大堂天棚上。倘若看见县令使用大印,必须赶快躲避!切切牢记不能忘了!限你辰刻去巳刻回来。若晚一刻,就用一根针刺在你心中,使你剧疼;晚两刻,刺两针;到第三针时,就使你的魂魄销毁了。”陈女听说后,浑身颤抖,飘然而去。瞬息间到了县衙,按道士所说的那样潜伏在天棚上。当时被拘的岭下居民都排列着跪在堂下,还没有审问。正好遇上要给公文盖印,陈女还没来得及躲避,官印已经出了匣子。陈女感到身体沉重疲软,天棚的纸格好像承担不了她的重量,突然暴裂出声,满堂人都惊讶地抬头看。县令再举官印,暴裂声音又响;第三次举官印时,陈女从天棚上翻坠地上。众人全都听见了。县令起身祝祷说:“如果是冤鬼的话,就应当直说,可以为你昭雪。”陈女哽咽着来到案前,一一诉说道士杀她的经过,派遣她来侦察的情形。

县令派衙役骑快马去,到了柳树下,道士果然还在那里。捉住他带回来,一审讯就服罪了。那些嫌疑犯于是都被释放了。县令问陈女:“你的冤情昭雪了,要到哪里去?”陈女说:“要跟从大人。”县令说:“我的官署中没地方可以容你,不如还是暂时回到你家去吧。”陈女过了很久说:“官署就是我的家,我这就进去了。”县令再问,已经寂然无声。他退堂后回到自己的住处,夫人刚刚生下个女孩来。


\subsection{1.5.25   义 犬}
\label{\detokenize{p00_u5176_u5b83/_u767d_u8bdd_u804a_u658b_u5fd7_u5f02:id198}}
山西潞安府某甲,他的父亲被人陷害入狱,将要处死。他搜括家里的积蓄,凑足了一百两银子,将要到郡城去托人说情。跨上骡子出了门,见自己喂养的黑狗跟在身后,他便呵叱它把它赶了回去;再往前走,狗又跟着走,用鞭子赶它也不回去。这样一直跟着走了几十里路。

某甲从骡子上下来,走到路旁去小便。随后就扔石块打狗,狗这才往回跑去。他刚开始走,见狗又急速而来,咬骡子的尾巴和蹄子。某甲生气地拿鞭子抽它,狗狂吠不已。忽然跳跃到前面,愤怒地咬骡子的头,好像想阻挡它的去路。某甲认为这是个不祥的预兆,更加生气,便掉转骡头往回奔驰驱赶狗。见狗已经跑远了,才返身疾驰,到了郡城时天已傍晚。等到一摸腰间的口袋,里面的银子丢了一半。他的汗水涔涔而下,就像丢了魂一样,翻来复去的一夜没睡好,顿时明白犬吠有因。

他早早地到城门口等候开门出披,仔细查看来的路上。又自念这南北大道上,行人多如蚂蚁,丢了的银子还能存在原处吗?某甲犹豫徘徊,到了下骡子解溲的地方,发现自己的黑狗死在草丛间,身上的毛被汗水湿透,就像水洗过的一样。提起它的耳朵一看,原来丢失的银子全在它的身子底下盖着。某甲非常感激狗的仁义,便买了棺材把它埋葬了。人们都把这个坟叫作“义犬冢”。


\subsection{1.5.26   鄱 阳 神}
\label{\detokenize{p00_u5176_u5b83/_u767d_u8bdd_u804a_u658b_u5fd7_u5f02:id199}}
益都县人翟湛持,去江西饶州任司理官职,路经鄱阳湖。见湖边有座神祠,便停车游览瞻仰。见里面雕塑着丁普郎等明太祖死难功臣诸像,翟姓的神像排在最末位上。翟司理说:“我们家族的人,怎么能在下首!”于是把翟姓神像和上首的神像换了个位置。

不久登船,突然大风刮断了船帆,桅杆歪到一边,全船人吓得大声哀号。顷刻间有只小船破浪驶来,靠近官船,急忙先把翟司理扶了过去,于是家人全都上了小船。仔细一看小船的主人,竟和祠中的翟姓神像一模一样。过了一会儿,风浪平息,再寻找小船主人已经不见了。


\subsection{1.5.27   伍 秋 月}
\label{\detokenize{p00_u5176_u5b83/_u767d_u8bdd_u804a_u658b_u5fd7_u5f02:id200}}
高邮人王鼎,字仙湖,为人慷慨,勇猛春力,交游很广。年十八岁,还没成亲,未婚妻就死了。他每次出去游历,常常是一年多不回来。哥哥王鼐,是江北的名士,对弟弟很友爱,劝弟弟不要再外出,要为他选个媳妇。王鼎不听,乘船到镇江拜访朋友。正巧朋友外出,王鼎便租了一家旅店的阁楼住下。楼外江水如练,碧波荡漾,金山尽收眼底,令王鼎心矿神怡。第二天,朋友来请他搬到家里去庄,王鼎推辞不去。

在楼上住了半月多。一夜,王鼎梦见一个女郎,大约十四五岁年纪,容貌秀丽端庄,上床跟他交合,醒后已经梦遗了。王鼎感到很奇怪,还以为是偶然的。到了夜晚,又做了个同样的梦。这样过了三四夜,王鼎非常惊异,睡觉时不敢熄灯,身子虽然躺在床上,心里却很警惕。刚闭上眼睛,梦见女郎又来了。正在亲热,王鼎猛然惊醒,急忙睁眼一看,一个美如天仙的少女,还在自己的怀抱中。少女见王鼎醒了,露出羞愧怯弱的样子。王鼎知道她不是人类,但很爱怜,来不及询问,又和她亲热起来。女子像是受不了,说:“如此狂暴,难怪人家不敢告诉你!”王鼎才开始询问她。女子说:“我姓伍,名叫秋月。先父是名儒,精通易理,对我很爱怜。但说我不长寿,所以不令我嫁人。我到了十五岁时果然死了,父亲便把我埋在阁东,坟墓和地一样平,坟上也没标志;只在棺材一边立了片石块,写着‘女秋月,葬无冢,三十年,嫁王鼎’。现在已过了三十年,正好你来了,我很高兴,急着想主动见你,心里害羞,所以借做梦和你相会。”王鼎也很高兴,又要求接着亲热。女子说:“我现在只有一点点阳气,要想复生,实在禁不起这番风雨。以后合好的日子还很长,何必非今晚不可?”于是起身走了。第二天,秋月又来了,跟王鼎对坐着,谈笑风生,欢乐得像旧相识。灭烛上床,就跟活人一样。只是她一起身,王鼎就遗泄淋漓,沾染床褥。

一晚,明月皎洁,王鼎和秋月在院子里散步,问她道:“阴间里也有城市吗?”秋月回答说:“和人世一样。阴间的城府不在这里,距这里有三四里路,但那里以夜间为白天。”王鼎问: “活人能看见吗?”回答说:“也可以。”王鼎请求去看看,秋月答应了。二人乘着月光走去,秋月飘飘忽忽地走着,像风一样快。王鼎极力追赶,片刻便来到一个地方。秋月说:“不远了。”王鼎四处眺望,什么也看不见。秋月便用唾沫涂在他的两眼上,王鼎睁开眼,觉得目力倍增,看夜间不亚于白天。立时便见一座城池矗立在烟雾迷茫之中。路上行人来来往往,像赶集一样。一会儿,见两个皂隶捆着三四个人经过,最后一人非常像王鼎的哥哥。玉鼎走近一看,果然是哥哥王鼐。惊骇地问:“哥哥怎么来了?”哥哥看见他,眼泪流了下来,说:“我也不知是为什么事,被强行拘拿了来。”王鼎愤怒说:“我哥哥是知礼君子,怎么像犯人一样捆着他?”请求两个皂隶释放了哥哥。皂隶不肯,傲慢地爱答不理。王鼎忿怒地要和他们争执,哥哥劝阻他说:“这是官命,应当守法。只是我缺少钱用,他们苦苦索贿,你回去后,要给我筹办些钱来。”王鼎拉着哥哥的胳膊,失声痛哭。皂隶大怒,猛地一拽王鼐脖子上的绳索,王鼐顿时被摔倒在地。王鼎见此情景,怒火中烧,再也忍耐不住,抽出佩刀,一刀把那皂隶的脑袋砍了下来;另一个刚要喊叫,王鼎又一刀杀了他。秋月大惊说:“杀了官使,罪不可赦!迟了就大祸临头了!请你们赶快找船北去,回家后不要摘丧幡,关门杜绝出入,七天后可保无事。”王鼎便搀着哥哥,连夜租船,火速北渡。回家后,见有很多吊唁的客人,才知道哥哥果然死了。关上门,下好锁,才进家,再看看哥哥,已经不见了。走进屋子,死去的哥哥已经苏醒过来,正喊:“饿死我了,快点备汤饼!”当时王鼐已死了两天了,一家人都非常惊骇,王鼎便讲了缘故。七天后打开门,去掉丧幡,人们才知道王鼐又复活了。亲戚朋友都来询问,王鼎就托词回答。

王鼎想念秋月,心烦意乱,便又南下,来到原来的那间阁楼上,点上蜡烛等了很久,秋月也没来。朦朦胧胧地正要睡觉,见一个妇人走进来。说:“秋月小娘子托我转告您:上次杀了公差后,因凶犯逃脱,把小娘子捉了去,现押在狱中,受狱卒虐待。小娘子天天盼着您,请您想个办法。”王鼎悲愤不已,跟着妇人去了。到一个城市,进了西关,妇人指着一个大门说:“小娘子暂押在这里。”王鼎进去,见房屋杂乱,囚禁着很多犯人,里面并没有秋月。又进了一个小门,见一间小屋子里有灯光。王鼎走近窗户往里一看,秋月正坐在床上,用袖子掩着脸哭泣。两个狱卒在一边摸摸她的脸,又摸摸她的小脚,嬉笑着调戏她。秋月哭得更急。一个狱卒搂着她的脖子说:“已经成了犯人,还要守贞洁吗?”王鼎大怒,顾不得说话,持刀冲进去,一人一刀,如斩乱麻,立时将两个狱卒杀了,将秋月救了出来,幸亏没人发觉。才回到旅舍,王鼎蓦然醒了过来。正在奇怪刚才做的梦太凶,忽见秋月含着泪站在一边。王鼎惊讶地起来拉她坐下,告诉她刚才的梦。秋月说:“是真的,不是梦!”王鼎吃惊地说: “这可怎么办!”秋月叹息说:“这也是定数,我本来要等到月底,才能复生。现在已经如此紧急,怎能再等?你赶快挖开坟墓,载着我一同回家,每天连声呼唤我的名字,三天后我就可以活过来。只是时日不满,我会骨软脚弱,不能为您操劳家务罢了。”说完,急匆匆地要走,又返说:“我几乎忘了,阴间里追究起来可怎么办呢!我活着时,父亲传我两道符,说三十年后,夫妇两人可以佩带上。”于是要来笔,飞快地写了两道符,说:“一道你自己佩,另一道贴在我的背上。”王鼎送她出去,记住她消失的地方,往地下挖了一尺多,便看见了棺材,已经朽烂了。一边有块小石碑,碑文果然和秋月说的一样。打开棺材一看,秋月面色如生。王鼎把她抱进屋中,衣裳随风化成了灰烟。贴好符,又用被褥紧紧地包起她来,背到江边,叫过一只船,假说是妹妹得了急病,要送回婆家。正巧刮起南风,天刚明,已到了家门。

王鼎把秋月抱进屋安置好,才告诉兄嫂。一家人都吃惊地来看,也不敢当面说王鼎中了邪。王鼎打开被子,长声呼叫秋月,夜里就拥抱着尸体睡觉。尸体渐渐温暖起来,三天后竟苏醒过来;七天后能走路,换了衣服拜见见嫂,轻盈盈的样子,不亚于神仙。只是十步之外,就要人扶着才能走,不然就随风摇曳,像要倾倒。看见的人以为她身体有这种病,反倒更添几分娇媚。秋月常劝王鼎说:“你罪孽太深,应该积德念经来忏悔。否财,恐怕寿命不长。”王鼎本不信佛,从此虔诚地拜在佛门,后来也没什么事。


\subsection{1.5.28   莲 花 公 主}
\label{\detokenize{p00_u5176_u5b83/_u767d_u8bdd_u804a_u658b_u5fd7_u5f02:id201}}
胶州窦生,名旭,字晓晖。一天他正在午睡,觉得好似有一个穿褐色短衣的人站在床前,惶恐四顾,好像有什么话要说。窦旭问他,他回答说:“我家相公想请您去一趟。”窦生问:“你家相公是什么人?”来人说:“就在附近。”窦旭随他出去。转过墙角,到了一个地方,只见亭台楼阁,重重叠叠,接连不断。两人曲曲折折地向前走着,窦生感到这千门万户,不似人间。又见宫人和女官众多,来来往往,熙熙攘攘,见褐衣人就问:“窦生请来了吗?”回答说:“请来了。”

一会儿,一位贵官出来迎接,见窦生恭恭敬敬。窦生说:“平素没有什么交往,故也未前来拜访。今天承蒙如此厚待,颇为疑惑不解。”贵官笑道:“我们君王久闻先生家族世代清廉,德望很高,非常倾慕,盼望与您会面。”窦生更惊异,又问:“大王是谁?”回答说:“少待一会,你自己就明白了。”

少顷,有两位女官到,手举一双长幅旌旗,导窦生入宫。进了几道门,见远远的大殿上一位大王坐在那里,见到窦生到来,走下台阶迎接窦生。两人按宾主的礼仪相互拜见后,摆宴坐席,酒宴十分丰盛。窦生仰头一看,殿上有一幅匾额,上题“桂府”二字,窦生心中局促不安,致使答对难以措辞。大王说:“能和你府上为邻,可见我们的缘分很深。开怀畅饮,不必猜疑畏惧。”窦生只是唯唯答应。

酒行数巡,只听殿下笙歌齐鸣。听不到锣鼓之声,但闻丝竹嘤嘤,幽细悦耳。乐队稍停,大王对左右说:“我偶然想到一个上联,请诸位把下联对上。这上联是:‘才人登桂府’,”众官正在思考,窦生应声说:“君子爱莲花。”大王一听大喜说:“奇怪啊!莲花是公主的乳名,对得如此贴切,莫不是夙有缘份?传话给公主,不可不出来见见这位才子。”过了一会儿,只听环佩之声叮咚渐近,兰麝之气浓而熏香,公主来到了。看上去十六七岁,绝美无双。大王让公主向窦生行见面礼,说:“这就是我的小女莲花。”公主施过见面礼,就回内殿去了。窦生一见公主,就心神动摇,呆呆坐在那里凝思冥想。大王举杯劝饮,窦生竟像没有看到。大王也似乎觉察到窦生的心意,便说:“我的小女儿和你很般配,但惭愧的是不同类,怎么办呢?”窦生怅然像是痴了一样,大王的这番话,又没听到。坐在他旁边的人,用脚悄悄地踩了窦生一下,说:“适才大王向您作揖你没看见,大王同你说话也没听到吗?”窦生茫茫然,若失魂魄,自觉惭愧,离开座位,说:“臣蒙大王厚礼相待,不觉饮之过量,有失礼仪,幸能宽恕。大王政务繁忙,我也到了该走的时候了。”大王也离开座位说:“这次见到窦君,我心中甚感惬意。为何这样仓促就要走呢?你既然不想住下,我也不敢强留。假若思念这里,我就派人再把你请来。”接着,就令内监引窦生出去。走在路上,内监对窦生说:“刚才大王说可以匹配,看样子想把公主许配给你,你为什么不发一言?”窦生后悔得直跺脚。边走边感到悔恨,不觉已经到家。

窦生忽然清醒过来,窗外夕照的残光,已经渐没。默坐回想起刚才发生的事,历历在目。晚饭后,吹熄了蜡烛,希望在幽冥中,再去寻求梦中境界。然而邯郸之路渺不可寻,只是悔恨叹惋而已。

一天晚上,窦生与朋友同睡在一张床上,忽然见到上次来送他的内监来了,传达了大王的命令,邀请窦生进宫去。窦生很高兴,就跟着去了。窦生见到大王,趋步向前参拜。大王急将他扶起,让他在一旁坐下,说:“自上次分别,知道你很眷恋小女;现在把小女许配于你,想你不会太嫌弃吧!”窦生立即叩头拜谢。大王命学士、大臣们陪同窦生宴饮。酒喝到正快乐时,宫中人前来报告说:“公主妆扮好了。”一会儿,见数十个宫女,拥簇着公主出来。用红色的锦绸盖着头面,迈着轻盈的纤步,被人搀扶到猩红的地毯上,与窦生拜天地成婚。交拜后,侍女们把他们夫妻送到宫廷馆舍。洞房中温和清凉,香气甜蜜。窦生说:“有公主在我跟前,真使人乐而忘死。只怕今天的艳遇是一场梦!”公主捂着嘴笑说:“明明是我与你在一块,哪里是梦啊!”

第二天清晨,窦生就嬉笑着给公主涂脂、敷粉、画眉;完了,又用带子量量公主的腰围,用手指量量公主的脚。公主笑问:“窦君疯颠了吗?”窦生说:“我每每被梦骗怕了,所以我特意地细细看看你,记下来。倘若再是梦,也足以记得清楚。”两人正在说笑间,一个宫女急急跑进来说:“妖怪闯进宫殿,大王躲到偏殿里,灭顶之祸不远了!”窦生大惊,急急去见大王。大王执着窦生的手哭泣着说:“蒙你不嫌弃,正图永久之好。谁料灭顶之祸从天而降,国运危在旦夕,这可怎么办啊!”窦生惊问这话从何说起。大王把桌案上的一份奏章,交给窦生看。奏章中写道:“含香殿大学士黑翼,为有非常之妖灾,祈求大王早日迁都,以保存国家事:据宫门看守者报告,自五月初六日,来了一条千丈长的巨蟒,盘踞在宫外,吞食城内外臣民一万三千八百多口;所经地方,宫殿尽成废墟,等等。臣子得知,奋勇前去探看,确见妖蟒一条,其头大如山岳,两眼如同江海;昂起头,则殿阁齐并吞掉;伸伸腰,则高楼墙垣尽覆。真是千古少见之凶恶,亦为万代少见之灾祸!国家危在旦夕!乞求大王早日携带家眷宫人,速速迁到安全地方。”

窦生看完奏章,面如灰土。立刻有宫人跑来报告:“妖物来了!”众人哀呼,极度悲惨。大王仓惶中,不知怎么办,只是哭泣着对窦生说:“小女拖累先生你了。”窦生一口气跑回到馆舍,见公主正与左右的人抱头大哭,见窦生进来,牵着他的衣襟说:“郎君怎么安置我呀!”见此情景,窦生悲痛欲绝,就握着公主的手腕,思考着说:“我家里很贫穷,惭愧的是没有金屋,只有草房三间,姑且一块躲到那里可以吗?”公主含着泪说:“事情紧急,还能有什么选择呢?只求携同速速离开这里!”窦生于是搀扶着公主出来。

不一会,到了窦生的家里。公主说:“这里是很安全的地方,比我们的国家好多了。然而我跟你来到这里,我父母依靠谁呢?请你再另外筑一间房舍,让全国人都来。”听此话,窦生很是为难。公主嚎啕大哭,说:“不能救人之急,要郎君有什么用?”窦生劝慰了公主一番,就自己走进内室。公主伏在床上悲啼不已,劝也不止。窦生正在焦急无术的时侯,忽然醒来,方知又是一场大梦,但耳畔嘤嘤啼声,一直在响。仔细一听,又非人声,而是两三只蜜蜂在枕边飞鸣。他大声叫道:“怪事,怪事!”

同床的朋友被惊醒了,问他出了什么事。窦生就把刚才梦中的情景一五一十地告诉了朋友。朋友听了,也感到很诧异。两人就共同起来看,蜜蜂飞在衣袖间,依依不去,拂之不走。朋友便劝窦生为之筑巢。

窦生按照朋友的话,督工为蜜蜂造巢。刚刚竖起两面墙板,大群的蜜蜂便从墙外飞来,络绎不绝,如一条黑呼呼的绳子。还没有盖顶,蜜蜂飞来的足有一斗。窦生按蜜蜂飞来的方向,遗踪它们是从哪里来的;发现原来是从邻居老头子的旧菜园子里。菜园子里有个蜂房,已经有三十多年了,繁殖的蜜蜂很多。

有人把窦生造蜂房的事告诉邻居老头。老头到菜园中察看,蜂房中寂静得没有一点声音。打开墙壁一看,有条大蛇盘踞在里面,有一丈多长。老头气愤地把蛇提出来杀死,才知道窦生梦中所言巨蟒,就是这条蛇。这群蜜蜂自从迁到窦生家,生殖繁延得更兴盛,也没有其它异常现象。


\subsection{1.5.29   绿 衣 女}
\label{\detokenize{p00_u5176_u5b83/_u767d_u8bdd_u804a_u658b_u5fd7_u5f02:id202}}
书生于景,字叫小宋,是益都人,在醴泉寺里读书。一夜,于景正在诵读,忽听窗外一个女子称赞说:“于相公读书很勤快啊!”于景心想,这深山中哪来的女子?正在疑惑问,女子已推门进来了,说:“很用功啊!”于景惊讶地站了起来,见这女子穿着绿衣长裙,生得美妙无比。于景知道她不是人类,再三追问她的家住哪里。女子说:“你看我并不是能吃人的,何必寻根究底呢?”于景心中很喜欢她,便和她一块睡了。女子脱去衣服,腰细得不满一把。天快亮时,女子轻盈地走了。从此,没有一天晚上不来。

一晚,两人一块饮酒。女子谈吐间很懂音律,于景便说:“你的声音娇柔细弱,如果能唱一曲,一定让人消魂。”女子笑着说:“不敢唱,怕消了你的魂。”于景执意请她唱,女子说:“我不是吝惜,是怕被别人听到。你一定要听,我只好献丑,但只能小声唱,你明白意思就行了。”接着用脚尖轻轻点着拍子,唱道:“树上乌臼鸟,嫌奴中夜散,不怨绣鞋湿,只恐郎无伴。”声细如蝇,刚刚能辨听清楚;而仔细一听。只觉宛转滑烈,动耳摇心。唱完,女子打开门看看外面,说:“提防窗外有人。”又出去绕屋子转了,一圈,才进屋来。于景说:“你怎么这样疑惧?”女子笑着回答说:“俗话说‘偷生的小鬼常怕人’,这就是说的我啊。”不一会儿睡下后,女子忽又不高兴,说:“平生的缘份,难道到此为止了吗?”于景忙问缘故,女子说:“我的心跳动不安,只怕是祸将临头了。”于景安慰说:“心动眼跳,本是平常的事,何至于说这种话呢?”女子才稍高兴一点,二人重又亲热起来。

天快亮时,女子披衣下床。刚要开门,犹豫了一回又返回来,说:“不知什么缘故,我心里总是怕。请你送我出门。”于景便起床,把她送出门外。女子说:“你站在这里看着我,我跳过墙去,你再回去。”于景说:“好吧。”看着女子转过房廊,一下子便不见了。正想再回去睡觉,只听传来女子急切的呼救声。于景奔跑过去,四下里看并没人影,听声音像在房檐间。他抬头仔细一看,见一弹丸大的蜘蛛,正揉弄着一个东西,发出声嘶力竭的哀叫声。于景挑破蛛网,除去缠在那个东西身上的网丝,原来是只绿蜂,已经奄奄一息了。于景拿着绿蜂回到房中,放到案头上。过了会儿,绿蜂慢慢苏醒过来,开始爬动。它慢慢爬上砚台,用自己的身子沾了一身墨汁,出来趴在桌上,走着划了个“谢”字,便频频舒展双翅,然后穿过窗子飞走了。从此,女子没有再来。


\subsection{1.5.30   黎 氏}
\label{\detokenize{p00_u5176_u5b83/_u767d_u8bdd_u804a_u658b_u5fd7_u5f02:id203}}
龙门县有个叫谢中条的人,为人轻薄,品行不端。三十多岁时妻子死了,留下两儿一女,一天到晚哭叫,谢中条很感劳累苦恼。想再聘娶个女人作妻子,但高不成,低不就,只好暂时雇一个老妈妈抚养子女。

一天,谢中条缓步走在山路上,忽然一个妇人从后面过来。他等妇人走近,偷偷一看,是一位俊俏女子,二十多岁,心申很喜欢她,就嬉笑着说:“娘子一个人行走,不害怕吗?”妇人只管走路也不应声。他又说:“娘子小脚纤弱,走山路很艰难啊。”妇人仍然不理他。谢中条见四周没人,便走近妇人身边,突然抓住她的手腕,把她拉到山谷中,要强行与她合欢。妇人愤怒地喊叫说:“哪里来的强盗,蛮横来侵犯!”谢中条只管拉着妇人走,一点也不放松。妇人步履艰难,跌跌撞撞,无计可施,就说:“你想与我合欢,就这样对我啊?放开我,我答应你。”谢中条答应了,两人一块走到僻静的沟壑中。亲热完了,妇人问他家住哪里,姓什么,谢中条如实告诉她。也问妇人。妇人说:“我姓黎,不幸早寡。婆婆也早早去世,我孤独一身,没有依靠,所以经常到娘家去住。”谢中条说;“我也是死了妻子。鳏居在家,你能跟着我吗?”妇人问:“你有没有子女?”谢中条说:“实不相瞒,如果说枕席之事,与我要好的女子也不少。只是儿啼女哭,叫人受不了。”妇人犹豫了一会说:“这是件难办的事。看你的衣服鞋袜样式也很平常,我自认为能做;但继母难当,恐怕受不了别人的闲话。”谢中条说:“请你不要疑虑。我自己不说你不好,别人为何干涉?”妇人有点同意了,转而又顾虑说:“我们已到了这种地步,我怎能不从呢?只是家中有凶悍的大伯子,时常把我当作得取钱财的奇货,恐怕不允许我们成亲,那又怎么办?”谢中条也忧愁起来,要妇人同他私奔。妇人说:“我也多次想过这个办法,所顾虑的是你的家人一旦泄露,对我们俩都不利。”谢中条说:“这是小事,我家中只有一个老妈妈,立刻就可以打发她走。”妇人很喜欢,就同谢中条一起回家。谢中条先把妇人藏在外面的屋里,接着进家打发老妈妈走了,打扫床铺迎进妇人,两人更加欢好。妇人就自己操持家务,还为儿女们缝缝补补,很是勤劳辛苦。谢中条自从得了妇人,异常宠爱她,每天只是关着门在家中与她闲谈,不再和客人来往。

过了一个多月,谢中条因公事外出,锁上门就走了。回来后,见堂屋的门紧闭着,怎么叫也没人答应。推开门扇进去,屋中没有人影。又来到卧室,一匹大狼突然冲出门来,几乎把他吓死。进去一看,子女都不见了,地上满是鲜血,只有三个人头还在。他返身去追狼,已经不知它的去向了。


\subsection{1.5.31   荷 花 三 娘 子}
\label{\detokenize{p00_u5176_u5b83/_u767d_u8bdd_u804a_u658b_u5fd7_u5f02:id204}}
浙江湖州的宗湘若,是个读书人。一年秋天,他去坡里查看农田时,见庄稼茂密处不住地摇晃,心中怀疑;于是走过田间小路去那里察看,原来有对男女正在地里野合。他笑了笑要往回走,只见那男的羞愧地系上衣带,草草离去。那个女子也赶忙起来,宗生仔细一看,女子长得非常秀丽,心里很喜欢她,想要和她亲热亲热,又实在羞于这种鄙陋的做法。于是走向前替她拂拭衣服上的尘土,说:“你们幽会得可快乐?”那女子只笑不说话。宗生靠近她的身体,解开她的衣服,摸她的皮肤,只觉细嫩滑腻,于是上下几乎摸遍。女子笑着说:“你这个迂腐的秀才!要怎样就怎样好了,这样狂荡地摸来摸去做什么?”宗生追问她的姓氏,女子说:“春风一度,即别东西,何用劳驾你审察?莫非要我留下名字立贞节牌坊?”宗生说:“在荒草野坡中私会,是山村放猪的奴仆干的事,我不习惯。以你的美丽姿质,就是偷偷约会,也应当自重才是,何必如此卑琐呢?”女子听了他的话,表示赞许。宗生又说:“我的书房离这里不远,若不嫌弃,请到那里去呆一会。”女子说:“我出来已经很久了,恐怕别人怀疑,我夜里可以去。”她详细问了察生门前的特征标记,然后匆忙奔向斜路,急急地走了。到了夜里一更天,女子果然来到宗生的书房。两人无限欢爱,极其亲热。这样过了很多日子,他们俩的事也没有人知道。

恰巧有个西域僧人住在本村庙里,见到宗生,惊讶地说:“你身上带有邪气,曾遇到过什么?”宗生说:“没有。”过了几天,宗生不知不觉地忽然得了病。女子每夜都带来好的果子点心给宗生吃。并殷勤慰问他,感情像夫妻一样好。但是,上床以后必定强让宗生与她相交。宗生身患大病,很难承受。心里怀疑这女子可能不是人类;然而也没有办法拒绝,或使她离去。于是说:“以前那个和尚说我被妖怪迷惑我还不信,现在果然病了,他说的话真灵验啊。明天委屈他来一趟,就求他贴符念咒。”女子听说后脸色马上变得很凄惨,宗生更加怀疑她。第二天,宗生派家人把实情向那个西域僧人讲了。僧人说:“这是个狐狸,它的道业还很浅,容易捉拿。”于是写了两道符交给家人,并嘱咐说:“回去找一个洁净的坛子,放在床前,用一道符贴住坛口;当狐狸一窜进去,就赶快在上面盖上一个盆,再把另一道符贴到盆上,然后把坛子放进开水锅用烈火猛煮,不多时它就会死去的。”家人回来按照僧人的吩咐办妥了。

夜深了,女子才来到。她从袖子里摸出一些金桔,刚要到床前探问宗生的病情,忽听到坛子口飕飕一声风响,就把女子吸到坛子里边去了。家人突然跳起来,迅速盖上盆并贴上符。想放进锅内去煮。宗生看到满地的金桔,想到以前两个人的感情那样好,心情悲伤感动,急忙叫人把她放了。于是揭了符拿掉盆,女子从坛内出来,极为狼狈,跪到地上说: “我多少年修行道业将要成功,一时几乎化为灰土!您真是个仁义之人,我誓必报答您。”说完就走了。

过了几天,宗生病情更加沉重,像将要死去的样子。家人急忙去集市为他购买棺材,在路上遇到了一个女子,问他说:“你是宗湘若家的仆人吗?”家人回答说:“是啊。”女子又说:“宗相公是我的表哥,听说他病得很重,本来想要去探望他,恰巧有事去不了。这里有灵药一包,劳驾你送给他。”家人接过药拿回家中。宗生想表亲中根本没有姐妹,知道是狐狸来报答他。吃了这药后,果然病便好了,十余天身体就完全康复。他心里非常感激狐女,便对空祝祷,希望能再见到她。

一天夜里,宗湘若关起门来自己喝酒。忽然听到有用手指轻弹窗子的声音。拔出闩出门一看,竟是狐女。宗生大喜,攥着她的手表示感谢,并请她坐下共饮。狐女说:“分别以来,心中时时不安,想来思去无法报答您的太恩大德。现在为你找了一个好伴侣,聊以塞责吧!”宗生问:“是个什么人啊?”她说:“这不是您所知道的。明天辰刻,您早一点去南湖,见到有采菱角的女子,其中有个穿白绉纱披肩的,就驾船向她急驶过去。如果分辨不清她的去处,就察看堤边,发现一支短杆莲花隐藏在叶子底下,你便采回来,点上蜡烛烧那花蒂,就能得到一位美丽的妻子;同时还能使您长寿。”宗生恭敬地记下了她说的话。不久狐女要告别,宗生再挽留她,狐女说:“自上次遭到灾难,我就顿悟正道,为什么要以枕席之爱换取别人的仇恨呢?”说完,面带厉色告辞而去。

宗生按照狐女说的话到了南湖,看到荷花荡中美丽的女子很多。其中有一个垂发少女,穿着用自绉纱做的披肩,真是个绝代佳人。便迅速划船向她逼进,忽然弄不清她到哪里去了。于是拨开荷花丛去找,果然有一枝杆长不到一尺的红莲花,便折下来拿回家中。宗生进门把红莲花放到桌子上,将蜡烛芯剪了剪,点上火要去烧花;一回头,莲花变成了美女。宗生又惊又喜,急忙伏地而拜。莲女说:“你这个痴书生,我可是个妖狐,将为你带来灾祸!”宗生不听。莲女又说:“这是谁教给你这样做的?”宗生回答: “我自己就能认识你,何用别人教我?”上前抓着她的胳膊往下拉,莲女随手而下,变成了一块怪石,高有一尺多,面面玲珑。宗生就把它安放到供桌上,然后点上香很恭敬地礼拜祝祷。

到了夜里,宗生关严门窗,惟恐怪石跑了。天明一看,又不是石头了,而是一件纱帔,远远就闻到一股香气。展开纱帔的领子和衣襟看去,上面仍然留存着莲女刚穿过的余痕。宗生拿到身边盖上被子抱着它躺在床上。天黑时他起身掌灯,等转过身来垂发女已经在枕上。宗生高兴极了,恐怕她再变了,哀求祷告然后和她亲热起来。莲女笑着说:“真是孽障啊!不知道是什么人多嘴,竟叫这疯狂儿纠缠死!”于是不再拒绝。两人亲热的时候,莲女好像承受不了,屡次求他停止,宗生不听。莲女说:“你不听,我就变化而去!”宗生怕她真的走,就此而罢。从此两人情深意笃,和谐无间。家里大箱小箱内金银绸缎常常满着,也不知从哪里来的。莲女见了人只是恭敬地打个招呼,似乎不善言词。宗生也避讳着不对人说她那奇异的来历。莲女怀孕十个多月后,计算时日应当分娩了,就走进房内,嘱咐宗生把门关紧,禁止别人叩门。自己竟然用刀从肚脐下割开,取出一个男孩,又让宗生撕下块绸缎把刀口包扎好,过了一夜就痊愈了。

又过了六七年,莲女对宗生说:“我们前世造下的这段缘分我已报答,请求与你告别了。”宗生一听眼含热泪说:“你才来我家时,我穷得不能自立,靠着你家里才富起来,你怎么忍心就说远离呢?况且你也没有亲族,将来儿子不知到母亲在哪里,也是一件很遗憾的事!”莲女伤心地说:“有聚必然有散,这本来就是常事。儿子有福相,你也能活百岁,还再求什么呢?我本姓何。倘若蒙你思念,抱着我的旧物呼唤‘荷花三娘子’,就能见到我。”说完挣脱出身子来,说了声“我走了”。宗生惊看时,她已飞得高于头顶;宗生急跳起来去拉她,结果抓住了一只鞋。鞋脱下来落到地上,变成了石燕,颜色比朱砂还红,内外晶莹明彻;像水晶一样。宗生拾起来收藏好。翻检箱子,见莲女初来时所穿的自绉纱披肩还在里边。于是每逢怀念她的时候,就抱着披肩呼唤“荷花三娘子”,披肩立即化成莲女,面带笑容,喜在眉梢,犹如真的一样,只是不说话罢了。


\subsection{1.5.32   骂 鸭}
\label{\detokenize{p00_u5176_u5b83/_u767d_u8bdd_u804a_u658b_u5fd7_u5f02:id205}}
淄川城西白家庄的某人,偷了邻居的一只鸭子煮着吃了。到夜里,觉得全身发痒;天亮后一看,身上长满了一层细细的鸭茸毛,一碰就疼,非常害怕,可又没有办法医治。

夜里,他梦见一个人告诉他说:“你的病是上天对你的惩罚,必须得到失鸭主人的一顿痛骂,这鸭毛才能脱落。”而邻居老翁平素善良,心胸宽阔,丢了东西从来就不去计较或流露不高兴的样子。偷鸭的人很奸滑,便撒谎告诉老翁说:“鸭子是某某人所偷,他非常害怕别人骂,骂他可以警告将来。”老翁笑道:“谁有那么多闲工夫生闲气,去骂这种品行恶劣的人。”终不肯骂。偷鸭的人很难为情,只好把实情告诉了邻居老翁;老翁这才肯骂,那人身上的鸭毛果然退了。


\subsection{1.5.33   柳 氏 子}
\label{\detokenize{p00_u5176_u5b83/_u767d_u8bdd_u804a_u658b_u5fd7_u5f02:id206}}
胶州的柳西川,是法内史的管家,四十多岁,生了一个儿子。柳西川十分溺爱他,什么事都由着儿子的性子,唯恐他不如意。儿子长大后,异常地浪荡奢侈,不几年便把柳西川的积蓄挥霍净光。后来,儿子生了病,柳西川本来养着些好骡子,儿子说:“肥骡子肉好吃,杀匹骡子给我吃了,病就好了!”柳西川便想杀匹老骡子,儿子知道后,愤怒地咒骂起来,病势也更加沉重,柳西川害怕,忙杀了匹肥螺子给他吃,儿子才高兴起来。但只吃一片骡肉,便扔在一边不吃,病情也没有好转,不久就死了。柳西川心疼得直想死去。

过了三四年,柳西川村里的人结香社去朝拜泰山。登到半山腰,见一个人骑着匹骡子迎面走来,模样非常像柳西川死去的儿子。等他走封近处一看,果然不错。那人见了众人,下骡作揖行礼,向每个人都问寒问暖。村人都很惊骇,也不敢提他已经死了的事,只是问他:“在这里干什么?”柳子回答说:“也没什么事,四处跑跑罢了。”便打听众人所住旅店主人的姓名,众人告诉了他。柳子拱拱手说:“我还有件小事,来不及叙谈了,明天去拜访你们。”说完,骑上骡子走了。

村人回到旅店,以为柳子未必真来。第二天一早等着他,他果然来了。把骡子拴在走廊的柱子上,走进屋子说笑起来。众人说:“你父亲天天想念着你,你怎么不回去探望探望他呢?”柳子惊讶地问:“你们说的是谁呀?”众人回答说就是柳西川。柳子一听,神色大变,过了好久,才说:“他既然思念着我,请你们回去后捎句话:我于四月七日,在这里等他!”说完,告辞走了。

村人回去后,把经过告诉柳西川。柳大哭,如期赶到那家旅店,又把缘故告诉了店主人。主人劝阻他说:“那天我见公子的神情很冷酷,似乎未必有好意。依我看来,还是不见为好!”柳西川哭泣着,不相信店主人的话。主人说:“我不是故意阻止你,鬼神无情,是恐怕你遭到不测。如果你一定要见,请你预先藏在柜子里,等他来后,看看他的言语和神色,如可以见你再出来。”柳西川按他说的办了。

一会儿,柳子果然来了,问店主人:“柳某来了吗?”主人回答说:“没有!”柳子气愤地骂道:“老畜牲怎么还不快来!”主人惊讶地说:“你怎么骂父亲?”柳子又骂道:“他是我什么父亲!当初我凭着义气和他结伴经商,没想到他包藏祸心,暗中侵吞了我的血本,还凶悍地赖帐不还!这次我一定杀了他才甘心,他哪里是我什么父亲!”说完,径直出门,边走还边骂:“便宜了他!”柳西川在柜子里听得清清楚楚,冷汗从头一直流到脚跟,大气也不敢出。直到店主人叫他,他才钻出柜子,狼狈地逃回了老家。


\subsection{1.5.34   上 仙}
\label{\detokenize{p00_u5176_u5b83/_u767d_u8bdd_u804a_u658b_u5fd7_u5f02:id207}}
康熙二十二年三月,我和高季文去济南,同住在一家客店,高季文突然得了病。恰巧高振美也跟随高念东先生到了济南,于是商量为高季文治病求药。听袁鳞先生讲:南城外面一个姓梁的人家里有狐仙,擅长医术,像战国名医长桑一样高明。于是共同去梁家求医。

梁氏,是个四十多岁的妇女,很有狐狸的神态。进入她家中,看到内室里面挂着红帘子。从帘子缝隙往里看,墙壁中间悬挂着观世音的画像。还挂着两三张画轴,上面画着跨马持戈的武将,身后跟着很多骑卒;北墙下面有几案,案两头有小座位,高不到一尺,上面铺着小锦褥,说是仙人来到,便坐在这里。

众人烧上香,站成一排拱手肃立。梁氏敲了三下念经的磬,嘴里隐约念念有词。祝祷完后,敬请求医的客人到外面坐下。梁氏站在帘子下面,理了理头发,手托着腮和客人说话,一五一十地叙述仙人的灵验事迹。过了很长时间,天渐渐到了傍晚时分。大家担心天晚了回不去,就请她再祝祷一下,粱氏于是又敲起磬重新祈祷。祈祷完,她转过身站起来说:“上仙最喜欢夜间谈话,其它时间常常遇不上。昨天夜里有些等候考试的秀才,带着菜肴和酒来与上仙聚饮;上仙也拿出好酒酬谢诸位客人,席间赋诗谈笑,散席时,已是黑夜将尽。”

梁氏的话还没讲完,忽听室内有微小的声音不住地在响。好似蝙蝠在飞着鸣叫。大家正在凝神细听的时候,忽然案子上好像落下了一块很大的石头,发出了剧烈的声响。梁氏转过身来说: “差点吓死我!”又听到案子上发出感叹声,像是一个健壮的老人。梁氏用芭蕉扇隔开北墙几案旁的小座位,只听小座位上大声说:“有缘分!有缘分!”接着高声让坐,又好像拱手行礼。随即问客人:“有什么见教?”高振美遵照念东先生的意思问:“见到菩萨了吗?”上仙回答说:“去南海普陀山,是我的老熟路,怎么能见不到呢?”高振美又问:“阎罗王也更换吗?”上仙回答说:“与人间一个样。”又问:“阎罗王姓什么?”回答说:“姓曹。”问完便为高季文求药。上仙说: “你们回去夜里祭祀茶水,我到观音大士那里求药回来奉送,什么病也能治好。”众人也问了各自想知道的事,上仙都详尽地作了分析判断,众人于是告辞返回旅店。过了一夜,高季文的病稍微好了,我和高振美整理行装先回家,就没有时间再去拜访了。


\subsection{1.5.35   侯 静 山}
\label{\detokenize{p00_u5176_u5b83/_u767d_u8bdd_u804a_u658b_u5fd7_u5f02:id208}}
吏部侍郎高念东先生说:“明朝崇祯年间,出了个猴仙,号叫静山。它的神灵托附在河间县的一个老人身上,能和别人谈论诗文,判断吉凶,讲起话来娓娓动听,不感到疲倦。如将肉类、果类食品放到桌子上,猴仙便吃得一片狼藉,只是不能见到他。”那时先生的祖父卧病在床,有人来信说:“侯静山,是个年老有道的人,不能不见见他。”于是高家便派仆人骑马去河间县请那个老人。这老人来到一整天了,而猴仙还没有来到,便烧香祭祀。忽然听到屋上大声赞叹说:“这真是家好人家!”众人惊讶地看去,又听屋檐上还这样说。河间老人站起来说:“大仙到了。”众人便跟着老人整理衣帽出去迎接,又听到拱手致意的声音。随后走进房内,大笑放声言谈。当时高侍郎的兄弟还是秀才,刚刚参加乡试回来。大仙说:“二公考得好,不过《五经》不熟悉,还需要努力,飞黄腾达之时不远了。”高公兄弟听完后很恭敬地询问祖父的病情,大仙说:“生死是件大事,其中的道理很难讲清楚。”于是都知道病人有不祥之兆。不久,先生的祖父就去世了。

起初有个耍猴子的人,到村子里去耍猴子。猴子把锁着它的铁链弄断逃跑了,没有追上,它便跑进了山里。过了几十年,人们仍然能见到。它走起来像飞一样,看到人就躲藏。后来渐渐跑进村里偷吃果饼,可村里的人都看不见。有一天,村里的人发现了它,跟着追到野外,用箭把它射死了。但是猴子的灵魂居然不知道自己死了,只觉得身子像树叶一样轻,瞬间就能走百里路;于是去依附河间老人,说:“你能敬奉我,我就让你发家致富。”于是自号叫静山。

湖南长沙有个猴子,脖子上系着条金链子,曾经往来于士大夫家。见到它的人必定会有喜庆幸运之事。给它果子,它也吃。但不知道它是从哪里来的,也不知它到哪里去。有位九十多岁的老人说:“我小时候好像见到它链子上有个牌子,上面有明代藩王官府的识记。”想来这猴子也成仙了。


\subsection{1.5.36   钱 流}
\label{\detokenize{p00_u5176_u5b83/_u767d_u8bdd_u804a_u658b_u5fd7_u5f02:id209}}
沂水县刘宗玉说:他的仆人杜和,偶然在园子里发现一个地方向外淌钱币,像流水一样,深浅、大小有二三尺多。杜和看到后又惊又喜,两只手满满地抓了两把,又俯身趴在钱流上面。不久起来一看,钱币已经没有了,唯鲁攥在手里的还在。


\subsection{1.5.37   郭 生}
\label{\detokenize{p00_u5176_u5b83/_u767d_u8bdd_u804a_u658b_u5fd7_u5f02:id210}}
郭生,是淄川东山人。从小就喜欢读书,但山村中没有可以求教指正的人,二十多岁了,写的字笔画错讹还很多。原先,家中曾经闹过狐狸。衣服、食品和其它器物,总是丢失,深受其害。

一天夜晚郭生读书,将书放在书桌上,被狐狸涂抹得一塌糊涂;厉害的地方,乱七八糟的连行数都看不清楚了。只好选择那些稍微干净点的来读,只有六七十首。郭生心里非常恼怒愤恨,但又无可奈何。郭生又把平日练习写作的文章收集起二十多篇来,准备让有学问的人指正。第二天早晨起来后,看见文章都翻腾开摊在桌子上,几乎全被浓的淡的墨汁涂抹尽了。郭生恨得要命。正好一位姓王的书生,因事来到山村中。王生平常跟郭生关系很好,顺便登门拜访。看到了被涂污的书,就问郭生是怎么回事。郭生把自己遇到的苦恼事情详细告诉了王生,并且拿出残留的稿子给王生看。王生反复审看,发现没有涂抹留下的文章,好像还有些好的语句。又看那些被涂抹掉的文字,都是冗杂繁琐可以删掉的。王生惊讶地说:“狐狸好像是有意这样做的,不但不能以此为患,还应赶快拜它为师呀。”过了几个月,郭生回过头来看自己原来写的文章,顿时觉得涂改得很正确。于是改写了两篇文章,放在书桌上,以观察它们的变化。等到天亮,又涂改了。过了一年多,狐狸不再涂改了,只用浓墨汁洒大黑点,淋漓满纸。郭生感到很奇怪,拿着去告诉王。王生看了以后说:“狐狸真是你的老师,文章写得很好,可以去参加考试了。”这一年,郭生果然考中了秀才。郭生因此很感激狐狸,总是准备下鸡和米饭,供狐狸吃喝。每次买八股文的选本,不自己选择,而是由狐狸来决断。因此两次府道考试,都名列前茅,考中副榜贡生。

当时叶、缪等先生的文章,风雅艳丽,家喻户晓。郭生有一手抄本,爱惜备至。忽然有一天,被狐狸倒了一碗浓墨汁在上面,沾污湿洇得几乎无一个字留下。郭生便又拟题,构思创作,自己觉得很惬意,谁知又全部被狐狸涂抹了。于是,郭生渐渐不信服狐狸了。没多久,叶公因纠正文体而被收押入狱,郭生又稍稍服气狐狸的先见之明。然而以后郭生每做一篇文章,都煞费苦心,却总被狐狸涂污了。郭生自以为前几次考试都名列前茅,心中盛气很高,就更加怀疑狐狸是妄改了。于是就誊录了以前被狐狸洒了许多墨点的文章试验它,狐狸又全涂抹了。郭生便笑着说:“你真是荒唐,为什么以前说好的,现在又说不好?”于是就不给狐狸设饭菜了,把所读的书本,锁到箱柜之中。早晨起来,看见封得很严实,丝毫未动。但打开一看,只见封皮上涂砸了四道墨汁,比手指还要粗。在第一章上画了五道,第二章上也画了五道,再往后就没有了。从此以后,狐狸竟消声匿迹了。后来郭生考试,有一次考了四等,二次五等,这才知道,其先兆已经寓于狐狸画的道道中了。


\subsection{1.5.38   金 生 色}
\label{\detokenize{p00_u5176_u5b83/_u767d_u8bdd_u804a_u658b_u5fd7_u5f02:id211}}
金生色,是云南晋宁人,娶了本村一个姓木的女子为妻。妻子生了个男孩刚满周岁,金生色忽然得了病。他预感自己必定会死去,就对妻子说:“我死了你一定要改嫁,不要守寡。”妻子听了,好言好语,恳切发誓,表示死守到老。金生色听了摇摇手,对母亲说:“我死后劳累您养育小孙子阿保,不要叫媳妇守寡。”母亲哭着答应了他。

不久,金生色果然死了。木母前来吊唁,哭完后对金母说:“天降灾祸,女婿突然死去。我女儿年龄还小,身体也弱,将来怎么生活啊?”金母悲痛中听木母说这番话,极为气愤,生气地说:“一定要守寡!”木母感到惭愧,也就没再说什么。夜里,木母陪女儿睡觉,私下对女儿说;“人人都可以做丈夫,凭我儿的好长相,还愁找不到个好男人?年纪轻轻不早找个人家,整天瞪着眼守着这个小儿,难道不是个傻子?你婆婆如果一定叫你守寡,决不能给她好脸看。”金母从门前过,正好听到这些话,非常愤恨。

第二天,金母对木母说: “我那死去的儿子有遗嘱,本来不叫媳妇守寡;现在你们既然这样急不可待,那就必须守!”木母听了就愤怒地回家去了。夜里,金母梦见儿子来到,哭泣着劝说母亲不要让媳妇守寡。金母感到很奇怪,就派人去告诉木母,约定等儿子出殡后任凭媳妇嫁人。但是,询问了好几个会看阴阳宅的先生,都说年内不宜举行葬礼。可金生色的媳妇一心想打扮得漂漂亮亮的好出嫁,因此戴着孝还涂脂抹粉。在金家还穿素服,一回到娘家,便打扮得花枝招展,特别鲜艳。金母知道后,感到媳妇行为不好,想到她终究要成为别人的媳妇,也就暗中忍耐。于是媳妇更加放肆。

这个村有个游手好闲、品行不端的人叫董贵,见到金生色的媳妇后很喜爱她,用金钱买通金家邻居的老妇人,求她牵线与金家媳妇私通。夜里,董贵从老妇人家跳墙到金家媳妇的房间和她鬼混。这样往来十余天,丑事传遍全村,唯有金母不知道。媳妇的房里夜间只有一个小丫头陪她,而且还是媳妇的心腹。一天晚上,董贵和金家媳妇正在偷情缠绵,听到金生色的棺材震响,声音如同放爆竹。小丫头在外间床上,看到死了的金生色从幔帐后面走出来,带着宝剑进入卧室。片刻,听到董贵和媳妇的惊叫声。不一会,董贵光着身子跑出来。又过了一会儿,金生色揪着媳妇的头发也走了出来,媳妇大声嚎叫。金母惊慌地起来,看见媳妇光着身子往外走去,正要开门,问她也不答话。金母追出门去看,四周寂静,什么声音也没有,竟不知道媳妇跑到哪里去了。金母回来走进媳妇的卧室,灯还亮着,看见有一双男人的鞋,于是呼叫小丫头。小丫头才战战兢兢地出来,把刚才发生的奇怪事情都说了,金母和她感到又害怕又奇怪。

董贵跳墙逃到邻家,身子抱成一团蹲在墙角。过了一段时间,听人声渐渐没有了,才站起来。董贵一丝不挂,冻得直打寒战,想找老妇人借套衣服。他看到院内有一间屋,双门虚掩,便暂时进到屋里。黑暗中摸摸床上,触到了女子的脚,知道这是老妇人的儿媳妇。他立刻产生奸淫邪念,乘那媳妇睡觉,偷偷上床贴近她。那媳妇醒来,问:“你回来了?”董贵说:“回来了。”那媳妇竟然一点不怀疑,任董贵猥亵。

原来,老妇人的儿子有事到北村去,临走时嘱咐妻子掩着门等他回来。他回来后,听到屋里有动静,便产生怀疑。仔细一听,话音神态极其放荡,不禁大怒,拿着刀冲进房内。董贵害怕,窜到床下面,老妇人的儿子立即上去把他杀死。接着又要杀他的老婆,他老婆哭着告诉丈夫错认了人,才把她放了。可不知道床下究竟是谁,便招呼母亲起来,一道点着灯去看,见那人被砍得仅能辨清面目,还有气息,问他从哪里来的,还能回答。但他身上有好几处刀伤,血流不止,不一会儿就死了。老妇人慌张得不知怎么办才好,对儿子说:“捉奸捉双,你单单杀了他,可怎么办?”儿子不得已,又把老婆杀了。

这天夜里,木翁正在睡觉,听到门外有劈劈啪啪的声音,出来一看,是屋檐起了火,而放火的人还在犹疑不定,似乎不知往哪里去好,木翁大声呼叫,家里人很快都来了。幸亏火刚点着不久,还容易扑灭。木翁命人拿弓箭,去搜寻放火的人。只见一个人身体矫健得像猴子一样,竟然跳墙而去。墙外就是木家桃园,园子四面环有坚固的高墙。几个家人登着梯子往里察看,没发现人影,只见墙下有个东西在微微活动。问话也不回答,用箭射去,那东西便瘫软了。开开门近前查看,发现一个女子光着身子躺在那里。箭穿在头上、胸部。他们拿着蜡烛仔细一照,原来是木家的女儿、金家的媳妇。众人非常害怕地报告了主人。木翁、木母也胆战心惊,不知道是什么原因。木女闭着眼睛,面如死灰,呼吸微弱。木翁叫人拔她头上的箭,拔不出来,后来用脚踩着她的头这才拔出来。木女呻吟一声,血喷出来,就没气了。木翁非常害怕,不知怎么办才好。天亮以后,木翁把实情告诉了金母,直挺挺地跪在地上求饶。金母也没怎么怨恨。只是把前面的事告诉了木翁,叫他自己家里埋了就是。

金生色有个叔伯兄弟叫金生光,愤怒地来到木家,痛斥木女所为。木翁惭愧沮丧,给了他一些钱让他回去了。但是,终究不知道和金家媳妇私通的人叫什么名字。不久,邻居老妇人的儿子以捉奸杀人投案自首。官府只是稍微责罚了他一下,便把他赶出来释放了完事。但是,他妻子的哥哥马彪平常好打官司,便写状子上告妹妹死得冤。官府传拘邻居老妇人,老妇人害怕,把事情的始末全供了出来。官府又传唤金母,金母推脱有病,派金生光代替去对质,金生光把底细都说了。于是前案并发,把木家老夫妇都牵连进去,一切情况都很容易地审查清楚了。木母因为教唆女儿嫁人,判纵淫罪。遭棍打,并命她拿钱自赎,因而家产荡然一空;邻居老妇人牵线导淫,乱棍打死。案子这才完结。


\subsection{1.5.39   彭 海 秋}
\label{\detokenize{p00_u5176_u5b83/_u767d_u8bdd_u804a_u658b_u5fd7_u5f02:id212}}
莱州有一个秀才,叫彭好古,在一座别墅里读书,离家很远。中秋节也没回家,一个人冷冷清清。想到村里的人没有能说话的,只有一个姓丘的书生,是本县的名士,但他平素又有些见不得人的恶行,彭好古非常鄙视他,不愿和他交往。圆月升上天空,彭好古更加感到无聊,迫不得已,只得写了封请柬让仆人去请丘生。

过了不久,丘生来了,二人赏月饮酒。忽听有叩门声,童仆答应着出去开了大门,见是一个陌生的书生,要拜见主人。彭好古离席将客人请进来,互相一揖,然后围着桌子坐下。彭好古便询问起客人的家乡住处。客人说:“我是广陵人,与你同姓,字海秋。值此佳节良宵,我一个人闷在旅店里太冷清,听说您高雅健谈,所以不请自来了!”看他虽是布衣,却很整洁,谈笑风雅。彭好古大喜,说:“是我的同宗人!今晚什么日子,遇上这样的佳客!”请客人喝酒,二人融洽得就和老朋友一样。看彭海秋的意思,似乎十分鄙视丘生;丘生每次巴结地和他攀谈,他都傲慢地不大答理。彭好古替丘生感到羞惭,便打断他的话头,说自己要先唱支民谣劝酒,接着唱起了李白的《扶风豪士之曲》。唱完,主客一同大笑起来。彭海秋说:“我不懂音律,不能回报你的阳春白雪之曲,找一个代替的可以吗?”彭好古说:“悉听尊便。”彭海秋问: “莱州城有名妓没有?”彭好古回答说:“没有。”彭海秋听说,默默地坐了很久,忽然跟童仆说:“我刚才叫来了一个人,现在门外,你去领她进来!”童仆出门,果然发现一个女子正在门外徘徊,便把她领进屋来。见那女子有十五六岁年纪,穿着柳黄色帔风,散发出阵阵香气,美丽得跟天仙一样。彭好古非常惊骇,拉她坐下。彭海秋慰问她说:“麻烦你千里跋涉而来!”女子含着笑连连答应。彭好古心中惊疑,询问她是从哪来的。彭海秋说:“贵地苦于没有佳人,我刚才从西湖里的船上叫了她来。”又对女子说:“你刚才在船上唱的《薄幸郎曲》就很好,请你再唱一遍。”女子便唱道:“薄幸郎,牵马洗春沼。人声远,马声杳,江天高,山月小。掉头去不归,庭中生白晓。不怨别离多,但愁欢会少。眠何处?勿作随风絮。便是不封侯,莫向临邛去!”彭海秋从袜子中掏出一支玉笛,伴着女子的歌声悠扬动听地吹起来,歌唱完了,笛声也停止了。彭好古惊叹不已。说:“西湖到这里,何止一千里路,片刻之间能叫她到这里,莫非是神仙吗?”彭海秋说:“不敢称仙。只是在我眼里,万里远的路不过就像这庭院一般。今晚西湖的风月比平时更美,不可不去游览游览。能跟我一起去吗?”彭好古有心要看看他的奇异本领,便答应道:“太有幸了!”彭海秋又问:“愿意乘船还是骑马?”彭好古想:还是坐船安逸,回答说:“愿乘船。”彭海秋说:“在这里叫船太远,天河中应该有摆渡的!”便高高地扬起一只手,向空中招呼道:“船来,船来!我们要去西湖,不吝惜船钱!”不一会儿,只见一只彩船,从空中飘飘落下,船四周缠绕着团团烟云,几个人一起登上去。见船上一人手持短桨,桨尾密密地排列着长长的鸟翎,形状像羽毛扇,摇起桨,只觉清风习习。彩船渐渐上升,直入云霄,然后又往南飞去,快得跟离弦的箭一样。

过了一刻工夫,觉得彩船落到水中。只听船外笙歌管弦,一片嘈杂。出船舱一望,见明亮的月光荡漾在烟波缭绕的水面上,数不清的游船正游来荡去。船家停下船桨,任彩船自由行驶。彭好古仔细一看,果然是西湖。这时彭海秋去船舱后取出些美酒佳肴,大家欢快地对喝起来。不一会儿,有只楼船渐渐驶近,依傍着彩船并行。彭好古隔着楼船的窗子往里一看,里面有两三个人正在笑闹着下围棋。彭海秋举起一杯酒对女子说:“用这杯酒为你送行。”女子喝酒的时候,彭好古对她恋恋不舍,惟恐她立即走了,暗暗地踢了踢她的脚。女子秋波一转,脉脉送情。彭好古更加动心,请求约定再见之期。女子说:“如你喜爱我,只要打听娟娘的名字,没有不知道的!”彭海秋把彭好古的绫巾送给女子,说:“我为你们代订三年后相会之约。”随即起身,把女子托在手掌里,说道:“仙人啊仙人!”伸手扳住邻船的窗户,把女子从窗格里塞了进去。窗格有只盘那样大小,女子伏身像蛇一样钻了进去,一点也不觉狭窄。一会儿便听邻船有人道:“娟娘醒过来了!”楼船渐渐荡了开去。

彭好古远远地见那只楼船已经停泊,船上的人纷纷下船走了,游兴顿时没有了。便和彭海秋说想上岸游览游览,刚商量着,船已靠了岸。于是大家弃船登岸。彭好古独自一人在前,漫步走了约一里多路,彭海秋从后面赶上来,手里牵着匹马,让彭好古骑上;自己又转回去,说:“等我再借两匹马来。”过了很久,也没回来。这时,路上的行人越来越少了。仰头一看,斜月西转,天色将明。丘生也不知去了哪里。彭好古牵着马,徘徊路边,不知如何办好。又骑马回到原来停船的地方,只见人和船都不见了。想到腰包里没带钱,彭好古更加忧愁惊慌。天大明后,他见马背上有个小口袋,伸手一掏,摸出三四两白银。用它买了点吃的,等着彭海秋和丘生回来,不知不觉已到了中午。他想,不如先去拜访娟娘,也可以慢慢访查丘生的消息,可等他打听娟娘的名字,并没有一个知道的。彭好古兴致索然,第二天只得骑马往回赶来。幸亏马比较温顺,性不烈,半个月才回了家。

起初,彭好古三人乘船上天时,他的童仆忙跑回家说:“主人已成仙升天了!”全家人都悲哀地哭起来,以为他不会回来了。彭好古返回家,把马拴好走进院子里,家里人见了,都惊喜地询问他。彭好古详细讲了奇异的经历。又想到自已一人返回老家来,恐怕丘家听说后会来追问丘生的下落,便告诫家里人不要声张。后来,彭好古说起马的由来,大家因为马是仙人送的,都好奇地到马厩里观看。到了马厩,马已没有踪影,只有丘生被用缰绳拴在马槽上!众人极为惊骇,叫出彭好古来看。见丘生垂着头站在那里,面如死灰,问他也不答话,只是两只眼一张一闭而已。彭好古很不忍心,把他解开扶到床上。丘生就像丧失了魂魄,彭好古给他灌些汤水,他稍稍能咽下去。到半夜,丘生多少清醒过来,急急忙忙地跑到厕所里,屙下来几个马粪蛋,又吃了点东西,才能开口说话。彭好古在床头细问究竟。丘生说:“我们下船后,彭海秋引着我边走边谈。等走到一处没人的地方,他开玩笑般地拍了拍我的脖颈,我只觉迷迷糊糊的,一下子跌倒在地。趴在地上稍定了定神,再看看自己,已经变成马,心里也明白,但是不能说话。这真是奇耻大辱,实在不能让妻子儿女知道,请求你不要泄露这事!”彭好古答应了,命仆人用马驮着送他回了家。

彭好古自回家后,一直思念着娟娘。又过了三年,他的姐夫在扬州做官,他便去探望。扬州有个姓梁的公子,跟彭家素有来往,设宴邀请彭好古。酒席上有几个歌女,都过来拜见梁公子。梁公子问娟娘怎么没来,回答说病了。公子发怒地说:“这奴婢自以为声价高,用条绳子去把她捆来!”彭好古听到娟娘的名字,惊疑地问是谁。公子回答说:“是个妓女,才貌广陵数第一。因为有点名气,所以敢傲慢无礼。”彭好古怀疑是偶然重名,但触动了心事,又着急地想见见她。

一会儿,娟娘来了,梁公子盛气凌人地斥责了她一顿。彭好古仔细打量了一下,果然是中秋节见过的那个娟娘。便对梁公子说:“她跟我有旧交,请你宽恕她。”娟娘看了看彭好古,显出惊愕的样子。梁公子来不及深问彭好古,便命娟娘斟酒。彭好古问她:“《薄幸郎曲》还记得吗?”娟娘更加惊骇,凝神注视了他一会,才开始唱起那支旧曲。听她的声音,跟当年中秋节时唱的一模一样。酒宴结束,梁公子命娟娘陪客人入寝。彭好古握着她的手说:“三年之约,今天才实现了吗?”娟娘说:“那天我跟人游西湖,喝了几杯酒,忽然像醉了一样,朦朦胧胧地只觉被一个人带到一个村子里。一个小童领我进入一家,席上有三个客人,你是其中的一个。后来乘船来到西湖,又把我从窗口送了回去,你拉着我的手恋恋不舍。我以为那都是幻梦,但绫巾真在,我至今还珍藏眷。”彭好古也讲了那件事的经过,两个人相互惊叹感慨了一会儿。娟娘扑到彭好古的怀里,哽咽着说:“仙人已给我们作了媒人,您不要以为我是个风尘女子,可以舍弃,就不再想念我这个苦海中的人!”彭好古说:“船中订下的约会,我一天也没忘。倘若你有意,我就是倾囊出资,再卖了这匹马,只要能把你赎出来,我也在所不惜!”

第二天一早,彭好古把这意思告诉了梁公子,又从姐夫那里借了一千两银子,削去了娟娘的乐籍,把她带回了老家。娟娘有次偶然到那座别墅去,还记得当年喝酒的地方。


\subsection{1.5.40   堪 舆}
\label{\detokenize{p00_u5176_u5b83/_u767d_u8bdd_u804a_u658b_u5fd7_u5f02:id213}}
沂州宋君楚侍郎家,一向崇尚看风水,连家中妇女们都能读看风水的堪舆书,通晓其中的道理。宋侍郎死后,他的两个儿子各立门户,为父亲选择营葬的风水宝地。凡听说有善相地脉、看风水的人,兄弟俩都不远千里,争着请了来,于是两家分别招罗了上百名风水先生。这些人天天骑着马去郊野看坟地,分成东西两路出入,就像是两支军旅。过了一个多月,两家分别寻到了自己中意的风水宝地。这个说把父亲埋在这里子孙会封侯;那个说把父亲埋在那里后代会拜相,两兄弟各说各理,互不相让,便都赌气不再商量,各自去营建坟墓,又是搭锦棚,又是插彩旗,两处都齐备了。

到了发丧那天,灵柩抬到岔路口,兄弟俩分别率领着自己门下的风水先生又争执起来。从早晨一直争到太阳西斜,还是决定不下。客人不耐烦,纷纷都走了。抬灵柩的役夫们换了十次肩,最后疲惫地撑不住了,干脆把灵柩扔在路边,也走了。两兄弟索性不葬了,就在停灵枢的路边,纠集工匠,搭起了茅棚,以遮蔽风雨。哥哥在一边建了房子,留下人看守;弟弟也学着哥哥的样,建了房子派了人。哥哥再建房子,弟弟也再建,这样三年后,这个地方竟成了村落。

又过了许多年,两兄弟相继去世了。嫂子与弟妹一块商量,打破了丈夫们水火不相容的议论,妯娌二人一齐乘车去野外,看那两座坟地,说都不好。于是二人重新准备聘礼,请风水先生另择宝地。每找到一个地方,必要先生画成地图量给她们看,以鉴别优劣。先生们每天都呈进好几份地图,妯娌两个全都指出了毛病。过了十几天,才找到一个地方。嫂子看了地图,喜欢地说:“可以了!”拿给弟妹看。弟妹看了后说:“埋在这地方,日后我们家当先出一个武举人。”葬后三年,宋侍御的长孙果然考中了武举人。


\subsection{1.5.41   窦 氏}
\label{\detokenize{p00_u5176_u5b83/_u767d_u8bdd_u804a_u658b_u5fd7_u5f02:id214}}
南三复,是晋阳地方的官宦之家,有一座别墅,离家十几里路,他每天骑马去别墅一趟。一次,路上遇雨,正好走在一个小村里,见一农人家,门里很宽敞,就进去避雨。临近村的人因南家是大户人家,所以都惧怕他们。

过了一会儿,主人出来邀请南三复进屋休息,样子十分谨慎恭敬。南三复走进一间斗大的小屋,坐下后,主人才拿扫帚殷勤地扫地,接着浸了蜜水当茶,招待南三复。南三复叫主人坐下,主人才敢坐。南三复问主人姓名,主人说:“姓窦,名廷章。”一会儿又献上酒,烹来鲜雏,伺候非常周到。

窦翁有一女儿,刚到束发年龄,来给南三复烫酒,时时等在门外,稍稍露出半侧身子来,年纪约十五六岁,美丽无比。南三复一见动了心。雨停后,他回到家里,日夜想念这个妙龄女子。

过了一天,南三复带了布匹粮食,又去小村窦家,想寻找增进关系的台阶。此后,他常常经过窦家,有时带了酒肴来在窦翁家留连。女子也渐渐与他熟悉了,不大避讳他,常常在南三复面前来往。南三复看她一眼,她就低下头微微一笑。南三复越来越神魂颠倒,不超过三天必到窦家一趟。

一日,南三复来,正好窦翁不在家,坐了很长时间,女子只好出来招待客人。南三复见别无他人,就拉住女子的胳膊想亲近她。女子非常羞惭,严肃地抗拒说:“我家虽穷,要嫁,也不能仗势欺人!”这时,正好南三复死了妻子,便对女子作揖说:“我若能得到你的爱怜一定不再娶别人。”女子叫他对天发誓,南三复就指天发誓表示永不相负,女子便应允了与他欢好。此后,每得知窦翁不在家,南三复就来与女子私会。女子催促他说:“我们这样往来,终日在帐篷底下过日子,总不是常法。若是找媒人来提亲,父母必然以为荣耀,一定不会不同意。你应该快一点办。”南三复嘴上答应着,可心里暗想,农人的女儿哪能当自己的配偶?暂且含糊其词拖延一下再说。

这时,一个媒人来给南三复提亲,说的是一家大户人家的女儿。开始南三复还有点犹豫,后来听说女子很漂亮,家中又富,就决心同意了这门亲事。这时窦女已经怀孕,她更焦急地催南三复与她早日结婚,南三复就再也不去窦家了。

过了不久,窦女生了个男孩。父亲大怒,责打女儿,女儿如实告诉了父亲,并说:“南三复一定会娶我。”窦翁放了女儿,叫人去问南三复,可南三复却矢口否认。窦翁便把小孩抛弃了,打女儿打得更厉害。女儿偷着哀求邻家妇女去告诉南三复自己的苦楚,可南三复仍是不理。

一天夜里,窦女偷着跑出门,看了看被她父亲抛掉的儿子还活着,便抱了去找南三复。到了南家,对看门的说:“我要见你家主人,听他说一句话,我就死不了了。他不念我俩的感情,还不念他的儿子吗?”看门的禀告南三复,南三复吩咐一定不叫她进门。窦女倚着南家的大门嚎啕大哭,一直到五更天才听不见哭声了。天明一看,她已抱着孩子僵死了。

窦翁气愤得不得了,立即上告了官府,官府知道南三复不仁不义,准备治他的罪。南害怕,拿一千两银子贿赂官府,得以免于治罪。

南三复新提亲的那个大户人家,忽然夜里梦见一个女子披头散发抱着孩子来告诉他:“一定不能把女儿许给南三复那个负心人,若是许给他,我就杀了她!”可是这家人家贪图南家富贵,还是同意把女儿嫁给南三复。到了娶亲的那天,大户人家陪送的嫁妆很丰盛,新娘子也很漂亮。但新人整日愁容满面,不见有笑容,睡在床上也泪湿枕席,问她,也不肯说。

又过了几天,大户人家来南家看女儿,一进南家大门就哭,南三复还没来得及问为什么,他们就进了女儿的屋子,看见女儿惊慌地说:“刚才在你家后花园,见我女儿吊死在一棵桃树上,现在这房子里的是谁?”女子听说立即变了脸色,一下扑到地上死了,大家仔细一看,竟是窦女。又到后花园看,新娘子果真已吊死在桃树上。

南家一家人都吓得不得了,赶快去告诉了窦翁。窦翁命人挖坟开棺一看,女儿的尸体已经没有了。窦翁以前的愤恨还未消,又添了新愤,悲愤已极,又去官府告状。官府因情节奇幻,没有马上断决。南三复又去贿赂官府,官府得到许多好处,此案又不了了之。

南三复经过这事后,家境逐渐衰败,名声也不好听;又加上家里的怪事不断传播,几年内没有人敢把女儿嫁给他。南三复不得已,就从百里外找了曹进士的女儿为妻。还没有来得及成亲,谣传朝廷要选美进宫,因此有女儿的人都纷纷把女儿送到女婿家去。

一天,一个老太婆领着一个女子,坐一辆马车,来到南三复家,说是曹进士送女儿来的。她扶着女子进了屋子,对南三复说:“选美女的事很急,仓促间不能举行婚礼,暂送小娘子来。”南三复问:“为什么没有别人来送?”老太婆说:“多少有些嫁妆,随在后面,马上就到。”她说罢匆匆就走了。南三复见这女子也还风流标致,便走过去和她调笑;女子低着头,手里玩弄着带子,神情很像窦女。南三复心里就有点厌恶,但没有说出来。到了晚上,女子上了床,用被子蒙住头就躺下,南三复认为这也是新人的常态,也没有在意。天已经黑了,曹进士家的人还没有来到,南三复就开始怀疑。他到床上掀开被子想问一下女子,一看女子已经僵死了。南吓得不知怎么是好,又不明白是怎么回事,就派人快去曹进士家问。可曹进士家却说没有送女儿这回事,这件奇事又传开了。这时,有个姚孝廉的女儿死去才埋葬了一天,夜里便被贼把尸体盗走了。姚家听到这事后,就到南三复家去验证,一看,果然是他女儿的尸体,掀开被,还赤条条光着身子。姚孝廉很气愤,就去告南三复。官府因为南三复品行不端屡次被告,也非常讨厌他,就按挖坟盗尸罪,判了他死刑。


\subsection{1.5.42   梁 彦}
\label{\detokenize{p00_u5176_u5b83/_u767d_u8bdd_u804a_u658b_u5fd7_u5f02:id215}}
徐州有个叫梁彦的人,患了一种鼻塞打喷嚏的病,很长时间也没治好。有一天,他正在睡觉,感到鼻子特别发痒,急忙起来打了一个大喷嚏。有个东西突然喷出来落到地上,形状像屋脊上的瓦狗。有指头顶那么大。又打了一次,又喷出一个。打了四次,喷出了四个。这四个小东西蠢蠢爬动,聚集到一起互相嗅闻。片刻之间,只见一个强健的吃了其中一个体弱的,吃下后身子顿时见长。一会的工夫,互相吞吃的结果,只剩下一个。身子比鼠还大。它伸出舌头转动着,去舔自己的嘴唇。

梁彦非常吃惊,用脚去踩,而它却沿着梁彦的袜子向上爬,逐渐爬到他的大腿上。梁彦抓着衣服用力抖动,可这东西粘在上面下不来。一会儿它钻入衣襟下,爬到粱彦腰侧时,就用爪子抓搔。梁彦非常害怕,赶忙解开衣服脱下扔到地上。一摸,那个东西已贴伏到腰上,用手推,推不动;用指甲掐,却很痛,竟然成了附在皮肤上的肉瘤。它的嘴和眼已经闭上,好像一只趴着的老鼠。


\subsection{1.5.43   龙 肉}
\label{\detokenize{p00_u5176_u5b83/_u767d_u8bdd_u804a_u658b_u5fd7_u5f02:id216}}
太史姜玉璇说:“天山南麓的沙漠中,有个叫白龙堆的地方,从地上挖下几尺以后,看到里面盛着满满的龙肉。人们可以任意去割,只是不能说出‘龙’字来。若有人说‘这是龙肉’,就会有霹雳震响,把人击死。”姜太史就曾经吃过这种肉,的确不是荒谬之谈。


\section{1.6   卷 六}
\label{\detokenize{p00_u5176_u5b83/_u767d_u8bdd_u804a_u658b_u5fd7_u5f02:id217}}

\subsection{1.6.1   潞 令}
\label{\detokenize{p00_u5176_u5b83/_u767d_u8bdd_u804a_u658b_u5fd7_u5f02:id218}}
宋国英,是东平县人,以教习资格被任命为潞城县令。他上任后,非常贪婪暴虐,尤其是催逼赋税,最为残酷。被他用棍子打死的老百姓,常常横七竖八地躺满了县衙大堂。我的同乡徐白山一次路过潞城县,见他如此横暴,便讽刺他说:“你作为百姓的父母官,威风气焰竟到了如此程度吗?”宋国英扬扬得意地说:“不敢,不敢!我官虽小,但到任一百天,已打死五十八人了。”

过了半年,宋县令正在伏案处理公务,忽然瞪着眼站了起来,手脚一顿乱挠,像是与人撑拒的样子,嘴里连连说着:“我有罪该死,我有罪该死!”衙役把他扶进后堂,一会儿便一命呜呼了。唉!幸亏还有阴曹地府在管理着人世间的事情,不然,像宋国英这样的“父母官”,杀人越货愈多,“政绩卓异”的名声也就传开了,流毒还有穷尽吗?


\subsection{1.6.2   马 介 甫}
\label{\detokenize{p00_u5176_u5b83/_u767d_u8bdd_u804a_u658b_u5fd7_u5f02:id219}}
大名有个秀才,叫杨万石,生平最怕老婆。妻子姓尹,性情出奇地凶悍。丈夫稍微违背了她,她就用鞭子毒打。杨万石的父亲已经六十多岁了,是一个鳏夫,尹氏拿他当奴仆看待。杨万石和弟弟杨万钟常常偷点饭给父亲吃,不敢让尹氏知道。但因为父亲常年穿着破衣烂衫,衣不蔽体,恐怕让人笑话,所以,兄弟二人从不让父亲见客人。杨万石四十多岁了,还没有儿子,娶了个姓王的妾,两人从早到晚都不敢说一句话。

一次,杨氏兄弟二人到郡城等侯乡试。遇见一个少年,容貌俊雅潇洒,二人便跟他交谈起来,谈得很投机。问他的姓名,少年说:“姓马,名叫介甫。”从此后,三人交往更加密切,不久,便结义成了兄弟。分别后,大约过了半年,马介甫忽然带着童仆前来拜访杨万石兄弟。正巧遇上杨万石的父亲坐在大门外,一边晒太阳一边捉虱子。马介甫以为他是杨家的仆人,便说了自己的姓名,让他去通报主人,杨父便披上破棉衣进去了。有人告诉马介甫:“这老头就是杨万石的父亲。”马介甫正在惊讶,杨万石兄弟二人穿戴得整整齐齐迎出门来。进屋行过礼后,马介甫便请求拜见义父。杨万石推辞说父亲偶然得了点病,不能见客,连连让马介甫坐下。

三人谈笑着,不知不觉天已黑了。杨万石说了多次已准备好了酒饭,却一直不见端上来。兄弟二人轮番出出进进好几次,才见有个瘦弱的仆人捧了把酒壶进来。一会儿酒便喝完了。又坐等了很久,杨万石频频地出去催促,急得满头大汗。又过了很久,才见那个瘦弱仆人送来饭。但饭做得实在不好吃,让人难以下咽。吃完饭,杨万石急匆匆地走了。杨万钟抱来床被子,陪客人住宿。马介甫责备他说:“过去我以为你们兄弟二人有很高的品德,才和你们结拜兄弟。现在老父亲实际上吃不饱穿不暖,让路人见了都替你们羞愧!”杨万钟流下泪来,说:“这其中的心事,实在难以出口。家门不幸,娶进了一个凶悍的嫂子,全家男女老少横遭摧残。如不是至亲好友,也不敢宣扬这件家丑。”马介甫惊叹了一会儿,说:“我本来打算明天一早就走。现在既然听你说了这桩奇异的事,倒不能不亲眼看一看。请你们借我一间空房子,我自己起伙做饭。”杨万钟听从了,打扫了一间屋子,让他住下。夜深后,又从家里偷来些蔬菜粮食,惟恐尹氏知道。马介甫明白他的意思,极力推辞不要。还把杨父请来,一起吃住。自己又进城去街市上买了布匹,替杨父做了新衣换上,父子三人都感动得哭泣起来。

杨万钟有个儿子叫喜儿,才七岁,夜里跟着爷爷和马介甫睡。马介甫抚弄着他说:“这孩子将来的福气寿数,要超过他父亲;只是少年时要受点苦难。”尹氏听说杨老汉竟然安安稳稳地有饭吃了,大怒,动不动就高声叫骂,说马介甫强行干涉她的家务事。起初还在自己屋里骂,渐渐地就在马介甫的屋子附近骂起来,故意让马听到。杨氏兄弟二人急得汗流浃背,犹豫着不敢去制止。但马介甫对骂声却充耳不闻。

杨万石的妾王氏,怀孕五个月了,尹氏才知道。她大发淫威,将王氏的衣服剥掉一顿毒打。打完,又喊杨万石来,让他跪在地上,扎上一条女人头巾,然后拿起鞭子往家门外赶。当时,正好马介甫站在外面,扬万石羞惭地不敢出去。尹氏用鞭子抽打着,逼他出去。杨万石忍受不了,只得跑出屋子,尹氏也随后追出来,双手叉腰,跳着脚大骂不止,围观的人挤满了大街。马介甫用手指着尹氏,大声喝斥说:“回去!回去!”尹氏不由自主地返身便跑,像被鬼撵着一样,鞋子都跑丢了,裹脚布弯弯曲曲地拖在路上,赤着脚跑回了家,面如死灰。稍定了定神,奴婢拿来鞋袜让她换上,尹氏才号啕大哭起来,家里的人谁也不敢劝她。

马介甫拉过杨万石,要替他摘下头巾。杨万石站在那里一动不动,大气不敢出,像是怕头巾掉下来。马介甫硬给他摘下来后,他还坐立不安,唯恐私摘头巾,要罪加一等。一直等到尹氏哭完了,杨万石才敢回家,提心吊胆地慢慢蹭了回去。尹氏见了他,默默地一句话没说,突然站起身,回房中睡觉去了。杨万石才放下心来,与弟弟都暗暗感到奇怪。家人也都感到惊异,凑在一起叽叽咕咕。尹氏听到一些,更加羞惭恼怒,将奴婢逐个打了一遍,又喊叫王氏。王氏上次被打伤了,一直卧床不起,尹氏说她伪装,跑到王氏的床前将她一顿暴打,直打得下身鲜血涌出流了产。杨万石在没人的地方,对着马介甫悲伤地痛哭。马介甫劝慰了一番,叫童仆备下酒菜,二人对饮,已经二更天了,仍然不放杨万石回去。

尹氏一人在卧室里,痛恨丈夫不回来,正在大发脾气,忽然听到一阵撬门声。她急忙呼叫奴婢,屋门已经大开,有个巨人走了进来,身影遮挡了整个屋子,面貌狰狞凶恶,像鬼一样。转眼间又进来几个人,手里都持着明晃晃的刀。尹氏吓得差点死过去,刚想号叫,巨人用刀尖一下顶住她的脖颈,说:“敢叫,立即杀了你!”尹氏急忙拿出金银绸缎,要买条命。巨人说:“我是阴司的使者,不要钱,特来取你这个悍妇的心!”尹氏更加恐惧,跪在地上连连磕头,直磕得头破血流。巨人毫不理会,一边用刀一下下划着她的胸膛,一边数落她的罪状说:“像某件事,你说该杀不该杀?”说一件,就划一刀;把尹氏的凶悍罪状列举完,刀子已在她的胸口处划了几十下。最后,巨人说:“王氏生了孩子,也是你的后代,你怎么竟残忍到把她打堕了胎?这件事绝对不能饶恕!”命那几个人将她的手反绑起来,要给她开膛破肚,挖出心看看。尹氏吓得叩头求饶,连连说已经知罪了,巨人才饶了她。一会儿听到大门开关的声音,巨人说:“杨万石回来了。你既然已经悔过,姑且先留下你这条命吧!”说完,都消失不见了。杨万石进屋来,见尹氏赤身裸体地被反绑着,心窝上的刀痕纵横交错,多得数不过来。便解开她询问缘故,得知事情经过,非常惊骇,暗地里怀疑是马介甫干的。

第二天,杨万石向马介甫讲述了昨晚的怪事,马介甫也流露出惊骇的样子。自那以后,尹氏的威风逐渐收敛了,连续几个月没再骂人。马介甫非常高兴,这才告诉杨万石说:“我实话告诉你,你不要泄露出去:前次是我用了点小小的法术,吓唬她一下。现在她既然已经改正,你们又和好了,我也就暂时告辞了!”他便收拾行装走了。

从此后,尹氏每天傍晚都主动挽留丈夫作伴,满脸堆笑地迎合他。杨万石终生没受过这般优待,突然之间真是受宠若惊,坐立不安,不知该怎么办好。有天晚上,尹氏想起那巨人的样子,还吓得瑟瑟发抖。杨万石想讨好她,泄露了那巨人是假的。尹氏一听,一骨碌坐起身,穷根究底地追问他。杨万石自知失言,后悔也晚了,只得实说了。尹氏勃然大怒,破口大骂起来。杨万石害怕,跪在床下不起来,尹氏不理。杨哀求到三更,尹氏才说:“想叫我饶了你,你必须自己用刀在你心口处也划上那么多口子,我才解恨!”于是起身到厨房拿菜刀。杨万石大为恐惧,连忙逃出了屋子。尹氏握着刀追赶出来,闹得鸡飞狗跳,一家人全都起来了。杨万钟不知是什么缘故,只是用身子左右挡护着哥哥。尹氏正在叫骂着,忽见杨老汉也走过来;又见他穿着崭新的袍服,更加暴怒,扑上前去,把老汉的衣服割成条条碎片,又猛打老汉的耳光,往下拔他的胡子。杨万钟见了大怒,拿起块石头砸过去,正中尹氏的脑门,一下子跌倒在地死了过去。杨万钟说:“只要父兄能活下去,我即使死了,也没什么遗憾了!” 说完便投井自杀了。等把他救上来,早已死了。尹氏不久又苏醒过来,听说杨万钟死了,才稍微解了恨。埋葬了杨万钟后,杨万钟的寡妻留恋儿子,不愿改嫁。尹氏对她动不动就辱骂,不给饭吃,硬逼她改嫁走了。只留下杨万钟的儿子孤单一人,天天遭受尹氏鞭打,等家人吃完后,才给孩子一点冷饭块吃。不过半年,就把孩子折磨得骨瘦如柴,仅剩下一口气了。

一天,马介甫忽然又来了,杨万石嘱咐家人不要告诉尹氏。马介甫见杨父又和以前一样衣衫褴褛,大吃一惊;又听说杨万钟死了,跺着脚悲叹不已。喜儿听说马介甫来了,便跑过来依偎在他身边恋恋不舍,连声叫着“马叔”。马介甫一时没认出他来,端详了很久,才认出他是喜儿,惊讶地说:“孩子怎么瘦弱成这个样子了?”杨父嗫嗫嚅嚅地对马介甫讲了一遍。马介甫生气地对杨万石说:“我过去说你不像人样,果然没说错。你们兄弟二人就这一根苗,孩子如被害死了怎么办?”杨万石一言不发,只会俯首帖耳地流泪。过一会儿,尹氏便知道马介甫来了。她不敢自己出来赶客人走,就把杨万石叫进去,一甩手就是几巴掌,逼他赶走马介甫。杨万石含着泪出来,脸上的掌痕还清清楚楚。马介甫发怒地说:“你不能制服她,难道就不能休了她吗?她殴打父亲,害死弟弟,你竟安心忍受,怎么做人?”杨万石听了,坐立不安,似乎被打动了。马介甫又激他说:“如她不愿走,理应用武力赶走她,就是杀了她也不要害怕。我有两三个知己朋友,都身居要职,一定会给你出力,保你无事!”杨万石答应,负气奔进内室,正好迎面碰上尹氏。尹氏大声责问:“你要干什么?”杨万石一下子变了脸色,双膝一软,不由自主地跪在地上说:“马生教我休了你。”尹氏更加狂怒,四处寻找刀杖。杨万石恐惧万分,急忙逃了出来。马介甫鄙夷地说:“你真是不可救药!”说完,打开一只箱子,取出一点药末,掺在水里让杨万石服下,说:“这药叫‘丈夫再造散’。我所以不敢轻易使用它,是因为这种药能伤害人。现在迫不得已,姑且试试吧!”杨万石喝下药后,顷刻便觉一股怒气从胸中冒出,像烈火烧着一样,一刻也忍受不了,径直奔进内室,喊叫声像打雷一样。尹氏还没来得及讲话,杨万石飞起一脚,把她踢出几尺以外,跌倒在地。接着又攥起块石头,往她身上砸了无数下,打得她几乎体无完肤。尹氏嘴里还在含混不清地怒骂不止,杨万石更加暴怒,从腰里拔出刀子。尹氏见了,叱骂说:“拔出刀子,你敢杀我吗?”杨万石一言不发,从她大腿上一刀割下巴掌大的一片肉扔在地上。刚要再割,尹氏已疼得哀叫着求饶。杨万石不听,又割下一块肉扔了。家人们见杨万石又凶又狂,急忙跑过来,死命将他拉了出去。马介甫迎上去,挽着他的胳膊慰劳了一番。杨万石还余怒不息,屡屡挣扎着要再去找尹氏,马介甫劝阻住他。又过了一会儿,药力渐渐消失,杨万石又变得垂头丧气起来。马介甫嘱咐他说:“你不要气馁!重振男子汉大丈夫之气,全在此一举。人之所以怕老婆,并不是一朝一夕就能形成的,而是有一个过程。就好比昨天的你已经死了。今天又复活了一个新的你,必须从此洗旧革新。再一气馁,可就无法挽回了!”说完,让杨万石进去看看尹氏动静。尹氏一看见杨万石,还吓得全身发抖,从心里服了,让奴婢硬扶自己起来,要跪爬过去迎接。杨万石阻止,尹氏才罢了。杨万石出来后告诉马介甫,杨氏父子都非常高兴。马介甫便要告辞,父子都挽留他。马介甫说:“我正要去东海,所以顺路来看看你们。回来时我们还能相见。”

过了一个多月,尹氏才渐渐伤好起床了,她对丈夫十分恭敬。可日子一长,她觉得杨万石黔驴技穷,似乎没什么别的能耐,对他先是亲昵,渐渐嘲笑,渐渐喝骂,不长时间,完全恢复了老样子。杨父忍受不了,深夜逃到河南当了道士,杨万石也不敢去寻找他。

过了一年多,马介甫来了,得知事情经过,愤怒地斥责了杨万石一番。立即叫过喜儿,把他抱到驴背上,撇下杨万石,赶着毛驴走了。从此后,村里的人都鄙视杨万石。学使驾临考核生员时,认为杨万石品行恶劣,革去了他的生员资格。又过了四五年,杨万石家遭受火灾,房子财物全部化为灰烬,还延烧了邻居家的房屋。村里的人把杨万石扭送到郡府,打起官司,官府罚了他很多银两。于是杨万石家产渐尽,连住的地方都没有了。邻村的人都相互告戒,谁也不要借给他房子住。尹氏的兄弟们愤怒她的所作所为,也拒绝接济,不让她回娘家。杨万石穷困不堪,只得把王氏卖给了大户人家,自己带着尹氏向南出走。到河南地界,旅费便没有了。尹氏不愿跟他走,一路嚷叫着要改嫁。正好有个屠夫死了老婆,便花三百吊钱把尹氏买走了。只剩杨万石一人,在附近的城市乡村中讨饭度日。

一天,杨万石到一个大户人家门前讨饭,看门的人斥责着赶他走。一会儿,有个官员从门里出来,杨万石急忙跪在地上哭泣着乞讨。那官员仔细端详他,又问了问姓名,惊讶地说;“是我伯父!怎么穷到这个地步!”杨万石细看,认出是弟弟的儿子喜儿,不禁失声痛哭,跟着喜儿进了家。只见高房大屋,金碧辉煌。一会儿,杨父扶着一个童儿出来,父子见面,相对悲泣。杨万石才讲述了自己的遭遇。原来,马介甫带走喜儿后,一直让喜儿住在这里。几天后,马介甫又去找了杨父来,让他们祖孙团聚。又请了先生,教喜儿读书。喜儿十五岁时考中了县学,第二年又中了举人。马介甫又替他娶了妻子,便要告别。祖孙二人哭着挽留他,马介甫说:“我不是凡人,是狐仙,道友们已等我很久了!”于是,告辞走了。喜儿说到这里,不禁感到心酸。又想起自己过去同庶伯母王氏倍受酷虐,越发悲伤。于是,喜儿派人带着银两,用华丽的车子,把王氏赎出接了回来。一年多,王氏生了个孩子,杨万石便把她扶作正妻。

尹氏跟了屠户半年,还是像以前那样凶悍狂悖。一次,屠户大怒之下,用屠刀把她大腿上穿了个洞,再用根猪毛绳从洞里穿过去,把她吊在了房梁上,自己挑着肉出门走了。尹氏号叫得声嘶力竭,邻居才知道。把她放下来,从伤口里往外抽绳子,每抽动一下,尹氏喊疼的叫声震动了四邻。从此,尹氏见了屠户就毛骨悚然。后来大腿上的伤虽然好了,但毛绳上的断毛留在肉里,走起路来终究还是一瘸一拐的。还得昼夜服侍屠户,不敢稍有松懈。屠户蛮横残暴,每次喝醉酒回来,就毒打尹氏一顿,毫不留情。到此时,尹氏才明白过去自己强加给别人的虐待,也是像自己今天的景况一样不好受。

一天,喜儿的夫人跟伯母王氏到普陀寺烧香,附近村庄的农妇都来拜见她们。尹氏也混在人群里,怅惘地不敢靠前。王氏看见了她,故意问:“这是谁呀?”家人禀告说;“她是张屠户的老婆。”呵斥尹氏上前,给太夫人行礼。王氏笑着说:“这个妇人既是屠户的老婆,应该不缺肉吃,怎么如此瘦弱?”尹氏听了又惭愧又愤恨,回家后便去上吊,但绳子太细,没能吊死,屠户也就更加厌恶她。

又过了一年多,张屠户死了。一次,尹氏在路上遇到杨万石,远远地望见他,便跪在地上爬过去,泪流如雨。杨万石碍着仆人在场,一句话没和她说。但回去后却告诉侄子,想接回尹氏,侄子坚决不同意。尹氏被村里的人唾弃,久久没有个归宿,便跟着乞丐们讨饭度日,杨万石还不时地和她在野外荒庙中幽会。侄子引以为耻,暗暗地让乞丐们把杨万石羞辱了一番,他才和尹氏断绝了关系。这件事我不知究竟,最后几行是毕公权撰写成的。


\subsection{1.6.3   魁 星}
\label{\detokenize{p00_u5176_u5b83/_u767d_u8bdd_u804a_u658b_u5fd7_u5f02:id220}}
郓城县人张济宇,一天躺在床上还没睡着,忽然看到一片光明照满屋内。他惊异地看着,见一个鬼拿着笔站着,像是魁星的样子。他急忙起来向魁星跪拜叩首,光明也随即消失了。

从此张济宇便自负起来,认为这一定是科考第一的预兆了。可是从这以后,他竟然潦倒失意,一事无成,家境也败落下来,亲人又接连死去,只有他一个人活着。

那个魁星为什么不给张济宇降福,反而降祸呢?


\subsection{1.6.4   厍 将 军}
\label{\detokenize{p00_u5176_u5b83/_u767d_u8bdd_u804a_u658b_u5fd7_u5f02:id221}}
有个叫厍大有的人,字君实,是陕西省汉中洋县人氏。他是个武举人,隶属祖述舜部下。祖述舜给他的待遇很优厚,多次提拔他,并晋升他为后周的总戎。后来,厍大有感到后周政权大势已去,就秘密偷袭祖述舜。祖述舜在格斗中奋力抗拒,结果伤了手,被捆绑起来。

厍大有归顺了总督蔡毓荣。来到都城,梦中到了冥王府。冥王因为厍大有不讲道义,非常生气。命令小鬼用滚沸的油浇在他的脚上。厍大有醒来后,感到双脚疼得难以忍受。后来他的脚肿烂了,脚指全都脱落,又增添了疟疾,总是连声呼叫着说:“我实在是负义之人!”终于死去了。


\subsection{1.6.5   绛 妃}
\label{\detokenize{p00_u5176_u5b83/_u767d_u8bdd_u804a_u658b_u5fd7_u5f02:id222}}
康熙二十二年,我在刺史毕际有公的绰然堂设馆教书。毕刺史家的花草树木极为茂盛,闲暇对我就跟从毕公漫步,得以尽兴地游赏奇花异草。

一天,我观赏完花木回到房内,因极度困倦想睡一觉,便脱下鞋来上了床。睡梦中见两个女子,衣着鲜艳华丽,走过来很恭敬地说:“有件事想拜托您,敢劳大驾前去。”我惊讶地急忙起来问:“是谁招呼我?”她们说:“是绛妃。”我被她们说糊涂了。不明白绛妃是谁,但也就连忙跟着她们去了。

不多时,就见一片宫殿楼阁,高接天际。下面有石砌的台阶,沿着台阶一层一层地往上攀登,大约上了一百多层才到了顶端。只见红漆大门敞开着,又有两三个美丽的女郎,急忙进去通报。一会儿,我跟着她们来到一座大殿外面。这大殿有金质的帘钩、碧绿的门帘,光闪闪地耀人眼睛。殿内有一个女子从台阶上走下来,身上佩带的玉佩发出铿锵悦耳的声响,样子像是皇宫的嫔妃。我正想向她施礼,女子却先说道:“委屈先生远来,理应先向你致谢。”便招呼身边的侍女,把毯子铺在地上,样子像是要给我行礼。我惶恐得手足无措,便对她讲道:“草莽微贱之人,有幸得到您的召唤,已经感到不尽的荣耀;又胆敢以平等的礼节拜见您,更加重了我的罪过,折损了我的福分!”

绛妃便叫使女们撤去地毯,摆设了宴席,对面坐下。酒过数巡,我即告辞说:“我喝不了几杯就醉,恐怕酒醉失态,有违礼仪。您有什么吩咐请赐教,以消除我的疑虑。”绛妃不说话,只是用大杯催促我喝酒。我几次请她指教,她才说:“我是花神,合家的眷属都寄居在这里,经常被封家的丫头蛮横摧残。今天想和她们作一决战,拜托您撰写声讨她们的檄文。”我惶恐不安地站起来说:“我学问浅薄,不善文辞,恐怕辜负了您的重托。只是奉您的命令,怎敢不竭尽我至诚的愚拙。”

绛妃很高兴,就在殿上赐给我笔和墨。众女郎拂拭几案座位,磨墨润笔。又有一个垂发少女把纸叠成文书格式,放在我的手腕下面。我才略写了一两句,便有两三个女郎凑过来观看。我平时不很敏捷,这时却觉得文思泉涌。不多时,就把稿子写完了,她们争着拿去呈给绛妃。绛妃展开稿子看了一遍,说写得很不错,于是又送我回到绰然堂。我醒来之后回忆这件事,每个情节都清楚地浮现眼前,只是那檄文中的词句多半记不起来了。因此,只能补上不足之处,使它成为完整的檄文:

“谨按封氏:飞扬成性,忌嫉为心。济恶以才,妒同醉骨;射人于暗,奸类含沙。昔虞帝受其狐媚,英、皇不足解忧,反借渠以解愠;楚王蒙其蛊惑,贤才未能称意,惟得彼以称雄。沛上英雄,云飞而思猛士;茂陵天子,秋高而念佳人。从此怙宠日恣,因而肆狂无忌。怒号万窍,响碎玉于王宫;澎湃中宵,弄寒声于秋树。倏向山林丛里,假虎之威;时于滟滪堆中,生江之浪。且也,帘钩频动,发高阁之清商;檐铁忽敲,破离人之幽梦。寻帷下榻,反同人幕之宾;排闼登堂,竟作翻书之客。不曾于生平识面,直开门户而来;若非是掌上留裙,几掠妃子而去。吐虹丝于碧落,乃敢因月成阑;翻柳浪于青郊,谬说为花寄信。赋归田者,归途才就,飘飘吹薜荔之衣;登高台者,高兴方浓,轻轻落茱萸之帽。蓬梗卷兮上下,三秋之羊角抟空;筝声入乎云霄,百尺之鸢丝断系。不奉太后之诏,欲速花开;未绝座客之缨,竟吹灯灭。甚则扬尘播土,吹平李贺之山;叫雨呼云,卷破杜陵之屋。冯夷起而击鼓,少女进而吹笙。荡漾以来,草皆成偃;吼奔而至,瓦欲为飞。未施抟水之威,浮水江豚时出拜;陡出障天之势,书天雁字不成行。助马当之轻帆,彼有取尔;牵瑶台之翠帐,于意云何?至于海鸟有灵,尚依鲁门以避;但使行人无恙,愿唤尤郎以归。古有贤豪,乘而破者万里;世无高士,御以行者几人?驾炮车之狂云,遂以夜郎自大;恃贪狼之逆气,漫以河伯为尊。姊妹俱受其摧残,汇族悉为其蹂躏。纷红骇绿,掩苒何穷?擘柳鸣条,萧骚无际。雨零金谷,缀为藉客之裀;露冷华林,去作沾泥之絮。埋香瘗玉,残妆卸而翻飞;朱榭雕栏,杂佩纷其零落。减春光于旦夕,万点正飘愁;觅残红于西东,五更非错恨。翩跹江汉女,弓鞋漫踏春园;寂寞玉楼人,珠勒徒嘶芳草。斯时也:伤春者有难乎为情之怨;寻胜者作无可奈何之歌。尔乃趾高气扬,发无端之踔厉;催蒙振落,动不已之阑珊。伤哉绿树犹存,簌簌者绕墙自落;久矣朱幡不竖,娟娟者霣涕谁怜?堕溷沾篱,毕芳魂于一日;朝荣夕悴,免荼毒于何年?怨罗裳之易开,骂空闻于子夜;讼狂伯之肆虐,章未报于天庭。诞告芳邻,学作蛾眉之阵;凡属同气,群兴草木之兵。奠言蒲柳无能,但须藩篱有志。且看莺俦燕侣,公覆夺爱之仇;请与蝶友蜂交,共发同心之誓。兰桡桂楫,可教战于昆明;桑盖柳旌,用观兵于上苑。东篱处士,亦出茅庐;大树将军,应怀义愤。杀其气焰,洗千年粉黛之冤;歼尔豪强,销万古风流之恨!”


\subsection{1.6.6   河 间 生}
\label{\detokenize{p00_u5176_u5b83/_u767d_u8bdd_u804a_u658b_u5fd7_u5f02:id223}}
河间县有个书生,在自家的场上积攒了一个像山丘那样大小的麦穰垛。家人天天从垛上撕麦穰烧,日子一长,把垛上撕了个洞。有一只狐就住在这个洞中,经常变化成一个老翁,去拜见书生。

一天,狐又变化成老翁,请书生去喝酒。到了麦穰垛前,狐翁拱手请书生入洞。书生很为难,狐翁再三邀请,书生才钻了进去。进洞一看,只见房屋走廊,华丽宽敞。坐下后,摆上来的茶、酒都芳香无比。只是日色昏黄,也分不清是白天还是晚上。喝完酒,出来再同头一看,又什么都没有了。

狐翁经常在晚上外出,直到第二天一早才回来,谁也不知他去了哪里。问他,便说是有朋友请他去喝酒。一次,书生请他带自己一同前去,狐翁不答应。书生再三恳求,狐翁才同意,挽住书生的胳膊,快如疾风地往前行去。走了有做顿饭的功夫,来到一个城市。二人走进一家酒店中,只见客人很多,一桌一桌地聚在一起喝酒,一片喧闹声。狐翁领着书生来到楼上,往下看下边喝酒的人,桌几上摆着的菜肴都历历在目。狐翁自己下楼,任意取拿桌上的酒果,捧上来让书生吃,喝酒的人竟一点也不察觉。过了一会儿,书生见楼下一个穿红衣服的人桌上摆着金桔,便请狐翁去拿。狐翁说:“那人是个正派人,我不能接近他!”书生听了这话,心里默想:狐跟我交游,一定是因为我有邪心的缘故;从今往后,我必定要做个正派人!刚想到这里,忽然身子不由自主,头一晕,从楼上掉了下去。楼下喝酒的人大吃一惊,都吵嚷起来,以为是妖怪。书生仰头往上一看,哪里有楼,原来刚才是在房梁上!书生将实情告诉了众人,众人审知他说的是实话,便给他路费,让他走了。书生问众人这是什么地方,得知是山东鱼台县,离河间县已一千多里路了。


\subsection{1.6.7   云 翠 仙}
\label{\detokenize{p00_u5176_u5b83/_u767d_u8bdd_u804a_u658b_u5fd7_u5f02:id224}}
梁有才,原籍山西,是个小商贩。暂住在济南。家里一无妻子二无田地,独身一人。

一天,梁有才跟别人去爬泰山。泰山在四月里去烧香的人很多。又有男女信徒一百多人,间杂着跪在神座下面,看着香烧完了才起来,叫作“跪香”。梁有才看见这些跪着的人中有一个女子,年纪有十七八岁,长得很美,非常喜欢她。他佯装香客,靠近女郎跪下。又装作膝盖没劲的样子,一俯身去摸女郎的脚;女郎回头看了下,似乎有点生气,就跪着走了几步,离梁有才远了一些。可梁有才也跪着走过去靠近了女郎,一会儿,又去摸女郎的脚。女郎察觉梁有才不怀好意,忽地站起来出门走了。梁有才也不跪了,去追踪女郎,可是出来看了看女郎的足迹,却不知向哪里去了,心里大失所望,没精打采地走着。半路上看见女郎跟着一个老妇人一起走,看样子像是女郎的母亲。梁有才跟上去。老妇人与女郎一面走路一面说话。老妇人说:“你能来给泰山娘娘叩头,是好事!你又没有弟弟妹妹,但求娘娘暗中保护,能找到个好女婿,只要孝顺,不一定是王孙公子。”梁有才听了,心中暗暗高兴,渐渐靠近老妇人与她搭话。老妇人自称姓云,女儿名叫翠仙,是她的亲生女,家住西山里,离此四十多里路。梁有才说:“山路很难走,大娘你年纪大了走路费力,小妹又这样细弱也走不快,什么时候才能到家?”老妇人说:“天已晚了,我们准备在她舅舅家里住一宿。”梁有才又说:“刚才您说找女婿不嫌穷,只要人好;我还没有结婚,我能使您满意吗?”老妇人问女儿,女郎没说话。问了好几次,女郎才说:“他没有福气,又行为浮荡,容易反复无常,我不能给这种薄情人作妻子!”梁有才听了,竭力表白自己诚实,还指天盟誓。老妇人听了很欢喜,竟答应了他的婚事。女郎很不高兴,变了脸色。老妇人又拍了一下梁有才,表示亲切。粱有才更加殷勤,拿出钱雇了两个二人抬,叫她母女坐,自己步行跟在后面,像个仆人一样。每逢不好走的地方,还喊轿夫慢点走不要摇摆,表现非常殷勤。

过了一会儿,进了一个村子,老妇人便邀梁有才一同到女郎舅舅家。舅翁及妗子出来相迎,老妇人称他们哥哥嫂嫂。对他们说:“梁有才是我的女婿,今天正是好日子,不要再选择日子了,今晚就叫他们成婚。”舅翁也很高兴,拿出酒肴招待梁有才。接着,云翠仙穿着礼服出来,三位老人就扫了床催他们早睡。

梁有才与女子入了洞房,女郎说:“我本来就知道你是个不义之人,但迫于母命,姑且嫁你。你若是好好为人,不愁白头到老。”梁有才唯唯地答应着。天明,早早起床,老妇人对梁有才说:“你先回家,我与女儿随后就到。”

梁有才回到家里,把房子、院子打扫干净,老妇人果然送女郎来了。母女进屋一看,什么也没有。老妇人便说:“这个样子怎么过日子?我马上回去,给你们点小小帮助。”便走了。

第二天,有几个男女送来衣被、用具,摆了满满一屋,连顿饭没吃就走了,只留下一个小丫鬟。梁有才从此坐享温饱,每日招呼一些无赖饮酒、赌博,渐渐偷妻子的首饰去赌。云翠仙多次劝阻,梁有才不但不听,还很不耐烦。翠仙无法,只好天天守着箱子,像防贼一样。

一天,赌徒们叫门找粱有才,偷着看见了云翠仙,非常吃惊,试着对梁有才说:“你太富贵了,还愁穷吗?”梁有才问原因。赌徒们说:“刚才见你夫人,实在是天仙一样,她与你的家道很不相称。卖给人家作妾,可得一百两银子;如卖到妓院,可得一千两银子。你一旦千两银子到手,还怕没钱饮酒赌博?”梁有才当时虽没有说什么,但心里却很以为然。回到家里时时对妻子叹气,说穷得没法过。翠仙也不理他,粱有才就天天敲桌子,硬板凳,扔筷子,骂丫鬟,作出种种姿态叫翠仙看。一天晚上,翠仙打了酒来与他对饮,忽然对他说:“你因为家里穷,天天焦心,我又无法使你不穷,不能替你分优,哪能不惭愧?但家里又没别的东西可卖了,只有这个丫鬟,卖了她,可能还稍稍解决点用度。”粱有才摇摇头说:“她能值几个钱!”又呆了一会,翠仙说:“我对于你,还有什么不能支持的?但真的一点办法也没有了。想我们穷到这个地步,就是死心地跟你过一辈子,不过是都受一百年苦,能有什么前途?不如把我卖给有钱的人家,都能得到好处,卖的钱可能比丫鬟多些。”粱有才故意装作惊讶地说:“何至于如此?”翠仙再三要求,脸色很认真。粟有才才高必地说:“再慢慢商量。”

粱有才于是便托宦官把妻子卖给官府的妓院。宦官亲自来看人,见了云翠仙,非常高兴,怕不能到手,立下字据,支了八百串钱,事情就办成了。翠仙说:“我母亲因为你穷,常常挂念,今天咱们断了情缘,我得回娘家一趟。况且咱俩就要分开了,哪能不告诉母亲一声?”梁有才顾虑她母亲阻拦。翠仙说:“这是我自己愿意的,保证不要紧。”粱有才听从了,便跟着翠仙去她娘家。

半夜才到了翠仙娘家,叫开门进了院子,见楼房非常华丽,丫鬟使女来来去去的很多。以前粱有才天天与翠仙在一起,经常要求来看岳母,翠仙都不同意;所以当了一年多女婿,还没有走一次岳母家。今天一见,十分惊奇,心里想,她家原是这样的大户人家,恐怕不甘心把女儿卖去当妓女。

翠仙领粱有才上了楼,老妇人一见,惊讶地问夫妻俩为何而来。翠仙抱怨说:“我原来就说他是个不义的人,如今果然不错!”便从衣服里边拿出两锭黄金披在桌子上,说:“这金子幸亏没有被小人偷了去,今天仍归还给母亲。”母亲惊奇地问缘故。翠仙说:“他将要把我卖了,我收着这金子也没有用处。”指着粱有才大骂:“豺狼!鼠子!以前你挑着担子,满脸是土,像鬼一样。结婚时,浑身汗臭气,身上的污垢掉下来几乎砸塌了床,脚上老皴一寸多厚,叫人整夜恶心。是我进了你家,你才坐吃三餐,脱了你那身穷鬼皮。母亲在上,难道我是说谎吗?”粱有才低着头,一声也不敢吭。翠仙又说:“我自己知道我没有倾国倾城的相貌,不配侍奉富贵的人;像你这样的男人,我自认为还配得过你。有什么亏待你的地方,竟不念一点香火之情?我岂不能盖楼房、买田地?就是看你一身穷骨头,天生乞丐相,早晚不能白头到老!”说完了,丫鬟婆子们连起手来,团团围住梁有才。看见小姐斥骂他,便一起唾骂,都说:“不如杀了他,何必多说!”梁有才害怕,跪在地上认错,直说自己知道悔改了。翠仙又生气地说:“卖妻子已是十恶不赦,够残忍的了,况且还把同床人卖去当妓女!”话还没有说完,众人都怒日圆睁,一起用簪子、剪刀刺梁的肋下、踝骨。梁有才嚎啕大哭,叫喊着求饶。翠仙制止住说:“可以暂时放了他,他就是不仁不义,我还不忍心看他害怕的样子。”便领着丫鬟使女们下楼去了。

梁有才坐着听了一会儿,没有动静了,心里想偷跑。一仰头,看见满天星斗,东方已发白了。四面一片苍茫的原野,灯也没有了,房子也没有了,自己坐在峭壁上,向下看是深不见底的山谷,心里害怕掉下去。身子一动,轰隆一声随着乱石就掉了下来。幸亏半山腰有棵枯树挡了一下,没有掉入山谷。他肚子挂在枯树上,手足都够不到东西。向下看茫茫然不知有多少丈深,身子一动也不敢动,连喊带怕,声嘶力竭,全身都肿了,眼、耳、鼻、舌、身,都一点劲也没有了。

太阳渐渐升高了,才有个打柴的人看见梁有才。他找了条绳子来,把梁放下山崖,已经奄奄一息。打柴的人把他送回家去。到了家,大门敞着,家里一片荒凉,像座破庙,桌椅板凳都没有了,只有一张绳结的床和一张破桌子,还是他家的旧物。零零乱乱地还放在那里。梁有才浑身无力地躺下,饿了,就向邻居家要口饭吃。接着身子肿处溃烂了,成了癞疮。乡里人看不起他,都不理他。梁有才没有办法,卖了破屋,住在土洞里,在街上乞讨,随身还带着一把刀。有人劝他用刀换点吃的,粱有才不肯,说:“住在野外,要防备虎狼,得用它自卫。”

后来,梁有才在路上遇到劝他卖妻子的那个人,走到近前与那人说话,忽然抽出刀来把那人杀了,于是被捕入狱。县官得知这里面的一些情由;没忍心虐待他,只是把他关起来,没有多久,梁有才便死在狱中。


\subsection{1.6.8   跳 神}
\label{\detokenize{p00_u5176_u5b83/_u767d_u8bdd_u804a_u658b_u5fd7_u5f02:id225}}
济南的风俗,民间有生病的人,就在闺房内求神占卜吉凶。请来老巫婆敲打带铁环的单面鼓,舞步婆娑,跃然作态。叫做“跳神”。而这一风俗在京城中尤其盛行。良家少妇们也时常自已这样做。在堂屋中,托盘里放着肉,盆子里装着酒,条几上点燃着大蜡烛,比白天还明亮。一个少妇扎着短裙子,弯屈起一只脚跳“商羊舞”,另有两人各抓着少妇一条胳膊,在两边架着她。少妇口中念念有词地絮叨着,像是在歌唱,又像是在祈祷,字句或多或少,长短不齐,虽然不合韵律,却拖着长腔。室内几面鼓同时乱打,犹如雷鸣,声音杂乱刺耳。少妇的嘴唇一启一合,掺杂着鼓声,听不清唱的什么。不久,少妇低下头来,眼睛斜视着一旁,站立全靠别人搀扶,不搀扶就向前倒下去。一会儿,少妇忽然伸着脖子高跳起来,离地一尺多高。室内各个女子都严肃起来,惊恐地张望着说:“祖宗来吃饭了。”便呼地一口气吹灭了灯,室内外一片昏黑。人们都惊惧地屏住呼吸立在暗中,谁也不敢交谈一句;即使说话也听不到,因为这时的鼓声太乱了。“祖宗”吃了不多时,就听到少妇厉声呼唤公婆和兄嫂的小名。这才一起点燃蜡烛,躬着腰询问吉凶。看那酒杯、盆子和托盘里,都已空空的了。人们看少妇脸色的变化,观察她面部表情是恼怒还是喜悦,恭恭敬敬地问长问短。少妇有问必答。问病的人中有在内心非议的,神已经知道,便指出某人讥笑我,大不敬,要脱下他的裤子。这个讥笑神的人自顾全身,已是光溜溜的裸体,每每在门外的树梢上找到裤子。

满洲的妇女,尊崇侍奉神尤其虔诚,即使有一点儿疑惑,也一定求神来判断。“跳神”的人常是穿着整洁,骑假虎假马,拿着长兵器,在床上舞动,叫做“跳虎神”。假马假虎的姿势显示出威武愤怒的样子。跳神的人声音粗重,有时自称是关羽、张飞或赵公明,都不一样。气势威严,阴冷可怖。男子如从窗纸上开个小孔往室内偷看,就立即被长兵刃从窗内穿出刺中帽子,挑进屋里去。一家里的老妇人、媳妇、姐妹,都严肃地瑟缩着,小心翼翼地像群雁排成“一”字形站在那里,不敢胡思乱想,也不敢轻举妄动。


\subsection{1.6.9   铁 布 衫 法}
\label{\detokenize{p00_u5176_u5b83/_u767d_u8bdd_u804a_u658b_u5fd7_u5f02:id226}}
有个姓沙的回民,学得了铁布衫大力法。他把五指并起来,用力砍下去,可以砍断牛脖子;横着捅过去,可把牛肚子穿一个窟窿。

他曾经在仇彭三公子的家里,把一块又粗又重的木头悬挂在空中,让两个体壮力大的仆人使足力气把悬木推出很远,然后使悬木猛然荡回来;沙某用赤裸裸的肚子迎接撞来的悬术,“砰”的一声响,悬木被顶出老远。沙某又掏出自己的生殖器,平放在石头上,用木槌子使劲砸,没有一点儿损伤;只是怕刀罢了。


\subsection{1.6.10   大 力 将 军}
\label{\detokenize{p00_u5176_u5b83/_u767d_u8bdd_u804a_u658b_u5fd7_u5f02:id227}}
查伊璜,是浙江人。有一年的清明节,他在野外一座寺庙里喝酒,见大殿前有口古钟扣在地上,这钟足有一个可盛两石的大水瓮那样大,钟身上和地下留着清清楚楚的用手抓过的新痕迹。他很惊疑,趴在地上往钟里看了看,里面藏着一只可装八升左右的小竹筐,筐里不知有什么东西。他便命几个人抓着钟耳,奋力一提,古钟纹丝没动。查伊璜更加惊疑,便继续坐下喝酒,等着那个往钟里藏东西的人来。

过了一会儿,走来一个年轻的乞丐,把讨来的饭堆在钟的一边;然后一只手掀开钟,另一只手把饭抓进筐里,一连掀了好几次,才把饭放完。然后仍把钟扣好,走了。过了不久,他又回来了。掀开钟抓把饭吃起来,吃完掀钟再取,轻松得像开个柜子一样。查伊璜和同座的人都惊骇不已。查伊璜起身问道:“你这样一个堂堂男子汉,怎么讨饭呢?”乞丐回答说:“我饭量大,没人愿雇我做工。”查伊璜见他力气极大,劝他从军,乞丐忧愁没有门路。查伊璜便把他带回家中,让他饱餐一顿,估计他的饭量,大概比普通人多吃五六倍。又替他换了新衣新鞋,赠他五十两银子作为路费。送他从军去了。

过了十多年,查伊璜的一个侄子在福建做县令。有个叫吴六一的将军忽然来拜访他。交谈间,将军问查县令:“查伊璜是你什么人?”查县令回答说:“是我叔父。不知他与将军在何处有过交往?”将军说:“他是我老师,分别十年了,我非常想念他。麻烦您告诉他一声,请他赏光来我家作客!”查县令漫不经心地答应了一声,心想:叔父是个名儒,怎么会有武弟子呢?

过了不久,查伊璜正好来到侄子这里,查县令便告诉了他这件事,查伊璜茫然记不起;因那将军问讯自己时很是恭敬迫切。查伊璜便命备马,带着仆人去登门拜访。将军急急忙忙地迎出大门来。查伊璜打量打量他,一点也不认识,心里怀疑将军认错了人。但将军对他却越发恭恭敬敬,将客人请进家,又穿过三四道门,忽见院中有女子来来往往,查伊璜知道这是将军的内院,不禁站住不前。将军又作揖请他再往里走,一会儿走进堂屋,只见掀门帘的、搬椅子的,全是年轻的侍妾。查伊璜落座后,刚想问个明白,见将军脸上微一示意,便有个侍妾给他捧来官服。将军匆忙站起来更衣,查伊璜不解他要干什么。众侍童帮着将军穿戴整齐,将军又命几个人过去按着查伊璜不让起身,自己大礼参拜起来,犹如拜见皇帝一样。查伊璜极为惊愕,不明白是怎么回事。将军拜完,又换上便服在一边陪坐,笑着说:“先生不记得那个举钟的乞丐了吗?”查伊璜才恍然大悟。过了会儿,将军摆上了丰盛的酒宴。下面奏起乐曲。喝完酒,将军去为查伊璜安排了住宿的地方,又命几个侍妾服侍着他,自己才告辞离开了。

第二天,查伊璜因为酒醉起得很迟,将军已在他卧室门外问候多次了。查伊璜得知后,心里很不安,想告辞回去。将军把大门锁上,不让走,查伊璜见将军连续几天不干别的,只是在清点家中的奴仆丫头、骡马器具和珍玩服饰,亲自监督着造簿登记,一再告诫不要遗漏了。查伊璜以为这是将军的家务事,所以也没有深问。一天,将军拿着全部家产的登记簿,对查伊璜说:“我能有今天,全出于先生当年的厚赐。现在的一个奴婢、一件器物,我都不敢独自享有,请把我的一半家产分给先生!”查伊璜大吃一惊,坚决推辞。将军不听,又拿出窖藏的数万两银子,一分为二。又按登记簿点出一半古玩、床几等物,堂屋内外都快摆满了。查伊璜再三阻止,将军不顾,又按姓名点出一半奴婢仆人,随即命点出的男仆收拾行李,女仆收拾器具,并且嘱咐他们要好好伺候先生,仆人们齐声答应。将军亲眼看着婢妾们登上车子,仆人们套好骡马,热热闹闹地上路了,才和查伊璜告别。

后来,查伊璜牵连到修史一案中,被逮捕入狱。最后终于无罪释放,都是吴将军从中出力的结果。


\subsection{1.6.11   白 莲 教}
\label{\detokenize{p00_u5176_u5b83/_u767d_u8bdd_u804a_u658b_u5fd7_u5f02:id228}}
白莲教首领徐鸿儒,得到了一本左道旁门的书,能够驱使鬼神为他做事。一次他稍微试验了一下,观看的人都感到惊恐,投奔到他门下的人很多。于是徐鸿儒暗暗萌发了造反的念头。一天,他取出一面铜镜,说能够照出人的一生祸福。他把铜镜悬在院子里,让人们自照,镜子里的人,有的戴着头巾,有的戴着纱帽,锦绣华服,貂蝉美饰,形象不一。人们更加感到惊奇。从此这个消息到处传播,上门请求照镜子的人接连不断。徐鸿儒于是宣称:“凡是镜子里照出的文武高官,都是如来佛祖注定龙华会里的人。大家应该努力,决不能退缩。”于是徐鸿儒当着众人的面照自己,便看到镜子里的他头戴皇冠,身穿袞龙服,俨然就像帝王一样。众人你看我我看你,感到十分惊讶,一齐跪倒在地。

徐鸿儒于是竖起反旗,众人无不欢腾雀跃相随,希望自己能成为像镜子里的形象那样的高官。不到几个月,徐鸿儒就聚集了一万多人,滕县、峄县一带官府望风逃窜。后来大队清兵前去剿捕,其中有一位彭都司,是长山县人,武艺高强,无人能敌。白莲教军中出来两个少女和他交战,她们都使双刀,锋利如霜;骑着高头大马,非常威武、她们飘忽盘旋,从早晨一直杀到傍晚,少女不能伤害彭都司,彭都司也没能取胜。这样厮杀了三天,彭都司累得精疲力竭,最后气喘而死。后来徐鸿儒兵败被杀,捉到他的同伙拷问,才知道少女用的是木刀,骑的是木凳子。假兵马累死了真将军,也够奇异的了。


\subsection{1.6.12   颜 氏}
\label{\detokenize{p00_u5176_u5b83/_u767d_u8bdd_u804a_u658b_u5fd7_u5f02:id229}}
顺天有个书生,家中很穷,遇上荒年,跟随父亲到了洛阳。他生性迟钝,十七岁了,还写不出一篇完整的文章。然而却仪表文雅,相貌秀美,很会谈笑,善写书信,看见他的人并不知道他肚子里其实没有多少学问。不久,父母相继去世,只剩下他孤身一人,在洛汭一带教私塾度日。

当时村子里颜家有个孤女,是名士的后代,从小聪明。父亲活着时,曾教她读书,只学一遍就记住了。十几岁时,就学父亲的样子吟诵诗文。父亲说:“我家有个女学士,可惜不是男的。”因此特别喜欢她,期望为她选择一个做高官的女婿。父亲死后,母亲仍然坚持这个选婿目标,三年没有成功,母亲也去世了。有人劝颜氏找个有才学的文人,颜氏认为很对,但还没有着落。

有一次,邻居的妇人翻墙过来,同她攀谈,拿着用字纸包着的绣线。颜氏打开一看,字纸原来是那个顺天书生写的书信,寄给邻居妇人的丈夫的。颜氏反复阅读,似乎有好感,邻家妇人看透了她的心思,悄悄对她说:“这少年风度翩翩,很秀美,同你一样也是没有父母,年龄也相仿。你如果有意,我嘱托丈夫为你们撮合。”姑娘脉脉含情,没有说话。邻妇回去,把意思告诉丈失。邻生本来就同这书生很要好,便告诉了书生。书生非常高兴,就把母亲遗留给他的金鸦指环,托邻生转给颜氏作聘礼。几天后举行了婚礼,夫妻二人如鱼得水,十分欢乐。及至看了书生的文章,颜氏笑着说:“你写的文章和你像是两个人,像这样什么时候才能考中?”于是早晚劝书生攻读,像老师一样严厉。到黄昏时,自己先点灯坐在桌前吟诵,为丈夫作表率,直到三更才罢休。

这样过了一年多,书生对科举应试的八股文章已很精通,可是几次投考都名落孙山,困顿失意,茶饭不进,寂寞愁闷得悲痛哭泣,颜氏责备他说:“你不像个男子汉!如果让我换了发髻改成男人衣冠,我看那高官显位,如同拾取草芥一样容易!”书生正在懊丧,听了妻子的话,怒视着她生气地说:“你是不出闺房的人,没到过考场,就以为功名富贵像你在厨房提水煮粥一样容易。如果把男人的冠给你戴在头上,恐怕也和我一样。”颜氏笑着说:“你不要生气。到了考试的日期,请让我换了衣冠,代你应考。倘若也像你一样落拓,我当再不敢小看天下的读书人。”书生也笑着说:“你不知黄柏苦的味道,真应该让你去尝一下。只怕你换装后露出破绽,让乡邻笑话。”颜氏说:“我不是说笑话。你说过顺天有你的老家,让我换上男装跟你回去,假称是你弟弟,你从婴儿时就出来了,谁能辨出真假?”书生答应了她。颇氏进了内屋,换了男人的衣服出来,说:“你看可以充作男人吗?”书生仔细看她,俨然一个顾影自怜的俊美少年。书生高兴极了,向邻居一一告别,有交情好的稍微给他点馈赠。书生买了一头瘦驴,载着妻子回了老家。

书生的堂兄还在,见两个堂弟美如冠玉,很喜欢,早晚都来照顾他们。又见他们起早贪黑地用功读书,更加爱护尊敬他们,雇了一个小僮供他们使唤。到了黑天,颜氏和丈夫就打发小僮回去。乡里有吊丧、喜庆之事,书生自已去周旋,颜氏总是在家中读书。住了半年,很少有人见过颜氏的面。客人有时求见,哥哥总是代为辞谢。有人读了颜氏的文章,惊奇地赞叹不已。有时有人忽然闯入来相见,颜氏作个揖便回避了。客人见其丰采,又都倾倒,由此名声更大起来。一些世家争相招赞做女婿,堂兄来商议,颜氏只是一笑;再强求,就说:“我立志取得高官,不考中决不结婚。”到了考试的日子,两人一齐去投考,书生又落榜,颜氏则以科试第一名而参加乡试,考中顺天府乡试第四名。第二年又考中进士,授桐城县令。因治理有方,不久又升迁河南道掌印御史,富贵如同王侯。后来托病请求退职,被赐卸任返乡。家中常常宾客盈门,但颜氏始终辞谢不见。从儒生开始到显贵,从不提婚娶,人们没有不觉得奇怪的。回乡后,颜氏渐渐购置婢女,有人疑心这里面有私情,堂嫂留心观察,确实没有不正当的行为。

没过多久,明朝灭亡,天下大乱。颜氏这才告诉堂嫂说:“实言相告,我是你堂弟的妻子。因为丈夫平庸,不能自立,我才负气女扮男装求得功名,深怕传扬出去,致使天子召问,让天下人耻笑。”堂嫂不相信,颜氏便脱下靴子,让堂嫂看自己的脚,堂嫂才惊异起来。再看靴子里,塞满了碎棉絮。此后,颜氏让书生承袭了官衔,她则闭门做起深闺女人。又因她一生没有怀孕,便出钱让丈夫买妾。还对书生说:“凡是身居显贵的人,都要买姬妾侍女供奉自己。我为官十年,还只一身;你是何等福泽,坐享佳妇丽人。”书生说:“你也可以购置一批男宠,请夫人自己办吧。”相互调笑取乐。这时书生的父母,已多次受朝廷封赐之恩。富贵绅士来拜访,对书生施以御史的礼仪。书生羞于承袭闺阁女子挣的名衔,只以一般儒生自安,终身没有坐过官轿。


\subsection{1.6.13   杜 翁}
\label{\detokenize{p00_u5176_u5b83/_u767d_u8bdd_u804a_u658b_u5fd7_u5f02:id230}}
杜翁,是沂水县人。一天他偶然从街市上出来,坐在墙下,等候一起来逛市的伙伴。他觉得有些困倦了,忽然像是进入梦中。见一个人拿着公文把他抓去,到了一处官府衙门里,这地方他从来没有到过。有个戴瓦垄帽的人,从官署里出来,细看原来是青州的张某,是过去的熟人。张某见到杜翁很惊讶地问:“杜大哥,你怎么来到这里?”杜翁说:“不知怎么回事,不过有拘捕人的文书。”张某怀疑有差错,要去为他查验一下,叮嘱他说:“要小心谨慎地站在这里,不要到其它地方去,怕万一迷失了路,就很难挽救你了。”张某说完就走了,很长时间不见出来。只有那个拿着公文的人来了,自己承认是抓错了人,当即释放杜翁回家。

杜翁告别了那人往回走,路上遇见六七个女郎,容貌长得很美好,心中喜欢她们,就跟在她们后面走。下了大道,走上小路,刚走了十几步,听见张某在后面大声呼唤:“杜大哥,你要到哪里去?”杜翁迷恋女郎们,情不自主地跟着走。眨眼间,见众女郎进入一个小门中,认得这是卖酒的王某家。他不觉探身门内,刚往里瞧了一眼。就见自己已在猪圈里,和许多小猪卧在一起。这才一下子明白过来,原来自己已经变成猪了。不过耳内还听到张某呼喊。他非常害怕,急忙用头碰撞猪圈的墙壁。听到有人说:“小猪得了羊痫风了。”他上下打量自己,已经又变成了人。急忙走出门来,就见张某已经等候在路上。张某责备他道:“本来叮嘱你不要到别处去,你为什么不听我的话呢?几乎坏了大事!”于是握着杜翁的手把他送到街市口,才告别去了。

杜翁忽然从梦中醒来,自己的身子还倚在墙根。他到姓王的家里去询问,王家说果真有一头猪自己触墙而死。


\subsection{1.6.14   小 谢}
\label{\detokenize{p00_u5176_u5b83/_u767d_u8bdd_u804a_u658b_u5fd7_u5f02:id231}}
渭南姜部郎的宅子里,有很多鬼魅,经常出来迷惑人,姜部郎一家因此迁走了。留下个仆人看门,死了;连换了好几个,都死了。姜家只得废弃了这座宅子。

同村有个书生叫陶三望,一向倜傥不羁,好和妓女玩耍,但每次喝完酒就走了。朋友故意让妓女夜晚投到他门上,陶生笑着收留,实际上终夜也不沾染。曾有一次,陶生在姜部郎家住宿,有个丫鬟夜晚私奔了来,他坚决拒绝,始终不乱。姜部郎因此很看重他。陶生家里贫穷,又死了妻子,几间草房,酷暑天受不了闷热,便向姜部郎请求,要借住到他家的废宅子里去。部郎因为那座宅子太凶,不同意。陶生便作了一篇《续无鬼论》献给部郎,还说:“鬼有什么能为?”部郎因为他执意恳求,便答应了。

陶生到废宅子里打扫了厅房,傍晚,把书放下,回去取别的东西;回来一看,书却没有了。陶生很奇怪,仰面躺在床上,屏住呼息看有什么变故。过了一顿饭的工夫,听见有脚步声。陶生斜眼一瞅,见两个女子从屋里出来,把丢失的书送还到桌子上。一个约二十岁,另一个约十七八,都很艳丽。两个女子悄悄地走过来站在床下,相视而笑。陶生一动不动。那年长的女子翘起一只脚,踹了下陶生的肚子,年小的那个捂着嘴偷偷地笑起来。陶生觉得心神摇荡,像要控制不住自己,急忙收回杂念,始终不理她们。大女子便走近他,用左手拔他的胡须,右手轻轻地拍他的脸,发出很小的响声。年小的女子笑得更厉害了。陶生猛然起身,大喝道:“鬼东西竟敢这样!”两女子大吃一惊,跑散了。陶生恐怕夜里被她们扰乱受苦,想搬回去,又怕人说他言而无信,便起来点上灯读书。只见暗处鬼影恍惚,陶生全然不睬。快到半夜,陶生点着蜡烛睡下,刚闭眼,觉得有人用根细的东西捅自己的鼻孔,非常痒。大声打了个喷嚏,听见暗处隐隐有笑声。陶生也不说话,假装睡着了等待着。一会儿,见那个少女拿着根细纸捻,悄悄地摸了过来。陶生突然起身,大声呵斥,少女飘然窜掉了。睡下后,少女又来捅耳朵,折腾了一夜,没得安宁,鸡叫后,才寂静无声了。陶生才大睡了一觉。白天便没看见和听见什么。

太阳落山后,鬼影又恍惚出现。陶生便在夜晚做起饭来,打算一夜不睡,熬到明天早上。那大女子渐渐过来,把胳膊伏到案几上,看陶生读书;接着伸手把书合上了。陶生恼怒地去抓她,女子却飘散了。过了会儿,她又过来合上书。陶生便用手按着书读。那个少女偷偷地走到他身后,用两手一下捂住了他的眼睛,转眼跑开,远远地站着嘲笑他。陶生指着她骂道:“小鬼头!捉住便都杀了!”女子丝毫不怕。陶生便又戏弄说:“男女房中术,我一点不懂,缠我没用。”两女子微微一笑,返身走到灶边,一个劈柴,一个淘米,替陶生做起饭来。陶生夸奖道:“你们两人这样做,不胜过傻蹦乱跳许多吗?”一会儿,粥做熟了,两人又争着拿勺子、筷子、碗放到桌子上。陶生说: “感谢你们伺候,怎么报答?”女子笑着说:“饭中掺了砒霜、鸩毒了。”陶生说:“我和你们从无怨仇,怎至于给我下毒呢!”吃完,两女子又给盛上,争着跑来跑去的侍奉,陶生大乐。以后天天如此,习以为常了。渐渐熟悉后,对坐倾谈,陶生问她们姓名。大女子说:“我叫秋容,姓乔,她是阮家的小谢。”陶生又追问她们的来历。小谢笑着说:“痴郎!都不敢献出一次身子,谁要你问门第,想准备嫁娶吗?”陶生严肃地说道:“面对美人,怎会无情!但阴间的鬼气,人中了必定会死。你们不乐于和我住一起,走就是了;乐于住一起,就要安宁。如果你们不爱我,我何必玷污两位美人;如果爱我,你们又何必弄死一个狂生呢?”两女子互相看了一眼,像都被打动了。从此后,便不很耍弄陶生,但有时还把手伸到陶生怀里,或者扯下他的裤子扔在地上,陶生也不见怪。

一天,陶生抄书,还没抄完就出去了。回来后见小谢趴在桌子上,正拿着笔代抄,看见陶生,扔下笔斜瞅着他笑起来。陶生走近一看,虽然字写得太拙,但行列倒还整齐。便夸奖说:“你还是个很雅的人呢!如果喜欢这个,我可以教你。”说完,把她抱在怀里,把着手腕教她写字。秋容从外面进来,见此情景,脸色一下子变了,像是嫉妒。小谢笑着说:“小的时候曾跟父亲学写字。很久不写了,真像在梦里。”秋容也不说话。陶生明白她的意思,假装没有察觉,也抱起她来,给她支笔说:“我看看你能写字吗?”秋容写了几个,陶生就站起来说:“秋娘真好笔力!”秋容才高兴起来。陶生便折了两张纸,写上字,让她们临摹,自己在另一个灯下读书,心中暗喜两个人都有了事做,再不会捣乱了。临摹完,两女子都站在陶生的桌前,让他评阅。秋容从没读过书,写的字让人认不出来。评判完,她自觉不如小谢,脸上现出羞惭的样子。陶生夸奖劝慰了一番,秋容的脸色才放晴了。两女子从此后拿陶生当老师侍奉,陶生坐着就替他挠背,躺下就给他按摩大腿,不仅再不敢欺侮,还争着讨好陶生。过了一个月,小谢的字竟然写得很端正秀气,陶生偶然夸赞了一句,秋容立即很惭愧。眼泪汪汪,泪珠如线,陶生百般安慰劝解,才作罢。此后,陶生就教秋容读书,秋容非常聪明,教一遍,不用再问第二遍,和陶生争着读,常常彻夜不眠。小谢又带了她的弟弟三郎来,拜在陶生门下。三郎约十五六岁,姿容秀美,拿来一支金如意作为送给老师的见面礼。陶生让他和秋容同学一经,只听满屋咿咿呀呀的念书声,陶生在这里设起鬼塾来了。姜部郎听说后,很高兴,按时供给陶生薪水。过了几个月,秋容和三郎就都能作诗,还经常互相唱和。小谢暗地里嘱咐陶生不要教秋容,陶生答应了;秋容暗地里嘱咐他不要教小谢,陶生也答应了。

一天,陶生要去考试,两女子涕泪相送。三郎说:“先生可假托生病不去;不然,恐怕会有不吉利时事。”陶生觉得托病不考太耻辱,还是去了。原来,陶生常以诗词讽刺时事,得罪了本县的权贵们,这些人天天想中伤陶生。暗地里贿赂学使,诬告陶生行为不检,将他下到了狱中。陶生花费用光,只得向犯人们讨饭,自以为活不成了。忽然一人飘飘忽忽地走了进来,原来是秋容,她给陶生送了饭来,面对着陶生悲伤地哽咽道:“三郎担心你不吉利,现在果然不错。三郎和我一块来的,他已去官府申诉了。”说了几句话,秋容就走了,别的人都看不见她。第二天,巡抚大人出门,三郎拦路喊冤,巡抚便命带他走。秋容入狱把这消息告诉了陶生,然后又返回去探听,三天没回来。陶生又愁又饿,度日如年。忽然小谢来了,凄伤得要死,说:“秋容回去时,经过城隍祠,被西廊里的黑判官强摄了去,逼她作小妾。秋容不屈服,现在也被囚禁起来了。我跑了一百多里路,奔波得十分疲乏,到北郊时。被荆棘刺破了脚心,痛彻骨髓,恐怕不能再来了。”说着,伸出脚来让陶生看,只见鲜血淋漓,湿透了鞋袜。小谢拿出三两银子,一瘸一拐地走了。

巡抚提审三郎,问知他和陶生没一点瓜葛。无故替陶生告状,要杖打他,三郎扑地而灭。巡抚很惊异,看了看三郎的状子,情词悲恻。便提审陶生,问道:“三郎是什么人?”陶生假装不知。巡抚醒悟他被冤枉,释放了他。

陶生回来后,一晚上没有一个人来。到了深夜,小谢才来了,凄惨地说;“三郎在巡抚衙门被官衙的守护神押到了冥司。阎王因为三郎很义气,已让他投生到富贵人家。秋容被囚禁了这么久,我写了诉状投到城隍府,又被压下,递不上去,怎么办呢?”陶生忿怒地说:“黑老鬼怎敢这样!明天我去推倒他的塑像,踏为碎泥。再数落城隍之罪,骂他一顿。手下的官吏如此横暴,难道他在醉梦中吗!”两人相对坐着,悲愤不已。不觉四更将尽,秋容忽然飘然来了。陶生和小谢惊喜万分,急忙询问。秋容哭着说: “我为了你,受尽了千辛万苦!那个黑判官天天拿刀杖逼我,今晚忽然放我回来,说:‘我没别的意思,原是出于喜爱你。既然不愿意,我也不曾玷污你。烦你告诉陶秋曹,不要责备我。’”陶生听说,稍微高兴了些,想跟她们二人同床,说:“今天我愿意为了你们去死!”两女子凄伤地说:“一向受你开导,现在很知道些道理,怎忍心因为爱你而害死你呢?”执意不肯。三人亲热地抱在一起,感情如同夫妻。两个女子患难与共,互相嫉妒的念头早已消散了。

一次,一个道士在路上遇到陶生,看着他说:“你身上有鬼气。”陶生很惊异,详细对道士说了。道士说:“这两个鬼很好,不能辜负了她们。”便画了两道符,交给陶生,说:“回去给那两个鬼,看她们的运气,如听到门外有哭女儿的,吞下符立即出屋,先到的可以复活。”陶生道谢,接下符回去给了两个女子。过了一个多月,果然听见门外有哭女儿的,两女子争相奔出。小谢匆忙之中忘了吞符。见有辆丧车经过,秋容径直跑过去,进入棺材不见了。小谢进不去,痛哭着返了回来。陶生出去一看,原来是富户郝家为女儿出殡。众人都见一个女子进入棺材不见了,正在惊疑,忽听棺内有声音,歇下肩开棺一看,女子已经苏醒。于是,众人把棺暂时寄放在陶生的书房外面,围护着。郝某询问女儿,女子回答说:“我不是你女儿。”就把实情讲了一遍。郝某不太相信,想抬她回家。女子不听,径直奔入陶生的书房,躺在床上不起来。郝某便认了陶生为女婿走了。陶生走近女子端祥了一下。面貌虽然不一样,但艳丽不亚于秋容。陶生大喜过望,两人高兴地叙起往事。忽听有呜呜的鬼哭声,原来是小谢在暗处哭泣。陶生心中非常可怜她,便端着灯过去,宽慰她。小谢哭得衣衫全是泪水,悲痛不已,直到天明才走了。

天亮后,郝某派丫鬟、婆子送来嫁妆,居然和陶生真正成了翁婿了。晚上,陶生和秋容进入洞房,小谢又哭起来。这样过了六七夜,陶生夫妇都觉凄惨,竟不能同床。陶生十分忧虑,想不出办法。秋容说:“那个道士,真是仙人。你再去求他,或许他会同情相救。”陶生认为很对,访查到那道士住的地方。便去跪在地上哀求。道士极力说:“没办法!”陶生哀恳不已。道士笑着说:“痴书生真能缠人!合该和你有缘,我就竭力使出我的法术吧。”于是跟着陶生回到家中,要了一间静室,闭上门坐着,告诫陶生不要询问。过了十几天,道士不吃也不喝。陶生偷偷地往屋里瞅了瞅,道士像睡着了一样。一天早晨起来,有个少女掀开门帘进来,长得明眸浩齿,光艳照人,微笑着说:“奔跑了一夜,累死了!被你纠缠不休,跑到百里之外,才找到一个好躯壳,道人载着一起来了,等看见那人,便交给她。”到黄昏,小谢来了,女子突然迎上去抱住她,顿时合为一体,倒在地上僵死过去。道士从房中出来,拱拱手径自走了。陶生再拜。送走道士回来,见女子已经苏醒过来,扶她到床上,精神渐渐复原,只是握着脚说脚趾大腿酸疼,几天后才能起来。

后来,陶生科考得中。有个叫蔡子经的和他同榜,因为有事来拜访陶生,住了几天。小谢从邻居家回来,蔡子经看见她,急忙跑过去跟着细看。小谢侧身躲避,心里暗怒他太轻薄。蔡子经对陶生说:“有件令人非常骇异的事,能告诉你吗?”陶生询问,蔡子经回答说:“三年前,我的小妹去世,过了两夜尸体忽然不见了,到现在我还在疑虑。刚才看见尊夫人,怎么这样像我的小妹呢?”陶生笑着说,“我的妻子很丑,怎敢和你妹妹相比?但我们既然是同榜,交情又好,不妨让她见见你。”于是进入内室,让小谢穿上原来的葬服出来。蔡子经见了大惊说:“真是我的妹妹!”便哭起来。陶生对他详细讲了事情经过。蔡子经高兴地说,“妹子没死,我要尽快回家,告慰父母。”于是走了。过了几天,蔡家全家都来了。后来,就像郝家一样,与陶生来往密切。


\subsection{1.6.15   缢 鬼}
\label{\detokenize{p00_u5176_u5b83/_u767d_u8bdd_u804a_u658b_u5fd7_u5f02:id232}}
有个姓范的读书人,住在一家旅店里。晚饭后,他点着蜡烛,没解衣服卧在床上,还没入睡。忽然有一个侍女走进来,将包着衣物的包袱放到椅子上;还有梳妆镜匣和梳妆盒子,一样一样摆放在案头上,便离去了。

不多时,一个少妇从房间里出来,打开梳妆盒子和镜匣,对着镜子梳妆;一会儿梳理发髻,一会儿插戴头簪,又对着镜子前后左右仔细打量自己的身影。这样过了很长时间,那个侍女又来了,端了水来让少妇净面。少妇洗完之后,侍女又捧上手巾,等少妇擦拭完了,又把洗脸水端走了。少妇解开包袱,取出裙子、披肩,光灿灿的全是新缝制的,便穿在身上。又掩掩衣衿,提提衣领,挽结束扎十分周到。

范生看到这一切,没说话,心里却有些儿疑惑、惊讶,心想这一定是个私奔的女人,要打扮得漂漂亮亮去幽会。

少妇梳妆完了,取出一条长长的带子挂到粱上,并挽了个扣子。范生不禁一惊。只见她从容自若地抬起两只脚跟,伸长脖子要上吊;脖子才伸进扣子里,眼睛就闭上了,眉毛也直竖起来,舌头伸在嘴外面两寸多长,脸色变得悲惨像鬼似的。

范生吓得慌忙跑出门外,呼喊着告诉旅店的主人。店主人忙去察看,少妇已经无影无踪了。店主人说:“以前我的儿媳就是吊死在这里的,莫非就是她吗?”

咳,真希奇呵!人已经死了还重演她惨死的样子,这是什么道理呢?


\subsection{1.6.16   吴 门 画 工}
\label{\detokenize{p00_u5176_u5b83/_u767d_u8bdd_u804a_u658b_u5fd7_u5f02:id233}}
吴门有个画工,忘了他叫什么名字。喜欢画吕洞宾祖师的像。每次想像着吕祖的样子,他都感到心领神会。他很想有幸能见到吕祖,这个虔诚的念头凝结在心中,使他无时无刻不想着吕祖。

一天,画工正好遇到一群乞丐在城郊外喝酒。其中一人穿着破衣,露出了胳膊肘,但神采奕奕,气宇轩昂。画工心里一动,怀疑他就是吕祖。仔细端详了一下,越发觉得确实无误。于是他一下子抓住那人的胳膊说:“您是吕祖!”那乞丐大笑起来。画工执意说他就是吕祖,跪拜在地上不肯起来。乞丐说:“我真是吕祖,你又要怎样呢?”画工连连叩头,求他指教。乞丐说:“你能认出我,也算是有缘。但这里不是说话的地方,我们夜间再相会吧。”画工还想再问,转眼间,乞丐已消失得无影无踪。画工惊叹着回了家。

到了夜晚,画工果然梦见吕祖来了,对他说:“念你心意诚恳,我特来见见你。但你骨气贪吝,不能成仙。我让你见一个人好了!”说完向空中一招手,便有一个美丽的妇人凌空而下,衣着打扮像是皇宫中的贵妃。美丽的容貌,华贵的服饰,把屋子都照亮了。吕祖说:“这位是董娘娘,你要仔细看看记住了!”一会儿又问画工:“记住了吗?”画工说:“记住了!”吕祖再次嘱咐说:“不要忘了!”过了会儿,妇人离去,吕祖也走了。画工醒后,感到很奇怪,便把梦中见的那个妇人,回忆着画了幅像,珍藏起来,但终究不解是什么意思。

又过了几年,画工偶然去京城游玩,正赶上皇宫中的董妃去世了。皇上念董妃贤惠,要为她画张像以流传后世,便召集画匠们,皇上描述了一番董妃的模样,让他们想像着去画,但没一个画得像。这画工听说这件事后,忽然想起梦见的那个妇人,莫非就是董妃吗?便将自已原来画的那张像呈给皇上。皇宫中的人传看了一遍,都赞叹说画得惟妙惟肖。画工由此被封了中书官;但他不愿做官,皇上便赐给了他一万两银子。

从此,这位画工名声大噪。富贵大家都争着用重金聘请他,为自己先辈们画像。他只需凭空想像一阵,便无不画得形像逼真。只十多天,这画工便又挣了上万两银子。

莱芜的朱拱奎曾见过这个画工。


\subsection{1.6.17   林 氏}
\label{\detokenize{p00_u5176_u5b83/_u767d_u8bdd_u804a_u658b_u5fd7_u5f02:id234}}
济南有个叫戚安期的人,平时行为轻佻、放荡,喜欢嫖妓。妻子婉言劝说,他不听。他的妻子林氏,美丽而且贤惠。一次正遇上清兵进入济南,林氏被俘虏去了。晚上,清兵在半路上住宿,一个兵要奸污林氏,林氏假装答应了他。正好这个兵把佩刀挂在床头上,林氏急忙抽下刀来自刎而死,这个兵就把尸体抛在了荒野里。第二天,清兵便拔营离去了。

有人传说林氏已经死了,戚生很悲痛,赶到出事地点,一看林氏还有微弱的气息。他急忙背着妻子回到了家里,见她双目渐渐活动起来,又听到她有轻轻的呻吟声,便扶正她的脖子,用竹管一滴一滴给她灌下点汤水,还能够咽下去。戚生安慰妻子说:“你如果万一能活过来,我要背弃你就不得好死。”

过了半年,林氏恢复了健康,只是她的头受脖子伤疤的牵制,常像是往左看的样子。戚生也不因此感到妻子丑陋,对她的爱恋胜过往日,逛妓院的恶习也从此断绝。林氏自觉容貌丑陋,要给丈夫娶妾,戚生坚决不同意。

又过了几年,林氏仍没有生育,于是又劝说丈夫收下她的丫鬟。戚生说:“我已经发誓不再找第二个女人,鬼神难道听不见吗?即使没有男孩继承宗嗣,也是我命中注定。倘若不该绝后,难道你已经老到不能生育了吗?”林氏于是假托有病,让丈夫独自住在一室;打发丫鬟海棠抱着被子睡在戚生的床下伺倏。

过了很久,林氏私下问海棠夜里有什么情况,海棠说没有什么事。林氏不相信,到了夜里,告诫海棠不要去了,自己到海棠睡觉的地方躺下。过了一会儿,便听列床上响起鼾声。林氏悄悄起来爬到丈夫床上摸索他,戚生醒来忙问是谁?林氏凑到他的耳边低语说:“我是海棠。”戚生拒绝说:“我已经对夫人盟了誓,不敢再变心了。如果像往年那样,还用着你来找我吗?”林氏这才下床出去了。

戚生从此独自睡一处。林氏就吩咐海棠假装成自己去和丈夫同床。戚生心想妻子一生从不肯作不速之客,怀疑此事,便用手摸了摸她的脖子,没有伤疤,知道是海棠,又训斥了她。海棠含羞离去。

次日,戚生就把这件事告诉了林氏,让她赶快把海棠嫁出去。林氏笑着说:“你也不必过于固执,倘若能得到一个儿子,也就很幸运了。”戚生说:“如果背弃了盟约,鬼神就要给我惩罚,还指望延续宗嗣吗?”第二天,林氏笑着对丈夫说:“大凡农家人,种上庄稼,是否出苗吐穗不一定知道,不过通常播种的农时是不能误的。晚间耕耘的日期到了。”戚生欣然一笑,表示领会。晚上,林氏媳灭了灯,叫海棠来,让她卧在自己的被子里。戚生来了,上床取笑地说:“种地人来了。很惭愧我的工具都钝了,怕是辜负了这么好的农田。”海棠没有说话。接着戚生开始和她作爱。海棠小声说:“我这儿有些儿肿,用力大了受不了。”戚生也就倍加体贴温存行事。过后,海棠佯装起来小便,就让林氏代替了她。从此以后,海棠每当月经来潮,就与戚生同房,然而戚生却不知道。

过了不久,海棠怀了孕,林氏每天让她静坐休息,不再让她侍奉自己了。又故意对丈夫说:“我劝你收下海棠,你不听。假设哪一天海棠冒充我时,你如信以为真,同床后她怀了孕,该怎么办呢?”戚生说:“留下孩子,卖掉母亲。”林氏没再说什么。过了一段时间,海棠生了一个男孩。林氏暗中雇了个奶妈,把孩子抱到娘家寄养。过了四五年,海棠又生了一个男孩和一个女孩,长男名叫长生,已经七岁了,在外祖母家读书。林氏每半月就借口走娘家,去看孩子。

海棠年龄日益大了,戚生时时催促着把她打发走,林氏总是答应着。海棠天天想念孩子,林氏也就满足了她的愿望,暗地里给她梳起了少妇的发髻,把她送到了娘家。林氏对丈夫说:“你天天说我不愿嫁出海棠,我娘家有个义儿,已经把海棠许配给他。”

又过了几年,孩子都长大了。正遇戚生过生日,林氏事先忙着准备筵席,等候宾朋到来。戚生感叹地说:“岁月过得真快,不觉已经过了半辈子。幸运的是我们都很健康,家境也不至于挨冻受饿。所缺少的就是孩子。”林氏说:“你太执拗了,不听我的话,这怨谁?然而要想得到儿子,两个也不难,何况一个呢。”戚生笑着说:“既然说不难,明天就问你要两个儿子。”林氏说:“容易,太容易了!”

次日早起,林氏派了车马到娘家,把两个男孩,一个女孩打扮一新,一同坐车回到了家。走进家门,林氏叫三个孩子排成一行,齐声喊父亲,又给父亲叩头祝福长寿。跪拜完了起来,互相看着嘻笑。戚生诧异不解。林氏说:“你要两个儿子,我再添一个女儿。”这才给丈夫详细说了事情的前后经过。戚生非常高兴地说:“你为什么不早告诉我呢?”林氏说:“早告诉你,恐怕你赶走孩子的母亲。今天孩子已长成人了,还能赶她走吗?”戚生极为感激,热泪禁不住流下来。于是驾车亲自把海棠迎接了回来,和睦相处白头到老。古来贤惠的妻子像林氏这样,真可以说是德才出众的闺范了!


\subsection{1.6.18   胡 大 姑}
\label{\detokenize{p00_u5176_u5b83/_u767d_u8bdd_u804a_u658b_u5fd7_u5f02:id235}}
益都县的岳于九,家里有狐精作祟,衣物、器具常被扔到邻家的墙头上。他家积蓄了许多细葛布,要取来做衣服,见成捆的葛布还像原来的样子,可解开一看,中间空空的,全被剪去了。像这类恶作剧很多,令他们一家不堪忍受。一次,全家人七嘴八舌大骂恶狐,岳生忙告诫说:“这样乱骂怕让狐听见。”话音未落,狐在屋梁上说:“我已听见了。”从这以后,狐作祟更加厉害了。
有一天,岳生夫妇躺在床上还没起来,狐偷偷地摄走了他们的被子和衣服。两人都光着身子蹲在床上,仰望着空中可怜地向狐祷告。忽然看见一个很美的女郎从窗口里进来,把衣服撂到床头上。看这女郎身材不高,穿着深红色的衣服,外面套着雪白的背心。岳生忙穿上衣服,给女郎作揖,说:“上仙既然有意关照,就不要再来扰乱我们了。请你给我做个女儿,怎么样?”狐女说:“我的年龄比你还大,你怎能妄自尊大呢?”岳生又请求结为姊妹,狐女才应允了。于是岳生就叫家里人都称之为胡大姑。

当时,颜镇的张八公子家里也有狐居住在楼上,常常与人说话。岳生问狐女说:“你认识它吗?”狐女回答说:“是我家的喜姨,怎么不认识!”岳生又问:“那个喜姨从来不扰乱人,你为什么不效法它呢?”狐女不听,还像往常那样扰乱人。不过不怎么伤害别的人,只是专门祸害岳生的儿媳妇,常常把她的鞋袜、头簪和耳环等物抛弃在街道上;每当吃饭时,就在她的粥碗里埋进死鼠和粪便等脏物。岳生的儿媳也每每把饭碗扔掉大骂骚狐,并不祷告求饶。岳生只好祈求说:“儿女们都尊称你大姑,你怎么一点儿也没有长辈的样子呢?”狐女说:“要是教你儿子休了他的老婆,我做你的儿媳,就会相安无事了。”岳生的儿媳一听气得破口大骂:“浪狐真不害臊,竟想跟别人争汉子!”当时。岳生的儿媳正坐在衣箱上,忽然看见一股浓烟从自己屁股底下冒出来,烟熏火热像蒸笼似的。她急忙掀开衣箱一看,里面的衣裳已全被烧成了灰烬,剩下的一两件,全都是婆母的。狐女又叫岳生的儿子休掉他的妻子,他不答应。过了几天又催促他,仍然不答应。狐女恼怒了,就用石头打他,额头被打破了,因流血过多几乎送了命。岳生更加犯了愁。

西山李成爻善于用符水驱怪,岳生就用聘金把他请了来。李成爻用泥金在红绢上写符,三天才写完。又把镜子捆在棍子上,举着照遍了宅子里的每个角落。让童子跟在后面看着,如果看到什么东西,就赶快报告。照到一个地方,童子说墙上像是有只狗在卧着。李成爻即刻并起食指与中指,指划着把符写在墙上。随后,他在院子里走着巫步,念着咒语,不多时就见家里的狗和猪一起来了,耷拉着耳朵,夹着尾巴,像是听从命令似的。李成爻一挥手,说:“都走开!”它们随即纷纷排成队一个跟一个地走了。李成爻又念起咒来,一群鸭子立即来了,他又挥手把它们拘走了。一会儿,一群鸡又来了,李成爻指着其中一只大声呵叱。其它的鸡都走了,只有这只鸡独自趴在地上,交叉着翅膀长鸣,说:“我不敢了。”李成爻说:“这个东西是你们家制作的紫姑神偶。”全家人都说从来没制作过。李成爻说:“紫姑现在还在家里。”于是全家人都回忆起三年前,确曾制作过紫姑,玩过这种游戏,这次狐狸作怪的蹊跷事就是从那一天开始的。大家到处搜寻,见那紫姑偶形还放在牲口棚的梁上,李戚爻把它拿下来投进火里,又拿出一只酒瓶,念了三遗咒,又厉声喝斥了三次,那只鸡从地上起来迳直走了。这时,听到酒瓶口说道:“岳四真狠呵,几年后,定当再来!”岳生请求李成爻把酒瓶投进热汤或火里,李不同意,带走了。

有人见李成爻家里的墙上挂着十几只瓶子,堵塞着口的都装着狐,说他逐个放了它们出去胡作非为,他却依靠这种办法获取聘金,把这些狐当成奇货。


\subsection{1.6.19   细 侯}
\label{\detokenize{p00_u5176_u5b83/_u767d_u8bdd_u804a_u658b_u5fd7_u5f02:id236}}
浙江省昌化县的满生,在本省余杭县设私垫教书。他偶然到街市上去。路经一家靠街的阁楼下,忽然有一只荔枝壳坠落在肩头上。他抬头一看,见一个少女倚在阁搂的栏干上,姿色艳丽,俊俏极了,不由得双目注视着她,像发了狂似的。那少女低头微笑着进了阁楼门里。满生一打听,才知道她是妓院鸨母贾氏的女儿细侯。细侯的名声与身价很高,满生知道很难实现自己的心愿。

满生返回书斋后,左思右想,整夜不能入睡。到了明天,他到贾氏妓院,送上名帖,与细侯见了面。两人说说笑笑,非常快乐,满生更加被少女迷住。他便借口有事向同人们借贷,凑了若干银子,又带着去见细侯,两人相亲相爱极为融洽。满生就在枕头上作一绝句赠给细侯道:“膏腻铜盘夜未央,床头小语麝兰香,新鬟明日重妆凤,无复行云梦楚王。”细侯听了忧愁不安地说:“我虽然污秽低贱,却想得到一位真心爱我的人敬奉他。你既然没有妻子,看我能给你当家吗?”满生大为喜悦,立即再三叮嘱,两人山盟海誓,订下终身。细侯很高兴地说:“作诗之类的事,我自认为不难,每当在无人的地方。也想仿效着作一首,又恐怕作得不好,让人听了看了讥笑。倘若能跟你在一起,希望你能指教我。”于是又问满生家有多少土地和房子。满生回答说:“有薄田五十亩,破屋几间罢了。”细侯说:“我嫁给你以后,咱们一定要天天在一起,不要再外出教书了。四十亩地将就可以自给自足。十亩地可用来种些黍子,再织五疋绢,在太平年间交纳赋税还有余呢。这样,我们可以闭门相对,你读书,我织绢,闲暇日子作诗饮酒消遣,那千户侯又有什么可贵的!”满生说:“你的身价大约值多少呢?”细侯说:“以母亲的贪心,哪能满足她?至多不过二百两银子就足够了。可悔恨的是我年轻,不知道重视钱财,得到银子总是交给母亲,自己的积蓄寥寥无几。你能筹集到一百两银子就好了,如超过这个数就更不必顾虑了。”满生说:“像我这样落寞,你是知道的。一百两银子怎么能自己办到?我有个曾经一起盟过誓的至友在湖南当知县,几次要我到他那里,我因为路途遥远,怕行路艰难没去。今天为了你,我定当前去找他想办法筹措银子。估计三四个月就可以回来,希望您能耐心等侯。”细侯答应了。

满生立即放弃了教学,去湖南觅友。到了那里,那县令已被免了官职,正居住在民房里,钱袋空虚,无法馈赠他。满生穷困失意,难以返回余杭,就只好在这个县里教书度日。过了三年,仍然不能回家。有一次,满生偶然用戒尺打了学生,这个学生投水自杀了。学生家长痛惜孩子,就控告了老师,满生因此被捕入狱。幸亏有其他的学生可怜老师没有过错,时常给他送去衣食等物,因而没有特别受苦。

细侯自从与满生相别之后,闭门不接待任何客人。贾母问知缘故,又没法强迫她改变主意,也就只好听任她了。有个富商,久慕细侯的名声,便托媒人致意贾氏,定要把细侯娶到手,不惜代价。细侯不同意。富商因为经商到湖南去,仔细地侦探满生的消息。这时,监狱将要释放满生,富商便用银钱买通了主管犯人的官吏,让他永久禁锢满生。富商回来告诉鸨母说:“满生已在监狱里关死了。”细侯怀疑富商的口信不一定确实。贾氏说:“不用说满生已经死了;即使不死,与其跟着穷酸过一辈子苦日子,哪如跟富商穿绫罗绸缎、吃山珍海味呢?”细侯说:“满生虽然贫穷,可是他的人品却很高尚;要跟着个肮脏商人,我实在不心甘情愿。况且那种路人的传言,怎么能轻信呢!”

富商又别生一计,叮嘱另外一个商人假作了一篇满生绝命书寄给细侯,以断绝细侯的希望。细侯收到了这份假绝命书,从早到晚地哀哭。贾氏说:“我从小对你抚养教育,实在辛苦。你长大成人才只二三年,能得到你报恩的日子也不多了。你既然不愿意隶属乐籍当妓女,又不同意出嫁,这样下去以什么谋求生活呢?”细侯不得已,就嫁给了那个富商。富商供给细侯的衣服、簪环极为丰富侈华。过了一年多,细侯生了一个男孩。

不久,满生得到学生的鼎力相助,获得昭雪,被释放出狱。这时,他才知道原来是那个富商搞阴谋把他禁锢在狱中的,可又想与富商素日并无仇恨,反复思考也不明白究竟为了什么。学生们都自动拿出钱资助他作路费,回到了家。当他听说细侯已经出嫁的消息,心情十分激动难过,于是就把自己的苦情,托市上卖浆的老妇转达给细侯。细侯得知此情非常悲伤,这才明白以前那些所谓满生已死的种种说法,都是富商搞的阴谋诡计。她趁富商到外地去,杀了怀抱中的孩子,携带着东西投奔满生去了,凡是富商家的衣物首饰,一件也不带走。富商回家后,愤怒地告到官府里。官府经过调查,弄清了事情的真相,把富商的状子搁置起来不予审理。

呵,这与三国时关云长从曹营毅然回归蜀汉,又有什么不同?不过细侯竟然杀了自己的儿子出走,也实在是天下的狠心人了!


\subsection{1.6.20   狼 三 则}
\label{\detokenize{p00_u5176_u5b83/_u767d_u8bdd_u804a_u658b_u5fd7_u5f02:id237}}
有个杀猪的人,卖肉回来,天已经黑了。忽然来了一只狼,看到担中的肉,好像垂涎三尺。杀猪人走,狼也走,尾随了好几里路。杀猪人害怕了,拿出刀来吓唬狼,狼就稍微后退几步;杀猪人再往前走,狼又跟着。杀猪人没有办法,心想狼想要的是担中的肉,不如暂时将肉挂到树上,明天一早再来拿。于是便用铁钩钩住肉,翘着脚挂到树叉上,又把担子让狼看看以示空了,狼才不再追他了。杀猪人就直接回家了。天刚放亮时,杀猪人去拿肉,远远看到树上悬挂着一个很大的东西,好像人吊死的样子,杀猪人很害情,小心翼翼地靠近一看,原来是只死狼。他抬头仔细查看,见狼口中含着肉,肉钩子刺在狼的上腭中,好像鱼吞了鱼饵一样。当时狼皮价格非常贵,能卖到十两银子,杀猪人因此发了一点小财。

一个杀猪人晚上回家,担子里的肉已经卖光了,只剩下一些骨头,路上遇到两只狼,在后面跟着他走了很远。杀猪人害怕了,扔出根骨头。一只狼得到骨头停住了,另一只狼仍然跟着。他又扔了一根骨头,这只狼停下了,可那只狼又来了。骨头已经扔光了,两只狼仍然像原先那样跟着他。杀猪人非常窘迫,恐怕被两只狼前后攻击。看到田野中有一片麦场,场主在场上堆积了一些柴草,用草苫遮盖着,同小山丘一样。杀猪人跑过去,倚在柴垛旁,放下担子拿起杀猪刀,狼不敢再向前走,只是虎视耽眈地盯着他。不多时,有一只狼径自离开了,另一只狼像狗一样蹲在面前,时间长了,狼的眼睛眯缝着,像睡着了一样,显出十分悠闲的样子。杀猪人乘其不备,突然跳起来,一刀劈中狼头,又砍了数刀,才把狼杀死。他正想走,转身看见柴草垛后面。另一只狼正在挖洞,想钻到他后面攻击他。狼的身子已经钻进一半,只剩下屁股和尾巴露着。杀猪人从后面砍断狼的腿,这只狼也被杀死了。杀猪人这才明白,前面的狼假装瞌睡,是以此迷惑他。狼也是很狡猾的,然而顷刻间两只狼都被杀死,禽兽的欺诈手段能有多少呢?只是给人们增添一些笑话罢了。

有一个杀猪人,傍晚赶路,被狼追逼着。见路旁有个农夫搭起的供夜耕用的草棚,便急忙跑进去趴下。狼从草苫中伸进一只爪子,杀猪人急忙捉住,不让它抽回去,但却没有办法杀死它。见身上只有一把不到一寸长的小刀,他便用刀子割破狼爪下的皮,用吹猪的办法吹狼。他用尽力气吹了一会儿,感到狼不再动了,才用带子绑住口。杀猪人出来一看,狼胀得像牛一样,大腿直挺挺伸着不能弯曲,嘴也张着合不起来,杀猪人就把狼背回家了。


\subsection{1.6.21   美 人 首}
\label{\detokenize{p00_u5176_u5b83/_u767d_u8bdd_u804a_u658b_u5fd7_u5f02:id238}}
有几个商人一同寓居在京城一家房舍中。房舍和邻屋相连,中间隔着一层木板壁;板上有个松节脱落了,洞穴像杯子大小。忽然有个女子从板壁洞穴中把头伸了过来,挽着凤髻,美丽无比;随即伸过一条手臂来,洁白如玉。众人害怕她是妖怪,想捉住她,但她已缩了回去。一会儿,美人头又露出来,只是隔着板壁看不见她的身子。等到人跑过去,美人头就又缩回去了。

有一个商人持刀藏到板壁下。顷刻间美人头又伸了过来,商人突然用刀砍去。美人头随刀而落,血溅满地。众商人惊告主人。主人害怕,带着美人头去告了官。官府逮捕了这些商人并审问他们,认为这事很荒唐。把他们羁押了半年,终究没问出符合犯事事实的供词来,也没人因人命来告状,这才释放了商人,掩埋了这个美人头。


\subsection{1.6.22   刘 亮 采}
\label{\detokenize{p00_u5176_u5b83/_u767d_u8bdd_u804a_u658b_u5fd7_u5f02:id239}}
听济南怀利仁说:历城的刘亮采公,是狐仙的后身。起初,他的父亲刘翁住在南山,有个老叟到他家拜访,自称姓胡。刘翁问他住在什么地方,胡叟说:“就在此山中。这里清闲人少,只有您和我两人,可以早晚相聚在一起,因此来拜识您。”刘翁于是和他交谈,见他言词意趣敏捷,很喜欢他。摆上酒菜欢饮,直到喝醉了才走。过了一天胡叟又来,两人的交情越加诚挚深厚。刘翁说:“自从与您相交,情谊就非常深厚。只是不知您住在什么地方,到哪里去给您请安问好呢?”胡叟说: “不敢隐瞒,我实是山中的老狐,和您有前世的缘分,因此敢到您门下相交。固然不能使您有福,但也不敢使您有祸,希望您相信我,不要害怕。”刘翁也不怀疑,对他更加敬重。就叙起年龄,胡叟为兄,往来犹如兄弟。即使是有小的吉凶事,胡叟也来告诉刘翁。

当时刘翁没有儿子,胡叟忽然说:“您不用忧愁,我定当作您的后人。”刘翁对这种奇怪的说法感到很惊讶。胡叟说:“我算着自己的寿数已尽,眼看到了去投生的日期了。与其投身到别人家里去,哪如生在故人家?”刘翁说:“您仙寿万年,怎么竟然到了这种地步?”胡叟摇头说:“这些事不是您所能知道的”。于是走了。到了夜里刘翁果然梦见胡叟来,说:“我现在已来到家了。”刘翁醒来,夫人生了个男孩,这就是刘亮采公。

刘公后来长大成人。身材很短,言词敏捷诙谐,很像胡叟。他从小就有才名,万历壬辰年成了进士。刘公为人好打抱不平,急人所急,因此秦、楚、燕、赵等地的客人,都进出于他的家门。哪些卖酒卖饭的人也都集中到这里来,家门前竟成了个市场。


\subsection{1.6.23   蕙 芳}
\label{\detokenize{p00_u5176_u5b83/_u767d_u8bdd_u804a_u658b_u5fd7_u5f02:id240}}
马二混,住在青州东门内,以卖面为生。家里穷,没有妻子,同老娘一起生活。有一天,老娘一人在家,忽然来了个十六七岁的美丽少女,虽然穿着简朴,但容貌光艳照人。老娘惊奇地看着她,追问她是什么人。那女子笑着说:“我因为你儿子老实本分,愿意给你家做媳妇。”老娘更加吃惊地说:“姑娘长得像仙女,有你这一句话,就折我们娘俩几年寿了!”女子再三请求,老娘认为她一定是大户人家逃出来的,更极力拒绝,女子只好走了。过了三天,那女子又来了,住下不走了。老娘问她姓什么,女子说:“老娘如果能留下我,我才说。不然的话,你就不用问了。”老娘说:“我们人穷命苦,娶你这样漂亮的媳妇,不般配,也不吉利。”女子笑着坐在床头上,舍不得离开。老娘催她说:“姑娘快走吧,别给我家惹祸!”女子这才走了。老娘看着她是往西去了。

又过了几天,西边巷子的吕妈妈来,对老娘说:“我邻家有个叫董蕙芳的姑娘,独自一人无依无靠,愿意给你儿子做媳妇,你怎么不收留她呢?”老娘把自己的顾虑告诉了吕妈妈,吕妈妈说:“哪有这种事!如果有差错,都包在我身上!”老娘听说十分高兴,就答应了。吕妈妈走了以后,老娘扫屋铺床,等儿子回来去娶亲。天刚黑,女子一个人飘然来了。进门参拜老娘,礼节周全。告诉老娘说:“儿媳有两个丫鬟,没得母亲的允许,不敢让她们来。”老娘说:“我们母子两人守着个穷家,不懂得使唤丫鬟。我们每天赚很少一点钱,只够填饱肚子。如今添了新媳妇,娇嫩坐吃,还怕吃不饱;再添两个丫鬟,难道让她们喝风过日子啊?”女子笑着说:“丫鬟来了不用母亲养活,她们能自己干活挣饭吃。”老娘问:“丫鬟在哪里?”女子便喊:“秋月、秋松!”声音未落,两个丫鬟就像飞鸟一样落在面前。女子叫她们跪拜母亲。不长时间,马二混回来了。老娘迎上去告诉他家里发生的事,马二混很高兴。进屋一看,家里布置得像宫殿一样,雕梁画栋,桌椅床帐,耀眼夺目。二混吃惊得很,不敢往里走。女子下床笑着迎了过来,马二混见她像仙女一样,更加吃惊,连忙后退。女子拉住他,坐下和他亲切地说话。马二混高兴得出乎意外,不知怎么好了。一会儿,他起来想出去打酒,女子说:“不用了。”就叫两个丫鬟准备酒菜。秋月拿出一个皮口袋,拿到门后格格地摇了一阵,然后伸进手去一样一样往外拿,壶里有酒,盘里有菜,样样都是热气腾腾。吃完饭,二人上床睡觉,毛毯被褥又光滑又暖和。天亮以后,二混出门一看,还是原来的破草房。母子俩都很奇怪。马大娘就去吕妈妈家,想察问那女子的来历。到了吕妈妈家,先谢她作媒的情意。吕妈妈惊讶地说:“我已经很久不去你家了,哪里有邻家女子托我作媒的事呢?”马大娘起了疑心,就把事情的原委说给吕妈妈听。吕妈妈大惊,跟着马大娘去看新媳妇。女子笑着迎出来,再三道谢吕妈妈作媒的好意。吕妈妈见慧芳艳丽聪明,惊愕了许久,也就不再辩解,只好应是。女子送给吕妈妈一把白木做的搔子,说:“我没有什么东西谢你,就送你这把搔子抓痒吧。”吕妈妈拿回家仔细看,搔子变成银的了。

马二混自从娶了妻子,就不卖面了。家里门户一新,箱中貂裘锦衣无数,任凭马二混穿用。但一出门,就变成朴素的布衣了,但仍然又轻又暖。蕙芳自己所穿的衣服也是这样。这样过了四五年,蕙芳忽然说:“我被罚降人间已经十几年了,因为和你有缘分,才暂时留在这里。如今我要走了!”马二混苦苦相留,蕙芳说:“请你另娶妻子,好延续香烟。我以后还会来的。”说完就忽然不见了。马二混便娶了秦家的女儿作妻子。过了三年,七月初七那晚,二混夫妻两人正在说话,蕙芳突然来了,笑着说:“新夫妇多么快乐,不想念故人吗?”马二混慌忙起来,非常悲伤地拉她坐下,诉说思念之情。蕙芳说:“我刚送织女过了天河,抽空来看看你。”两人恋恋不舍,说不完的知心话。忽然听空中有人喊:“蕙芳!”蕙芳急忙起身告别。马二混问是谁喊,蕙芳说:“我是和双成姐姐一块来的,她等得不耐烦了!”马二混出去送她,蕙芳说: “你能活到八十岁,到那时,我来安排你入土。”说完就不见了。马二混现在已六十多岁了。此人除了老实厚道外,也没有别的长处。


\subsection{1.6.24   山 神}
\label{\detokenize{p00_u5176_u5b83/_u767d_u8bdd_u804a_u658b_u5fd7_u5f02:id241}}
益都县的李会斗,偶然到山上去,遇到几个人坐在地上饮酒。他们见李生来到,都很高兴地嚷着站起来,把李生拉入座内,竞相为他敬酒。看那些盘子里的菜肴,陈列着很多珍馐美味。过了一会儿,大家喝得非常高兴。只是酒味太淡而且苦涩。

忽然远远地来了一个人,脸又窄又长,大约有二三尺的样子,帽子的高矮粗细和脸孔很相称。众人惊慌地说:“山神来了!”立即纷纷四散。李生也伏身藏匿在深坑中。过了不久起来一看,菜肴和酒全没了,只有破陶器中积存的尿液,还有瓦片上盛着的几条蜥蜴罢了。


\subsection{1.6.25   萧 七}
\label{\detokenize{p00_u5176_u5b83/_u767d_u8bdd_u804a_u658b_u5fd7_u5f02:id242}}
徐继长,是临淄县人,家住在城东的磨房庄。他读书没取得功名。就到官府做了小吏。一次偶然去看亲戚,经过于家的坟地。傍晚,徐继长酒醉回家,仍路过那片坟场,见到路边一片楼阁瓦舍,十分繁华富丽,有一老汉坐在门口。徐继长喝了许多酒,很口渴,想水喝,就向老汉行礼,讨点米汤。老汉站起身来,请他进去,到堂屋里给他拿水。徐继长喝完后,老汉说:“天已晚了,路不好走,暂且住一夜,明天早晨再走,怎样?”徐继长也感到疲乏困倦,就很乐意地答应了。老汉让家里人准备酒菜待客,又对徐继长说:“老夫有句话,请您不要怪我莽撞。您门风清白,威仪令人仰望,我们可以结成姻亲。我有个小女儿还没有出嫁,想给您做侍妾,希望能攀附上您。”徐继长又恭敬又不安,不知说什么才好。老汉便派人告诉了自己的亲戚和本家,又传话让他的女儿梳妆打扮。

过了一会儿,四五个高冠宽带的人,先后来到。那女郎艳妆而出,容貌俏丽,举世无双。于是大家入席喝酒。徐继长精神迷乱,只想快快睡觉。他喝了几杯后,就坚决推辞,再也不肯喝。老汉就让小丫鬟领着徐继长夫妻进了洞房,尽享新婚之乐。徐继长问少女的家族姓氏。少女说:“姓萧,排行第七。”徐继长又仔细询问她的门第,少女说;“我虽然出身低下,但嫁给你做小吏的也不算辱没你,为什么苦苦追问根底?”徐继长溺爱她的美貌,竭力地亲昵温存,再也不怀疑了。少女说:“这地方不能为家。我知道你家大姐很和善,或许不会阻拦。你回家打扫出一间房子,我自已就会去。”徐继长口里应着随即搂住少女,一会就睡了。

一觉醒来,怀里已经空空的了。天色也已大亮,松树遮住了日光,身下垫的谷穰有一尺来厚。徐继长惊恐地回到家,把这事告诉了妻子。妻子耍笑他,就打扫出一间房子,在里边安了一张床,关上门,出来说:“新娘子今夜就来了。”两人都笑。到了傍晚,妻子嘲弄地拉着徐继长开了房门说:“新娘子是不是已在屋里了?”进去以后,就看到一位美女穿得很华丽地坐在床上。她看见徐继长夫妻进来,连忙起身相迎。夫妻二人非常惊奇,美女却捂着嘴吃吃地笑,恭敬地行了礼。徐继长妻子就整治了酒菜。让他们饮合欢酒。七姐每天很早就起来做家务,不用别人指派。

一天,七姐对徐继长说: “我姐姐们想来咱家看一看。”徐继长担心仓促间没有好东西待客。七姐说:“她们都知道咱家不富裕,会先送些菜肴和炊具来,只麻烦我家姐姐做一做罢了。”徐继长告诉了妻子,他妻子同意了。早饭后,果然有人挑了酒肉来,放下担子就走了。徐妻就当了厨师。午后,来了六七位女子,年纪大点的四十来岁,她们围着桌子坐下一起喝酒,谈笑声充满了房间。徐妻趴在窗户上偷偷一看,只看见丈夫和七姐对面坐着,别的客人却看不见。她们一直玩到很晚,才高兴地离去。七姐送客还没回来,徐妻进屋看一看桌子上,杯盘都光光的,就笑道:“这些丫头想必是都饿坏了,就像狗舔的一样干净。”不多时,七姐送客回来,殷勤地向徐妻道劳,夺过杯盘去洗,并催促徐妻去睡。徐妻说:“客人来到我们家,让她们自带酒莱,也太笑话了。明日应该再请一次。”

过了几天,徐继长按照妻子的话,让七姐再请客人来。客人到了以后,尽情地吃喝,唯独留下了四碗菜没动筷子。徐继长问为什么,她们笑着说:“夫人认为我们太没出息,所以留下给她吃。”席间有一女子,大约十八九岁,七姐称她为六姐,白鞋子白衣服,说是才死了丈夫,但神情妖冶艳丽,很能说笑。她和徐继长渐渐融洽起来,就用诙谐的话相互挑逗。行酒令时,徐继长做令主,禁止说笑话。六姐违反了好几次,接连喝了十几杯,面红大醉,娇美的身子没有力气,软弱的难以支持。不久,她就躲开了。徐继长拿着蜡烛去找她,却见她已藏进帐子里睡熟了。徐继长近前去吻她,她也不觉得。徐继长伸手到她下衣里摸了摸,不禁神魂摇动。忽听酒席上乱喊徐郎,便急忙理好六姐的衣服,见她袖子里有一块绫巾,偷拿起来出了帐子。

到了半夜,客人们都离了座,六姐还没醒来。七姐进去摇晃她,她才打着呵欠起来,系好裙子,梳理好头发,跟着大家回去。徐继长心里念念不忘,想到没人处展玩偷来的绫巾,但找时已经不见。他怀疑是送客时丢在路上了,就端着灯仔细地照台阶,却没找到,心里很不自在。七姐问他,他随便答应着。七姐说;“你不要骗我了,那手巾人家已拿去,白费心思。”徐继长很惊讶,便如实告诉了她,并说很想六姐。七姐说:“她和你没有宿缘,就这么一点缘分罢了。”徐继长问其中的原因。七姐说:“她的前身是个妓女,你是读书人。你见了她后很爱她,但被你的父母所阻拦,愿望得不到实现,因此患了重病,生命垂危。你让人告诉她说:‘我的病已没法医治了,假若你能来,哪怕只让我摸一下你的身体,我死了也不遗憾!’她被你的痴情所感动,就答应了你的请求。恰巧她被杂事缠身,没有立即去;第二天去,你已经死了。这是她的前世和你只有摸一下的缘分。超过了这个界限,就不是你所能得到的了。”以后再摆酒招待那些女眷,只有六姐不来。徐继长怀疑七姐嫉妒,很有怨言。

一天,七姐对徐继长说: “你因为六姐的缘故,胡乱责怪我。她实在是不肯来,跟我有什么相干?现在我们八年的情爱,就要分手了。让我尽力为你谋划一下,以解除你以前的迷惑。她虽然不肯来,难道能挡住我们不去?我们上门去找她,或许能人力胜过天意,也未可知。”徐继长十分高兴,随着她前往。七姐握住徐继长的手,飘然离地,很快到了她家。只见黄砖大厅,重门曲折,和第一次见到时没有区别。岳父和岳母一起迎出来,说:“我女儿多年来承蒙你的照顾,我们年高懒惰,很少去探望,你不会责怪我们吧?”立即摆酒举行宴会。七姐便问姐姐们的情况。她母亲回答说:“她们都各自回家去了,只有六姐还在这里。”随即喊丫鬟请六姐出来见客,很久还不出来。七姐进去,把她拉了出来。六姐低头不语,不像从前那样有说有笑。一会儿,父母告辞走开。七姐对六姐说:“姐姐自命清高,让人家怨恨我!”六姐微微冷笑说: “轻薄之人不宜和他亲近!”七姐端起两人的酒杯,强迫他们交换喝下,说:“都已经亲吻过了,为什么还要作态?”不多时,七姐也走开了,屋里只留下两个人。徐继长突然起身逼她,六姐兜着圈子躲闪撑拒。徐继长拉住她的衣服跪在地上苦苦哀求,六姐的脸色渐渐平和起来,两人手拉手进了里间。刚刚解开衣扣,忽听到外边叫喊声震天动地,火光照亮了房门。六姐大惊,忙推开徐继长说:“灾祸临头了,怎么办?”徐继长仓促间不知怎么做才好,而六姐已经逃窜没了踪影。徐继长惆怅地坐了一会,房屋也全不见了。这时有十几个猎人架着鹰拿着刀来到跟前,吃惊地问:“你是什么人?怎么半夜里坐在这地方?”徐继长推托迷了路,说了自己的姓名。一个猎人说;“刚才我们追赶一只狐跑到这里来了,你见到没有?”徐继长回答说:“没见到。”仔细辨认那地方,原来是于家坟地,便很不高兴地回了家。此后,他非常希望七姐再来,早晨盼着喜鹊叫,晚上盼着灯花爆,然而最终也没有消息。这个故事是董玉玹讲述的。


\subsection{1.6.26   乱 离 二 则}
\label{\detokenize{p00_u5176_u5b83/_u767d_u8bdd_u804a_u658b_u5fd7_u5f02:id243}}
学师刘芳辉,是京都人。他有个妹妹许聘给戴生,出嫁的日期眼看就到了。遇上清兵入境,父兄恐怕她这样一个细弱女子成为负担,打算把她妆扮好送到戴家。还没妆饰完,清兵纷纷而入,父子分头逃奔。刘女被清兵的小头目俘虏而去。跟随了好几天,小头目对她绝无不庄重的行为,夜晚就睡在别的床上,对她的饮食照顾非常周到。后来又掳掠了一个年轻人来,年纪和刘女差不多,容貌秀美仪态风雅。小头目对他说:“我没有儿子,想让你来继承家世,愿意吗?”年轻人答应了。又指着刘女对他说:“如果愿意的活,就让她作你的妻子。”年轻人很高兴,愿意按他说的办。小头目于是让年轻人和刘女睡在一起,二人感情融洽,非常快乐。随后在枕上各自说出姓氏,原来年轻人就是刘女的未婚夫戴生。

陕西某公,任职盐官,因家室累赘没带到任上。遇上姜瓖据城抗清的事变,家乡成了他们聚集的地方,某公和家庭的音信便隔绝了。后来事变平息,某公派人探问,而百里以内人烟断绝,无处可以打听消息。恰遇某公进京向朝廷述职,身边有个老差役死了妻子,家贫不能续娶,某公便给他几两银子让他去买个妻子。当时清兵凯旋而归,俘获了无数妇女,插上草标押到市场上,像牛马一样地卖。老差役携带银子到市场上去选购女人,他自知钱少,不敢问年轻女人的价钱。见其中有个老年妇女很整洁,就拿银子赎买回来。老妇人坐在床上,仔细地认了认说:“你不是某差役吗?”问她是怎么知道的,她回答说:“你跟随我的儿子服役,怎么不认识!”差役大惊,急忙告诉某公。某公过去一看,果然是自己的母亲,因而痛哭,加倍偿赐了差役。老差役因为银子多了,不愿意再买年老妇女,见一妇人年纪三十多岁,风度仪容超脱不俗,就赎买了她。往回走时,妇人一边走一边看他,说:“你不是某差役吗?”差役又惊问她。她回答说:“你跟随我的丈夫服役,怎能不认识!”差役更加惊奇,领着她去见某公。某公一看,果真是他的夫人,又悲痛失声。一天当中母亲、妻子重新和他团聚,高兴得不得了。于是用一百两银子为老差役娶了一个美貌的妻子。看来必定是某公有大德,因此鬼神被他感动并报答了他。可惜说这事的人忘了此公的姓名,秦中或许还有能说出他的姓名的人。


\subsection{1.6.27   豢 蛇}
\label{\detokenize{p00_u5176_u5b83/_u767d_u8bdd_u804a_u658b_u5fd7_u5f02:id244}}
山东泗水县的山中,早先有座佛寺,四周没有村庄,很少有人到这里来,有一个道士便住在这座寺院里。有人说寺里有很多大蛇,所以游人更远远地躲着这里。

有一个少年进山用网逮鹰,一直走到山的深处。天晚了,远远看到有座寺院,便前去投宿。道士惊讶地说:“居士从哪里来?幸好没被我那些孩子们看见。”让他坐下,拿粥饭给他吃。还没吃完,一条大蛇爬进来,足有十多抱粗,昂头看着客人,愤怒的目光像闪电一样一闪一闪的。少年大吃一惊,恐惧万分。道士用手掌拍拍蛇的头,呵斥说: “去!”大蛇就低下头爬进东屋里,弯弯曲曲爬了好一会儿,身子才全进去,盘伏在屋里,一间东屋全塞满了。少年人更加害怕,浑身打颤。道士说:“这蛇是我平时豢养的,有我在这里,不要紧。怕的是你自己遇到它。”少年刚坐下,又一条蛇进来,比前一条略小一点,约有五六抱粗。看见客人立即停住了,怒目闪闪,吐着舌信子,像前一条一样。道士又呵斥它,这条蛇也进了室内。东屋里没有它卧的地方了,它就一半身子绕在梁上,墙壁上的土被哗哗地摇落下来。少年更害怕,整夜睡不着,早早就起来想回去。道士送他,出了屋门,只见墙上、台阶下,到处都是蛇,大如盆粗、酒杯粗,爬着的、卧着的,种种不一。蛇看到生人,都露出要吞吃的样子。少年害怕,依偎着道士的胳膊跟他走,一直让道士送出山口,少年才自己回去。

我乡里有些客居中州的人,寄宿在蛇佛寺中。寺里的僧人准备了晚饭,肉汤很鲜美,而且肉段都是圆的,形状像鸡脖子。客人疑惑地问寺僧:“杀了多少鸡,能有这么多的脖子?”僧人说:“这是蛇肉段。”客人大惊,有的跑出门去呕吐。客人们睡下后,觉得胸膛上有东西爬动,用手一摸,是蛇!顿时吓得叫喊着爬起来。僧人起来说:“这是平平常常的事,有什么可怕的!”说着用火把照照墙壁,只见大大小小的蛇满墙都是,床上床下也是蛇。第二天,僧人领着客人们来到佛殿,见佛座下有一口大井,井里的蛇有瓮粗,把头探出井边,却不出来;点上火把向井下看,里面蛇子蛇孙数百万条,都簇拥在井中。僧人说:“过去蛇从井里出来祸害人,自从修了佛像坐在上面把它们镇住以后,它们才不敢出来为害了。”


\subsection{1.6.28   雷 公}
\label{\detokenize{p00_u5176_u5b83/_u767d_u8bdd_u804a_u658b_u5fd7_u5f02:id245}}
安徽亳州人王从简,他的母亲坐在屋里,遇到小雨天,夜色昏暗,看见雷公手持大槌,闪动着翅膀飞进屋来。她吓得不得了,急忙端起便盆把尿液泼向雷公。雷公身上沾了污秽,好像被刀斧砍中,反身快逃,但翻来复去地跳跃,就是走不了。最后跌倒在院中,吼声如牛。这时天上的乌云慢慢低下来,渐渐和屋檐齐平。云中有叫声像马嘶鸣一样,和雷公相互应和。一会儿,大雨猛然倾泻下来,雷公身上的脏东西全被冲洗干净,这才打了个霹雳升空而去。


\subsection{1.6.29   菱 角}
\label{\detokenize{p00_u5176_u5b83/_u767d_u8bdd_u804a_u658b_u5fd7_u5f02:id246}}
有个叫胡大成的,是楚地人。他的母亲素来信奉佛教。大成跟随着塾师读书,去私塾的路上经过观音祠,他的母亲嘱咐他每次路过一定进去叩拜观音。这一天,大成走进祠庙,看见有个少女领着一个小孩在里面游玩。少女的头发才掩住脖颈,但风致却非常美好。这年大成十四岁,心里对她产生了好感。于是问她的姓氏,那少女笑着说:“我是祠西焦画工的女儿菱角,你问我有什么事吗?”大成又问:“你有婆家了吗?”少女羞红了脸,说道:“没有。”大成说:“我做你的丈夫,好吗?” 少女羞惭地说:“我不能作主。”说话间目光晶莹含情,偷偷地上下打量大成,看起来好像欣然同意的样子。大成走出祠,少女追过去远远地告诉她:“崔乐诚是我父亲的朋友,请他作媒人,事情没有不成功的。”大成说:“好。”想到菱角聪慧多情,心中更加爱慕她。回到家里,向母亲表白了心愿。母亲只有这一个儿子,总怕违背他的心意,就赶忙央求崔乐诚作媒。焦父要聘礼很多,婚事差点没有说成。崔乐诚极力夸耀大成是清白人家,人才出众,焦父这才答应。

大成有个伯父,年老无子,在湖北担任教官。伯母在当地病逝后,母亲让大成去湖北奔丧。过了数月,大成将要返回时,伯父又病了,不久也去世了。大成已经停留了很长时间,适逢强盗占据湖南,与家中信息隔断,流浪到民间,孤立无依,惶惶不可终日。这一天,有个四十八九岁的妇女,在村中绕来绕去。太阳西斜也不走。她自我介绍:“我和亲人离散了,没有办法回家,要把自己卖掉。”有人问她的价钱,她说:“我不屑于作别人的奴仆,也不愿成为别人的妻子,但只要有把我当作母亲的,我就随他,不计较价钱。”周围听的人都嘲笑她。大成走近细看,女人眉目间有一二分很像他的母亲,触动心怀悲伤不已。他想自己孤单一人,连缝缝补补的人也没有,于是邀请妇人回家,以儿子的礼节对待她。妇人大喜,便替大成做饭织鞋,辛苦劳累,就像母亲一样。若大成违背了她的心愿就责备他,但大成稍有点疾苦时,却体恤爱护胜过了亲生儿子。有一天,妇人忽然对大成说:“这里太平,幸而没有什么可担忧的事。然而你年龄大了,虽然流落在外,但伦常大道不可偏废,再过两三天,应当为你娶亲。”大成落泪了,说:“儿子已经有媳妇了,只是阻隔在南北两地不能成亲。”老妇说:“大乱时期,人事皆非,为什么还要像守株待兔那样空自等待呢?” 大成又哭着说:“且不说结发的盟约不敢违背;又有谁家愿意把娇贵的女儿嫁给我这像浮萍一样漂泊不定的人呢?”妇人不回答,只是帮助整治窗帘、帷幔、被子、枕头等,并且准备得很周全,也不知她从哪里弄来的。

一天,太阳已经西落,妇人嘱咐大成:“点着蜡烛坐着,不要睡觉,我去看一看新娘子来了没有。”于是走出家门。已经过了三更,妇人还没有回来,大成非常疑虑。过了一会儿听到屋外有喧哗声。走出一看,见一女子坐在庭院中,头发蓬乱,正在哭泣。大成惊问:“你是谁?”她也不回答。过了好一会儿,才说:“把我娶来,肯定没有福分,我只有寻死!”大成大惊,不知道其中的原因。女子说:“我年少时受聘于胡大成,没料到他到湖北去,音信断绝。父母强迫我嫁到你家。身子可以强得到,但志向不可改变!”大成闻听哭道:“我就是胡大成,你是菱角吗?”女子停住哭泣,非常惊异。但又不相信是真的。两人互相拉着走进屋内,在灯下认真细看,说道:“莫非这是一场梦?”于是转悲为喜,互相诉说离别的痛苦。

起初战乱发生后,湖南百里内,荒无人烟,鸡犬不闻。焦画工携带全家流落到长沙东面,又接受了周生的订亲聘礼。战乱中不能成亲,约好今晚送菱角到周生家。菱角大哭,不肯梳妆,家里人强行把她推入车中。到了中途,菱角颠落车下。于是有四个人带着轿子赶到,自称是周家迎亲的,立即把菱角扶到轿中,快走如飞,到了这里才停下。一个妇人把菱角带进来,说:“这就是你的夫家,只管进去不要哭泣。你家婆婆明晚就会赶到。”说完离去。大成问知实情,才醒悟妇人是神人。夫妻二人焚香共同祈祷,希望母子能重新团聚。

大成的母亲自从战事起后,和同乡妇女一起奔到涧谷中。藏身一夜,有人鼓噪说强盗来了,大成母亲于是和众人惊慌地四处躲藏。这时有个童子把骑的马交给大成母亲,大成母亲焦急中顾不得细问,扶着童子的肩膀上了马。马跑起来轻灵神速,转眼间到了湖上,马踏水奔腾,蹄下不起波浪。不久,童子把大成母亲扶下来,指着一处房子说:“这里面可以居住。”大成母亲才要张口感谢,回头见那匹马化作金毛犼,有一丈多高,童子跳上飞驰而去。大成母亲用手敲门,门豁然一下自动打开。有个人从里面走出询问,大成母奇怪声音这么耳熟,仔细一看,原来是大成。母子俩抱头痛哭。菱角也被惊起,一家人重逢非常欢慰。猜测那个妇人是观音化身。从此吟观音经更加虔诚。大成一家人于是客居在湖北,买田盖屋,过起了日子。


\subsection{1.6.30   饿 鬼}
\label{\detokenize{p00_u5176_u5b83/_u767d_u8bdd_u804a_u658b_u5fd7_u5f02:id247}}
有个叫马永的,是齐地人。为人贪婪,是个无赖,家底终于被耗尽了,同乡人戏称他为“饿鬼”。到三十多岁时,日子更加艰难,衣服破烂不堪,常常两手交叉着搭在肩上,在集市上偷拿食物吃。人们都厌弃他,对他不屑一顾。

同乡有个朱姓老头,年少时携带家眷住在繁华都市,干着不正当的行业。晚年回到家乡,被士人大加非议。但朱氏修正品行,广做善事,人们开始稍有礼貌地对待他。一天,正赶上马永拿食物吃不给钱,店铺里的人不依不饶。朱氏可怜他,替他付了钱,把马永领回家,赠给他数百钱作本钱。马永拿去后,不肯自谋职业,坐吃老本。不久,本钱花光了,又重蹈旧辙。他惧怕和朱氏相遇,于是逃到临邑。夜晚住在学宫中,冬夜寒气袭人,马永就摘下圣人塑像冕冠上的玉串,烧了冕板取暖。学官知道后,大怒,要用大刑。马永苦苦哀求赦免,愿意为学官积蓄钱财。学官大喜,把他放走了。马永探得某书生家财殷富,便登门强行勒索钱财,故意挑动那人大怒,然后马永用刀自伤,诬陷那书生伤人,把他告到学官。学官勒索得重赂,才没把那书生除名。其他学生都很愤恨,共同告到县官那里。县官探访到实情,打了马永四十大棍,用枷锁锁住他的脖子。过了三天,马永就死在狱中。

这一夜,朱氏梦见马永穿戴整齐地进来,说:“我辜负了您的大恩大德,今晚来报答。”梦醒了,恰逢妾生了个儿子。朱氏知是马永转世,就给孩子起名叫马儿。马儿自小就不聪明,可喜的是他还能读书。二十岁时,朱氏竭力活动,才让马儿进了县学。后来,马儿去考试时,住在旅店里,白天卧在床上,见墙壁上糊满着旧时的八股文章,仔细一看,有《犬之性》的题目,心里感到很难作,就读完这篇文章并且记住了。进入考场,考的恰好是这个试题,马儿便把记忆的抄录下来,得了个优等,考中了。直到六十岁时,马儿才补了个临邑训导。做官数年,没有一个道义之交。要是从袖子里拿出钱给他,他就露出贪婪的笑容;否则,眼皮一耷拉,摆出一副威严的样子。好像不认识。县官偶而判令他对小有过失的学生进行轻罚,他就残酷掠夺,如同惩治盗贼。有要起诉学生的,就来叩门送礼。这样多次,学生们都不堪忍受。马儿年近七十,体态臃肿又聋又瞎,常常向人们索取能使白须变黑的药物。有个狂放的学生,折了茜根来哄骗他。天亮了,大家一看,马儿的胡须成了红色,就像庙中灵官塑像的模样。马儿大怒,要拘捕这个学生,而这个学生已经在前一夜晚逃走了。因为这个缘故,马儿恨气郁结,过了数月就死了。


\subsection{1.6.31   考 弊 司}
\label{\detokenize{p00_u5176_u5b83/_u767d_u8bdd_u804a_u658b_u5fd7_u5f02:id248}}
闻人生,是河南人。有一次,他生病卧床,躺了一整天,见一个秀才走进来,跪在床下拜见,非常谦恭有礼。既而秀才又请他出去走走,一路上秀才拉着他的胳膊,边走边絮絮叨叨地说个不停。一直走了几里路,还不告别。闻人生站住脚,拱拱手要告辞。秀才说:“请您再走儿步,我有一件事求您!”闻人生问他什么事,秀才说:“我们一些人都归‘考弊司’管辖。‘考弊司’的司主名叫‘虚肚鬼王’,凡初次拜见他的人,按照旧例,都要从大腿上割下一块肉献给他。我想求您去给讲讲情,饶了我们!”闻人生惊讶地问:“犯了什么罪至于受这种刑罚?”秀才回答说:“不必犯罪,这是‘考弊司’的老规矩。如果给鬼王送重礼,才能免了;但我们都太穷了,送不起礼!”闻人生说:“我和那鬼王素不相识,怎能为你效力呢?”秀才说:“您的前世是鬼王的爷爷辈,他应该听您的话。”

二人正说着,已走进一座城市,来到一个衙门前。见官衙的房屋建筑不很宽敞,只有一间厅堂又高又大。堂下东西两边立着两块石碑,上面刻着斗大的字,涂着绿色。一个刻的是“孝悌忠信”,另一个刻的是“礼义廉耻”。二人登上石阶,又见大堂上方悬挂着一块匾,上书大字“考弊司”。大堂柱子上,挂着一副板雕绿字的对联,上联是:“曰校、曰序、曰痒,两字德行阴教化,”下联是:“上士、中士、下士,一堂礼乐鬼门生。”两人还没看完,一个官员从里边走了出来。见那官头发卷曲,腰背弓着,像有几百岁的样子,一对鼻孔朝天,短短的嘴唇翻开着,露出一嘴獠牙利齿。随从的一个师爷,人身上却长着颗虎脑袋。又有十几个人在两边排列伺候,大半都狰狞凶恶,像是山精山怪。秀才对闻人生说:“那就是鬼王。”闻人生早吓得魂飞魄散,返身想走。鬼王却已看见他,忙从台阶上走下来,恭敬地行礼,将闻人生请进了大堂,又问候他的日常起居,闻人生只吓得连连说“是”。鬼王问他:“有什么事来到这里?”闻人生便把秀才求自己的事说了。鬼王一听勃然变色,说:“这是有旧例的,就是我亲爹来讲情,我也不敢听从!”说完,面如冰霜,像是一句人情话也听不进去。闻人生不敢再说别的,急忙站起身告辞。鬼王又侧着身子,恭恭敬敬地把他送到大门外才回去。

闻人生出门后不往回走,又返身偷偷走进来,想看看那鬼王到底要干什么。来到大堂下,只见那秀才和另外几个人都已被绳索反绑起来,一个面目凶恶的人拿着一把刀子走过来,先脱下秀才的裤子,然后从大腿上一刀割下一片三指宽的肉来。秀才疼得大声号叫,把嗓子都喊破了。闻人生年轻气盛,见此情景,怒不可遏,大喊道:“如此惨毒,成何世界!”鬼王吃了一惊。从座上站起来,命暂停割肉,自己从座椅上下来,迎接闻人生。闻人生已气忿忿地走了出去,遍告路人,要去上帝那里控告。有人讥笑他说: “真愚蠢啊!蓝天茫茫,到哪里去找上帝申诉冤屈?这些鬼跟阎王倒挺近,到阎王那里上告,或许还管点用!”便指给他路。闻人生沿路赶去,一会儿来到阎王殿,见气象十分威严,阎王正在大殿上坐着。闻人生跪在台阶下,大声喊冤。阎王叫上他来询问清楚,立即命鬼率拿着绳索提着锤子去捉鬼王来。过了不久,鬼王和秀才一起被拿来,阎王审知闻人生说的都是实情,大怒,斥骂鬼王说:“我可怜你生前一生苦读,所以暂时委给你这个重任,等候让你投生到富贵大家去。你现在却敢如此无法无天!我要剔去你身上的‘善筋’,再给你添上‘恶骨’,罚你生生世世永远不得做官!”一个鬼卒便上前,将鬼王一锤子打翻在地,连门牙也碰掉了。鬼卒又用刀割破鬼王的指尖,抽出一条又白又亮、像丝线一样的筋来,鬼王痛得杀猪般地大声嗥叫。直到把他手上、脚上的筋都抽完,才有两个鬼卒押着他走了。

闻人生给阎王磕了头,便退出了阎王殿。秀才在后面跟着,对闻人生很是感激,挽着他的胳膊,送他走过街市。闻人生看见有个人家,门口挂着红门帘,帘后有个女子,露出了半张脸,模样非常艳丽。闻人生问:“这是谁家?”秀才回答说:“这是妓院。”已经走过去后,闻人生对那女子留恋不舍,于是坚决不让秀才再送。秀才说:“您是为我的事来的,却让您一人孤孤单单地回去,我怎么忍心呢?”闻人生坚决告辞,秀才只好离去。闻人生见秀才走远,急忙回身走进那家妓院。那女子立即出来迎接他,面现喜色。进入室内,女子让闻人生坐下,互相说了姓名。女子自称姓柳,小名叫秋华。这时一个老婆子出来,为他们准备下酒菜。喝完酒,二人上床,极尽欢爱,山盟海誓地订下了婚约。天亮后,老婆子进来说:“没钱买柴买米了,无奈只得破费郎君几个钱了!”闻人生顿时想起腰包里空空的,没带钱,惶恐惭愧地一语不发。过了很久,才说:“我实在没带一文钱,我给你们立下字据,回去后立即偿还。”老婆子一下子变了脸,说:“你听说过有妓女外出讨债的吗?”柳秋华也皱着眉头,一句话不说。闻人生只好脱下外衣,当作抵押。老婆子接过衣服,讥笑说:“这件东西还不够偿还酒钱的!”嘴里絮絮叨叨的,一副很不满意的样子,跟那女子进了内室。闻人生非常羞惭。又过了会儿,闻人生还在盼望着女子出来和他道别,再重申订下的婚约,等了很久,寂无声息,闻人生便暗暗进去察看,见老婆子和柳秋华自肩部以上都变成了牛头鬼,目光闪闪地相对而立。闻人生大惊,急忙返身逃了出来。他想回家,可是岔路极多,不知走哪条路好。询问街市上的人,并没有知道他的村名的。闻人生在街上徘徊了两天两夜,辛酸悲伤,加上饥肠辘辘,真是进退两难。忽然那个秀才从这里经过,看见闻人生,惊讶地说:“你怎么还没回去,却这样狼狈?”闻人生红着脸不好意思回答。秀才说:“我知道了,你莫不是被花夜叉迷住了吧?”说完,秀才便气昂昂地往那家妓院走去,说:“秋华母女怎么这样不给人留面子?”过了一会儿,秀才就把衣服抱来交给闻人生说:“那淫婢太无礼,我已经叱骂过她了!”秀才把闻人生一直送到家后,才告辞走了。这时,闻人生已突然死了三天,此刻才苏醒过来,说起阴间的经历,还记得清清楚楚。


\subsection{1.6.32   阎 罗}
\label{\detokenize{p00_u5176_u5b83/_u767d_u8bdd_u804a_u658b_u5fd7_u5f02:id249}}
沂州的徐星,自已说夜里当了阎罗王。这个州里还有个马生,也说自已夜里当了阎罗王。徐星听说后,就到马家去拜访,问马生昨晚阴间处理过什么事。马生说:“没有别的事,只是送左萝石升天。天上掉下莲花来,花朵和房屋一样大。”


\subsection{1.6.33   大 人}
\label{\detokenize{p00_u5176_u5b83/_u767d_u8bdd_u804a_u658b_u5fd7_u5f02:id250}}
长山县孝廉李质君,一次去青州府时,路上遇到六七个人,听口音像是燕地人,细看他们两颊,都有瘢痕,像铜钱大小。李质君很惊异,就问他们为什么都得一样的病。客人说:“前几年我们客居云南时,一天天黑后迷失了道路,走进了大山里。绝壁悬崖,崇山峻岭,找不到路出来。见山谷中有一棵大树,几尺长的树条,软绵绵地下垂着,树荫足有一亩地大。我们无计可施,便都拴住马,解下行李,依傍在这棵大树下休息。夜深了,老虎、豹子、猫头鹰,此起彼伏地嗥叫着,我们都抱着腿面面相觑,不敢睡下。忽然看见一个大人走来,有一丈多高。我们吓得团团趴在地上,大气不敢出。那个大人来到跟前,用手抓住马就吃,六七匹马顷刻就被他吃光了。接着他又折下树上的长条,抓住我们的脑袋,像串鱼那佯,把我们串成一串。提着走了几步,枝条发出脆弱断裂的声音。大人好像怕坠落到地上,就把枝条弯曲起来,用一块巨大的石头压住枝条两端走了。我们觉得他走远了,便拿出佩刀砍断串着的枝条,忍着疼痛急忙逃跑。刚跑几步,见大人又领着一个人来了。我们害怕万分,忙趴下藏在草丛中。看见领来的那个人更加高大,来到树下,往来巡视,好像找什么东西又找不到。一会儿,这人发出喊叫声,好像大鸟在叫,很生气的样子,怨恨大人欺哄它,用手掌打他的耳光。大人弯着身子,很恭敬地挨着打,不敢有一点争执。不一会儿,二人都走了,我们才仓惶逃出来。

在荒野中逃了很久,远远看见山岭上边有灯光。大伙一块跑过去,到了一看,是一间石屋,有个男人住在里面。我们进去,都拜见了他,并且讲了刚才受的苦。男子拉起我们,让我们坐下。说:“这东西特别可恨,但是我也没有办法制服它。等我妹妹回来,再同她商量。”不多会儿,一个女子挑着两只老虎从外边进来,问客人是从哪里来的。我们忙伏地叩头,告诉她来由。女子说:‘早就知道这两个东西为害,没料到凶顽到这程度!应当马上除掉它。’说着,从石屋中拿出一柄铜锤,重三四百斤,然后出门不见了。那男人便煮老虎肉招待我们。肉还没熟,女子已经回来了,说:‘他们看见我想逃,我追了数十里路。打断了他一个指头就回来了。’说着把指头扔到地上,见那截指头有人的腿骨那么粗。众人惊骇极了,问她的姓名,女子不答。稍过了一会儿,虎肉熟了,我们腮上的创伤疼得我们不能吃东西。女子便用药屑给我们涂抹伤处,疼痛立刻止住了。天明了,女子送我们来到那棵大树下,行李都在。我们各自背起行李走了十多里路,经过昨天夜里争斗的地方,女子指给我们看,石洼中残留的血迹还有一盆多。一直送我们出了山,女子才告别返回山中。


\subsection{1.6.34   向 杲}
\label{\detokenize{p00_u5176_u5b83/_u767d_u8bdd_u804a_u658b_u5fd7_u5f02:id251}}
向杲,字初旦,是太原人。他与庶母所生的哥哥向晟友情最为敦厚。向晟结交了一位妓女,名叫波斯,与向晟有割臂为誓终生永好的盟约。困为波斯的鸨母要的价钱太高,两人始终没有如愿。正好鸨母想要从良,愿意先把波斯嫁出去。有一个姓庄的公子,一向喜欢波斯,向鸨母请求买下波斯做妾。波斯对鸨母说:“既然母亲和我愿脱离这地狱而登入天堂,如果把我卖给别人做妾,与当妓女又有什么区别!如果肯依从我的志向,只有向晟才合我的意。”鸨母答应了,并把她的意思转告向晟。当时向晟的妻子死了,尚未再娶,他非常高兴,竭尽家中所有的钱财,把波斯娶了回家。庄公子听说后,恼怒向晟夺走了他喜欢的女人。一次在路上偶然碰到向晟,便大骂一顿。向晟不服,庄公子就叫随从毒打向晟,打得快要死了,他们才走。向杲听到消息急忙跑去看,他的哥哥已经死了。

向杲不胜哀痛悲愤,写好了状子到郡城去告状。庄公子对上下官府都行了贿,使他有理得不到伸张。向杲心中愤怒郁结,没有地方控告诉说,只想着要在路上刺杀庄公子。每天揣着锋利的刀,伏在山间路旁的草丛里。时间长了,机密逐渐泄露出去。庄公子知道了他的打算,每出门就戒备森严。听说汾州有个叫焦桐的人,很勇敢而且擅长射箭,庄公子便用很多钱把他聘来做护卫。向杲没有办法实旋他的计划,但还是每天在路边等着。有一天,他刚刚藏好,忽然下起了倾盆大雨,全身上下都湿透了,冻得打颤,颇吃苦头,不一会狂风四起,又下起了冰雹。向杲身上忽然没有了知觉,不知痛痒。山岭上以前有座山神庙,他强支撑着跑到那里。进了庙以后,就看见一个他认识的道士在那里。从前,这个道士曾经在村里讨饭,向杲经常给他饭吃,因此道士也认识向杲。道士见向杲的衣服都湿透了,就给他一件布袍,说:“暂且把这件布袍换上。”向杲换上布袍,忍着寒冷,像狗一样蹲着。自己看着身上,忽然长出了皮毛。身子变成了老虎。道士已不知道哪里去了。向果心中既吃惊又愤恨。可转念一想,这样能找到仇人而吃他的肉,办法也不错。就下山到原来藏身的地方。看见自己的尸体趴在草丛中,才明白自己的前身已经死了。他还恐怕自己的身子被乌鸦和老鹰吃了,时时巡回守护着它。

过了一天,庄公子才从这里经过,老虎猛然窜出,把庄公子从马上扑落下来,咬下庄公子的脑袋,吞了下去。焦桐掉转马头,向老虎射了一箭,射中老虎的肚子,老虎倒下接着就死了。向杲在荆棘丛中,恍然好像一场大梦初醒。又过了一个晚上,才能行动走路,昏昏沉沉地回到家里。家里人因为他一连几晚上不回来,正在惊骇疑虑,见到他,都高兴地来安慰探问他。向杲只是躺着,反应迟钝不能说话。过了一会儿,家人听说了庄公子被虎咬死的消息,争着到床头高兴地告诉他,向杲才自己说:“老虎就是我。” 接着就讲述了他奇异的经过。这事从此传播出去。庄公子的儿子悲痛父亲死得太惨,听说以后很恼火,就去告了向杲。官府认为这件事很荒诞,而且没有证据,置于一边不予理睬。


\subsection{1.6.35   董 公 子}
\label{\detokenize{p00_u5176_u5b83/_u767d_u8bdd_u804a_u658b_u5fd7_u5f02:id252}}
青州的董可畏尚书,家里的规矩很严厉,内宅外宅的男人和女人,不敢互相说一句话。一天,有个丫鬟和男仆在中门以外调笑,公子看见便怒叱了他们,两人各自奔散。

到了夜晚,公子和书僮睡在书房中。当时正是盛夏,房门大敞着。夜深的时候,书僮听到床上有剧烈的声响,被惊醒了。在月光下,见白天和丫鬟调笑的那个仆人提着一样东西出门走了。因为他是家里的仆人,书憧也没多疑,就又睡了。忽然听见有大声走路的靴子声,一个魁伟的男子,红脸长须,好像汉寿亭侯关羽的模样,手里提着一颗人头走了进来。书僮非常害怕,便像蛇那样钻入床底下。听到床上吱吱咯略的响,像是抖衣服,又像按摩腹部,过了一会才完事。靴子声重又晌起来,那人就出去了。书僮伸着脖子慢慢从床底下出来,见窗棂上有了晨光。用手摸到床上,沾了一手湿漉漉的东西,闻了闻有血腥气味,吓得他大呼公子,公子正好醒来。书僮便把见到的情形告诉他,并拿火来照,一看血满枕席。公子和书僮大惊,却不知道这是怎么回事。

忽然有官府的差役前来敲门,公子出门接见,差役很惊讶,嘴里直说怪事。公子追问,差役告诉说:“刚才衙门前有一个人神色迷乱恍惚,大声说:‘我杀了主人了!’众人见他衣服上有血污,便抓住他告了官。经过审问,知道是您家的仆人。他说已经杀了您,把人头埋到关帝庙旁边了。我们前往那里去验证,看到挖的坑土还很新,但却没有人头。” 公子大为惊异,便赶往公堂,见那犯人就是先前和丫鬟调笑的仆人。于是叙述了事情的奇异经过。官听了非常怀疑惶恐不安,便狠狠地打了仆人一顿后释放了他。公子不想和小人结下仇怨,就把那个丫鬟许配给这个仆人,让他们回家去了。

过了几天,这个仆人的邻居夜间听到他房子里一声震响,像是什么东西崩裂了,急忙起来喊他,没人应声。撞开门进去一看,见仆人夫妇连同他们睡的床,都截然而断,成了两半,木头和肉体上全有刀削的痕迹,好像是一刀砍断的。

关公显灵的事迹非常多,但是没有比这件事更神奇的了。


\subsection{1.6.36   周 三}
\label{\detokenize{p00_u5176_u5b83/_u767d_u8bdd_u804a_u658b_u5fd7_u5f02:id253}}
泰安州的张太华,是个很富裕的州吏。他家里有狐骚扰,虽多次驱赶、遏止,也不起作用。他把这事说给知州听,知州也无能为力。当时州的东面也有狐居住在村民家里,人们都看见过狐是一个白发老头。这老头和村里人互通礼仪往来,如同常人一样。他自称排行第二,人都叫他胡二爷。恰巧有个秀才来拜见知州,谈话间提到了胡二爷的奇异。知州便为张吏出谋,让他前去问胡叟。那时胡叟住的村子里有个在州衙当差的人,张吏向他打听胡叟的情况,果然不假,于是和他一同前往,就在衙役家里设筵请胡叟。胡叟来到,礼让敬酒,和常人没有不同的地方。张吏便把请求驱狐的事告诉了胡叟。胡叟说:“我本来很清楚地知道这回事,只是不能为您效力。我的朋友周三,寓居在岱庙,他能降伏它们,我定当代您求他。”张吏大喜,再三致谢。胡叟临走时和张吏约好,让他明天在岱庙的东面设筵等待。张吏都答应照办。

第二天,胡叟果然领着周三来到约定的地点。周三的脸像铁一样,上面长满卷曲的胡须,穿一身骑马服装。酒过数巡,周三对张吏说:“刚才胡二弟把您的意思告诉了我,事情已经知道得很详细了。只是此辈确实有很多同伙,不可好言相告,难免动用武力。请允许我就借居在您家,有什么吩咐也在所不辞。”张吏转念一想,去掉一狐,再来一孤,是用凶暴换凶暴,因而迟疑不决,没敢立即答应。周三已知道了他的心思,说:“不用害怕,我和那些狐不一样,而且和您还有同住一起的缘分,请勿怀疑。”张吏答应了他。周三又嘱咐他明日和全家人一起关上门坐在屋子里,不要喧哗。

张吏回到家中,全都遵照周三的吩咐安排好了。不久便听到院子里有攻击刺斗的声响,过了一个时辰才静下来。开门出来一看,鲜血点点洒满台阶,台阶上有好几个小狐狸头,像碗、杯大小。又去看为周三清扫准备的房间,见他端坐在里面,拱手笑着说:“蒙您重托,妖类已全部消灭了。”周三从此住在张家,相见如宾主一般。


\subsection{1.6.37   鸽 异}
\label{\detokenize{p00_u5176_u5b83/_u767d_u8bdd_u804a_u658b_u5fd7_u5f02:id254}}
鸽子种类繁多。山西有“坤星”,山东有“鹤秀”,贵州有“腋蝶”,河南一带有“翻跳”,吴越一带有“诸尖”,这都是品种出色的上好鸽子。另外有靴头、点子、大白、黑石、夫妇雀、花狗眼等,名类繁多,数不胜数,只有玩鸽内行的人,才能辨识清楚。

邹平县有位张幼量公子,特别喜好鸽子。他按照《鸽经》上所列的名堂,四处搜求,力求搜寻到天下所有品种。他养鸽子,如同养育婴儿一样,天冷了,就用甘草粉给鸽子疗护;天热了,就给鸽子吃点盐粒。鸽子好睡觉,但睡得太多了,容易得麻木症死掉。张公子在扬州花十两银子买到一只鸽子,身材最小,很喜欢走动,把它放到地上,盘旋着走动,没有停止的时候,不到死不会停下来。所以,平日常常需要人把着它。夜间,便把它放到鸽群中,使它惊动其它鸽子,可以防止麻痹病。这种鸽子,人们叫它“夜游”。山东一带养鸽子的行家,以张公子家为最著称,张公子也常以善养鸽子,自我夸耀。

一天夜晚,张公子独坐在书斋中,忽然一位身着白衣的少年叩门进来。张公子一看,素不相识,问他是什么人,回答说:“四处漂泊的人,姓名有什么可说的?听传闻说公子蓄养的鸽子最多,这是我生平中最爱好的,愿意观赏您养的鸽子。”张公子就把自己所蓄养的鸽子,全都展示出来,各种颜色的鸽子都有,五光十色璀璨如锦。少年笑着说:“人传说的真不假啊!公子真可称得上包罗天下名鸽的人了。我也养有一两头,公子愿意观赏吧?”张公子听罢很高兴,就跟着少年去了。

月色朦胧,旷野中显出一片萧条的景象,张公子心里有些怀疑畏惧。少年向前指着说:“请再走一段路,我的住处就在前边不远。”又走了几步,见一座道院,院内仅有两间屋子。少年拉着张公子的手走了进去,院里暗淡,没有灯火。少年站立在院子的中央,口里学着鸽子的叫声。忽然有两只鸽子飞了出来,形状如同平常的鸽子,但身上的羽毛纯白,飞到房檐那么高,边叫边斗,每次相扑,必定翻筋头。少年一挥胳膊,两只鸽子一齐飞去了。少年又紧撮起嘴唇,发出一种奇异的声音,又有两只鸽子飞出来,大的如同鸭子大,小的才如拳头;两只鸽子并立在台阶上,学着仙鹤起舞。大的伸长脖颈,张开两只翅膀,作孔雀开屏的样子,旋转着边叫边跳,好像在引着小鸽子;小鸽子上下飞鸣着,时而飞到大鸽子的头顶上,翅翼翩翩,如同燕子飞落在蒲叶上,声音细碎,如同敲击小鼓;大的伸长脖颈不敢动。叫的声音越急,声音就变得如同磐石一般清脆悦耳。两只鸽子鸣叫相合,相互间杂,很合节拍。接着,小鸽子飞起来,大鸽子就上下摆动着逗引它。张公子赞赏不已,感到自己的鸽子委实比不上,望洋兴叹。

张公子向少年行礼,请求少年能够割爱。少年不同意,张公子又恳切地乞求。少年让两只舞蹈的鸽子飞去后,又学着以前唤鸽的声音,招两只白鸽来,伸手捉住,对张公子说:“若不嫌弃,就把这两只白鸽送给您,聊以塞责。”张公子把两只白鸽接到手,细心地观看着,只见白鸽两只眼睛在月光映照下,呈现琥珀色,两眼通明透亮,好像中间没有间隔一样,中间的黑眼珠,圆如花椒粒。掀起鸽子的翅膀看,肋间的肌肉,如同晶莹的水晶,五脏六腑都看得清楚。张公子感到很奇异,但还是觉得不满足,乞求少年再送给他几只。少年说:“我还有两种未敢奉献,现在不敢再请您观赏了。”两人刚在争执间,家人点着麻杆火把来找主人。张公子回头看少年,已化为一只白鸽,大如鸡,冲天飞去。又看眼前的院落、房舍,都消失了,只有一座小坟墓,两棵柏树。张公子与家人抱着白鸽,惊骇叹息而归。回到家中,试验着让白鸽飞翔,异常驯良,边飞边斗如初见时一样,虽然算不上少年养鸽中的优良品种,但也是人世间绝无仅有的。张公子对两只鸽子爱惜备至。过了两年,这对白鸽又生了小公鸽小母鸽各三只,虽然亲朋好友,也得不到。

有一位张公子父亲的朋友,是个贵官。一天,见到公子,问:“你养了多少只鸽呵?”张公子谨慎地回答几句,就退下来。怀疑某公是爱好鸽子的,想赠送两只鸽子,但是实在舍不得。又想到长辈来索求,不能过于抹他的面子,而且也不敢以平常的鸽子送给他应付差使,就选两只白鸽,用笼子盛着去送给他,自己以为就是送千金的礼物,也不如这两只鸽子珍贵。

过了几天张公子见到某公,自己脸上很有居功得意之色,而某公说话间,并无一语感谢赠送鸽子的事。张公子不能忍耐,便问:“前天我送的鸽子可中意?”某公回答说:“也挺肥美。” 张公子惊讶地说:“大人把鸽子烹了?”某公回答说:“是啊!”张公子大惊地说:“这不是寻常的鸽子,就是平常所说的佳种‘靼鞑’的。”某公回想了一下说: “味道也没什么特殊的。”张公子听罢,悔恨地回到家里。

到夜里,张公子梦见白衣少年来见他,责备说:“我原以为你能很爱惜鸽子,所以把子孙托付于你。你怎么能把明珠投到黑暗中,致使我的子孙丧身于锅、鼎!今日我就率子孙去了。”说罢,化作鸽子,张公子所豢养的白鸽全都跟着它飞走了。

天明,张公子去看笼中的白鸽子,果然都不见了。心中很悔恨,接着把所养的鸽子,分别赠送给自己的好友,几日内就分光了。


\subsection{1.6.38   聂 政}
\label{\detokenize{p00_u5176_u5b83/_u767d_u8bdd_u804a_u658b_u5fd7_u5f02:id255}}
明代的怀庆潞王,荒淫无德。他经常到民间去,发现有美女,总要抢夺到手中。有个王生的妻子,被潞王看上了,便派遣车马径直进了她家。王妻号啕大哭不服从,被强抬着出了门。王生逃了出去,藏身在聂政墓地,希望妻子经过这里,能远远地和她诀别。不多时,妻子到了这里,望见丈夫,便大哭着扑到地上。王生悲痛的心情无法抑制,不觉哭出声来。跟从的人知道了他是王生,就抓住他,要用鞭子抽打他。

忽然坟墓中出来一个男子,手握利剑,气势威猛,厉声说道:“我是聂政!良家女子岂容强占。看在你们身不由己的份上,暂且饶恕你们。给那个昏王捎句话,若再不改恶行,没几天就将割他的脑袋!”众人大惊,弃车而逃,男子也进入坟墓不见了。王生夫妇叩拜了聂政墓回家,仍然害怕潞王再派人来。过了十几天,竟然毫无消息,心情才安定下来。潞王的淫威从此也有所收敛。


\subsection{1.6.39   冷 生}
\label{\detokenize{p00_u5176_u5b83/_u767d_u8bdd_u804a_u658b_u5fd7_u5f02:id256}}
山西平城有个姓冷的书生,小时候很迟钝,到了二十多岁,还没能读通一经。忽然来了个狐,和他住在一起。此后常听见冷生整夜说话,就是兄弟追问他,也不肯泄露。这样过了很多天,他忽然得了精神失常的毛病,每次得到题目作文,就闭门寂坐,过一会儿,便放声大笑。偷偷一看,他手不停地写着,一篇八股文很快就完成了,脱稿后竟然文思精妙。当年他考中了秀才,第二年又成了廪生。每逢考试便大笑,声音响彻考场堂壁,由此“笑生”的名声大噪。幸亏学政当时外出不在场,没有听见。后来遇上某位学政规矩严肃,整日端坐在考场大堂上,忽然听见笑声,愤怒地把他抓来,将要责罚。执事官代为说明他精神失常,学政的怒气才稍微消了一点,虽把他释放了,却除去了他的生员名籍。从此他便装疯沉缅于诗酒。著有《颠草》四卷,超群绝俗可供诵读。


\subsection{1.6.40   狐 惩 淫}
\label{\detokenize{p00_u5176_u5b83/_u767d_u8bdd_u804a_u658b_u5fd7_u5f02:id257}}
一书生买了一处新居,经常遭到狐的侵扰。一切衣服器物,多被毁坏,并且经常把尘土撤在汤饼里。一天,有朋友来拜访,恰巧书生有事外出,很晚也没回来。书生的妻子就做了饭菜款待客人。客人吃完以后,她才和丫鬟一起吃剩下的饭菜。

书生平日行为不检点,喜欢在房里偷藏春药。不知什么时候,狐把春药放到了粥里。妇人吃时,闻着有一股麝香味,就问丫鬟,丫鬟说不知。妇人吃完后,觉着欲火中烧,一霎也忍耐不住;自己强行压制,欲望更加强烈。想到家里再也没有别的男人,只有客人留宿,就跑去敲客人的房门。客人问她是谁,妇人就如实告诉了他;客人问她要干什么,妇人不回答。客人告罪说:“我和你丈夫是知己朋友,不敢有这样的禽兽行为。”妇人还舍不得走开。客人就斥骂说:“我朋友的文章道德,都被你丧尽了!”隔着窗户朝她吐唾沫。妇人非常羞愧,这才回到自己房里。于是想道,我怎能做出这样的事来?忽然想起吃饭时碗里的麝香味,莫非是丈夫的春药?她赶忙查看纸包里的春药,果然乱七八糟撒了一桌,瓦盆、酒杯里都是。妇人平时知道喝凉水可以解除,于是喝了下去。一会儿便觉得心里清醒,羞愧得无地自容。她躺在床上翻来覆去过了很久,已经更尽,更加担心天亮后难以见人,就解下衣带上了吊。丫鬟发觉后把她救了下来,已经没了气息。到了辰时,才有了微弱的呼吸。客人早已在夜里离去。

书生直到黄昏后才回家,见妻子躺在床上,问她怎么了,她不回答,只是跟含清泪。丫鬟把她上吊的事告诉了他,书生大吃一惊,就苦苦追问原因。妇人把丫鬟遣开,才把实情告诉了丈夫。书生叹息说:“这是对我淫欲无度的报应,怎能责怪你?幸亏遇到了好朋友,要不的话,可怎么做人?”就从此痛改前非,狐患也就绝迹了。


\subsection{1.6.41   山 市}
\label{\detokenize{p00_u5176_u5b83/_u767d_u8bdd_u804a_u658b_u5fd7_u5f02:id258}}
奂山的山市是淄川县有名的八景之一,可往往几年都见不到一次。有位名叫孙禹年的公子,同几位朋友在楼上饮酒,忽然看见奂山山头有座孤塔高高耸起,直插青天。在座的人都很惊疑,心想,附近并没有这么个禅院啊;一会儿,又出现了几十座高大的宫殿,碧绿色的琉璃瓦,飞翘的殿檐,人们这才明白是出现了山市。又一会儿,又变幻成一座长达六七里又高又厚的城墙,城墙上有一个个垛子;城中有的似高楼,有的像厅堂,有的像牌坊,历历在目,像有亿万家。

忽然,大风刮起,尘土飞扬,城市依稀可见。接着风定天清,刚才看到的一切全没有了,只有一座高楼直插云霄,每层楼有五个窗户大开着,闪着五点亮光,那是透过窗口看到的蓝天。一层层地指着数,楼越高亮点越小,数到第八层,亮点就如星星一般大了;又往上数,则虚无飘渺看不清楚,没法计算层次了。楼上的人住来奔忙,有倚窗的,有站立的,各不一样。过了一会,楼房慢慢低矮下来,可以看见楼顶了,慢慢地又像平常的楼一样了,又渐渐地像座高房子,猛然间又只像拳头那么大,像豆粒那么小,接着就什么也看不见了。又听说有起早赶路的人,看见山上有商店集市人来人往,和人世间没有两样,所以又叫“鬼市”。


\subsection{1.6.42   江 城}
\label{\detokenize{p00_u5176_u5b83/_u767d_u8bdd_u804a_u658b_u5fd7_u5f02:id259}}
江西临江的高蕃,年少聪慧,仪表秀美。十四岁入了县学,富豪人家争着把女儿许配给他。高蕃挑选妻子很严苛,屡次违背父亲的意旨。他的父亲名叫高仲鸿,六十多岁,只有这一个儿子,非常宠爱他,不忍心违背一点儿子的心意。

当初,东村有个樊翁,在一家店铺中教授儿童启蒙,租赁高蕃家的房屋携家居住。樊翁有个女儿,乳名叫江城,与高蕃同岁,当时都是八九岁,两小无猜,每天一同玩耍。以后樊翁迁走了,过了四五年,两家没有再通过消息。

一天,高丫在小巷中看见一个女郎,艳美绝伦。跟着一个小丫鬟,仅六七岁。高蕃不敢正面对视,只是斜眼偷看女郎。女郎停步凝视着他,好像有话要说。高蕃仔细一看,原来是江城,顿时非常惊喜。两个人都没有说话,你看我,我看你,呆呆地站着。过了会儿才走开,两人都流露出恋恋不舍的样子。高蕃临走时故意把一条红巾掉在地上,小丫鬟拾起来,欢喜地交给少女。女郎把红巾掖入衣袖中,换成自己的手帕,假装对丫鬟说;“高秀才不是外人,不要匿藏他丢失的东西,你快追上还给他!”小丫鬟果然追上交给了高蕃。高蕃得巾大喜,回家请求母亲去求婚。高母说:“江城家无半间屋,到处流浪,怎么能和我家般配呢?”高蕃说:“我自己要娶她,绝对不后悔!”高母决定不下来,和高仲鸿商量,仲鸿执意不同意。

高蕃听说后心里闷闷不乐,吃不下饭。高母忧虑,对高仲鸿说:“樊氏虽然贫穷,也不是那些市侩无赖可比的。我去他家拜访,倘若他女儿般配,也没什么不可。”仲鸿说:“好。”高母便假托到黑帝祠烧香,到樊家探问,见江城明眸秀齿,容貌娟丽,心里非常喜欢,于是拿很多钱和绸缎赠给樊家,把结亲的想法实说了。樊母起初谦让推辞,后来还是接受了婚约。高母回来述说详情,高蕃才开始露出笑容。过了年,选择良辰吉日把江城娶过来,夫妻二人相处很和美。

但是江城善怒,翻脸不认人,又好絮烦,常在耳边吵嚷。高蕃因为爱恋她的原因,都忍住了。高蕃父母听说后,心里不高兴。一次私下里责怪儿子,被江城听到了,大怒,更加痛骂高蕃。高蕃稍微反驳,江城更怒。把高蕃驱赶出屋,关上房门。高蕃在门外冻得索索发抖,也不敢敲门,抱住膝盖呆在屋檐下过夜。江城从此把高蕃视为仇人。起初,高蕃长跪就可以讨饶,逐渐地这一招也不灵了,遭受的痛苦逐渐加深。公婆略微说江城几句,江城那顶撞不服的样子,实在无法形容。公婆愤怒,把她休回娘家。樊翁心里惭愧,央求熟悉的人在高仲鸿面前求情,仲鸿不答应。

过了一年多,高蕃外出遇到岳父。岳父邀他到家中,不住地表示歉意。让女儿妆扮好出来见丈夫,夫妻相见,内心不觉酸楚。樊翁就买了酒款待女婿,非常殷勤地劝酒。到了傍晚,又恳切地让高蕃住下过夜。整理另一张床,让夫妻二人共寝。天要亮时,高蕃告辞回家,不敢把实情告诉母亲,掩饰得非常严密。从此每隔三五天,高蕃就在岳父家住一夜,父母一直不知道。

樊翁一天亲自去拜访高仲鸿,仲鸿起初不肯见面,后来迫不得已,只得出来相见。樊翁跪着上前,请求让女儿回来,仲鸿不肯,借口儿子不愿意。樊翁说:“女婿昨晚住在我家,没有听说有什么不满意的话。”仲鸿惊问:“何时在你那里住宿?”樊翁把详情告诉了他。仲鸿羞惭地说:“我确实不知道。既然他爱江城,我本人何必仇视江城呢?”樊翁离开后,仲鸿叫过儿子,痛骂不绝。高蕃只是低着头不答话。说话间,樊父已把江城送来。仲鸿说:“我不能为子女承担过错,不如各立门户,就麻烦你主持签订分家的契约。”樊翁劝阻,仲鸿不听。于是让高蕃夫妇在另一院居住,派一侍女服侍他们。过了一个多月,相安无事,高蕃的父母私下暗自快慰。可是不久,江城又渐渐放肆起来,高蕃脸上时常有手指抓破的痕迹。父母明明知道,也强忍着不过问。

一天,高蕃实在忍受不了毒打,奔到父亲的住所躲避,惊惶得好像被扑打的鸟雀一样。父母正要询问,江城已操着木棒追赶进来,竟然在公婆身旁抓住丈夫痛打。公婆大喊住手,可江城一点不顾,直打了几十下,才悻悻地离去。高父驱赶儿子说:“我是为了避开喧闹,才和你分开过。你既然喜欢这样,又为什么逃到我这儿呢?”高蕃被驱逐出来,徘徊在外,没地方可去。高母怕他受挫寻死,让他独自居住,供给他食物;又把樊翁召来,让他调教女儿。樊翁走进房中,万般劝说开导,江城始终不听,反而用恶言恶语挖苦父亲。樊翁拂袖而去,发誓跟女儿一刀两断。不久,樊翁因愤恨而生病,和老妻相继死去。江城怨恨父母,也不回娘家去吊丧,只是每天隔着墙壁谩骂,故意让公婆听见,高仲鸿都置之不理。

高蕃独自居住,虽然好像离开了汤火的煎熬,只是觉得有点凄凉孤独。便偷偷用金钱买通媒婆李氏,托她找了个妓女收在书房中,来往都乘夜晚。时间久了,江城微微听到风声,到书房中谩骂。高蕃极力表白,指天发誓,江城才回去。从此江城每天伺机寻找高蕃的把柄。有一次李氏从书房中出来,恰好和江城相遇。江城急忙喊叫她,李氏神色慌张,江城更加怀疑,对李氏说:“据实说出你的所作所为,或许可以免罪!如果还隐瞒真情,我把你的毛发揪光!”李氏战战兢兢地说:“半月来,只有妓院李云娘来过两次。刚才公子说,曾在玉笥山遇见陶家媳妇,爱慕她的两只小脚,嘱咐我把她招来。她虽然不是贞洁女人,也未必就愿来过夜,能否成功不敢肯定。”江城因她说出实情,姑且饶恕。李氏要走,江城不许。等到太阳西落,江城喝斥她说:“你先去吹灭他的蜡烛,就说陶家媳妇来了。”李氏只得照江城说的那样办。江城跟着急忙走进房中。高蕃喜坏了,挽着江城的的手臂拉她坐下,述说了自己怎样如饥似渴。江城默不作声。高蕃在暗中摸到她的脚,说:“山上一见您的仙容,忘不了的就是这双脚。”江城始终不语。高蕃说:“昔日的夙愿,今天才得以实现,为什么见面却不相认呢?”自己举灯就近一照,原来是江城!高蕃大惊失色,吓得把蜡烛掉在地上,跪在地上浑身哆嗉,好像刀子已经搁在脖子上。江城捏着耳朵把高蕃提回去。用针把两条大腿都扎遍了,才让他躺在下铺休息,自己醒过来就大骂一顿。高蕃从此害怕妻子犹如虎狼,即使江城偶尔给他好脸色,高蕃在枕席之上也不能正常行事。江城就打他的嘴巴,把他喝斥走,更加厌弃他没有男人样。高蕃每天虽身在芝兰芳香之室,却犹如监狱里的犯人,仰事狱吏之尊严。

江城有两个姐姐,都嫁给了秀才。大姐心地平和善良,寡言少语,和江城相处得不融洽。二姐嫁给了一个姓葛的,她为人狡诈善辩,搔首弄姿,虽长得不如江城,但凶悍妒忌却不相上下。两姊妹相逢没有其它的话,只是以在家中如何施威而自鸣得意,因此两人关系最好。高蕃拜访亲戚朋友,江城总是嗔怒;只有到葛家,知道了也不禁止。一天,高蕃在葛家饮酒,已经喝醉了,葛氏嘲弄说:“您为什么这样害怕内人?”高蕃笑着说:“天下事有很多难以理解,我之所以害怕内人,是因为内人美貌;还有那种内人不及我内人美貌,但却比我更惧怕内人的,不是更加令人疑惑不解吗?”葛氏非常羞惭,不能回答。丫鬟听到这话,告诉了二姊。二姊大怒,立刻操杖迫出来。高蕃见她气势汹汹,来不及提鞋想要逃走,擀面杖挥起,已打在了腰脊部,打了三杖,高蕃三次倒在地上,再也爬不起来。又一杖误打在头上,血流如注。二姊离去,高蕃才蹒跚着回家。江城见了惊问怎么回事。起初高蕃因为触犯了二姊,不敢实说,江城再三追问,才说出详情。江城用丝帛包住高蕃的头,愤然说:“人家的男人,何劳她痛打呢!”换上短袖衫,怀藏木棒,带着丫鬟迳直赶去。到了葛家,二姊笑脸相迎。江城一语不发,一棒打去,二姊倒在地上,撕裂了裤子,痛苦不堪,牙齿被打落了,嘴唇豁开了,屎尿都流了出来。江城回去后,二姊羞愤,派丈夫赶到高家算帐。高蕃急忙赶出来,极力好言劝慰。葛某小声说:“我这次来是身不由己。悍妇不仁不义,幸而借妹妹的手惩罚了她,我们两人何必产生矛盾呢?”江城已经听到,急忙出来,指着葛某骂道:“龌龊贼!妻子吃了亏你反而私下和外人交好,这样的男人,怎不该打死呢?”大声喊人,寻找擀面杖。葛某大窘,夺门窜出。高蕃从此再也没有一处可以来往的人家了。

同学王子雅经过这里,高蕃殷勤地挽留喝酒。饮酒间,谈些闺阁的事情,互相戏谑打逗,言语颇为猥亵。江城恰好来瞅客人,把全部的话都偷听去了,暗中把巴豆投在汤中端上去。不长时问,王子雅上吐下泻不可忍受,只存奄奄气息。江城派丫鬟问王子雅:“还敢无礼吗?”王子雅这才醒悟患病的来由,呻吟着请求饶恕。这时绿豆汤早已准备好了,喝下去,吐泻就止住了。从此,相识朋友互相暂诫,不敢再到高家喝酒。

王子雅有座酒馆,酒馆里有很多红悔,王设宴款待同辈朋友。高蕃假托要到文社去。告诉江城后就去了。太阳西落,酒意正浓时,王子雅说:“恰好有个南昌名妓,流落在此地,可以招来共饮。”众人都非常高兴,只有高蕃离席,极力肯辞。众人拉住他说:“闺阁中耳目虽长,也不会听见看见这里。”于是共同发誓不走漏风声,高蕃这才重新坐下。过了一会儿,妓女果然来了,年纪约十七八岁,戴的玉佩叮当作响,如云的发鬟梳得高高的。问她的姓名,她说:“姓谢,小字芳兰。”说话吐气,非常高雅,举座若狂。而芳兰尤其对高蕃有意,屡次以眉目传情,被众人发觉了,故意拉两人并肩坐在一起。芳兰暗自抓住高蕃的手,用手指在高蕃手掌上写了个“宿”字。高蕃此时,要去不忍心,要留又不敢,心如乱麻,不可言喻。两人低着头说悄悄话,高蕃醉态更加放纵,床上的“胭脂虎”也都忘在脑后了。再喝一会儿,夜已经很深了,酒馆中客人更加稀少,只有远座上一个美少年,对烛独饮,有个小僮拿着餐巾侍奉在旁边。众人私下议沦少年气质高雅。不久,少年饮完走出酒馆。小僮返回来,对高蕃说:“主人等待着有句话要对你说。”众人都茫然不解,只有高蕃颜色惨变,来不及和众人告别,便匆匆而去。原来那个少年便是江城,小僮是她的丫鬟。高蕃跟随着回到家,伏着受鞭打。从此江城禁锢得更加严密,丧喜事都不让他去参加。文宗来讲学,高蕃因为误讲而被降为青衣。一天,高蕃和侍女说话,江城怀疑二人私通,把酒坛罩在侍女头上痛打。又把高蕃和丫鬟都绑庄,用绣剪剪下两人腹部的肉皮,再交换着补上,解开绳子后令他们自已包扎。过了一个多月,补的地方竟然弥合了。江城常常光着脚把饼踩在尘土巾,喝斥高蕃拿起来吃下去。像这样的折磨,种种不一。

高母因为想念儿子,偶尔到他的房子去,见儿子骨瘦如柴,回家痛哭欲绝。夜晚梦见一老叟告诉她说:“不用忧烦,这是前世的因果报应。江城原是静业和尚所养的长生鼠,公子前世是学子,偶尔游览那座寺庙,误把长生鼠打死了。现在得的恶报,人力不可挽回。你每天早起,虔诚诵读心经观音咒一百遍,一定会有效。”高母醒世来把此事讲给高仲鸿听,两人心里感到怪异,于是夫妻照着办了。虔诚诵念了两个多月,江城仍和从前那样蛮横,变得更加狂纵,听到门外有锣鼓声,梳妆未完就握着头发跑了出去,假痴不呆地远远观看。上千人指着看她,她却很坦然,不以为怪。公婆都为此感到耻辱,却管不住她。

忽然有个老僧在门外宣讲佛法因果,观看的人围得如一堵墙。老僧吹动鼓上的皮发出牛叫声。江城奔过去,见人多没有缝隙,就让婢女搬出座位,她爬上去站着看。众人的眼光都向她看去,她如同没有感觉。过了一会儿,老僧论说佛事将完时,索取一盂清水,拿着面对江城宣祷道:“莫要嗔,莫要嗔!前世也非假,今世也非真。咄!鼠子缩头去,勿使猫儿寻。”宣讲完,吸一口水喷射到江城脸上,粉脸湿漉漉的,一直流到襟袖上。众人大惊,认为江城会暴怒。江城却一声不吭,擦擦脸自己回去了。老僧也离开了。江城进室呆坐,茫然若失,一整天也不吃不喝,打扫床铺迳自睡下。半夜江城忽然把高蕃唤醒,高蕃以为她要解溲,捧进尿盆。江城不接,暗自拉住高蕃手臂,拉进被中。高蕃明白,但却浑身抖动,好像捧的是圣旨。江城感慨地说:“害得您这样,我怎么配作人呢!”于是用手抚摸着高蕃的身体,每摸到刀杖疤痕处,就嘤嘤啜泣,用指甲掐自己,恨不得立即去死。高蕃见此情形,心里很不忍,耐心地反复劝慰安抚。江城说:“我觉得那老僧必是菩萨化身,清水一牺,好像换了我的肺腑。现在回想起我从前的所作所为,都如同隔世一般。我从前莫非不是人吗?有丈夫而不能同欢,有公婆而不能侍奉,这到底是什么心思!咱们明天可以搬回家去,仍和父母同居,以便于早晚请安。”絮絮叨叨说了一夜,如同叙说十年离别之情。第二天天未亮,江城就起来,整好衣服,理好家具,丫鬟带着箱簏,江城亲自抱着被褥,催促高蕃前去父母处叩门。高母出来,见此情景惊讶地询问,高蕃把意思告诉了她。高母还在迟疑不决,江城已和丫鬟走进来。高母随后进屋。江城伏在地上流泪哀求,只求免死。高母觉察她是出自真心实意,也流泪说:“孩儿何以一下子变成这样了?”高蕃对母亲详细叙说夜里的情形,高母才醒悟从前的梦灵验了。大喜,唤奴仆为他们打扫从前的房子。

江城从此看着公婆的脸色,顺着长辈的意志行事,胜过孝子。每当遇见生人,就腼腆得像新娘子。有人开玩笑叙说往事,她马上就涨红了脸。江城又勤俭,又善于积累,三年中,公婆不过问家事,但已积蓄起万贯家财。高蕃这年乡试大捷,考中举人。江城常对高蕃说:“当日见过芳兰一面,现在还是想着她。”高蕃因为不受虐待,心愿已满足,非分想法不敢再有,只是点头而已。正巧高蕃赶到京城会考,几个月才返回。进屋,见芳兰正和江城下棋。高蕃惊奇地询问这事,才知道江城用几百两银子赎买芳兰脱离妓院了。这件事情浙中王子雅说得非常详细。


\subsection{1.6.43   孙 生}
\label{\detokenize{p00_u5176_u5b83/_u767d_u8bdd_u804a_u658b_u5fd7_u5f02:id260}}
孙生,娶了仕宦人家的女子辛氏为妻,辛氏初入门,穿着裆裤,裤子上有很多带子,浑身上下束缚得很紧,拒绝和孙生同床,还在床头上常常放着锥、簪之类的东西用来自卫。孙生屡次被刺伤,只好在另一张床上自宿。过了一个多月,仍不敢和妻子共寝。两人即使是白天相逢,妻子也从没给以好言笑脸。

孙生有个同学知道这事后,私下对他说:“你夫人能喝酒吗?”孙生答道:“喝得很少。”这人和孙生开玩笑说:“我有替你们调停的方法,方法绝妙并且可行。”孙生问:“什么方法?”那人说:“把迷魂药放在酒中,骗她喝下去,你就可以为所欲为了。”孙生笑了,但暗自佩服这个方法很好。于是向医生询问,亲自用酒煮乌头,把煮好的酒放在桌上。到了晚上,孙生斟上另一种酒,独自喝了几杯睡下了。像这样过了三晚上,妻子始终不喝那药酒。一天夜里,孙生在另一张床上躺下时,看到妻子还孤寂地坐着,孙生故意发出鼾声。妻子于是下床,取过酒来在炉子上煨热,孙生暗自欢喜。过了一会儿,妻子喝干了一杯,又斟上再喝;大约喝了半杯左右,把剩下的酒仍倒进壶中,收拾床铺睡下了,过了很长时间没有出声,但灯明煌煌的还没有熄灭。孙生怀疑她还醒着,故意大喊:“锡灯台烧化了!”妻子不答应,再喊仍然不应声。孙生光着身子过去一看,妻子已醉睡如泥。孙生揭开被子轻轻躺进去,层层割断她身上的束结。妻子终于觉察到了,只是不能动,也不能说话,任凭他轻薄而去。妻子醒过来后,心里感到怨恨,便上吊自杀了。孙生梦中听到吼喘声,起身奔去查看,见妻子的舌头已伸出大约两寸左右了。孙生大惊,割断绳索,把她扶到床上,过了一个时辰,她才苏醒过来。孙生从此非常厌恶妻子,夫妻避道而行,相遇就各自低下头。过了四五年,两人没说一句话。妻子有时在家中正和别人嘻笑着,一见丈夫来了,脸色立变,严肃得像蒙上了一层霜。孙生曾寄宿在书斋中,整年不回家;即使强迫他回家,也只是面对墙壁消磨时光,默然就枕罢了。孙生的父母对此非常忧虑。

一天,有一个老尼来到孙生家,见了孙妻,倍加赞誉。孙母没有说话,只是长叹。老尼询问缘故,孙母就把详情说了。老尼说:“这是很容易办的事情。”孙母高兴地说:“倘若能使媳妇回心转意,我不在乎花费多少报酬。”老尼偷看室内无人,低声说:“买一副春宫图,三天后为你镇服她。”老尼离去后,孙母就买好东西等着。过了三天,老尼果然来了,嘱咐说:“此事必须非常保密,不要让他夫妻二人知道。”于是剪下图中人物像,又拿三枚针,一撮艾草,一并用白纸包好。纸包外面绘上几幅像蚯蚓形状的图画,让孙生的父母把妇人骗了出去,偷拿到她的枕头,撕开线缝,把那些东西放进去。然后仍缝好,放回原处,老尼这才离开。到了晚上,孙母强让儿子回家睡觉,并派一老妇去偷听。将过二更时,听到孙妻呼唤孙生小名,孙生不答。过了一会儿,孙妻再唤,孙生厌烦,没有好声气。天亮后,孙母走进儿子的房间,见夫妻二人脸面相背,以为老尼的计策足骗人的。她把儿子叫到无人处,婉转地劝说他。但孙生一听到妻子的名字便大怒,咬牙切齿。孙母怒骂儿子,孙生头也不回地走了。第二天,老尼又来了,孙母告诉他事情无效,老尼大加怀疑。老妇人于是叙说偷听到的情形。老尼笑着说:“以前你只说是妻子憎恨丈夫,所以单镇服她。现在妻子的心已转变,而男方还没有。请再用先先前使用的方法镇服双方,一定有效。”孙母听从,拿过儿子的枕头像上回那样收拾好,又叫儿子回家睡觉。过了一更,好像听到两张床上有转侧声,两人不时咳嗽,都像睡不着的样子。时间久了,听到两人在一张床上说话,但隐隐约约听不清楚。天将亮,还听到吃吃的嘻笑声,不绝于耳。老妇人把事情告知孙母,孙母大喜。老尼来后,赠给她丰厚的谢礼。孙生和妻子由此琴瑟和好,生下一男一女,夫妻十余年中再没有发生过口角。周围的人私下询问其中的原因,孙生笑着说:“先前看到妻子的身影就发怒,以后听到妻子的声音就喜欢,我自己也不能解释这是什么心情。”


\subsection{1.6.44   八 大 王}
\label{\detokenize{p00_u5176_u5b83/_u767d_u8bdd_u804a_u658b_u5fd7_u5f02:id261}}
甘肃临洮冯生,原是富贵人家的后代,后来家事衰败。有一个以捉鳖为业的人,欠他的债务偿还不起,打到鳖就献给他抵债。一天,献给他一只个头很大的鳖,额顶上有白点。冯生以为鳖的形状不同一般,就把它放了。后来,他从女婿家回来,走到恒河的岸边,天色已经黄昏,见到一个喝醉酒的人,跟着二三个僮仆,跌跌绊绊地走来。很远见到冯生,就问:“什么人?”冯生漫不经心地说:“走路的人。”喝醉酒的人生气地说:“难道没有姓名?胡说是走路的人!”冯生因赶路的心很急切,把他的问话放在一边不回答,迳直走过去。喝醉酒的人更生气,捉住冯生的衣袖不让他走,一股酒臭气熏人。冯生更不耐烦,然而用力拉也解脱不了,就问道: “你叫什么名字?”他好像说梦话似地说:“我是南部过去的县令,你要干什么?”冯生说:“世界上有这样的县令,沾污了世界!幸亏是旧县令,假若是新县令,还不杀光了走路的人吗?”喝醉酒的人很愤怒,样子要动武。冯生口气很大地说:“我冯某并不是受人打的!”喝醉酒的人听到,变愤怒为高兴,踉踉跄跄地下拜说:“原来是我的救命恩人,冒犯了切勿怪罪!”从地下起来,呼唤随从的人,先回去准备酒菜。冯生推辞,他不同意,握着冯生的手,走了好几里路,来到一个小村落。走进院里,见房廊屋舍都很华丽,好似贵人之家。醉人酒稍醒,冯生才询问他的姓名。他说:“我说了你可切勿惊怪,我是洮水上的八大王。刚才在西山青童那里饮酒,不觉喝醉了,对你有不恭之处,实在感到惭愧而又害怕。”冯生听了,知道它是妖怪,因为他的话语殷勤实在,就不害怕了。

一会儿,就设了丰盛的宴席,与冯生亲热地喝起酒来。八大王饮酒最豪放,一连干了好几杯。冯生恐怕它再喝醉了,来打扰自已,就假装已经喝醉,要求睡下。八大王明白他的意思,笑着说:“先生是不是怕我发狂啊?请您不要惧怕。凡是喝醉酒的人行为不端,并说自己隔一夜就不再记得,那是欺人的。饮酒的人无德,故意犯错误的十个中就有九个。我虽然不足以与你同辈相处,但还未敢以无赖的行为来对待您,为什么这样拒绝我呢?”冯生就又坐下,态度郑重地劝谏说:“既然自已知道,为什么不改正自己的行为?”八大王说:“老夫为县令时,沉醉于饮酒,比今天尤甚。自从触怒了天帝,被贬谪回归到岛屿上,就尽力改正以前的行为,十多年了。现在我已经是快进棺材的人,潦倒不能飞黄腾达,所以旧毛病又犯了。我自已也说不清楚,现在特意听您的指教。”倾心谈话间,远处的钟声已经响了。八大王起身捉住冯生的手臂说:“我们相聚时间不长了,我藏有一件东西,聊以报答您的厚德。这东西不可以长久佩戴,满足自己心愿后,就当归还我。”从口中吐出一个小人,仅仅有一寸高。八大王以指甲掐冯生的手臂,疼痛得如同皮肤裂开。八大王急忙把小人按捺在上边,放开手,小人已经进入皮里,指甲的痕迹还在,而臂上慢慢地突起,好似一块疙瘩的形状。冯生惊奇地问他,他笑而不答,只说:“先生可以走了。”把冯生送出来,八大王自已返了回去。回头看,村庄田舍全都不见,唯有一只巨大的鳖,蠢笨地爬进水中。惊讶了很长时间,自己想,所得到的必定是“鳖宝”。

自这以后,他的眼特别明亮,凡是藏有珍珠宝贝的地方,即使在很深的地下,都可见到;即使平日所不认识的东西,也可随口说出它的名字。在他睡觉的房间中,掘出埋藏在地下的数百串钱,他的生活用度已很充足。后来,有出卖一所旧宅子的,冯生看到它里面藏有无数成串的钱,就用很多的钱购买来。从此与王公大臣同等富裕。家中奇珍异宝应有尽有。他还得到一面镜子,背面有突起的凤纽环儿和水云湘妃的图,它的光亮能照一里多,胡须和眉毛都可数清楚。美丽的女人一照,影子就可留在里面,磨也磨不掉。假若改换妆梳重照,或者再更换一位美人,前面所照的影儿就消失掉。当时,肃王府的三公主生长得绝世的美丽,冯生久已仰慕她的名字。正巧遇到三公主去游崆峒山,他就事先到山中藏下来,等待三公主下车时,就用镜子照了她。回来后,把镜子放置在书案上,细细察看。见到美人在镜中,用手拈巾微笑,嘴好像要说话,眼波也像在流动,冯生高兴地藏起来。

一年多后,这件事让他妻子泄露出去,传到肃王府。肃王大怒,把冯生捉起来,把镜子追去,拟将冯生斩首。冯生贿赂宦官,请他们告诉肃王:“大王如果能赦免,天下的最值钱的宝贝不难弄到。若不然,只有死,而对王也没有什么益处。”肃王想抄他的家,把他迁到别的地方去。三公主说:“他已经偷看到我的容貌,即使死十次也解脱不了这种玷污,还不如嫁给他。”肃王不允许。三公主生气,把自己关在房子里不吃东西。肃王的妃子很忧愁,尽力说服肃王。肃王就释放了冯生,命宦官把这个意思向冯生说明。冯生推辞说:“糟糠之妻不下堂,我宁愿死掉,也不能从命。肃王如果准我自赎,即使倾家荡产也可以。”肃王愤怒,又把冯生逮捕起来。王妃把冯生的妻子召进宫中,想把她用毒药毒死,见到她,冯妻把一个珊瑚镜台赠送给王妃,说话言语也很温和动人。王妃喜欢她,让她参见三公主。公主也喜欢她,两人订为姊妹,让人转告冯生。冯生告诉妻子说:“王侯的女儿,不可以用先来后到论定嫡与庶。”妻子不听,回到家里置备聘礼,送进王府。去送礼品的有千人,珍宝玉石之类,王家也不知道它的名字。肃王大喜,释放冯生回家,把三公主嫁给他。三公主仍然携带着镜子归去。

冯生一天晚上独自睡下,梦见八大王高大的身躯走进来,说:“所赠送的东西,应当还给我了。佩戴久了,耗费人的心血,折损人的寿命。”冯生答应了,留下八大王一起宴饮。八大王告辞说:“自从聆听你的教诲,酒已经戒了三年。”就以口啃冯生的手臂,冯生痛得很厉害。醒来一看手臂,肿块已经消失了。自此以后,冯生仍然和平常人一样。


\subsection{1.6.45   戏 缢}
\label{\detokenize{p00_u5176_u5b83/_u767d_u8bdd_u804a_u658b_u5fd7_u5f02:id262}}
淄川县有个人,一向轻佻无赖。一次,他偶然在村外游玩,见一个少妇骑着马走过来,他便跟同伴说:“我能让她一笑!”同伴们不信。双方约定打赌,谁输了请客喝酒。

无赖突然跑到少妇马前,连声嚷叫着说:“我要死!我要死!”说着,从墙头上把一根高梁秸横抽出一尺多,解下腰带挂在上面,伸进脖项,作出上吊的样子。少妇走过他身边,果然被逗笑了。夫家也都笑起来。少妇过去了,无赖仍然站在那里一动不动,大家更加大笑起来。走近一看,只见他舌头伸了出来,眼睛紧闭着,已经真的吊死了!

在高梁秸上能吊死人,这事不也太奇怪了吗?这件事可以作为那些轻薄人的警戒了。


\section{1.7   卷 七}
\label{\detokenize{p00_u5176_u5b83/_u767d_u8bdd_u804a_u658b_u5fd7_u5f02:id263}}

\subsection{1.7.1   罗 祖}
\label{\detokenize{p00_u5176_u5b83/_u767d_u8bdd_u804a_u658b_u5fd7_u5f02:id264}}
即墨县有个叫罗祖的人,小时候家里贫穷。有一年,恰好他们姓罗的族中摊着要个人去北部边疆当兵,族人决定叫他去。

罗祖在北疆的好几年里,娶了媳妇,生了个儿子。队伍上的守备官待他很好。不久,守备升了官,要去陕西当参将,打算把罗祖也带了去。他把妻子和孩子托付给一位姓李的朋友照顾着,便跟守备去了陕西。一去就是三年。

一次,罗祖听说参将想给北疆去一封信,就申请把送信的任务交给他,也好借这个机会看望久别的妻子和儿子。参将同意了。

罗祖到家见妻子很健康,感到很欣慰。可是发现床底下有一双男人的鞋,心想,我三年不在家,哪来的男人鞋?莫非……便和妻子到李姓朋友家,感谢他三年来的照顾。姓李的朋友见他回来,赶紧做菜摆酒,热情地劝他夫妇吃喝;妻子也说三年来姓李的对她照顾多么多么好,简直是个大恩人,罗祖也说了好多感谢的话。第二天,罗祖对妻子说:“我得替参将送信去,晚上回不来,不要等我了。”说完,骑马走了。实际上他并没有去送信,而在近处找了个地方藏起来,到了夜里二三更的时候又回来了。一进门,听见妻子跟姓李的正在床上睡觉,说些无羞耻的话,他气极了,撞开门进了内室。妻子与姓李的吓坏了,在地上跪着爬到他面前,说:“我们不是人,我们该死!” 罗祖把刀抽出来,真想一刀结果了这两个狗男女,但沉思了一下,又把刀插入刀鞘,对姓李的说;“我原来把你当人看待,你既然这样,说明你是个禽兽,杀你反而玷污了我的刀。这样吧,我的妻子和儿子你要,我的兵也由你替我当,马匹和武器都在这里,我走了!”说罢就走了。

罗祖的乡邻知道了这件事,一齐告到了官府。官府便把姓李的提去,拷问。姓李的全部招供了。但除了李的供词,一没有人证,二没有物证,没有充分的根据给他定刑。派人到处找罗祖,一点影子一点消息也没有。官府便怀疑是姓李的因奸情杀了罗祖,便对姓李的及罗妻施以更重的刑罚。过了一年,这两个男女都死在狱中,官府就把罗祖的儿子送回了他的即墨老家。

又过了好久,石匣营村有个打柴的人进山,经常看见一个道人坐在一个山洞里,可从来没见他下山化过缘求过吃。消息传来,大家都觉得很奇怪:他吃什么活着呢?就一齐给他送去吃的。有人认识这个道人不是别人,就是罗祖。送来的吃食都放满了山洞,罗祖始终也没吃一点。看他的意思是讨厌这么多人去看他,渐渐地,就很少有人去了。好几年后,洞外的乱草长得像树那么高了,偶尔有人到洞内看见他仍坐在那里没动地方。又过了好久,有人见他在山上走动,待接近他时,却又没了。再回洞中找他,还在洞中坐着,衣服上往日的尘土都没变样。大家更加奇怪,又过了几天再去看,只见他的鼻梁都塌陷了,这才知道他早已坐着死了。

乡邻为了纪念他,建了一座罗祖庙。每年三月来烧香的络绎不绝。他的儿子去烧香,人们都喊他小罗祖,香火钱都给了他。至今他的后代还年年去收香火钱呢。

这个故事是沂水刘宗玉对我讲的,很详细。我笑笑说:“现在出家的和尚道士不想当圣贤,却想成佛祖,请告诉他们,要想立地成佛,得把手中的刀放下。”


\subsection{1.7.2   刘 姓}
\label{\detokenize{p00_u5176_u5b83/_u767d_u8bdd_u804a_u658b_u5fd7_u5f02:id265}}
淄川县有个姓刘的人,习性凶狠蛮横,真像个披着人衣的老虎。后来这人从淄川迁到沂县,恶习没有改掉,乡里人都害怕他,厌恶他。刘某有几亩地,和一家姓苗的地界挨着。姓苗的很勤快,在地边种了很多桃树。桃树刚开始结果时,苗家的儿子去摘。刘某见后,怒气冲冲地将他赶下树,指着那些树说是他的。姓苗的儿子哭着回家告诉了父亲。姓苗的正在惊讶时,刘某已赶到门前辱骂起来,并扬言要到衙门告状。姓苗的笑着安慰他,刘某怒气不消,忿怒而去。

这时,刘某同县老乡李翠石在沂县开当铺。刘某拿着状纸进城,恰好和他相遇。因是同乡又很熟悉,李翠石便问他:“干啥去?”刘某就把进城打官司的事告诉了他。李翠石听后,笑着说: “你的名声,众所共知;我和姓苗的素来相识,他平生很善良,怎么敢占骗你呢?你不要将事情说反了啊!”说完就撕碎他的状纸,拉他进了当铺,说以后给他俩调解,不要再争执下去。刘某怨恨仍不消,暗中拿铺里的笔,重新写了状纸,准备过后再告。一会儿,姓苗的来到铺里,把事情前因后果详细告诉了李翠石,哀求李翠石为他解除这场纠纷。姓苗的又说:“我是个庄稼人,半辈子没见过当官的,只要不打官司,几棵桃树,谁还敢占为已有。”李翠石叫出刘某,把苗家退让的意思告诉了他。刘某又指天画地,大骂不休;姓苗的光说好话,一句也不敢辩驳。

过了四五天,李翠石碰见刘某村里的人说他死了。李翠石听后很吃惊,叹息不止。后来李翠石外出,见迎面走来一个拄拐杖的人,很像刘某。走到跟前,果然是他。刘某热情向他问候,并请他到家里去作客。李翠石不敢靠近他,说道:“前几天听说你去世了,这是从哪里传来的谎言。”刘某不答话,一个劲地拉他进村,到家摆好酒菜后才说:“以前的传言,一点也不假。前天我出门,见来了两个人,要捉我去官府。问什么事,二人只说不知道。我想,我出入衙门十几年,不怕见官长的人,也就跟着他俩去了。走进公堂,见上面坐着的官,脸上带着怒气,说道:‘你是刘某吗?罪恶满盈,自己不肯悔改;又把别人的东西占为己有,像你这种蛮横凶暴的人,按例应当放到油锅里炸死!’旁边一个人查过簿册,说:‘这个人行过一次善事,按例不应当死。’那个当官的看过簿册,脸上的怒气稍微消了些,说道:‘暂时先送他回去吧!’几十个人齐声呵斥撵我走。我说:‘因什么事把我捉来?又因什么事送我走?请求向我说明白。’衙役拿着簿册走下来,指着上面的一条给我看。上面写着:崇祯十三年,用钱三百,救活一对夫妻,使他们得到团聚。衙役说:‘没有这一条,今日命当绝,让你投生为畜类。’听后,我很害怕,急忙跟抓我的那两个人出来。两人向我索贿,我愤怒地说:‘你们不知道我刘某出入衙门二十年,是专勒索别人的钱财,怎么竟敢向老虎要肉吃呢!’两人不敢再要,把我送到村口,向我拱手说道: ‘这趟差事没得到你的一口水喝。’两人走后,我进门就苏醒过来了,这时我断气已经两天了。”李翠石听后,感到这事很奇怪,就问他行的那件善事。原来,崇祯十三年,遇上了大灾荒,出现人吃人的情景。刘某那时在淄川县衙当捕隶。一天,遇见一男一女哭得很伤心,问他们为何这样?回答说:“俺俩结婚才一年多,今年遇上灾荒,不能一块儿活下去,只好悲伤罢了。”过了不多时,在一个油店门前又遇上他俩,好像在和店主争什么。刘某走到跟前,问怎么回事。油店的店主姓马,说:“他俩饿得快要死去,每天靠讨吃我的麻酱才活下来。今天又想把老婆卖给我,我家里已买下十多口,这事也好,只要价钱便宜,我就收下她,否则罢了。那有像你这样可笑的男子,没完没了地缠磨人!”那男子便道:“眼下小米贵得如同珍珠,若不要得三百文钱,就不够我逃命的路费。本想卖掉老婆能使我们都活下来,如果老婆卖掉后我还脱不了死,那又何必呢?我不敢讲价钱,只求你行个好,积个阴德罢了。”刘某很可怜他俩,便问马店主能出多少钱。马店主说:“如今一个妇女最多值一百个大钱。”刘某请马店主不要少他要的三百文。他愿替出上一半。马店主坚决不答应。刘某年轻气盛,便对那男子说:“这个人粗俗小气,不值得再和他争。我情愿送你三百文钱,你能逃荒,夫妻俩又能在一起,不是更好吗?”于是解囊取出钱交给了他。夫妻俩哭着向刘某拜谢后才离去。刘某讲完这件事,李翠石对他大加赞叹。

自此以后,刘某先前的那种恶习全改了。现在刘某已经七十多岁,身体还很健康。去年李翠石去周村,碰上刘某和人争吵,围着许多人劝他,他也不听。李翠石笑着对他说:“你又想告桃树状吗?”刘某一听马上停止了争吵,脸上也没了怒气,一句话没再说,径直而去。


\subsection{1.7.3   邵 女}
\label{\detokenize{p00_u5176_u5b83/_u767d_u8bdd_u804a_u658b_u5fd7_u5f02:id266}}
太平地方有个叫柴廷宾的,妻子姓金,娶进门来不会生孩子,又特别爱“吃醋”。为了要孩子,柴廷宾花很多钱买了人小老婆,金氏就狠狠虐待,一年就死了。气得柴廷宾一个人睡了好几个月,再不进妻子的屋。

这一天,柴廷宾过生日,妻子好言好语,还用丰厚的礼物给他祝寿。柴不忍拒绝,这才重新与她有说有笑。妻在卧室里设下酒宴,请他进去,他推说喝醉了,不去。金氏打扮得漂漂亮亮,自己又来到丈夫屋里,说:“为了你过生日,我伺候了一整天,即使您真的醉了,也请去饮一杯。”柴廷宾这才进了卧室,边饮酒边与妻子说话。金氏从容地说:“上回害得你买回来的妾死了,我现在还后悔,可是你就记了仇。结发之情一点都没有了吗?从今往后你找十二个女人我也不说你点不是。”柴廷宾听了,更加欢喜,就留在妻子卧室和她同寝,从此和原来一样相亲相爱了。于是金氏就明里请媒婆给丈夫物色好的女人,暗中却又叮嘱媒婆拖延,即使真的找到了好的,也不要告诉丈夫,而她自己又装出着急的样子去督促媒婆。这样过了一年多,柴廷宾等急了,又托亲友花钱买妾,果然买到一个林家的养女。金氏见了,表面上很喜欢,让林女与自已一同吃饭,什么化妆品呀,首饰呀,由着林家女使用。

林女是被林家收养的私生女,没学过针线活儿,除了会绣花鞋,其它衣物都得依仗别人。金氏就批评说:“俺家从来节俭,不像王公贵族家,要你当画看。”就把些好看的花绸缎给她,叫她学女红,像严师教学生。开始还仅仅训斥两句,后来就渐渐发展到用鞭子打。柴廷宾见了,又心疼又没办法。金氏对林女却比过去更加爱护,常亲自替她打扮,帮她穿戴,给她搽粉。只是有一条:林女哪怕鞋跟有一点皱褶,金氏就用铁棍敲她的脚;头发稍乱一些,就用巴掌扇她的脸,逼得林家女受不了,终于上吊死了。柴廷宾心里十分难过,说了些埋怨妻子的话。金氏听了,反而发怒说:“我替你调教女人,难道错了吗?”这时,柴廷宾才明白了妻子的险恶用心,又和妻子翻了脸,发誓永远断绝夫妻关系,暗中在另一块宅基上盖了房子,打算再买到个女子,另过日子。

眨眼间半年,没找到。

这一天,柴廷宾参加一个朋友的葬礼,见到一位十六七岁的姑娘,美得耀眼。柴廷宾眼睛都看直了,魂都跑了。那姑娘不喜欢他这样子,转开目光不理他。柴廷宾一打听,姑娘姓邵,父亲穷,只有这么个女儿,从小聪明过人,读书过目不忘,尤其爱读《内经》和《冰书》,父亲很溺爱她,凡来说媒的,都叫她自己拿主意,可是不论富家子弟还是穷人后生她都不同意,因此十七岁了还没定下婆家。柴廷宾知道了这些情况,明白这是个不容易娶的姑娘。但心里总萦绕着她的影子,又希望因家中穷,多给钱财或许能打动她的心,就托媒人去说。找了几个,没一个敢去做媒的,柴也就灰了心,不抱希望了。

有一天,忽然有个姓贾的媒婆因贩卖珍珠路过柴家,柴廷宾就对她说了自己的愿望,并给她很多钱,说:“我只求你把我的意思通报给邵家,成不成都不怪你;万一有成功的希望,花钱再多我也不在乎。”贾媒婆贪图钱财,答应了他。到了邵家,有意识地和邵女的母亲拉家常。谈话间偶然看见了她美丽的女儿,故作惊讶说:“好俊的闺女,如是选进昭阳院,赵家姊妹算得什么?”又故意问:“女婿是谁家的公子?”邵母说:“还没找人家呢。”贾婆说:“这么好的闺女,还愁找不到王侯公子作女婿吗?”邵母叹气说:“王侯贵族我们不敢高攀,只求找个知书识礼的后生也就不错了。俺家这个小冤家,给说媒的也不少了,挑来挑去,十个里也没挑中一个,也不知她究竟想嫁个什么样的。”贾媒婆说:“夫人不用愁,这么好的闺女,不知哪家后生前世里修了多少德才有娶她的福份。昨天有件让人好笑的事:那个叫柴廷宾的书生,在谁家的葬礼上见过你家姑娘,相中了,说宁愿出千金聘礼呢。这不是癞蛤蟆想吃天鹅肉吗?真可笑,早叫我挖苦跑了。”邵母听了笑笑,不置可否。贾婆又说:“一般穷秀才不用谈了,若是有钱的人家,哪怕不是什么读书人,却也图个富贵,似乎还可以。”邵母仍然只笑不说话,叫人摸不透她的心思。贾媒婆忽然一拍巴掌,装出一副认为邵母已经同意了她的观点的神气,说:“哎呀呀,若真那样,我自己反不合算了。您想想,尽管夫人您没有架子,我多咱来多咱跟我促膝谈心,茶酒相待,若是您有了富亲戚,出入有车马,往来尽是楼阁大户,我再来了,,怕您那看大门的仆人还嫌我寒伧,喝斥我呢。”邵母听了,沉吟了许久,起身到后堂和丈夫说话去了。过了一会儿,听见叫他们的女儿。又过了一会儿,邵母和她丈夫、女儿一块儿出来了,笑着对贾婆说:“你说这个妮子怪不怪,多少好人家不愿嫁,听说去做妾,倒愿意了。不叫人家读书人笑话吗?”贾婆说:“不妨事,过了门,若生个男孩,正房妻子又能拿她怎样?”说完,又传达了柴廷宾准备把她女儿安顿在另一处房宅的意思。邵母更高兴了,对女儿说:“闺女,快向贾姥姥下个保证:这门亲事是你自己同意的,不后悔。以后不如意了,不埋怨爹妈。”邵女有些难为情地说: “爹娘放心,以后女儿一定好好孝敬二老。女儿自知命不好,若找个太好的人家,反倒活不长;找个不太好的人家,受点罪,受些委屈,也不见得是坏事。上回见柴家公子,看相貌是个有福之人,他的儿孙一定会有出息的。”

听了这话,贾媒婆高兴得去告诉柴廷宾。柴廷宾喜出望外,马上下了千金聘礼,用华贵的车马把邵女娶到别墅里。这件事,除了金氏,柴家上下全知道,可是谁也不敢说。

安顿下来以后,邵女对丈夫说:“郎君,你这个办法,就好比燕子把窝筑在飘动的布上,长不了的,还嘱咐家人不要走漏消息,这样的事要想永远瞒着是不可能的。依我看,不如早早进家去住,祸反而会小些。”柴廷宾怕她受金氏虐待,邵女说:“天下没有不可感化的人。我若是处处小心不犯过错,她有什么理由虐待我呢?”柴延宾同意她的道理,可不敢照着去办。

这天,柴有事不在家,邵女穿了朴素的衣服,吩咐一名老男仆牵匹老马,命一个老女仆带上个包袱,果断地到了金氏的住所,跪着把自己怎么到金家,怎么住在别院等原委如实说了。金氏这才知道还有这等事,而且发展到这等程度了,自己还蒙在鼓里,立时气了个半死。待要朝邵女发作吧,一想人家主动来向我坦白,是可以原谅的。又见她穿戴朴素、态度谦卑,气就消了些,于是吩咐丫头把好衣服拿来给她换了,悻悻地说:“姓柴的这个没良心的,对外人说我多么凶,我平白无故地被人家嚼舌头。其实全怪他,怪那个贱女人气的我。你想想,背着老婆另找女人,这还算个人吗?”邵女说:“我仔细观察他,好像有点后悔。不过放不下大男人架子,不肯在你面前认错罢了。俗话说‘大的不向小的低头’。按常礼,妻子和丈夫的大小,好比儿子和父亲,妾和正室那样。如果夫人您稍稍缓和一下,给他点好颜色,我看过去的隔阂就能消除。”金氏说:“他自己不来,我有好脸色给谁看去?”这时,金氏心静了,见邵女老跪着也不成样子,就吩咐使唤丫头给邵女收拾房间,叫她住下来。尽管心里还不是滋味儿,但总算暂时平安无事了。

柴延宾出门回来,听说邵女到了金氏那里,吓坏了,心想,羊进了虎群,早给金氏嚼得只剩骨头渣儿了。赶紧过去,进了门,见家里没一点动静,才放了心。邵女在门口迎着他,劝他快到金氏那边去。柴延宾有些为难,邵女就掉下泪来了。柴延宾这才接受了她的建议。邵女又到金氏面前说:“柴郎回来了,觉得没脸见你。我求你去给他个笑脸,好言好语说说话吧。”金氏听说柴回来了。心中就来气,不肯过去。邵女进一步劝道:“我不是说过么,夫和妻有大小之分。古时候有个叫孟光的女子,对待丈夫那真是恭敬极了,每逢吃饭,把饭端到额头高送到丈夫面前,别人知道了,不认为这是丢面子。为什么呢?因为她做的符合自己的妻子身份,符合大礼,夫人您主动去见柴郎,不失身份的呀。”金氏这才听从了她。

一见丈夫,金氏气哼哼地说:“好哇,你既然跟兔子一样有三个窝,还回来干什么?”柴延宾低头不语。邵女赶紧用胳膊肘碰碰他,他才无可奈何地笑了笑。妻子见他有了笑容,态度也就和缓下来。要转身回屋。邵女又推柴延宾快跟进去,一面又吩咐厨子准备酒菜,叫他们对饮了几杯。

从此,夫妻和好如初。邵女每日早早起来过去向金氏问安,伺候洗脸,洗了脸又递手巾,像婢女那样恭敬金氏。柴延宾若要到她屋飘来,她苦苦拒绝,十几天才留她住一夜。因此,金氏也觉得她贤惠知礼。但是又觉出自己不如邵女,由惭愧渐渐积累成了嫉妒。然而邵女处处谨慎,又找不出她的毛病。偶尔斥责她两句,她也俯首帖耳地听着。

一天夜里,柴、金二人吵了嘴。起床后梳妆时金氏还没消气,恰巧邵女不小心,将伺候她梳头的镜子掉在地上摔破了。金氏立刻火冒三丈,攥着还没梳好的头发,眼珠都要瞪出来了。吓得邵女赶紧跪下来求饶。金氏好容易抓住她的把柄,不肯饶她,拿起鞭子就抽了一顿。柴延宾实在看不下去,咚咚跑过来拉起邵女出了屋。金氏骂咧咧地还要追着打。柴延宾急了,夺过鞭子抽起她来,抽得她脸上流了血,她才退回房去。夫妻又跟仇人一样了。

从此,柴廷宾不准邵女再到金氏房中去。邵女不听,次日清晨,跪着用膝盖走到金氏门外,等她起床好伺候她。金氏知是邵女来了,捶着床骂,叫她滚。对邵女,她恨得咬牙切齿。拿定主意,等丈夫不在家狠狠收拾她。柴廷宾知道她有了这个念头,干脆不出门,跟外界不来往了。金氏就天天打女仆出气,打得下人们叫苦连天。自从夫妻决裂,邵女夜里也不敢留柴廷宾住了。弄得柴廷宾夜夜独宿。金氏知道后,明白了丈夫并未被邵女独占,心里稍稍好受了些。

柴家有个稍大点的婢女,很精。一次与主人偶然说了句话,金氏发现后怀疑她与丈夫有私情,就狠打了她一顿。恨得婢女常在背地里骂她。这天,轮到这婢女夜间伺候金氏。邵女嘱咐柴廷宾说:“今夜别到夫人房里去,我看那婢女面带杀机,不知安的什么心呢。”柴廷宾觉得有理,把那婢女叫来,诈问她:“今晚你想干什么?”婢女以为主人察觉了她的秘密,吓得说不出话来。柴廷宾见她这副佯子,更加疑惑,搜她身上,发现她带了一把锋利的刀子。这下,婢女无话可说,跪下来求饶说:“我该死,我该死。” 柴想打她,邵女劝阻说:“别忙。你一打她,事情就张扬开了。若被夫人知晓,这婢子还活得了吗?她的罪固然是不可饶恕的,我看不如把她卖出去,既可保住她的性命,咱家又可得点收入不是?”柴廷宾同意,正好有个人家要买妾,柴就赶紧把她卖了。

金氏发现少了那个婢女,一问,知道是丈夫卖了,就怪丈夫不同她商量;又听说丈夫是采纳了邵女的意见,又怪起邵女来,用很恶毒的话骂她。连柴廷宾也埋怨邵女:“都是你自找的。你若不管闲事,容那婢女杀了她,哪还有这些麻烦?”金氏听了“杀”字,感到奇怪。问下人,没一个知道的。问邵女,邵女也不说。金氏又纳闷又生气,提着裙子跳着脚骂。柴廷宾听不下去,就把事实告诉了她。金氏大吃一惊,才知是邵女救了自己,对邵女就温和了些,可是心中又怪邵女为什么不早说。柴廷宾见金氏态度缓和以为没事了,就出了远门。

金氏趁丈夫不在家,把邵女叫来数落她:“不该饶了那个要杀我的小蹄子,你为什么把她放走了?”邵女一时找不到合适的话回答。金氏想:这回可抓住你的不是了——跟杀主人的婢子一鼻孔出气呀,非狠狠治你不可!就把铁烧红,烙邵女的脸,想把她的面容毁了。家中女仆全替邵女抱不平。每烙一下,邵女就哀号一声,佣人们哭着请求替邵女受刑。金氏不答应,又改用针刺邵女的胸肋,连刺了二十多下,这才觉得出了气,说:“滚!”

过了些日子,柴廷宾回来了。见邵女脸上有烙伤,问明情由,气得立刻要找金氏算帐。邵女拉住他的衣服劝道:“是我自愿来跳这火坑的。我嫁你,难道因为你家是天堂吗?我自知命不好,只有找罪受,老天爷才能消气。只要我受得了,就受,这样或许有个出头之日。若再触怒了老天爷,不就像填坑填了一半又去挖一样前功尽弃吗?”她就用烫伤药自己搽伤,几天就好了。一照镜子,高兴地说:“柴郎,为我庆贺吧。夫人这一烙,把我脸上那条倒霉的纹给烙断了!”便一如往常地侍奉金氏。

金氏见上回全家的佣人都为邵女痛哭求情,明白大家都恨自己,有点懊悔,就常和颜悦色地叫邵女跟自己一块儿做事情。过了一个多月,金氏突然得了打嗝病。一吃饭就嗝得厉害,影响饮食。柴廷宾本来就恨她死得晚,根本不管她的病。她的肚子几天后胀得像鼓那么大。一天到晚只想睡觉,下不来床。邵女顾不上吃饭和休息,伺候她。她很感激,邵女又对她讲些医药方面的道理,可金氏怀疑;我过去对她太惨酷,她会不会弄毒药毒死我?金氏不听邵女的什么医理,还装出感谢的样子,病当然不见好转。

金氏这个人,尽管人人恨,还是有优点的,那就是治家很严,佣人很服从她;自她得病后,不能过问家政,佣人就懒散了。有些活儿就没人干。柴廷宾只好自己管理,累得够呛还管不好,甚至有人往外偷东西。柴廷宾这才感到金氏这个内当家的重要,就认真给她请医生治病。对自己的病,金氏心里也没数,别人问起来,只说是得了气鼓。大夫们也就确诊为积住气了。换了几个大夫,都不见效。病越来越重,都快不行了。

这天又煎药,邵女建议说:“医生开的这药,吃一百副也不顶用,甚至越吃越重。”金氏不信,还叫她照老方子煎。邵女偷偷换了方剂,金氏服下,一顿饭功夫泄了三次,马上觉得好了,就笑话邵女刚才说的不对,还是老方子好,还笑着讽刺她:“喂,你这个女华佗,怎么样啊?”邵女和佣人都忍不住要笑。金氏被笑得莫名其妙,追问起来,邵女才把实情说了。金氏感动地说:“该死!我天天受你的爱护,竟还蒙在鼓里。从今天起,家里的事全听你的。”不久,病全好了。柴廷宾高兴地摆酒席为她庆贺,邵女站着执酒壶。金氏不让,夺下酒壶拉邵女挨着自己坐下,亲热得不行。到了夜深,该安歇了,邵女找了个借日要离开,好让他们夫妇同眠。金氏不依,派两个婢女硬把邵女拉住,硬要她和自己一床。从此,两人同吃同住,同宿同商量,赛过亲姊妹。

不久,邵女生了个男孩,产后总是闹病,金氏像孝敬母亲一样伺候她。

不多天,金氏又病了,心口疼,疼起来脸都发青,恨不得死了才好。邵女赶紧买了几根银针给她按穴位扎上,疼得要死的金氏立刻不疼了。十来天又犯了,再扎;六七天又犯了,再扎。弄得金氏天天提心吊胆地怕再犯。一天夜里,她梦见到了一座庙里,大殿里的鬼神全能活动,一个神问她:“你是金氏吗?你的罪孽太重了,早该死,念你已有悔改表现,才只让你害病,表示神灵对你的谴责。你害死过两个女人,是她们应得的报应。可是邵女有什么罪?你对她这么狠毒!你用鞭子打她,已由你丈夫替神灵报应给你了,这个可以抵消;另外,你还欠了一次烙和二十三次针扎的帐,现在邵女已经扎过你三次,刚刚报应了零数,你的病就想除根呀?明天又该犯了。”醒来后,金氏心中害怕,又认为梦不可信,早饭后真的又犯了病,而且疼得更厉害了。邵女也纳闷,说:“光用针扎怎么老除不了病根呢?我看得用烧红的针扎,把穴位烧烂了也许能除根,可就是怕夫人您受不住。”金氏想起了梦,并不怕,同意了。她边挨针边想,欠下的十九针,不知道还要害什么样的怪病才能抵偿,不如一天扎够,也许能免了受不完的苦。扎过了一柱香的功夫,又求邵女再扎,邵女笑道:“针是随便乱扎的吗?得按穴位。”金氏说:“什么穴位不穴位,你给我扎十九下就是了。”邵女又笑了:“不行,不行。”金氏在床上跪起来苦苦哀求,邵女总是不忍心。金氏把梦告诉了她,她才约摸着经络上的有效部位给她扎了十九针。

从此,金氏完全康复,没再犯。又因真正悔过,心理平衡,在下人面前也没有了愧心的样子。

邵女的儿子叫柴俊,聪慧过人。邵女常说这孩子有作翰林的相貌。八岁,人称神童;十五岁,中了进士。这年,柴廷宾夫妇四十岁。邵女三十二三岁。孩子做了大官,车呀马的回家看老父母,乡亲们都夸奖。邵女的父亲自从千金卖了闺女,就富起来了;但也真的被读书人瞧不起,直到柴俊有了功名,才有人跟他往来。


\subsection{1.7.4   巩 仙}
\label{\detokenize{p00_u5176_u5b83/_u767d_u8bdd_u804a_u658b_u5fd7_u5f02:id267}}
有一个姓巩的道士,没有名字,也不知道是什么地方人。一次,他去求见鲁王,看门人不给通报,这时有位宫中的宦官出来,道士便求他引见。宦官见他又穷又土,将他赶走了。可是道士马上又回来了,宦官很生气,派人边打边撵。赶到没人的地方,道士笑着拿出百两黄金,请追赶的人回复宦官:“就说我不是要见鲁王,听说王宫后院的花草树木、亭台楼阁是世间最美的景致,如果能领我看一看,这一生就满足了。”接着又拿出些银子给他,那人高兴地回报去了。宦官也很高兴,领道士从王府的后门进去,游览了所有的景地。道士又跟着登上楼台。宦官走到窗口眺望,被道士一推,只觉得身子从楼上掉下来,腰被细藤缠住,悬挂在半空中;往下一看深不见底,头晕目眩,细藤也隐隐发出格崩的断裂声。他害怕极了,大声号叫起来。有几个内监闻声赶来,见状惊恐万分。见他离地很高,上楼一看,细藤拴在窗棂上,想拨藤救他,又怕藤太细会拉断。到处寻找道士,却不见踪影。实在没有办法,只好禀报鲁王。鲁王亲自去察看,也感到非常惊奇。便令人在楼下铺上茅草和棉絮,以便将细藤割断。楼下刚铺垫好,细藤“砰”的一声崩断了。宦官竟然离地不到一尺。大家忍不住笑了起来。

鲁王命人去寻访这位道士,得知他住在尚秀才家,便派人去问,说出游没有回来。差人回府途中正巧遇上了道士。便领他去见鲁王。鲁王设宴款待,请道士表演幻术。道士说:“我是个山乡野人,没有别的本事,承蒙您的厚待,就献一班歌女为大王祝寿吧。”说完,从袖子中拿出个美人放在地上。那美人向鲁王叩拜。道士命美人扮演“瑶池宴”为鲁王祝寿。美人说了几句开场白,道士又拿出一人,那人自称王母娘娘。一会儿,董双成、许飞琼等仙女都先后出场;最后,织女出来拜见,并献上一件天衣,宫里顿时金光灿烂,一片通明。鲁王怀疑天衣是假的,想要来看看,道士急忙说:“不可!”鲁王不听,拿来一看,果然是无缝天衣,不是人间可以做的。道士很不高兴地说:“我实心实意奉承大王,才从天孙那儿暂时借来天衣,如今天衣被俗气玷污,让我怎么还给主人呢?”鲁王又觉得仙女也一定是真的,想留下一两个,可仔细一看,原来都是自己宫中的歌女。又怀疑刚才唱的曲子并不是她们熟悉的,一问,歌女们果然连自己也不知道。道士把那件天衣烧了,然后把灰放在袖中,再搜看时,却什么也没有了。鲁王因此对道士十分敬重,想留他住在府中,道士说:“我游荡惯了,这宫殿就如同牢笼,不如住在秀才家里自由。”从此道士经常出入王府,但每到半夜必然回去。有时坚决留他,也偶尔住下。道士常在宴席间表演四季花木颠倒时序的游戏。鲁王问他:“听说仙人也不忘男女之情,是真的吗?”道士回答: “也许是这样吧,可我不是仙人,所以心如枯木。”一天晚上,道士住在府里,鲁王叫一个年轻貌美的妓女去试探他。妓女进了房门,连叫几声,没人答应,点了灯一看,道士像死人一样闭着眼坐在床上。摇晃他,眼一睁又闭上了;再摇他,打起了呼噜。推他,又顺势倒下,卧床而睡,酣声如雷。妓女用手弹弹他的额头,发出像敲击铁器一般的声音,便急忙去禀报鲁王。鲁王让人用针刺道士,针扎不进去,推他,重得摇不动。又召来十几个人把他举起扔到床下,就像一块千斤重石落在地上。天亮以后去看看,道士仍然睡在地上。道士醒后笑着说:“睡得真死,掉下床来也不知道!”以后这些妓女们常在道士坐卧时按着他玩,刚按时还软和,再按就硬得像石头一样了。

道士住在尚秀才家经常半夜不回来。有时尚秀才锁了门,等天明开开房门一看,道士已经睡在屋里了。以前,尚秀才和一个叫惠哥的歌妓很要好,两人立誓结为夫妻,惠哥歌唱得特别好,演奏技艺也超群出众。鲁王听说惠哥很有名气,就召入宫内侍奉自己。从此,惠哥和尚秀才断绝了交往,虽然常相互思念,却无法见面。一天晚上,尚秀才问道士: “你在宫中见过惠哥没有?”道士说:“那些歌女我都见过,但不知谁是惠哥。”尚秀才把惠哥的年龄相貌描述了一遍,道士想了起来。尚秀才求他再去时给转达一句话,道士笑着说:“我是世外之人,不能替你捎书传信。”尚秀才苦苦哀求,道士只好展开袖袍说:“你如果一定要见惠哥,就请钻进我的袖子里来吧。”尚秀才往袖子里一看,见里面大得像屋子,便伏身进去,里面光明洞彻,宽若厅堂,桌椅床帐无所不有,而且在里面一点也不觉得气闷。道士来到王府内,与鲁王下棋。他见惠哥走来,便佯装用袍袖拂尘,将惠哥装进袖内,别人一点也没发觉。尚秀才正独坐沉思时,忽见从屋檐掉下一个美人,一看是惠哥。两人惊喜万分,你拥我抱,亲热异常。秀才说:“今日奇缘,不能不记下来。我们来对诗吧。”说完先在墙壁写了:“侯门似海久无踪,”惠哥续写:“谁识萧郎今又逢,”秀才写:“袖里乾坤真个大,”惠哥续道:“离人思妇尽包容。”刚题完,忽然进来五个人,头戴八角帽,身穿淡红衣,都是不相识的人。他们一声不响,把惠哥提了就走。尚秀才吓得不行,不知怎么回事。道士回到秀才家里,把秀才叫出来,问他在里面的事情。秀才隐瞒着没有全部说出来。道士微笑着把衣袖翻过来让他看,秀才见上面隐隐约约有些字迹。细得像虮子一样,仔细辨认,原来是他题的诗句。过了十多天,尚秀才又求道士带他去了一次。先后共去了三次。惠哥告诉秀才说:“我已感到腹中胎动,非常担忧,只好用带子把腰扎紧。可是王府中耳目众多,倘若有一天临产,小孩一哭,往什么地方藏?麻烦你和巩道士商量一下,见到我三叉腰时,请他设法救我。”尚秀才答应了。回去后见了道士跪在地上不起来,道士扶起他来说:“你要说的话,我都知道了。请你放心,你尚家就靠这一点骨血传宗接代,我怎敢不尽力帮助呢?但从现在起你不能再进王府了。我所以报答你的,原不在儿女私情呀!”几个月过后,道士从外面回来,笑着说:“我给你把儿子带来了,快拿小孩包被来!”尚秀才的妻子非常贤惠,快三十岁了,生了几胎只活下一个儿子。最近又生了个女儿,刚满月就死了。听尚秀才一说,惊喜地走出来。道士从衣袖中取出婴儿,脐带还没断,睡得正甜呢。秀才的妻子接过来抱在怀里,婴儿才呱呱啼哭起来。道士脱下衣服说:“产血溅在衣服上,是道家最大的忌讳。今天为了你,二十年的旧物,只好扔了!”尚秀才为道士换了一件新衣袍,道士嘱咐他说:“旧衣服不要扔了,烧一钱灰吃了,可治难产,堕死胎。”尚秀才记在心里。

道士在尚秀才家又住了一些时候,忽然对秀才说:“你收藏的那件旧衣服,应当留下一些自己用,我死了你也别忘了!”尚秀才觉得道士的话不吉利。道士转身就走了。道士进王府对鲁王说:“我快要死了!”鲁王很惊奇,道士说:“人的生死都是有定数的,还有什么可说的呢?”鲁王不信,强把他留下。道士刚下了一盘棋,急忙起身要走,鲁王又把他拉住。道士请求到外屋休息,鲁王答应了。鲁王去看时,见道士已经死了。鲁王备了上等棺木,按当地礼节把他葬了。尚秀才亲到坟前哭吊一场,这才醒悟到道士原先说的话是预先告诉他的。道士留下的旧衣用来催生,十分灵验,求尚秀才医治的人接连不断。开始只是剪被产血玷污的袖子给人,后来衣袖用完了,又剪领襟给人,也很有效。他想起道士嘱咐的话,怀疑妻子日后必定难产,就剪下巴掌大的一块血布珍藏起来。后来鲁王有个爱妃临盆三天生不下来,医生都没有办法。有人告诉鲁王尚秀才能治,鲁王立刻召他进府。那妃子只服了一剂就生下来了。鲁王非常高兴,赠给尚秀才银钱绸缎,尚秀才全部推辞不要。鲁王问他要什么,秀才说: “我不敢说。”鲁王请他说,秀才叩头,说:“实在要赏我,就请把歌女惠哥赐给我,我也就心满意足了。”鲁王把惠哥召来,问她年龄,惠哥说:“我十八岁入府,至今已十四年了。”鲁王觉得惠哥年龄太大,便命将全部歌妓都叫来,任尚秀才挑选。秀才却一个也不喜欢,鲁王笑着说:“真是个书呆子!你们俩十年前就定了婚约吗?”尚秀才将实情说了。鲁王备好车马,仍把尚秀才辞掉的银钱、绸缎给惠哥当嫁妆,把他们送到家中。惠哥生的儿子取名秀生,取“秀”与“袖”同音之意,这年秀生十一岁。尚秀才家时刻不忘巩仙人的恩德,每逢清明都到他坟上祭扫。

有个长年旅居四川的客人,在路上遇见巩道士。道士拿出一本书说:“这是王府的东西,我来时匆忙没来得及归还,麻烦你捎去。”客人回来听说道士早死了,不敢贸然去见鲁王。尚秀才知道后替他回奏了。鲁王打开书一看,果然是以前道士借去的。鲁王起了疑心,挖开道士的坟墓一看,却是一副空棺材。后来,尚秀才的大儿子年龄不大就死了,全靠秀生顶立尚家的门户,传宗接代。固而,尚秀才更佩服巩道士的先见之明了。


\subsection{1.7.5   二 商}
\label{\detokenize{p00_u5176_u5b83/_u767d_u8bdd_u804a_u658b_u5fd7_u5f02:id268}}
莒县有个姓商的人家,哥哥家很富,弟弟家很穷,两家只隔一道墙。康熙年间,一个灾荒年,弟弟穷得揭不开锅。一天,天过晌了,弟弟还没生火做饭,饿得肚子咕噜叫,愁得走来走去,没有一点办法。妻子叫他去求哥哥,二商说:“没用!要是哥哥可怜咱们穷的话,早就来帮助我们了。”妻子执意要他去,二商就让儿子去。过了一会儿,儿子空手回来了。二商说:“怎么样?我说的不错吧?”妻子详细问儿子大伯说了些什么,儿子说:“大伯犹豫地看看大伯母,伯母对我说:‘兄弟已经分家,各家吃各家的饭,谁也不能顾谁了。’”二商两口子无活可说,只好把仅有的破旧家什卖掉,换点秕糠来糊口。

村里有三四个无赖,窥测到大商家里很富裕,半夜里翻过墙头,钻进大商家。大商两口子听见动静,从睡梦中惊醒,敲起脸盆大声喊叫。邻居们因为大商家太刻薄,谁也不去援救。大商家没有办法,只得大声呼喊二商。二商听到嫂子呼救,想去救助,妻子一把拉住他,大声对嫂子说:“兄弟已经分家,谁有祸谁受,谁也顾不了谁呀!”不一会,强盗砸开屋门,抓住大商两口子,用烧红的烙铁烙他们,惨叫声阵阵传来。二商说:“他们虽然不讲情义,可哪有看到哥哥被害死而不去救的!”说着带领儿子大声喊叫着翻过墙头。二商父子本来就武艺高强,远近闻名;强盗又怕招来众邻援助,就四散逃走了。二商看到哥嫂的两腿都被烙焦了,忙把他们扶到床上,又把大商家的奴仆召集起来,才回家去。大商家虽然人受了酷刑,而钱财却一点没丢。大商对妻子说:“如今咱能保全财产,全靠弟弟解救,应该分一点给他。”妻子说:“你要是有个好弟弟,还不受这份罪呢!”大商不再吭声了。二商家连糠菜都没有了,满以为哥哥会送点东西来报答他。可是过了很久,也没听到动静。二商的妻子等不得了,叫儿子拿着口袋去借粮,结果只借了一斗粮回来。二商妻子嫌少,生气地让儿子送回去,二商劝住了。又过了两个月,二商家穷得实在熬不住了。二商说:“如今实在没有办法可以糊口了,不如把房子卖给哥哥。哥哥如果怕我们离开他,或许会不接受我们的房产,想办法接济我们呢。就算不是这样,卖得十来两银子,也可维持度日啊!”妻子觉得也只有这样了,就让儿子拿了房契去找大商。大商把这事告诉妻子,说:“就算弟弟不仁义,也是同胞手足。他们如果走了,我们就孤立了,不如归还田契,再周济他们一点。”妻子说:“不行。他说走是要挟我们。如果信了他,就正好中了他的圈套。世上没有兄弟的人难道都死了吗?我们把院墙加高,足可以自卫了。不如收下他的房契,他爱上哪上哪好了,也可以扩大我们的宅院。”商量好了,就叫二商在房契上签字画押,付给房钱。二商只好搬到邻村去了。

村里那几个无赖,听说二商走了,又来抢劫,抓住大商鞭抽、棍打,用尽毒刑。大商只好把所有的金银财物,都用来赎命。强盗临走的时候,打开大商家的米仓,招呼村里的穷人随便拿。顷刻之间米仓就空了。第二天,二商才听说这事,急忙赶来看望。可是,大商已经神志昏迷,不能说话了。他强睁开眼,看见弟弟,只能用手抓挠床席,不一会儿就死了。二商忿怒地去找县官告状。可强盗头子早已逃走了,没有逮到,那些抢粮食的都是村里的穷人,州官对他们也无可奈何。大商撇下的小儿子,才五岁。自从家中穷了以后,他常常自己到叔叔家,好几天不回去。送他回去,就哭个没完,二商的妻子对这孩子白眼相待,二商就说:“孩子的父亲不仁义,孩子有什么错呢?”就到街上买了几个蒸饼,送孩子回去。过了几天又背着妻子,偷偷地拿了一斗米给嫂子送去,让她抚养儿子。就这样常常接济他们。又过了几年,大商媳妇卖掉了他家的田产,母子俩的生活能维持了,二商才不再接济她们。又一年,闹灾荒,路上到处可以看见饿死的人。二商家吃饭的人多了,不能再去照顾别人。侄子这年只有十五岁,年小体弱不能干重活,二商就让他挎个篮子,跟哥哥们卖烧饼。一天晚上,二商梦见哥哥来了,神情凄惨地说:“我被老婆的话所迷惑,丢了手足情分。弟弟不计较从前的怨仇,更使我羞愧得无地自容。你以前卖给我的房产,如今空着,你就搬去住吧。屋后乱草下面的地窖里藏着一些钱,把它拿出来,也能过上温饱日子。就让我的儿子跟着你吧。那个长舌头老婆,我最恨她!你就别管她了。”二商醒来以后,觉得很奇怪,就用高价租回房子。住进去以后,果然在房后挖出了五百两银子。从此,不再做小买卖,而让儿子和侄子在街市上开了一家店铺。侄儿非常聪明,帐目从来没有差错,又忠厚诚恳,就是出入很少一点钱,也一定告诉哥哥,二商非常喜爱他。一天,侄儿哭着为母亲要点米,二商的妻子想不给她。二商看在侄儿的一份孝心上,就按月给嫂子一些粮食。过了几年,二商家越来越富裕了。不久,大商媳妇生病死了。二商也老了,就和侄儿分了家,把家产的一半分给了侄子。


\subsection{1.7.6   沂 水 秀 才}
\label{\detokenize{p00_u5176_u5b83/_u767d_u8bdd_u804a_u658b_u5fd7_u5f02:id269}}
山东沂水有个秀才,在山中温习功课。夜里,有两个美女进了屋,含笑不说话,各自用长袖拂了一下床,就挨着坐下了。她们的衣服轻软,不带一点声息。一会儿,一个美女站起来,将一条白绫巾展放在桌子上,巾上有草书文字三四行,秀才也没仔细看看写的是什么词句。另一个美女起身把一锭白银放在桌子上,大约有三四两的样子,秀才便把银子放进自已的袖子里。两个美女拿起白绫巾,拉着手笑着出了门,说:“真是俗不可耐!”秀才用手一摸袖子里,银子早已没有了。

美人坐在面前,投以情愫,秀才竟然置之不顾,而却把银子拿起来,这纯是一副乞丐相,能令人可耐吗!讨人喜欢的狐女,那高雅的样子可以想见。

朋友说了这件事,使我又想到了一些令人不可耐的事情,一并附记在这里:穷酸俗气;大老粗拽文;炫耀富贵;秀才装名士;谄媚丑态;不住嘴的信口扯谎;入座时苦让上下位;强逼人听看不像样的诗文;守财奴哭穷;喝醉了无理纠缠;学作满洲腔调;摆一付硬逼人说话的架势;开低级下流的玩笑;娇怂自己的孩子爬登筵桌抓肴果;凭借别人余威装模作样;低劣的科甲出身者大谈诗文;说话之间屡称自己是权贵亲戚。


\subsection{1.7.7   梅 女}
\label{\detokenize{p00_u5176_u5b83/_u767d_u8bdd_u804a_u658b_u5fd7_u5f02:id270}}
太行人封云亭,青年丧妻,十分寂寞,便到府城星去散心。有一天正在旅店里歇息,一阵睡意朦胧,隐隐约约地看见墙上显出一个年轻女子的身影,像是一幅画悬在那里。起初封生还嘲笑自己想老婆想疯了,可凝神注视了好半天,画影并不消失;再凑近细瞧,更清晰了:真真切切一个少女,却是一脸苦相,伸着舌头,脖上还挂着绳套。封生正在惊愕不定,那少女却像要从墙上慢慢走下来。封生知道碰上吊死鬼了,然而大白天,胆子总是壮些,便说:“娘子不必吓唬小生。您如有奇冤,小生可以为您效力。”这一说,女子身影真地落下来了,说:“你我萍水相逢,怎敢贸然以大事相托呢?然而九泉之下的枯骨,这么多年了,舌头缩不回去,绳套也脱不掉,实在是苦不堪言。求求您,让主人砍断这屋梁,烧掉它,您对我就恩重如山了。”封生答应去办,影子也就消失了。封生就招呼店主人来,打听这是怎么回事。店主人介绍说:“十多年前,这里是梅家的住宅。一天夜里小偷进来,被梅家逮住了,送到县府里交给典史。不料典史接受了小偷的三百文钱贿赂,竟诬陷梅家女儿与小偷通奸,要把梅女拘上大堂,让法医检验。梅女听说后,就上吊死了。不久,梅家夫妇也相继去世,宅院就归了我。这些年,旅客常说见鬼见怪的,可总也没法儿让它安静下来。”封生便把吊死鬼的要求转达给店主人。店主人一盘算,拆掉房顶换大梁,耗资太大,负担不起,面有难色。封生便慷慨解囊相助,完成了这项工程。修好之后,封生依旧住在这座房子里。

夜间,梅女来了,翩翩然一个万福,向封生表示感谢。言谈之间,喜气洋洋,举手投足,窈窕轻盈,原来是个十分秀气的姑娘。封生不禁油然而生爱慕之心,侮女却凄然而又羞涩地说:“鬼的阴气,对您是有害的。再说这样私合,我生前的耻辱,岂不是淘尽两江之水也洗不清了吗?咱们将来肯定会美满地结合,现在还不到时候。”封生忙问:“要到什么时候?”梅女嫣然一笑,不再作声。封生说:“喝点酒吧?,梅女说:“我不会饮酒。”封生不禁笑起来:“面对美人,光是默默地对着眼儿看,又有什么味道啊!”梅女说:“我生平的喜好,只有下打马棋。可是只两人下也不热闹;再说深更半夜的,也没处去找棋盘。的确,长夜也够难打发的,那我就跟您玩翻线花的游戏吧。”封生只好依他。两人促膝盘坐,封生叉开手指,梅女翻弄起来。真没想到,这小小玩艺儿,竟然变幻无穷。工夫一长,封生竟糊涂起来,不知该如何动作了。梅女笑着教他,又用眼神示意,愈变愈奇,愈奇愈妙。封生乐不可支地说:“这真是闺房里的绝技啊!”梅女说:“这玩法是我自己悟出来的。只要有这两根线,就可以织成任何花纹图案,不过一般人不细心揣摩罢了。”夜深了,玩累了,梅女就让封生就寝。她说:“我是阴间的人。是不睡觉的。你自己歇息吧。我小时候懂点按摩术,愿意奉献小技,帮您做个美梦吧。”梅女开始按摩,先是两手叠起,轻揉慢搓,从头到脚按摩一遍。梅女细手所过之处,封生觉得骨肉松缓,像醉了似的,懒洋洋的。接着梅女又轻握拳头细细捶擂了一遍,封生更觉得如同被棉絮团儿敲打一样,浑身舒畅,妙不可言。擂到腰间,已经闭目合眼,懒懒地要睡了。到大腿,已经沉沉进入梦乡。

封生一觉醒来,已是第二天中午。起床后只觉骨节轻松,浑身清爽,心里更加爱慕梅女,绕着屋墙呼唤她的名字,却没有声音答应。晚间,梅女才来了。封生心急地问:“你究意住在哪里?叫我呼唤了个遍!”梅女笑笑说:“鬼哪有一定的住处,总之在地下就是了。”封生忙问:“地下有缝,能容下你吗?”梅女又说:“鬼不见地,如同鱼不见水一样。”封生握住梅女的手说:“只要能让你活过来,我倾家荡产,在所不惜!”梅女笑了笑说:“也用不着倾家荡产。”两人又开始玩翻线花的游戏,直到深夜。封生又苦苦逼迫梅女,梅女说:“你别缠我。有个浙江妓女,名叫爱卿,挺风流标致的,新近就住在北邻。明天晚上我招她来暂且陪你如何?”第二天晚上,梅女果然领来一个少妇,看去约三十岁,顾盼巧笑,媚眼飞情,一派风骚放荡,这便是妓女爱卿了。三人凑在一起下“打马棋”,棋罢梅女告辞,爱卿陪封生过夜。封生询问爱卿的家世,爱卿含含糊糊,不肯明说,只是说:“您如果喜欢我,就用手指弹弹北间的墙壁,小声喊‘壶卢子’,我就会来。如果喊三声还没人答应,那就是我没空儿,就别再喊了。”天明时,爱卿果然隐身到北墙上消失了。第二天晚上,梅女一个人来了,封生问爱卿为何不来,梅女说:“被高公子招去陪酒去了。”两人坐下剪明灯烛叙谈起来。正在兴浓之际,梅女却沉默了。一会儿动动嘴唇,像有话要说,可话到嘴边又不出口。封生再三追问,梅女只是抽泣流泪,始终不肯明言。封生勉强拉她翻线花,到底打不起精神来,四更天便走了。

此后,梅女常与爱卿一起到封生住处来,说笑声通宵达旦,因而这事传遍了全城,远近皆知。恰巧有位典史,家庭本是浙江的世族,因妻子与仆人通奸,被他休掉了;又娶了一个顾氏,感情倒是很好,不幸才一个多月就死了,所以心里老是思念她。现在听说封生有两个鬼友,想向他打听一点阴间情况,看自己与顾氏还有无缘分,于是骑马来拜访封生。起初,封生不肯应承,经不起这位典史苦苦哀求,便设筵请典史饮酒,答应晚间招鬼妓来商量。日落天黑,室内暗下来之后,封生走到北墙,边敲边小声呼唤了三声。话音未落,爱卿已经出现了。谁知她抬头一见典史,面色突变,扭头便走。封生正要上前拦阻,这位典史早已气得抓起一个大碗猛投过去,随着“哗啦”一声响,爱卿飘然消失了。封生大吃一惊,正要问是何缘故,忽然一个老太婆从暗室里冒出来,开口便骂:“你这贪财害命的黑心贼!你砸坏了我家的摇钱树!得赔我三十吊钱!”一边骂,一边抡起拐杖就打,恰巧打到典史的头顶上。典史抱头哀哭着喊:“那女子是顾氏,我老婆呀!我还正为她年轻轻的死了而哀痛呢,谁想到她作了鬼还不正经!可这与你这老婆子有何相干呢?”老太婆气冲冲地斥责他说:“你本不过是浙江的一个无赖地痞,花钱买了这个臭官,戴上这条乌角带子,鼻梁骨就倒竖起来朝了天啦!你当官有什么黑白?袖里有三百钱贿赂你,就是你亲爹!你这神怒人怨的东西,死期就在眼前了!是你爹娘在阴司里再三哀求,情愿让你媳妇入青楼当妓女,替你偿还那些贪债,你自己还蒙在鼓里哪!”说罢,抡起拐杖又打,典史吓得在地上打滚哀叫。封生在旁边又惊讶又着急,又想不出办法排解。忽见梅女从房中出来,一见典史,登时气得张目结舌,脸色全白了,扑过来摘下头簪照典史就刺。封生更吓坏了,赶紧用身子遮住典史,劝说:“他即使有罪,可死在这里,小生就不好交待了。请您千万投鼠忌器吧!”梅女一想,这才住手;又拉住老太婆:“那就为我封郎着想,暂时叫他再活一煞吧!”这位典史一见,慌忙抱头鼠窜而去。听说回到衙门就患了头疼,半夜就死了。

第二天晚上,梅女来了,一见面就兴高采烈地说:“真痛快!总算出了这口恶气!”封生这才问:“你们究竟有何仇怨?”梅女说:“不是早就告诉你了吗?受贿诬奸的,就是这家伙!我含冤已经多年了。每每想求你替我伸冤昭雪,总是自愧对你还没半点好处,所以才欲言又止。昨天碰巧听见打架,偷偷一听,没承想正是仇人!”封生也惊讶地说: “原来他就是诬害你的那个坏蛋!”梅女说:“他在这县里当典史十八年了,我含冤而死也十六年了!”封生又问老太婆是谁,梅女说是一个老鸨儿;又问爱卿,梅女说:“她正在生病呢。”

大冤已报,梅女这才微笑着对封生说:“我当初说过结合有期,现在不远了。你曾说过情愿倾家荡产赎我,自己还记着吗?”封生说:“今天还是那份心思。”梅女说:“实话告诉你吧:我死的那天就已经转生在延安展孝廉家了。只因为大仇未报,所以至今滞留在这里。现在请你用新布做一个小口袋把我的鬼魂装上,让我随着你去。你到那里就向展家求婚,我保证他家一定答应。”封生还担心两家门第相差悬殊,不一定成功。梅女说:“放心,只管去吧。”又嘱咐封生说:“途中千万别呼唤我。待到成婚的晚上,将小布袋挂在新娘子头上,赶紧呼唤‘莫忘莫忘’,就大功告成了。”封生一一答应着。准备停当后,封生把小布袋打开,梅女跳了进去,然后一齐上延安。

延安果然有个展孝廉,有个姑娘,长相挺俊,就是有痴呆病,舌头又常伸在唇外,就像大热天狗喘气一样,难看又吓人,所以十六岁了,没有敢来提亲的,这简直成了爹娘的一块心病。封生先登门递上帖子,介绍了自家情况;然后托媒说亲。展家自然高兴,便把封生招赘到家中来。举行婚礼的时候,新娘子依然傻乎乎的,什么礼节也不懂,两个婢女一边一个扶着拖着才进了洞房。婢女们离开后,她竟然解开上衣大襟,露出乳房,直冲着封生憨笑。封生便取出小布袋挂在新娘子头上低声呼唤起来:“莫忘莫忘!” 新娘子听到呼唤声,沉思起来,凝神对封生端详着,目光渐渐亮起来。封生笑着说:“您不认得小生了吗?”又举着小布袋摇晃摇晃,新娘子清醒了,这才急忙掩上衣衿,两人亲亲热热说笑起来。第二天清早,封生先上堂拜见岳父。展举人安慰他说:“我闺女痴呆无知,蒙你看得起,既然成了亲,你如有意,我家有些聪明丫鬟,你看中哪个,我一定赠给你,决不吝惜。”封生竭力辩白,说小姐并不傻,举人倒疑惑不解起来。一会儿,女儿也上堂来拜亲,举止大方知礼,举人更加惊异,女儿微微一笑而已。举人询问其中缘故,女儿羞涩难说,还是封生从旁把情由大体述说一番。举人更加高兴,比以前更疼爱这个女儿。从此让儿子大成与封生一块儿读书学习,一切供应都很丰盛。

过了一年多,先是大成逐渐对封生流露出瞧不起的神色,郎舅之间不再和睦;接着奴仆们也看人下菜碟,开始在主人面前讲封生的坏话。展举人听多了流言蜚语,对封生的礼数也不那么讲究了。展女觉察到这些,就劝封生说:“丈人家终究不是长久住处。那些长住丈人家的,全是些废物。趁现在还没有大裂痕,咱还是早点回家吧。”封生也深以为然,于是向岳父告辞。举人想留下闺女,展女不愿意。这一来,父亲加兄长都火了,索性不给车马。展女便拿出自己的首饰变卖了,雇了一套车马回家。后来举人还写信让女儿回娘家看看,展女坚持不去。直到封生中举,两家才通好往来。


\subsection{1.7.8   郭 秀 才}
\label{\detokenize{p00_u5176_u5b83/_u767d_u8bdd_u804a_u658b_u5fd7_u5f02:id271}}
广东有个姓郭的秀才,傍晚从朋友那里回来,走到山中迷了路,走进了一片乱树丛里。到了一更时,听到附近的山头上有人说话,他急忙朝那里奔去。见十多个人,正围坐在地上喝酒,瞧见郭秀才,一齐大声说:“座中正好少一个客人,你来到,太好了,太好了!”郭秀才入了座,见在座的多半是读书人,便请教回家的路。其中一个笑起来:“你这个人真酸气,舍弃这大好的明月不观赏,怎么想着回家呢?”立刻递给他一杯酒。郭秀才尝了一点,香味扑鼻,一仰头就喝了下去。接着又一个人拿壶给他倒酒。郭秀才本来就很喜欢喝酒,又加上奔跑得口渴,一连喝了十杯。大家拍手称赞说:“有气魄!真是我们的朋友!”郭秀才为人放达,爱开玩笑,能学各种鸟叫,无不学得维妙维肖。他离坐去旁边小便时,偷偷地学燕子叫。大家怀疑地说:“半夜里哪来的燕子?”又学杜鹃叫,大家越加感到惊疑。郭秀才回到坐处,只笑不说话。正当大家在纷纷议论时,郭秀才回过头去,学作鹦鹉说:“郭秀才喝醉了,快送他回去吧!”大家很惊讶,可侧起耳朵再听,四周只是一片寂静,再也没听到叫声。过了一会儿,郭秀才又学鹦鹉叫。大家这才发现是郭秀才学的,一起大笑起来。大家都撮起嘴跟他学,没有一个学得像。一个人说:“今晚可惜青娘子没来。”又一个人说:“中秋佳节,我们还在这里聚会,郭先生不能不来。”郭秀才恭敬地答应下来。这时,一个人站起来说:“郭先生有学鸟叫的绝技,我们来表演叠人游戏,怎么样?”于是吵吵嚷嚷,一块站起来,前边的一个人挺身站立;立刻有一个人飞快地跳到他的肩上,也直立起来;一连上了四个人,高得不能再向上跳。继续再上的人,抓着臂,踩着肩,像爬梯子一样;十多个人,一会儿全爬上去了,看上去高入云霄。郭秀才正惊愕时,他们就像一根直立的木柱,直挺挺地倒在地上,变化成一条细长的小道。郭秀才惊骇地站了很久,顺着小道竟回了家。

笫二天,郭秀才腹内痛得厉害;尿出的尿是绿色的,像铜青,碰到东西就能染上色,也没有尿味,一连尿了三天才好了。郭秀才又去察看他们集会的地方,只见满地肉骨剩菜,杯盘狼藉,周围一片丛棘茂草,并没有道路。到了中秋节,郭秀才想去赴会,朋友们把他劝住了。假若大着胆子再去见一下青娘子,必定会有更多奇异的事。可惜啊,他当时动摇了!


\subsection{1.7.9   死 僧}
\label{\detokenize{p00_u5176_u5b83/_u767d_u8bdd_u804a_u658b_u5fd7_u5f02:id272}}
有一个道土,外出云游,天已很晚,投宿到荒野的一个寺院里。他见僧人住的房子紧紧地关闭着,就垫下蒲团,盘腿坐在廊下。夜已经深了,周围很静,他听到开门与关门的声音。转眼间,他见走来一个僧人,浑身上下都沾满血污。僧人好像没见到道士,道士也装着没看见他。僧人径直进入大殿,登上佛座,抱着佛像的头大笑,很长时间才离去了。

第二天,道士看看房舍,门关得好好的。他感到奇怪,到村子中,便告诉大家他晚上所见到的事。大家一块来到寺院,开门查验,看见僧人被杀死在地上。寝室中的席子和箱子都被掀翻了。大家知道这是被盗贼抢劫了。可是僧鬼还笑,这是什么原因呢?大家一块查看佛像的头,发现佛像头后有微小的痕迹,便用刀挖开,里面藏着三十多两银子。大家就用这些钱,把死僧埋葬了。


\subsection{1.7.10   阿 英}
\label{\detokenize{p00_u5176_u5b83/_u767d_u8bdd_u804a_u658b_u5fd7_u5f02:id273}}
甘玉,字璧人,庐陵县人。父母早就死了。留下个弟弟叫甘珏,字双璧,从五岁起就由哥哥抚养。甘玉性情友爱,对待弟弟如同自已的儿子。后来甘珏渐渐长大,生得一表人材,秀美出众,而且很聪明,文章写得好,甘珏更加喜爱他,常说:“我弟弟才貌出众,不能不找个好媳妇。”但是由于过分挑剔,始终没有找到满意的。

这时,甘玉在匡山寺庙里读书。一天夜里,刚躺下,听到窗外有女子说话的声音。他偷偷一看,见有三四个女郎席地而坐,几个婢女正摆酒上菜,都长得特别漂亮。一个女子说:“秦娘子,阿英为什么没来?”那个坐在下座的女子说:“昨天她从函谷关来,被恶人伤了右臂,不能一起来玩,她正因此遗憾呢。”另一个女子说:“我前天夜里做了个恶梦,今天想起来还吓得冒汗呢。”下座的女子忙摇手说:“不要说!不要说!今晚上姐妹们欢聚相会,说了吓人的话让人不痛快。”女子笑着说:“看你那胆怯的样子!难道真有虎狼把你给叼去吗?你要是不让我说,必须唱一首歌,为我们助酒。”那女子便低声唱道:“闲阶桃花取次开,昨日踏青小约未应乖。嘱咐东邻女伴少待莫相催,着得凤头鞋子即当来。”唱完,满座人无不赞赏。正说笑着,忽然一个高大的男人从外边闯进来,鹰样的眼睛,闪闪发光,相貌丑陋可怕。女郎们哭喊着:“妖怪来了!”像惊弓之鸟一样一哄而散。只有刚才唱歌的那个女子体态柔弱,落在后面,被那男人抓住。女子痛苦地哭叫着,拼命挣扎。那男人生气地大吼一声,咬断了她的手指,就势嚼了起来。女子躺在地上,像死了一样。甘玉怜悯之心顿起,再也忍耐不住,急忙抽出剑,开门冲出去,向那男人一剑砍去,砍掉了大腿,那人忍痛逃走了。甘玉扶女子进屋,见她面如土色,血流满了衣袖。看她的手,右手拇指已经断了。甘玉撕下一块布,给她包好,女子才呻吟着说:“救命之恩,让我怎样报答呢?”甘玉自从看到她时,心中已经暗暗为弟弟盘算,就把自己的意思告诉了女子。女子说:“我这样一个残疾之人,不能操持家务了。应当另外为令弟找一个好的。”甘玉问她姓氏,女子回答说;“姓秦。”甘玉给她铺好被褥,让她暂时在这里休养,自己抱着铺盖到别处去睡了。第二天一早,甘玉来看那女子,床上却已经空了。甘玉想,她一定是自己回去了。但是访察了邻近的村子,并没有姓秦的。他又到处托亲戚朋友打听,也没有个确实的消息。回去与弟弟说起这事,还悔恨得像丢失了什么宝贝似的。

甘珏一天偶尔到野外游玩,遇见一个十五六岁的少女,风姿美好,看着甘珏微笑,像有话要说。她四面看了看,问甘珏说:“你可是甘家的二郎吗?”甘珏回答说:“是。”少女说:“令尊曾给你和我订过婚约,你怎么今天想违背婚约,另外跟秦家订婚呢?”甘珏说:“小生幼年失去父母,家中的亲戚朋友我都不知道。请告诉我你的家世,我回去问我哥哥。”那少女说:“没必要细说,只要你一句话,我自己会到你家去的。”甘珏以没有禀告哥哥为由推辞了,少女说:“呆郎君!你就这么怕你哥哥呀!我姓陆,住在东山望村。三天之内,等你的回信。”说完就告辞走了。甘珏回家,跟哥嫂讲了这事,哥哥说:“简直是胡说八道!父亲去世时,我都二十多岁了,如果有这事,我能没听说?”又觉得那女子一人在野外行走,还与男子随便搭话,更加看不起她。甘玉又问起她的相貌,甘珏面红耳赤,一句话也说不出来。嫂子笑着说: “想来一定是位美人了?”甘玉说:“小孩子哪分得出美丑?就算美,也肯定比不上秦姑娘。等秦姑娘的事不成,再考虑她也不晚。”甘珏默默地退了下去。

过了几天,甘玉在路上见一个女子在前面边哭边走,便垂下鞭子按住缰绳,微微斜眼一看,见是个举世无双的美丽少女,便叫仆人去问她为什么这样伤心。少女回答说:“我以前许给甘家老二,因为家里穷,搬到很远的地方去了,跟甘家断绝了音信。最近刚回来,听说甘家三心二意要背弃前约。我想去问问大伯子甘璧人,怎么安置我?”甘玉惊喜地说:“甘璧人就是我。先父在世时订的婚约,我实在不知道。这儿离家不远,请你到家里再商量。”于是从马上下来,把缰绳交给少女,让她骑着,自己牵马步行,一块回家。少女自己说:“我小名叫阿英,家里没有兄弟,只有一个表姐秦氏和我住在一起。”甘玉这时才明白阿英就是弟弟遇见的那个美人。甘玉想告诉她家里的人,阿英再三阻止。甘玉暗暗高兴弟弟得到这么一位俊媳妇,但又怕她轻浮不庄重,招人议论。可住了很长时间,阿英非常矜持端庄,又温柔会说话,对嫂子像母亲一样,嫂子也非常喜欢她。

到了中秋佳节,甘玉夫妻俩正在吃酒说笑,嫂子让人来叫阿英。甘珏心里有些不高兴。阿英就让来人先回去,说自己马上就到。可她端坐在那儿跟甘珏说笑了很长时间,也没有去的意思。甘珏怕嫂子等久了,就连连催促她。阿英只是笑,一直没有去。

第二天一早,阿英刚梳妆完,嫂子亲自过来问候,说:“昨天夜里在一起时,为什么老是不快乐?”阿英微微笑了一下,没说话。甘珏觉得奇怪,再三询问,发现了破绽。嫂子大吃一惊说: “如果不是妖怪,怎么会有分身术!”甘玉也害怕起来,隔着帘子告诉阿英说:“我们家世代积德行善,从来没跟人结过怨仇。如果你真是妖怪的话,请马上走,别伤害我弟弟!”阿英不好意思地说:“我原本不是人,只是因为老公公在世时订的婚约,所以秦家表姐也劝我来完婚。我自己明白不能生男育女,曾经想离开你们;之所以恋恋不舍,是因为兄嫂待我太好了。如今既然被怀疑,就从此永别吧!”说完,转眼就变成一只鹦鹉,翩翩飞走了。当初,甘父在世时,养了一只鹦鹉,非常聪明,甘父常常亲自喂食。当时甘珏才四五岁,问父亲说:“养鸟干什么?”甘父开玩笑说:“给你作媳妇啊。”有时鹦鹉没食吃了,甘父就喊甘珏说:“还不拿吃的给鹦鹉?要饿煞你媳妇了!”家里人也都拿这话来取笑甘珏。后来,鹦鹉挣断锁链,不知飞到什么地方去了。甘玉想到这里,才醒悟女子说的婚约指的就是这个。可是甘珏明知她不是人,仍然想着她。嫂子想得更厉害,整天伤心落泪。甘玉也很后悔,但也无可奈何。

两年后,甘玉为弟弟聘娶了姜氏女,可甘珏始终觉得不如意。甘氏兄弟有个表兄在广东当司李。甘玉到广东去探望他,去了很久也没回来。恰在这时,家乡遇上土匪作乱,附近的村落多半都成了废墟。甘珏非常害怕,带领全家人躲避到了山谷里。在山谷避难的人很多,谁也不认识谁。甘珏忽然听见有个女子小声说话,声音很像阿英。嫂子催促他过去看看,甘珏走近一看,果然是阿英。甘珏高兴极了,捉住阿英的手臂不放松。阿英对同行的人说:“姐姐先走吧,我看看嫂子就来。”阿英来到嫂子跟前,嫂子看见她,伤心地哭起来,阿英再三劝说,又说:“这里不是安全的地方。”劝他们回家。甘珏害怕土匪会到村里去,阿英再三说:“不要紧。”就和他们一同回去了。阿英撮了一些土拦在门外,嘱咐家里人安心在家中住着,不要出门。坐着说了几句话,阿英转身想走。嫂子急忙握住她的手腕,又叫两个婢女捉住她的双脚。阿英没有办法,只好住下了。但是阿英却不到甘珏房里去,甘珏约她三四次,她才去一次。嫂子常对阿英说新娶的姜氏媳妇不能让小叔子满意。阿英便每天早上起来给姜氏梳妆打扮,梳理好了头发,又细心地为她搽匀脂粉。人们再看姜氏,比往日漂亮了几倍。如此三天,姜氏居然变成一个美人。嫂子觉得奇怪,就对阿英说:“我没有儿子,想买个小妾,暂时没空去买。不知道婢女是不是也能变成美人?”阿英说:“没有人不可以变美,只是本质好一点的容易些罢了。”嫂子就把所有的婢女叫来,让阿英相看。只有一个又黑又丑的婢女,有生男孩的相貌。阿英就把她叫来,给她洗了澡,洗了脸,然后用浓浓的粉和了药末给她抹上。过了三天,这个婢女的脸渐渐由黑变黄;又过了几天,粉脂的光泽慢慢沁入肌肤,居然变得很好看了。

阿英他们每天关着门在房里说笑,根本不想土匪的事。一天夜里,村里突然一片吵嚷声,全家人都吓得不知怎么办好。不一会儿听到门外人喊马叫,土匪纷纷离去。天亮以后,才知道村里已被烧光抢尽了。强盗们一队一伙地四处搜寻,凡是藏在山谷洞穴里的人,都被搜出来杀了,或是抓走了。于是甘家的人更加感激阿英,把她当作神仙看待。阿英忽然对嫂子说:“我这次来,是因为忘不了嫂子的情义,暂时为你们分担离乱的忧愁。哥哥不久就要回来了,我在这里,就像俗话说的,非李非桃,不伦不类,让人笑话。我要走了,有时间我会再来看望你们的。”嫂子问:“你哥哥在路上没事吧?”阿英说:“最近有大难。但这不关别人的事,秦家姐姐受过哥哥的恩惠,我想一定会报答他,所以不会有什么事。”嫂子留她过夜,天不亮阿英就走了。

甘玉从广东回来,听说家乡闹土匪,便日夜兼程地往回赶。路上遇到贼寇,主仆二人把马扔了,各自把银子扎在腰间,钻进棘丛中躲避。这时,一只秦吉了鸟飞落到荆棘上,展开翅膀遮盖住了他们。甘玉见它的脚缺一个指头,心中感到奇怪。不一会儿,贼寇从四面包围过来,在荒草丛棘中走来走去,好像在寻找他们。主仆二人吓得连气都不敢出,直到盗寇走光了以后,那只鸟才飞走了。甘玉回家后,一家人各自述说了自己的见闻,甘玉才知道那只秦吉了鸟就是自己曾救过的秦姑娘。

后来,每当甘玉外出不回来,阿英晚上一定来。估计甘玉要回来了,第二天便早早地走了。甘珏有时在嫂子房里遇到阿英,瞅机会邀她到自己屋去,阿英只是答应,却不肯去。

一天夜里,甘玉出门了。甘珏想阿英一定会来,就藏在一边等她。不久,阿英果然来了,甘珏突然出来,把她拦截住,硬要她到自己房里去。阿英说:“我与你的情缘已经完了,强合在一起,恐怕会遭到上天的惩罚。不如留些余地,以后我们还可以常见面的,怎么样?”甘珏不听,又和她做起夫妻之事。天亮后,阿英去见嫂子,嫂子怪她为什么昨夜没来,阿英笑着说:“半路上被强盗劫了去,让嫂子白等了一个晚上。”说了几句话,便转身走了。不多久,有一只很大的狸猫叼着一只鹦鹉从嫂子卧室门前经过。嫂子一见,惊骇极了,怀疑那鹦鹉是阿英。当时嫂子正在洗头发,赶忙住手大声呼叫。家里人一起连打带喊,才把它救下来。鹦鹉的左翅膀沾满了血,已经奄奄一息了。嫂子把它放在膝盖上,抚摩了很久,才渐渐苏醒,自己用嘴整理着翅膀。一会儿,鹦鹉在房子里转着圈子飞起来,大声说:“嫂子,告别了!我怨甘珏呀!”振动着翅膀飞走了,从此再也没回来。


\subsection{1.7.11   桔 树}
\label{\detokenize{p00_u5176_u5b83/_u767d_u8bdd_u804a_u658b_u5fd7_u5f02:id274}}
陕西的刘公,是兴化县的县令。有一个道士来献给他一棵栽在盆里的小树。县令仔细一看,原来是一棵纤细如指的小桔树,他不喜欢,不想接受。刘公有个小女儿,这时才六七岁,正好那天过生日。道士说:“这盆小树不足以供您赏玩,姑且送给女公子祝她福寿吧。”于是刘公便接受下来。女儿一见这棵小桔树,非常喜爱。把它放在自己的闺房里,早晚护理,唯恐它受到损伤。刘公任期满了的时候,桔树已经有一把多粗。这一年它第一次结果。刘公一家收拾行装准备离开,认为桔树太重,带着累赘,商量着不要了。小女儿抱着桔树撒娇地哭起来。家里人哄她说:“只是暂时离开,过不了多久就会回来。”小女儿相信了这些话,才不哭了;但她又恐怕这棵树被力气大的人扛走了,非要看着家里人把树移栽到台阶下,这才离去。

女儿回到家乡,长大后嫁给了一个姓庄的。姓庄的在丙戌年考中进士,被委任为兴化县令。他的夫人十分高兴,心里琢磨,十多年了,那棵桔树可能已不存在了。到了兴化,原来那桔树已经有十围粗了,而且果实累累,数以千计。问以前的差役,都说:“刘公走了以后,这棵树长得很茂盛,就是不结果,这是它第一次结果。”夫人更加惊异了。姓庄的在任三年,桔树年年硕果累累。第四年,桔树忽然憔悴不堪,不像从前那样茂盛。夫人说:“夫君在这儿的任期大概不长了。”到了秋天,庄县令果然被解任。


\subsection{1.7.12   赤 字}
\label{\detokenize{p00_u5176_u5b83/_u767d_u8bdd_u804a_u658b_u5fd7_u5f02:id275}}
清朝顺治乙未年冬天的一个晚上,天上出现了火一样的红字很大,内容是:“自苕代靖否复议朝冶驰。”


\subsection{1.7.13   牛 成 章}
\label{\detokenize{p00_u5176_u5b83/_u767d_u8bdd_u804a_u658b_u5fd7_u5f02:id276}}
牛成章,是江西的一个布商。妻子姓郑,生了一个儿子,一个女儿。牛成章三十三岁时病死了。儿子牛忠,当时才十二岁;女儿不过八九岁罢了。母亲不能守节,卖掉家里的东西,改嫁而去。留下兄妹二人,难以生存下去。牛成章有个叔伯嫂子,已经六十岁,孤独一人没有依靠,就收留了两个孤儿一块生活。

几年后,老太太去世了,家中生活更加困难。牛忠渐渐长大,想继承父业,但苦于没有本钱。这时,妹妹嫁给了一个姓毛的商人,家中很富有,她哀求丈夫借几十两银子给了哥哥。

牛忠跟着别人去南京,途中遇上了海寇,身上带的钱都被抢光,他没法回家,只好到处流浪。一天,偶然走进一个当铺,见铺主极像他的父亲;出来后,秘密访查打听,姓氏名字都和父亲一样。牛忠十分惊讶,不明白其中的缘故。只是每天在当铺旁边转来转去,暗地察看铺主对他有没有反应。铺主对他却毫不理会。牛忠经过三天的观察,铺主的说笑举动,真是自己的父亲,一点不错。当下又不敢拜认,就向铺中的佣人自我介绍,请求以同乡的身份,到铺中做佣人。立好契约后,铺主看他的姓名,家乡住地,似乎心里有所触动,问他从哪里来。牛忠哭着说出了父亲的名字。铺主听后,怅然若失,像有心事一般。待了很久,又问:“你母亲好吗?”牛忠又不敢说父亲死去,委婉地回答说:“父亲六年前出外经商,至今还没有回家。母亲改嫁,幸亏有伯母抚育,不然,早就埋到山沟里了。”铺主十分悲惨地说:“我就是你父亲啊。”于是,父子拉着手,悲哀万分。随后,父亲领他到内室拜见后母。后母姓姬,三十多岁,没有生育,牛忠来到,她很高兴,在内室设宴招待他。

自从牛忠来到之后,牛成章始终闷闷不乐,就想回老家一趟。妻子担忧铺中没人照管,没让他走,牛成章便带领儿子处理铺里的事务。过了三个月,他把铺中所有的帐册托给儿子,自己急忙整理行装回了老家。

父亲走后,牛忠把父亲已去世的实情告诉了后母。后母听了很吃惊,说:“他经商来到这里,过去和他交往的好友,留下他开了这个当铺;娶我来已经六年,怎么说他死了呢?”牛忠又详细叙说了一遍。二人都产生了疑念,不明白其中的因由。

过了一天一夜,父亲从老家返回来,手里拉着一个妇人,头发乱蓬蓬的。牛忠一看,原来是自己的亲生母亲。牛成章揪着她的耳朵,跺着脚大骂:“为什么抛弃我的儿子!”妇人非常害怕,趴在地上一动也不敢动。牛成章用嘴咬她的脖子,妇人大声叫牛忠,说:“儿呀快来救救我!儿呀快来救救我!”牛忠再也忍不住,急忙向前用身子把他俩隔开。牛成章还在忿怒时,妇人突然不见了。众人很惊讶,大声嚷叫有鬼。再看牛成章,脸色突然变得苍白,穿的衣服一下子落到地上,化为一股黑烟,也不见了。母子二人惊叹不已,将牛成章的衣服、帽子埋葬了。

牛忠继承父亲的家业,成了富有万金的大户。后来牛忠回老家问起生母,原来她在父亲回去的那天去世了,家里人都说见过牛成章。


\subsection{1.7.14   青 娥}
\label{\detokenize{p00_u5176_u5b83/_u767d_u8bdd_u804a_u658b_u5fd7_u5f02:id277}}
霍桓,字匡九,是山西人。父亲做过县尉,很早就死了。霍桓是家中最小的孩子,聪明过人,十一岁时就考中了秀才,被人称为神童。然而霍桓的母亲对他过分爱惜,从不让他迈出家门,所以都十三岁了,还分不清叔伯、甥舅。同村有个姓武的评事,喜好道教,进山访道一去不返。武评事有个女儿名叫青娥,十四岁了,生得美貌无比。小时候偷看过父亲的书,非常羡慕何仙姑的为人。自从父亲进山修道后,她立志不嫁,母亲也拿她没有办法。

一天,霍桓在家门口看见青娥,尽管他还是个孩子不懂什么,但觉得非常喜欢她,只是表达不出来。回家后就告诉了母亲,让母亲托媒人去说亲。母亲知道青娥立志不嫁,觉得不好办,霍桓便整日闷闷不乐。母亲怕儿子不顺心会闷出病来,就托人去武家提亲,武家果然不答应。霍桓无时无刻不在想着这事,终究想不出点办法。这天有一个道士在门外,手中握着一把一尺来长的小铁铲。霍桓借过来看了看,说:“这东西有什么用?”道士回答说:“这是挖掘药材的工具。别看它小,坚硬的石头也能铲进去。”霍桓不太相信。道士就用铲砍墙上的石头,石头随手而落,像砍豆腐一样,霍桓非常惊讶,拿在手中玩着,爱不释手。道士说:“公子喜欢,我就把它赠给你吧。”霍桓高兴极了,拿钱酬谢他,道士不收钱走了。

霍桓把小铲拿回家,在砖石上试了几次,毫不费力就把砖石砍碎了。他顿时想道:如果从墙上挖个洞,不就可以见到武家的美人了?但却不知道这么做是非法的。等到夜深人静,霍桓翻墙出去,一直来到武家的墙外,挖透了两道墙,才到了正院。看见小厢房中还有灯光,就趴在窗上偷偷往里看,只见青娥正在卸妆脱衣。不一会儿,灯灭了,寂静无声。霍桓穿过墙壁进去,青娥已经睡熟了。他轻轻脱下鞋子,悄悄地爬到床上。又怕把青娥惊醒了,自己一定会遭到大骂而被赶走,就偷偷地躺在青娥的被子旁边,略略闻到女子的香气,便感到心满意足了。没想到他挖墙掏洞忙了半夜,已经十分疲乏,才一合眼就睡着了。青娥醒后,听到有呼吸声,睁眼一看,见有亮光从被凿开的墙洞中透进来,太吃一惊。她急忙起来,轻轻地拉开门栓出门,敲窗叫醒了丫头、老妈子,一同点了火把,拿着棍棒来到卧房。只见一个未成年的书生,酣睡在床上。仔细一看认出是霍桓。婢女们把他推醒,霍桓急忙起来,目光灼灼像流星一样,好像不怎么害怕,只是羞答答地不说一句话。婢女们都说他是贼,吓唬他,责骂他,他才哭着说:“我不是贼!实在是因为我太爱小姐,想看看她的美丽容貌。”大家又怀疑一连凿透了几道墙,不是一个孩子能办到的。霍桓拿出小铲子说出它的奇异用途。大伙一同试了试,既惊讶又害怕,认为是神仙给他的,要去告诉夫人。青娥低头沉思,好像不愿意。婢女们知道了青娥的意思,都说:“这个人的名声门第,倒也不玷污小姐,不如放他回去,让他们托媒人来说亲。等明天,就告诉夫人说昨夜遭了强盗,怎样?”青娥没有说话,婢女们就让霍桓快走。霍桓要小铲子,婢女们笑着说:“傻小子!还忘不了凶器!”霍桓看见青娥枕边有一股凤钗,就偷偷装进袖中,可是被婢女看见了,急忙告诉青娥,青娥不说话,也不生气。一个老妈子拍着霍桓的脖子说:“别说他傻,心眼儿机灵极了!”就拉着他,仍然让他从墙洞里钻了出去。

回家后,霍桓不敢如实告诉母亲,只是嘱咐母亲再托媒人到青娥家去提亲。母亲不忍心拒绝他,便到处托媒人,急着为儿子另选良姻。青娥知道后,心里又急又慌,暗暗让心腹人给霍母透露风声。霍母非常高兴,托媒人去武家说亲。恰巧有个小婢女泄漏了那天晚上的事,武夫人感到很耻辱,非常气愤。媒人一来,更触发了她的怒气,气得她用手杖戳着地,太骂霍桓和他母亲。媒人害怕,逃窜了回去,把详情告诉了霍母。霍桓的母亲也很生气,说:“不成器的儿子做的这些事,到现在我还被蒙在鼓里,怎么能这样无礼对待我?当他们在一起睡觉时,为什么不将荡儿淫女一块杀了!”从此霍母见了武家的亲属,便宣扬这事。青娥听说后,羞愧得要死。武夫人也很后悔,但却没法禁止霍母不让她说。青娥暗自让人去婉转地告诉霍母,发誓说自己非霍桓不嫁。青娥盼话那样悲切,霍母很感动,就不再说那件事了。但是两家的亲事也不再提了。

当时秦中的欧公在这个县当县令,见霍桓的文章好,非常器重他。时常把他召进县署,极力优宠。一天,县令问霍桓:“结婚了吗?”霍桓回答说:“没有。”县令细问原因,霍恒说:“从前我和已故武评事的女儿有过婚约。后来因为两家有隔阂,就终止了。”县令问:“你还愿意同她成亲吗?”霍桓不好意思,没说话。县令说:“我一定为你办成这事。”就委托县尉、教谕,去给武家送聘礼。武夫人很欢喜,婚事就这样定了。过了年,把媳妇娶进门。青娥一进家门,就把小铲子扔在地上说:“这贼寇用的东西,快拿回去吧!”霍桓笑着说:“不能忘了媒人。”珍重地佩戴着它,从不离身。

青娥为人温厚善良,沉默寡言。一天三次拜见婆母,其余时间只是关门静坐在书房里,不太留心家务事。婆母有时因红白公事出门到别的地方去,她便事事都过问,处理得井井有条。过了一年多,生了个儿子,取名孟仙。青娥把孩子委托给乳妈照料,好像不大关心似的。又过了四五年,青娥忽然对霍桓说:“我们的美满姻缘到现在已经八年,如今就要长久分离了。有什么办法呢!”霍桓惊讶地问她怎么回事,青娥默默地一句话不说。妆扮好了拜见了婆母,接着转身回到屋里。霍桓同母亲追到房中问她,她已躺在床上咽了气。母子二人十分悲痛,购置了上好的棺材安葬了她。

霍母已经年老力衰,常常抱着孙子思念儿媳。从此得病,卧床不起。不想吃饭,只想吃鱼羹。但是附近没有鱼,只有到百里之外才买得到。这时家中的小厮和马匹都被差遣出去了,霍桓十分孝顺,急不可待,便带着钱自己去买鱼了。白天黑夜不停地赶路,返回时走到山中,太阳已经落山了。霍桓两脚磨起了泡,一瘸一拐地走着,十分艰难。这时后面一个老头赶上来,向他,“脚是不是起泡了?”霍桓连连答应。老头便拉他坐到路旁,敲石取火,用纸包着药末,给霍桓熏脚。熏完,让他试着走一走,脚不但不疼了,步履反而更加矫健。霍桓非常感激:向老头道谢。老头阿:“什么事这样急?”霍桓回答母亲有病,又说了母亲生病的缘由。老头问:“为什么不再另娶呢?” 霍桓回答说:“没找到合适的。”老头指着远处一个山村说:“那地方有一个很好的姑娘。你如能跟我去,我愿意给你做媒。”霍桓推辞说母亲有病,急等鱼吃,没有空闲去。老头便拱手告辞,约他改天再去,进村只要问王老头就行,接着就走了。霍桓回家后,把鱼烹好端给母亲吃。母亲多少能吃点东西,几天后病就好了。霍桓这才叫仆人备马一起到山村去找那老头。

霍桓来到和老头相遇的地方,却找不到那个村子。他来回彷徨了多时,夕阳渐渐落山了。山谷重重叠叠,看又看不远,就与仆人爬上山头,四下一望,却看不见一个村子。无可奈何,只得往山下走,但回去的路又找不到了。霍桓心中急躁得如同着了火。正在东奔西跑时,昏暗中一脚踏空,从绝壁上掉了下去。幸亏数尺下有一条细长的平台,霍桓正好掉在上面。平台窄得刚刚能容下他的身子,往下黑得看不见底。他害怕极了,一动也不敢动。又幸亏崖边上长满了小树,像栏杆一样围护着他。他慢慢移动了一下身子,看见脚旁有个小洞口,心中暗暗高兴,就背贴着石头,慢慢蠕动着滚进洞中,心中才稍平稳了些,希望等到天亮时叫人搭救。不多时,看见山洞深处有星星大的亮点,霍桓慢慢走近,走了约三四里路,忽然看见有房屋。虽没有灯火,但却像白天一样光明。一个美丽的女子从屋里出来,霍桓仔细一看,原来是青娥!青娥看见霍桓,惊奇地问:“你是怎么来的?”霍桓顾不上说话,抓着她的手呜呜地哭了起来。青娥劝住他,问起婆母和儿子。霍桓就把家中的苦处述说一遍,青娥也惨然泪下。霍桓说:“你死了一年多了,这是不是阴间啊?”青娥说:“不是,这里是仙府。我并没有死,所埋葬的,不过是一根竹杖。你今天来这里,也算是有仙缘。” 就领他去拜见父亲。只见一个留着长胡子的老头,坐在堂上。霍桓上前拜见,青娥说:“霍郎来了!”老头吃惊地站起来,握着霍桓的手简单说了几句话,就说: “女婿来了,太好了。应当留在这里。”霍桓推辞说母亲盼他回去,不能久留。老头说:“我也知道。但迟三四天回去,不会有什么关系吧。”就让人摆酒菜招待他,又叫婢女在西堂上放了床,铺了锦绣被褥。霍桓吃完饭,约青娥同床睡觉。青娥说:“这是什么地方,能容许狎亵!”霍桓捉住她的胳膊不放。窗外传来婢女的嗤笑声,青娥更加羞惭。正在争执时,老头进来,叱责说:“俗骨玷污了我的洞府!马上走!”霍桓一向高傲,如今羞愧得无法忍受,变了脸色说:“儿女之情,人所不免!你作为长辈怎么能监视我们?想叫我走并不难,但你女儿必须跟我去!”老头理屈词穷,叫女儿跟他走,打开后门送他。骗霍桓刚离开门,父女俩把门关死回去了。霍桓回头一看,只见峭壁x岩,一点缝隙也没有。自己孤单一人,不知往什么地方去好。看天上斜月高悬,星斗稀疏,他惆怅了很久,由悲伤变为怨恨,对着石壁号叫,始终没有应声的。霍桓气愤至极,从腰中拿出小铲,奋力挖凿石壁,边挖边骂,瞬息间已凿进三四尺。隐隐听见石壁里有人说活:“孽障啊!”霍桓凿得更急。忽然洞底两扇门豁然打开,推青娥出来,说:“走吧!走吧!”石壁又复合上了。青娥埋怨说:“既然爱我做你的媳妇,哪有这样对待丈人的?是哪里的老道士,给你这件凶器,把人缠得要死!” 霍桓得到青娥,心愿已经满足,不再说什么,只是担忧道路艰险难以回家。青娥折了两根树枝,两人各自跨上一根,树枝随即化作马匹。一路奔驰,不一会儿就来到家,这时霍桓已经失踪七天了。

起初,霍桓同仆人失散后,仆人找不到他,就回家告诉了霍母。霍母派人搜遍山谷,也没有踪影。正忧虑恐慌的时候,听说儿子回来,欢喜地出来迎接,抬头看见儿媳,几乎把她吓死。霍桓简单述说了经过,霍母更加喜欢。青娥因为自己形迹奇异,担心别人知道了会议论,便请求母亲搬家。霍母听从了她的意见。霍家在外郡有房产,就选了吉日搬迁过去,人们都不知道。霍桓与青娥又一块生活了十八年,生了一个女儿,嫁给了本县一个姓李的。后来霍母老死了,青娥对霍桓说:“我家的茅草地里,曾经有一只野鸡在那儿抱了八只蛋,那里可以埋葬母亲。你们父子俩一同扶棺材回去安葬母亲。儿子已经成家立业,可以留在那里守护坟墓,不用再回来。”霍桓听从了她的话,埋葬母亲后自己返回来。过了一月多,孟仙来探望父母,可是父母已经杳无踪影。问他们的仆人,却说;“去给老夫人送葬还没回来。”孟仙心中明了白,只有感叹而已。

孟仙文才出众,名声很大,但是考场上总是失利,四十岁了还没有考中。后来他以拔贡的身份到京城参加考试,在考场上遇见一个年纪十七八岁的少年,神采俊逸。孟仙很喜欢他,看他的卷子上,写着顺天廪生霍仲仙,孟仙不由吃惊地瞪大了眼睛。就把自己的姓名告诉那少年,仲仙也感到奇怪,就问孟仙的家乡是哪里,孟仙把一切都告诉了他。仲仙高兴地说:“小弟赴京时,父亲嘱咐说,在文场中如遇到山西一个姓霍的,是我们的同族,要与他好好相处,如今果然如此。可是我们的名字怎么这样相近啊!”孟仙问了仲仙的高祖、曾祖及父母的姓名后,惊讶地说:“这是我的父亲啊!”仲仙怀疑年龄对不上,孟仙说:“我的父母都是仙人,怎么能以相貌看他们的年龄呢。”就把过去的事情都告诉他,仲仙才相信了。

考试完毕,二人顾不上休息,就叫仆人驾车,兄弟俩一同回了家。刚进家门,家人就迎出来说:昨天夜里,老太爷和老夫人突然不见了,兄弟俩大吃一惊。仲仙进屋去问媳妇,媳妇说:“昨天晚上还在一块饮酒,母亲说‘你们夫妇年轻不懂事,明天大哥来了,我就没有牵挂了。’今天早晨进屋一看,已经寂静无人了。”兄弟俩听了,伤心得跺脚。仲仙还想追出去寻找。孟仙认为没用,才没去。这次考试仲仙中了举人。因为祖坟在山西,就跟随哥哥一块回老家去了。他还希望父母仍在人世,走到哪里都要打听,但始终没有音讯。


\subsection{1.7.15   镜 听}
\label{\detokenize{p00_u5176_u5b83/_u767d_u8bdd_u804a_u658b_u5fd7_u5f02:id278}}
山东益都县的郑氏兄弟,都是文学士。大郑早就出了名,父母偏爱他,因此对大儿媳也好;二郑科场失意,父母不太喜欢他,也就厌恶二儿媳,至于耻于把她当作儿媳,这样相比之下一冷一暖,兄弟二人心里就有了隔阂。

二郑媳妇对丈失说:“都是同样的男子汉,为啥就不能为老婆争口气?”拒绝和丈夫同宿。从此二郑发愤努力,专心致志地勤学苦钻,也终于有了名气。父母对他的看法稍好了点,但终究不如对哥哥好。

二郑媳妇盼望丈夫显贵的心情非常急切,这一年正好是乡试大比之年,在除夕晚上她偷偷用镜听的方法为丈夫考试占卜吉凶。出了门,听见有两人才起来,互相推搡着闹着玩,说:“你也凉凉去!”二郑媳妇回到家里,弄不明白这句话是啥意思,也就放下这事不再提了。

乡试考完以后,兄弟二人都回家了。当时天还很热,两个媳妇在厨房里为忙秋的人做饭,热得她俩很难受。忽然有骑马的人登门来报喜讯,说大郑考中了举人。郑母赶紧跑进厨房喊大儿媳说:“老大考中了,你可凉凉去。”二郑媳妇又气又难过,一边掉泪一边做饭。不一会儿,又有人来报喜说二郑也考中了举人。二郑媳妇听说,用力一扔擀面杖起来,说道:“我也凉凉去!”这句话是她心中气忿之情所激,不知不觉顺口说出来的;可过后再一想,才知道正好应验了镜听占卜的结果。


\subsection{1.7.16   牛 癀}
\label{\detokenize{p00_u5176_u5b83/_u767d_u8bdd_u804a_u658b_u5fd7_u5f02:id279}}
陈华封,蒙山人。盛暑的一天,因为天气炎热,他来到野外的一棵大树下躺下乘凉。忽然一个人奔跑过来,头上戴着围领,匆匆忙忙地跑到树荫下,搬起一块石头坐下,挥动着扇子扇个不停,脸上汗流如汁。陈华封坐起来,笑着说:“如果把围领解下来,不用扇也可以凉快。”来客说:“脱下容易,再戴上就难了。”二人便攀谈起来。客人言词含蓄文雅,说:“这时没有别的想法,如能得到冰浸的好酒,一道清冷的芳香直入咽喉,炎热的暑气就可消去一半。”陈华封笑着说:“这个愿望很容易,我可以满足你。”便握着客人的手说:“我家就在附近,请赏光。”客人笑着跟他走了。

到了家,陈华封从石洞中拿出藏酒,酒凉得震牙,客人高兴极了,一口气喝了十杯。这时天快黑了,忽然下起雨来,陈华封便在屋里点上灯。客人也解下围领,二人开怀痛饮。说话间,陈华封看见客人脑后不时漏出灯光,心中疑惑。不多会儿,客人酩酊大醉,睡在床上。陈华封移过灯来偷偷一看,见他耳朵后边有一个洞,有酒杯大小,里面好几道厚膜间隔着,像窗棂一样,棂外有软皮垂盖,中间好像空空的。陈华封骇怕极了,暗暗从头上拔下簪子,拨开厚膜看看。里面有一物,形状像小牛,随手飞出来,冲破窗户飞走了。陈华封更加害怕,不敢再拨动了,刚想转身走,客人已经醒了,吃惊地说:“你偷看我的隐秘了。把牛癀放了出去,可怎么办?”陈华封询问缘故,客人说:“既然已经这样,还隐瞒什么。实话告诉你:我是六畜的瘟神。刚才你放跑的是牛癀,恐怕方圆百里内的牛就要死绝了。”陈华封本来以养牛为生,听了非常害怕,向客人恳求解救的办法。客人说:“我都免不了罪责,哪有什么办法解救?只有苦参散最有效了,你要广传这个方子,不要存私念就可以了。”说完,拜谢了陈华封,告辞出门。又捧了一把土堆在墙壁的龛中,说:“每次用一合便有效。”客人拱拱手就不见了。

过了不久,牛果然病了,瘟疫漫延开来。陈华封想自己专利,把治病的方子秘藏起来,不肯传人,只传给他弟弟。弟弟按方子一试,很神验;但陈华封自己照方子给牛吃药,却一点效力也没有。他有四十头牛,都快死光了,只剩下四五头老母牛,也奄奄一息了。他心中懊恼,无法可使,忽然想起龛中的那捧土,心想也未必有效,姑且试试吧。过了一夜,牛便都起来了。他这才醒悟到,药之所以不灵,原来是神对他私心的惩罚。几年以后,母牛繁育,又渐渐恢复到原来的境况。


\subsection{1.7.17   金 姑 夫}
\label{\detokenize{p00_u5176_u5b83/_u767d_u8bdd_u804a_u658b_u5fd7_u5f02:id280}}
浙江会稽县有个梅姑祠。梅姑神本姓马,老家在山东东莞。没出嫁,未婚夫就死了,便决心不再嫁人;到三十来岁上,她也死了。族人为她立祠纪念,称她为梅姑。

二百多年后的丙申年,浙江上虞县有个姓金的举子进京考试路过这里,进庙参观梅姑像,很是感慨。到了晚上,梦见有个穿青色衣服的丫鬟来传话说梅姑请他,他随着那人去了。进了祠,见梅姑正在屋檐下等他,笑着说:“白天受到先生您的关心,很是感激。若不嫌弃我丑陋拙笨,我愿给你当使唤丫头。”金某一声声答应着。梅姑送他时说:“先生您先回去,等我给你安排好地方就去接您。”金生醒了一想,梦见死鬼要嫁给我,这是啥事儿!别扭!这天夜里,这一带的居民都梦见了梅姑说:“上虞的金生是我的丈夫,你们应该在庙中为他塑个像!”天明后,村里的人们见了都说做了同样的梦。村中的族长怕为金生塑像玷污了梅姑贞洁的名声,不依从大家的意见。不久,族长一家全病了。族长害了怕,便在梅姑的上首塑了金生的像。像塑好后,金生告诉自己的妻子说:“梅姑要接我去呢。”于是穿戴得整整齐齐地死了。他妻子恨死了梅姑,到祠中指着她的像骂了一通脏话;还不解气,上了神座,又打了她一顿耳光才走。直到今天,梅姑娘家的马姓人还称金生为金姑夫。


\subsection{1.7.18   梓 潼 令}
\label{\detokenize{p00_u5176_u5b83/_u767d_u8bdd_u804a_u658b_u5fd7_u5f02:id281}}
进士常大忠,是山西太原人,在京城候选官职。抽签的前一夜,梦见梓潼帝君持名帖前来拜见。第二天抽签一看,得梓潼县令一职,常大忠很感奇怪。

后来,他母亲病故,卸职回家为母亲守孝。三年期满后,回到京城等候补官,夜里又做了个同样的梦。暗想:难道又去梓潼任职吗?不久,公文下来,果然不错。


\subsection{1.7.19   鬼 津}
\label{\detokenize{p00_u5176_u5b83/_u767d_u8bdd_u804a_u658b_u5fd7_u5f02:id282}}
有个姓李的人,白天躺在床上休息。看见从墙中走出一个妇人,头发蓬散得像个乱草筐,头发垂着遮挡着脸,走到床前用手把头发一分,露出脸来,又胖又黑,像个丑八怪。李某十分害怕,想要逃跑,妇人突然跳到床上,用力抱住他的头,便与他接吻,用舌头把唾液送到他嘴中,冷凉得如同冰块,渐渐流到喉咙。李某想不咽下去,但又不能喘气,咽下去又稠又粘,能塞住喉咙。才喘一口气,接着嘴中又堵得满满的,气急得喘不上时,就咽一口。这样过了很久,憋得他再也不能忍受。听到门外有人行走的脚步声,妇人才放开手匆忙而去。

从此之后,李某肚腹胀得喘不过气来,几十天吃不下饭。有的人告诉他饮用参芦汤,饮后吐出些像鸡蛋清一样的东西,病才好了。


\subsection{1.7.20   仙 人 岛}
\label{\detokenize{p00_u5176_u5b83/_u767d_u8bdd_u804a_u658b_u5fd7_u5f02:id283}}
有个叫王勉字黾斋的人,家在灵山,很有才气,考试总考第一。他心气很高,善讥讽人,不少人都受过他的奚落。

这天他偶然遇到个道士,道士打量了他一番说:“你的相貌主大贵,可惜被你轻薄的缺点给抵消了。凭你的聪明才学,如果不读书,去修道,还有可能成仙。”王勉讥笑说:“谁将来有多大的福,这是不可能知道的。我只知道世上并没有什么仙啊神的。”道士说:“你的见识怎么这样浅薄?仙人,不用找,我就是。”王勉更笑他荒唐。道士说:“我这个仙还没什么特别,你如随我去,立刻能叫你见上几十个真正的仙人。”王勉问:“去哪儿?”道士说:“近得很。”于是把拿的木杖夹在腿间,把另一头交给王生,叫他学自己的样子,嘱咐他闭上眼,叫声“起”,王生就觉得木杖忽然粗得像能盛五斗粮食的布袋,腾空飞起。王生悄悄一摸,一片片的鳞甲刺手,吓坏了,动也不敢动。一会儿,道士又叫一声“住”!就把木杖抽去,落到一所大宅院里。

只见楼阁重重,像帝王家,有个丈把高的台子,台上有座大殿,前后竖着十一根柱子,非常宏大华丽。道士拉他上去,就吩咐童子设宴,招待客人,殿内一下了摆了几十桌筵席,那个阔气,叫人眼花缭乱。道士换了好衣服等侯。不多会儿,诸位客人从天上来了,骑龙的、跨风的、骑虎的,等等不一,还带着乐器。有女有男,有赤脚的。内中有个美丽的妇人,骑彩凤,宫中打扮,有童子替她抱着乐器。乐器五尺来长,不是琴,也不是瑟,叫不出名来。

酒宴开始,佳肴满桌。王勉只觉吃起来又香又甜,一点也不像平常菜肴。王勉无言静坐,只看那美妇,心中喜欢她,惟恐她一直不弹。酒饮得差不多了,一位老者倡议说:“多亏崔真人邀请,今天可算盛会,当然该饮个尽兴。请大家以乐器分类,同一类的来个大合奏吧!”于是各自找伴儿配合,演奏起来。美妙的音乐响彻云霄。唯有那骑彩凤的,跟谁也搭不上伴儿。大家奏完,童子才打开乐器袋,摆到案子上。女的伸出白皙的手腕。像拨筝那样,开始演奏。声音比琴响亮得多。雄壮处使人胸怀开阔,柔婉处勾人魂魄。奏了半顿饭工夫,整个大殿里静得很,连个咳嗽的也没有。曲终,“当”的一声收住,像鼓磐那样清脆。众人齐称赞说:“云和夫人真是绝技啊!”大家起身告别,一时龙吟,凤鸣,都散了。道士安排了上等床铺被褥,供王勉休息。

王勉见丽人时,已经心动,听了她的音乐更加思念了。想想以自己的文才,做大官不难,那时要什么没有?……顷刻间心绪乱极了。道士好像知道了他的心事,说:“你前生与我一起学道,后来因意志不坚定,才坠入尘世。我不是勉强你,实在是想使你从恶浊的尘世中自拔;谁知你陷得太深了,懵里懵懂的,唤不醒了。现在我就送你走,以后也不一定不能见面。但想做天仙,还得再受劫难才行。”于是就指着石阶下一条长长的石头,叫他闭上眼坐了,嘱咐他不能睁眼看。说完,用鞭子把石头一抽,石头飞起来,王生耳边呼呼有风,不知飞了多远。忽然想,在天上看下界,是个什么样子;偷偷将眼睁开条细缝向下一看,见大海茫茫,无边无际,吓得赶紧闭眼,可是连石带人已经掉下去了。嘭!跟海鸥潜水似的,一下子扎进水里,幸亏他过去住在海边,会一点游泳。这时,便听见有人拍巴掌,说:“这一跤摔得真美啊!”正危急间,一女子拉他上了船,并说:“好事儿,好事儿,秀才‘中湿’啦!”王生一看,见这女子有十六七岁,很漂亮。王冻得打哆嗦,求她弄火烤烤。女子说:“跟我到家,就安排。以后如得意了,可别忘了我。”王生说:“什么话!我是中原大才子,偶然这样狼狈,以后该用一生来报答你,岂止是不忘!”

女子摇橹划船,行驶如风,一会儿靠岸,在舱中拿出采来的一束莲花,领他一起走。约走了半里路,进了个村庄。见有一朝南的红漆大门,进去后又过了几道门,女子先跑了进去。不久,出来一个四十来岁的男人,作揖请王生进屋。又吩咐佣人快拿衣帽鞋袜叫王生换了,然后,问起他的家世。王生说:“我不是说假话,我的才学还是有点知名度的,崔真人很喜欢我,邀请我去了天宫,我取功名做大官容易得很,所以不愿隐居。”那男子肃然起敬,说:“这地方叫仙人岛,与人世隔绝。我姓桓,叫文若,几辈子住在这僻静地方,没想到有接近名士的荣幸。”便殷勤地设下宴席,又不紧不慢地说:“我有两个女儿,大的叫芳云,十六岁了,至今未遇见佳偶,打算叫她侍奉您,怎么样?”王勉以为一定是那个采莲姑娘,赶紧站起来表示感谢。男子叫人在邻居中请几位德高望重的人来作陪,又让仆人马上去叫女儿。顷刻问,袭来一阵浓香,十几个美女簇拥着芳云出来了。明媚光艳,像朝霞中的莲花,拜见了客人,坐下。十几位美女中就有那个采莲人。酒过三巡,又出来个十来岁的女孩,姿态俊秀,笑着偎在芳云胳膊下边,一双水灵灵的大眼转来转去地看。桓文若说:“闺女家不在绣房里,出来干啥?”又对客人说:“这是绿云,是我小女儿。挺聪明,能背《三坟》、《五典》呢!”就叫她对客人吟诗,绿云即刻朗诵了三首“竹枝词”,稚嫩宛转的声音很好听。桓便叫她挨着姐姐坐下。又跟王生说:“像王郎这样的大才,一定写过很多好文章,能不能叫大家听听呀?”王勉痛快地背诵了一篇近体诗作,涌完自豪地左看右看。其中有这么两句:

“一身剩有须眉在,小饮能令块磊消。”

邻居老者再三地念诵。芳云低声告诉他说:“上句是孙悟空离火云洞,下句是猪八戒过子母河呀。”说得满座都鼓掌大笑。桓又请王生再诵别的,王勉又诵了一首水鸟诗:“潴头鸣格磔,” 忽然想不起下句来了,他略一沉思,芳云对妹妹叽叽咕咕地耳语了几句,捂着嘴笑起来。绿云对父亲说:“姐姐给姐夫续上下句了,是狗腚响绷巴。”一席人都笑得闭不上嘴。王勉很不好意思。桓文若生气地用眼睛瞪了下芳云,王勉表情才平静了些。桓又请王介绍自己的文章。王勉想,这些与世隔绝的人一定不懂八股文,就炫耀起自己应试得了第一名的一篇,题目是:“考哉闵子骞。”王勉文章破题的头句说:“圣人赞大贤之孝……”绿云看看父亲,说:“圣人没有赞美学生的,‘孝哉……’一句应该是别人的话。”王勉听了,情绪马上低落下来。桓笑着说:“小孩子懂什么!别挑剔这个,评评文采怎么样吧。”王勉又接着背诵,每诵几句,姐妹俩就叽叽咕咕耳语,好像是些批评的话,只是咕咕哝哝听不清。王勉念到得意处,连考官的评语也念出来了,有句评语是:“字字痛切。”绿云对父亲说:“姐姐说:该把‘切’字删去。”大家都不明白为什么。桓文若怕她又说出叫王生难堪的话,不敢往下问。王勉把文章背完,又介绍考官的总评语。有这样的句子:“羯鼓一挝,则万花齐落。”芳云又捂着嘴跟妹妹嘁喳,两人笑得前仰后合。绿云又对父亲说:“姐姐说:‘羯鼓应该挝四下’。”大家又不懂。绿云想说,芳云忍住笑吓唬她:“妮子敢说,看不打死你!”众人更不明白了,纷纷猜测是什么话。绿云忍不住,终于说:“删去‘切’字,说‘痛’就‘不通’;再敲四下鼓,鼓声不是 ‘不通不通’吗?”众人听了大笑起来,桓文若生气地斥责她们,赶快亲自起身斟酒,陪不是。

王勉开始时还吹嘘自己的才名,目中无人;到这时,再没那么神气,只有淌汗的份儿了。桓文若又夸将了他两句,想给他个机会让他下台,说:“我刚想起一句,请你即席联个下句:“‘王子身边,无有一点不似玉’。”大家还没来得及想,绿云应声说道:“黾翁头上,再着半夕即成龟。”芳云失声笑了出来,呵了手膈肢她。绿云脱身跑掉,回头看着姐姐说:“关你什么事?你一遍遍地骂,别人才骂一句就不行了?”桓文若喝斥她,她才笑着走了。

邻居老人告辞。婢女们领王勉夫妻进内室休息。内室里屏风床铺,陈设精美齐全,灯烛照耀。再看洞房里,满架子的函套,什么书都有。问她个生僻的问题,她没有答不上的。到这时候,王勉才觉出自己学问差远了,应该知羞才是。芳云喊“明珰”,采莲女就小跑过来,这才知道她的名字。刚才被挖苦得够呛,唯恐妻子瞧不起自己;幸好芳云虽然嘴厉害,对丈夫还是极尽温柔,王勉也就安下心来。没事儿就吟几句诗文,芳云说:“我有句忠言,不知你听得进听不进。”王问:“什么忠言?”芳云说:“从此别作诗,也是个掩饰短处的办法。”王勉一听,惭愧得很,就不再写文章。

日子长了,王勉和明珰渐渐亲昵,对芳云说:“明珰对我有救命之恩,希望你对她好些。”芳云立刻同意了。有时夫妻在卧室中玩耍,也叫上明珰一块儿。两人感情更深了,慢慢就发展到使眼色,打手势进行暗示的程度。芳云觉察出来,责备王勉,他唯唯地听着,好歹混过去了。

这天晚上,王勉与芳云对吟,觉得寂寞,建议把明珰喊来,芳云不许。王勉说:“您读了那么多书,怎么不记得‘独乐乐’几句?”芳云说:“我说你不通,这不更证明了。连句读也不懂啊?‘独要,乃乐于人要;问乐,孰要乎?曰:不’。”夫妻一笑而罢。

碰巧芳云姐妹去邻女那儿赴约,王勉得了空儿,赶紧叫来明珰,尽情欢娱了一番。当天晚上,王勉就觉得小肚子痛,痛过后,生殖器全缩回去了。吓得告诉了芳云,芳云笑了,说:“一定是报了明珰的恩了!”王不敢隐瞒,实说了。芳云说:“自找的祸,我实在没办法,不痛不痒的,随它去。”几天不愈,王勉心中郁闷,芳云知道,也不问,只是用明亮的眼睛看他。王勉说:“你正应了一句古话:‘胸中正,则眸子瞭焉’”。芳云笑笑说;“你也应了一句话:‘胸中不正,则瞭子眸焉。’”原来“没”的方音读如“眸”,故意用这话开他的玩笑。王勉也笑了,向芳云哀求治疗的办法。芳云说:“你不听忠告,在这以前,还可能怀疑我忌妒呢。你不知道,这婢女本来不可亲近,过去我说那些,实因为是爱你,可你……当成了耳旁风,我才故意不可怜你。唉,没办法,给你治吧。可医生必须观察观察。”就把手伸进他衣服里,口中念道:“黄鸟黄鸟,无止于楚!”王勉听他说得有趣,不觉大笑,笑过病就好了。

过了几个月,王勉因双亲年老,孩子年幼,常苦苦思念,将这个告诉了芳云。芳云说:“想回家不难,可是再见面就不知哪天了。”王勉听了,泪流满面,求芳云与自己一起走。芳云考虑再三,才答应了。桓翁设宴为他们饯行。绿云提了个篮子进来说:“姐姐要远远地离开我们了,没什么可送,怕你到了海上没地方住,连夜替你造了房子,可别嫌粗糙。”芳云施礼接受了,凑近看是些用细草制成的楼阁,小的有橙子那么大,大的像桔子。大约有二十多座,每座的梁栋椽檩,根根清楚,里面的床帐桌倚,芝麻粒儿大小。王勉以为是小孩子玩艺儿,可心里也叹服她的手巧。芳云说:“实告诉你吧,我们是地仙,因为与你有缘份才嫁了你。本来我不愿与你同去凡尘,仅为你老父亲在,不忍违背你的意思。等老父亲百年后,还得再回来。”王勉恭敬地答应着。桓翁问:“从水路走,从旱路走?”王勉受过水上风险,表示愿从旱路走。出了门,车马已等在那里了。

告辞了桓翁一家登上归程。马行迅疾,很快到了海岸边,王勉正愁无路可走,芳云拿出一匹白绸子,向南一抖,立刻变成了一带长堤,一丈多宽,转瞬问就过去了,长堤也慢慢收了起来。又到一处,有一片无边无际的潮水。芳云止住车马不让再走,下了车,取出篮子中用草做成的房舍,同明珰等婢女按一定布局摆好,转眼间变成一处大宅院。一齐进去卸下行装,跟岛上的房邸并无两样,连同房屋的桌呀、床啊也和原来一样。这时,天色已晚,就住下了。

次日早晨,叫王勉把父母接来赡养。王勉打发人骑马到老家,到了才知道家中宅屋已经换了主人;问邻里,说是他母亲和儿子都早死了,只有老父亲还活着。王勉的儿子爱赌博,家产全输光了,爷爷孙子连住的地方都没有,临时借宿在西村。

王勉刚回来时,还想取功名,所以不把家境放在心上,直到了解了这些情况才非常难过地想,富贵纵然好,可是跟梦中之花有什么区别?骑马到了西村,见老父亲穿得又脏又破,衰老得可怜。父子都失声痛哭,问起他那个没出息的儿子,说是出去赌没回来。王勉就用马接了父亲回来。芳云拜了公爹,烧了水请老人洗澡。送来绸缎衣服,让他住在熏了香的卧室里。又送信请来了公爹的老友陪他说话,老人的生活超过了名门大家。

一天,王勉的儿子找来了,王勉不让他进家门,只给了二十两银子。叫人告诉他:“用这些钱娶个媳妇过日子吧。再敢来,用鞭子打死!”儿子哭着走了。

王勉自从回了老家,不大与别人来往,可是老朋友偶然来到,一定款留几天,说话比原来谦虚多了。其中独有个黄子介,是老同学,也是个不及第的名士,王勉留他住了好多天。还常说些秘密话,送的钱物也多。住了三四年,王老头去世,王勉花一万两银子请人看茔地,厚礼葬了。这时,儿子已成了家,媳妇管得严,儿子赌博也少了,在爷爷的丧事上,儿媳才拜见了公婆。芳云一见,断定她能操家,又赐三百两银子作为买田产的资本。

第二天,黄子介带了王勉的儿子一同去拜谒,王勉住的房舍院落都已消失,不知他们到哪里去了。


\subsection{1.7.21   阎 罗 薨}
\label{\detokenize{p00_u5176_u5b83/_u767d_u8bdd_u804a_u658b_u5fd7_u5f02:id284}}
某巡抚的父亲,早先在南方做总督,去世已经很久了。

一天夜里,巡抚梦见父亲来,脸色哀伤恐惧,对他说:“我一生没多少罪恶,只有一旅边防军队,不应当调遣而错误地调遣了,途中遇上海寇,全军覆没。现今他们告到阎王那里,阴司里的刑罚残酷歹毒,实在叫人害怕。阎王不是别人,明天有个经历官押送粮草来,那人姓魏,他就是阎王。你要替我哀求他,不要忘了啊!”巡抚醒来,慌得这事很奇怪,心里不很相信。刚又睡下,又梦见父亲来,让他一定照说的去办,还说:“父亲遭遇灾难,还不铭记在心,怎么把它当作妖梦置之不理呢?”巡抚醒来,越加感到这事奇异。

第二天,巡抚留心查看名册,果然有个姓魏的经历,转运粮草第一个来到,巡抚立刻传话叫他进来。叫两个衙役把他按到座上,随后按拜见官长的礼节向他叩拜起来。叩拜完毕,直挺挺跪在地上,两眼垂泪,把梦中的事向魏经历说了。魏经历不承认自己是阎王,巡抚趴在地上不起来。魏经历才说:“是的!有那样一件事。但是阴间的法律,不像人间昏暗不明,可以上下联手,串通作弊,恐怕我无能为力。”巡抚苦苦哀求他。魏经历无可奈何,就答应下来。巡抚又请求迅速办理。魏经历反复筹划,考虑没有个安静的地方处理这事。巡抚请求把接待宾客的公馆清扫出来让他用。魏经历同意后,巡抚才从地上站起来。又要求审理时跟去看一下,魏经历不同意。他再三要求,才答应他去,嘱咐说:“到了那里不要出声。阴间刑罚虽然残忍,可是与人间不同,一处治就像死了,其实没死。如果你看见了什么,千万不要惊怪。”

到了夜里,巡抚藏在公馆的一旁,见公堂台阶下,受审的犯人,断头的,折臂的,乱纷纷不计其数。在一块空地上放着一口油锅,几个人在油锅下烧起了火。忽然看见魏经历穿着官服走出来,坐到大堂上,神气威猛,和白天见的大不一样。那些断头折臂的人,一齐趴到地上,同声叫喊冤枉。魏经历说;“你们都是被海寇杀害的,冤有头债有主,为什么乱告官长呢?”众鬼大声喊着说:“按规定不应该调遣,我们是被错误地调动后,才遭到杀害,这是谁给我们造成的灾难呢?”魏经历又多方为巡抚的父亲解脱。众鬼大声叫冤,乱成了一片。于是,魏经历叫过鬼卒,说:“可将那个官放到油锅,稍微炸一下,于理也是应当的。”看魏经历的用意,似乎想借此平息一下众鬼的怨愤。当下就有两个恶鬼把巡抚的父亲捉来,用锋利的钢叉刺入油锅。巡抚见此情景,心里又惊又痛,无法忍受,不觉脱口喊了一声。刹时,庭中寂然无声,眼前的一切都不见了。巡抚惊叹不已,悄悄地回去了。天明之后,巡抚去看魏经历,见他已经死在公馆里。


\subsection{1.7.22   颠 道 人}
\label{\detokenize{p00_u5176_u5b83/_u767d_u8bdd_u804a_u658b_u5fd7_u5f02:id285}}
从前有个疯颠的道士,谁也不知道他姓什么叫什么。他居住在蒙山的寺庙里,有时唱有时哭,很不正常,谁也猜不透他,有人曾见他煮石头当饭吃。

一次正逢重阳节,本县有个贵人带着酒登山,乘坐着华丽的车子游玩。喝完了酒从寺庙经过,才到门前,只见疯颠道士光着脚穿着破道袍,自己撑着一把大黄伞,学着给帝王清道的声音从庙里出来,意思很有嘲弄这位富贵人的味道。这位贵人很羞惭恼怒,指挥着他的仆人们追赶辱骂道士。道士大笑,转身向后跑。仆人们追得很急,道士便扔了他打的那把伞。仆人们一起上前撕破了伞,结果一片片伞布变成了鹰隼,到处乱飞。众人这才害怕起来。伞柄转动,又变成了一条巨大的蟒蛇,红色的鳞片非常耀眼。众人喊叫着想跑开,有一个同来蝣玩的人制止他们说:“这不过是迷惑人眼的幻术罢了,哪能咬人?”说完持刀直奔蟒蛇。蟒蛇张着口愤怒地迎上来,把他吞进口里咽了下去。众人更加害怕,护拥着那个贵人急忙奔跑,跑到三里之外的地方才停下来歇息。派好几个人小心翼翼地到寺庙去侦探,见道士和蟒蛇都不见了。刚要回去,听到老槐树内有气喘如驴的声音,他们害怕极了。开始时不敢走近老槐树,后来慢慢隐蔽着靠近,见老槐树已经腐朽,中间空空的,有一个洞像盘子那么大。有一个人试着爬上去往洞里一看,只见那个斗蟒蛇的人倒立在树洞之中,而洞孔大小只能容进两只手,没有办法把那人弄出来。急忙用刀劈树,等到把树劈开,那人已经昏死过去。过了一些时候,稍微苏醒过来,抬了回去,道士不知道到哪里去了。


\subsection{1.7.23   胡 四 娘}
\label{\detokenize{p00_u5176_u5b83/_u767d_u8bdd_u804a_u658b_u5fd7_u5f02:id286}}
程孝思,是四川人,自小聪明,能写文章。父母很早就去世了,家里非常贫困,无衣无食,只好求胡银台雇佣他干点文书差事。胡银台试着让程生写了篇文章,看了后非常高兴,说:“这人不会长久贫困,可以把女儿许给他。”胡银台有三个儿子、四个女儿,都是还在怀抱中时就跟大户人家订了亲的。只有小女儿四娘,是妾生的,母亲早就死了,十五岁了还没订亲。胡银台就把四娘许给了程生,招赘他为女婿。有人讥笑胡银台,认为他老糊涂了,胡乱许亲。胡银台毫不理会,打扫了房子,让程生住下,饭食、衣服都优厚周到地供给。公子们鄙视程生,不愿和他同吃,连仆人、奴婢们也常常挪揄程生。程生默默地忍受着,毫不计较,只是刻苦攻读。众人在一边故意厌恶地讽刺他,程生照旧读书,停也不停;那些人又鸣锣敲钟,前后捣乱,程生干脆拿起书本,到卧室里读。

起初,四娘还没出嫁时,有个神巫能预知人的贵贱。神巫把胡银台的子女们挨个看了一遍,都没有奉承的话。只有四娘来后,才说:“这是真正的贵人!”等到四娘嫁给程生,姊妹们都叫她 “贵人”,以此嘲笑她。但四娘性情端庄,寡言少语,听到别人这么叫她,就像没听见。渐渐地连丫鬟、婆子们都这么叫起来。四娘有个丫鬟叫桂儿,十分不平,大声说:“怎知我家郎君就不会做贵官?”二姊听到后,嗤之以鼻,说:“程郎如做了贵官,挖了我的眼睛去!”桂儿发怒地说:“到那时,恐怕舍不得两颗眼珠子!”二姊的丫鬟春香说:“二娘如果食言,我用我的双眼代替!”桂儿更加愤怒,拍着巴掌发誓说:“管教你们都成了瞎子!”二姊恼恨桂儿言语冲撞,甩手就给了她几巴掌,桂儿号啕大哭。胡夫人听说这件事后,也不置可否,只是微微冷笑了一声。桂儿吵嚷着向四娘哭诉,四娘正纺着线,听后不动怒也不说话,照旧纺织。

正赶上胡银台做寿,女婿们都来了,带来的贺礼摆满了屋子。大媳妇嘲笑四娘说:“你家送的什么寿礼呀?”二媳妇就说:“两肩挑着张嘴呗!”四娘面色坦然,一点也不羞惭。人们见她事事都像傻子一样,更加欺侮她。惟有胡银台的爱妾李氏,是三姊的生身母亲,总是敬重四娘,经常照顾怜恤她。还常嘱咐三娘说:“四娘外表憨厚,内里聪明,精明不外露。你那些姊妹兄弟们都在她的包罗之中,自己还不知道。况且程郎昼夜苦读,怎会久在人下?你不要效仿他们,应该善待四娘,将来也好见她。”所以三娘每次回娘家,总是加意和四娘交好。

这年,程生因为胡银台帮助,考中了秀才。第二年,学使驾临进行科考,正好胡银台去世了。程生披麻戴孝,像儿子一般悲痛。因为这事程生没能赶考。丧期过后,四娘赠给程生银子,让他补进“遗才”籍。嘱咐说:“过去你在这里住了这么久,之所以没被赶走,只因为有老父亲在。现在是万万不行了!倘若你这次去能考中举人,回来时还可能有这个家。”程生临别,李氏、三娘都赠送了很多礼物给他。

程生进了考场,发愤揣摩,仔细构思,以求务必考中。不久,放榜了,他竟榜上无名。程生没能实现夙愿,气怒不堪,没脸回家。幸亏银子还多,就带着行李进了京城。当时,胡家的亲家们大都在京城做官,程生恐怕他们讥笑自己,便改了名,编了个家乡住址,向大官家谋求差事做。有个姓李的御史大夫,是东海人,见了程生后很器重他,收他做了幕宾,并资助费用,给程生捐了个“贡生”,让他去参加顺天科考。这次,程生连战连捷,被授予“庶吉士”的官职。程生便跟李公讲了实情。李公借给他一千两银子,先派了个管家去四川,为程生买宅子。这时,胡大郎因为父亲亡故,家里亏空,要卖一处别墅,这个管家就买了下来。然后,又派车马去接四娘。

原先,程生考中以后,来了个报喜的。胡家一家人都厌恶听到这种消息。又审知名字不符,将报喜人赶走了。正好三郎结婚,亲戚朋友们都来送礼庆贺。姑嫂姊妹都在,惟独四娘没被兄嫂请来。这时,忽然有个人奔跑了进来,呈上寄给四娘的一封信。兄弟们打开一看,面面相觑,脸上失色。此时酒宴中的亲戚们才请见四娘。姊妹们惴惴不安,恐怕四娘怀恨不来。不一会儿,四娘竟翩然而来。那些人纷纷凑上去,祝贺的、搬座的、寒喧的,屋里一片嘈杂。耳朵听的,是四娘;眼睛看的,是四娘;嘴里说的,也是四娘。但四娘仍像以前一样凝重端庄。大家见她不计较过去,心中才稍微安宁了点。一会儿忽见春香跑了进来,满脸鲜血。众人一起询问,春香哭得回答不上来。二娘呵斥了她一声,春香才哭着说:“桂儿逼着要我的眼睛,要不是挣脱,眼珠子让她挖了去了!”二娘大为羞惭,汗流满面,把粉都冲下来了。四娘依旧不动声色,漠然置之。满座人一片寂静,接着便陆续告辞。四娘盛妆而出,惟独拜了李夫人和三姊,然后出门,登上车走了。大家才知道买别墅的就是程家。

四娘初到别墅,日用东西都很缺。胡夫人和公子们送来了仆人、丫鬟和器具,四娘一概不要,只接受了李夫人赠送的一个丫鬟。住了不久,程生请假回来扫墓,车马随从如云。到了岳父家,先向胡银台的灵柩行了祭礼,然后参拜了李夫人。等胡家兄弟们穿戴整齐要拜见程生时,程生已上轿打道回府了。

胡银台死后,他的儿子们天天争夺财产,把他的棺材扔在那里不理会。过了几年,棺木朽烂,渐渐地竟要把屋子当作坟墓了。程生见了十分伤心,也不和胡家兄弟们商量,自己出资,选了下葬的日子,事事尽礼,隆重安葬。出殡那天,车马接连不断,村里的人都赞叹不已。

程生做官十几年,两袖清风,乡亲们凡遇难事,他无不尽力。胡二郎因为人命案被牵连入狱,审案的官员,是和程生同榜考中的,执法非常严明。胡大郎央求岳父王观察写了封信给这个官员,人家却置之不理。胡大郎更加害怕,想去求四娘,又觉没脸见她,便让李夫人写了封信,自己拿着去了。来到京城,胡大郎不敢冒然进程家,看见程生上朝走了后,才登门求了见。盼望四娘念手足之情,忘记过去的嫌隙。门人通报后,便有原来的一个老妈子出来,领着他走进内厅,草草地摆上酒菜。吃喝完,四娘才出来,脸色温和,问道:“大哥在家事情很忙,怎么有时间不远万里来到这里?”大郎跪倒在地,哭泣着说了来由。四娘扶起他来,笑着说:“大哥是个好男子汉,这算什么大事,值得这样?妹子一个女流,你啥时候见跟人呜呜哭泣来?”大郎便拿出李夫人的信,四娘看了后说:“嫂子们的娘家,都是些了不起的天人,各自去求求自己的父亲、哥哥,就了结了,何必奔波到这里?”大郎哑口无言,只是哀求不已。四娘变了脸色,说:“我以为你千里跋涉而来是为了看妹子,原来是拿大案求‘贵人’来了!”一甩袖子,进了内室。大郎既羞惭,又恼恨,只好出来。回家后详细一说,一家大小无不痛骂四娘,连李夫人也觉得四娘太忍心了。

过了几天,胡二郎竟被释放回家。全家大喜,还讥笑四娘不肯相救,徒落了个被众人怨恨。一会儿,有人来报,四娘派了仆人来问候李夫人。李夫人叫进来人,那人送上带来的银子,说:“我家夫人为了二舅的案子,忙着派人料理,没顾上写回信铪您。让我送上这点礼物,以代信函。”此时,大家才知道,二郎的回来还是程生和四娘出力的结果。

后来,三娘家渐渐贫困,程生更加周到地接济她。又因为李夫人没有儿子,程生就把她接到自已家,像母亲一样养起来了。


\subsection{1.7.24   僧 术}
\label{\detokenize{p00_u5176_u5b83/_u767d_u8bdd_u804a_u658b_u5fd7_u5f02:id287}}
黄生,是官宦世家的子弟,富有才情,志向很高。他居住的村外有座寺院,里面住着一个僧人,跟黄生交情深厚。后来僧人外出云游,去了十多年才回来。他看见黄生,感叹地说:“我以为你早就飞黄腾达了,到如今还是一个平民百姓吗?看来你的福运很薄,请让我为你贿赂贿赂阴间的神灵。你能给置备十千银钱吗?”黄生回答说:“不能。”僧人说:“请你勉强置一半吧,其余的我代你借上。我们以三天为约。”黄生答应了,回家后抵押家当,勉强凑够了五千的数目。

三天后,僧人果然拿来五千钱交给黄生。黄家原来有一眼水井,井深得探不到底,有人说通着河海。僧人让黄生把钱捆好放在井边。嘱咐他说:“你约摸我到了寺里后,就把钱推进井中。等到半顿饭光景,井中会有一个大钱浮起来,你就拜它。”说完就走了。黄生不明白这是什么法术,转念一想,灵验不灵验还说不定,如果把十千钱都投进井中,未免太可惜,就把九千藏起来,只投进了一千钱。稍过了一会儿,井中突然凸起来一个大水泡,铿的一声破了。接着就有一个钱浮起来,像车轮一样大。黄生害怕极了,赶快跪拜,又取出四千钱投进去,落井后发出碰击声,原来是被大钱隔挡着,沉不下去。

天快黑时,僧人来了,责备黄生说:“为什么不把钱全投进去?”黄生说:“已经都投进去了。”僧人说:“阴府的使者只拿了一千去,为什么要说假话?”黄生只得把实情讲了。僧人叹息说:“鄙吝的人成不了大器。你命中注定到老也就是个贡生,不然的话,立即就能中进士。”黄生非常后悔,求僧人再给他祈祷,僧人坚决推辞,走了。黄生看见投到井中的四千钱还浮着,便用井绳把它钓上来,大钱就沉下去了。这一年,黄生果然仅考了个副榜贡生,到死也如同僧人所说的,仅是个贡生。


\subsection{1.7.25   禄 数}
\label{\detokenize{p00_u5176_u5b83/_u767d_u8bdd_u804a_u658b_u5fd7_u5f02:id288}}
有个既有名声又有地位的人,总做损人利己的坏事,妻子常用善恶报应劝他,他就是不听。

有个炼丹成仙的,据说能预知人的寿限,这人就去找他。炼丹人说:“你呀,再吃二十石米,四十石面,就死。”回来跟妻子说了,算一算一个人一年最多吃两石面,那么我还能活二十年呢。即使做坏事,看样子二十年内也死不了,于是仍像以往那么坏。

过了一年,忽然得了糖尿病,饭量大增,而且一会儿就饿,一天一夜吃十几顿还要吃,不到一年就死了。


\subsection{1.7.26   柳 生}
\label{\detokenize{p00_u5176_u5b83/_u767d_u8bdd_u804a_u658b_u5fd7_u5f02:id289}}
周生是顺天府官宦人家的后代,和柳生是好朋友。柳生得到过高人的传授,精通相面。曾对周生说:“你呀,这辈子得不到多大的功名;可是要想成为百万富翁,还可以想办法。可惜你的妻子生了一副没福气的薄命相,怕不能协助你发展家业。”不久,他妻子果然就死了。

妻子死后,家不像个家,日子简直没法过了。就想起了朋友柳生,打算请他帮忙再找一房妻室。进了柳生家的客厅,柳生在里屋好久不出来。周生喊了好几遍他才出来。对周生说:“我天天给你物色佳偶,现在才找到。刚才我是在屋里作了点法术,求月老给你系红绳呢。”周生听了很高兴,问他究竟进行得怎么样了,柳生说:“刚才有人提了个布袋出去,你看见了吗?”周生说:“看见啦,一身破衣服,像个乞丐。”柳生说:“哎,那是你未来的岳父,你应该尊敬他才是。”周生苦笑说:“我因为你是我的好朋友,才跟你讨论私事儿,你怎么跟我开这么大的玩笑?我尽管家境不好,好歹还是官宦世家,怎么就到了跟市井小人联姻的地步?”柳生说:“不对,犁牛还能生出红毛牛来呢。乞丐又有何妨?”周生问:“你见过他女儿吗?”柳生答道:“没有。我从来不认识他,连他的姓名还是问了以后才知道的。”周生笑道:“连犁牛都不知道,你又怎么知道小牛是什么颜色的呢?”柳生说:“我是算出来的。这个人凶恶而贫贱,可命中该有个福气大的闺女。但是勉强把你们撮合到一起一定有大灾大难。等我再问问神明。”

周生回家后,不大相信柳生的话;托媒人说了好几家。一家也没成。一天,柳生忽然来了,说:“有个客人,我已经替你下了请柬了。”周生问:“是谁呀?”柳生说;“先别问,快准备酒饭。”周生不明白,按柳生的意思准备。一会儿,客人到了,原来是个姓傅的兵。周生心中不愉快,表面上敷衍着。但是柳生却表现出很恭敬的样子。不大功夫,上来了酒菜,只是餐具非常粗劣。柳生站起来对客人说:“周公子早就仰慕您的大名,常托我替他找您;几天前才有幸见到您,又听说您很快要远征,决定立刻请您来,时间太仓促,准备得不好。”饮着酒,傅姓的兵谈到了他的马有病,不能骑了。柳生也低着头替他想办法。

等客人走了以后,柳生批评周生说;“这位朋友是千金也买不到的,你怎么对人家这么冷淡?”就借了周生的马,骑了回家去,又谎称是周生的意思,把这匹马送给了姓傅的。后来周生知道了,虽然不大高兴,也没办法了。

第二年,周生要去江西投奔到臬司幕下做事,找柳生给算算此行是吉是凶。柳生说:“大吉!”周生笑笑说:“找你算算也没别的意思,只为了一件事:在江西如果收入些钱财,我就买个好媳妇,以证明你以前说的话并不灵验,你说能吗?”柳生回答说:“你一切都能如愿。”

周生到了江西,正赶上大股贼寇叛乱,三年回不了家。后来局势稍平静了些,拣了个好日子登上归途。中途又被贼寇掳了去。一同遭难的有七八个人,他们都是被劫去了钱财以后获得了释放。只有他自己被带到贼窝里,贼头领问过了他的家世,说:“我有个小闺女,想把她嫁给你,你不要推辞。”周生不吱声。贼头儿生了气,命令立刻将他斩首。周生害了怕,寻思不如暂时应下,以后再慢慢摆脱。先保住性命要紧,便说:“小生之所以不敢答应,因为我是个文弱书生,当不了兵打不了仗,不更成了您的累赘了吗?您若答应我们小两口一起走,我会感激您的大恩的。”贼头儿说:“我正愁这丫头拖累我呢,这有什么不可以的。”说罢,领周生进了内宅,叫女儿妆扮好了出来与周生相见。周生一看,是位十八九岁天仙一样的美人。当晚就同了房,比周生想象中的好媳妇还要好上几倍。问起媳妇的姓名家世,才知她父亲就是当年那个提布袋的叫花子。话题扯到柳生的预言,夫妇二人都感叹了一番。

过了三四天,贼头儿要送他们走了,忽然大队官兵铺天盖地攻来,贼头儿全家都被捉住了。官军里三名将官负责监视他们,先把这姑娘的爹娘斩了。眼看轮到了周生,周生心想:这回活不成了。正在害怕,一位将官瞅了瞅他,说:“这不是周生吗?”原来,姓傅的兵已经因为立了军功,升为副将军了。傅对同僚说:“这人是我家乡一带大户人家的名士,怎么能是贼呢。”给周生松了绑,问他怎么到了贼窝。周生撒谎说:“我从江西娶了媳妇回家,谁想中途落到贼人手里。幸亏您来救了我,您的恩德太大了。只是我妻子和我在乱军中走散了,我求您帮我找找,叫我们团聚。”傅将军就命令俘虏们排成队,叫周生认人,果然找到了。傅将军给他们吃喝盘缠,说:“过去您对我有赠马的恩惠,我一天也没忘。您急着回家,时间仓促,来不及正经准备礼物,只送您两匹马、五十两银子帮助您回北方老家吧。”又派了两个骑兵,拿了通行证护送他们。

路上,姑娘对周生说: “我那傻爹不听劝,害得我娘搭上了命。俺娘儿俩早知道有今天这场祸。我为什么还希望多活两天?因为我小时候被一个相面的相过面,他说我命大,有福;我活下来好为老人收尸骨呀。我知道一个地方,埋着好多银子,挖出来把爹娘的尸骨赎出来,剩下的咱带回家去,够咱过日子的。”说完,嘱咐骑兵在路旁等一等,两人到了埋藏银子的地方,在烧成灰烬的房屋里用佩刀在地里掘出了银子,全装进包袱,回到原路,用一百两贿赂了骑兵,叫他把她爹的尸骨安葬;又领周生拜别了她娘的坟墓,才踏上归途。到了河北地界,又给了骑兵一笔厚厚的赏钱,就朝家中走去。

周生好久没回家,佣人们说准是死在外头了,就把家产哄抢光了。及至听说主人回来了,吓得全逃了,只有一个老婆子,一个婢女,一个老仆没走。周生觉得自己死里逃生已经够幸运了,就不追问。去访问柳生,已经不知哪去了。

女的持家比男人还强,在邻里中找忠厚老实的,给了资本叫他们去做生意,自己提成。若是这些做买卖的在屋檐下算帐,女的就在帘子里边听;外边算盘打错了一个珠,女的就能指出错在哪里。因此家里家外没一个敢欺骗她的。几年以后,联络的商人上了百,而家产就积累到几十万了。这才派人把双亲的遗骨移到自己家乡,用隆重的葬礼重新安葬了。


\subsection{1.7.27   冤 狱}
\label{\detokenize{p00_u5176_u5b83/_u767d_u8bdd_u804a_u658b_u5fd7_u5f02:id290}}
朱生,是阳谷县人,年龄不大,却性情轻薄、好开玩笑。一天,他因为死了妻子,去求一个媒婆给自己说亲。路上碰到那媒婆邻居的妻子,朱生瞟了一眼,见那妇人很美,便跟媒婆开玩笑说:“刚才碰见你的邻居,真是既文雅又秀丽,你若为我求偶,她就可以。”媒婆也开玩笑说:“你先杀了她男人,我再替你想办法。”朱生笑着说:“说定了。”

过了一个多月,媒婆的邻居出去讨债,被人杀死在野外。县令拘拿了死者的邻居和地保,拷问实情,却仍无头绪。只有那个媒婆招供了她和朱生开的玩笑话,县令因此怀疑到了朱生头上,将他逮捕了,朱生却坚决不承认。县令又怀疑死者的妻子跟朱生私通,谋害亲夫,将那妇人抓了去,用尽了各种酷刑拷打。妇人忍受不了折磨,胡乱招认了。县令又拿妇人的供词审问朱生。朱生说:“她一个柔弱妇人,受不了刑罚,她说的全是假的!既然她将要冤死,还要被加上不贞洁的名声;纵使鬼神无知,我又于心何忍呢?我实招了吧:想杀死她的丈夫再娶了她,都是我一个干的,她实在不知情!”县令问:“你有什么凭证吗?”朱生说:“有血衣可以作证。”县令便派人到朱生家搜取血衣,搜来搜去,却怎么也找不到。县令再次拷打朱生,打得他几次死去活来。朱生便说:“这是我母亲不忍拿出物证来让我去死,等我自己去取!”县令命衙役押着他回到家中。朱生告诉母亲说:“给我血衣,我是死;不给我也是死。反正都是死,还不如快点死去,也免得多受折磨。”他母亲听了,哭着进了内室。不一会儿,取出一件衣服来交给他。县令检查到衣服上确有血迹,人证、物证俱在,便判了朱生死刑。以后经两次复审,也都没有不同的证词。过了一年多,朱生马上就要被处决了。

一天,县令正在审案,忽有一人径直冲上公堂,瞪着眼大骂县令道:“你如此昏庸糊涂,怎么治理老百姓!”几十名衙役见状,一拥而上,想绑起他来,那人振臂一挥,衙役们呼啦啦倒了一片。县令大惊,站起身想逃,那人大喊道:“我是关帝跟前的将军周仓!昏官敢动,立即要你的狗命!”县令浑身颤抖,一动不敢动。那人说:“杀人的是宫标!与朱某有什么关系?”说完就一下子倒在地上,像死了一样。过了会儿才苏醒过来,还面无人色。等询问他的姓名,才知他就是宫标。县令拷打他,宫标招供了全部杀人罪行。

原来,宫标本是个无赖,知道那邻居讨债回来,以为他腰包里一定有很多钱,就在野外杀了他,没想到竟什么也没有。后来听说朱生被屈打成招,他暗自庆幸。这天,他稀里糊涂地冲进县衙,自己也不知是怎么回事。县令又问朱生那件血衣是哪里来的,朱生也不知。叫他母亲来询问,才知是他母亲割破自己的胳膊染的!检查朱母的左臂上,果然刀伤还没好,县令也大吃一惊。后来,县令因为这个案子被告发罢官,罚款赎罪,在羁留时死在狱中。

过了一年多,死者的母亲让媳妇改嫁,那妇人感激朱生的义气,便嫁给了他。


\subsection{1.7.28   鬼 令}
\label{\detokenize{p00_u5176_u5b83/_u767d_u8bdd_u804a_u658b_u5fd7_u5f02:id291}}
有个姓展的教谕先生,性情洒脱,有名士风度。然而,当喝酒后便狂放不羁,不拘小节。每逢外出喝酒回来,总是骑着快马驰过殿前台阶。台阶两侧有很多古柏。有一天,他喝醉了酒,骑马飞奔而来,撞到树上,碰破了头,自己说:“这是子路气我无礼,打破了我的脑袋!”半夜里就死了。

城里有个商人,到展先生家乡去做买卖,夜里住到庙里。到了深夜,忽然看到四五个人,带着酒菜来庙里喝酒,展先生也在里边。酒过三巡,一个人用字行酒令,说:“田字不透风,十字在当中;十字推上去,古字赢一锺。”一个说:“回字不透风,口字在当中;口字推上去,吕字赢一锺。”一个说:“囹字不透风,令字在当中;令字推上去,含字赢一锺。”又一个说:“困字不透风,木字在当中;木字推上去,杏字赢一锺。”最后轮到展先生,他深思了很久,也没想出来。大家笑着说:“既然说不出来,应当罚一锺。”其中一个很快递给他一锺。这时展先生说:“我有了:曰字不透风,一字在当中;……,”大家又笑着说:“推作什么字?”展先生端起锺来一饮而尽说:“一字推上去,一口一大锺!”引得大家捧腹大笑。过后没多久,他们一齐出门走了。

商人不知道展先生早已去世,以为他是卸官回了家。又回到乡里询问,才知道展先生死去多年。商人恍然大悟,夜里所见的是些鬼罢了。


\subsection{1.7.29   甄 后}
\label{\detokenize{p00_u5176_u5b83/_u767d_u8bdd_u804a_u658b_u5fd7_u5f02:id292}}
洛城有个刘仲堪,从小笨,过分爱读书,经常关门苦读,不和外界来往。

这天,他正读书,忽然闻到屋里充溢着一种奇异的香气。一会儿,又有裙子上的玉环声。惊愕中见进来一美女,头上金银首饰光彩照人,随从们也是皇宫内的打扮。刘仲堪吓得赶快伏在地上。美女扶起他说:“先生怎么从前那样傲慢,现在又这样恭敬呢?”刘仲堪更害怕了,说:“您是哪里的天仙,我还不认识您,什么时候对您无礼过?”美女笑了,说:“才几时不见你就忘了?正儿八经地坐着磨砖的不是你呀?”命令随从们铺下绣花绸地毯。摆了上等酒宴,拉他坐下饮酒,谈古论今,学问非常渊博。刘仲堪丈二和尚摸不着头脑,对答不上来。美女说:“我仅仅去王母娘娘的瑶池赴了一次宴会,你已经历了几死几生,一点灵性也没有了!”就吩咐佣人送来了“汤沃水晶膏”叫刘仲堪喝了。喝过之后,刘忽然觉得明白清彻起来。不一会儿,天黑了,随从都退去。息了蜡烛,二人脱了衣服尽情欢娱。

天不亮,佣人们又都来了。美女起床,头发一点不乱,不用梳妆。刘仲堪心中充满柔情,苦苦地问她的姓名,美女说:“告诉你也不要紧,只是怕你更起疑心。我就是甄后,你,就是刘公干再世。当年你为我犯了罪,我心不忍,现在相会,是为了稍稍报答你对我的一片痴情。”刘问:“魏文帝现在哪里?”美女说:“曹丕不过是曹操老贼的一个糟儿子而已。我偶然从上界下凡跟着他游戏了几年富贵生涯,事情过去,也不再想它了。曹丕因为他父亲曹操作恶多端,在地狱里呆了好久,现在不知他的消息。倒是他弟弟曹植,为天帝掌管典籍,有时能见着。”一会儿,看见院中停下一辆龙车,美女便赠给刘仲堪一个胭脂盒作纪念,道了别,上车驾着云去了。

刘仲堪从此文章才思大见长进,可是想那美女想得他如呆如痴。几个月后,身体渐渐要垮了。他母亲不知原因,很犯愁。家中有个老女佣,忽然对他说:“少爷您是不是想念什么人呀?”刘仲堪听她说中了自己的隐情,便将实话说了。女佣说:“少爷,您写封信,我能给送到。”刘听了惊喜地说:“你有这样的本事,以往我怎么没发现?真能给我送信,我忘不了你的好处。”于是写了信,交给她带去了。半夜,女佣回来了,说:“幸好没误事,我到了人家门口,看门的以为我是妖怪,想把我绑起来。我把少爷写的信拿出来,看门人拿了去,一会儿叫我进去了。那位夫人看了信也感叹不已,说不能再相会了,就想写回信。我说:‘我家公子病得不轻,不是写封信能治好的。’夫人沉思了一会儿,放下笔说:‘先捎个口信去,我会给刘郎送个俊媳妇去的。’我临走又嘱咐我:‘刚才的话是刘郎的终身大事,不要外传,就可以长久了。’”刘仲堪听了,高兴地等着。

第二天,果然有一老妇领个女郎到了他母亲那边,姑娘漂亮得世上少有。老妇自我介绍说:“姓陈,这是我亲生闺女,叫司香,愿做您的儿媳。”他母亲挺喜欢,就谈到下聘礼。老妇一点聘礼不要,直等到女儿跟刘仲堪成了婚才去。一家人只有仲堪知道这姑娘不是凡人,私下问她:“你是天上那夫人的什么人?”姑娘答:“我是曹丞相铜雀台的宫女。”仲堪怀疑她是鬼,担心夫妻不会长久,她说:“不是,我和夫人都名列仙籍了,因有过错,罚到人间。夫人已被召回,我罚期还不满。夫人在天上给我讲了情,临时叫我留在人间侍奉您。我去或留,全在夫人,所以咱夫妻不是暂时的。”

一天,有个瞎婆子牵条黄狗来要饭,敲着板唱小曲儿,姑娘出去看,还没站稳,黄狗挣断绳子要咬她,她吓得往回跑,已被黄狗咬破褂子。刘仲堪赶来,用棍子打狗,狗怒叫着仍然把咬下的布条嚼碎了。瞎婆抓住狗脖子的毛又拴上绳子走了。刘仲堪进屋看妻子,见她吓得还没平静下来。刘说:“你是仙人,怎么还怕狗?”妻说:“郎君不知,那狗是曹操变的,为我没守‘分香’的戒律,生我的气呢。”刘一听,想把那狗买来用棍子打杀,妻不同意,说:“上帝罚他为狗,哪能随便杀他?”住了两年,凡见过她的人都为她的美丽倾倒,可是问起她的身世,又总是含混其辞。于是都怀疑是妖怪,刘母问儿子,仲堪向母亲透露了一点儿,母亲害怕,叫儿子将那女子赶走,儿子当然不听。母亲便找了会驱妖的术士,来到院中作法。刚刚在地上划出筑法坛的位置,女的就知道了,悲戚地说:“本来盼望与郎君白头到老,现在老母怀疑,咱们的缘份到头了。要我走,也不难,但不是用这种驱妖术可以办到的。”就捆了一束柴,点了火,扔到台阶下,浓烟立刻将房屋遮住,对面不见人,伴随着雷一棒的响声,等烟消了,见那术士七窍流血死在那里。进屋去看,女子已不见了。呼喊老佣,也不知去向。这时刘仲堪才对母亲说:“老女佣大概是个狐狸精。”


\subsection{1.7.30   宦 娘}
\label{\detokenize{p00_u5176_u5b83/_u767d_u8bdd_u804a_u658b_u5fd7_u5f02:id293}}
温如春是陕西的一个世家子弟,从小就酷爱弹琴,即使出门在外住在旅店里,也一时一刻离不开琴。

一次,他外出到了山西,途中经过一个古寺,便下马进去休息。进了庙门,看见一个穿着布袍的道士,盘腿坐在走廊里。道士的竹杖倚在墙上,花布袋子里装着架古琴。温如春一看到琴就触动了自己的爱好,于是就问道士:“您也会弹琴吗?”道士答:“只是弹不好,愿意向行家学习学习。”说着,就把琴从布袋子里取出来递给温如春。温如春接过来观看,见琴纹理精妙,试着勾拨了一下,声音非常清脆悠扬。温如春很高兴,为道士弹了一支曲子,道士微微一笑,似乎感到还不够满意。温如春就把自己拿手的本领都用上弹了一番。道士笑着说:“还好,还好!可要做贫道的师傅还不够格啊!”温如春听他的口气很大,就请他弹几曲。道士把琴接过来放在膝上,才拨动了几下,就觉得和风徐来;又过一会儿,百鸟群集,庭院里的树上都落满了。温如春非常惊奇,就拜道士为师,向道士求教。道士把刚才的曲子又重新弹了几遍。温如春细细地听,用心地记,才稍微领会了曲子的节奏。道士试着让他弹,又加以指点引导,然后说:“学会了这些,在人间就没有对手了!”从此以后,温如春精心钻研,成了身怀绝技的高手了。

后来,温如春动身回故乡,离家还有几十里,天色已晚,又下起暴雨,一时找不到住处,看到路旁有个村庄,就赶快跑过去。进村顾不得选择,见有一个门户,便急匆匆躲了进去。进了屋,寂静无人,一会儿,出来一个十七八岁的姑娘,长得像天仙般美丽。她抬头见有生人,吓得急忙退回去了。温如春还没有娶亲,对这个姑娘产生了爱慕之情。这时,一位老太婆出来问他是干什么的。温如春说出了自己的姓名,并且要求借宿。老太婆说:“在这里住宿是可以的,只是没有床铺,如不嫌委屈自己,可以用草搭个地铺。”不多一会,老太婆点了蜡烛来,又把草铺到地上,显得很热情。温如春问她姓什么,她回答:“姓赵。”又问刚才那位姑娘是什么人,老太婆说:“她叫宦娘,是我的侄女儿。”温如春说:“我不自量,欲攀附高门结为婚姻,怎么样?”老太婆皱起眉头,现出为难的样子,说:“这件事却是不敢答应你。”温如春问她为什么,老太婆只说:“很难讲。”温如春感到失望,只好不再提了。老太婆走了之后,他看到铺草又潮又烂,没法睡上去,就端坐在那里弹琴,以便度过漫漫长夜。雨停之后,温如春不等天明就起身回家了。

县里有个退休在家的部郎葛公,很喜欢有文才的人。温如春有次去拜访他,他要温如春弹奏几曲。温如春弹琴时,只见帘幕后隐约有个女子在偷听。忽然,一阵风吹开了帘子,现出了一个十六七岁的姑娘,美貌无双。原来葛公有个女儿,乳名叫良工,善于词赋,是当地有名的美人儿。温如春动了爱慕之心,回到家中跟母亲说了,母亲便请了媒人前去提亲。葛公嫌温家家境破落,没有应承。但良工自从听了温如春的弹奏之后,心里暗暗倾慕,时常盼望再次聆听那美妙的琴声。而温如春因为亲事不成,愿望不能实现,心情沮丧,再也不登葛家的大门了。

有一天,良工在花园里散步,拾到了一张旧信笺,上面写着一首题为“惜余春”的诗词:“因恨成痴,转思作想,日日为情颠倒。海棠带醉,杨柳伤春,同是一般怀抱。甚得新愁旧愁,铲尽还生,便如青草。自别离,只在奈何天里,度将昏晓。今日个蹙损春山,望穿秋水,道弃已拼弃了!芳衾妒梦,玉漏惊魂,要睡何能睡好?漫说长宵似年,侬视一年,比更犹少:过三更已是三年,更有何人不老!”良工把诗词吟诵了三四遍,心里很喜欢,便把诗笺带回屋里,拿出精致华美的信笺,认真地抄了一遍,放在书案上;过后再找却找不到了,心想也许被风吹走了吧。正巧,葛公从良工绣房门口经过,抬到了这张锦笺,以为是良工作的词,厌恶词句轻佻,心里很不高兴,就将它烧了,又不好明讲出来,便打算把良工快嫁出去。这时,临县刘布政的公子正好派人前来提亲,葛公很高兴,但还想亲眼看看这位公子。刘公子来到葛家,衣着华美,长得大方英俊,葛公非常满意,对公子热情款待。既而公子告别之后,在他的座位下遗失一只绣花女鞋。葛公顿时憎恶刘公子的轻薄行径,把媒人叫来,告诉了这件事。刘公子一再替自己辩解;葛公不听,终于拒绝了刘公子的求亲。

原先,葛公种有一种绿色的菊花,自己珍藏着不外传。良工把这种绿菊花养在她的阁房里。这时,温如春的院子里有一两棵菊花也变成了绿色,朋友们听到这个消息,就上门来观赏;温如春也极为珍视这种绿菊。一天早晨,温如春去看菊花,在花畦边抬到写有《惜余春》的信笺,反复读了几遍,却不知道从哪里来的。因为“春”字是自己的名字,就更加喜爱它,便在书桌上详加评点,评语写得轻薄放荡。

葛公听说温如春的菊花变成了绿色,觉得很奇怪,便亲自到温的书房来探访,看到桌上的诗笺,拿起来便读。温如春觉得自己的评点有些不雅,伸手夺过来揉成了一团。葛公只看到一两句,认出了正是良工房门口拾到的那篇《惜余春》词,心中大疑;进而连温如春的绿菊,也猜想是女儿良工赠送的。葛公回家把这些事告诉给夫人,叫夫人审问良工。良工感到委屈,哭着要寻死。这事没有见证,无法证实。夫人也担心这事传扬出去名声不好,盘算着不如把女儿嫁给温生。葛公赞同,将此意转告给温如春,温如春喜出望外。这天,温如春遍请亲友参加观赏绿菊的宴会,焚香弹琴,直到深夜才结束。回房睡下后,书僮听到书房里的琴自己响起来,开始还以为是别的仆人弹着玩的,可仔细看琴旁并没人,这才向主人报告。温如春亲自到书房察看,确实是琴不弹自响。那琴声生硬而不流畅,好像是想学自己的弹法,可又没有学会。温如春点起蜡烛突然闯进去,房里空无一人。温如春便将琴带回自己的卧室,那琴一夜没有再发出声响。温如春认为是狐仙弹奏的,想拜自己为师学习弹琴。于是他就每晚弹奏一曲,将琴摆放原处任其弹拨,夜夜藏着偷听。到了第六七个夜晚,那琴弹奏的曲调,满可以听上一听了。

温如春成亲之后,和良工谈起过去的那篇《惜余春》词,才明白了他们所以能够成亲的原因,可始终不知道那诗词是从哪里来的。良工听到琴能自鸣的奇事,就去听了一次,说:“这不是狐仙,弹奏的曲调凄切痛楚,有鬼声。”温如春不相信,良工说她家有面古镜,可照出鬼怪的原形。第二天派人去取了来,等着琴自己响起来时,温如春握着镜子突然进了书房,用灯火一照,果然有个女子在,只见她慌慌张张地躲在房角,再也藏不住身了。温如春过去一看,原来是从前避雨时遇见的那位赵宦娘。温如春大为惊奇,就追问她。宦娘含着眼泪说:“替你们当媒人,不能说对你们不好吧,为什么这样苦苦地逼我呢?”温如春收起镜子,要宦娘不要再躲避,宦娘答应下来。温如春就把古镜装进镜袋。宦娘远坐一旁,说:“我是太守的女儿,已经死了一百年了,从小就喜欢琴和筝,筝懂得了一些了。只是琴没有得名师指点。所以在九泉之下,仍感遗憾!那次你冒雨进了我家,听到你的琴声,十分钦佩;你向我家求亲,我恨自已是死去的人,不能和你结成伴侣,所以暗地里设法帮助你们二人结成美好姻缘,来报答你对我的眷恋之情。刘公子丢失的红绣鞋,还有那篇《惜余春》词,都是我做的事,我报答教师不能说不尽心了。”温如春夫妇听了她的话,都非常感激地拜谢她。

宦娘又对温如春说:“你弹的琴我能领会多半了,可是还没有学到其中的神韵和道理,请你再为我弹一次吧!”温如春答应了,一面教她弹琴,一面讲解指法。宦娘特别高兴,说:“真是太好了,我能领会了!”说着起身要告辞。良工原来喜欢弹筝,听说宦娘擅长弹筝,就想听她弹一曲。宦娘答应了,就演奏起来。宦娘弹的声调和曲谱好极了,都不是人间能够听到的。良工边听边打着拍子赞叹,请求向她学习。宦娘执笔写了十八章曲谱后,又起身告辞,温如春夫妇再三恳切地挽留她。宦娘悲切地说:“你们夫妻俩多么幸福,知己知音,感情深厚,我这个苦命人哪有这样的福气!如果有缘,只能下辈子相见了。”说着她将一卷画像给了温如春,说:“这是我的肖像,若是你不忘媒人,可以挂在卧室里,高兴的时候,点上一柱香,对着我的像演奏一曲,那我就如同亲自领受了!”说罢,宦娘走出房门,消失不见了。


\subsection{1.7.31   阿 绣}
\label{\detokenize{p00_u5176_u5b83/_u767d_u8bdd_u804a_u658b_u5fd7_u5f02:id294}}
海州的刘子固,十五岁时,到盖县探望他的舅舅。看见杂货店里有一个女子,姣丽无双,心中便喜爱上了她。他悄悄来到店中,假托说买扇子,女子就喊她父亲。见她父亲出来,刘子固很沮丧,便故意跟老头压了个低价,走了。远远看见女子的父亲到别处去了,他又回到店里。女子又要找她父亲,刘子固忙阻止说:“不要去找了,你只要说个价,我不计较价钱。”女子听了他的话,故意说了个高价。刘子固不忍心和她争价,把身上所有的钱都给了她,就走了。

第二天,刘子固又来了,还像昨天一样。付了钱刚走出几步,女子追出叫他:“回来!刚才我说的是假话,价钱太高了!”便把一半钱还给了他。刘子固更感到她诚实。此后,趁她的父亲不在时,刘子固常来店里,慢慢跟她熟了。女子问刘子固:“你住在什么地方?”刘子固如实告诉她,又反过来问她姓什么?女子说:“姓姚。”刘子固临走时,女子把他所买的东西用纸包好,然后用舌尖舔一下纸边粘上。刘子固怀揣着包裹回去后舍不得打开,怕把女子的舌痕弄乱了。过了半个月,刘子固的作为让仆人发现了,私下告诉了他舅舅,硬让他回去。刘子固情意恳切,恋恋不忘,把从女子那里买的香帕脂粉等东西,秘密放置在一个箱子里。没人时,就关起门把东西拿出来看一遍,触景生情,思念不已。

第二年,刘子固又到盖县来。刚放下行李,就到店里去找那女子。到那里一看,店门关得紧紧的,刘子固失望地回去了。他以为女子同她父亲偶尔出门没有回来,第二天便早早又去,店门仍然紧关着。刘子固向邻居打听,才知道姚家原来是广宁人,因为这儿生意不好,所以暂时回广宁了,谁也不知他们什么时候再回来。刘子固神情沮丧,失魂落魄。住了几天,就怏怏不乐地回家了。母亲为他提婚事,他老是阻止。母亲觉得奇怪,又很生气。仆人偷偷把以前的事告诉母亲,母亲对他管制防范得更加严了。从此他再不能去盖县了。刘子固整日恍恍惚惚,吃不下饭,睡不着觉。母亲愁得没法,心想不如满足了儿子的心愿。于是,立即选了个日子,准备好行装,让儿子到盖县转达母亲的意思,让舅舅托人向姚家提亲。舅舅马上就去姚家,过了一会,舅舅回来,对刘子固说:“不好办了,阿绣已经许给广宁人了。”刘子固垂头丧气,心灰意冷。回家后,捧着箱子抽泣;常常徘徊思念,希望天下有第二个阿绣。

这时有媒人来提亲,夸赞复州黄家姑娘长得漂亮。刘子固担心媒人说的不确实,命仆人驾车到复州去看看。进了西城门,刘子固看见朝北的一家,两扇门半开着,门里有一个姑娘很像阿绣。再凝神一看,姑娘边走边回头看着进去了,一点不会错。刘子固大为动心,于是就去东边邻居家打听,知道这姑娘姓李。刘子固反复思索,疑惑不解,天下怎能有如此相像的人呢!住了几天,也没找着机会去见姑娘。只有两眼直盯盯地看着姑娘的家门,希望姑娘还能出来。一天,太阳正要落山,姑娘果然出来了。忽然看见刘子固,立即返身回去,用手指指身后,又将手掌放在额头上,然后进屋了。刘子固高兴极了,但不知姑娘是什么意思。沉思了好一会儿,就信步来到她家的房后。只见一座荒园寂静空旷。西边有一堵矮墙,只有齐肩高。刘子固豁然明白了姑娘的意思,于是就蹲下藏在草丛中。待了很久,有人从墙上露出头来,小声说:“来了吗?”刘子固答应着起来,仔细一看,真是阿绣。他悲痛万分,泪落如雨。姑娘隔着墙,探身用毛巾给他擦泪,不断地安慰着他。刘子固说:“我想尽了办法,愿望也没实现,自以为今生是没有希望了,怎想到还会有今天?你怎么到这里来的?”姑娘说:“李氏是我表叔。”刘子固请阿绣过墙来,阿绣说:“你先回去,把仆人打发到别的地方住,我会自己到的。”刘子固听从了她的话,坐在家里等着,一会儿,阿绣悄悄来了。没有浓妆艳抹,袍裤还是以前穿过的。刘子固挽着她坐下,详细诉说自己的相思之苦。于是又问:“你已许配人家,怎么还没有过门?”阿绣说:“说我已经许配人家,是骗你的。我父亲因为你家太远,不愿跟你们结亲,所以托你舅舅用假话骗你,以打消你的念头。”说完两人上床躺下,男欢女爱,不可言喻。四更刚过,阿绣急忙起来,翻墙走了。刘子固从此不再想黄家姑娘的事,住在这里忘了回去。一个月了还不回家。一天夜里,仆人起来喂马,见刘子固房里还亮着灯,偷偷一看,见是阿绣,非常惊骇,但不敢跟主人说。第二天一早起来,仆人到集市上访查了一番,才回去追问刘子固说:“夜里跟你交往的那人是谁呀?”刘子固开始不愿告诉他。仆人说:“这座房子太冷清了,是鬼狐聚集的地方,公子应当自爱。他姚家的姑娘,怎么会到这里来?”刘子固不好意思地说:“西邻是她表叔,有什么好怀疑的?”仆人说:“我已详细访查过了。东邻只有一个孤老太太,西边那家只有一个小孩,没有什么亲戚住在家里。你所遇到的一定是鬼怪。不然,哪有穿了几年的衣服还不换的?况且她面色太白,两颊略瘦,笑起来没有酒涡,不如阿绣美。”刘子固反复想了想,才非常害怕地说:“那怎么办?”仆人出谋说等她来时拿着家伙一块打她。天黑后,姑娘来了,对刘子固说:“我知道你怀疑我。但我没别的意思,不过是想了却过去的缘分罢了。”话还没说完,仆人推门进来,姑娘大声呵叱他:“把你的家伙扔了!快摆上酒来,我与你主人告别!”仆人一听便扔了兵器,就像有人夺走一样。刘子固更加害怕,勉强摆上酒席。姑娘像往常一样有说有笑,举手指着刘子固说:“知道你的心事,我正打算尽我的微力为你效劳,你为何想暗中害我!我虽然不是阿绣,但也自以为不比阿绣差。你看我真不如你过去的那个人吗?”刘子固吓得毛发倒竖,话也说不出来了。姑娘听着打三更了,拿起酒杯喝了一口,站起来说:“我暂时走了。待你洞房花烛之后,我再与新媳妇比比美丑。”一转身不见了。

刘子固听信了狐精的话,跑到盖县抱怨舅舅骗他,不愿住在舅舅家。搬到邻近姚家的地方住,托媒人给自己说亲,用丰厚的彩礼打动姚家。姚家妻子说:“我家小叔子为阿绣在广宁选了个女婿,阿绣的父亲为此到广宁去了,成不成还不知道。须等他回来后再跟他商量。”刘子固听了这些语,惶惶不安,没了主张,只好坚守在这儿等他们回来。

过了十几天,忽然听说要打仗。开始刘子固怀疑是讹传,时间长了,才知道是真的。他急忙收拾行装走了。中途遇到战乱,主仆二人失散,刘子固被军队的前哨抓住了。士兵认为刘子固是个文弱书生,便疏忽了对他的防备,刘子固便偷了一匹马逃走了。到海州地界时,看见一个女子,蓬头垢面,步履艰难,快走不动了。刘子固骑着马从她身边走过,女子忽然大声呼喊:“马上的人不是刘郎吗?”刘子固停下马仔细看她,原来是阿绣!他心中仍然害怕她是狐狸,说:“你真是阿绣吗?”女子问:“你怎么说这种话?”刘子固把他遇到的事说了一遍。女子说:“我真是阿绣。父亲带我从广宁回来,路上被士兵抓住。他们给我一匹马骑,可我老是从马上跌下来。忽然有一个女子,握着我的手腕拉我逃跑,我们在军队中乱窜,也没有人盘问。那女子跑得像飞鹰一样快,我拼命跑也跟不上。跑百十步就掉好几次鞋。跑了很久,听到人喊马叫渐渐远了,那姑娘才放开手说:‘告别了。前面的路都很平坦,你可以慢慢走。爱你的人就要到了,你同他一块回家吧。’”刘子固明白那女子是狐狸,非常感激她。刘子固就把留在盖县的原因告诉了阿绣,阿绣说他叔叔在广宁为她提了一个姓方的女婿,还没等送聘礼,战乱就开始了。刘子固这才知道舅舅说的不是假话。他把阿绣抱到马上,两人骑着一匹马回了家。

进门看到老母亲安然无恙,刘子固很高兴。他把马系好,向母亲讲述了事情的前后经过。母亲也非常高兴,急忙为阿绣梳洗打扮。妆扮好了,阿绣容光焕发,母亲拍着手说:“怪不得我那傻儿子在梦中都忘不了你。”接着铺好被褥让阿绣跟自己一起睡。他们又派人到盖县,送书信给姚家。没过几天,姚家夫妇一块来了,选定了吉日办完婚事就回去了。

刘子固拿出收藏的那只箱子,里面的东西原封没动。有一盒子粉,打开一看,脂粉已变为红土。刘子固很奇怪,阿绣掩口笑着说:“几年前的骗局,你今天才发觉。那时见你任凭我给你包裹,从来都不检查真假,所以就跟你开了个玩笑。”正在嬉笑时,一个人掀开门帘进来说:“你们这样快活,应当谢谢媒人吧?”刘子固一看,又是一个阿绣,急忙喊母亲,母亲和家里人都来了,没有一个人能辨认真假的。刘子固回头一看也迷惑了;看了很久,才朝一个“阿绣”作揖感谢。“阿绣”要了镜子自己看了一下,害羞地转身跑了,再找她时已没了踪影。刘子固夫妇感激她的恩情,在屋里设了一个灵位祭祀。

一天晚上,刘子固喝醉了酒回家,屋里黑黑的没有人。他刚要自己点灯,阿绣来了,刘子固拉着她问:“你去哪儿了?”阿绣笑着说:“看你醉成这样,臭气熏人,真让人讨厌。你这样盘问人,难道我跟男人幽会去了?”刘子固笑着捧起她的脸颊,阿绣说:“你看我与狐狸姐姐谁美?”刘子固说:“你比她好。但只看外表看不出来。”说罢关上门,两人亲热起来。一会儿有人叫门,阿绣起身笑着说:“你也是只看外表的人。”刘子固不明白她的意思,走去开门,却是阿绣进来。他十分惊愕,这才明白刚才那个是狐狸。黑暗里又听到笑声,刘子固夫妻望空中祈祷,祈求狐狸现身。孤狸说:“我不愿见阿绣。”刘子固问:“为什么不变成另一个相貌?”狐狸说:“我不能。” 刘子固问:“为什么不能?”狐狸说;“阿绣是我妹妹,前世时不幸夭折。活着时,她和我一块随母亲到天宫去,见了西王母,我们心里都暗暗爱慕她。回家后,我们就精心模仿西王母。妹妹比我聪慧,只一个月就学得非常神似;我学了三个月才学像了,但始终赶不上妹妹。如今又隔了一世,我自以为超过她了,没料到还跟从前一样。我感激你二人的诚意,所以此后会不时来一趟的,现在我走了。”于是不再说话。

从此狐狸三五天就来一次。家中一切难办的事都能解决。每当阿绣回娘家,狐狸常来住几天,家里人都害怕地避开她。每当家中丢了东西,她就打扮得整整齐齐,端立着,头上插着几寸长的玳瑁簪子,召集家人来庄重地告诉他们:“所偷的东西,今天晚上必须送回原来的地方;不然的话,就头痛大作,后悔也来不及。”天亮后,果然会在原来的地方看见被偷的东西。三年后,狐狸再没有来,偶然丢失了金银等贵重东西,阿绣模仿狐狸的妆扮做法,吓唬家人,也常常见效。


\subsection{1.7.32   杨 疤 眼}
\label{\detokenize{p00_u5176_u5b83/_u767d_u8bdd_u804a_u658b_u5fd7_u5f02:id295}}
有一个猎人,夜间到山中打猎。刚埋伏下,看到一个小人,身长约有二尺,孤零零地在沟底行走。一会儿,又来了一个小人,大小高矮和前一个一样。他俩相遇,互相问到哪里去。前一个说:“我要去看望一下杨疤眼。前天见他脸上气色不好,恐怕有大难临头。”后一个说:“我也是要去看望他,你说的一点不错。”猎人知道他俩不是人,便大声喊叫,霎时,两个小人都不见了。

这天夜里,猎人打倒一只狐狸,发现它的左眼皮上,有一块像铜钱那么大的疤眼。


\subsection{1.7.33   小 翠}
\label{\detokenize{p00_u5176_u5b83/_u767d_u8bdd_u804a_u658b_u5fd7_u5f02:id296}}
王太常,是江浙一带地方的人。他童年时,有一次白天卧床休息,忽然天色变得黑暗,雷电交加,一只比猫大一点的动物跳上床,躲在他身边.辗转不肯离开。一会雨过天晴,那动物便走了。这时他才发现不是猫,怕得不得了,隔着房间喊他哥哥。兄长听他讲明原委,高兴地说:“兄弟将来一定会做大官,这是狐狸来躲避雷劫的。”后来,他果然少年就中了进士,从知县一直做到监察御史。

王太常有个儿子名叫元丰,是个傻子,十六岁了,还分不清雌雄。就因为傻,乡里人谁也不肯把女儿嫁给他。王太常很是发愁。

有一天,有个老妇人领着一个姑娘找上门来,说是愿把姑娘嫁给王家做媳妇。那姑娘满脸带笑,漂亮得像天上的仙女。王太常全家很高兴,问那老妇人姓名,她自称姓虞,女儿名叫小翠,已经十六岁了。商量聘金时,老妇人说:“这孩子跟着我,吃糠还不得一饱。一旦住在这高房大屋里,有丫头仆妇供她使唤,有山珍海味给她吃,只要她舒心如意,我就心安了。这又不是卖青菜,还要讨价吗?”王夫人大喜,热情地招待了她们。老妇人叫女儿拜见了王太常夫妇,吩咐道:“这就是你的公公婆婆,你得好生侍奉他们。我很忙,先回去三两天,以后还要来的。”王太常叫仆人备马相送。那老妇人说她家离这儿不远,不必麻烦了,说完出门径自走了。小翠倒也没显出悲伤和依恋不舍的样子,就在带来的小箱子里翻寻花样,准备做活。王夫人见她很大方,心里很是喜欢。过了几天,老妇人未如约而来。王夫人问小翠家住哪里,她只是露出一副痴憨的样子,竟不知家住在哪里,怎么个走法。王夫人便收拾了另外一个院子,让小夫妇完婚。亲戚们听说王太常找了个穷人家的女儿做媳妇,不免暗地嘲笑一番。可后来见小翠伶俐漂亮,都大吃一惊,从此就再也不议论什么了。

小翠很聪明,会看公婆的脸色行事,老夫妇也特别疼爱她,唯恐她嫌元丰傻。小翠却有说有笑,好像满不在乎的样子。只是小翠太爱玩耍,常用布缝成个球,踢着玩,穿上小皮鞋,一踢就是好几十步远,骗元丰跑去拾取。元丰和丫鬟们跑来跑去,往往累得满身大汗。一天,王太常偶然经过,球从半空中飞来,拍的一声,正好打在脸上。小翠和丫鬟们连忙溜走,元丰还傻乎乎地跑过去拾。太常大怒,拣起块石子投过去,正打中儿子。元丰趴在地上又哭又闹。王太常回到房里,将事情的经过向夫人说了一遍,夫人过来斥责了小翠一顿。小翠一点不在意,低头微笑着,用手指在床沿上划来划去。夫人走后,她又照样胡闹,把胭脂粉抹在元丰的脸上,涂得五颜六色,像个花面鬼。夫人一见,气极了,叫小翠来怒骂一顿。小翠靠着桌子玩弄衣带,不害怕,也不吭声。夫人无可奈何,只得拿儿子出气,把元丰打得大哭大叫,小翠这才变了脸色,跪在地上求饶。夫人消了气,丢下棍子走了出去。

小翠把公子扶到卧室里,替他掸掉衣裳上的尘土,用手绢给他擦脸上的泪痕,又拿红枣、粟子给他吃。元丰止住啼哭,又高兴起来。小翠关上房门,把元丰扮做楚霸王,自己穿上艳丽的衣服,腰束得很细,扮成虞姬,姿态轻盈地跳起舞来。有时又把公子装扮成沙漠国王,自己头上插上野鸡翎子,手抱琵琶,丁丁铮铮地弹个不停,满屋子里充满了笑声。一天到晚,总是这样。王太常因为儿子傻,也就不忍心过分责备、埋怨小翠,即使偶而听到,也只好装聋作哑。

与王家同一巷子里,还住着一位王给谏,中间相隔只十几家,但王太常和王给谏向来不和。那时正逢三年一次的官吏考核,王给谏嫉妒王太常做了河南道台,想找机会暗算一下。王太常知道了,心里很着急,可是想不出对付的办法来。

一天晚上,王太常睡得很早。小翠穿上太官上朝的服装,装扮成吏部尚书的模样,剪了一些白丝绒做成大胡子戴上,又叫两个丫鬟穿上青衣装成官差,偷偷地从马棚里牵出马来,说是“去拜见王先生”。到了王给谏的大门口,便用马鞭打自己的从人,说:“我是要看王侍御的,谁要看什么王给谏啊!”拨转马头就走。到了自家门口,门房以为真的是吏部尚书来了,赶紧跑到上房向王太常禀报。王太常连忙起身出外迎接,才知道是儿媳妇开了个大玩笑。王太常气得脸色发白,一甩袖子回到房里,对夫人说:“人家正找咱的岔,想整治咱家,这可倒好,媳妇反而闹出这种丑事,咱家灾难临头了!”夫人也气得不得了,跑到小翠房里,又是训斥,又是责骂。小翠只是嘿嘿地傻笑,并不分辩。打她吧,不忍下手;休掉她吧,又无家可归。夫妇二人百般悔恨,一宿都没有睡好。

这时吏部尚书某公正声势显赫,他的穿着打扮和那天小翠装扮的一模一样。因此王给谏也以为真是吏部尚书,屡次派人到王太常门口打听消息。等了半夜,还没见吏部尚书出来,他怀疑吏部尚书和王太常正在商议什么机密大事。第二天早朝,王给谏见了王太常,便问道:“昨晚尚书到府上拜访了吧?”王太常以为他有意讥讽,满面羞惭,只是低声含糊地应了两个“是”字。王给谏越发怀疑了,从此不敢再暗算王太常,反而极力和他交好。王太常探得内情,暗暗高兴,但私下仍叮嘱夫人劝小翠以后不要再胡闹了。小翠也笑着答应下来。

过了一年,朝中首相被免职。恰好有人写了一封私信给王太常,误送到王给谏家里。王给谏大喜,便先托一位和王太常有交情的人,以此为要挟,向他借一万两银子。王太常拒绝了。王给谏又亲自上门来谈。王太常忙寻找官服,哪知怎么也找不到了。王给谏等了好一会,以为王太常摆架子,有意怠慢,气忿地正要离开,忽见元丰身穿皇帝的龙袍冠冕,有个女子从门内把他推了出来。王给谏一见吓了一跳,假意含笑,抚慰公子,把衣冠脱下来,交给从人带走了。等到王太常赶出来,客人已经走了。

王太常得知缘故,立时吓懵了,脸色如土,大哭道:“真是祸水啊!闯下这滔天大祸,眼看咱全家就要被抄杀满门了!”说着和夫人拿着棍杖去打小翠。小翠早已知道了,关紧房门,听凭他们叫骂,全不理睬。王太常见此情景,更是火上浇油,拿起斧子要劈门。这时,小翠在门里笑着劝公公说:“爹爹不要生气,有我在,各种刑罚自然由我承担,定不要您二老受牵连。爹爹要劈死我,这是想杀人灭口吗?”王太常一听有道理,这才把斧子扔下。

王给谏回去,果然上奏皇帝,揭发王太常谋反,有龙袍、皇冠为证。皇帝惊讶地打开验看,原来所谓皇冠是高梁秸子编的,龙袍乃是个破旧的黄布包袱皮。皇帝生气了,责怪王给谏诬陷好人。皇帝又把元丰叫来,一看,原来是个白痴。皇上笑了:“这样的傻瓜能当皇帝吗?”就交给法司看管。王给谏又指控王太常家中有妖人。司法官吏把王家的丫鬟仆人拘去审讯,大家都说:“哪有妖人?只有个疯疯颠颠的媳妇和一个痴呆呆的儿子,整天闹着玩儿罢了。”四邻八舍也是这样讲。这件案子才审定了,判王给谏诬告,充军云南。从这以后,王太常觉得小翠很不平常,又因为她母亲一去不回,就揣度媳妇莫非是个仙女吧!就让王夫人去询问。小翠只是笑,一句话也投有。夫人再三追问,小翠捂着嘴,笑道:“我是玉皇大帝的亲生女儿,娘还不知道吗?”

过了不久,王太常又升了官。这时他已经五十多岁了,经常为没有孙子而发愁。

小翠过门已经三年了,每夜都和公子分床睡眠。夫人就派人把公子的床搬走,嘱咐他和小翠睡一张床。过了几天,公子就找夫人告状了:“那张床搬走了,怎么老不归还?小翠每夜都把脚搁在我肚皮上,压得我都喘不过气来!又好掐人家的大腿……”丫鬟仆妇们听了都捂着嘴吃吃地笑,夫人连喝带打地把他赶走了。

一天,小翠在房里洗澡,元丰见了,要和她同浴。小翠笑着拦阻他,叫他等一下。小翠洗完澡出来,把热水倒在大瓮里,然后给公子脱去衣裳,和丫鬟扶着他下了瓮。公子觉得非常闷热,大叫着要出来,小翠不听,又用被子给他蒙上。过了一会儿,没有声响了,打开一看已经死去。小翠很坦然地笑着,一点也不惊慌,慢慢地把公子抬出来放在床上,给他擦干身子,随后盖上两床被子。夫人听到儿子洗澡给闷死了,嗷嗷哭着跑了来,骂着说:“疯丫头,怎么把我儿子给弄死了!”小翠微微一笑,说:“这样的傻儿子,还不如没有哩!”夫人一听这活,更是气得发疯,用头去撞小翠。丫鬟们连忙把夫人拉开。正闹得不可开交,一个丫鬟跑来报告:“公子哎哟着起来啦!”夫人收住眼泪,过去抚摸元丰,见他咻啉地喘着气,浑身冒大汗,把棉被也湿透了。过了一顿饭的功夫,汗也完了,元丰睁开了两眼,四下张望。看家里的人,好像一点不认识,开口说:“回想过去的事,真像做梦一样,这是怎么回事呀?”夫人听了这话,好像不是出自傻子之口,觉得很奇怪,领着他见王太常。太常多方试探,果然不傻了。一家都高兴得不得了,真是如获至宝。老两口又暗暗地叫仆人把原先抬走的床再抬回去,放在原处,铺好被褥。第二天再去看,被褥一动没动。从那以后,元丰的痴病再也没有复发,夫妻二人非常和谐,出出进进,形影不离。又过了一年多,王太常被王给谏一党的人弹劾,罢了官,还要受处分。王太常家中有个广西巡抚赠送的玉瓶,价值几千两银子,准备拿出来贿赂大官。小翠很爱这花瓶,常拿在手里玩。一次一不留神掉在地上,摔个粉碎。她十分羞愧,忙去告诉公婆。老两口正为丢官而烦恼,一听玉瓶摔碎了,气上心头,齐声责骂小翠。小翠气忿地走出房门,对元丰说:“我在你家几年,替你家保全的不止一只花瓶,怎么就这么不给我一点面子?老实对你说,我不是凡间女子,只因我母亲遭受雷劫时,受了你父亲的庇护,又因为咱们俩有五年的缘份,这才让我来到你家,一则是报恩,二则是了却这一点心愿。我在你家不知挨了多少骂,真是数也数不清了。我之所以没走,是咱俩五年缘分未满。如今我还能呆下去吗?”说罢,小翠气冲冲地走了出去。元丰追到门外,已经不知去向了。

王太常觉得自己做得过分,但后悔已来不及了。元丰走进房里,见到小翠用过的脂粉和留下的首饰,睹物思人,不禁号啕大哭起来。白天不吃饭,晚上不睡觉,一天天瘦下去。王太常很着急,想赶快为他续娶,以便解除他的悲痛,可是元丰仍不快乐,只是找来一位名画师,画了一张小翠的像,每天供奉祷告不已。

这样差不多过了两年。一天,元丰偶然因事从外地归来。那时天色已晚,明月当空。村外原有他家一座花园。他骑马从墙外经过,听到墙里有笑声,便停下来,叫马夫拉住马,自己站在鞍子上,隔着墙朝里望去,看见有两个姑娘在园中戏耍,因为月亮被云彩遮着,朦胧不明,看不甚清楚。只听得一个穿绿衣裙的姑娘说:“死丫头,该把你赶出去!”穿红衣裙的姑娘说:“这是俺家的花园,你反倒赶我,到底该赶谁呀!”绿衣姑娘说:“真不害羞,不会做媳妇,被人家休了出来,还敢冒认是你家的花园哩。”红衣姑娘说:“总比你这没有主的老姑娘强得多!”元丰听话音很像小翠,便连忙喊她。绿衣姑娘一边走一边说:“我暂时不跟你争论,你的汉子来了!”红衣姑娘走过来,果然是小翠。元丰高兴极了。小翠叫他攀上墙头,接他过去,说:“两年不见,你竟瘦得只剩一把骨头架子了。”元丰握着她的手,泪流满面,把思念之情详细给她讲了。小翠说:“我都知道,只是没脸再进你家大门。今天跟大姐在这里游玩,没想碰到了你,可见姻缘是逃不掉的。”元丰请她一同回去,小翠不肯;请她留在园中,她答应了。

元丰打发仆人回家回禀夫人。夫人一听,又是惊,又是喜,便坐着轿子赶来。走进花园,小翠迎接跪拜。夫人拉着小翠的胳膊,老泪纵横,真诚地检讨以前的过错,简直不能谅解自己。又说:“如果你心里不怀恨我,就请你一同回去,让我的晚年得到安慰。”小翠坚决推辞,不肯答应。夫人因为这花园太荒凉,打算多派些丫鬟仆人来侍奉。小翠说: “别的人,我都不愿见,只要原先的那两个丫头。相处的日子长了,我很相信她俩,就让她俩来吧。照应大门,派个老仆人就行。别的人一概用不着。”夫人就按小翠说的做了,对外人就说是元丰在花园里养病。每天送给他们食物和日常用品。

小翠常劝元丰另外娶亲,元丰不依。过了一年多,小翠的面孔和声音渐渐和从前不一样了。把画像取出来一对,简直判若两人。元丰非常奇怪。小翠说:“你看我比以前美吗?”元丰说: “今天你美倒是美了,但是跟从前不一样了。”小翠说;“你这意思是说我老了?”元丰说:“你才二十几岁,怎么会老呢?”小翠笑了笑,把画像烧了,元丰要去拿,已经变成了灰烬。

一天,小翠对元丰说: “公公说我到死也不会生孩子。现在双亲都年老了,你又孤零零一个弟兄也没有,我不会生育,怕要贻误你们的宗嗣。你还是另娶一房妻子,早晚可以侍奉公婆,你两面跑跑没有什么不方便的。”元丰答应了,就向钟太史家求亲。迎亲的日子快到了,小翠给新妇做了新的衣服和鞋袜,然后送到钟家去。新娘进门,她的容貌、言谈和举止,竟然跟小翠没有丝毫差异。元丰十分惊奇,到花园去找小翠。小翠已不知去向,问丫鬟,丫鬟拿出一块红巾,说:“娘子回娘家去了,留下这个叫我交给公子。”元丰展开红巾,上面系着一块玉玦,这是表示她永远与元丰分别了。元丰知道她不会再回来了,便带着丫鬟回去。元丰虽然时刻想念着小翠,幸而见到新娘犹如见到了小翠一样。

元丰这才明白:和钟家女儿成亲的事,小翠早已料到了,因此她先化成钟家姑娘的模样,这样就可以安慰元丰后来对她的思念啊!


\subsection{1.7.34   金 和 尚}
\label{\detokenize{p00_u5176_u5b83/_u767d_u8bdd_u804a_u658b_u5fd7_u5f02:id297}}
金和尚,是山东诸城人。他的父亲是个无赖,以几百钱的身价把他卖给了五莲山的寺院。因为金和尚从小无知愚笨,不能育经参禅,所以只能干些放猪赶集的杂事,就像个佣人一样。

后来他的师傅死了,遗留下很少的一点银子,金和尚就把银子揣在怀里离开寺院,作小商贩去了。他最善长干那些投机倒把、牟取暴利的勾当,数年间竟成了个大富户,在水坡里买了住宅和土地。他的徒弟非常多,吃饭的人数日以千计,村子四周有成百上千亩良田。他在村里盖起了几十座宅院,只住和尚不住杂人;即使有,也是些没有产业的穷人,携带着妻子儿女,来这里租赁他的房子和地当佃户。每一座宅院门内,四周房子相连,都是些佃户住在里面。和尚住的房舍在宅院中间:前边有大厅,重粱挂柱,彩绘金碧,耀人眼目;大厅里的几案、屏风,晶莹光亮,可以照出人影;再后边是寝室,里面挂着红色帘子和绣花帷幔,兰麝香味四溢喷鼻;檀木床上镶着螺壳画,上面铺着锦缎褥垫,折叠得有一尺多厚;壁上有很多名家的美人山水画,悬挂粘贴得几乎没了空隙。

金和尚只要一声长呼,等在门外的几十个仆人,便如雷鸣一样齐声答应。这些人头戴红缨帽,脚穿皮靴,都像乌鸦般聚集过来伸长脖子站着。他们接受吩咐时都用手掩着嘴说话,侧着耳朵听。若有客人突然来到,十几桌宴席只要哟喝一声,很快就可以办好。蒸熏烧煮的各种美味佳肴,纷纷摆上来,满桌上热气腾腾如下起了雨雾。只是不敢公开蓄养歌妓;但却有十几个美少年,都聪明伶俐讨人喜爱,他们头缠皂纱,口唱艳曲,让人听了看了觉得也很不错。

金和尚若是一出门,十几个骑马的随从便前呼后拥,腰里挎着弓、箭互相碰击发出声响。奴仆们称呼金和尚叫“爷”。就是本县的那些平民百姓,有称呼他“爷爷”的,有称呼他“伯伯、叔叔”的,而没有叫他“师父”、“上人”的,更无称呼他的法号的。他的徒弟出门,声势比金和尚略差一点,但是他们都骑着很威风的骏马,也和一般的贵公子大致相同。

金和尚又广为结纳,就是远在千里之外也有人和他及时互通消息,以此掌握地方军政长官的把柄。这些官员若偶而气盛冒犯了他,就先自己战战兢兢吓得不得了。金和尚的为人,粗俗不雅,从头到脚没有一块雅骨。他一生没有奉诵一经,没学会一咒,从来不到寺院;他的住室中未曾有过诵经用的金铙和法鼓这类器物,他的徒弟从未见到过,而且也没听说过。

凡是来租赁房屋居住的佃户,家中的妇女们打扮得就像京城里的人那样浮华艳丽,她们用的香脂、头油、花钿、铅粉,都是和尚们供给的,而和尚们对这类花销也毫不吝惜,因此村里顶名务农并不种地的人家有上百户。经常发生不守法的佃户砍下了和尚的脑袋埋在床下的事情,金和尚对此也不太追究,只是把这类佃户赶出村去就算完了,他们历来的习俗就是这样。

金和尚后来又买了个异姓人家的孩子,让他做自己的儿子。还专门请了个教书先生,教儿子学习科举功课。他的儿子聪明有文采,就让他进了县学,随即按照惯例成了太学生,不久,参加顺天府乡试,考中了举人。由此金和尚被人们称为“太公”并叫响了。过去称金和尚为“爷”的如今再加上个“太”字,原来对他行常礼的人现在都垂手改行儿孙礼了。

过了不久,太公和尚死了。金举人披麻戴孝,身卧草垫头枕土坯,面对灵床自称孤哀子;金和尚的徒弟们用的哭丧棒堆满了床榻;然而在灵帏后面嘤嘤细声哭泣的,惟有金举人的夫人一人而已。士大夫们全都盛装而来,揭起灵帏吊唁,官员们的伞盖、车马多得堵塞了道路。

到了出殡那天,搭的棚阁像云彩一样连成一片,旌幡幢盖遮天蔽日。用草扎的殉葬品,都用金帛装饰。车马伞盖和仪仗几十套;马有千余匹,美女近百人,都栩栩如生。方弼和方相两个开路神,是用硬纸壳制成的巨人,头束皂帕身穿金甲;里面虽是空的但却用木架支撑着,让活人在里面扛着它走。还在里面安装上能转动的机关,使开路神须屑飞舞,目光闪烁,像要呐喊一样。观看的人都感到很惊奇,有的小孩远远地看见它就吓得哭着跑了。为金和尚制作的冥宅壮丽得犹如宫殿,楼阁房廊连接足有几十亩地,里面千门万户,人进去就能迷路出不来了。祭品上的麟、凤、龟、蛇四灵物,人们大多都叫不出名字来。会合到这里来行送葬礼的人车盖相接,上自地方官员,他们都躬着腰进来,恭恭敬敬地按朝见的仪式起拜;下至本县的贡生和小吏,他们只能手扶地面行叩首礼,不敢劳累金举人和那些师叔们。

这个时候,人们倾城出动都来瞻仰,男男女女气喘挥汗,络绎不绝;有带着老婆抱着孩子的,有呼喊兄长寻找妹妹的,真是人声鼎沸。再掺杂上锣鼓吹打的喧闹声,各种杂耍戏剧的铿锵声,连人的说话声都听不见了。那些看热闹的人的身子自肩以下都被挤得看不见了,只能看到千万个人头在攒动。人群中有个孕妇肚子疼急了要分娩,几个女伴便张开裙子当作帷帐,围绕守护着她;只听到婴儿的啼哭,也来不及问是男孩女孩;裂下一块衣服包好孩子抱在怀里,有扶着她的,有拉着她的,很费劲地挤出去走了。这真是一大奇观啊!

金和尚入葬以后,把他所遗留下来的资产一分为二:一份归他的儿子金举人,另一份归他的徒弟们。金举人得到了一半家产,在他住宅的东西南北四周,都是和尚们的地盘;然而金举人与和尚们都是兄弟相称,他们之间的利益仍旧休戚相关。


\subsection{1.7.35   龙 戏 蛛}
\label{\detokenize{p00_u5176_u5b83/_u767d_u8bdd_u804a_u658b_u5fd7_u5f02:id298}}
徐公做齐东县令时,在他的县衙中有一座楼,是用来贮藏菜肴和食品的。可里面的东西经常被偷吃,还弄得地上狼藉一片。家人为此常常受到呵斥和责备,因此,就偷偷地藏在一边看看究竟是怎么回事。只见有一只大蜘蛛,像斗那么大。家人吓得连忙去告诉徐县令。徐县令感到很奇异,每天派奴婢们送些食物给蜘蛛吃。蜘蛛更加驯服,饥饿了就出来依附于人,吃饱后就离去。

这样总共过了一年多。徐县令一次偶尔批阅公文,大蜘蛛忽然爬到他的桌子上来趴着。徐县令以为它饿了,刚呼唤家人取食物,转过头来见两条蛇夹着蜘蛛卧在那里,蛇粗细如同两根筷子。蜘蛛爪子蜷起,肚子也缩着,好像非常畏惧。转瞬间,两条蛇突然暴长,像鸡蛋一样粗。徐县令大惊失色,想逃走,这时,雷霆大作,徐县令全家人都震昏了。过了一会,徐县令苏醒过来;奴婢仆人连同他的夫人共被击死了七人。徐县令病了一个多月,也死了。

徐县令为人正直,廉正爱民。在发运灵柩的那一天,老百姓自愿敛钱给他送葬,哭声遍野。


\subsection{1.7.36   商 妇}
\label{\detokenize{p00_u5176_u5b83/_u767d_u8bdd_u804a_u658b_u5fd7_u5f02:id299}}
天津有个商人,要出远门做买卖,从一个富人那里借了几百两银子作本钱,不幸被小偷看见了。到了晚上,小偷预先藏在他屋里等他回来;但商人因那天是个好日子,拿到钱就出发了。小偷等得时间久了,只听商人妻子在床上翻来复去,像难以入睡。一会儿,墙上忽然开了个小门,屋里通亮,门里出来个年轻漂亮的女子,手拉一条带子,走近床边递给商人妻子。商人妻用手推开,年轻女子固执地再递给她,她就接过去,起床,拴在梁上,伸进脖子,上吊了。年轻女子也就走了,墙上小门也关上了。小偷大惊,推开门逃了。

天明后,家里人见主妇吊死,报了案。官府捉商人邻居去,严刑拷打,邻居忍受不了折磨,只得承认杀了人,几天后就要被处决了。

小偷为邻居的冤枉不平,到官府自首,说了那夜亲眼见到的事实。官府不信,对他用刑,他也不改口供,说那是真的,邻居便免了罪。官府向其他邻人调查,都说那宅子的旧主人曾经有年轻媳妇吊死过,年龄、相貌跟小偷说的完全符合,因而知道那是年轻妇女的鬼魂。

人说暴死的人必然找人作替身,真是这样吗?


\subsection{1.7.37   阎 罗 宴}
\label{\detokenize{p00_u5176_u5b83/_u767d_u8bdd_u804a_u658b_u5fd7_u5f02:id300}}
静海有一个姓邵的书生,家里很穷。在母亲生日那天,他在院子里准备了供品做寿,磕了头起来,桌上的供品却全没有了。邵生很害怕,就去告诉母亲;母亲怀疑他因为家里穷买不起供品,故意诓她。邵生无法为自己辩白,只好默默不语。

不久,考官来到静海,邵生苦于没有路费,借了一点点钱去应试。途中遇到一人,在路边恭敬地等候他,还殷勤地请他去做客,邵生便跟他去了。看见楼台殿阁列满街路,进了门,一个大王坐在大殿上。邵生跪下磕头,大王态度和悦地叫他坐下,赐他酒食,说:“前些天从贵府经过,我的手下人又饿又渴,叨扰了你的好酒菜。”邵生不明白是怎么回事,大王又说:“我是地狱的四殿阎君。你不记得给你母亲过生日那天吗?”吃过酒,拿出一包袱白银说:“吃了你的酒和肉,用这个略作报答吧。”邵生接过包袱来,回头一看,宫殿、人一下子全没了。只有几棵大树孤零零立在道旁。看看赠的银子,是真的,称了称,足足五两。考试完毕,仅花了一半,便将剩下的银子拿回去孝敬母亲。


\subsection{1.7.38   役 鬼}
\label{\detokenize{p00_u5176_u5b83/_u767d_u8bdd_u804a_u658b_u5fd7_u5f02:id301}}
山西有个姓杨的医生,擅长针灸,还能叫鬼为他做事。一出门,那些牵骡的、拿鞭的都是些鬼。

曾经有天夜里杨医生从外地回家,和朋友一路同行。途中看见迎面走来两个人,又高又大,同常人大不一样。朋友很震惊,杨医生向前便问:“你们是什么人?”回答说:“长脚王、大头李前来敬迎主人。”杨医生说:“给我前边带路。”两人转身飞快向前走去,见杨先生落到后边时,就站住等等他,好像奴隶一样。


\subsection{1.7.39   细 柳}
\label{\detokenize{p00_u5176_u5b83/_u767d_u8bdd_u804a_u658b_u5fd7_u5f02:id302}}
细柳姑娘,是中原一个读书人的女儿。因为她的细腰柔软可爱,有人便半开玩笑地称呼她“细柳”。

细柳从小很聪明,善解文字,喜欢读相观的书籍。但她平素沉默寡言,从不评论别人好坏;只是有来求婚的,她必定要亲自暗中相看。看了很多求婚的人,都没相中,而她的年龄已经十九岁了。父母生气地对她说:“若天下始终找不到中意的男人,你还想梳着丫髻当一辈子老闺女吗?”细柳说:“我本想以人力胜天;可看了这么久没见有合适的男人,这也是我命该如此。从今往后,完全听凭父母作主。”

当时有个姓高的书生,是个出身于官宦世家的知名人士,听说了细柳的好名声,就和她订了亲。结婚以后,夫妇二人感情很好。高生的前妻死时留下一个儿子,小名叫长福,如今已经五岁,细柳抚养他很周到。有时她回娘家,长福总是又哭又叫地要跟着她,就是喝叱也不能阻止。过了一年多,细柳生了个儿子,给孩子取名叫长怙。高生问她取这个名字的含义,她回答说:“没有别的意思,只是希望他能长在身边罢了。”

细柳对于针线活很粗疏,常不在意;但是对于家里田地的位置,应纳赋税的数量,却都按着帐册查对,惟恐知道得不详细。过了很久,她对丈夫说:“家中的事务请你放下不要管了,留给我自已来办,看我能否当好这个家?”高生就按她说的做了。半年多时间家里的事情没有一件办不好的,高生也很佩服她的才能。

一天,高生到邻村喝酒去了,正巧来了个催交赋税的差役,在外敲门嚷叫。细柳叫奴仆出去说好话劝慰,可差役就是不走。细柳于是赶紧派童仆去把丈夫叫了回来。催税的差役走了以后,高生笑着说:“细柳,如今你才知道再聪明的女人也不如个痴愚的男子吧?”细柳听说这活,难过地低下头哭了起来。高生很惊异地挽起她的手劝解她,细柳始终也不高兴。高生不忍心让家务累坏了她,仍然想自己管家,细柳不同意。她早起晚睡,更加辛勤地料理家务。每次都是提前一年,就先储备下来年要交的赋税,因此整年也见不到催税的差役再登家门。她又用这种方法来计划吃穿,从此家里的开支更加宽裕了。于是高生这才大为高兴,一次曾和她开了个玩笑,说道:“细柳何细哉:眉细、腰细、凌波细,且喜心思更细。”细柳听完也给他对上了个下联,说:“高郎庄高矣:品高、志高、文字高,但愿寿数尤高。”

村里有个来卖好棺材的,细柳不惜重价买下来,钱数凑不起来,又多方向亲戚邻居求借。高生认为这东西不是急用之物,便一再劝她别买,细柳不听。棺材在家里存放了一年多,有家富户家里死了人,想用加倍的价钱登门来买。高生因为有利可图而和细柳商议卖掉棺材,细柳不让:问她为什么不愿卖,却又不说;再问她,眼里晶莹的泪花就要掉下来。高生心里很奇怪,但是又不忍心再违背她的意愿,也就算了。又过了一年,高生已经二十五岁,细柳坚决不让他再出远门。有时他回家稍晚了点儿,僮仆们便一个接一个地跑去又叫又请。于是同仁们都以此拿他开心。有一天,高生到朋友家里去喝酒,忽然觉得身体不舒服,就赶快往回走,到了半路掉下马来,竟然死了。当时正是炎热的暑天,幸好死者用的衣服被子都是细柳以前早预备好了的。村里的人这才都佩服细柳娘子能料事如神。

长福到了十岁那年,才开始学习作文。父亲死了以后,他娇惯懒惰得不肯读书,经常逃学出去跟着放牧的孩子玩耍。细柳先是责骂,见他不改,又用板条子打,但长福仍然愚顽如故。细柳对他无可奈何,就喊他过来告诉他说:“既然你不愿意读书,何必再勉强你呢?只是穷人家没有闲饭养活闲人,可换下你的衣裳来,去和僮仆们一块干活。不然的话,就用鞭子抽你,不要后悔!”于是给他穿上破衣服,叫他去放猪。回家就让他自已拿个碗,和那些仆人们一起去吃饭。过了几天,长福吃不了这个苦,哭着跪到堂下,表示愿意再去读书。细柳回过脸去朝着墙,置之不理。长福不得已,只好拿着鞭子哭着出了门。

残秋将要过去,长福还光着个膀子没有衣服,打着赤脚没有鞋穿。冷雨淋湿了,他缩着头顶活像个要饭的花子。村里人见了都可怜他,那些续娶后妻的人,都以细柳娘子为戒,很多人都对她的做法不满,议论纷纷。细柳对此也渐渐听说了,但却漠然置之,不往心里去。长福实在受不了这个罪,便丢下猪逃走了。细柳也不去追问。过了几个月,长福没处讨饭了,才面容憔悴地回了家;但又不敢急着进门,只好哀求邻居老太婆去和母亲说。细柳说:“他若能受得了一百棍子打,可以来见我;不然的话,他还是早一点离去。”长福听了这话,骤然进门,痛哭流涕地愿受棍打。细柳问道:“你今天知道悔改了?”长福说:“我悔改了。”细柳说:“既然知道悔改,就不必打了,可以老老实实地去放猪,要再犯了决不饶你!”长福大哭着说:“我愿意挨一百棍子打,请母亲再叫我去读书吧。”细柳不听,邻居老太婆在一边劝解,最后才答应了长福读书的请求。给他洗了头换上衣服,让他和弟弟长怙同师学习。长福自此发奋勤学,与以前大不相同,三年就考中了秀才。巡抚大人杨公,见了长福的文章很器重他,让官府每月都供给他粮食,资助他读书。

长怙非常迟钝,读了好几年书竟然写不了自己的姓名。母亲只好叫他弃学务农。长怙游手好闲惯了,怕干活劳累。母亲愤怒地说:“士、农、工、商四行各有自己的本业,你既不能读书,又不能种地,岂不要饿死填了沟壑吗?”说着立时用棍子打了他一顿。从此长怙带领奴仆们种地,若是一早晨晚起,母亲就责骂他。衣服饭食,母亲总是把好的给哥哥长福。长怙对此虽然不敢说,但是心中却暗自不平。农活干完了,母亲出钱让他去学习经商。长怙好淫嗜赌,到手的钱全弄光了,却谎称遇上了盗贼运气不好,以此欺骗母亲。母亲发觉后用棍子几乎把他打死。长福久久地跪在地上苦苦哀求,愿代替弟弟挨打,母亲的怒气才消了。从此只要长怙一出门,母亲就暗中探察他。因此长怙的劣行略微收敛了一下,但他并不是真心愿意这样的。

有一天,长怙去请求母亲,打算跟着几个商人去趟洛阳,实际上他是想借出远门的机会,痛痛快快地为所欲为。然而他却提心吊胆,惟恐母亲不答应。母亲听他说完了,毫无疑虑,立即拿出三十两碎银并为他准备好行装,最后又拿出枚银锭交给他,说:“这是你祖父做官时钱袋里的遗物,不能花掉,只可用它压装,以备急用。况且你是初次出远门学着经商,也不指望你赚大钱,只要这三十两银子亏不了本钱就心满意足了。”临走时母亲又一再叮嘱他。长怙满口答应着出了门,很庆幸自己的的计谋实现了。

到了洛阳,长怙便不再和商人们在一起,而是独自住在了有名的娼妓李姬的家里。才住十几宿的功夫,三十两碎银子就眼看花光了。他自以为有那锭大银子在钱袋里压底,一开始并没有想到自己身上会缺了钱;但等到拿出那银锭一看,才知道竟是假的。他简直吓坏了,脸都变了色。李老太婆看见他这番模样,便冷言冷语地对他不客气了。长怙心里很不安宁,然而钱袋空了又无处投奔,仍寄希望于李姬能看在这些天的情意上,不会立即就赶他走。不一会儿,有两个人手拿绳索进来,突然套住了他的脖子。长怙惊恐地不知道怎么办才好,悲哀地询问是怎么回事,原来李姬早已偷了那锭假银去告到了公堂上。长怙被带去见官,自己又不能辩解,受到了严刑拷打,几乎丧了命。他被押在监狱里,身无分文,又受狱吏的虐待,没办法只得向同牢的囚犯们讨点吃的,暂且苟延残喘。

起初,长怙刚一上路,母亲就对长福说:“你记住等二十天以后,要让你去一趟洛阳。我的事情多,恐怕忘了这事。”长福便问去干什么,母亲难过得要掉下泪来。他也不敢再问,就退了出来。过了二十天,长福又去问母亲。她叹了口气说道:“你弟弟现在轻浮放荡,就跟你以前逃学一样。当初我若不冒着个后娘虐待你的坏名声的话,你哪里会有今天?人们都说我心狠,可是我泪水淌满枕席的时候,人们就不知道了!”她一边说着一边流泪。长福恭恭敬敬地站在旁边听着,不敢再问。母亲掉完了泪,这才说: “因为你弟弟放荡之心不死,为此我故意给了他那锭假银子使他受点挫折,我估计他现在已经被逮进狱中了。巡抚杨大人待你很厚,你前去求他,这样既可以解脱长怙的死罪,也能使长怙感到惭愧而真正悔改。”

长福立刻就上了路。等到他进了洛阳,弟弟已经被逮起来三天了。他接着赶到监狱中去探望弟弟,见长怙面孔变得像鬼一样。长怙一见到哥哥就哭得抬不起头来。长福也和他一同大哭起来。当时长福在巡抚杨大人面前很受宠,因此远近的人都知道他的大名。县令知道了他是长怙的哥哥后,就急忙把长怙释放了。

长怙回到家,还怕母亲在生自己的气,便用膝盖跪行到她的面前。母亲看着他说:“这回可遂了你的心愿了?”长怙流着眼泪不敢再作声,长福也一同跪下了,母亲这才呵叱长怙起来。

从此长怙下决心痛改前非,家里的各种事务,他都很勤快地去办理;即使偶然懒散点,母亲也不责问他。过了几个月,母亲也不再提让他去经商的事,他想自己去请求又不敢,只好把意思告诉了哥哥。母亲听说后很高兴,尽力借贷了一大笔钱给了长怙。仅半年时间他就赚回了一倍的利息。这一年秋天,长福考中了举人,又过了三年考中了进士;弟弟长怙经商也聚积了上万两银子。

淄川县有个客居洛阳的人,说他曾偷着见过这位太夫人细柳。虽然已年过四十,却仍像三十多岁的人,而且她的穿戴也很朴素,和平常人家没有两样。


\section{1.8   卷 八}
\label{\detokenize{p00_u5176_u5b83/_u767d_u8bdd_u804a_u658b_u5fd7_u5f02:id303}}

\subsection{1.8.1   画 马}
\label{\detokenize{p00_u5176_u5b83/_u767d_u8bdd_u804a_u658b_u5fd7_u5f02:id304}}
山东临清的崔生,家中简陋贫穷,院墙破败不堪。崔生每天早晨起来,总看见一匹马躺在草地上,黑皮毛,白花纹,只是尾巴上的毛长短不齐,像被火燎断的一样。把它赶走,夜里又再回来,不知是哪里来的。

崔生有一位好友在山西做官。崔生想去投奔他,苦于没有马匹,就把这匹马捉来拴上缰绳骑着去,临行前嘱咐家人说:“如果有找马的,就说我骑着去山西了。”

崔生上路后,马一路急驰,瞬间就跑了一百多里路。到了夜里马不大吃草料,崔生以为它病了,第二天就拉紧马嚼子,不让它快跑,但马却乱踢着嘶叫不已,口喷着沫,同昨天一样雄健。崔生便任它奔驰,中午便到达山西。此后,崔生时常骑着马到集市上,看到的人无不称赞。晋王听到消息,用高价买这匹马。崔生怕丢马的人来找,不敢卖。住了半年,也没人找马,崔生就以八百两银子卖给了晋王府,自己又从集市上买了一匹健壮的骡子骑着回家。

后来晋王因为有急事,派遣校尉骑着这匹马到临清。刚到临清,马跑了,校尉追到崔生东邻家,进了门,却不见马,便向主人索要。主人姓曾,说确实没有见过马。等进到主人的房里,看见墙壁上挂着陈子昂的一幅画马,其中一匹毛色很像那匹马,尾巴上的毛被香头烧了一点,这才知道,那匹马原来是画上的马成妖了。晋王的校尉因为难复王命,就告了姓曾的。这时崔生有了卖马的钱,家中居积盈万,自愿找姓曾的赔偿马钱,交付校尉回去复命。姓曾的很感激崔生的恩德,却不知道崔生就是当年卖马的人。


\subsection{1.8.2   局 诈}
\label{\detokenize{p00_u5176_u5b83/_u767d_u8bdd_u804a_u658b_u5fd7_u5f02:id305}}
有个御史的家人,一次偶然站在街市上,一个穿戴华丽的人,过来和他攀谈。那人逐渐问起这个家人主人的姓名、官阶门第,家人都告诉了他。那人自我介绍说:“我姓王,是公主家里的内使。”两人越谈越投机。姓王的说:“如今仕途险恶,那些做大官的都依附贵戚作靠山,你家主人依靠的是什么人?”家人说:“没有。”姓王的说:“这叫舍不得小财,却忘了大祸!”家人说:“那么投靠谁好呢?”那人说:“我家公主以礼待人,能庇护他人。某侍郎就是通过我引荐给公土的。如果肯出一千两银子作见面礼,见公主是很容易的。”家人很高兴,就问他住在什么地方。姓王的指着他的大门说:“天天同住在一条巷子里,你不知道吗?”

家人回家把这事告诉了御史,御史也很高兴。准备了丰盛的筵席,叫家人去请那姓王的,姓王的高兴地来了。在宴席上,王某把公主的性情及生活中的琐事详细地讲了一遍,并且说:“若不是看在同住一条巷子的情谊,就是赏赐我一百两银子,我也不会效劳。”御史更加敬佩感激他。临别时,两人约定晋见公主的日子,姓王的说:“你准备好礼物吧。我瞅机会和公主说说,早晚一定会告诉你的。”

过了好多天,姓王的才来,骑着一匹很美的骏马,对御史说:“你快准备打扮,带上礼物跟我走。公主事太多,拜见她的人接连不断。从早到晚,没一点空闲。现在恰好有一点空,心须赶紧去,如果耽误了,还不知什么时候能见呢。”御史忙拿出金银厚礼,跟着他去了。弯弯曲曲走了十几里路,才来到公主府第的门前。御史下马恭候。姓王的先拿着礼物进去了。等了很久,姓王的才出来宣告说:“公主宣召某御史!”接着就好几个人一声接一声地传呼。御史弓着腰进了府门,见高堂上坐着一位美丽的女子,容貌姿色像天仙一样,服饰华丽耀眼。侍女都穿着锦绣衣服,站立两边。御史跪下参拜,公主传命在檐下赐坐,用金碗献茶。公主同他说了几句客套话,御史就恭恭敬敬地退了下来。接着从宫里传出赏赐的东西:一双缎靴和一顶貂帽。

回到家里,御史很感激那位姓王的,就拿名帖去登门拜谢。到了王某家,只见大门紧闭,里面没人。御史怀疑姓王的侍候公主还没回来。三天去了三次,始终也没见到。派人到公主那里打听,那里的大门也锁得紧紧的。询问附近的居民,都说:“这里从来没什么公主。前几天有几个人租过这几间房子,现在已经走了三天了。”使者回去告诉御史,主仆只有垂头丧气而已。

某副将军,带着很多钱晋京,想升官进爵,苦于没有门路。一天,一个穿皮袍骑骏马的人来拜访,自我介绍说:“我的内兄是皇帝的近侍。”喝完茶,他请副将军与他私下谈谈,说:“眼下有个地方正缺一位将军,你如果舍得多花些钱,我去嘱咐内兄在皇帝面前多宣扬你的能力,可以得到这个位子,有权势的人也夺不去的。”副将军怀疑他在胡言乱语,那人说:“这事你用不着犹豫,我不过想从内兄那儿抽一点小份子,将军这儿,我一文钱也不企望。咱们说定数目,立下文书,等待皇帝召见后,你再如数交钱。如若没有结果,你的钱还在你手里,谁还能从你怀里抢走呢?”副将军高兴地答应了。

第二天,那人来领副将军去见他内兄。他内兄自称:“姓田。”家里很富有,像王侯之家。副将军参拜时,姓田的非常傲慢,待答不理的。引见人拿着写好的文书对副将军说:“刚才同内兄商量,这事没有一万两银子是办不到的。请你在这后面画押。”副将军照办了。姓田的说:“人心难测,我怕他事后反悔。”那个引见人笑着说:“兄长过虑了,你既然能给他官做,还不能把他的官免掉吗?况且满朝将相,还有那么多愿意交接咱们还高攀不上的呢!将军前程远大,应该不会丧尽良心的。”副将军急忙表白,发誓,走时那人送他说:“三天之内一定给你准信。”

过了两天,太阳刚落山,有几个人大声吆喝着跑进来说:“皇帝正等着你晋见呢!”副将军惊喜万分,急忙赶到皇宫。只见皇帝坐在金銮殿上,武士们威严地站在两旁。副将军忙跪行大礼,三呼万岁。皇帝赐他坐下,慰问了几句话,看了看两旁的大臣说:“听说副将军武艺高强,英勇善战,今日一见,果然是个将军的材料!”又对副将军说:“那地方很险要,现在委派你去当将军,不要辜负了朕的一番心意,委任的公文很快就下了。”副将军谢恩退了出来。接着前几天那个穿皮袍骑骏马的人跟到他家里,按照字据上的数目把钱拿走了。于是副将军便放心地等着委任的公文,整天向亲朋好友吹嘘他的荣耀。过了几天,探听到消息,那个将军的空缺已被别人补上了。副将军惊怒交加,跑到兵部大堂忿怒地争辩说:“我是皇上亲自封到那地方的将军,你们怎么另派别人去了?”兵部长官很奇怪,听他讲述皇帝召见时的情景,一多半倒像在梦境里。兵部长官火了,把他抓起来押到狱中。这时副将军才供出引见他的那个人的名字,可是朝中并没有这么个人。副将军又花了一万两银子,才被革职释放了。

奇怪呀!这个武官虽然呆傻,难道朝廷也是假的吗?这当中一定使用了幻术吧!所谓真正的大盗并不拿刀枪,就是指这些人了。

嘉祥县有个姓李的书生,琴弹得很好。一次他偶尔去东郊游玩,看见二人从土里挖出一架古琴,就用很少的钱买了下来。回到家中把琴擦干净,琴身发出一种奇异的光彩。安上弦弹奏,音调非常清烈,李生高兴极了,如同得到了一块宝玉,用锦囊装起来,藏进密室里,就是至亲好友也不拿出来给看看。有个新上任的县丞姓程,拿着名帖去拜访李生。李生性格孤癖,很少交朋友,因为县丞是先来拜访他,他只好去回拜了。过了几天,县丞又来请他喝酒,李生推托不掉,就去了。县丞风流文雅,谈笑潇洒不俗,李生心里很喜欢他。过了一天,李生拿了请帖回请县丞。两人欢声笑语,谈得十分融洽。从此,花前月下,两人常在一块饮酒谈笑。

过了一年多,李生在县丞的住处,偶然看见桌子上有一架用锦囊裹着的琴。李生便拿出来弹了几下,县丞问:“你也懂琴吗?”李生说:“这是我平生最爱好的。”县丞惊讶地说:“咱们交往不是一天了,你的绝技我怎么从来没听到过?”于是拨开香炉,烧起沉香,请李生弹奏。李生弹了一曲,县丞说:“果然是高手!我也愿献小技,请不要见笑!” 接着弹了一首“御风曲”,声音清脆悦耳,给人一种飘然欲仙、超脱尘世之感。李生非常佩服,愿拜他为师。从此两人又成了琴友,友情更加深厚。又过了一年多,县丞将自己的琴技全都教给了李生。然而,县丞每次到李生家,李生还是拿一般的琴给他弹奏,从没有泄露珍藏的古琴。

一天晚上,两人喝得略有醉意,县丞说:“我新演习了一首曲子,你愿意听吗?”说完,弹了一曲“湘妃”,如泣如诉,声调幽怨,李生连声称赞。县丞说:“遗憾的是没有一架好琴!如果有一架好琴,音调会更加动听。”李生高兴地说:“我藏着一架古琴,这琴非同一般。如今遇到知音,怎敢藏着不拿出来呢?”于是到密室,打开柜子拿出古琴。县丞用衣襟掸掸琴上的尘土,放在桌上,弹了一曲,音调果然强弱分明,弹出的曲子精妙入神,李生听得不停地打着拍子。县丞说:“我这点拙笨的琴技,辜负了这架好琴。如果能叫我妻子弹奏,可能还有一两声中听的。”李生惊奇地说:“你妻子也精通琴技吗?”县丞说:“刚才的曲子就是从我妻子那儿学来的”。李生说: “可惜在闺房之中,小生听不到她弹奏。”县丞说:“我们俩关系密切,不必受俗礼约束。明天,请你带琴到我家去,我叫她隔着帘子为你弹奏。”李生高兴地答应了。

第二天,李生拿着琴去了。县丞准备了酒菜,两人相对痛饮。过了一会儿,县丞将琴拿进去,转身又出来坐下。这时见帘内隐隐约约出现一个美人,浓郁的香气透过帘外。又过了一会儿,琴弦声幽幽飘来,李生也听不懂弹的什么曲子,只觉得心猿意马,神魂颠倒。曲弹完了,便有人掀开帘子一角往外偷看。李生一瞅,原来是一位二十来岁的绝代佳人。县丞用大杯劝酒,帘内又弹起了“闲情之赋”。李生意动神摇,喝了一杯又一杯,最后酪酊大醉,离席告辞,索要古琴。县丞说:“你喝多了,怕路上跌倒摔了古琴。明天你再来,我让妻子把她的绝技献出来。”

第二天,李生去拜访县丞,只见县丞的住处静悄悄的,只有一个老仆看门。李生问老仆,老仆说:“五更天就带着家眷走了,不知道干什么去了。说是三天以后回来。”三天后,李生又去程家,等到天黑,也没有踪影,县里的官吏和衙役们都起了疑心,报告了县令。打开县丞的房门一看,屋里什么也没有了,只剩下桌椅和空床。就将此事报到省府,也弄不明白是怎么回事。

李生丢了古琴,吃不下饭睡不着觉,只好不远千里到县丞的老家湖南去找。三年前,县丞拿钱在嘉祥买了官做——李生按他的姓名,到他的家乡打听,湖南并没有这么个人。有人说:“有个姓程的道士,会弹琴,传说还有点金的法术。三年前,忽然走了,没有再回来。”李生怀疑就是这个道士,又详细询问了年龄、相貌,完全一样。这才知道程道士所以花钱买官做,全是为了骗那架古琴。两人交往一年多,从不谈音乐方面的事,渐渐拿出琴来,渐渐卖弄琴技,又渐渐用美人来迷惑他,下了三年功夫,终于把古琴骗走了。程道士对琴的嗜好,更甚于李生。天下的骗子,诡计多端,像程道士这样,可算是骗子中最风雅的了。


\subsection{1.8.3   放 蝶}
\label{\detokenize{p00_u5176_u5b83/_u767d_u8bdd_u804a_u658b_u5fd7_u5f02:id306}}
长山县进士王(山斗)生,作县令的时候,每次审理案件,总是按照违法的轻重程度,罚犯人交纳蝴蝶来为自已赎罪。大堂上常常同时放出千百只蝴蝶,好像风吹着无数碎锦在飘舞。每逢此时,王(山斗)生就拍案大笑。

一天夜间,王(山斗)生梦见一位女子,穿着华丽的衣服,从容地走了进来,对他说道:“因为遭受你的虐政摧残,我的姊妹们有许多都夭亡了。应当让你因自命风流先受到一次小小的惩罚。”说完,化为一只蝴蝶。迥旋飞翔而去。第二天,王(山斗)生正独自在县衙中喝酒,忽然下人来报告说直指使来到了。王(山斗)生慌忙出来迎接,他妻子跟他开玩笑,用簪子插在他头上的一朵白花也忘记摘了下来。直指使看见了,认为他态度不恭,将他狠狠地责骂了一顿。从此,罚犯人交纳蝴蝶自赎的命令就停止了。

青城县的于重寅,性格狂放荒诞。他当司理的时候,有一年的元宵节晚上,他把烟花爆竹捆缚在一头驴身上,从头到尾都捆满了。他牵着驴来到太守门前,敲着梆子请太守出来,说:“我献上一头火驴,希望太守出来看一看。”当时太守因爱子正患天花,心情很不好,就推辞不出来。于重寅坚持请求,太守不得已,就叫守门的仆人打开锁。门刚刚打开,于重寅就用火点燃了引芯,把驴推进门内。爆竹爆炸驴子受到惊吓,又跑又窜,狂奔乱跑,身上的烟花喷火射人,没人敢靠近它。火驴穿过大厅进入室内,盆盆罐罐被撞被踏毁,很多东西都烧成了灰尘,窗纱也都烧成了灰烬。家人们大声惊呼。生天花的儿子受到惊吓,当天夜里就死了。太守对于重寅非常痛恨,准备向上级弹劾他。于重寅请托各司、道长官出面说情,并亲自登门负荆请罪,才免于遭受弹劾。


\subsection{1.8.4   男 生 子}
\label{\detokenize{p00_u5176_u5b83/_u767d_u8bdd_u804a_u658b_u5fd7_u5f02:id307}}
福建总兵杨辅家有个美少年,肚子里蠕动。满了十个月,梦见神人把他的两肋骨割去,醒来发现左右有两个男婴在啼哭。起身看看自己的肋下,割的痕迹还在。两个儿子一个起名天舍,一个起名地舍。


\subsection{1.8.5   钟 生}
\label{\detokenize{p00_u5176_u5b83/_u767d_u8bdd_u804a_u658b_u5fd7_u5f02:id308}}
钟庆余,是辽东名士。因参加乡试,来到济南府。听说藩王府邸有一位道士,能预知人的吉凶祸福,心中很想去看看。

二场考完后,他来到趵突泉,正巧在这里遇到道士。道士看上去六十多岁,长长的胡须飘在胸部以下,是一位银须白发的道长。聚拢在道士四周询问凶吉的人,像堵墙一样围得水泄不通。道士只用几句简单的话回答他们。道士在众多的人中看见钟庆余,很高兴地与他握手,并且说:“你的心术品行,令人敬佩。”说完,挽着钟的手登上阁楼,避开别人,问他说:“莫不是想知道你的将来如何?”钟庆余说:“是的!”道士说:“你的福命太薄,但这一科中举,是有希望的。但是,你荣归以后,恐怕就不可能见到你的母亲了。”

钟庆余是一位孝子,听到道士的话,流下泪来。便不想再继续考下去,想回家乡。道士说:“你若错过这个中举的考试,以后恐怕不会有这样的机会了。”钟生说:“母亲临死不得相见,将来让我再怎么作人,即使贵为公卿将相,又有什么意思?”道士说:“我前世与你有缘,眼下,我定尽一切力量帮助你。”说完就取出一丸药送给钟生说:“你可以先打发一个人连夜赶回去,将这丸药给你母亲服了,可延命七天。待考毕再赶回去,你母子还来得及见面。”

钟生将丸药藏好,就匆匆地离开道士,精神颓丧。心中想,母亲寿终为期不多,早归一天,就可对母亲多奉养一天,就带着仆人赁了头驴子,马上东归。赶着驴子走了一里多路,驴子忽转头向后跑。仆人在后头赶,它不驯服;牵着笼头,它就尥蹶子。钟生无计可施,急得挥汗如雨。仆人劝说先停下,钟生不听。又另赁一头驴,结果也是一样。看着日已落山,不知到底该怎么办。仆人又劝说:“明天就要考完了,何必去争这一早一晚?请让我先回去,这个办法也可以”。钟生迫不得已,就听从了仆人的话。

第二天,钟生潦潦草草地考完,即刻动身,顾不上吃饭睡觉,披星戴月而归。回到家,听说母亲病势垂危,吃下道士送的丹药,渐渐地痊愈了。钟生走进母亲的房间,见到母亲,在床边就流下泪来。母亲摇摇头,不让钟生哭,拉着他的手欢喜地说:“刚才作梦,我到了阴间,见到阎王,神色很和气,说:‘查看你的一生,没犯过大罪恶;现今念你的儿子很孝顺,再赐你阳寿十二年。’”钟生听了很高兴。过了几天,母亲的病果然平复了。

又过了不几天,钟生听到自已考中的消息,便辞别了母亲,来到济南府。到藩王府邸,送了点礼品给内监,让内监致意道土。道士很高兴地从里面出来,钟生便跪下给他磕头。道士说:“你既考中举人,太夫人又增了寿数,这些都是你自已盛德的报应,贫道哪有这回天之力啊!”钟生从话中,又惊讶其先知,于是就向道士拜问自已终身的祸福。道士说:“你没有多大的富贵,只要能活到八十九岁也就满足了。你的前身与我同是和尚,因用石头打狗,误将一只青蛙致死,这只青蛙已投生为驴。按前生的定数,你应当意外地早死。今因你的孝德,感动了神灵,已有解星进入你的命运之中,所以,应当没有别的危险了。但是你的妻子,前生不贞节,命里注定该年轻守寡。现今,你因为德行而延长了寿数。她就不再配作你的妻子了,恐怕一年之后,你妻子就要死的。”钟生悲伤很久,又问续娶的妻子在什么地方。道士说:“在河南,今已十四了。”道士临分手时嘱咐说:“倘若以后遇到危难,应逃向东南。”

一年后,钟生妻子果然死了。钟生的舅父在西江一个县作县令,母亲让钟生去探望舅父,顺便路过河南,验合当娶继室的预言。偶然到一个村庄,正遇上在河边演戏,男男女女处在一起。钟生刚想驱骡快点赶过去,有一匹失去缰绳的公驴,跟随着他而行,惹得钟生的骡子老尥蹶子。钟生回头,用鞭子击打驴耳,驴受惊狂奔。这时,正巧有一位王子,才六七岁,奶妈正抱着坐在河堤上,驴冲过来,侍从人员没来得及提防,把小王子挤到了河里。众人大喊大叫,想把钟生抓起来。钟生放开骡子,拼命地跑;又想起道士的话,极力向东南奔去。

大约跑了三十多里,到了一个山村,有一位老汉站在门旁,钟生下骡行礼。老汉把他请到屋里,自己介绍说:“姓方。”就问钟生从何处来。钟生叩头在地,将所遭遇到的如实说了。老汉说:“这不妨。请暂且住在这里,我会派人去打听消息的。”到晚上,得到消息,才知被惊的是小王子。老汉惊骇地说;“别的事,我尚能帮忙,这件事,我真是爱莫能助。”钟生哀求不已。老汉出计谋说:“没有别的办法。请你在这里住一晚,听听缓急,我们再作打算。”钟生忧愁恐怖,一夜没有入睡。第二天,老汉派人出去探听消息,听说官府已行文追查逃犯。谁若藏匿逃犯,杀头示众。老汉很为难,默默地进到屋里。钟生又疑虑又恐惧,惶惶不安。

半夜,老汉走进来,问: “家中夫人多大了?”钟生告诉说自己鳏居。老汉高兴地说:“我有办法了。”钟生问他,老汉回答说:“我的姐夫仰慕佛道,在南山出家,姐又死去。遗有小女,跟着我过活,这孩子也颇聪慧,将她嫁给你为妻怎样?”钟生高兴正符道士的预言,而且有了亲戚关系,可以得到救助,便说:“小生实在荣幸。但是,我这远方的罪人,恐怕连累岳丈。”老汉说:“这是为你着想。我姐夫道术颇深,但他很久不与人世来往。结婚后,你自己与我外甥女筹划一下,去求他必定有好办法。”钟生很高兴,就作了老汉的外甥女婿。

女郎才十六岁,容貌艳丽,是世上无双的美人。钟生常对之欷觑慨叹。女郎说:“我虽然长得不漂亮,也不至于这么快就被你嫌恶呀?”钟生道歉说:“娘子长得如同仙人,我能与你相配,实是万幸。但我有祸患,非常担心好事反成坏事。”就将实情告诉了女郎。女郎埋怨说:“舅舅行事,不通人情!这等弥天大祸,是没法子的,事前也不与我说明白,这不等于把我推到陷阱里么。”钟生长跪说:“是我死命地哀求舅舅,舅舅虽然慈悲,但他自己也没办法,知道你能起死回生。我诚然不足称得上是一位好丈夫,然而我家的门第,倒也不辱没您。倘若我有再生之日,诚心诚意地供养你,是指日可待的。”女郎叹气说:“事情已到这地步,我有什么可推辞的?可是,父亲自从削发出家,儿女之情已经断绝。没有别的法子,与你一同去哀求他,恐怕要受些挫折和凌辱。”于是,两人一夜未睡,用毡绵作了厚厚的护膝,藏在衣服里面;然后,叫来轿子,进了南山十多里。山路曲折险峻,再也无法乘轿了。下轿后,女郎走路很艰难。钟生用手臂搀扶着她,摔了无数跤才攀上去。不远,就见到寺院的山门,他二人坐下,稍微休息一会。女郎气喘吁吁,汗水淋漓,脸上的粉一道道流下来。钟生见了,心中很是不忍,说:“为了我的事,使你受这样的苦。”女郎面色惨然地说:“恐怕这还算不得是苦。”

疲乏稍解,二人就相互搀扶着进了寺庙,给佛施过礼,就向里走。转弯抹角地进了禅房,见一位老僧盘腿坐在那里,双目似闭,一位童子在一边持拂侍候他。方丈室中,打扫得光洁清静;在老僧的座位前,布满了沙砾,密如繁星。女郎不敢选择地方,进来就跪在上面;钟生也跟着跪在后头。老僧开眼一看,又闭上了。女郎参拜说:“好久未来探望父亲了,今女儿已经嫁人,特地携同女婿来拜见您。”老僧待了好久,才睁开眼说:“你这妮子,太带累人了。”就不再说话了。夫妻二人跪了好久,筋疲力尽,沙子与石块快要压到骨头里了,痛得再也支持不下去。又过了一会,老僧说:“把骡子牵来了没有?”女郎说;“没有。”老僧说:“你夫妻马上回去,可快快地把骡子送来。”夫妻二人叩拜而起,狼狈地走出寺庙。

回到家里,遵照父亲的话,将骡子送进寺庙,但不解其意,只是躲在家里,探听外面的风声。过了几天,听传闻说:罪犯捉到了,已经绑到刑场上,砍了脑袋。夫妻得知,相互庆幸。没多久,山中派一童来,把一条砍断的拐杖交给钟生,说:“代替你被砍的,就是这位君子。”便嘱咐钟生,将拐杖埋葬掉,还要礼拜祭奠,以解竹木代死之冤。钟生细看,那被砍断的地方,还有血痕。钟生祈祷后,将拐杖埋葬。夫妻二人不敢在此久居,连夜离开中州回了辽阳。


\subsection{1.8.6   鬼 妻}
\label{\detokenize{p00_u5176_u5b83/_u767d_u8bdd_u804a_u658b_u5fd7_u5f02:id309}}
泰安人聂鹏云和妻子感情很好。妻子得病死后,他整天悲哀,掉了魂似的。

一天晚上,他正在屋里闷坐着,妻子忽然推门进来了。他吃惊地问:“你怎么来了?”妻子笑着说:“我已成了鬼,被你深切的哀悼感动,哀求阴间主管允许,来跟你暂时相会。”聂欢喜非常,拉着妻子上床睡觉,觉得与她生前并无两样。从此日夜往来,转眼一年多,聂也不提再娶妻子。族中弟兄怕他断了后,私下劝他再娶。聂听从了,聘了一个良家女子。但他怕鬼妻不高兴,保着密。不久,到了迎亲的日子,鬼妻知道了这事,责备他说:“我因为郎君讲夫妻情义,才冒着在阴间受责罚的风险来与你相会;谁知你不坚守诺言,情义深厚原来就是这样的吗?”聂说这是族人的意思。鬼妻总是不高兴,没跟他亲热就走了。聂觉得他可怜,可是实现了再娶的打算,也觉宽慰。

新婚之夜,夫妇都睡下后,鬼妻突然来了,从床上用巴掌扇新媳妇并大骂:“你怎么敢占我的床!”新媳妇起身和她撕打。聂吓得光着身子蹲在床上,一个也不敢保护。一会儿,鸡叫天亮,鬼才去了。

新媳妇怀疑聂的妻子并没有死,责备丈夫骗了自己,想上吊自尽。聂对她讲了缘由,新媳妇才信是鬼。天黑鬼就来,新媳妇吓得躲开;鬼也不再与聂同床,只用指甲掐他的肉,再就是对着蜡烛气呼呼地用眼瞪他,也不说什么。聂愁得不行。邻村有人会驱鬼术,削桃木橛子楔在她坟的四角上,才不闹鬼了。


\subsection{1.8.7   黄 将 军}
\label{\detokenize{p00_u5176_u5b83/_u767d_u8bdd_u804a_u658b_u5fd7_u5f02:id310}}
黄靖南,字得功。年轻时候与两个孝廉进京,路上遇上了响马贼。孝廉害怕,跪下拿出钱来买命。黄气坏了,手里又没有武器,就两手抓住骡子,奋力举起来投过去。贼没想到这一招,猝不及防,连人带马一起被砸倒。黄又一顿拳头把贼的胳膊打断,从贼袋子里拿出钱来还给孝廉。孝廉赞赏他的勇力,资助他当了军人。后来他多次立大功,直到被封为将军。


\subsection{1.8.8   三 朝 元 老}
\label{\detokenize{p00_u5176_u5b83/_u767d_u8bdd_u804a_u658b_u5fd7_u5f02:id311}}
本朝有位中堂,做过明朝的宰相。因为曾投降过流寇,名声很不好。他告老还乡后,盖了一座供奉祖宗的祠堂。祠堂竣工那天,有几个人在里面过夜。天亮后,看见堂上悬挂着一块匾,上面写着“三朝元老”,还有一副对联:“一二三四五六七、孝悌忠信礼义廉。”也不知是什么时候挂上的。大家都觉得奇怪,弄不懂是什么意思。有人推测出来说:“上句隐指亡(王)八,下旬隐指无耻。”

洪经略奉命南征,凯旋回到南京后,祭悼阵亡的将士。有个过去的门客进见他,参拜完了,要向他呈献一篇文章。洪经略很久以来厌恶文章,便托词老眼昏花,不接受。那人说:“请你只坐着听就行了,容我读给你听。”接着就从袖筒里掏出一篇文章,高声朗读。原来是明朝崇祯皇帝亲笔写的祭祀洪经略战死辽阳,为国殉难的祭文。读完后,大声哭着走了。


\subsection{1.8.9   医 术}
\label{\detokenize{p00_u5176_u5b83/_u767d_u8bdd_u804a_u658b_u5fd7_u5f02:id312}}
沂水有一个贫民,姓张。一天,他在路上碰见一个善相面的道士,道士看了看他,说:“你定当以术业致富。”姓张的问道士:“我应从何业?”道士又看了看他,说:“医术就行。”姓张的说:“我斗大的字不识两个,怎能从医?”道士笑着说:“你这人太愚,名医何必多识字?只要去干就行。”

姓张的回到家里,贫困无业,便搜集一些偏方,到市上摆了个地摊,摆上鱼牙、蜂房等东西,给人治病,赚一点粮、钱糊口。而人们也没把他当成什么良医来对待。

一次,正碰上青州太守得了咳嗽病,下文到各县为他找医生诊治。这沂水一带本来就是山村僻壤,很少会医术的。县官害怕找不到医生没法向上司交差,就叫乡里负责人自报医生。于是就共同推举姓张的。县官听了,立即召见他。这时,姓张的也正在患痰喘病。自己还治不了自己。接到命令后非常恐惧,便一再推辞,可县官不听,还是命人把他送往青州。这一路上都是大山,姓张的渴得要死,但是山中的水比玉液还贵,挨门要水喝,都没有给的。走到一个地方,碰见一个农妇在漉野菜汁子。漉了半天,野菜虽多,但水却没有多少,盆里的菜汁浓浊得像口涎那样。姓张的实在渴极了,就乞求妇人把剩下的一点菜汁给自己喝。他喝下去以后,顿时觉得不渴了,咳嗽也止住了。他心里暗想,这大概就是治咳嗽病的好方子。

到了青州,各县来的医生都已经给太守诊治了一遍,都没有见好。姓张的来了后,要求单独给自己安排一间房子,装着考虑开药方子。然后派人到民闻索要藜蕹。野菜要来后,他就按那个农妇的方法漉出汁儿来,请太守服下。太守喝下,不长时间,果然药到病除。太守很高兴,赐给姓张的许多钱、物,并赠了一幅金匾挂在了姓张的门上。从此,姓张的名声大震,来找他看病的人,络绎不绝,门庭若市。所治的病,没有不见效的。

一次,有个患伤寒病的人来找姓张的,进门说了病症后,要求给他处方。这时,姓张的正喝醉了酒,马马虎虎地错把治虐痰的药给了病人。待一会酒醒过来才想起出了错,但又不敢再对别人说。可谁知三天后,忽然有人带着礼物来拜谢他。他不知是什么事,一问,才知道那个伤寒病人吃了药后大吐大泻,病就全好了。像这类事情很多。姓张的从此发了大财,家中成了富户。身价也因此抬高了。凡来请他治病的,不出大价钱,不用车马接送,他就不去。

益都县有个姓韩的老头,是个有名的医生。他还没有出名以前,到四乡卖药。一次他晚上没处住宿,就投奔了一家人家。正好这家人的孩子患伤寒病快要死了,知道他是卖药的,就请他给孩子治病。韩老头十分为难,自己想:不给孩子治病今晚无处住,治吧又没有办法治,就在屋里走来走去,两手搓着身子,搓来搓去,身上的污垢搓下来一大片,顺手又把泥垢捻成了一个个丸子。老头灵机一动,心想:给病人吃了这个丸子先应付一下,反正治不好病也没有多大害处;就算治不好,也已赚得个白住白吃。想罢,就给病人吃下这个泥垢丸子。到了半夜,忽然主人来敲门,而且叫得很急。韩老头想:糟了,一定是那孩子死了!他怕主人不饶他,就起来爬墙逃跑。主人见他跑,就在后面追。一直追了几里路,韩老头看看逃不掉了才住了脚。这时主人才告诉他,孩子吃了丸子,出了一身大汗,病全好了。把韩老头又请了回去,好好招待了一番。临走,又赠给他许多钱物。


\subsection{1.8.10   藏 虱}
\label{\detokenize{p00_u5176_u5b83/_u767d_u8bdd_u804a_u658b_u5fd7_u5f02:id313}}
乡里有一个人,偶然坐在一棵老树下面休息。忽从身上摸到一个虱子,便用一片纸把它包了起来,塞到树身上一个洞里就走了。

过了二三年,他又经过那个地方。猛然想起树洞里的虱子,便走到树下,见纸包还完好地放在里边。拿出纸包打开一看,虱子已经干瘪得像麸皮一样了。他又把它放在手中,仔细地观看起来。一会儿,手心感到特别痒,虱子的肚腹却渐渐地鼓了起来。他赶紧把虱子扔掉,就回家去。到家后,手心肿起一个像核桃一样大的疙瘩。肿疼了好几天,那人就死了。


\subsection{1.8.11   梦 狼}
\label{\detokenize{p00_u5176_u5b83/_u767d_u8bdd_u804a_u658b_u5fd7_u5f02:id314}}
白翁是河北人。大儿子白甲,在江南做官,一去三年没有消息。正巧有位姓丁的瓜葛亲戚,来他家拜访。白翁设宴招待他。这位姓丁的平日常到阴间地府中当差。谈话间,白翁问他阴间事,丁对答了些虚幻不着边际的话;自翁听了,也不以为真,只是微微一笑罢了。

别后几天,白翁刚躺下,见到了丁姓亲戚又来了,邀请白翁一块去游历。白翁跟他去了。进了一座城门,又走了一会,丁指着一个大门说:“这里是您外甥的官署。”当时,白翁姐姐的儿子,是山西的县令。白翁惊讶地说:“怎么在这里?”丁说:“如果你不信,就进去看个明白。”白翁进了大门,果然见外甥坐在大堂上,头戴饰有蝉纹的帽子,身穿绣有獬豸图案的官服,门戟与旌旗列于两旁,但没有人给他通报。丁拉他出来,说:“你家公子的衙署,离这里不远,也愿去看看吗?”白翁答应了。走了不多一会儿,来到一座官府门首,丁说:“进去吧。”白翁探头向里一看,有一巨狼挡在路上,他很畏惧,不敢进去。丁说:“进去。”白翁又进了一道门,见大堂之上、大堂之下,坐着的、躺着的,都是狼。再看堂屋前的高台上,白骨堆积如山,更加畏惧。丁以自己的身体掩护着白翁走进去。这时,白翁的公子白甲,正好从里面出来,见父亲与丁某到来,很高兴。把他们请到屋里坐了一会儿,便让侍从准备饭菜。忽然,一只狼叼着一个死人跑进来,白翁吓得浑身哆嗦,说:“这是干什么?” 儿子白甲说:“暂且充当疱厨做几个菜。”白翁急忙制止他。白翁心里惶恐不安,想告辞回去,一群狼挡住去路。正在进退两难的时候,忽然见群狼乱纷纷地嗥叫着四散逃避,有的窜到床底,有的趴伏在桌上,白翁很惊异,不明白这是什么缘故。一会儿,有两个身着黄金铠甲的猛士闯进来,拿出黑色的绳索把白甲捆起来。白甲扑倒在地上,变成一只牙齿锋利的老虎。一个猛士拔出利剑,想砍下老虎的脑袋;另一个猛士说:“别砍,别砍,这是明年四月间的事,不如暂敲掉它的牙齿。”于是,就拿出大铁锤敲打老虎的牙齿,牙齿就零零碎碎地掉在地上。老虎痛得吼叫,声音震动了山岳。白翁大为恐惧,忽然被吓醒,才知道这是一个梦。

白翁心里总觉得这个梦很奇异,马上派人去把丁某请来,丁推辞不来。白翁把自己梦的经过记下来,让次子送到白甲做官的官府,信中劝诫白甲的言语很沉痛悲切。次子到白甲处,见白甲门牙都掉了;惊骇地问他,说是因为喝醉酒,从马上掉下来磕掉了。细细考察一下时间,正是白翁做梦的日子,更加惊骇。他把父亲写给他的信拿出来,读完信,脸色变得苍白。略沉思了一会说:“这是虚幻的梦,是偶然的巧合,有什么值得大惊小怪的。”那时,白甲正在贿赂当权的长官,得到优先推荐的机会,所以并不以这个稀奇的梦为意。弟弟在白甲的官府中住了几天,见蠹役满堂,行贿通关节的人,到深夜还是不断。弟弟流着泪劝谏白甲不要再这样干了,白甲说:“弟弟你自小居住在乡间土墙茅屋中,所以不了解官场的诀窍啊。官吏的提升与降职的大权,是在上司的手里,而不是在老百姓手里。上司喜欢你,你就是好官;你爱护百姓,有什么法子能使上司喜欢呢?”弟弟知道白甲是无法可劝了,就回到家里,把白甲的行为告诉了父亲。白翁听到后,悲痛大哭。没有别的法子可行,只有将其家中的财产捐拿出来周济贫苦的人,天天向神灵祈祷,求老天对逆子的报应,不要牵累到他的妻子儿女。

第二年,有人传说白甲以首荐,推举到吏部做官,前来祝贺的人挤满了门庭;白翁只有长嘘短叹,躺在床上推说有病,不愿接见客人。不久,又传闻白甲在回家的路上遇到了强盗,与仆从都已丧生。白翁就起来,对人说:“鬼神的暴怒,只殃及到他自己,保佑我全家的恩德不能说不厚。”就烧香纸感谢神灵。来安慰白翁的人,都说这是道听途说的消息,但白翁却据信不疑,限定日期为白甲营造坟墓。

可是白甲并没有死。原来四月间,白甲离任调往京都,才离开县境,就遇到强盗。白甲把携带的行装,全部献出来,众强盗说:“我们到这里来,是为全县百姓申冤泄愤的,那里是专为这些东西而来的!”接着就砍下了白甲的头;又问白甲的家人:“有个叫司大成的是哪一个?”司大成是白甲的心腹,专帮他干坏事。家人都指着那个叫司大成的人,强盗们也把他处死。还有四个贪婪的衙役,是为白甲搜刮百姓钱财的爪牙,白甲准备带他们到京城。强盗们也把他们从仆从中找出来杀了,才把白甲的不义之财分了带到身上,骑马急驰而去。白甲的魂魄伏在道旁,见一位官员从这里经过,问道说:“被杀的这个人是谁?”走在前边开路的人说:“是某县的白知县。”官员说: “这是白翁的儿子,不应该叫他这大年纪见到这样凶惨的景象,应当把死者的头再接上。”立即有一个随从把白甲的头安上,并且说:“这种邪恶之人,头不应使它正当,让他用肩托着下巴就行了。”安上头就都走了。过了一些时候,白甲苏醒过来。妻子去收拾他的尸体,见他还有一点气息,就把他用车子载走,慢慢地给他灌点汤水,他也能咽下去。可是住在旅店中,穷得连路费都没有。半年多,白翁才得知儿子的确实消息,就派二儿子去把他接回来。白甲虽说是活了,但两只眼睛只能顾看自己的脊背,人们都不拿他当人看待。白翁姐姐的孩子从政声望很好,这一年被考核进京做御史。这些都和他梦中所见完全相符。

邹平县李匡九进士,做官为政廉洁。当时,常有富裕家的人,被官府中差役罗织罪状关进监狱。一次,一个差役讹诈被抓来的富人说:“县太爷要你交二百两银子,快送来,不然,就要出事。”富人很害怕,答应给一半。差役说:“不行。”富人向他哀求,差役说:“这事不是我不给你出力,怕的是县太爷不同意。到听审时,我当堂给你讲讲情。你可亲眼见到是允许,还是不允许。这样你可了解我的一片苦心了。”

过了一会,李匡九开始审理案件。差役心知李匡九最近戒烟,故意走到近前,低声地问他要不要吸烟。李匡九摇摇头表示不吸。差役便走到富人跟前说:“我才禀报说你出白银一百两,他摇头不答应,这是你亲眼见到的!”富人相信了他的鬼话,答应给二百两银子。差役知道李匡九爱喝茶,就靠近问道:“冲点茶吧?”李匡九点点头。差役又到富人跟前说:“成了。老爷点头同意了,你亲眼看见了吧!”后来案子结了,富人果然无罪释放。这位差役不但收到二百两银子,还得到额外的谢金。

唉,做官者自以为为政清廉,而骂他们贪官的大有人在。这就是自己放纵差役去作恶,如同豺狼,而自己还在稀里糊涂不自觉啊。世上这种糊涂官很多,这件事,可为一心为政廉洁的当官者,作一面镜子啊。


\subsection{1.8.12   夜 明}
\label{\detokenize{p00_u5176_u5b83/_u767d_u8bdd_u804a_u658b_u5fd7_u5f02:id315}}
有个客商,乘船在南海里行驶。夜里三更时,船舱里突然亮了,像天明了一样。客商起来一看,见海里有个庞然大物,半个身子露出水面,如同一座大山;眼睛像两个初升的太阳,光芒四射,把整个大海都照得通明。客商很震惊,询问船上的人,并没有一个人知道这是什么东西。大家一齐趴在船舱中观察它。过了一会儿,那个怪物渐渐沉入水中消失了。于是,天又黑了下来。

后来,客商到了福建,那里的人都说有天夜里突然亮了一阵又黑了,当作怪事相传。客商计算人们传说的日期,正是在船上见到怪物的那个夜晚。


\subsection{1.8.13   夏 雪}
\label{\detokenize{p00_u5176_u5b83/_u767d_u8bdd_u804a_u658b_u5fd7_u5f02:id316}}
丁亥年七月初六,苏州下了大雪。百姓吓得了不得,一齐到大王庙去祈祷。大王忽然附在一人身上说话了:“现如今叫谁老爷,前面都加了‘大’字;难道因为我这个神小,担不得一个‘大’字吗?”众人惊得赶忙喊:“大老爷!”雪立刻住了。

由此看来,神也喜欢有人奉承。怪不得当官的门前车马多呢。


\subsection{1.8.14   化 男}
\label{\detokenize{p00_u5176_u5b83/_u767d_u8bdd_u804a_u658b_u5fd7_u5f02:id317}}
苏州木渎镇,有个乡下人的女儿,一天夜晚坐在院子里乘凉,忽然从天上落下一块陨石,正好打中她的头,倒在地上死过去了。父母年纪很大,又没有儿子,只有这个女儿,悲伤地哭喊着急救。过了不多时,女儿才渐渐苏醒过来,笑着对父母说:“我现在变成男孩了!”察看她,果然不错。他们家里不认为是妖孽,反而窃喜家中突然得了儿子。真是奇怪呀!

这是顺治丁亥年间发生的一件事。


\subsection{1.8.15   禽 侠}
\label{\detokenize{p00_u5176_u5b83/_u767d_u8bdd_u804a_u658b_u5fd7_u5f02:id318}}
天津的一座寺院里,鹳鸟将巢筑在了屋脊之端的鸱尾上。在大殿的顶棚上面,藏着一条盆一样粗的大蛇。每当幼鹳的羽毛翅膀快要长全的时候.大蛇就爬出来,将小鹳一个个地吞吃干净。老鹳悲鸣哀号了好几天才飞走了。这样过了三年,每次人们都料想老鹳必定不会再来了,可到了第三年,老鹳仍然把巢筑在了原来的地方。

到了第四年,幼鹳又快要长成时,老鹳忽然飞走了,三天后才飞回来,进巢呀呀地鸣叫着,跟以前一样哺育着雏鹳。那大蛇又蜿蜒着从天棚上爬了下来,刚接近鹳巢,两只老鹳惊慌地飞起,急切地哀叫着,迅速飞上蓝天。瞬时,只听刮起大风,天昏地暗。大家惊骇异常,只见一只大鸟振动着翅膀,遮天盖日,从天空疾飞而下,如急风骤雨,用爪猛抓大蛇,蛇头立刻掉了下来,连大殿的一角都毁坏了好几尺。大鸟振动着翅膀飞去了。大鹳马上跟在大鸟的后面,好像送别恩人一样。鹳巢也已经翻了下来,两只幼鹳,一只死了,一只还活着。寺院的老僧把活着的小鹳安置到钟楼上。一会儿,老鹳返回,仍然到钟楼上哺育小鹳。等到小鹳的羽毛丰满翅膀长成,老鹳就带着它飞走了。


\subsection{1.8.16   鸿}
\label{\detokenize{p00_u5176_u5b83/_u767d_u8bdd_u804a_u658b_u5fd7_u5f02:id319}}
天津有个专门打鸟的人,一次打到一只鸿雁。他带着雁回家去,那只雄雁也跟着飞到他家,围绕着他房子飞来飞去。悲哀地鸣叫,直到天黑下来时,它才飞去。

第二天,打鸟的人很早就出去,见到那只雄雁早已飞来,飞叫着跟在他后边;接着就飞落在他的脚下。打鸟的人准备把雄雁一块捉住。但见它伸长脖子前俯后仰,吐出半锭黄金。打鸟的人才恍然明白它的用意,说:“你要用金子来赎你的妻子啊?”于是,就放了那只母雁。两只大雁在地上走来走去,好像是悲喜交集,接着就结伴飞走了。打鸟人称称金子,有二两六钱多点。噫!禽鸟有什么理智,竟能这样钟情呢!人生悲莫悲于生离别,动物也是这样吗?


\subsection{1.8.17   象}
\label{\detokenize{p00_u5176_u5b83/_u767d_u8bdd_u804a_u658b_u5fd7_u5f02:id320}}
广东有个猎人,带着弓箭进山打猎。他偶尔躺在地上休息,不觉睡过去了,被一头大象用鼻子卷了去。自己想,这次肯定遭象残害。

不一会儿,大象把他放在一棵树下,点了点头。又叫了一声,一群象便纷纷来到,四面围绕着他,似乎对他有什么企求。刚才卷他的那只大象趴在树下,仰头看看树,又看看猎人,好像让他上树。猎人领会了它的意思,就踏着象背爬到了大树上。虽然爬到了大树顶,却不知大象要他干什么。

不一会儿,一只狮子来了,大象都趴伏在地上。狮子在群象中挑了一只肥的,看样想把它撕着吃掉。群象害怕地颤抖着,没有一只敢逃跑的,只是都抬起头来仰望着树上,好似哀求猎人可怜搭救。猎人会意,就朝着狮子射了一箭,狮子中箭立刻断了气。群象仰头看着天空,意思是向猎人拜舞。猎人爬下树,象又趴在地上,用鼻子牵动猎人的衣服,好像让他骑在自己背上。猎人跨到大象背上,大象驮着他走了。到了一个地方,大象用蹄子挖地,挖出无数象牙。猎人从大象背上下来,把象牙捆绑起来,放在象背上。大象驮着他送出大山,才返了回去。


\subsection{1.8.18   负 尸}
\label{\detokenize{p00_u5176_u5b83/_u767d_u8bdd_u804a_u658b_u5fd7_u5f02:id321}}
有个樵夫到市场上卖完柴,扛着扁担回来,忽觉扁担后面如有重物。回头一看,见一个没有头的人悬挂在上面。樵夫大吃一惊,抽出扁担乱打,死尸便看不见了。樵夫吓得抱头飞奔,跑到一个村庄边,已是黄昏了,见有几个人打着火把照着地上,好像在找什么东西。樵夫上前一打听,原来他们几个人刚才正围坐在一起,忽然从空中掉下一个人头,须发蓬乱,一转眼就又没有了。樵夫也讲了自己所看见的,合起来正好是一个人,但谁也推究不出他是从哪里来的。

后来,有人挎着篮子走,别人忽然看见他篮子里有个人头,惊讶地询问他,他这才大惊失色,把人头倒在地上,然而一转眼又不见了。


\subsection{1.8.19   紫 花 和 尚}
\label{\detokenize{p00_u5176_u5b83/_u767d_u8bdd_u804a_u658b_u5fd7_u5f02:id322}}
诸城县的丁秀才,是丁野鹤先生的孙子。他是少年名士,患病多年而死。但过了一夜,他竟然又活了,说:“我悟道了。”当时,诸城县有一位僧人对于佛理奥妙很有研究。丁秀才叫家人把这位僧人请来,让他在床前讲解《楞严经》。但每听僧人讲解一节,他都说不是这样。于是说:“假若我的病能痊愈,验证佛理有何难?但是只有附近的某生,能治愈我的病,应该诚心诚意地去把他请来。”原来,丁秀才讲的这位书生,精于医术,却从不以行医为业,请了他三次,他才来。书生根据医理出方下药,丁秀才吃了几副,病就痊愈了。

这位书生给丁秀才看过病回到家里,一位女子从外边进来,对他说:“我是董尚书家中的丫鬟,紫花和尚与我有冤仇,现在他得到应有的报应,你又想把他治活?假若你再去给他治病,大祸将临到你的头上。”话说完,女子就隐没不见了。书生很恐惧,丁秀才家人再来请他,他坚决推辞。丁秀才的病复发后,丁家人执意要请他去看病,书生就把不去的原因讲了。丁秀才慨叹说;“罪业是前生所造,今天死,也是我所应得的。”说完就死了。后来,寻问诸城人,果真有一位紫花和尚,是一位很有道业的僧人。青州董尚书的夫人,曾经把他供养在家中,也没有人知道其冤仇所结的缘由。


\subsection{1.8.20   周 克 昌}
\label{\detokenize{p00_u5176_u5b83/_u767d_u8bdd_u804a_u658b_u5fd7_u5f02:id323}}
淮上有一位贡生,叫周天仪,年已五旬,膝下只有一子,名叫克昌,周天仪对他非常溺爱。克昌长到十三四岁,出落得潇洒俊雅,但他天性不喜读书,总是逃学,跟孩子们一块戏闹玩耍,经常整天不回家。周天仪也听之任之。一天,天黑了克昌还没回家,全家人才开始寻找,竟不见踪影。周天仪夫妻号啕大哭,几乎痛不欲生。

过了一年多,克昌忽然自己回来了,说:“我被道士迷了去,幸没被害。正巧他外出,我才逃了回来。”周天仪喜出望外,也不加追问。此后再让克昌读书,只觉他比原来聪明了好几倍。过了一年,克昌文思大进,既而又参加了府学考试,于是成了名士。世族大家都争着向他许亲,克昌都不乐意。赵进士有个女儿,很有姿色,周天仪硬强为儿子娶了过来。媳妇进门,夫妻二人调笑甚欢,但克昌总是独宿,似乎从未沾染过妻子。第二年,克昌考中举人,周天仪越加感到欣慰。但自己渐渐衰老,天天盼望抱孙子,所以曾暗示克昌这件大事。克昌却漠然置之,似乎不懂。母亲再不能容忍,一天到晚絮絮叨叨,克昌变了脸色,出门而去。说:“我早就想走了,之所以没立即走,是因为难忘父母养育之恩。实在不能探讨闺房中的事情,以安慰父母的心愿。请让我仍旧离去,那个会顺从父母意愿的人马上要来了!”家中的佣妇追上去拉他,克昌已跌倒在地,衣冠像蝉蜕皮一样堆在地上,人却不见了。全家人都非常惊骇,怀疑克昌已死,所见到的是他的鬼魂。但也只有悲叹而已。

第二天,克昌忽然骑着马带着仆人回来了,全家人惊慌失措。走近他略微问了几句,也是说被恶人劫了去卖给一个富商家。这个富商没有儿子,拿他当亲儿子一般看待,买了他后,富商忽生一子。他想家,富商便把他送了回来。问他以前学过的东西,则跟过去一样愚钝。于是才知道这个是真克昌,而那个考中举人的是假的,是鬼变的。但周家暗喜这件事没有泄露出去,便让克昌承袭了举人的名份。行房事时,妻子倒是非常亲热熟练,但克昌很羞惭。面有愧色,好像新婚一样。刚过一年,克昌便生了儿子。


\subsection{1.8.21   嫦 娥}
\label{\detokenize{p00_u5176_u5b83/_u767d_u8bdd_u804a_u658b_u5fd7_u5f02:id324}}
山西太原人宗子美,随父亲游学四方,后来到扬州,就住了下来。

子美的父亲,与红桥下的林婆子平素就有交往。一天,宗子美与父亲路过红桥,正巧遇到林婆子。林婆子再三请他们父子到家中作客,喝茶叙谈。到家见有位女子站在一旁,生得很漂亮,宗翁极力赞美。林婆子说:“你家郎君温柔和顺,真像个大姑娘,是有福之相。假若你们不嫌弃,便把我的女儿许配给郎君,怎么样?”宗翁笑着,督促儿子快向林婆施礼,说道:“你这一句话,可是值千金啊!”原先,林婆子独居,这女子忽然间自己来到她家中,述说了孤苦之情。林婆子问她名字,说叫嫦娥。林婆子很爱怜她,就把她留下,其实,她是把嫦娥当奇货。当时宗子美刚十四岁,一见嫦娥,心中暗喜,自念父亲必定找媒人订婚;可是回来后,他父亲好像把这事忘了。宗子美心里火烧火燎一般,喑暗地把这事告诉了母亲。父亲得知后说:“那是与贫婆开玩笑的。她不知要将这女儿卖多少黄金呢,这事怎能说得那么容易。”

过丁一年,宗子美的父母都去世。但宗子美仍不能忘情嫦娥,服孝快要满期,就托人向林婆子求婚。林婆子起初不应允,宗子美气忿地说:“我生平从来不轻易向别人折腰,为什么你这老婆子把我的真心诚意看得一钱不值!假若你背弃以前的婚约,得将我折腰的诚意还我。”林婆子就说:“以前那是与你父亲开玩笑许下的事,也许是有的。但当时没有正式说定,过后也都忘却了。今天你既然这详说,我还想留着女儿嫁给王子不成?我天天把她梳装打扮得这样美妙,实指望能换得千金;今天我只要你半价,可行吧?”宗子美自己忖度难以办成,也就把这事放到了一边。

当时,正巧有一位寡妇赁居在西邻。她有个女儿刚到待嫁的年龄,小名叫颠当。宗子美偶尔遇见过她,典雅的丽质,不在嫦娥之下。子美很思慕她,每每以赠送礼物为由接近她。时间长了,他们间也较熟悉了,见面时往往以目送情。二人想说话,也没有机会。一天晚上,颠当越过垣墙来借火。宗子美欢喜地拉住她,于是二人就完成燕好之事;并约定迎娶颠当,她推辞说哥哥在外经商还未回来。自此以后,他们一有机会就相互往来,但不露行迹。

一天,宗子美偶然经过红桥,见嫦娥正巧站在门里,宗子美很快地走过去。嫦娥望见,向他招手,宗子美站住脚;嫦娥又招呼他,他就进了嫦娥的家门。嫦娥以背约责难宗子美,子美向她述说了其中的缘故。嫦娥进屋,取来黄金一铤交给宗子美。宗子美不接受,推辞说:“我自己还以为永远不会再与你有缘分了,就与别人订了婚约。现在我若接受你的黄金,娶你为妻,就辜负了别人;若接受你的黄金,却不娶你,就辜负了你的好心。所以,这黄金我是不敢接受的。”嫦娥待了好久说:“你的婚约之事,我也知道,这件事是必定不能成的。即使成了,我也不怨君负心。你赶快离开这里,妈妈要回来了。”宗子美仓促间,也不知该怎么办好,接了黄金就回到家里。过了一夜,把这事告诉了颠当。颠当认为嫦娥说的话对,但劝宗子美专心钟爱嫦娥。宗子美沉思不语;颠当说她愿意在嫦娥之下,宗子美这才高兴起来。马上派媒人,携带着黄金交给林婆,婆子无话可说,就把嫦娥交给了宗子美。嫦娥进门后,宗向嫦娥叙述了颠当的话。嫦娥微笑,怂恿纳颠当为妾。宗子美很欢喜,急于一见颠当,而颠当却很久不来了,嫦娥也知道颠当是为了自己,因此就暂且回家,特意给他们个机会。嘱咐宗子美,与颠当相见时,把她佩的香囊窃来。不久,颠当果然来了,宗子美与她商量迎娶的事,颠当说不着急。颠当解开衣襟和他调笑时,胁下露出一个紫色的荷包,宗子美趁空摘取,颠当突然变了脸色说:“你与别人一心,与我是二心,是负心!请从此以后,断绝来往。”宗子美百般解释、挽留,颠当不听,走了。一天,宗子美从她家门前过,那房子已被另一位吴姓的赁去;说颠当母女已搬走很久时间了,连点影迹都见不到,没有办法去打听。

宗子美自娶了嫦娥,家中骤然富裕起来,楼阁长廊,连接街巷。嫦娥喜于嬉戏玩耍。一次,他们见到一幅美人的画卷,宗子美对嫦娥说:“我常说,美丽如同你的人,天下真是无双。只恨不曾见过传说中的杨贵妃、赵飞燕啊。”嫦娥笑着对宗子美说:“你想见识杨贵妃、赵飞燕,这也不难。”于是,拿起画卷仔细看了一遍,便急忙走进屋里,自己对着镜子修饰打扮一番,学着赵飞燕翩翩起舞的轻盈风姿;又学杨贵妃慵懒娇媚的醉态。长短肥瘦,随着舞姿的变化而变化。表现出的那种娇柔风情,与画卷上的样子一模一样。嫦娥刚扮妆起舞时,有一个婢女从外边走进来,见了嫦娥几乎都认不出来了。惊讶地问她的同伴姐妹;再仔细端详,才恍然大悟而笑。宗子美说:“我得到你这位美丽的娇妻,历史上的美人,也就都在我的屋子里了。”

一天夜里,刚刚睡下,忽然数人把门撬开进来,火光将墙壁照得通亮。嫦娥急忙起来,惊呼:“盗贼进来了!”宗子美刚刚醒来,正想大呼,一个人用刀按在他的脖子上,吓得他连大气都不敢喘。另一个人将嫦娥背到身上就跑了,这群强盗哄然而散。这时,宗子美才大声叫喊,家中的仆役都集拢来,看看房子中的珍贵的珠宝细软,没丢失一点儿。宗子美很悲痛,惊吓得连个主意也没有了。他们告到官府,官府下通牒追捕,但没有半点消息。渐渐地,三四年的时间过去了,宗心情郁闷无聊,借着到省城赴试的机会,顺便到京都里散散心。居住了半年,算卦问卜,各种方法都施尽了,也没有打听到嫦娥的下落。

一次,偶然路过姚家巷,遇到一位女子,蓬头垢面,衣衫褴褛,慌惧如同讨饭的乞丐。宗子美停下脚步,细细看她,原来是颠当!惊讶地说:“颠当,你怎么憔悴成这样子?”颠当回答说: “自与你分别后,就南迁了,老母亲也去世了。我被恶人抢去卖到旗下,遭到挞辱与冻馁,无法忍受。”宗子美听了,凄然流下眼泪,问道:“在旗下,能赎出来吗?”颠当说:“很难。要花费好多钱,是没有办法作到的。”宗子美说:“实话告诉你吧,几年来,我家中颇富,可惜我客居于此,囊中钱不多。如果把行李与马卖了,能够赎你的话,我也不敢推辞。假若所需的钱数过大,那我就回家去操办。”颠当与他相约,明天在西城的丛柳下相会;并嘱咐一定要他一个人去,不要让别人跟从。宗子美答应说:“就这样。”

第二天,宗子美按照约定,早早就去了。到了西城,颠当早就等在那里了,身着鲜艳明丽的旗袍,与昨天所见,大不一样。宗子美惊奇地问她,颠当笑着说:“昨天我是试一试你的心,幸亏故人之情未变。请到我的寒舍叙叙,我一定好好地报答你。”宗子美跟着颠当向北走了一段路,就到了她的家。颠当拿出菜肴、美酒款待他,二人欢笑异常。宗子美约她一块回家去。颠当说:“我这里俗事累赘太多,不能跟你走。可是,嫦娥的消息,我颇知道点。”宗子美迫不及待地问嫦娥在哪里。颠当说:“她的行踪飘忽不定,具体地方,我也说不准。西山有位老尼,瞎了一只眼,去问她,自会告诉你。”当晚宗子美就宿在颠当的家里。天明,颠当给他指明路。宗子美到了那里,见有一座古寺,周围的墙垣都倒塌了。在一丛竹子里有间茅草屋,老尼正在补缝衣服。见到来人,待答不理的。宗子美给她行礼,老尼这才抬起头来问他要作什么。宗子美将自己的姓名报上,接着告诉了自己所要求的事。老尼说:“我是个八十岁的瞎子,与世隔绝,那里能知道美人的消息?”宗子美苦苦地哀求她,老尼才说: “我实在不知道。有二三家亲戚,明天晚上来访,或者小女子们能知道这事,也说不定。你明天晚上来吧!”宗子美就出来了。

第二天再到那里,老尼不在家,破门紧紧地锁着。在这里等了很久,夜已经深了,明月高高地挂在东方的天空,宗子美走来走去,没有办法。远远地望见二三位女郎从外边走进来,其中的一个就是嫦娥。宗子美太高兴了,猛然间起来,急忙拉住嫦娥的衣袖。嫦娥说:“莽撞的郎君,吓死我了!可恨那多嘴的颠当,又让你用儿女情来缠磨我。”宗子美拉着嫦娥坐下,握着她的手,叙说别离后的艰辛,不觉悲伤地流下泪来。嫦娥说:“实话告诉你:我是天上嫦娥被贬谪下界,浮沉于人世间。现期限已满,便假托寇劫回到天上。之所以这样做,是为了断绝君的希望。那位老尼,是给王母娘看门的。我最初被谴时,承蒙她的关照收留下来,所以,有时间常来看望她。如果你能放我走,我就想法将颠当给你娶过来。”宗子美不放她,低着头流泪。嫦娥回头张望说:“姊妹们来了。”宗子美四处张望,嫦娥不见了。宗子美失声大哭,不想再活在人世间,就解带自己上吊。恍恍惚惚地觉得自己的魂已经离开躯体,迷迷糊糊地不知飘荡到哪里。忽然,见到嫦娥,捉住自己双脚,离地而起,又进入寺中,在树上取下尸体推挤着,呼唤着:“痴郎!痴郎!嫦娥在这里!”宗子美忽若梦醒。稍定,嫦娥气忿地说:“颠当贱婢!害了我又杀了郎君,我不能轻饶她。”二人下山就赁了一辆车子,回到寓所。宗子美就命家人准备行装,自己返身到西城去答谢颠当。但到了那里,原先的房舍完全变样了,宗子美惊愕慨叹而归。暗想,幸亏嫦娥未发现。进门,嫦娥迎笑说:“你见到颠当了吗?”宗子美惊愕得说不上话来。嫦娥说:“你想背着我嫦娥,怎么能见到颠当呢?请老实地坐在那里,她一会儿就会自来的。”不多会儿,颠当果然来了,仓惶地跪在床下。嫦娥用指头弹着她的头说:“小鬼头,害人不浅!”颠当连连扣头,但求免死。嫦娥说:“把别人推到火坑里,而自己想逍遥天外?广寒宫中十一姑,不几天就要下嫁,需要绣枕头百幅,鞋百双,可以跟我去,共同完成。”颠当恭恭敬敬地说:“只要分给我,定按时送来。”嫦娥不许,对宗子美说:“你若同意的话,就放她走。”颠当瞪眼看着宗子美,但他只笑不说话。颠当生气地看着他。颠当乞求回家告诉一声,嫦娥答应了,颠当于是就回家去了。宗子美向嫦娥问起颠当的生平、身世,才知她是西山的一只狐狸。宗子美买好车子,等待着。第二天,颠当果然回来了,他们就一块返回家乡。

嫦娥这次回来,举止很持重,平日从不轻率地与家人说笑。宗子美强迫嫦娥扮装游戏,她从不肯,只是偷偷怂恿颠当去做。颠当很聪慧,善于谄媚男子。嫦娥喜欢单独过夜,宗子美每想与她过夜,她常以身体不舒适推辞。一天夜里,已是三更天了,还听到颠当房中,吃吃笑声不断。嫦娥让婢子偷偷去看个究竟。婢子回来,什么话也没有说,只是请夫人自己去看看。嫦娥伏在窗上,向里看,只见颠当凝妆扮作自己的形状,宗子美抱着她,呼叫嫦娥。嫦娥轻蔑地一笑,回到屋里。不大会儿,颠当心头暴痛,急忙披上衣服,拉着宗子美到嫦娥房中,进门便跪下。嫦娥说:“我又不是医生与巫婆,哪里能治病?你自己想效仿西施捧心学娇。”颠当只是在地下叩头,声言知罪。嫦娥说了声“好了”。颠当便从地下起来,失笑而去。颠当暗中对宗子美说:“我能使娘子学观世音菩萨。”宗子美不相信,于是就与颠当开玩笑打赌。嫦娥每次盘腿打坐,总是双目若闭。颠当悄悄地用玉瓶插上柳枝。放到茶几上,自己就垂发合掌,侍立于侧,樱桃般的嘴唇半开,瓠子般的牙齿微露,双目一眨也不眨。宗子美在一旁笑她。嫦娥睁开眼问她,颠当说:“我学的是龙女伺候观世音。”嫦娥笑着骂她,罚她学着童子样,给自己施礼。颠当将发束起来,就四面向上参拜,伏在地上,变化各种形态,左右转辗,那舞动的姿式,脚都可以磨着耳朵。嫦娥笑了,用脚去踢她。颠当抬起头,用口咬着嫦娥的脚尖,轻轻地用牙齿衔着。嫦娥正开心嬉笑,忽觉得一丝媚欲之情,从脚趾而上,直到心头,春情已动难以忍受,自已也控制不住。嫦娥急忙收神镇静下来,呵斥说;“狐奴才!你想死,迷惑人也不选择一下。”颠当害怕,急忙松开口,伏在地上。嫦娥又严厉责备她,但众人不解其故。嫦娥对宗子美说:“颠当这婢子,狐性不改,刚才差点儿被她愚弄。若不是我道业根深,很容易堕落进她的圈套。”自这以后,每见颠当,则自提防之。颠当羞惭畏惧,告诉宗子美说:“我对于娘子的一手一足,无不亲爱;但正因爱之深,不觉媚惑她就过分。如果说我有别的心,不但不敢有,我心里也不忍。”宗子美把这实情告诉嫦娥,嫦娥改变了对她的态度,如同当初一样。然而,因为嬉闹没有个节制,屡次劝戒宗子美,宗子美听不进去;因而,大小婢妇,都效仿他们,争相狎戏。

一天,两个婢女扶着一个婢女,扮作杨贵妃醉酒。两个婢女使了个眼色,趁这位婢女醉态朦胧之时,两人把手一放,婢女突然跌到台阶下,被摔的声音如同推倒一堵墙。众人大声惊呼,近前一摸,装扮贵妃的婢女,像贵妃一样,薨于马嵬坡,一命归西了。众人惧怕,赶快把这事告诉了主人。嫦娥惊骇地说:“闯祸了,我说的话怎么样!”去验看,已不可救了。派人去告诉婢女的父亲。婢女的父亲某甲,平素为人就无德行,哭闹着跑来,把女儿的尸体背到厅房里,又喊又骂。宗子美吓得关上门,不知怎么办才好。嫦娥自己出面责备他,说:“主人即使虐待婢子致死,法律上也没有偿命这一条。况且你孩子是偶然暴死,怎么知道她就不会再活了。”某甲叫嚷着说:“四肢都冰凉了,哪有再生之理!”嫦娥说:“不要乱吵,纵然是活不了,还有官府在。”于是,进了大厅,用手抚摸尸体,婢女马上苏醒过来。再用手抚之,随手而起。嫦娥返转来,愤怒地说:“婢子幸亏没死,贼奴才怎么这样无理!可用绳子绑送官府。”甲无话可说,长跪哀求饶恕。嫦娥说:“你既然知罪,暂且免于追究、处分。但无赖小人,反复无常,把你女儿留在这里,终是惹祸之根,应该把她领回去。所购之原价若干,要赶快措办,如数送来。”派人押送回去,让他请二三个村里的老人,在证券后划押作保。完了之后,才把婢女叫来,让甲自己问,说:“没有伤着吧?”婢女回答说:“没有。”就把婢女交给甲,让他领走。事情处理完后,嫦娥把婢女们喊来,数落她们的罪责,一个个被扑打。又把颠当唤来,严禁她再干这类的事。对宗子美说:“方今知道,主子一笑一颦,也不敢轻率。戏谑自我开始,竟使弊端屡禁不止。世间凡是哀伤的事属阴,欢乐者属阳;乐过了头就要走向反面,这是万物循环的规律。婢子的祸殃,是鬼神给我们的预告。再执迷不悟,就要闯大祸了。”宗子美听从了嫦娥的话。颠当哭泣着要求嫦娥解脱她。嫦娥用手指着颠当的耳朵,过了一会儿松开手。颠当在迷茫中恍惚了一会儿,忽然间,如大梦初醒,伏地便拜,高兴得手舞足蹈。自这后,闺阁中清净严肃,没人敢再随便喧哗。那个婢子,回到家中,没有病,自己就死了。甲因为赎婢子的钱赔偿不了,就请村中老者代为哀求怜悯,嫦娥答应了。又因婢女扶持主人的感情,施舍了一口棺木。

宗子美常以无子为忧。嫦娥肚子中,忽然听到儿啼的声音,于是就用刀割破左胁,取出婴儿,是个男孩;没有多久,嫦娥又怀孕了,又开刀破右胁取出婴儿,是个女的。男孩很像他父亲;女孩很像她母亲。长大成人后,都与大户人家成了婚。


\subsection{1.8.22   鞠 药 如}
\label{\detokenize{p00_u5176_u5b83/_u767d_u8bdd_u804a_u658b_u5fd7_u5f02:id325}}
鞠药如,是青州人。妻子死了之后,便离家出走了。几年以后,他身穿道服,背着蒲团来到家乡。在家住了一宿想走,亲戚族人硬留下了他的衣杖。鞠药如推托随便走走,到了村外,屋里的衣杖服具,都冉冉地飞了出来,随他一块而去。


\subsection{1.8.23   褚 生}
\label{\detokenize{p00_u5176_u5b83/_u767d_u8bdd_u804a_u658b_u5fd7_u5f02:id326}}
顺天府的陈孝廉,十六七岁的时候,曾经跟随一位私塾先生在僧寺中读书。同学很多,其中有一个姓褚的同学,自称是山东人,刻苦攻读钻研,一刻也不休息。而且这个同学寄宿在学校里,从未见他回过一次家。陈生与他最要好,因而就询问他为什么不回家。褚生回答说:“我家里很穷,筹措学费不容易。我即使做不到珍惜每一寸光阴,如果每天加上半个夜晚,那么我的两天就可以抵得上别人的三天。”陈生听了他的话很受感动,就想搬来床铺和他一起住。褚生劝阻他说:“暂且不要这样做!我看先生的学识已经不能做我们的老师了。阜城门有一位吕先生,年纪虽然很老了,却可以做我们的老师,请你和我一同到他那里去求学吧。”原来京城中教私塾的大多按月计算收取学费。月底学费用完了,学生们可以按自己的意愿继续留下或者离开。于是陈、褚二人就一同去拜见吕老先生。吕老先生是浙江有名的读书人,因落魄穷困而不能回乡,只好靠教儿童启蒙糊口,这实在不是他的志向。所以,他得到陈、褚两个学生后非常高兴。而褚生又很聪明,读书一过目就懂了,吕先生特别器重他。陈、褚两人感情十分亲密,白天同桌读书,晚上共睡一床。到了月底,褚生忽然请假回去了,十几天没再回来,大家都感到很奇怪。

一天,陈生因为有事到了天宁寺,在廊下遇见了褚生,他正在把苘麻劈城小条,蘸上硫黄,制成引火的用具。褚生见到陈生后,现出羞惭不安的神情。陈生问道:“你为什么突然停止读书了?”褚生握住陈生的手,把他请到一边,很难过地说:“因为家穷没有钱给先生作学费,所以必须做半个月的买卖,才能读一个月的书。”陈生感慨了很长时间,说:“你尽管回去读书,我自当尽力帮助你。”就叫仆人收拾起褚生的工具,两人一同回到学校。褚生嘱咐陈生不要泄漏这件事情,只假托个理由去告诉先生。陈生的父亲本是个开店铺的商人,靠做买卖致富,陈生常偷父亲的钱,替褚生交纳学费。陈父因为丢失了钱而责问陈生,陈生就把实情告诉了父亲。陈父认为陈生是个书呆子,就叫他停学了。褚生感到十分惭愧,拜别老师准备离去。吕老先生知道了其中的缘故,就责备他说:“你既然这样贫困,为什么不早告诉我?”于是把钱全部退还给陈父,留下褚生像往常一样读书,跟他一起吃饭,对待他就像自已的儿子一样。陈生虽然不入馆读书了,却常常邀请褚生到酒店共饮。褚生本来为了避嫌不肯去,可陈生邀请他也越发坚决,常常流下泪来。褚生不忍心拒绝他,于是与陈生就来往不断了。

过了两年,陈生的父亲死了,陈生又要求跟吕老先生读书,吕老先生被他的诚意感动了。就收下了他。但陈生由于停学已经很久,和褚生相比差距很大了。过了半年,吕老先生的长子从浙江一路讨着饭到京城来寻找父亲。学生们凑集了一些钱给吕老先生作回乡的路费,可是褚生只有流着眼泪依恋不舍而已。吕老先生临别时,嘱咐陈生要把褚生当作自己的老师对待。陈生答应了,请褚生住到自己家里当自已的老师。过了不久,陈生考中了秀才,又以“遗才”科的身份参加乡试。陈生担心自己不能把文章写完,褚生主动请求代替他去参加考试。到了乡试的日期,褚生带了一个人来,说是自己的表兄刘天若,嘱咐陈生暂时跟着他去。陈生刚刚出门,褚生忽然从后边拉他,陈生身体几乎跌倒,刘天若急忙挽住他一同走了。他们游览眺望了一阵子以后,就一同在刘天若家中住下了。刘家没有妇女,他就让客人住在了内院。住了几天,不知不觉到了中秋节了。刘天若说:“今天李皇亲的花园中游人很多,我们也应当去舒散一下心头的烦闷,顺便送你回家。”于是,刘天若就叫马僮挑着茶炊、酒具前去。只见水中楼台,梅花形的亭阁里,人声喧哗嘈杂,不能进入。走过了一道水闸,便见在老柳树下横着一条画舫,他们就互相扶着登上船去。两人酒过数巡,很感寂寞。刘天若伸头对书僮说:“梅花馆最近有新来的妓女,不知在家不在?”书僮去了一会儿,就和一位女子一同回来了。原来是妓院的李遏云,她是京城的名妓,诗写得很好,又善于唱歌。陈生曾经和朋友一起在她家喝过酒,所以认识她。两人相见,略为问候了几句。李姬脸上带着悲哀忧愁的神色。刘天若叫她唱歌,她就唱了一支《蒿里曲》。陈生很不高兴地说:“我们主、客两人即使不能使你满意,也不致于对着活人唱挽歌!”李姬站起来表示了歉意,勉强露出了笑脸,就唱了一支词曲浓艳的歌曲。陈生高兴了,握住李姬的手腕说:“你过去写的《浣溪纱》词我读了好几遍,现在都忘了。”李姬就吟道:“泪眼盈盈对镜台,开帘忽见小姑来,低头转侧看弓鞋。强解绿蛾开笑面,频将红袖拭香腮,小心犹恐被人猜。”陈生反复吟咏了好几遍。不一会儿,船靠岸停下。他们上岸后走过长廊时,陈生见长廊壁上题诗很多,就拿起笔来把那首《浣溪纱》词写在壁上。这时天色已近傍晚了,刘天若说:“考场中的人快出来了。”就送陈生回家。陈生进了家门,刘天若就告别回去了。陈生见室内昏暗无人,稍一迟疑,褚生已经走进门来,他细看了看,却不是褚生。正感到疑惑的时候,那人忽然走近自己身边跌倒了。这时家里仆人们说:“公子疲劳极了!”一齐来拽他扶他;陈生又觉得跌倒的不是别人,而是自己。他站起来后,看见褚生站在旁边,恍恍惚惚好像是在梦境中。陈生屏退了仆人追问这件事,褚生说:“告诉你不要吃惊:我其实是个鬼,很早就应该投生转世了。所以在这里一拖再拖,是因为不能忘记你对我的深厚情谊,因此附在你身上,以便代替你去参加乡试。现在三场考试已经结束,我的愿望也了结了。”陈生又请求他代替自己参加会试。褚生说:“你的父辈福薄,悭吝之人的骨格,承受不了诰赠的荣耀。”陈生问:“你将要到哪里去?”褚生说:“吕老先生与我有父子的情分,我经常挂在心上而不能忘。我的表兄在阴间衙门里掌管簿籍,我求他向地府主管者说情,或者能有希望作他的子嗣。”于是告别而去。陈生觉得这件事很奇异,天明后就去拜访李姬,想问问泛舟游湖的事,但是李姬已经死了好几天了。他又到李皇亲花园中去,见廊壁上自己题的那首《浣溪纱》词仍在,只是墨色淡而模糊,好像就要磨灭了一样。这才明白题写诗的是自己的魂,而作词的李姬是个鬼。到了晚上,褚生满面喜色地来了,说:“所求之事幸而成功了,现在要向你告别了。”于是伸出两只手来,叫陈生在手心里各写上“褚”字作为标志。陈生要设酒宴为他饯行,褚生摇头说:“不必了。你如果不忘我们往日的交情,乡试发榜以后,不要怕路途遥远艰险,到浙江来看望我。”陈生流着泪送他,看见有一个人在门外等候着,褚生还在依依不舍,这个人用手按着褚生的头顶,褚生就随着他的手变扁了。这个人用手把褚生捧起来放入一个口袋中,背着走了。

过了几天,陈生果然考中了举人。于是治备了行装前往浙江。吕老先生的妻子几十年不生育,五十多岁了,忽然生了一个男孩,两只手紧紧握住不能伸开。陈生到了吕家,要求见见小孩,并说手掌中一定有个“褚”字。吕老先生不信。小孩见了陈生,十指自动伸开了,一看他的手掌心,果然各写着一个“褚”字。大家很惊奇地问起原因,陈生就把这事原原本本地告诉了他们。大家又喜欢又惊奇。陈生丰厚地赠送给吕老先生一笔钱,才告辞回家。后来,吕老先生以岁贡的资格,到京城参加廷试,住在陈生家里。这时吕老先生的小儿子才十三岁,已经考中秀才了。


\subsection{1.8.24   盗 户}
\label{\detokenize{p00_u5176_u5b83/_u767d_u8bdd_u804a_u658b_u5fd7_u5f02:id327}}
清朝顺治年间,山东滕县、峄县一带,十个百姓中就有七个是盗寇,官府也不敢抓捕他们。后来,这些盗寇受了招抚,归顺了朝廷,县官把他们另立户册,称之为“盗户”。凡“盗户”与一般老百姓发生争执,官府总千方百计地袒护他们,为的是怕他们重新造反。后来打官司的人便往往冒称是“盗户”,而另一方却极力揭发对方是假的。每每打官司时,诉讼双方先不去争论是非曲直,而是苦苦争辩谁是真盗假盗,还得烦劳官府去查阅户籍。正巧,官署里经常有狐狸作祟。县官的女儿被狐狸迷住了,请了法师,用符咒捉住了狐狸,放进了一个瓶子里,准备用火烧死它。这时,狐狸在瓶子里大声喊叫:“我是盗户!”听到的人无不暗笑。


\subsection{1.8.25   某 乙}
\label{\detokenize{p00_u5176_u5b83/_u767d_u8bdd_u804a_u658b_u5fd7_u5f02:id328}}
城西的某乙,过去是个小偷,他的妻子为此感到很恐惧,多次规劝他,某乙于是幡然悔悟。

过了两年,某乙贫困得不能忍受,就想再去当一次小偷而后就不干了。于是假托去做买卖,到一个算卦人那里去算算到什么地方去吉利。算卦人算了算,说道:“东南方向吉利,利于小人,不利君子。”卦家隐隐约约与他心中的想法相吻合,他暗暗高兴。于是他就向南走,到了苏州、松江一带,每天在村庄、城镇中游逛,这样过了好几个月。

一天,他偶然进入一座寺院中,见墙角上堆着两三块石子,心里知道这里边有些古怪,他也拣了一块石子放上去,然后就一直走到佛龛后边躺下了。天黑了以后,寺中有些人聚在一起说话,好像有十几个人。忽然其中一人数了数石子,很惊讶地发现多了一块,因而一起到佛龛后边搜寻,发现了某乙,就问他:“放石子的是你吗?”某乙承认了。又盘问他的住址、姓名,某乙用假话回答他们。于是他们给了某乙一件武器,领着他一同出去。到了一座高大的宅院外,有人拿出了软梯,大家争着越墙而入。因为某乙是从远处来的,对路径不熟悉,就叫他潜伏在墙外,负责传递财物和看守口袋。过了一会儿,墙上扔下一个包裹;又过了一会儿,用绳子缒下一只箱子。某乙举手接住箱子知道装着东西,就把箱子打破,用手摸索着拿,凡是沉重的东西,全部放进一个袋子里,背起袋子急忙逃走了,终于寻路回到了家中。

从此某乙建楼阁,买良田,并且用银子为儿子捐了个功名。县令给他家大门上挂了匾,称他为“善士”。后来这件大案被破获了,群盗都被抓获,只有某乙没有姓名、籍贯,没有办法追查,才免于被捕。事情过去了很久之后,某乙喝醉了酒自己说出了这件事。

曹州府有个大强盗,抢到一大笔财物回到家后,毫无顾忌地安然睡去。有两三名小盗,越过院墙进入他家中,把他捉住了,向他要钱。大盗不给,他们就鞭打、烧烙。把大盗的所有财物都逼索到手,才离去。大盗向人说:“我不知道炮烙的痛苦如此厉害!”于是对盗贼深感痛恨,就投到衙门里充当了马捕,把本地的盗贼差不多都捕捉尽了。有一次捕到了以前抢他财物的几个盗贼,就用他们对自己施用过的刑罚惩治了他们。


\subsection{1.8.26   霍 女}
\label{\detokenize{p00_u5176_u5b83/_u767d_u8bdd_u804a_u658b_u5fd7_u5f02:id329}}
朱大兴,河南彰德府人。家中很富裕,但为人吝啬,如果不是儿女婚嫁之事,家中从没有宾客,厨房中也从无肉类。然而,他却喜欢女色,只要是他看上的女人,花钱多少,从来不吝惜。每天晚上,爬墙串村,去找淫荡女人睡觉。

一天夜里,朱大兴遇到一少妇独自行路,心知是逃亡的妇女,便强逼着她来到家中。点灯一看,漂亮极了。妇女自己说:“姓霍。”再细致地问,妇女很不高兴,说:“既然把我带到家中,又何必盘根寻声地问呢?如果怕受连累,不如早让我走好了。”朱不敢再问,便留下她一块睡了。但是霍女不安于粗茶淡饭,又讨厌吃肉汤之类的东西,最喜欢吃的是燕窝、鸡心、鱼肚白作的羹汤,只有这样才能吃饱肚子。朱大兴没有办法,只有尽力供奉。霍女又爱生病,每天须一碗参汤补养身体。起初,朱大兴很不愿意。但霍女痛哭呻吟,眼见就要快死的样子,无可奈何,给她煮了一碗人参汤,病好像一下子就消失了。自此以后,习以为常。霍女穿的衣服必须是绵绣之类,穿了几天就厌烦了,要换新的。就这样,一个多月,计算起来花钱无数。朱大兴渐渐地供不起。霍女哭泣着不吃饭,要求离开这里,到别处去。朱怕她走,只好委曲顺应她的要求。霍女经常感到苦闷无趣,每每让朱大兴每隔十数日便招戏班为她唱戏。唱戏时,必须让朱大兴在帘外设一凳,让她抱着儿子观看;即使这样,她也无笑容,经常对朱大兴责骂,朱大兴也不去与她辩解。过了两年,朱家渐渐衰落。朱大兴向霍女婉转地说,每日消费是否可以稍减一成。霍女同意了,每日用度减了一半。时间长了,朱家仍然不能供给,霍女每天喝点肉汤也能过得去。又渐渐地,没有珍馐海味也能用得下。朱大兴暗暗自喜。忽然一夜,霍女开门逃跑了。朱大兴怅然若失,到处打听,才知道在邻村何姓家中。

姓何的是邻村大户人家,是宦官之后,他性格豪放无拘束,好结交客人,家中常是灯央亮到天明。忽然有一美丽的女子,半夜来到他的寝门。他细盘问,知是从朱家出逃的小妾。朱大兴的为人,姓何的一向藐视他;又喜欢这女子貌美,竟然把她留下了。二人在一起私混了几天,何某越发被这女人迷惑,生活穷奢极欲,对她的一切供给,如同朱大兴一样。朱大兴得知消息,就到他家要人,姓何的根本不当会事。朱告到官府。官府因为这女子的姓名来历不明,放到一边,也不追问。朱大兴变卖家产,向官府行贿,才准拘捕审问。霍女对姓何的说:“我在朱家,原本也不是通过媒人,纳彩礼而定的,怕他作什么?”姓何的很高兴,准备到公堂上与朱打官司。在座的客人劝谏说:“收纳别人逃跑的妻妾,已经是违法的行为。况且这个女人进门之后,挥霍无度,就是千金之家,怎能支撑得了?”姓何的恍然大悟,就把女人送给了朱大兴。

过了一二天,霍女又外逃了。

有个姓黄的书生,家中很贫寒,未曾娶妻。一天夜里,忽然间来了一位女人敲他的门。女人进门后,自己向黄生说是来给他作妻的。黄见到这样一位美貌的女子,而且是自投到他家,惊慌恐惧,不知该怎么做才好。黄生平素守本分,坚决拒绝。女人也不离去。与黄生应对之时,黄生发现这个女人柔美可爱,不禁心中有点动情,就把她留下了。但又担心她不能安心这贫寒的家庭。但是,女人每天起得很早,操持家务,勤劳超过过门多年的妻子。黄为人蕴藉,举止潇洒,很会取得妻子的欢心。两人相见恨晚,只恐将风声走漏出去,二人的欢快日子不能长久。而朱大兴自从倾产起诉后,家中更加贫穷;又考虑到这个女人不是安分守己的人,也就把追寻她的事,放到了一边。

霍女跟黄生一起过了数年,二人恩爱诚笃。一天,霍女忽然说要回家探亲,要求用车马送她。黄生说:“以前你说无家,为什么前后说法不一样?”霍女说;“以前我是随便说说,我是镇江人。往日,我跟着荡子,流落江湖,就落到这步田地。我家中颇富裕,你把所有的钱财都带去,我必定亏待不了你。”黄生听从她的话,赁了一辆车,与她同去。

到了,扬州地界,把船停泊在江边。霍女正凭窗向外看,有一位巨商的儿子从旁边过去,惊叹她的美丽,又反转船跟在后头。黄生不知道这情况。霍女对黄生说:“你家很贫穷,现在有一个解救穷困的办法,不知你能不能听从我的?”黄问她,霍女说:“我跟你多年,未能为你生一男半女,也是件未做好的事。我虽说不漂亮,幸亏现在还未老,若有人肯出千金的话,你就把我卖给他。有了这份钱,妻室、田庐就都有了。这个办法怎样?”黄生脸面失色,不知这是什么原因。霍女笑着说:“郎君不要着急,天下本来多佳人,谁肯花一千金来买我呢?那是一句玩笑话给旁人听的,看看外面有没有买主。卖与不卖我,本来就在郎君你自己。”黄生不肯这样办。霍女自己把这件事告诉船夫的妻子,船夫妻子用眼看黄生,黄生随便应了一下。船夫妻去后不大会儿,回来说:“邻船有一位商人的儿子,愿意出八百金。”黄生故意摇头,说这事难成。船夫妻又出去了,过了一会,回来说:“同意如数交千金,请马上过船去,一边交钱,一边交人。”黄生微微一笑。霍女说:“叫他暂且等等,我嘱咐黄郎几句话,马上就去。”霍女对黄生说:“我每日以千金之躯侍奉郎君,你今天才知道吧!”黄生问霍女:“你以什么话来推辞掉人家呢?”霍女说:“请你马上过船去签署卖身契约;去与不去,本来就在我自己。”黄生认为不可。霍女逼着催促他去,黄生不得已,去了。立刻兑付清楚。黄生让人把千金封存起来,并加上印记对商人子说:“我虽然贫寒,竟然真的把妻子卖了,马上分离,真是难以割舍!假若妻子必不肯听从,仍然将这金原封不动地归还你。”刚把千金搬运到船上,霍女已同船夫的妻子从船后头登上商人之子的船了,远远地与黄招手作别,无一点依恋的样子。黄生惊骇得魂不附体,咽喉气塞,一句话也说不出来。一会儿商船解缆,如同离弦之箭远远而去。黄生大声呼唤,想追上去与之并行。船夫不听他的,开船南行。很快到镇江,把银子搬到岸上,船夫急急解船而去。

黄生在岸边守着行装苦闷地坐着,举目无亲,到什么地方,自已也不知道。望着滔滔的江水,东流而去,真像万箭穿心。黄生正在掩面哭泣时,忽听到娇滴滴的声音,在唤“黄郎”。黄生愕然回头一看,原来是霍女,已在前边的路上等着。黄生高兴极了,背起行李就跟从她出了,并问:“你怎么回来得这么快?”霍女笑着说:“若再迟回来数刻时问,恐怕你对我就产生疑心了。”黄生仍然认为她的举止不一般,又细细追问。霍女笑说:“我一生办事,对于那些吝啬的人,就破费他的钱财;对于那行为不端邪恶的人,就诳骗他们。假若我如实地把我要作的事告诉你,你必定不肯与我合作,这样,我们到哪里去弄这千金呢?袋里有了充足的钱,我又安然无恙地回到你的身边,你应该感到幸福和满足。你这样穷问到底作什么?”于是,就雇了一个脚力,背负着行李,一块走了。

进了镇江城水门内,有一座门朝南的宅子。他们直接进去。不大会儿,老头老婆男人女人,纷纷出来迎接,都说:“黄郎来了。”黄生就进屋去拜见岳父岳母。有两位年轻人,向黄生作揖施礼,坐下来与黄生说话。他们是霍女的兄弟大郎和三郎。宴席上菜肴不多,四个玉盘就把一张桌子摆满了。鸡、蟹、鹅、鱼。都用刀切成大块。年轻的人用大碗喝酒,谈吐豪放无拘束。宴会结束后,有仆人将他们夫妇领到另一个院子中,让他俩住在一块。床上的铺盖与枕头,滑腻细软,而床,是用熟制的皮革代棕藤制成。每天有婢女及老太太送来三餐。霍女有时整整一天也不出门。黄生在这里单独居住感到苦闷,屡次说要回家,但霍女坚决不让。一天,霍女对黄生说:“今天我为你打算:请你买一位女人,是为了你的子孙后代着想。但是,若买婢女小妾,价格一定很高;你假装当我的兄长,由我父亲出面与别家论婚,这样找一位良家女子是不难的。”黄生认为不可。霍女不听。

有一位张贡士,他的女儿新近死了丈夫。跟他协商的结果,要一百吊钱,霍女强为黄生取来。新妇小名叫阿美,性格和顺,生得也很漂亮。霍女喊她作嫂子,黄生局促不安,霍女反而坦然无事。有一天,霍女对黄生说:“我将和大姐到南海,去看望大姨,要一个月的时间才能返回,请你们夫妻俩安生地过日子。”说完就走了。

夫妻二人独居一院中,霍女家仍然按时给他们送饮食,对他们也很敬重。然而,自从进了这个门后,就不曾有一个人再到他们这房里来。每天早晨,阿美按时去给老太太请安,说一两句话就退出来。妯娌们站在一旁,也只是相视一笑而已。即便留恋不舍多坐一会,他们也不殷勤应酬。黄生去拜见岳父,也是这样。偶尔遇到诸兄弟在一起聚谈,黄生来了,大家都不作声了。黄生心中苦闷,又无处诉说。阿美发觉了这种情形,问黄生说:“你与他们既然是兄弟,为什么一月来都像生疏的客人?”黄仓促间回答不上来,结结巴巴地说:“我在外十年,现今足刚归来。”阿美又细细审问老头与老太太家的门第,以及妯娌们的住处。黄生窘迫,再也不能隐瞒了,就把实底全告诉了她。阿美哭泣着说:“俺家虽贫穷,也不至于卑贱到作你家的小老婆,无怪妯娌们都看不起我。”黄生听了惶惑害怕,不知有什么办法应付,只有跪在地下任凭阿美处置。阿美收住哭泣,用手把黄生拉起来,反而请黄生想办法。黄生说:“我哪里还敢想别的法子,只想让你回娘家去。”阿美说:“既然嫁你了,我再回娘家,于心不忍。那霍女虽说是先跟了你,但那是私奔,不是明媒正娶;我虽说是后嫁的,却是明媒正娶。不如暂且等她归来,问一下她,她既然出了这佯的计谋,将准备如何处置我?”

住了几个月,竟然没见霍女回来。一天晚上,听到客房里有吵闹的饮酒声。黄生偷偷去看,只见二位客人身着戎装坐在上座:一个头裹着豹皮的头巾,威严得像是天神;东首的那个人,戴着虎头的皮革做的头盔,虎口衔着他的额头,虎鼻虎耳俱全。黄生惊骇地回来,把这事告诉阿美,二人猜测一通,也弄不清霍氏父子是什么人。夫妻二人感到疑虑难解,很畏惧,二人谋划着迁到别处居住,又恐引起霍氏父子的猜疑。黄生说:“实话告诉你,那去南海的人,即使回来,当面对证已定,我也不能再住在这里。现在,我想带着你离开这里,又恐怕你的父亲说别的。不如我们二人暂且分手,二年当中我必定再来。你能等待就等待;假若想另嫁他人,也听你便。”阿美要回家告诉父母,跟黄生一块走。黄生不答应。阿美哭泣流涕,要他发誓,他才离别阿美,动身回家。

黄生去给老头老太太告辞。正巧其他诸史弟都出去了,老头挽留他,等女儿从南海回来再走,黄生没听,就告辞走了。黄生上船,心中很凄惨,像失魂落魄一样。船行至瓜州,忽然回头见有片帆飞驶而来;渐近了,看到船头,按剑而坐的是大郎。大郎老远就招呼说:“你想急着回去,为什么不再商量商量。撇下夫人自己独身走了,二三年的时间,谁能等待呢?”说话间,船已靠近。阿美从船中走出来。大郎挽扶着她登上黄生所乘的船,自己跳回船上,径直而去。这以前,阿美回到家中后,刚向父母哭诉,忽然大郎驾车登门来,按着剑威胁他们,逼着他女儿快走。全家人被吓得大气不敢喘,没有敢阻挡的。阿美向黄生述说了刚才的经过,黄生也猜不透他们是什么意思。但自己得到阿美,心中很高兴,就解船出发。

到家后,黄生出钱经营,很富有。阿美时常挂念她父母,想让黄生与她一块回镇江探望双亲;又恐怕把霍女引来,嫡庶问大小尊卑有争执。居住了不久,阿美的父亲打听着来了,见到他们家中房宅整齐,心中颇安慰。对女儿说:“你出门后,我接着到霍家去探访,见他家大门已关,房主也说不清楚,时过半年,竟无消息。你母亲日夜哭泣,说是让奸人把你骗去,不知流落到哪里去了。今天才知道你没出事。”黄生把实情告诉他老岳父,他们猜测着霍家一门为神人。后来,阿美生了个儿子,就取名叫仙赐。到十多岁,母亲让他去镇江、扬州,仙赐在旅社中住下后,随从的人都出去了。有一位女子进来,拉着他的手到她的房间里,放下帘子,将他搁在膝上,笑着问叫什么,仙赐便告诉她自己的名字;又问他:“叫这个名字,是什么意思?”孩子答:“不知道。”女子说:“回去问你的父亲便知道。”就为他在头上挽了个髻子,摘下自己头上的花给他簪上;又拿出一副金钏戴到他的手腕上;又将黄金放到他袖子里,说:“拿去买书读。”仙赐问她是谁,她说:“你不知道你还有一个母亲?回去告诉你父亲:朱大兴死了,但没有棺材埋葬,应当帮助他,不要忘了。”老仆人回到旅店后,不见了仙赐;寻找到另一室中,听到仙赐正与人说话,从外向里一看,是老主母。在帘外轻微咳嗽,好像要有话给她说。女人把仙赐放到床上,恍惚间,已经看不到。仆人问旅店的主人,并没有人知道。数天后,从镇江返回,把这事告诉了黄生,并把所馈赠的东西拿出来。黄生听罢,慨叹不已。等到去询问朱大兴的消息,他已经死去三天了,尸骸暴露在外,未能埋葬,黄生给了他家很多钱,便厚葬了他。


\subsection{1.8.27   司 文 郎}
\label{\detokenize{p00_u5176_u5b83/_u767d_u8bdd_u804a_u658b_u5fd7_u5f02:id330}}
山西平阳府,有位叫王平子的秀才,大比之年,到北京参加顺天府乡试,在报国寺里赁了一间房子住了下来。报国寺中,在他之前就来了一位浙江余杭县的秀才,和他作邻居。王平子递上自己的名片,要求与他相见。但余杭生不答理他。早晨或傍晚与他相遇,余杭生也表现得很傲慢。王平子很恼火他这种狂妄的样子,就打消了与他交往的念头。

一天,有一位少年到报国寺游览,穿着白色的衣裳,头戴一顶白色的帽子,望去很有点不凡的气魄。王平子来到少年跟前与他交谈,少年言谈诙谐,妙趣横生。王平子从心里对这位少年感到敬佩,问起他的乡里门第,他说:“家住登州府,姓宋。”于是,王平子叫老仆人拿座位来,两人相对谈论起来。恰巧余杭生从这里经过,他们两人就都起来给余杭生让座。余杭生居然坐了上座,一点不谦让,又问宋生说:“你也是到顺天府来参加乡试的吗?”宋生回答说:“不是。我是一个才能低下的人,没有腾达的志向。”又问:“你是哪一省的?”宋生就告诉他家住山东省。余杭生说:“竟然没有进取功名之心,足见你是很高明的。山东和山西,没有一个通晓文字的人。”宋生回答说:“北方通晓文字的人确实很少,但是不通晓的人,未必是我;南方通晓文字的人确实很多,然而通晓者未必是你。”说完就鼓掌,王平子与他一唱一和,因而哄堂大笑。余杭生惭愧得很,气呼呼地竖起眉毛,捋起袖子,大叫大囔说:“你们敢当面出八股题,比试一下吗?”宋生不在意地看着别的地方,微笑着说: “这有什么不敢的呢?”余杭生便急忙回到寓所,拿出一本《论语》交给王平子,让他出题。王平子随手把书一翻,指着说:“‘阙党童子将命’。”余杭生站起来,寻找笔墨和纸。宋生拉住他说:“不用写了,随便用口说就可以了。我的破题已经作出来:‘于宾客往来之地,而见一无所知之人焉。’”王平子捧腹哈哈大笑。余抗生愤怒地说:“你是完全不会作文章的,只会骂人,是个什么样的人!”王平子尽力为他两人调解,请另找一道好题。又翻出一个题目说:“‘殷有三仁焉’”宋生立刻答道:“三子者不同道,其趋一也。夫一者何也?曰:仁也。君子亦仁而已矣,何必同?”余杭生一听,便不作了,站起来说:“你这个人也算稍有点才气。”接着就走了。

王平子因为这事就更加尊敬宋生。一天,特邀宋生到自已的寓所中,两人谈了好长时间。王平子拿出自已所写的全部文章,向宋生请教。宋生看得很快,一会儿就看完了上百篇。然后说: “你写文章的功底很深,然而在你下笔为文时,没有一个必定追求的信念,而只是存有一种侥幸取得成功的心理,这样,你的文章就落到下等里去了。”接着取出已看过的文章,一一给王平子解释。王平子很高兴,以老师之礼来对待他。让厨房里的人,用蔗糖作水饺。宋生吃了水饺,很香甜,说:“我平生还未吃过这样甜美的水饺,请你改日再作一次给我吃。”这以后,两人的感情更加投合。宋生三五天必来一次,而王平子必作水饺给他吃。余杭生偶而遇到,虽然谈的不多,但傲慢的气概大大减少了。

一天,余杭生把自己写的文章拿来给宋生看。宋生见上面圈圈点点极多,还有不少赞美之词儿。看了一遍,就放在桌子上了,一句话也不说。余杭生怀疑宋生未看,再次向他请教。宋生说已经看完了。余杭生又怀疑宋生看不懂。宋生说:“这有什么难懂的?只是不好罢了!”余杭生又说:“你只看了圈圈点点和赞语,怎知不好呢?”宋生便背诵他的文章,好像早已读熟了似的。一面背诵,一面指出文章的毛病。余杭生局促不安,汗流浃背,没有说话就走了。

过了一会儿,宋生离去,余杭生进了屋子,坚决要看王平子的文章。王平子不给看。他硬是搜出王平子的文章,看到上面圈圈点点密密麻麻,嘲笑道:“这真像水饺子!”王平子本来性格朴实,不善于说话,这一来,只能是含羞地听着他说而已。

第二天,宋生又来了,王平子诉说了昨天的事。宋生非常气愤地说:“我以为‘南人不复反矣’,这卑鄙的小子竟敢这样欺人!有机会,我一定要报复他!”王平子极力劝他,说对人不要过分刻薄。宋生听了深受感动。

考试结束后,王平子把试卷拿出来,请宋生看,宋生十分欣赏。一天,他俩偶然走进大殿游玩,看到一个瞎和尚正坐在走廊里,摆着药摊,行医卖药。宋生惊讶地说:“这是一位奇人!他最懂得文章,不可不向他请教。”就让王平子回到寓所去把文章取来。王平子回到寓所正遇到余杭生,就与他一同前来。王平子走到和尚跟前,称他老师。那和尚以为他是来求医的,便问他患的是什么病。王平子说是来请教写文章的道理的。瞎和尚笑道:“是谁多嘴多舌啊?我没有眼睛,怎能评论文章呢?”王平子请他用耳朵代替眼睛,自已来念给他听。瞎和尚说:“三场的文章有二千多言,谁能耐着性花那么多时间听下去?不如把文章烧了,让我用鼻子闻一闻就可以了。”

王平子遵从他的意见。每烧一篇文章,那和尚就闻一闻,点点头说:“你是初次仿效几位大名家的手笔,学得虽然不十分像,也做到近似了,我刚才是用脾领受的。”王平子问他:“这样的文章能考中么?”和尚答道:“也能考中。”余杭生听了,不十分相信,先把古代名家的文章烧了一篇试试。瞎和尚用鼻子闻一闻说:“妙啊!这篇文章我是用心受的。不是归友光、胡友信等的手笔,怎么能写这么好呢!”余杭生大为惊讶,便开始烧自己的文章。那瞎和尚说:“刚才领教了一篇,尚未体会到全部妙处,为什么忽然另换一个人的文章呢?”余杭生假意说:“朋友的文章,只是那一篇,这篇才是我写的。”和尚闻了闻余下的纸灰,咳嗽了好几声,说道:“不要再烧了,实在咽不下去,现在勉强咽到胸膈;再烧,我就要呕吐了。”余杭生非常惭愧地退出去了。

过了几天,乡试发榜了,余杭生竟考中举人;王平子反名落孙山。宋生和王平子跑到瞎和尚那儿告诉他,瞎和尚便叹了口气说:“我虽然瞎了眼睛,但并没有瞎了鼻子,那些考试官简直连鼻子也瞎了!”一会儿,余杭生来了,得意洋洋地说:“瞎和尚,你也吃了人家的水饺么?现在究竟怎样?”瞎和尚笑道:“我只是谈论文章罢了,并不与你论命运。你不妨把考官们的文章,各取一篇用火烧掉,我就知道谁是你的老师。”余杭生和王平子一同搜索,只找到了八九个人的文章。余杭生说:“如闻错,拿什么惩罚?”那和尚气愤地说:“把我的瞎眼睛剜掉!”余杭生烧了起来。每烧一篇,瞎和尚都说不是;烧到第六篇,和尚忽然对着墙壁大呕大吐起来,而且放屁如雷,人们都笑起来。瞎和尚擦了擦眼睛,对余杭生说:“这才是你真正的老师呢!起初我不知道,骤然一闻,鼻子和肚皮都受了刺激,膀胱里也容纳不下,直接从肛门里放出来了!余杭生大怒,要走,并说道:“明天我还来看你,你别后悔、别后悔!”过了两三天,他却未来,到他寓所一看,已经搬走了。这才知道他正是那位考官的门徒。

宋生安慰王平子说:“凡是我们读书的人,不应该怨别人,应当严格约束自己。不埋怨别人,道德可以更高;严格约束自己,学问就会越来越深。当前的不得意,固然是运气不好;但平心而论,文章不是已经写得很好了么!今后只要加倍努力,天下总有不瞎的人。”王平子听了,肃然起敬。又听说第二年还要举行一次乡试,就不回家了,留在北京,以便向他求教。

宋生对王平子说:“京城柴米太贵了,但你不要有后顾之忧,屋后有个地窖子,埋着许多银子,可以掘出来用。”并告诉他埋在什么地方。王平子谢道:“宋朝的窦仪和范仲淹虽然很穷,却非常廉洁。现在我尚能自给,哪敢玷污自己的名声呢?”

一天,王平子醉后睡了,他的仆人和厨师便偷偷地去挖掘金窖。王平子忽然醒来,发觉屋后有声,偷偷出去一看,银子都堆在地上了。他们见事情败露,都吓得跪在地上。正要呵斥他们,发现一些金酒杯上刻着字,仔细一看,都是祖父的名字。原来王平子的祖父曾在南方做官,入京后住在这里,后来得急病死了,这些银钱正是老祖所留下来的。王平子大喜,一称,共八百余两。第二天,告诉宋生,并拿出金杯给他看,想与他平分,宋生坚决推辞了。王平子又拿了一百两银子送给瞎和尚,瞎和尚已走了。此后几个月,他越发刻苦读书了。

考期又到了,宋生说: “这次如果再考不中,那真的是命运了!”谁知,王平子竟因违犯场规被取消了考试的资格。王平子还没有什么怨言,宋生却大哭起来,王平子反而安慰起他来。他说:“上天嫉妒我,让我潦倒困苦了一辈子,今又连累了好友,真是命啊,真是命啊!”王平子说:“世间凡事本来都有定数的。像宋先生本无意求取功名,我考不中当然与你的命运毫无关系了。”他擦着眼泪说:“我早就想对你讲,实在是怕你惊怪,我并非是世上活着的人,而是一个飘泊无定的游魂。我年轻时,很有些才名,却一直不得志,连连落第。一气之下到了京都希望得到一位知音,把我的著作传下去。谁知,李自成进攻北京那一年,竟死于战乱。这样一年一年地到处飘泊,幸亏遇到你,相知相爱,所以我想极力帮助你;让好朋友得以实现我自己的宿愿。谁知今天,我们在文场上的命运是如此的不幸,谁又能无动于衷呢!”王平子也感动得掉下眼泪,问他:“为什么一直被埋没?”他说:“去年上帝有命令,让孔老夫子及阎罗王核查历劫的鬼魂,上等的在官署中备用,其余转生人世。我的名字已被录用,之所以未去,因为我想看到你考中后的快乐。现在我们只好告别吧!”

王平子问他考的是什么官职,他说:“梓潼府里缺一名司文郎,暂时叫一个耳聋的书僮代理,这就是文运颠倒的原因。万一侥幸得到这个官职,一定要圣教得以宏扬光大。”

第二天,宋生高高兴兴地来了,说:“我的愿望实现了。孔夫子让我做一篇《性道论》,看完后,非常高兴,说我可以做司文郎了。阎罗王一查生死簿,要以我说话无约束为罪名,不录用我;幸亏孔老夫子力争,才保住这个官职。我叩头拜谢。孔老夫子又把我叫到案前,嘱咐我说:‘现在因为怜惜你的才能,才选拔你充任这个清高的要职,你要改过自新,认真办事,不要再犯以前的错误了!’由此可知,阴曹对于道德,比文学更为看重。你一定是品行尚未修行好,今后只要积累善行不要懈怠就可以了。”王平子又问:“果真如此,那么,那个余杭生的德行如何呢?”他说:“不知道。阴曹赏罚分明,毫无错误,就是前几天我们看到的那个瞎和尚,也是一个鬼,他是前朝的名家,只因生前抛弃的字纸太多,罚他做瞎子。他想借替人医病,来赎以前的罪过,所以他常到热闹地方来。”王平子命人准备酒菜。宋生说:“不必了。终年打扰你,剩的时间不多了,再为我准备些水饺就足够了。”王平子非常难过,一点也不想吃,让他自己在那儿吃。一会的工夫,宋生就吃了三碗,捧着肚皮道:“这一顿饭,可以三天不饿。我这样做,乃是表示不忘你待我的好处。从前我吃你的水饺,都埋在屋后,已经变成蘑菇了。采集下来,藏起来做药,小儿吃了,可以变得更聪明。”王平子问他,以后什么时候再相会,宋生说:“既然做了官,就应该避开嫌疑。”又问:“如果到文昌帝君庙里祭奠,能达到你那儿吗?”他说:“这都没有什么好处!九天太远了,只要你洁身自好,多多积善,自有地府的人通报,那么,我是一定会知道的。”说完,向王平子告别后就不见了。王平子到屋后一看,果然长着许多紫色的菌。采集下来,藏在罐中。旁边有新土坟突起,宋生吃的水饺好像都在那里。

王平子回家后,更加刻苦读书。一天夜里,梦见宋生乘着车,上面张着伞盖来了,并说:“你从前因为发了点怒,误杀了一个婢女,在福禄簿上削去了官职、功名,如今你的德行已经把你的罪行赎掉了。但是你的命太薄了,还是没有做官的希望。”这一年,他参加顺天府乡试,考取了举人;第二年,又考中进士。王平子从此以后,也不图进取了。有两个儿子,其中一个生来很笨,脑袋迟钝,王平子给他吃了那些蘑菇,就很聪明了。后来,因为别的事情到南京,巧而遇到余杭生也到南京办事。谈到阔别之情,很是谦逊,然而两鬓已是斑白了。


\subsection{1.8.28   丑 狐}
\label{\detokenize{p00_u5176_u5b83/_u767d_u8bdd_u804a_u658b_u5fd7_u5f02:id331}}
有一个姓穆的书生,是长沙人,家里非常贫穷,到了冬天还没棉衣穿。

有天晚上,穆生正独自在家里闷坐,忽然进来个女子,衣着华丽耀眼,但长得却又黑又丑,笑着问穆生说:“你不感到冷吗?”穆生惊讶地询问她是什么人,女子回答说:“我是个狐仙。可怜你寒冷寂寞,想和你同床共枕。”穆生害怕她是狐狸,又厌恶她相貌丑陋,大声号叫起来。女子掏出一块元宝放到桌上,说:“你如答应,我把这个赠给你!”穆生见了元宝,高兴地同意了。床上没有被褥,女子便将自己的外衣脱下来铺上。二人直睡到天快明时,女子起床嘱咐说:“我给你的元宝,你快拿去买布来做被褥,剩下的钱,做件棉衣,买点酒菜,足够了。只要你和我永远相好,就不用愁贫困!”说完就走了。

穆生把这事告诉了妻子,妻子也很高兴,马上买布来缝制被褥。狐女晚上来后,见被褥一新,喜欢地说:“你家娘子太勤劳了!”临走前又留下银子作为酬劳。从此后,狐女每晚都来,每次离去,必定赠些钱物。这样过了一年多,穆生家的房屋变得整洁华美,全家人的衣着也都里外一新,居然成了富裕人家。

穆生富裕后,狐女赠给他的东西渐渐少了。穆生因此越来越厌恶她。一次,他请了个会驱狐的道士,画了张符贴在门上。狐女来后,把符咬下来扯碎,扔到地上,进屋指着穆生骂道:“忘恩负义,你可算是登峰造极了!你这样做又怎能奈何得了我!你若厌烦我,我自己会走的。但情义既已断绝,你过去从我这里接受的东西,须要再还给我!”说完,忿忿地出门走了。穆生害怕,忙告诉了那个道士。道士便布置起法坛,准备驱狐。还没布置完,那道士忽然摔倒在地,血流满面。一看,已被割去一只耳朵。众人大吃一惊,四散逃窜。道士也捂着耳朵狼狈逃走了。这时,像盆一样大的石块,纷纷砸到穆生屋里,门窗锅盆,全被砸烂,没一件囫囵的。穆生钻到床底下,蜷曲着身子,吓得冷汗直流。一会儿,见狐女怀中抱着个猫头狗尾巴的怪物进来,把怪物放在床前,唤它说:“嘻嘻!快去啃那坏蛋的脚!”怪物一口就咬住了穆生的脚,牙齿锋利得像刀刃一样。穆生十分恐惧,想缩回脚来,但四肢却不能动弹。怪物嚼起他的脚趾,发出咯咯吱吱的脆响。穆生疼痛万分,衷恳不止。狐女说:“所有的金银财宝都给我拿出来,不要隐瞒!”穆生连忙答应,狐女叫了声:“呵呵!”那怪物就不咬了。穆生爬不起来,只是告诉狐女藏钱的地方。狐女自己去搜寻,除了首饰衣服之外,只翻出了二百两银子。狐女嫌少,叫了声:“嘻嘻!”怪物又啃起穆生的脚来。穆生哀叫着求饶,狐女限他十天内还清六百两银子,穆生答应,她才抱着那怪物走了。

又过了很久,家人渐渐聚集起来,从床下把穆生拖了出来。只见他脚上鲜血淋漓,已没有了两个脚趾头。看看室内,财物被搜寻一空,只剩下当年的一床破被子还在。众人便拿来盖在穆生身上,让他躺下。穆生害怕狐女十天后再来,卖了使女和衣服,凑齐了六百两银子。十天后,狐女果然又来了。穆生急忙将银子交给她,狐女收下,默默地走了。从此后再没来过。穆生脚上的伤,医治了半年才好,家里又像从前那样一贫如洗了!

后来,狐女又跟了邻村一个姓于的。于某是农民,家境贫寒。过了三年,于某除了按时交纳官府的粮税外,还建起了成片的高房大屋,一家人所穿的华丽衣服,多半是原来穆生家的东西。穆生见了,也不敢问。一次,穆生偶然到野地去,在路上碰到狐女,他急忙跪在路边。狐女默默不语,只用白手巾包上五六两银子,远远地扔给穆生,返身便走了。后来,于某去世后,狐女还不时到他家中,但家里的财物往往随之消失了。于某的儿子再看见她来,便行礼参拜,远远地祷告说:“父亲去世,我们都是您的孩子。即使不怜恤我们,又怎忍心坐视我们贫困呢?”狐女听了,便走了。从此再没到过于家。


\subsection{1.8.29   吕 无 病}
\label{\detokenize{p00_u5176_u5b83/_u767d_u8bdd_u804a_u658b_u5fd7_u5f02:id332}}
洛阳有个叫孙麒的公子,娶了蒋太守的女儿为妻,夫妻二人感情极好。后来蒋氏二十岁时死去,孙麒悲痛不已,离家住到了山中一座庄园里。

一天,正碰上阴雨天气,孙麒躺在床上休息,屋里别无他人。忽然看见门口门帘下露出一双女人的小脚,孙麒惊疑地问是谁。只见门帘一掀,进来一个女子,年纪约十八丸岁,衣着朴素整洁,面色微黑,长了很多麻子,像是穷人家的女儿。孙麒以为是村中来赁房的,呵斥她说:“有什么事应当去告诉我的家人,怎么竟闯到我的屋里来了?”女子微笑着说:“我不是村里的人。我祖籍山东,姓吕。父亲是文学士,我的小名叫无病。跟随父亲客居到这里,父亲早已去世了。我孤独无靠,仰慕公子出身于大家,又是名士,愿意投奔您这个郑康成做您手下的文婢。”孙麒笑着说:“你的心意倒很好。但在这里我跟仆人们住在一起,实在不方便。等我回家后,再用顶轿子聘了你来。”女子踌躇地说:“我自料才疏貌丑,怎敢奢望做您的配偶呢?只想供你在书斋里驱使,我倒还不至于把书捧倒了!”孙麒说:“就是收你做婢女,也得挑个吉日啊!”说着,用手指指书架,命她把《通书》第四卷取来,意思是试试她的学问。女子翻检了一通,找到了书,自已先浏览了浏览,才交给孙麒,边笑着说:“今天河魁星不在房里。”孙麒听了,不禁动了心,便把她留下了,藏在室内,不让外人知道。

无病闲着没事,替他抹桌子、整理书籍、焚香、擦香炉,把房间整理得光洁一新,孙麒大为高兴。到了夜晚,孙麒命仆人都到别处去睡,只让无病伺候。无病察言观色,服侍得更加殷勤周到。直到叫她去睡觉,她才端着蜡烛走了。孙麒半夜一觉醒来,觉得床头上像躺着个人,用手一摸,知道是无病,便摇醒了她。无病惊恐地起身站在床下。孙麒责备她说:“怎么不到别处去睡?我的床头是你睡觉的地方吗?”无病怯怯地说:“我胆小,不敢独睡。”孙麒可怜她,让她睡在床里边。忽然,他闻到无病身上传来一种莲花一般的清香气息,大感惊异,便叫她和自己同枕一个枕头。孙麒心神摇荡,渐渐拉无病同睡一个被窝,二人欢爱一场,孙麒十分喜欢她。孙麒又想:老这样让无病躲藏着,总不是办法。又怕领她一同回家会惹人议论。孙麒有个姨母,跟这里只隔着十几家,他便和无病商量着让她先避到姨母家,以后再接她回来。无病觉得这办法好,便说:“你阿姨我早就很熟,不用你先去通知,我这就去。”孙麒送她,她就越墙走了。

孙麒的姨母是一个寡老太太。天明后她打开门,一个女子闪身走了进来,她忙询问,女子回答说:“你外甥让我来问候阿姨。公子想回家,因路远缺马,留我暂时借住在阿姨这里。”老太太相信了,便留住了她。

孙麒搬回家后,假称姨母家有个婢女,姨母想送给自己,派人把无病接了回来。从此后,便让她坐卧不离地服侍自己。日子一长,孙麒更加宠爱无病,便娶了她作妾。有高门大户想和他结亲,他一概不答应,大有和无病白头到老的意思。无病知道后,苦苦地劝他娶妻,孙麒只得又娶了许家的女儿为妻,但终究还是宠爱着无病。许氏非常贤惠,从不和无病争床第之欢,无病侍奉她也越发恭敬,因此二人关系很好。后来,许氏生了个儿子,取名叫阿坚,无病对待孩子像自己亲生的一样爱护。孩子刚三岁,常离开乳妈,跑去跟无病一块睡。许氏叫他回去,也不走。过了不久,许氏因病死去,临死前嘱咐孙麒说:“无病最爱护我的儿子,孩子就算是她亲生的好了;把她扶正作嫡妻,也可以。”埋葬了许氏后,孙麒便要按许氏的遗言去做,把这事告诉亲族后,大家都说不可,无病也坚决推辞,这事也就罢了。

本县有个王天官的女儿,新近守寡,托人来孙家求婚。孙麒非常不愿意结这门亲事。王家再三请求,媒人也极力宣扬王氏的美貌;加上孙麒的亲族仰慕天官大人的势力,一昧怂恿他,孙麒动摇了,到底还是娶了王氏。王氏果然生得非常艳丽,但性情却异乎寻常的骄悍。平时的衣服用具,一不称意,就乱毁乱扔。孙麒因为喜欢她,不忍违了她的性子。过门才几个月,便霸住丈夫,不让他和无病同房。还经常把怒气迁移到丈夫身上,几次三番地大吵大闹。孙麒受不了,便一个人独宿。王氏更加恼怒。孙麒烦恼不堪,找了个借口跑到京城中,避难去了。王氏又把孙麒的出走归罪于无病,尽管无病看着她的脸色,小心伺候,但王氏还是不高兴。有一天夜里,她让无病睡在床下伺候,阿坚总是跟着无病。每次叫起无病来支使,阿坚就啼哭不休。王氏厌烦地痛骂阿坚,无病急忙叫乳妈来抱走他。阿坚不走,想强让他走,他哭得更厉害了。王氏大怒,从床上蹦下来,将阿坚一顿毒打,他才跟着乳妈走了。阿坚从此后被吓出了病,不吃不喝。王氏禁止无病去照料阿坚,阿坚整天啼哭。一次,王氏呵斥乳妈把阿坚摔到地上,孩子哭得声嘶力竭,喊着要水喝,王氏不让给;直等到天黑,无病窥见王氏不在,偷偷地拿了水去给阿坚,阿坚看见她,丢了水扯住她的衣服号啕大哭。王氏听见,气势汹汹地走了出来。阿坚听到她的声音,立即憋住哭声,腿一伸,吓得背过气去了。无病见状,不禁失声痛哭起来。王氏大怒,骂道:“贱婢少做这种丑态!想用孩子的死威胁我吗?不用说是孙家的小崽子,就是杀了王府的公子,王天官的女儿也担当得起!”无病听了,只得抽泣着忍住眼泪,请求葬了阿坚,王氏不许,立命把他扔了。王氏离去后,无病摸了摸阿坚,觉得身上还温热,便暗对乳妈说:“你快抱了去,在野地里等等我,我马上就去。如果孩子死了,我们一块埋了;如果能活过来,我们就一同抚养他。”乳妈答应着走了。

无病回到房里,带上自己的一些首饰,偷偷地跑出家门,追上了乳妈。两人一块看看阿坚,见孩子已苏醒过来,二人非常喜欢,商量着到孙麒的庄园去,投奔姨母生活。乳妈担心无病走不动,无病便先走一步等着她。只见她走起来快得像风一样,乳妈使出全身的力气才能赶上她。约二更时分,阿坚的病又变得沉重起来,没法再继续赶路。二人便抄近路进了个村庄,来到一个农家的门前,在门口直站到天明,才敲开人家的门,借了间屋子住下。无病又拿出首饰,卖了换成钱,找来巫婆和医生给阿坚治病,可是仍不见好转。无病掩面哭泣着说:“乳妈好好看着孩子,我找他父亲去!”乳妈正惊讶她说得太荒唐,无病却一下子不见了,乳妈惊诧不已。

同一天,孙麒在京城中,正躺在床上休息,无病悄无声息地走了进来。孙麒吃惊地起身说:“我刚睡下就做开梦了吗?”无病抓住他的手,只是跺脚,哽咽得说不出话来。过了好久好久,才失声说道:“我受尽了千辛万苦,和孩子逃到杨——”话没说完,放声大哭,一下子倒在地下不见了。孙麒吓呆了,还怀疑是在梦中。忙叫仆人一块来看,见无病的衣服、鞋子还仍然在地上,众人大惑不解。孙麒急忙整治行装,星夜往家赶来。到家后,听说儿子已死,无病远逃,孙麒捶胸大哭,骂了王氏几句。王氏却反唇相讥。孙麒怒发冲冠,顺手摸起把刀子,丫鬟婆子们急忙拦阻他,孙麒走不近王氏,远远地把刀子抛了过去,刀背正砸中王氏的额头,血流了出来。王氏披头散发,鬼哭狼嗥地跑出家门,要去告诉娘家。孙麒将她捉了回来,索性痛打一顿,直把她的衣服都打成了碎条,疼得她转不动身,才命将她抬回房中护养,想等她伤好后再休了她。王氏的弟兄们听说这件事后,率领众人骑着马打上门来。孙麒也聚集起自家健壮的仆人,准备抵御。双方互相叫骂了一整天才散。王家没赚到便宜,不肯罢休,又打起官司。孙麒也让人护送着赶进城去,向官府申辩,控诉王氏种种的凶悍劣迹。县令不能使孙麒屈服,便把他送到专管风俗教化的学官那里惩戒,以此讨好王家。学官朱先生,是世家子弟,为人刚正不阿,察知实情后,愤怒地说:“县令老爷以为我是天下最卑鄙的教官、专门勒索伤天害理的财物给人舔屁股的无耻之徒吗?这种乞丐相,我做不来!”竟不接受县令的命令,让孙麒堂而皇之地走了。王家无可奈何,便示意亲朋好友,为他们两家调停,让孙麒到王家谢罪。孙麒不肯,调解人往来十多次,还是没有结果。王氏的伤也渐渐好了,孙麒想休了她,又怕王家不要人,只得不了了之。

孙麒因为无病逃走,孩子又死了,日夜伤心。想找到乳妈,问个实情。想起无病曾说过“逃在杨……”的话,邻村有个杨家疃,他怀疑她们逃到了那里,便去察问,结果没一个知道的。有人说五十里外有个村子叫杨谷,孙麒忙派人骑着马去访查。果然找到了乳妈和阿坚。原来,阿坚并没有死,病也渐渐痊愈了。相见之后,都非常欢喜,派去的人把她们接了回来。阿坚看见父亲,放声大哭,孙麒也流下了眼泪。王氏听说阿坚还活着,气势汹汹地跑出来,还想咒骂他。孩子正在哭着,一睁眼看见王氏,恐惧地一下子扑在父亲怀里,像是要藏起来。孙麒忙抱起来一看,阿坚已死过去了。急忙大声叫他,过了会儿才苏醒过来。孙麒怨恨地说:“不知如何酷虐,把我的儿子吓成这个样子!”立即写下离婚文书,送王氏回娘家。王家果然不要人,又把王氏送了回来。孙麒迫不得已,自己和儿子另住一个院子,再不与王氏来往。乳妈跟孙麒详细讲了无病的一些奇怪事情,孙麒才醒悟无病是鬼。十分感激她的情义,便将她的衣服、鞋子葬了,立了一块碑,上题“鬼妻吕无病之墓”。

又过了不长时间,王氏生下一个男孩,她却亲手把孩子掐死了。孙麒更加忿怒,再次休了王氏。王家却又把她用车子送了回来。孙麒便写下状子,告到官府。官府因为王氏是天官大人的女儿,对孙麒的状子都不受理。后来,王天官死去,孙麒仍在不停地上告,官府便判决将王氏休回了娘家。孙麒从此后再没娶妻,只是纳了个奴婢作妾。

王氏回娘家后,因为凶悍的名声远扬在外,住了三四年,没有一个来提亲求婚的。王氏这才幡然悔悟,但过去的事情却已无法挽回。后来,有个曾被孙家雇佣过的老妈子来到王家,王氏殷勤地款待她,还对着她流了不少眼泪。揣测王氏的心思,像是怀念原来的丈夫。老妈子回去后便告诉了孙麒,孙麒一笑置之。又过了一年多,王氏的母亲也死了。她孤单一人,无依无靠,几个兄嫂弟妹又都厌恶嫌恨她。王氏越发走投无路,只落得个天天泪水涟涟。有个贫寒的读书人死了妻子,王氏的哥哥便想送给一份厚厚的嫁妆,让她嫁给那个读书人,王氏不肯。她多次托来来往往的人给孙麒捎信,哭泣着说自己已为过去感到悔恨,孙麒始终不听。

一天,王氏带着一个婢女,从家里偷了头驴骑着,跑到孙家来。孙麒正好走出家门,王氏迎面跪在台阶下,哭得泪流不止。孙麒要赶走她,王氏拉住他的衣服再次跪下。孙麒坚决推辞说: “我们如再次复婚相聚,平时如无纷争还好;一旦有纠纷,你弟兄们个个如狼似虎,再想离婚,可就难了!”王氏说:“我这次是偷跑来的,绝没有再回去的道理。你愿意留下我,我就留下;否则,只有一死而已!况且我自二十一岁跟了你,二十三岁被休回娘家,即使我有十分的罪恶,难道就没一分的情义吗?”说完,从手腕上脱下一只金钗,并起双脚,套上金钗,用袖子盖在上面,说:“我们成亲时焚香立下的誓言,难道你不记得了吗?”孙麒热泪盈眶,让人把她扶进内室,但仍然怀疑王氏在欺骗自己,想得到她弟兄们的一句话作为证据。王氏说:“我私自逃了出来,有什么脸再去见我的弟兄?如不相信,我身上藏着自尽的工具,请让我断指以明心迹!”说着,从腰里掏出一把刀子,把左手搁在床边,一刀砍去了一截手指,鲜血进流。孙麒大吃一惊,急忙为她包扎伤口。王氏疼得脸色惨变,却不呻吟。笑着说:“我今天才从黄粱梦中醒来,特来借一间斗室,做出家的打算,你又何必猜疑我呢?”孙麒便让儿子和妾另外住一间房子,自己天天两处来回跑。又多方寻求好药,替王氏医治手上的伤口,一个多月才好了。王氏从此后不吃荤腥,只是关着门念佛而已。

又过了很久,王氏见家务废驰,没人管理,便对孙麒说:“我这次来,本想什么事都不管不问的;但现在见全家开支如此浪费,入不敷出,恐怕将来子孙们会有饿死的。没办法,我就再厚着脸皮料理料理吧!”于是,她召集女仆们,按日定量让她们纺线织布。家人因为她是自己跑上门来的,十分瞧不起她,私下里讥讽嘲笑她。王氏像是听不见。既而检查纺织数量时,凡是懒惰没完成定额的,都挨了她一顿鞭子,毫不客气,众人这才怕起她来。王氏又亲自监督管帐目的仆人,事事精心算计。孙麒十分高兴,让儿子和妾每天都去拜见王氏。这时,阿坚已九岁了,王氏对待他加倍温存,每天早上他去了私塾,王氏常常留下好吃的东西等他回来。因此,孩子也渐渐地和她亲近起来。

一天,阿坚用石块打麻雀,正好王氏经过,石块掉下来砸中了她的脑门,王氏一下子摔倒在地,昏迷过去。孙麒大怒,痛打儿子。王氏醒过来,极力劝阻,还喜欢地说:“我过去虐待过儿子,心中老觉得有块心病,这下可以抵消我的旧恶了!”孙麒听了,越发宠受她。但王氏常常拒绝和他同房,让他去和妾睡。过了几年,王氏屡次生产,但每次婴儿都夭折了。王氏说:“这是我过去杀死亲生儿子的报应啊!”阿坚结婚娶妻后,王氏便把外事委托绐儿子,家务事委托给儿媳妇。一天,她忽然说:“我某日就要死了!”孙麒不信。王氏自己料理起葬具,到了那天,她更换衣服,自己进入棺内去世了。面色还如活着时一样。这时,只闻到室内充满了一种奇异的香味,直到把她入敛后,香味才渐渐消失了。


\subsection{1.8.30   钱 卜 巫}
\label{\detokenize{p00_u5176_u5b83/_u767d_u8bdd_u804a_u658b_u5fd7_u5f02:id333}}
夏商,河北河间县人。他的父亲名叫东陵,十分富豪,但生活奢侈,吃包子就将包子的两角丢掉,扔得狼藉满地;加以他长得很肥胖,人们就给他个绰号,叫“丢角太尉”。到了晚年,夏东陵家中穷困,每天连饭都吃不上;两只胳膊极瘦,皮耷拉着像条布袋,人们便又呼他“募庄僧”——说他像挂着袋子四处化缘的和尚。到他临死时,对儿子夏商说:“我一生任意浪费的东西太多,冒犯了上天,所以使我无吃无穿地死去。你当珍惜自己的福气,好生去为人,以挽回你爸爸的过失。”

夏商严格遵守父亲临终时的遗嘱,为人诚实质朴,没有歪道,亲自耕作生活。乡亲们都很敬重他。本村中富人某翁,同情他家中的贫寒日子,借给他钱,让他学着经商。但夏商不会作买卖,结果亏了本,自己感到很惭愧,没有能力偿还人家的本钱,就要求雇给这个富翁作佣人。富翁不肯,夏商很不安,就把自己的耕地房屋都卖掉,把换得的钱给富翁送去。富翁问清情况,更加怜悯他,强把夏商卖掉的田产房屋赎回来;又重新借给他更多的资本,让他去作买卖。夏商推辞说:“我借你的十多两银子已亏本偿还不了,怎么还想让我来世作驴子再还您的债呢?”富翁就让他和别的商人结伴而行。几个月后回来,仅能不亏本;富翁不要他利息,让他再出去经商。过了一年多,夏商到南方购置了满满一车货物,回来时在江上遭到飓风,船差点翻了,货物丧失了一半。回家后,计算了一下剩下的货物,仪能够偿还所借贷的钱。夏商就对其他同伙说:“上天要让你贫穷,谁能挽回呢?这都是我连累了你们。”就按帐本的记载,把钱付给商人,自己退出了买卖行当。富翁再强使他经商,他坚决不干,就在家中老老实实地耕种。他常自己慨叹说:“人活在世上,都有几年的好日子,为什么我竟落魄到这种地步?”

一天,夏商遇到一位从外地来的算卦先生,说他能用钱占卜,能知道一个人一生的运气。夏商特地去找他,到那里一看,占卜的人是一位老婆子。她住的房舍精致而清洁,当中设有神的座位,香气熏染。夏商进去,拜完神位之后,占卜的婆子就向他收费。夏商给了她一百个钱,巫婆将钱全装到筒中,用手拿着在座前跪下,用手摇响竹筒,作出祈祷的样子。接着就起来,将钱倒在手中,然后在桌案上按次序摆开。她的占卜方法是以钱的字(正面)为“否”卦,以钱的幕(背面)为“亨”卦;她数到五十八个钱,皆出字,以后的钱则都出幕,便问夏商:“多大岁数?”夏商回答:“二十八岁。”巫婆摇摇头说:“早啊!早啊!您现在交的是先人运,并不是您本身的运。五十八岁方才交您自己的运数,才无盘曲交错。”夏商问:“什么叫先人运?”巫婆说:“若先人生前行善,他的福自己未享尽,则后人就享他的福;若先人生前有不善之事,他所造的祸,自己未受尽,则后人要接着受。”夏商屈指一算说:“再过三十年,我已经成了老头子,也快进棺材了。”巫婆说:“你五十八岁以前,有五年运数稍转,但也无大益,只能免于饥饿。五十八岁这一年,应有一笔大的钱财来到你手中,不需要你费力气去追求。先生你一生中无有过头的行为,就是到来世,你也享受不尽。”夏商告别巫婆回到家中,心里半信半疑。安心地过着贫寒的日子,不敢有别的想法。到五十三岁,他就很留意验证巫婆的话是否灵验。当时,刚开春,农田里的活开始耕作,夏商患了虐疾,不能下地。病好了又遇上天大旱,早种上的作物都枯死了。到秋上,才下了一场雨,家中也没有别的粮种,夏商把几亩地都种上谷子。接着又是大旱,荞麦豆子半数被旱死,只有耐旱的谷子长得还好;后来又下了几场雨,谷子生长得更好,较往年多收一倍。第二年的春天,又遇上饥荒年景,家中老小总算没有挨饿。夏商因为这件事,就相信巫婆说的话是灵验的。他便向那个富翁借钱,做一些小本买卖,结果有一点收获;有人便劝他去作大本买卖,夏商不肯。待到五十七岁,夏商偶而修葺垣墙,挖地时得到一个铁锅;揭开后,从地下冒出如同白絮般的烟气,夏商弄不清原因,也不敢再挖了。过了一会,烟气冒尽了,见到满满的一瓮白银子。夫妻一块搬运,一秤,共一千三百二十五两。夏商心中暗想,巫婆所卜的还是有点差错。邻居的妻子到夏商家,见到这许多的白银,回家告诉她的丈夫。她丈夫忌妒他们,就偷偷地告了官府。河间县的县令最贪婪,就把夏商捉来,向夏商诈索银子。夏的妻子想藏起一半,交一半,夏商说: “这并不是我们应该得到的钱财,留下来也招致祸患。”就把所有的银子全献给县令。县令得到银子,恐怕夏商有所匿藏,又向他追索盛银子的那口大瓮,把所有的银子放进去,瓮满了,才把夏商放了。没有多久,县令调任南昌府同知。过了一年,夏商因行商到南昌,这个县令已死,县令的妻子要回故乡,把粗重的东西卖掉了;有几篓桐油,夏商看了很便宜,就都买下来,全部运回家中。运到家后,见有的油篓渗漏,便把桐油倒在其它器具中,结果发现篓内有白银二铤;试探一遍,每个篓都有二铤白银。兑换后,正与所掘银数相符。夏商由此暴富,越发照顾贫穷的人,每每慷慨接济他们。妻子劝夏商积蓄点留给子孙,夏商说;“我这样做,就是遗留给子孙的福。”那位告发他的邻居,穷得光光,想向夏商借贷几个钱,而心中老觉自己作了亏心事。夏商得知,告诉他说:“过去的事,那是我的时运未到,所以鬼神假借你的手,把事告发,这与你有什么关系?”邻人感动得流下泪来。后来,夏商活到八十,子孙相继,历数代而不丧败。


\subsection{1.8.31   姚 安}
\label{\detokenize{p00_u5176_u5b83/_u767d_u8bdd_u804a_u658b_u5fd7_u5f02:id334}}
姚安,是临洮县人,生得秀美,风度潇洒。同村中有个姓宫的,有个女儿名叫绿娥,长得很艳丽,通晓诗书,一直没有选到合适的女婿。绿娥的母亲对别人说:“心须是门第和风采都像姚安一样,我才将女儿嫁给他。”姚安听说后,就哄骗妻子到井边去,将妻子推下井。接着就续娶绿娥为妻。

姚安娶了绿娥后,夫妻二人很恩爱。然而,姚安因为绿娥艳美,所以不很放心,经常怀疑她,整天把她关在家中守着她。绿娥只要一行动,他就紧跟着;绿娥想回娘家,姚安就用两肘撑着袍子,遮盖着绿娥出去,等绿娥上了轿子,姚安就把轿门加上封条,作个记号,完了后,自己跟随在后头,在娘家住一夜,就催着绿娥一块回来。绿娥心中受不了,气忿地说:“我若有男女私情,哪里是你这卑琐的举止所能管得了的!”姚安每次有事出门,就把绿娥关在家中。绿娥更加讨厌他这种行为,等他走了,故意将钥匙放到外边,以使他生疑。姚安归来看见钥匙大怒,质问绿娥,这钥匙是哪里来的?绿娥愤然地说:“不知!”姚安越发疑心,偷着对绿娥戒备更严。

一天,姚安从外回家,在门外偷听了很久,才开锁敞门。他怕门发出响声,悄悄从门的狭缝中塞进去。进屋,见一个男人头着貂皮帽子,躺在他的床上。姚安一见大怒,拿过刀跑进屋里,狠狠的就是一刀。走近一看,是绿娥白天睡觉,因怕寒冷,用貂皮帽子盖着脸。姚安大惊,跺着脚很是后悔。绿娥的父亲气忿地告到官府。官府下牒收捕了姚安,扒掉他的衣服,施以酷刑。姚倾家荡产,用很多的钱贿赂上下官吏,才得免死。但从此后,他便精神恍惚,若有所失。一次,正好他自己独坐,见绿娥同一满腮胡子的男人在床上亲热。姚安很厌恶,手持着刀过去。然而,刚到床前就不见了;姚安生气地转过来坐着,又见到这种景象。姚安怒不可遏,用刀去砍床,床上的席与被褥都破碎了。他又愤怒地持着刀,到床边上候着,见到绿娥与自己面对面站着,看着他笑。骤然挥刀砍去,立即将头砍断;刚坐下,绿娥又出现在原来的地方,如老样子笑着。夜晚将灯熄灭,就听到淫慝声,不堪入耳。每天都是这样。姚安再也不能忍耐下去,就把自己的田宅全卖掉了,想搬到别处去住。到夜里,小偷又挖开墙壁进来,将他所有的钱都偷走了。自这以后,姚安穷得无立锥之地,他在气愤中死去。邻居们用一张苇席卷着,把他埋葬了。


\subsection{1.8.32   采 薇 翁}
\label{\detokenize{p00_u5176_u5b83/_u767d_u8bdd_u804a_u658b_u5fd7_u5f02:id335}}
明朝覆灭的时候,到处发生战乱。於陵刘芝生聚集数万军队,准备渡江到南方去,忽然一个肥胖的男子来到军营栅门前,敞着衣襟露着肚腹,求见军队头领。刘芝生请他进去,跟他一谈,非常高兴。问他的姓名,那人自称采薇翁。刘芝生便留他在军中帮着参谋军事,又要给他兵刃,采薇翁说:“我自己有锐利的兵刃,不须你的矛戟。”刘芝生问在什么地方,采薇翁将衣襟捋起,露出肚腹。只见肚脐很大,可容鸡子。他憋住气鼓起肚腹,忽然肚脐发出嗤嗤的声音,突出一柄剑把。采薇翁用手握住剑把抽出剑,白刃如霜。刘芝生大惊,问:“还有吗?”采薇翁笑笑,指着肚子说:“这是武器库,什么没有啊!”刘芝生就让他取弓箭,采薇翁又像先前一样,取出一把雕弓,稍微一屏气,就有一支箭飞出来,接着又连续飞出无数支箭。后来他把剑柄插入肚脐中,马上就全都不见了。刘芝生觉得他太神了,与他常在一起,非常敬重他。

这时,军营中号令虽严,但士兵却是一群乌合之众,时常有人到老百姓家中抢劫东西。采薇翁说:“军队中重要的是纪律。如今,你统兵数万,但不能震慑住人心,这是自取败亡。”刘芝生很高兴,便纠察士兵,有到百姓中抢掠妇女、财物的,杀头示众。从此军中纪律稍好起来,但抢掠的事情却没有断绝。采薇翁不时骑马出去在军队中巡视,每次巡视时军中那些悍将骄兵的脑袋常常是自己掉下来,不知什么原因。大家都怀疑是采薇翁干的。先前,他向刘芝生建议严饬军纪,士兵们已经是又怕他又厌恶他;现在出了这神怪事,就更加怨恨他。各部的首领都向刘芝生诬告说:“采薇翁用的是妖术,自古以来的名将只听说靠智慧,没听说靠妖术。剑侠神仙之道,最终也逃脱不了灭亡。如今,无辜的将士往往自己失落了脑袋,大家群情激愤,人人自危。将军与他相处,也是危险的,不如想办法除掉他。”刘芝生听从了他们的话,打算等采薇翁睡下的时候杀掉他。派人察看他,见他正袒露着肚腹躺着,鼾声如雷。大家很高兴,让士兵包围住他的住处;两人拿刀进去砍下他的头;等抬起刀,头已经复合起来,喘息同原来一样。众人大惊,又砍他的肚腹,肚腹裂开却没有血,腹内的刀枪剑戟白森森的,都露出了锋利的尖。大家更惊骇,不敢靠近,站在远处用矛一拨,箭一下子放出来,射中了好多人。众人吃惊地散开了。回去告诉刘芝生,刘芝生急忙去看时,采薇翁已经不见了。


\subsection{1.8.33   崔 猛}
\label{\detokenize{p00_u5176_u5b83/_u767d_u8bdd_u804a_u658b_u5fd7_u5f02:id336}}
崔猛,字勿猛,是建昌府大户人家的子弟,性情刚毅。童年时在私塾中,同学们稍有触犯他,他就挥拳殴打。先生屡次劝戒,他依旧不改。他的名和字都是先生起的,也是劝他不要太刚猛的意思。

长到十六七岁,崔猛更是勇猛无比,更兼有手绝技:能手拄长杆,飞房越脊。他为人喜好抱打不平,因此,四邻八乡的人都佩服他,找他告状申诉的人挤满了庭院。崔猛锄强扶弱,不怕结仇。那些坏蛋稍违背了他,他就石头砸,棍子敲,直把他们揍得腿断胳膊折。每当他盛怒时,没有敢劝的。但他对母亲最为孝敬,不管有多大的怒气,母亲一到就烟消云散。母亲管教他最严厉,往往痛加斥责,他当时唯唯听命,但一出门就忘得干干净净。

崔猛的邻居家有个凶悍的婆娘,天天虐待她的婆婆。婆婆快要饿死了,儿子偷着给她一点饭吃,那婆娘知道了,百般辱骂,吵得四邻不安。崔猛大怒,翻墙过去,将那婆娘的耳朵鼻子、嘴唇舌头全割了下来,不一会儿就死了。崔母听说后,大吃一惊,急忙叫过那婆娘的丈夫来,极力安慰,并把自家的一个年轻奴婢许配给他为妻,这事才算了结。为了这件事,崔母气得痛哭流涕,也不吃饭。崔猛害怕,跪在地上请母亲处罚自己,还说自己已经很后悔。母亲只是哭泣,也不答理他。崔猛的妻子周氏见此情景,也跪在了地上求情,崔母才用拐杖痛打了儿子一顿;又用针在他胳膊上刺了个十字花纹,涂上红颜色,以免磨灭,让他牢记这次训戒。崔猛接受了,母亲才开始进食。

崔母平时喜欢布施化缘的和尚、道士,常让他们尽情吃饱。一次,有个道士来到家门口。崔猛正好走过,道士端详了端详他,说:“你满脸都是凶横之气,恐怕难保善终。你们积德行善的人家,不应当如此。”崔猛刚刚领受了母亲的训戒,听了道士的话,肃然起敬,说:“我也知道这点。但我一见不平之事,就苦于控制不住自己。我尽力去改正,能免了灾祸吗?”道士笑着说:“先别问能免不能免;请先问问自己能改不能改。只要你痛改前非,即使有万分之一的希望,我会告诉你一个解脱死亡灾难的法术!”崔猛平生最不相信道士的法术,因此听了道士的话,只笑不答。道士说:“我本来就知道你不相信。但我所说的法术,不是巫婆们搞的那一套。你照着去做了,固然是积德的事;假设没有效验,对你也没什么妨碍。”崔猛便向道士请教。道士于是说:“在家门外正有个年轻人,你应当跟他结成生死之交。将来即使你犯下死罪,他也能救你!”说完,把崔猛叫出门外,把那个年轻人指给他看。原来,那人是赵某的孩子,名叫僧哥。赵某,本是南昌人,因为遭了灾荒,领着儿子流落到了建昌。崔猛从此后努力结交僧哥,请赵某在自己家设馆教书,待遇十分优厚。僧哥这年十二岁,拜见了崔猛的母亲后,和崔猛结成了兄弟。过了一年多,赵某就领着儿子返回老家去了。音讯从此断绝。

崔母自从邻居那婆娘死后,对儿子管束得更严。有来家诉说冤屈的,她一律撵出去。一天,崔母的弟弟去世了,崔猛跟着母亲去吊丧。路上碰到几个人,用绳子捆着个男人,连打带骂,催促快走。围观的人挤住了路,崔母的轿子过不去。崔猛便问路人是怎么回事。这时有认得他的人,抢着向他诉说原委。原来,有个大官家的公子,横行一方,无人敢惹。这恶少窥见李申的妻子生得美貌,便想夺到手,但没有个借口。他便叫家人引诱李申去赌博,借给他高利贷,让他拿妻子作抵押,还要立下字据。李申输完,又借给他钱。李申赌了一夜,输了好几吊钱。半年后,连本带息,已欠那恶少三十吊。李申还不上,恶少便派爪牙将他妻子强抢了去。李申跑到恶少门上痛哭,那恶少大怒,将李申拉去绑到树上,百般毒打,逼他立下“无悔状”。崔猛听到这里,气塞胸膛,把马猛抽一鞭,就要冲上前去,看样子又想动武。他母亲急忙拉开轿帘喝道:“住手!又要犯老毛病吗?”崔猛只好停住。

吊完丧回家后,崔猛不说话,也不吃饭,只是呆坐着,眼光直直的,像是在跟谁呕气。他妻子问他,也不答话。到了夜晚,他穿着衣服躺在床上;辗转反侧,直挨到天明。第二天夜里,又是如此。后来他忽然起身下床,开开门走了出去;一会儿又回来躺下,像这样一连折腾了三四次。他妻子也不敢问他,只是屏住呼吸,听着他的动静。最后,他出去很长时间后才回来,关上门上床熟睡了。

这天夜晚,那恶少被人杀死在床上,开膛破肚,肠子都流了出来。李申的老婆也赤裸着身体被杀死在床下。官府怀疑是李申干的,将他逮了去严刑拷打,脚踝骨都打得露了出来,李申还是不承认。拖了一年多,李申忍受不了酷刑折磨,终于屈打成招,按律被判死刑。这时,正好崔母去世了。埋葬了母亲后,崔猛告诉妻子说:“杀死那恶少的人,是我!以前因为有老母在,所以不敢招认。现在为母送终的大事已经了结,我怎能拿我的罪责让别人遭殃呢?我要去官府领死了!”他妻子听了,吃惊地扯住他的衣服,崔猛一挥手,挣开妻子,径自去了官府自首。官府听他说了事情的经过,大吃一惊,立即给他戴上刑具,押入狱中,释放了李申。李申却不走,坚决申明人是自己杀的。官府也没法判明,便将两个人都下到狱中。李申的亲属们都讥讽李申太傻,他说:“崔公子做的事,正是我想做却做不到的;他替我做了,我怎忍心看着他死呢!今天就算他没有自首好了!”一口咬定是自己杀了人,和崔猛争着偿命。时间长了,衙门里的人知道了事情的真实情况,强将李申赶了出去,判崔猛死刑,马上就要处决了。

正好刑部的赵部郎,驾临建昌巡视。他在提审死囚案时,看到崔猛的名字,便让随从都出去,然后把崔猛叫上来。崔猛进来,仰头往大堂上一看,原来那赵部郎就是赵僧哥!崔猛悲喜交集,照实说了事情的经过。赵部郎考虑了很久,仍叫崔猛先回狱中,嘱咐狱卒好好照顾他。不久,崔猛因为自首,依律减罪,充军云南。李申自愿跟随着他,服刑去了。不到一年,崔猛按惯例被赦罪回家。这都是赵部郎从中出力的结果。

李申从云南回来后,便跟着崔猛生活,为他料理家业。崔猛给他工钱他也不要,倒是对飞檐走壁、拳脚刀棒之类的武术很感兴趣。崔猛优厚地对待他,替他买了媳妇,并送给他田产。崔猛经过这次变故后,痛改前非,每每抚摸着臂上的十字花纹,想起母亲生前的训戒,就痛哭流涕。因此,乡邻再有不平之事时,李申总是以崔猛的名义自己为他们排解,从不告诉崔猛。

有一个王监生,家里十分富豪。远远近近的那些无赖不义之徒,经常在他家进进出出。本县中的殷实富裕人家,很多都被他们抢劫过。有谁如惹了他们,他们就勾结强盗,将他杀死在野外。王监生的儿子也非常荒淫残暴。王监生有个守寡的婶母,父子两个都和她通奸。王监生的妻子仇氏,因为多次劝阻丈夫,王监生便将她用绳子勒死了。仇氏的兄弟们告到官府,王监生用钱财买通了官吏,反说他们是诬告。仇氏兄弟们有冤无处申,便到崔猛家来哭诉。李申听了两句,打发他们走了。

又过了几天,崔猛家里来了客人。正好仆人不在,崔猛便让李申去泡茶。李申默默地走了出去,跟人说:“我与崔猛是朋友,跟着他不远万里,充军云南,交情不可算不深。可他不但从没给过我工钱,还拿我当仆人支使,我再不甘忍受了!”便忿忿地走了。有人告诉了崔猛,崔猛谅讶他忽然变了心,但还没感到有什么奇怪的。李申忽然又打起官司,告了崔猛三年没给他工钱。崔猛这才大感惊异,亲自去官府和他对质,李申忿忿地和崔猛争执不休。官府认为李申在无理取闹,将他赶了出去。

又过了几天,李申忽然夜间闯进王监生家,将王监生父子连同王监生的婶婶一并杀死,还在墙上贴了张纸条,写上自己的名字。等到官府追捕他的时候,他早巳逃得无影无踪了。王家怀疑李申是崔猛主使的,官府却不相信。崔猛此时才恍然大悟:李申和自己打官司,原来是怕杀人后连累了自己。官府行文附近州县,紧急追捕李申。不久,正赶上闯王李自成打进北京,这件案子也就搁了起来。明朝灭亡后,李申才携带家眷回来,仍旧和崔猛住在一起,二人和好如初。

当时,正值天下大乱,贼寇蜂拥而起。王监生有个侄子叫王得仁,聚集起叔父生前所招的一帮无赖之徒,占山为盗,烧杀抢掠,无恶不作。一天夜晚,王得仁率领群盗倾巢而出,以报仇为名,攻打崔家。崔猛正好有事外出,强盗攻破崔家大门后,李申才发觉,急忙翻墙逃出,趴在暗处。强盗搜不到崔猛、李申,便将崔猛的妻子掳了去,将所有的财物都搜掠一空。李申回去后,见家里只剩下一个仆人,又气又急。他找来一股绳子,砍成几十段,把短的交给仆人,长的自己揣到怀里。嘱咐仆人摸到强盗巢穴的背后,爬上半山腰,用火点着绳子头,散挂在山上的荆棘丛中,然后立即返回。仆人答应着去了。李申曾见强盗们腰里都扎着根红带子,帽子上系着红绢,他也依样打扮好了。正好家里有匹老母马,刚生了小马驹,强盗们没要,丢弃在门外。李申便把马驹拴在门口,自己骑上母马,直向强盗们的老巢冲去。

强盗们占据了一个大村子,李申将马拴在村外,翻墙越院,摸进村内。见强盗乱纷纷的到处都是,手里还都拿着兵器。李申私下问了个强盗,知道崔猛的妻子正在王得仁处。一会儿,听见有人传令,让大家都休息,群盗们轰然答应。这时,忽然有人大声叫喊东山上有火,强盗们一齐往东望去,果然见有火光。最初只有一二点,既而多得像天上的星星一样。李申乘机大叫东山上有敌人。王得仁大惊,急忙披挂整齐,率众前去迎敌。李申乘机溜到后面,窜进王得仁的住处。见有两个强盗守卫着,李申假说:“王将军忘了带佩刀。”两个强盗听了,争着去找,李审从他们背后一刀砍去,一个中刀倒在地上,另一个忙回头看,李申又一刀斩了他,背着崔妻翻墙而出。跑到村外,李申解下那匹母马,把缰绳递给崔妻说:“娘子不识得路,只管放开马跑吧!”母马恋驹,一路奔跑回家,李申在后面跟着。出来谷口,李申把怀中的长绳头掏出来,用火点着,遍挂在山谷上,才回家来。

第二天,崔猛回来,听说了这件事后,认为是自己的奇耻大辱,气得暴跳如雷,想单人匹马去踏平贼窝,李申劝阻住了他,召集村里的人一块商量个对付的办法。但大家都害怕强盗,没有敢出头的。李申再三劝导,才凑了二十来个敢和强盗作战的壮丁,却又苦于没有兵器。这时,正好从王得仁的亲属家里抓到了他的两个奸细,崔猛便想杀掉他们,李申认为不可。他们叫那二十来个壮丁都手持白木棍,排成一队,将那两个奸细拖来,当众割去了耳朵,让他们走了。众人都埋怨说:“咱们这点人,本来就怕强盗知道底细,现在反而把实情泄露给他们,假如他们倾巢而来,全村可就保不住了!“李申说:“我正想让他们来!”

李申先把窝藏强盗奸细的人全部杀了,又派人四下里出去借弓箭、火铳,还到县里借了两尊大炮。傍晚,李申率壮士来到谷口,先把火炮安放在谷口要道,派两个人拿着火捻子埋伏着,嘱咐他们看见强盗来了,就点火放炮。然后又带人在山口的东边,伐了很多树木堆在山坡上。一切布置完,李申和崔猛各率十几人,分别埋伏在山谷两旁。一更快完的时候,远远听见战马嘶鸣,强盗果然蜂拥而来,人马络绎不绝。等强盗们都钻进了山谷,李申命将砍下的树木全部推落谷底,阻断了强盗的退路。接着,火炮轰鸣,喊杀声震动山谷。强盗急忙往后退,自相践踏,一片混乱。退到谷东口,树木阻路出不去,强盗们挤成了一个蛋。这时山谷两边火铳齐放,万箭齐发,势如暴风骤雨。强盗们断头折足、横七竖八地躺在谷底,最后只剩下二十来人,跪在地上哀求饶命。李申派几个人将他们绑起来押送回去,自己率队乘胜直捣强盗的老巢,守卫的强盗们闻风而逃。李申将强盗的辎重全部缴获了来,大胜而回。

崔猛高兴万分,询问李申当初救自己妻子时设置火绳的道理。李申说:“在东山放火绳,是把强盗们都吸引到东边,防止他们往西追赶,因为我们从西边撤退。火绳短,很快就烧完了,是怕强盗们侦察到山上没人。最后把火绳放在谷口,是因为谷口狭窄,一人当关,万人莫开,强盗们就是追了来,看见火光必然害怕。这都是一时没有办法而想出的冒险的下策。”把俘虏的强盗押了来审问,果然他们追进山谷后,望见谷口有火光,就吓得撤退了。李申把俘获的二十多个强盗全部砍掉鼻子后放走了。从此,李申威名大振。远远近近的避乱逃难的人,都投奔他。他由此办成了一个有三百多人的团练。各处的强盗没有敢来侵犯的,使这一片地方得到了安宁。


\subsection{1.8.34   诗 谳}
\label{\detokenize{p00_u5176_u5b83/_u767d_u8bdd_u804a_u658b_u5fd7_u5f02:id337}}
范小山,是青州府人,以贩卖毛笔为生,在外经商没有回来。

四月间,他的妻子贺氏独居家中,夜间被人杀死。这天夜里,细雨濛濛,人们在出事地点的泥中发现了一把题诗的扇子,是王晟赠送给吴蜚卿的。王晟,不知是什么人;吴蜚卿,是益都城里殷实之家,与同邑的范小山相识。吴蜚卿平日为人很轻浮、佻达,所以同乡人见到这把扇子,都认为人是他杀的。县衙把他捉去审问,他不承认;当用了惨酷的大刑后,他承认了,就定了案。这个案子送到府里;府里又转到县里,经历了十多个判官的手,无一人提出异议。吴蜚卿自己认为是死定了,便嘱咐他的妻子,把家中所有的财产都拿出来,救济那些孤苦的人。有到他家门前诵读佛经一千遍的,就给一条棉裤。于是,他家门前来来去去讨饭的,每天就像集市一样。诵读佛经的声音,在十多里外都可听到。因此,家中很快贫穷下去,每天只能依靠出卖田地房屋维持生活。吴蜚卿自己感到无生路可想,就暗地里买通了监守的,买来毒酒,想自杀。夜间梦到神人告诉他说:“你不要死,往日是‘外边凶’,眼下是‘里边吉’啊!”再睡觉。又梦见这些话,于是,他就没有自杀。

没有多久,周元亮起补山东青州海防道,当他读到囚犯吴蜚卿的案子时,感到这起案件审理有疏失,就问:“吴蜚卿杀人,有什么确凿的证据?”范小山说有扇子一把为证。周道台反复看了看那把扇子,问:“王晟是什么人?”回答说不知道。周先生又把审讯时的记录取出来看了一遍,立刻命令除掉吴蜚卿的死牢刑具,将他从重犯的内监解到关押轻犯人的外仓。范小山力争说不妥,周道台愤怒地说:“你想冤杀一个人了事呢,还是想得到真正的仇人才甘心呢?”大家怀疑周道台与吴蜚卿有私情关系,都不敢追问。周道台掷下一支红色的签子,立刻拘捕南部某店的主人。店主人恐惧,不知为什么。拘捕到以后,周道台就问:“你店的墙壁上有东苑李秀才的题诗,是什么时候题的?”回答说:“是去年,提学大人来青州府考试时,日照县的两三个秀才醉后所题,但不知他们住在哪里。”周道台便派人到日照,拘捕李秀才。数日后,李秀才被押解到。周道台在大堂上,问:“你既然身为秀才,为什么谋杀人呢?”李秀才跪下叩头,不知所措,惊惶地说:“没有这回事。”周道台把扇子掷到他的面前,让他自己看,说:“这分明是你作的诗,为什么伪托王晟?”李秀才审视后说:“诗,是我作的,但字并不是我写的。”周道台问:“既然知道你的诗,那人当然是你的朋友了,那么这是谁写的?”李秀才说:“这字迹,好像是沂州府王佐所写。”周先生又立即派遣差役到沂州府拘捕王佐。王佐被押到公堂,周道台审讯他,其过程和审问李秀才的情形一样。王佐说:“这是益都城铁商张诚求我写的,说王晟是他的表兄。”周道台说:“盗贼就在这里啊。”把张诚捕来,一审他就全部招认了。

原来,张诚见到贺氏很美丽,想去勾引她,但怕她不答应。自己想若作这件事,须用金蝉脱壳之计,如伪托吴蜚卿,人们必定都会相信的,故托人题一把扇子落款吴蜚卿。若事情作得很顺利就把自己的名字告诉贺氏,倘若中间多磨,就用此扇为证,嫁祸于吴蜚卿,本意并不想杀死贺氏。张诚翻墙进去,强追贺氏。贺氏因为独居,平日常将把刀放在自己的身边,以防万一。这次,她觉察到有恶人,就捉住张诚的衣服,手拿着刀起来。张诚害怕了,从贺氏手中夺过刀来,但贺用力拉住他的衣服,使张诚无法逃脱,关且大声地呼叫。张诚觉得困窘无法,就举刀将她杀死,丢掉扇子逃跑了。就这样,三年的一桩冤狱,一朝被昭雪,人们无不称赏周道台断案如神。吴蜚卿这时方悟神人所说“里边吉”就是个“周”字啊。但是,始终不解周道台如此断案的原因。

后来,益都城的一位绅士,乘一个机会向周元亮问起这件事。周元亮笑着说:“这案很容易看破。我细细翻阅这个案子的审讯记录,贺氏是四月上旬被杀死的。这天夜里,又是细雨濛濛,天气还有寒意,扇子并不是急需之物,哪里有在匆匆急迫的时候,反而携带这多余的累赘东西?凶手想嫁祸别人的用心是可以看出的。以前,我在城南避雨,见到墙壁上题诗一首,与扇子上的题诗完全相同。所以,我最初没有根据地猜测李秀才,结果,还是由这条线索把真正的杀人犯挖了出来。”在座的人听了,无不佩服。


\subsection{1.8.35   鹿 草}
\label{\detokenize{p00_u5176_u5b83/_u767d_u8bdd_u804a_u658b_u5fd7_u5f02:id338}}
关外山中,鹿很多。当地人常常在头上戴一个假鹿头,蹲伏在草丛中,口中含着一片卷曲的叶子,吹作鹿鸣之声,引得群鹿都集拢来。但群鹿中,公的少,母的多。公鹿的本性,常是一次交配,千百只母鹿必配一遍,所以交配完后,公鹿也就累死了。母鹿用鼻子嗅一嗅,知公鹿已死,于是大群的母鹿,就分别跑到山谷中,去寻觅一种具有异香气味的草,放在公鹿的嘴旁熏它。已死的公鹿嗅到这种气味,顷刻间,就苏醒过来。这时,蹲伏于草丛中的人,就急忙敲锣、放火铳,群鹿惊慌逃走。人们就将这种神奇的草取去。据说它可以起死回生。


\subsection{1.8.36   小 棺}
\label{\detokenize{p00_u5176_u5b83/_u767d_u8bdd_u804a_u658b_u5fd7_u5f02:id339}}
天津有个船夫,一天夜里,梦见一个人来跟他说:“明天,有个人来租船载运竹筒,一定要向他索要一千两银子;如他不出这个价,就不给他运。”船夫醒来,不相信这回事。刚睡下,那个人又来对他说了一遍,并且还在墙上写下“xxx”三个字,嘱咐说:“倘若那人舍不得出钱,你马上写这三个字给他看。”船夫醒来,越发感到奇怪。但他不认识这三个字,也不明白是什么意思。第二天,船夫留心过路的旅客。太阳快落山时,果然有个人赶着骡子,装载着竹筒,来向他租船。问到租价时,船夫照梦中的价要。那人笑他要价太高。两人争执了很长时间,船夫便抓过那人的手,用手指写了那三个字。那人见了,非常惊讶,转眼就不见了。船夫查看装载的货物,原来是几万只小棺材,每只比手指头大一点,里面都装有几滴血。船夫把那三个字让远近的人看,没有一个认识的。事过不久,吴三桂叛逆的密谋暴露了,党羽全部被杀,被杀的人数和小棺数几乎一样。这件事是徐白山说的。

注:“xxx”中的这三个字,打不出来。第一个字:“厂”字里面两个“贝”字(左边一个贝、右边一个贝)。第二个字:“厂”字里面三个“贝”字(上面一个贝、下面两个贝)。第三个字:“厂”字里面四个“贝”字(上面两个贝、下面两个贝)。


\subsection{1.8.37   邢 子 仪}
\label{\detokenize{p00_u5176_u5b83/_u767d_u8bdd_u804a_u658b_u5fd7_u5f02:id340}}
滕县有个杨某,跟白莲教党徒学了些邪门歪道的妖术。徐鸿儒被杀后,杨某侥幸漏网逃脱,于是凭借邪术遨游四方。杨家田园楼阁很多,称得上相当富有。他到泗水边上某士绅家表演幻术,士绅家的妇女出来看热闹。杨某发现士绅的女儿长得很漂亮,回家后谋划用幻术将她弄到手。杨的后妻朱氏,也颇有风韵,杨给她穿上华丽的衣服,假扮为仙女,又给她一只木鸟,教她驾驶方法,就从楼顶把她推下去。朱氏自己觉得体轻像树叶,飘忽忽地在云中飞行。不久,到了一个地方,云挡住去路,不能前进,知道已经到了。这天晚上,月色皎洁,俯看下界之物清清楚楚。朱取出木鸟投掷,木鸟振动翅膀飞了出去,直飞到士绅女儿的房间。女孩看见彩鸟飞进来,叫丫鬟把它捉住,鸟又从窗户飞出。女孩出来追它,鸟落到地上发出振动翅膀的声音。女孩靠近它,它钻到女孩的衣裙下面,转眼功夫,背起士绅女儿飞上天空,直冲云宵。丫鬟见了大哭起来。朱氏在云中说:“下界的人不要惊恐,我是月宫的嫦娥,她是王母娘娘第九个女儿,偶然被贬谪尘世。王母娘娘天天想念她,暂时招她去聚一会儿,就送回来。”朱氏便和士绅女儿把衣襟结在一起飞行。到了泗水边,刚好有人放飞爆竹,斜刺里撞上木鸟的翅膀,鸟惊翻坠地,把朱氏也拉了下来,跌落在一家秀才的院子里。

秀才叫邢子仪,家里一贫如洗,但性格耿直刚正。曾经有邻家少妇夜里私奔他被他拒绝了。这少妇恼羞成怒走了,在她丈夫面前诬陷邢子仪挑逗调戏她。她丈夫本来是个无赖.早晚跑到邢家门口辱骂。邢子仪便变卖家产,到别的村子租房住下。有个姓顾的看相先生能看出人的福禄寿命,邢子仪登门请他看相。顾先生看了他的面相后笑着说:“你富足得有千钟收入,为什么要穿烂衣服见人?难道以为我有眼无珠吗?”邢子仪嗤笑他胡说。顾先生又仔细端详了一会儿说:“是的。目前你虽有些破落,但是你的财源不远了”。邢又说他乱讲。顾先生说:“你不但会发横财,而且还会得到漂亮的老婆。”邢子仪始终不相信。顾先生把他推出来说:“先去先去,等证明我说的话灵验后再向你要酬金。”这天夜里,他独自坐在院子里赏月,忽然两个女子从天而降,仔细一看,都极为漂亮,他惊奇地以为是妖怪,盘问她们的来历,开始不肯说。邢子仪要喊村里人来,朱氏害怕了,才把实情说出,还嘱咐他不要泄漏出去,愿意终身跟着他。邢子仪想世家女儿不同于妖人的老婆,便派人通知丢失女儿的士绅。士绅夫妻自从女儿飞升后,整天落泪惶恐,忽然收到报告女儿情况的书信,这突如其来的喜讯真是出人意料,立刻备车马日夜兼程赶赴邢家。为了报答邢子仪,给他一百两银子,带着女儿回来了。邢子仪得到漂亮老婆,正担心没有一点钱财,有了士绅送的百两银子,高兴极了,跑去酬谢顾先生。顾又仔细看了他的面相,说:“还没到还没到,才开始交上好运,百两银子算什么!”仍然不接受他的酬谢。

士绅回到家后,请求地方长官逮捕杨某。杨预先知道消息,逃跑了,不知跑到哪里去了。接着抄了他家,发公文追查朱氏。朱氏很害怕,牵着邢子仪的衣襟哭泣。邢也拿不出办法,只好暂且用钱去贿赂拿公文抓人的差役.租车辆带朱氏去找那个士绅,哀求他帮忙解脱朱氏。士绅感激邢送还女儿的义举,竭力帮他奔走,结果官府答应交钱赎罪。士绅还留邢子仪夫妇住在他的别墅里,感情融洽得和亲戚一样。士绅女儿小的时候许配给刘家。刘家是个大官僚,听说她在邢家寄住了两晚上,认为受了侮辱,退回了婚书,断绝了婚姻关系。士绅打算另外给女儿找个婆家,女儿对父母说她立誓嫁给邢子仪。邢听了很高兴,朱氏也乐意,表示愿当小老婆。士绅担心邢没房子安家,当时官府拍卖杨某的房子,士绅替邢买下。夫妻便回去把士绅过去酬谢的银了拿来买些家具,雇丫鬟和仆人,十来天就把钱花光了。只好希望士绅女儿嫁过来时带些钱来。一天晚上,朱氏对他说;“我那个做孽的前夫杨某,曾把一千两银子埋在楼下,只有我知道。刚才我去看过埋银子的地方,盖在上面的砖和石块都和原来一样,也许窖里埋的东西还在。”两人一起去挖,果然挖出一千两银子。便相信顾先生相面术的神奇,送给他丰厚的酬金。后来士绅女儿过门,带来了丰厚的嫁妆。不到几年,邢子仪便成了本地区首富。

异史氏说:“白莲教被歼灭了,杨某却没有死,而且还发了财。几乎叫人怀疑天网恢恢的法条疏而有漏了。谁能知道上天把他留下,原来为了成全邢子仪。不然,即使邢子仪结束了贫困交上好运,也不能在极短的时间里盖起楼阁亭台,积累大笔财富。他不贪色跟邻妇苟合,而上天报答他两个漂亮女人。唉!造物主虽然不说话,而它的心意是可以知道的。”


\subsection{1.8.38   李 生}
\label{\detokenize{p00_u5176_u5b83/_u767d_u8bdd_u804a_u658b_u5fd7_u5f02:id341}}
商河县人李生一心向佛。村外一里多路的地方有座佛寺,他在寺内修建诵经修行的禅房三间,在里面打坐。四处化缘的和尚道士,来往都寄宿在这里,李生就和他们倾心交谈,热情地供给所需之物,从不厌烦。

有一天,下着大雪,特别寒冷,来了一个老和尚,挑着行李要求借个床睡觉,谈吐特别玄妙。住了两晚就要走,李生执意挽留,又住了几天。刚好有别的事李生要回家住,老和尚嘱吩他早些回到寺庙里,意思是要和他告别。鸡叫时分他又回来了,敲门没人答应。他跳墙进到寺中,看见禅房灯火通明,怀疑和尚在做什么事情,隐藏下来偷偷向里看。只见和尚整顿行李,一头瘦驴拴在灯架上。细看不像真驴,很像殉葬的纸驴。可是耳朵尾巴不时地动一下,咻咻地喘着气。一会儿收拾好了,开门牵着驴走了出来。李生偷偷地跟在后面,门外有个大水池,老和尚把驴拴在池边的树上,赤身跳入水中,把全身洗干净。穿上衣服牵着驴进入水池,也把驴子洗了一遍。然后放好行李,跳上驴背,跑得飞快。李生才喊他。和尚只是远远地拱手致谢,说的话没听清,已经走得很远了。王梅屋说:李生是他的朋友,曾经到过李生家,见堂上悬一匾额写着“待死堂”,也是通达之士。


\subsection{1.8.39   陆 押 官}
\label{\detokenize{p00_u5176_u5b83/_u767d_u8bdd_u804a_u658b_u5fd7_u5f02:id342}}
赵公,是湖广武陵县人。曾在太子宫中做过詹事官,年老后退休还乡。

一天,有个少年人来到赵公门口,恳求赵公收留他掌管文书。赵公将他叫进屋,见他生得文雅秀气,便询问他的姓名。少年人自称叫陆押官,还说情愿不要工钱,赵公便留下了他。陆押官非常聪明,胜过其他仆人。赵公的往来书信,他随便一写,便无不精妙;有时主人和客人对弈,他在一边看看,一指点,主人就赢了。赵公因此更加宠爱他。其他仆人见他得到主人的青睐,便闹着要他请客。陆押官答应了,问道:“共有多少同事?”正好赵公田庄里的管家们都来了,一下子聚集了三十多人。大家便把这些人也算进去,想为难为难他。陆押官说:“这太容易了。但客人太多,仓促间来不及现办酒席,我们到酒店去吧!”于是,遍请同事们,到临街一家酒店去。

大家进店坐下后,酒菜马上就上来了。刚要开始喝,有个人一把按住酒壶,站起身说:“大家先不要喝。请问今天谁是东道主?应当先拿出钱抵押在这里,大家才能开怀痛饮。不然,最后一下子花掉上千钱,大家一哄而散,跟谁要钱去?”大家听了,一齐看陆押官。陆押官笑着说:“莫不是以为我没钱吗?我有的是钱!”说着起身向面盆中抓了一块拳头大小的面团,又一点一点掐下来扔到桌子上;小面团随扔随变成了老鼠,满桌子乱窜。陆押官随便捉住一只老鼠,用手一裂,哧地一声肚子破了,取出一小块银子;再捉一只,又取出块银子。顷刻之间,老鼠都捉完了,碎银摆满了桌面。陆押官对大家说:“难道这些钱还不足以供大家喝酒吗?”众人见了,大感惊异。于是一起痛饮。喝完洒,算了算帐,花了三两多银子。大家再称称桌上的碎银,刚好符合这个数目,不多不少。有个人便要了一枚碎银揣在怀里,回去后跟主人禀报这件奇异的事。主人听了命他拿出银子来看看,他忙往怀里一摸,银子却没有了。于是他又回酒店去告诉店主,店主一看,那些碎银都变成了蒺藜。仆人回来把这事又告诉了主人。赵公便询问陆押官是怎么回事。陆押官说:“朋友们逼着我请客喝酒,我正好口袋里没钱,小时候学了点小戏法,所以现在试了试。”大家又要他还酒店钱,陆押官说:“我不是那种骗酒喝的人。某处田庄有个麦穰垛,再去扬扬场,可得两石小麦,足以偿还酒钱了!”于是他央求一个人同去。正好那座田庄的管家要回去,便和陆押官一路同行。一到场中,只见几斛小麦已堆在那里了。众人由此对他更加感到惊奇了。

一天,赵公去一个朋友那里赴酒宴。朋友家堂屋中有盆兰花,开得十分茂盛。赵公见了非常喜欢,回来后还在赞叹不已。陆押官说:“大人如真喜欢这盆兰花,也不难弄来。”赵公不太相信。第二天凌晨,赵公到书房中去,忽闻异香扑鼻,一盆兰花赫然入目。箭叶的多少跟在朋友家看到的那盆完全一样。赵公怀疑是陆押官偷来的,便询问他。陆押窟说:“我家里养的花,有成百上千盆,何须偷呢?”赵公不信。正好那个朋友来了,见了兰花惊异地说:“怎么这么像我家的那一盆!”赵公说:“我刚买了来,也不知这盆花出自哪里。只是你临来时,见你的那盆还在吗?”朋友说:“我来时没去书房,那盆花还在没在,实在不知。但如果这盆是我的,它怎么会跑到这里来了呢?”赵公听了,眼睛盯着陆押官。陆押官说:“这很好分辨:您家的那盆兰花,盆子破了,有修补的地方;这盆却没有。”大家一检查,果然不错。到了夜晚,陆押官告诉主人说:“刚才我说我家有很多花卉,现在请您前去,乘月观赏。但别的人不能跟随,只有阿鸭可以去。”阿鸭,是赵公的童仆。赵公听从了。一出门,已有四个人抬着顶小轿,等在路边。赵公坐上后,只觉轿子走得比马跑得还快。一会儿,便进入一座深山。但闻异香扑面,沁入骨髓。来到一个洞府,见房屋非常华丽,一点也不像是人间。到处都装饰着花石,一盆盆奇花异草,流光溢彩,散发出阵阵香气。仅兰花一种,就大约有几十盆,都开得非常茂盛。欣赏完后,仍如来时那样乘轿返回家来。

后来,陆押官跟随了赵公十几年。赵公无病去世后,陆押官便和阿鸭一同走了,谁也不知去了哪里。


\subsection{1.8.40   蒋 太 史}
\label{\detokenize{p00_u5176_u5b83/_u767d_u8bdd_u804a_u658b_u5fd7_u5f02:id343}}
太史蒋超,记得自已前世,是四川峨嵋山的和尚。他曾数次梦到自己到前世居住的庵前池塘中洗脚。他生平笃信佛经,一心归宗于天台佛教这一派。蒋太史虽然很年轻就供职于翰林院,但心中常存有出世的念头。

晚年,蒋太史告病假还乡。但走到江苏高邮时,就不想回家了。儿子苦苦地挽留,他也不听。转道到了四川,先是居住在成都的金沙寺。住了一段时间,又到了蛾嵋山的伏虎寺,患病而终。他自己写了一幅偈子,说:“自己本是超脱尘世而与猿鹤为亲的老僧,无缘无故地堕于世俗的尘网中。妄想到如滚沸的油锅中去逃避炎热,哪里能使自己从尘世的苦海中去求得超脱?尘世中所追求的功名富贵,就像那被世人戏耍的木偶一样。娇妻爱子,也只不过是一堆枯骨中的人罢了。只是君王、父母恩未报,只有生生世世求佛祖保佑他们。”


\subsection{1.8.41   邵 士 梅}
\label{\detokenize{p00_u5176_u5b83/_u767d_u8bdd_u804a_u658b_u5fd7_u5f02:id344}}
进士邵士梅,是山东济宁人。初任山东登州府教授时,有两位老秀才前来拜见。邵士梅看他们名字,似乎很熟悉。回忆了好长时间,忽然醒悟到他前身的事情。便问学舍杂役:“某生是不是某村人啊?”又细说了他的相貌风度,都一一吻合。一会儿,两位秀才径直进来,邵士梅拉着他们的手倾谈,好像老朋友一样。谈话间,邵士梅问起高东海的情况。二位秀才说:“他已死在监狱里二十多年了,现在家中还有一个儿子。他只是乡间的平民百姓,您怎么也知道?”邵士梅笑着说:“他是我故旧亲戚。”

原先,高东海素以无赖闻名;然而为人却很豪爽,轻于财物,好义气。有个人因欠财主租子而被逼得出卖孩子,高东海倾囊帮助他,将孩子代赎回来。他与一婆子很要好,这位婆子因为成了盗贼的窝主,官府追捕她甚急。婆子逃到高东海家躲藏起来。官府得知实情后,将高东海捉了去,旋尽残酷的刑罚,他始终不服,很快就在监狱中死去。高东海死的那一天,正是邵士梅降生的日子。

后来,邵士梅亲自到高东海所在的村子里,抚恤他的妻子。事情传出去,乡里远远近近的人,都感到奇异。这个故事是高念东跟我谈的,邵士梅是高念东长子高冀良的同科进士。


\subsection{1.8.42   顾 生}
\label{\detokenize{p00_u5176_u5b83/_u767d_u8bdd_u804a_u658b_u5fd7_u5f02:id345}}
江南有个顾生,一次客住在济南府的一家客店里,眼睛突然肿了起来,疼得昼夜呻吟,各处求治都不见效。十多天后,疼得稍轻点了;可是每当他一合上眼时,总看到一座很大的宅院,有四五进院落,大门都敞开着,最里边的院子里有人来来往往,但远远的看不清楚。

一天,顾生又在聚精会神地看着,忽然觉得自己的身子进入宅院中。走了三道门,没看到一个人影。有一座南北大厅,里边红毡铺地。他偷偷一看,见满屋都是婴儿,有坐着的、躺着的、爬着的,不计其数。顾生正在惊愕,一个人从屋后过来,看见他说:“小王子说有远方来的客人到了,果然不错。”就邀请顾生进屋。顾生不敢进去,那人强拉着他往里走。顾生问:“这是什么地方?”那人说:“是九王世子住的地方。世子得疟疾刚刚痊愈,今日亲朋前来祝贺,你很有福气啊。”话没说完,有人跑来催促他们快点走。

一会来到一个地方,雕榭朱栏,一座殿堂坐南朝北,殿前有九根大柱子。顾生登上台阶进入大殿中,见已经坐满了宾客。有一少年面朝北坐着,顾生知道这就是王子了,就跪伏在堂下拜见。满堂的客人都站了起来。王子拉着顾生,让他面向东坐下。一会儿,摆上酒来,鼓乐齐奏,歌妓们来到堂上,演“华封祝”的戏文。刚演了三折,客店的主人和仆人喊顾生吃午饭,靠在他床头频频喊他。顾生听得非常清楚,心中害怕王子知道,就假托上厕所走出大殿来。抬头看看太阳,已是中午;又猛然见他的仆人站在床前,顾生这才醒悟,自己始终未离开客店。他急欲想返回王子的殿堂,急忙循原来的路进去,经过原先有婴儿的那座大厅,看到里边并没有婴儿,只有几十个老妇人蓬头驼背,在屋里或坐或躺。她们看见顾生,恶声恶气地说:“谁家的无赖子弟,来这里偷看!”顾生害怕,不敢辩解,急忙来到后庭。走上殿堂坐下,见王子颔下已长出了一尺多长的胡须。王子看见顾生笑着说:“你到哪里去了?戏已演过七折了。”就拿了大杯罚他喝酒。不多时,戏演完了,有人呈上戏单,顾生点了“彭祖娶妇”。歌妓们用椰瓢行酒,能盛五斗多。顾生站起来推辞说:“我眼睛有病不敢过量。”王子说:“患眼病,有太医在这里,让他给你诊治。”东边座上一个客人,便离开座位过来,两指撑开顾生的上下眼皮,用玉簪点进了一些白色的药屑,嘱咐顾生闭上眼稍睡一会儿。王子命侍从带顾生到里边屋里,让他躺下。顾生躺了一会儿,觉得床帐又香又软,就睡熟了。睡了不多时,忽然听到锣鼓乱响,还以为是戏没结束;睁眼一看,原来是客店中的狗在舔油锅。眼病却完全好了,再闭上眼,什么东西都看不到了。


\subsection{1.8.43   陈 锡 九}
\label{\detokenize{p00_u5176_u5b83/_u767d_u8bdd_u804a_u658b_u5fd7_u5f02:id346}}
陈锡九是江苏邳县人,他的父亲陈子言是本县的名士。本县大富翁周某很仰慕陈子言的声望,就和陈家订为儿女亲家。陈子言接连几次参加科举考试都没有考中举人,家业渐渐衰败下来。后来陈子言到秦地去游学,一去好几年没有音讯。

周某对跟陈家的婚约,暗暗感到后悔。他把小女儿嫁给王孝兼做了继室,王家送的聘礼非常丰盛,送聘礼的仆从、车马十分气派,周某因此越发憎恶陈锡九的贫寒,打定主意要断绝与陈家的婚约。他去询问大女儿,大女儿却坚决不同意退婚。周某大怒,给女儿穿戴上破旧的衣服首饰,把她送到了陈锡九家。

陈家穷得整天无法生火做饭,周某一点也不体恤照顾。有一天,周某派一个年老的女仆用食盒给女儿送了些食物去。这老婆子一进门就对陈锡九的母亲说:“我家主人叫我看看我家姑娘饿死了没有?”周女恐怕婆婆羞惭,勉强笑着说了些别的话叉开话题,接着就把食盒中的菜肴点心拿出来,放在婆母面前。老女仆忙阻止说:“不要这样!自从姑娘来到她家,哪里从她家换得过一杯白开水?我家的食物,料想老太太也没脸去吃。”陈母非常气愤,声音和脸色都变了。这老女仆还不服,用很难听的话来顶撞陈母。正在吵闹着,陈锡九从外边进来了,问清情况后非常愤怒,揪着老女仆的头发狠狠打她耳光,一边打着一边把她赶出门去。

第二天。周某来接女儿回家,周女不肯回去。明日又来了,而且增加了人数,七嘴八舌,吵吵嚷嚷,好像要寻衅打架。陈母劝周女回去,周女泪流满面地拜别婆母,上车走了。过了几天,周某又派人来,硬逼着索要一份离婚文书。陈母强迫陈锡九写了离婚书给了他们。母子二人只盼望着陈子言回家,再想别的办法来处理这件事。

周家有人从西安来,得知陈子言已经死了的消息。陈母又悲伤又气愤,得了病死了。陈锡九在悲伤窘迫中还希望妻子能回来。但过了很长时间,一点消息也没有,陈锡九越加悲伤愤怒。他把家里的几亩薄田卖掉,给母亲购置了办丧事的用具。办完了丧事,陈锡九就一路讨着饭前往陕西,寻找父亲的遗骨。

到了西安,访问遍了本地居民,有人说:“数年前有一位书生死在旅馆里,被埋葬在东郊,现在那座坟墓已经找不到了。”陈锡九实在没办法了,只好白天在街市上讨饭,晚上在野地寺庙里住宿,希望能遇见一个知道父亲情况的人。

一天晚上,他正经过一片乱葬岗子时,有几个人拦住了去路,逼着他要饭钱。陈锡九说:“我是一个外乡人,在城里城外讨饭,哪里会欠人家的饭钱?”这些人愤怒了,把他揪倒在地上,用埋死孩子的烂棉絮塞住他的嘴。陈锡九声嘶力竭,渐渐地快要被闷死了。忽然这些人一齐惊叫说:“哪里的官府的人来了!”立刻就放开了手,四周变得静悄悄的。一会儿有车马到了,有人便问道:“躺在那里的是什么人?”立即就有几个人把陈锡九扶到车边。车中的那个人说:“是我的儿子啊!恶鬼怎能这样对待他!应当把他们全都捆来,不要漏掉一个。”陈锡九觉得有人去掉了他嘴里的烂棉絮。他稍微定了定神,仔细辨认了一下,车中人果然是父亲,不禁大哭着说:“儿子为了寻找父亲的尸骨受尽了苦难,没想到您如今仍然活在人间啊。”父亲说;“我不是生人,是阴世间的太行总管。这次来也是为了孩子你。”陈锡九哭得更加哀痛了,父亲劝慰开导他。陈锡九哭着述说岳父家强逼离婚的事。父亲说:“不必担忧,现在你媳妇也在你母亲那儿。你母亲非常想念你,你可以暂时去看一看。”于是就和锡九同坐一辆车,奔驰得像风般快速。

过了一会儿,到了一座衙门前,下了车穿过几道门,果然陈母在那里。陈锡九痛哭得快要晕过去了,父亲劝止他,陈锡九啜泣着答应了。他看见妻子在母亲身边,就问母亲说:“我媳妇也在这里,莫非她也成了九泉之下的人了?”母亲说:“不是,是你父亲接来的,等到你回家的时候,还要把她送回去。”陈锡九说:“儿子侍奉父母,不愿意回去了。”母亲说:“你辛辛苦苦跋山涉水来到这里,是为了寻求你父亲的遗骨。你不回去,那么当初你立志是为了什么呢?况且你的孝行上帝已经知道了,赏赐给你白银万斤,你夫妻享福的日子还很长久,为什么说不回去呢?”陈锡九低头哭泣。父亲几次催促他动身,锡九痛哭失声。父亲生气地说:“你还不动身吗!”锡九害怕了,这才停止了痛哭,询问父亲埋葬的地方。父亲拉着他的手臂说:“你动身吧,我告诉你:离那个乱葬岗一百多步的地方,有一大一小两棵白榆树,就是我埋骨之处。”父亲拉着他走得很急,竟没有来得及向母亲告别。门外有一个身体健壮的仆人,拉着马在等着他。陈锡九上马之后,父亲又嘱咐他说:“你平日睡觉的地方,有一点钱,可以赶快置办行装回去,向你岳父追要你媳妇,不得到你媳妇,决不要罢休。”陈锡九答应着走了。马奔跑得非常快,鸡叫的时候,已经到了西安。仆人把他扶下来,他刚要拜托仆人向父母问候,那仆人和马已经杳然无踪了。

陈锡九找到从前住宿的地方,倚着墙壁闭上眼睛休息,等待天亮。他觉着坐着的地方有块拳头大的石头硌着屁股,天亮后一看,原来是一块银子。他买了棺木赁了车,寻找到那两稞榆树之下,得到了父亲的遗骨,就回乡了。他把父母的遗骨合葬之后,家里穷得只有四堵墙壁了。幸亏乡亲们同情敬重他的孝行,都给他饭吃。陈锡九准备到岳父家去索回媳妇,自己考虑一下不能用武,就约本家哥哥陈十九一起去。到了周家大门口,守门的拒绝给他们通报。陈十九本是个无赖,骂出的话污秽不堪。周某只好派人劝陈锡九回家,愿意立即把女儿送去,陈锡九这才回家。

当初,周女刚回到娘家时,周某当着她的面辱骂陈锡九和他的母亲。周女不说话,只是面朝着墙壁流泪。陈锡九的母亲死了,周家也不让她知道。周某得到离婚书,向女儿面前一扔说: “陈家已经休了你了!”周女说:“我从不凶悍忤逆,为什么休我?”想要回婆家质问一下原因,周某又把她关了起来。后来陈锡九到西安去了,于是周某就伪造陈锡九死了的消息,以断绝女儿的心志。这个凶信一传播出去,杜中翰家里便来人商议向周女说亲,周某竟然答应了,快到迎亲的日子,周女才知道这件事。于是她哭泣,不肯吃饭,用被子蒙着脸,气如游丝,奄奄一息。周某正束手无策,忽然听说陈锡九找上门来,说话很不客气,他估计女儿必死,于是就派人抬着送回陈锡九家,打算等到女儿死了,就以此作为要挟,发泄自己的愤恨。

陈锡九回到家,送周女的人也到了,他们还恐怕陈锡九见周女病了不肯收留,刚一进门,放下就走了。邻居们都替陈锡九担忧,一起商议着抬着送回去。陈锡九不同意,扶着周女安置到床上,这时她就断了气。陈锡九这才感到很害怕,正惊慌失措的时候,周某之子领着好几个人,手持凶器闯了进来,把门窗都砸毁了。陈锡九逃走躲了起来,周家的人苦苦搜索他。乡亲们都为陈锡九感到不平。陈十九纠集了十几个人挺身而出打抱不平,周家子弟都被打伤,这才抱头鼠窜。周某越发愤怒,就向官府告状,要求逮捕陈锡九和陈十九等人。锡九准备逃走,把周女的尸首托邻居老大娘照看。忽然听见床上好像有喘息的声音,走近一看,妻子的眼睛微微转动了。又过了一会儿,已经能够转动侧身了。陈锡九大喜,就亲自到官府去说明了情况。县令对周某的诬告十分恼怒。周某害怕了,送给县令一笔很重的贿赂,才免于治罪。锡九回到家里,夫妻相见,悲喜交集。

在这以前,周女奄奄一息地躺着,自己发誓一定要死。忽然有人把她拉起来说:“我是陈家的人,赶快跟着我去,夫妻可以相见;不然,就来不及了!”周女不知不觉地身子已来到门外,有两个人扶着她上了轿子,顷刻之间来到了一座官署之中,看见公公婆婆都在这里,周女就问道:“这是什么地方?”婆母说:“不必问,不久就会送你回去。”又一天,看见陈锡九也来了,她十分高兴,可是见面不久就匆匆分别了,心里觉得十分奇怪。公公不知为了什么事,常常好几天不回来。昨天晚上忽然回来说:“我在武夷山中耽搁了,迟回来了两天,难为锡九这孩子了。可要赶快送媳妇回去了。”于是用车马送周女动身。周女忽然看见了陈家的大门,就像做了一场梦一样醒过来了。周女与锡九共同回述往事,都感到又惊又喜。

从此夫妻团聚,但每日生活都无法自给。陈锡九在村中开设了私塾,同时自己刻苦攻读。他常常私下里念叨:“父亲对我说:老天爷要赐给我黄金,现在我家除了四堵墙之外,一无所有,难道靠教书能发家致富吗?”

有一天,陈锡九从私塾中回来,遇见两人个,问他说;“先生是陈锡九吗?”锡九回答说:“是的。”那两个人就掏出锁链锁住他。锡九也不知是为了什么事。过了一会儿,村里人都聚集过来,一齐问那两人,才知被郡里的强盗所牵连。众人同情锡九冤枉,就凑钱贿赂差役,因此,押解途中他没有吃苦。到了府城见了太守,详细地叙述了自己的家世。太守很惊讶地说:“这是名士的儿子,温和有礼,举止斯文,怎么会做贼!”就命令解去绳索。从牢里捉出强盗严刑审问,强盗才供出是周某贿买他诬陷陈锡九。陈锡九又诉说岳父与他结仇的原因,太守更加愤怒,立刻命人拘押周某。太守请陈锡九到后衙中,与他谈论起先辈的交情。原来太守是从前的邳县知县韩公的儿子,也是跟着陈子言学习过的学生。于是太守就赠给他百两银子作为求学的费用,又赠给他两头骡子当坐骑,使他能常到府城来,以便考核文章。太守又对各位上司宣扬陈锡九的孝行,自总督以下各官员对锡九都有馈赠。锡九骑着骡子回到家中,夫妻都感到很欣慰。

有一天,陈锡九的岳母哭着来了,见了女儿就伏在地下不肯起来。周女惊骇地问她,才知道周某已经被枷铐起来,押在狱中了。周女哭着责备自己,只想去寻死。陈锡九不得已,就到府城去为周某说情。太守释放了周某并令他自己赎罪,罚他一百石谷子,又批示赐给孝子陈锡九。周某被放回来以后,拿出仓里的谷子,掺上一些糠秕后用车子送到锡九家,陈锡九对妻子说:“你父亲是以小人之心度君子之腹。怎么知道我一定会接受而不怕麻烦地掺进一些糠秕去呢?”就笑着把谷子退了回去。

陈锡九家里虽然小康了,但院墙仍然破败。一天夜间,群盗摸了进来。仆人觉察后,大声呼叫,强盗只偷了两头骡子去。过了半年多,陈锡九有一天晚上正在读书,听到敲门的声音,问了问却没有回答,就喊仆人起来去看看。门才一开,两头骡子窜了进来,原来正是以前被偷走的那两头。骡子直奔牲口栅中,全身淌汗,咻咻地喘着。点上蜡烛照着一看,两头骡子各驮着一个皮口袋。解开袋口一看,里面装满了白银。锡九心中十分惊奇,不知两头骡子是从哪里跑来的。后来听说,这天晚上强盗抢劫了周家,装得满满的离开了。正碰上巡逻的士兵,追得很急,强盗就扔掉抢来的东西逃走了。骡子认得旧主人的家,就一直跑回家来了。周某从狱中放回后,受刑的创伤还很重,又遭了强盗抢劫,生了一场大病死了。

一天夜里,周女梦见父亲带着枷锁来了,说:“我一辈子的所作所为,后悔也来不及了。如今在阴间受到惩罚,非你公公不能帮助我解脱。你替我求求女婿,写封信给他父亲。”周女醒了后还伤心地哭泣,锡九问她,她把梦中的情景都告诉了丈夫。陈锡九早就想到太行去一趟,于是当天就出发了。到了以后,准备了三牲祭品,酹酒祭奠之后,就露宿在那里,希望能见到父亲,可是一夜都没有什么怪异之事,于是就回家了。

周某死了以后,妻子和儿子更加贫困,依靠二女婿养活。王孝廉考试候补当了县官,因贪污受贿被罢官,全家被发配到沈阳去了。周家母子越发无依无靠了,陈锡九就常常资助周济他们。


\section{1.9   卷 九}
\label{\detokenize{p00_u5176_u5b83/_u767d_u8bdd_u804a_u658b_u5fd7_u5f02:id347}}

\subsection{1.9.1   邵 临 淄}
\label{\detokenize{p00_u5176_u5b83/_u767d_u8bdd_u804a_u658b_u5fd7_u5f02:id348}}
临淄有个老头,女儿是太学生李某的妻子。还没出嫁时,有个算卦先生给她算命,说她将来定受官府刑罚。老头听后大怒,既而笑着说:“怎么胡说到这种地步!先不说大户人家的女子必定不会涉足公堂;难道凭着一个监生还不能庇护自己的妻子吗?”

女儿嫁给李某后,非常凶悍,对丈夫动辄大骂,习以为常。李某忍受不了她的虐待,气愤地告到官府。县官邵大人准了他的诉状,立刻发签拘捕审理。老头听说后,十分震惊,带领子弟赶到县衙,哀求邵大人销了此案。邵大人不答应。李某此时也感到后悔,也去恳求撤拆。邵大人发怒说:“官府内的事,难道办与不办都依着你吗?一定要拘审!”衙役把李某的妻子带到公堂,邵大人只问了几句,便说:“真是个凶悍的泼妇!”命令衙役重打三十大板,打得她腚上的肉都掉了下来。


\subsection{1.9.2   于 去 恶}
\label{\detokenize{p00_u5176_u5b83/_u767d_u8bdd_u804a_u658b_u5fd7_u5f02:id349}}
北平陶圣俞,名叫下士。顺治年间,他去赴乡试,住在省城郊外一家旅店里。

这一天,他偶然出来散步,见一个人背着书箱在路上徘徊,样子像找不到地方住。陶生就上前与他搭话,那人放下书箱与他攀谈。说话当中,陶生见那人很有名士风度,心里非常高兴,就请那人与自已同住一个旅店;那人也很同意,便进了旅店住在一起。那人自我介绍说:“我是顺天府人,姓于,字去恶。”因陶生年纪稍长一点,于是就叫他兄长。

于去恶性情喜静不喜动,常一人独坐在屋里,但他的桌子上又不见书籍。陶生不与他说话,他也不做声,就一个人默默地躺着。陶生觉得这人很奇怪,便看他书箱里有啥东西;但里面除了笔墨纸砚,其它什么东西也没有。陶圣愈感到很奇怪,因此就问于去恶,于笑着说:“我们读书人,哪能临渴掘井?”

一天,于去恶向陶生借了本书,自己关上门抄书,抄得非常快,一天抄五十多页,抄了后又不见他装订成册。陶生纳闷,就偷偷瞅他,见他每抄一页就烧一页,烧成的灰一口吃了。陶生越发觉得奇怪,于是便问他,于回答说:“我这是以吃代读罢了。”接着他就背诵所抄的书,一会儿功夫背了好几篇,并且一字不差。陶生十分高兴,要求于去恶传授这种方法,于说不行。陶生认为于太保守,不够朋友,就说话刺他。于去恶说:“老兄你太不谅解我了,有些事想不对你说,我自己也解释不清楚,可是骤然与你说了,又怕吓你一跳,这怎么办?”陶生一再请求说:“你说吧!不妨事。”于这才说道:“我不是人,而是鬼。现在阴曹中以考试任命官吏,七月十四日奉命考核考官;十五日应考的士子入场,月底张榜揭晓。”陶生又问:“考核考官干什么?”于说:“上帝为了慎重起见,对无论什么样的官吏,都得要进行考试。凡文采好的便录用为考试官,文理不通的就不录用了。因为阴曹中也有各种各样的神,就像人间有太守、县令一样。得志的人,便不再读古籍经史,他们只是以古籍当敲门砖以求取功名罢了。一旦敲开门,当上官,就全丢了;如果再掌管文书十几年就能当上文学士了,胸中哪还能留下几个字!人间之所以无才的人能当上官,而有才的人却当不上官,就是因为少者这一考试啊。”陶生听了,认为于说得很对。从此,越发对于敬重了。

一天,于去恶从外面回来,面带愁容,叹了口气说:“我活着的时候就贫贱,自已本以为死后可以免于贫贱了,不料倒霉先生又跟我到了阴间。”陶生问他是怎么回事,于去恶说:“文昌星奉命去都罗国封王,考官的考试他暂不参加了。几十年的游神、耗鬼,都夹杂在考试官里,我们还有什么希望?”陶生问:“那些人都是些什么样的人?”于说: “就是说出来,你也不认识。只说一二人,你可能知道。譬如说乐正官师旷、司库官和峤就是那样的人。我自己想:一不能听命运摆布,二不能依仗文才进取,别又没有出入,还不如就此罢了。”说罢怏怏不乐,便整理行装要走。陶生一再挽留并诚恳地安慰他,于才又住了下来。

到了七月十五日的晚上,于去恶忽然对陶生说:“我要去考试了,请你黎明时,到东郊去烧上柱香,连叫我三声去恶,我就来相见。”说完就出门走了。陶生准备了酒、菜,等他回来。东方天亮时,陶生就去东郊烧了香,叫了三声去恶。不一会儿果然于去恶回来了,还领了一个少年来。陶问少年是谁,于去恶说:“这位是方子晋,我的好朋友,刚才在考场碰到,听见你的大名,很想认识一下,交个朋友。”于是他们三人一起到了住处,掌上灯,见了礼。这个少年风流潇洒,态度非常谦逊。陶生对他十分尊敬,便问:“子晋的大作,一定非常满意吧?”于说:“说来可笑,场上出了七道题,子晋已作了一半了,一下看到主考官的姓名,包起东西就退出考场,真是个奇人!” 陶生一面在炉子烧酒,一面问:“考场出的什么题?于兄定能考个一二名吧?”于去恶说:“以四书命题的八股文一篇,以五经命题的八股文一篇,这个什么人也能写;策问文体中有这样几句:‘自古以来,邪气固然很多。到了今天,奸邪之情,丑恶之态,却越来越多得不计其数;不用说十八层地狱不能都用上,就是都用上也容不下这些罪人,到底有什么办法呢?有的说再增加一二层地狱,然而这样太违背了上帝的好生之心。到底是增加地狱还是不增加?或是还有别的办法能堵住犯罪根源,你们可以提出建议,不要隐讳。’小弟对上述策问,答得虽不够好,但却是非常痛快。还有拟表:‘拟天魔殄灭,赐群臣龙马天衣有差’再就还有‘瑶台应制诗’、‘西池桃花赋’这三种。我自认为考场上无人能与我相比。”说罢鼓掌。方生笑着说:“这时的快乐心情,只是你自己感觉如此罢了;过几个时辰后不痛哭,才算真正男子汉。”

天明后,方生要告辞回去。陶生留他住下,方生不同意,陶生就要求他晚上回来。以后,方生一连三天竟没有来。陶生托于去恶去找方生。于生说:“不必去找,子晋很诚实,一定是有什么事,不然他绝对不会故意不来。”

太阳快落时,方生来了,拿出一卷稿子给陶生,对他说:“三天没有来,我失约了。我抄了旧诗百余首,请你欣赏。”陶生接到手里,非常高兴,马上捧读,读一句赞一声,约读了一二首,就珍藏在自己的书箱里。当晚,他们谈话谈到深夜,方生便留下与陶生一起睡下。自此以后,方生没有一晚上不来,而陶生也是一晚上不见方生,便睡不着觉,他俩亲热异常。

一天晚上,方生忽然怆惶进屋,对陶生说:“阴曹的地榜已接晓,于兄落第了!”于去恶正睡间,听到这话,立刻起来,十分痛苦,满脸是泪。陶、方二人极力劝他,安慰他,于生才止住了泪水。然而三人都心里难过,相对无语。待了一会,方生才说:“听说张桓候要来巡视,我想这可能是不得志的人造谣;若是真的话,这次考试可能有反复。”于去恶听说,脸上出现喜色。陶生问他为什么又高兴,于说:“桓侯张翼德,三十年巡视一次阴曹,三十五年巡视一次阳间,两世间的不平之事,等他老来解决。”接着起身拉着方生一起走了。

隔了两夜,于、方二人又回来。方生对陶生说:“你不祝贺一下于兄吗?桓候前天晚上来,扯碎了地榜,榜上的名字,只留下三分之一。桓候逐个看了一遍余下的考卷,见到于兄的考卷很赞赏,推荐于兄任交南巡海使,很快就来车马接于兄上任。”陶生听了十分高兴,马上摆了酒席庆贺。酒过数巡,于问陶生:“你家里有多余的房子吗?”陶生问: “你要做什么?”于说;“子晋孤单一人,没有家,他又不忍心老麻烦你,所以我要借你的房子与他相依为命。”陶生非常同意,说:“这太好了。就是没有房子,咱们同床共寝又有何妨!但是家里还有父亲,必须先向他说一声。”于说:“早知道你父亲仁慈宽厚,十分可信,你马上就要应考了,子晋如不等在这里,就先回去怎么样?”陶生留他们一起住在旅店里。等自己考完了试,大家一块回家。

第二天,太阳刚落山,就有大队车马来到门口,说是迎接于去恶去上任的。于起来向陶、方二人握手话别。对他二人说:“我们要分别了,我有一句话要说,又担心这话会给你泼冷水。” 问:“有什么话?”于说:“陶兄命运不好,生不逢时,这一科考中的可能性只有十分之一;下一科,桓侯巡视人间,公道可能分明些,但成功的可能性也只有十分之三;再一科考试,可望成功。”陶生听后,觉得这科没有什么希望,就想干脆不考了。于去恶说:“这不行,这是天数,就是明知考不上,也要经历一下这命中注定的艰苦。”接着他又对方生说:“不要再久留于此,今天是个好日子,我马上用车送你回去,我自己骑马去上任。”方生欣然同意,拜别而去。陶生心中迷乱,不知怎么是好,只是哭着送他二人走。遥望车、马分道而去,陶生心里十分空虚。稍镇静了一下,才后悔子晋北去他家,没有向他交待一句话,可现在已经来不及了。

陶生三场考下来,考得不够满意,一路奔波回了家。进门就问方子晋是不是来了,可是家里的人没有一个知道方子晋的。他便向他父亲详细说了在外面碰到的情况。父亲高兴地说:“若是这样的话,那客人早就来了。”原来在陶生未回家前,陶公白天睡觉,梦见一辆马车停在门前,一个美少年从车子里出来,到堂上来拜见。陶公问他从哪里来,少年回答说;“大哥允许借我一间屋住,因为大哥没考完试,所以我先来了。”说罢,要求进内房拜见母亲。陶公正推辞时,家中老佣人来报告说;“夫人生了个小公子。”陶公恍然醒来,觉得十分奇怪。今天陶生所说,正好与梦相符。才知到小儿就是方子晋来投胎托生的。陶氏父子非常喜欢这孩子,给起了个名字叫小晋。

小晋刚生下来,半夜里好哭,母亲非常苦恼。陶生说:“他若是子晋,我见了他,他就不哭了。”可是当时有旧风俗,刚生下来的孩子不能见生人!所以没有让他们相见。后来,因孩子哭得实在不能叫大人忍受了,才叫陶生进屋看他。陶生对孩子说:“子晋不要哭,我回来了。”小孩正哭着,听到陶生说话,马上就止住了哭声,直瞪着眼看陶生,像在辨认他一样。陶生用手摸了一下他的头顶,就出去了。

自从陶生去看了小孩儿以后。孩子再也不哭了。过了半月,陶生就不大敢见他了;因为一见他,小孩就非要陶生抱着不行;不抱,就哭个没完。陶生也越来越喜欢他。小晋长到四岁,就离开母亲跟陶生一块睡。陶生出去有事,他就装作睡着了,一直等陶生回来。每天陶生都在床头上教他读《毛诗》,诵诗的声音呢呢喃喃,一晚上背会四十行。拿原来方子晋的诗教给他,他非常乐意读,一读就能记住。再试其它诗文,他就记不住了。八九岁时,长得眉眼明亮,很像方子晋的模样。

后来,陶生两次参加考试,都没有考中。丁酉年,考场作弊事件被揭发,考试官大多数诛杀或贬职,考试作弊的事得到肃清,原来是张桓侯下界巡视的结果。陶生下一科中了副榜,接着成为贡生。陶生此时对前程已灰心,便隐居乡间,一心一意教小弟弟读书。经常对人说:“我有现在这样的快乐,当官也不换。”


\subsection{1.9.3   狂 生}
\label{\detokenize{p00_u5176_u5b83/_u767d_u8bdd_u804a_u658b_u5fd7_u5f02:id350}}
刘学师说:济宁有个行为狂放的书生,性好饮酒,家里穷得从来余不下一斗米,然而只要一得到钱就买酒喝,根本不把穷困放在心上。这时正遇上新刺史到济宁上任,这位刺史很能喝酒,但没有对手。听说狂生能喝酒,就招他来一起共饮,十分喜欢他。以后刺史就时常找狂生谈笑对饮。狂生倚仗着与刺史关系亲密,凡有打小官司想求得胜诉的,他就接受点贿赂,为他们去说情。刺史常常答应他的请求。狂生习以为常了,刺史心里就讨厌他了。

一天早上,刺史升堂处理公务,狂生拿着个条子来到堂上。刺史看着条子只是微笑,狂生厉声喝道:“大人同意我的请求,就答应;不同意我的请求,就否定它。何必笑呢!我听说,士可杀而不可辱。其它的事我固然无法报复,难道笑一笑也不能报复吗!”说完了就放声大笑,笑声震荡着大堂四壁。刺史大怒说:“你怎么能这样无礼!你没听说过‘灭门令尹’这样的话吗?”狂生竟然一甩胳膊走了,还大声喊道:“小生无门可灭!”刺史更加愤怒,就把他抓了起来。后来打听他的家庭情况,原来他并没有田产宅第,只带着妻子在城墙上住。刺史听到这种情况,就把他释放了,只下令驱逐他,不让他在城墙上住。朋友们很同情他的狂放行径,给他买了一小块地,买了一间小屋。狂生搬过去住下,叹息道:“从今以后可就害怕灭门令尹了!”


\subsection{1.9.4   澂 俗}
\label{\detokenize{p00_u5176_u5b83/_u767d_u8bdd_u804a_u658b_u5fd7_u5f02:id351}}
澄海地方的人,能变化成多种动物,跑出院子寻求食物。有个客商刚到这里时,住在旅店,常看到一群老鼠钻进米罐中,赶它们,则马上逃走。客商守在一旁,见它们又进去后,急忙用东西盖住罐口,拿瓢子舀水灌到里边。一会儿,老鼠全被淹死了。这时,客商发现店主全家人突然死去,只剩下一个孩子。客商被告到官府,县官审知实情后,宽恕了他。


\subsection{1.9.5   凤 仙}
\label{\detokenize{p00_u5176_u5b83/_u767d_u8bdd_u804a_u658b_u5fd7_u5f02:id352}}
刘赤水是平乐县人,从小聪明俊秀。十五岁便考入府学读书。因为父母早早去世,他天天游荡,放纵,荒废了学业。他的家产还不到中等人家的水平,但他天性爱好修饰打扮。连家里的被褥家具都十分精致华丽。

一天晚上,刘赤水被人请去喝酒,忘记把蜡烛熄灭就走了。等喝过了几巡酒后,他才想起了这件事,急急忙忙返回家中。忽然听到屋内有人小声说话,他俯身偷偷向里一看,只见一个少年拥抱着一个漂亮姑娘躺在床上。刘赤水的家就靠着一所权贵人家荒废的宅第,宅第中常有怪异的事,所以他心里知道这对男女是狐狸,也不害怕,闯进去喝道:“我的床上岂能容别人睡觉!”那两人惊慌失措,抱起衣服光着身子逃走了;却丢掉了一条紫色的绢裤,裤带上还系着一个针线荷包。刘赤水心中大喜,但又恐怕他们偷回去,就藏在被子中紧紧抱住。一会儿,一个头发蓬松的丫鬟从门逢中进来了,向刘赤水讨要丢失的东西。刘赤水笑着索要报酬,丫鬟答应送给他酒,刘赤水不答应;丫鬟又说赠给他金子,他也不答应。丫鬟笑了笑就走了。接着又返回来说:“我家大姑说:你如果赐还东西,一定给你找个漂亮的妻子作为报答。”刘赤水问道: “你家大姑是谁?”丫鬟答道:“我家姓皮,大姑小名叫八仙,和她睡在一起的是胡郎。二姑水仙嫁给了富川县的丁官人。三姑凤仙比那二位姑娘更漂亮,从来没有看见她而不满意的。”刘赤水恐怕她不守信用,就要求坐在这儿等候消息。丫鬟去了一会儿又回来说:“大姑叫我告诉先生:好事怎么能一下子就办成呢?刚才跟三姑说了这件事,遭到她的斥骂。只要缓几天等待着,我们家不是轻易许诺而不守信的人家。”刘赤水就把东西还给了她。

过了好几天,一点消息也没有。一天傍晚,刘赤水从外边回家,关上门刚刚坐下,忽然两扇门自动开了,有两个人手提着一床被子的四个角,兜着个女郎进来了,说:“送新娘来了!”笑着放到床上就走了。刘赤水走近一看,女郎酣睡未醒,还散发着芳香的酒气,红红的脸儿带着醉态,娇美的容貌可以倾倒世间所有的人。刘赤水高兴极了,替她抬起脚来脱去袜子,抱着她的身子轻轻脱去衣服。这时女郎已经稍微有些清醒了,睁开眼睛看着刘赤水,但四肢仍不能随意活动,只恨恨地说:“八仙这个浪丫头出卖了我!”刘赤水拥抱着她亲热。女郎嫌他皮肤冰凉,微笑着说:“今夕何夕,见此凉人!”刘赤水说:“子兮子兮,如此凉人何!”于是互相欢爱起来。过了一会儿,凤仙说:“八仙这个丫头真不害羞,玷污了人家的床褥,却用我来换她的裤子!我一定好好地报复她一下!”从此凤仙没有一天晚上不来,两个人盛情缠绵,十分亲热。

一天,凤仙从袖子中取出一枚金钏说:“这是八仙的东西。”又过了几天,凤仙怀里揣着一双绣鞋来了。绣鞋嵌着珍珠,用金线绣着花纹,制作精巧极了,凤仙嘱咐刘赤水拿出去宣扬。刘赤水就拿着绣鞋在新朋中夸耀,要求观看的人都用钱、酒作为礼物,从此刘赤水就把绣鞋当作奇货珍藏着。一天晚上,凤仙来了,说了些别离的话,刘赤水很奇怪,就问她,凤仙回答说:“姐姐因为绣鞋的缘故怨恨我,想带着全家远远地离开这里,隔绝我和你相好。”刘赤水害怕了,情愿把鞋还给她。凤仙说:“不必还她,她用这个方法要挟我,如果还给她,正中了她的计谋了。”刘赤水问:“你为什么不独自留下来?”凤仙说:“父母远去,一家十余口都托付给胡郎照顾,如果不跟随去,恐怕八仙这个长舌妇会给我造谣生事。”从此凤仙就不再来了。

过了两年,刘赤水十分思念凤仙。有一天,他在路上遇见一个姑娘,骑着马慢慢走着,一个老仆人拉着马缰绳牵着马,和他擦肩而过。那女郎回头掀起面纱偷偷看他,丰满的姿容美丽极了。不一会儿,一个少年从后边走过来,问他道:“这个女子是什么人?好像挺漂亮的。”刘赤水赞美不止。少年向他拱手致礼,笑着说:“太过奖了,那就是我的妻子。”刘赤水惶恐惭愧地向他表示歉意。那位少年说:“没有关系。但是南阳诸葛三兄弟中,你得到了其中那位卧龙,其余的两个小人物又哪值得称赞呢?”刘赤水对他的话感到诧异,少年对他说:“你不认识曾经偷着睡在你床上的人了吗?”刘赤水这才明白他就是胡郎。于是互相叙起连襟之谊,谈笑得十分欢畅。胡郎说: “岳父母刚刚回来,我们要去拜见,你愿意一起去吗?”刘赤水十分高兴,就跟着他们进入萦山。山上有本地人过去躲避战乱时居住的宅第,胡郎下马进去了。一会儿,好几个人出来看,说道:“刘官人也来了。”两个进了门,拜见了岳父母。另有一位少年已经先在那儿了,靴袍华美,光彩耀目。岳父介绍说:“这是富川县姓丁的女婿。”他们互相见礼后备自就坐。一会儿,酒茶纷纷端上来,大家互相谈笑,十分融洽。岳父说:“今天三位女婿一齐来了,可说是难得聚会,又没有外人,叫女儿们出来吧,大家团聚一次。”不一会姊妹们都出来了。老人吩咐摆上座位,各靠着自己的女婿坐下。八仙见到了刘赤水,只是掩着嘴笑,凤仙就和她互相开玩笑;水仙的容貌差一点,但是稳重温婉,满座的人都在热烈谈笑,她却只端着酒微笑而已。于是靴鞋交错,兰麝香气熏人,大家喝得十分高兴。刘赤水看见床头上摆着各种乐器,于是拿起一只玉笛,请求允许他吹一曲为岳父祝寿。老翁很高兴,就叫擅长乐器的人各自都献一项技艺。于是满座的人争着去拿乐器,只有丁婿和凤仙不去拿。八仙说:“丁郎不熟悉音律,可以不拿;你难道是手指弯曲伸不开的人吗?”说着,便把拍板扔到凤仙怀中。于是大家便络绎不绝地奏起了各种曲子。老翁非常高兴地说:“天伦之乐好极了!你们姊妹几个都能歌善舞,何不各自尽力表演自己擅长的技艺?”八仙站起来拉着水仙说:“凤仙从来都把她的歌喉看得比金子还珍贵,不敢劳动她的大驾,我们两个人可以合唱一曲《洛妃》。”两人的歌舞刚刚结束,正好有个婢女用金盘端着水果进来,大家都不知道这种水果叫什么名字。老翁说:“这是从真腊国带来的水果,叫‘田婆罗’。”顺手抓了几个送到丁婿面前。凤仙很不高兴地说:“对女婿难道因贫富不同就爱憎不同吗?”老翁有点不高兴,却没有说什么。八仙说:“爹因为丁郎是异县人,所以算是客人。若按长幼论,难道只有凤妹妹有个拳头大的酸女婿吗?”凤仙始终很不高兴,脱去了华美的衣服,把鼓拍交给婢女,唱了一折《破窑》,声泪俱下。唱完以后,一甩袖子就走了,满座的人都为此不高兴。八仙说:“这个丫头的任性和过去一模一样。”就去追凤仙去,不知到哪里去了。刘赤水感到很丢脸,也告辞了回去。到了半路上,看见凤仙坐在路旁,凤仙招呼他坐在自已身旁,对他说:“你也是一个男子汉大丈夫,难道就不能为妻子争一口气吗?功名富贵都在书中,希望你自己好好努力!”抬起脚来说:“匆匆忙忙出门,荆棘刺破了我的鞋子。以前给你的东西,带在身边没有?”刘赤水拿出绣鞋,凤仙拿过来换上。刘赤水请求把旧鞋给他,凤仙微笑着说:“郎君也是个大无赖!哪里见过自己妻子的东西还藏在怀里的人?如果你爱我,我有一件东西可以送给你。”立刻拿出一面镜子交给他说:“你想见我,应当从书卷中寻找;不然的话,再要想见面就没有日子了。”说完了话,就不见了。刘赤水十分惆怅地回到家中。拿出镜子看看,见凤仙背着身子站在镜中,好像望着相距百步之外的人那样。因而想起了凤仙的嘱咐,就谢绝宾客,闭门苦读。

有一天,刘赤水看见镜中的凤仙忽然现出正面,脸上充满了笑意,因而越发珍爱这面镜子。没有人的时候,就和镜中的凤仙互相对望着。过了一个多月,发愤读书的志向逐渐衰退了,游玩起来常常忘了回家。回到家中一看,镜中凤仙的影子,面容悲伤好像要哭的样子;隔了一天再看,又背面而立,像开始时那样了。这才明白是因为自己荒废了学业。于是就闭门苦读,昼夜不停。过了一个多月,凤仙的影子又面向外了。从此刘赤水就用这面镜子来检验自己的学业:每当荒废了学业,镜中人的面容就悲伤;刻苦攻读几天,镜中人的面貌就微笑。于是他把镜子日夜悬在面前,好像面对着老师一样。刘赤水这样苦读了二年,就一举考中了举人,他欣喜地说:“现在可以对得起我的凤仙了!”拿过镜子来,只见凤仙黛色的眉毛又弯又长,雪白的牙齿微微露着,笑容可掬,好像就站在自己面前。刘赤水心里爱极了,不转眼珠地长久凝视着。忽然镜子中的凤仙笑着说:“‘影子里的情郎,图画中的爱人’,就是说的今天这种情景吧。”刘赤水惊喜地向外看看,原来凤仙已经站在他的身边了。他握住凤仙的手,问候岳父岳母的情况。凤仙说:“我自从和你分别之后,就没有回家,藏在附近的山洞里,以此来分担你的辛苦。”

刘赤水到府城去赴宴,凤仙请求和他同去,两人同坐一辆车去赴宴,别人对面也看不见她。宴会结束后将要回去的时候,凤仙私下里与刘赤水商议,她假作是刘赤水在郡中的媳妇。凤仙回来以后,才开始出来见客人,经手管理家务。人们都惊讶她的美貌,而不知她是狐狸。

刘赤水是富川县令的学生,有一次他去看望老师,遇见了丁生。丁生热情地邀请刘赤水到他家里去,招待得优厚周到,并说:“岳父母最近又迁居到别的地方了。我妻子回家探亲,快回来了。我一定寄一封信告诉他们你高中的喜讯,和他们一起去拜访祝贺。”刘赤水当初怀疑丁生也是狐狸,等到仔细询问了他的家世,才知道他是富川县大商人的儿子。

当初,丁生有一次晚上从别墅回家,遇见水仙在独自赶路。丁生见她生得很美,偷偷地瞧她,水仙就要求跟着他一同赶路。丁生十分高兴,就把她带回自己书房里,与她同居了。水仙能从窗棂缝隙中出入,丁生才知道她是狐狸。水仙对他说:“郎君不必怀疑我,我因为你忠厚老实,所以才愿意嫁给你。”丁生宠爱她,竟不再娶亲。

刘赤水回家以后,借隔壁权贵家荒废了的大宅子,准备给来祝贺的客人住宿。房子打扫得十分整洁,只苦于没有帐幔可用。隔了一夜再去看时,屋里的陈设焕然一新了。过了几天,果然有三十多个人,带着酒礼等物来了,车马络绎不绝,挤满了街道小巷。刘赤水行礼让岳父及丁、胡进入客舍,凤仙迎接母亲及两位姐姐到内室里。八仙说:“小丫头你现在富贵了,不怨我这个大媒人了吧?我的金钏和绣鞋还在吗?”凤仙找出来给了八仙,说道:“绣鞋还是那双绣鞋,不过已被千万人看破了。”八仙用绣鞋拍打着凤仙的背说:“打你寄在刘郎身上。”于是把绣鞋扔到火里,祝告说:“新时如花开,旧时如花谢;珍重不曾着,妲娥来相借。”水仙也接着祝告说:“曾经笼玉笋,着出万人称;若使姐娥见,应怜太瘦生。”凤仙拨着火说:“夜夜上青天,一朝去所欢;留得纤纤影,遍与世人看。”于是就把烧成的灰捏在盘子中,分堆成十几份,望见刘赤水来了,托着盘子送给他。只见满盘都是绣鞋,都和原来那双的样式一样。八仙急忙赶出来,把盘子推跌到地上,地上还有一二只绣鞋在那里;八仙又伏在地上吹它们,绣鞋的踪迹才没有了。第二天,丁生因为路远,夫妻二人先回去了。八仙贪图和妹妹戏耍,老父及胡生屡次督促她,到了中午才从内室出来,跟大家一起回去了。当初他们来的时候,仪仗仆从十分气派,来观看的人群如赶集的一样。有两名强盗看到有这样漂亮的女人,连魂都飞走了,因而计谋在途中劫持她们。侦察到她们离开了村庄,就在后边跟随着,距离不到一箭远。马车奔驰很快,强盗们赶不上。到了一个地方,两边山崖夹道,车马走得便慢了。一个强盗赶上了他们,拿着刀大声吼叫,人们都吓跑了。强盗下马掀开车帘一看,原来是个老太婆坐在里面。正怀疑错劫了女郎的母亲,向两边张望的时候,飞来一刀砍伤了右臂,顷刻间被人捆绑了起来。强盗凝神仔细一看,山崖并不是山崖,而是平乐县城的城门。车中的老妇是李进士的母亲,正从乡下回来。另一个强盗随后赶到,也被砍伤马腿捉住了。守城门的兵丁绑着他们送到太守衙门,一经审讯,强盗就招供了。当时有大盗未能捕获归案,一审问,就是这两个人。

第二年刘赤水考中了进士。凤仙怕招祸惹事,全部推辞了亲戚朋友们的祝贺。刘赤水也不再另娶别的女人。到了他升任郎官时,才纳了一房妾,生了两个儿子。


\subsection{1.9.6   佟 客}
\label{\detokenize{p00_u5176_u5b83/_u767d_u8bdd_u804a_u658b_u5fd7_u5f02:id353}}
董生是徐州人,喜爱剑术,为人慷慨仗义。

有一次,他偶然在路上遇见一位旅客,两人骑着驴子同路行走。董生同他交谈,那人谈吐豪爽。又问他的姓名,那人说:“我是辽阳人,姓佟。”董生问:“你到哪里去?”他说:“我出门在外二十年了,这是刚从海外回来。”董生说:“你遨游四海,认识的人很多,有没有见到过异人?”姓佟的旅客说:“什么样的才算异人?”董生就说自己喜好击剑,只恨得不到异人传授。佟客说:“异人什么地方没有呢?但必须是忠臣孝子,异人才肯把他的武术传给他。”董生又毅然说自己就是那种人,接着抽出剑来,弹剑而歌;又用剑斩断路旁的小树,以显示剑的锋利。佟客捻着胡子微微一笑,要剑观看。董生将剑递给他,佟客看了看说:“这剑是用劣质铁铸造的,又被汗臭蒸熏,是最低劣的剑。我虽不懂剑术,但有一把剑很好用。”于是从衣下取出一柄尺把长的短剑,用它削董生的剑,董生的剑就像瓜一样脆,随手断开,断口如同马蹄一般。董生非常惊骇,也请佟客递过剑来看看,再三拂试后才还给他。

董生邀请佟客来到自己的家里,执意挽留他住两宿。董生向他请教剑法,佟客推辞说不懂。董生便双手按在膝上,夸夸其谈,大讲剑术,佟客只是恭敬地听着而已。

到半夜,忽听隔壁院子里人声嘈杂,吵吵嚷嚷,不知道出了什么事。隔院住着董生的父亲,董生非常惊疑,就到墙下凝神细听,只听有人愤怒地说:“叫你儿子赶快出来受刑,就放了你!” 一会儿,又听到用棍棒打人的声音,那呻吟不绝的人果然是董生的父亲。董生拿起长刀要过去搭救,佟客拉住他说:“你这是去送死,得想个万无一失的办法。”董生惶惶不安,向他请教。佟客说:“强盗指名找你,必定抓到你才甘心。你没有其他亲骨肉,应该把后事嘱咐给妻子儿女。我去开门,给你把仆人叫醒。”董生答应了,进去告诉妻儿。妻子扯住他的衣服痛哭起来,董生搭救父亲的念头立刻全消了。于是夫妻二人一起跑到楼上,寻找弓箭,防备强盗来攻。慌慌张张地还没准备好,听到佟客在楼檐上笑着说:“幸亏盗贼已经走了。”董生掌灯一看,果然强盗都没影了。董生犹豫地出了大门,看见他父亲到邻居家喝酒,提着灯笼刚回来;只是院子里有一些烧剩的草灰而已。董生这才知道佟客就是一位异人。


\subsection{1.9.7   辽 阳 军}
\label{\detokenize{p00_u5176_u5b83/_u767d_u8bdd_u804a_u658b_u5fd7_u5f02:id354}}
沂水某人,明朝末年在辽阳军中当兵,正赶上辽阳城被清兵攻破,这人被乱兵杀死。脖子虽然被砍断了,但还没有彻底死去。到了夜里,一个人拿着本簿册,按照上面的名字一个个查对。查到他时,那个人说他不应当死,叫左右随从把他的头接好,送他回去。于是,随从们一齐去把他的头取来安到脖子上,很多人扶着他,只听得风声簌簌地响,走了不多时,就放开他回去了。这人一看这个地方,正是自己的家乡。

沂水县令听说了这件事,怀疑他是私自逃回来的。派人把他抓来一问,才知道了事情经过。县令很不相信;又察看他的脖子,连一点断痕都没有,就要处罚他。这人说:“我说的话,我自己也没证据,只请求先把我关在牢中。断头的事可以是假,辽阳城被攻破的事不会是假。假若辽阳城安然还在,然后再让我受刑不迟啊。”县令听从了他。过了几天,辽阳来信说城被清兵占领了,城被攻下的日期,同某人说的完全一样。于是,县令便释放了他。


\subsection{1.9.8   张 贡 士}
\label{\detokenize{p00_u5176_u5b83/_u767d_u8bdd_u804a_u658b_u5fd7_u5f02:id355}}
安丘有个张贡士,因生病仰躺在床头上。忽见从自己的心窝里钻出来一个小人,身长仅有半尺高。他头戴着读书人的帽子,穿着读书人的衣服,动作像个歌舞艺人。他唱着昆山曲,音调清彻动听。道白、自报的姓名籍贯都和张贡士的一样了;所唱的内容情节,也都是张贡士生平所经历的事情。四折戏文都唱完了,小人又吟了一首诗,才消失不见了。张贡士还记得戏文的大概内容,为人讲述过。


\subsection{1.9.9   爱 奴}
\label{\detokenize{p00_u5176_u5b83/_u767d_u8bdd_u804a_u658b_u5fd7_u5f02:id356}}
河间府有个姓徐的书生,在恩村当私塾先生。进了腊月,徐生放寒假回家,路上遇见一位老者。老者看了看他说:“徐先生不在恩村教书了,明年去哪儿教?”徐先生回答说:“还教着呢。”老者说:“我叫施敬业,有个外甥,想找个好老师,刚才他托我去东疃村请吕子廉先生,可是人家已经收了稷门街的聘礼。先生您若屈尊到我家来,报酬比恩村的多一倍。”徐生辞谢说与恩村有约应守信用。老者说:“守信是君子风度,可是到明年开学还早呢。我先给您黄金一两作聘金,暂到我那里教几天,过年再商量,怎么样?”徐生答应了。老者下了马把聘金双手呈给他,说:“我家不远,宅院狭小简陋,喂不开牲口。您能不能把仆人和马打发回去,咱下步走着也挺好吗。”徐生同意,把行李放在了老者的马上。

走了三四里路,太阳要落山了,才到老者的家。徐生见大门上有一排排鼓出来的大钉和装饰成野兽头的门环,显然是有身份的人家。老者喊外甥出来拜老师,徐生一看,是个十三四岁的少年。老者说:“我妹夫叫蒋南川,生前做过指挥使,就留下这一个孩子,倒不笨,只是娇惯了些。有先生您教他一个月,一定胜过他读十年书。”不一会儿,摆上丰盛的酒宴,但斟酒上菜的全是女子。一个婢女拿着酒壶在一旁侍候,她约十五六岁,风度模样很美,徐生有点动心。宴罢,老者吩咐给徐生准备了床铺休息才辞去。天不亮,少年就来读书。徐生刚起来,就有婢女捧着毛巾脸盆来了。这婢女就是昨晚那个拿壶的。一日三餐,全是她伺候。晚上,她又来打扫床铺。徐问:“为什么没有男仆?”婢女只笑不言语,铺好了被子就走了。第二天晚上又来,徐用调戏的话试探她,她仍是笑,也不拒绝,徐生便跟她一块睡了。婢女对徐说:“俺家没男人,外头的事全靠施舅舅。我叫爱奴,夫人很尊敬您,怕别的婢女干活不干净,才派我来。今天这事儿千万保密,免得被人发觉了,咱俩都丢脸。”

有一夜,两人睡过了头,公子来上课,碰上了。徐很难堪,心中不安。到了晚上,爱奴来说:“幸亏夫人看重您,不然就坏了。公子进去把咱的事揭发了,夫人赶忙捂住他的嘴,好像怕您听见,仅仅告诫我不要在您书房里逗留得太久而已。”说完,就走了。徐生很是感激夫人。可就是她儿子不愿念书,批评他,他母亲还常讲个情;开始是派婢女,慢慢地就亲自出面,隔着窗户跟老师讲话,说着说着甚至掉了泪。每天晚上还一定要问明白了她儿子白天学得怎么样。徐生很不耐烦,生气地说:“你又由着儿子懒,又要求我把孩子教好,这号老师我当不来!我不干了!”夫人派婢女来认了错,徐才算了。

徐生自从来当先生后,常想到外面看看风景散散心,夫人老是把他关在家里。有一天,徐生喝了酒,有点醉,心里不痛快,把婢女叫来问原因。婢女说:“也没别的意思,就是怕耽误了公子的学业。先生如果真想出去走走,不是不行,请在晚上。”徐生一听,生了气:“拿了人家几两金子,就该憋闷死呀?!夜间我上哪去?白吃人家饭,我惭愧了多少天了,给我的聘金还在我包里呢。”于是拿出金子放在桌上,立即收拾行李要走。夫人走出来,一句话也不说,只用衣袖遮了脸哽咽。叫婢女把金子还给徐生,打开锁,敞了门送他走。徐生出门,觉得门很窄小;走了几步,射来了阳光,才发现自己是从一座塌陷的土疙瘩中出来。四下看看,荒凉得很,原来是座古墓。徐生非常害怕,又感激夫人待他的仁义,便用她赏给的金子雇人把坟墓培了土,在周围种上树才回家去了。

一年过去了,徐生又经过这里,向坟墓行了礼又赶路。远远看见那姓施的老者走来,微笑着向徐生问候,恳切地邀请他去做客。徐生心中明知他是鬼,但是很想问问夫人近来的情况,两人便进了村,在酒馆买了酒一起喝,不知不觉天就晚了。老者起身付洒钱,说:“我家离这儿不远了,我妹妹刚巧回来走娘家,盼先生走一趟,替老夫驱除祸事!”出了村几步,又一个院落,敲门进去,点了蜡烛与客人对坐。一会儿,老者的妹妹蒋夫人从内室出来,徐生第一次看见她本人,仔细端详,原来是位四十岁左右的美妇人。蒋夫人向徐施礼感谢,说:“我这样败落了的家庭,门户冷落,先生您能把恩德布施给已死的人,真不知怎样才能报答。”说完,掉下泪来。一会儿,蒋夫人喊:“爱奴!”又对徐生解释说;“这个婢女,是我平常所喜欢的,现在把她赠给先生,也可安慰您旅途中的寂寞。您需要什么,她能懂得您的意思。”徐生一一答应着。不多时,老者兄妹都走了,爱奴留下侍候先生睡觉。鸡叫头遍,老者就来督促起床,为他送行。蒋夫人也出来了,嘱咐爱奴以后好好侍奉先生,又对徐说: “从今往后,您该小心地保守秘密,咱两家的来往很奇特神秘,怕好事的人造出些谣言来,就不好了。”徐生答应着,告了别。与爱奴一匹马骑了,到了教书的书馆,自己单要了一间屋子,与爱奴一起生活。偶然有客人来,爱奴也不回避,别人也看不见她。徐生若想要点什么,才一想,她就给拿来了。她又擅长巫术,有点小病,她一按摩,立刻就好了。

又到了清明节,徐生回到那古墓地方,爱奴告辞下马。徐嘱咐她代向夫人问候,爱奴说:“是。”于是就不见了。几天后,徐生回来找她,刚想观察坟墓,忽见爱奴穿了一身华丽的衣裳在树底下坐着呢,于是和她一起上路。这样年年同来同去,就习惯了。徐生打算领她一同回家去,她坚决不同意。

到了年底,徐生辞了书馆返回老家,和爱奴约好再会的日子。爱奴送他到自己坐过的大树那儿,指着一堆石头说:“这就是我的坟。夫人出嫁前,我便在她身边伺候,我死后就埋在这里了。先生您若再从此经过,烧一柱香凭吊我,咱就能相见的。”

徐生告别爱奴回到家中,非常想念她,怀着敬爱之情去坟上烧香,并没见有她的影子。就买了口棺材,掘开坟墓,打算装了骨头带回家,重新安葬,以寄托爱恋之情。坟墓掘开后,徐生亲自进去看,见爱奴的面色和活人一样;皮肤虽然未腐烂,可是衣裳却已像灰那样腐败,头上的金玉首饰都和才做的一样新鲜。再看腰上,有裹着几块金子的包袱。他把包袱卷起来,揣到怀里,这才脱下袍子,盖上尸体,抱到棺材里,租了辆车拉回家去。停到另一所宅院里,给她换上身绣花新衣,自己睡在旁边,希望出现奇迹。

忽然,爱奴从门外进来了,笑着说:“挖人家坟的贼在这儿呀!”徐生惊喜地问候她,她说:“前些日子到了东昌府,三天后回来一看,我住的房子没有了。几次受您的邀请,没有跟随您来,是因为我从小受了夫人的大恩,不忍心离开她。现在您既然已经把我抢了来,并将我埋葬好,便是您对我最大的恩德了。”徐问她:“古人有死了后又活了的,如今你的身体与生前一样,为什么不仿效古人复生呢?”爱奴叹口气说:“这都是天命。世间传说的死后复生,多半是假的。要想再站起来走路,又有什么难处?但是不能和活人完全一样,所以,没那个必要了。”说完掀开棺材进去,尸体就自己站起来了,苗条的身段很可爱,摸摸她怀里却雪样冰凉。于是爱奴又想进棺内再躺下,徐好容易阻止住她。她说:“夫人对我太宠爱了,我家主人从外国带回数万黄金,我偷偷地拿了些,主人也不追问。后来我病危,又没有亲属,便藏在身上做了殉葬品。夫人为我的死哀痛得不得了,又用金玉首饰给我入殓。我的身体能不腐烂,只因为得了金宝之气,如果在人世间,哪能长久?若是真想让我保持活人似的身体,千万别强迫我吃饭。不然,灵气一散,我的游魂也就消失。”徐生就建造了精美的房子,与她一起住。她的言谈,笑声全和平常人一样,只是不吃不睡,不见陌生人。

一年以后,有次徐生喝了点酒,有些醉意,举杯把剩下的几滴酒强灌她,她立刻倒在了地上,嘴里流出血水,一天功夫尸体就腐烂了。徐生后悔已晚,用隆重的葬礼安葬了她。


\subsection{1.9.10   单 父 宰}
\label{\detokenize{p00_u5176_u5b83/_u767d_u8bdd_u804a_u658b_u5fd7_u5f02:id357}}
青州有个人,五十多岁了,又娶了个年轻媳妇。两个儿子怕后妈再生孩子,趁父亲醉酒,把睾丸给他割开,掺了些药进去。父亲醒后,谎称有病,不说这件事。日子一长,伤口愈合了。

一次他与妻子同房,刀口裂开,流血不止,很快就死了。妻子知道了原因,告到官府。官府对他儿子用刑,果然招供了。审讯的官员惊骇地说:“我如今成了单父宰啦!”把两个儿子一起处死了。

我家乡有个王生,结婚一个月就把妻子休了。妻子的父亲告到官府,当时淄川县令是辛公。问王生为什么休妻,回答说:“没法说呀。”辛公执意让他说,他只好说:“因为她不能生孩子。”辛公说:“荒唐!才结婚一个月,怎么知道她不能生孩子?”好久,王生才不好意思地说:“她阴道太偏。”辛公笑了,说:“对呀,偏了,害得家庭都不完整了。”

这个故事可以和“单父宰”一块儿当笑话说。


\subsection{1.9.11   孙 必 振}
\label{\detokenize{p00_u5176_u5b83/_u767d_u8bdd_u804a_u658b_u5fd7_u5f02:id358}}
某地孙必振,一次坐船过江,船到江心时,遇上了狂风暴雨,船身颠簸得很厉害,他同船上的人非常害怕。

这时,忽然看到一尊金甲神站在云中,手拿金字大牌朝着下面;大家一齐抬头看去,上面写着‘孙必振’三个大宇,很清楚。大家对孙必振说:“一定是你有罪,天神前来捉拿你。请你赶紧到别的船上,不要连累了我们!”孙必振还没来得及回答,大家不管他同意不同意,见旁边有一只小船,就一齐将他推了上去。孙必振刚登上船,回头一看,先前坐的那只船已翻到江中不见了。


\subsection{1.9.12   邑 人}
\label{\detokenize{p00_u5176_u5b83/_u767d_u8bdd_u804a_u658b_u5fd7_u5f02:id359}}
淄川县有一个乡人,一向无赖、霸道。有一天早晨起来,突然有两个人将他带走了。走到集市上,看见一个屠夫将半扇猪肉挂到肉架上,两个人便一个劲地朝肉架那边推挤他。他忽然感到自己的身子和架上的猪肉合到了一起,那两个人径自走了。过了一会儿,屠夫开始卖肉,拿刀砍割肉时,乡人就觉得砍一刀便疼一疼,痛彻骨髓。后来,邻居一个老头来买肉,他和屠夫讨价还价,又添肥搭瘦,片片碎割,那种疼痛更加难忍。

屠夫卖完肉后,乡人才寻着路回去,到家时已是八九点钟了。家里人说他起得太晚,他就详细地讲了刚才的遭遇。叫来邻居老头询问,老头买肉才回来,说起买肉的片数和斤数一点都不错。一早晨之间,便受到了一次凌迟酷刑,不也是很奇怪吗


\subsection{1.9.13   元 宝}
\label{\detokenize{p00_u5176_u5b83/_u767d_u8bdd_u804a_u658b_u5fd7_u5f02:id360}}
广东临江那里的山,崖高险峻,常有元宝嵌在岩石上。崖下面波涛汹涌,船不能停泊。有人划船冒险靠近山岩,伸手摘取,可元宝牢牢嵌在岩石上,坚不可动。如果某人命里注定要得此宝,则一摘就落到手里;回头看时,刚才摘元宝的地方,又生出了新元宝。


\subsection{1.9.14   研 石}
\label{\detokenize{p00_u5176_u5b83/_u767d_u8bdd_u804a_u658b_u5fd7_u5f02:id361}}
王仲超说:“洞庭湖的君山有个石洞,高大得可以在里面行船,又深又黑不见底,湖水在里面流出流进。我曾经点了蜡烛乘船进去过,看见两边石头像漆那样黑,用手按按却是软的。抽刀去割,像切下一块硬豆腐,可以随心所欲做成块研台。等出了洞,一见风,就比别的石头还硬,用来磨墨,非常好。那些雇船游览的人很多,洞中有这么好的石头不知弄出去用,它的好处也得依赖我这样好奇的人给它宣传、评论呢!”


\subsection{1.9.15   武 夷}
\label{\detokenize{p00_u5176_u5b83/_u767d_u8bdd_u804a_u658b_u5fd7_u5f02:id362}}
武夷山有一峭壁,高一千丈。人们常常在峭壁下捡到沉香玉块。太守听说后,命数百人赶造云梯,想爬到峭壁顶上看有什么怪异。三年才造好了云梯。太守登梯向上攀,快到山顶时,忽见一只大脚伸下来,脚拇指比捶衣棒还粗。一声大喝道:“不下去,就把你踹下去!”太守大惊,急忙快下,刚刚踏上地面,那云梯就像腐朽烂木一样折断,四散崩裂得没有踪迹了。


\subsection{1.9.16   大 鼠}
\label{\detokenize{p00_u5176_u5b83/_u767d_u8bdd_u804a_u658b_u5fd7_u5f02:id363}}
明朝万历年间,皇宫中有种大老鼠和猫一样大,为害很严重。朝廷向民间征集了很多好猫来捕大老鼠,结果都被大老鼠吃掉了。

正巧,这时候从外国进贡来一只狮子猫。这只狮子猫全身毛白如雪。大家把这只猫抱到有大老鼠的房子里,关上门,然后从门缝里悄悄偷看猫的动静。狮猫蹲了好久,那大老鼠才从洞穴里探头探脑地出来。它一见狮猫,就发怒地扑过来。狮猫躲避开大老鼠,跳到几案上;大老鼠追上来,狮猫又跃到地上,就这样上上下下有上百次。大家都认为狮猫害怕大老鼠,是个无用的东西。后来,大老鼠跳跃得渐渐迟慢了下来,肥大的肚子喘得一鼓一鼓的,蹲在地下稍息。狮猫见机突然猛扑而下,用爪子抓住大老鼠头顶的毛,张口咬住大老鼠的脖颈,猫鼠在地上咬斗,狮猫呜呜地吼叫着,大老鼠吱吱地扭动挣扎着。人们急忙开门进去看,大老鼠的头已被狮猫咬碎了。

大家这才明白,狮猫开始躲避大老鼠,并不是害怕,而是避开大老鼠的锐气,待消耗完它的体力后,乘其疲惫松懈时再攻击。你来我走,你走我来,狮猫是在用智谋。哎,那种匹夫之勇的粗人,只会怒目按剑,和这只大老鼠有什么不同呢!


\subsection{1.9.17   张 不 量}
\label{\detokenize{p00_u5176_u5b83/_u767d_u8bdd_u804a_u658b_u5fd7_u5f02:id364}}
有个商人,到河北去。途中,忽然下起了冰雹,他急忙到庄稼地里躲起来。这时,听到天空有人说:“这是张不量的地,不要伤害他的庄稼。”商人觉得很奇怪,暗地里想,姓张的既然“不良”,为什么还要庇护他呢?冰雹停止后,商人走进村里打听那个人,并且询问那人名字的意思。

原来,姓张的是富户人家,粮食积蓄很多。每年春天青黄不接时,贫民就到他家借粮食。归还时,他不计多少,都收进来。从来没见他用斗量过。所以取名“不量”,不是“不良”啊。村里人走到田中,见庄稼被冰雹砸得像乱麻一样,唯独张不量家所有的地,没受到损坏。


\subsection{1.9.18   牧 竖}
\label{\detokenize{p00_u5176_u5b83/_u767d_u8bdd_u804a_u658b_u5fd7_u5f02:id365}}
有两个牧童,在山里发现了一个狼穴,里面有两只小狼。牧童商量好了,每人捉了一只各自爬到一棵树上,两棵树之间大约相隔几十步远。

一会儿,大狼回来了,进洞一看,两只小狼不见了,非常惊慌。一个牧童在树上扭小狼的爪子和耳朵,故意让小狼嗥叫。大狼听见后,仰起头寻找,愤怒地奔到树下,一边嚎叫着一边抓爬着树干。另一棵树上的牧童也扭着小狼让它哀鸣。大狼听到后,停止嚎叫,四面环顾,发现了另一棵树上的小狼,于是便丢下这个,急奔到另一棵树下连抓带嚎。这时,前一棵树上的小狼又嚎叫起来,大狼又急忙转身奔到第一棵树下。就这样,大狼不停地嚎叫,不停地奔跑,来回跑了几十趟,渐渐地脚步慢了,嚎叫的声音也弱了,最后奄奄一息地僵卧在地上,很久不再动弹。两个牧童从树上爬下来细看,大狼已经断气了。

现在有些豪强家的子弟动不动就气势汹汹,横眉竖眼地舞枪弄剑,好像要把人吃掉似的。而那些逗他们发怒的人,却关上门走了。这些子弟们声嘶力竭地叫喊,更认为再也没有敌过他的,于是便以为自己是威风凛凛的英雄了。可他们不知道这种如同禽兽的威风,不过是人们故意戏弄他们取乐罢了。


\subsection{1.9.19   富 翁}
\label{\detokenize{p00_u5176_u5b83/_u767d_u8bdd_u804a_u658b_u5fd7_u5f02:id366}}
有个富翁,很多买卖人向他借钱,这天出门,一个少年跟在富翁的马后面。富翁问他干什么,他说想借本钱,富翁答应了。到了家,正巧桌上有几十枚钱,少年就很熟练地将钱一摞摞垒来垒去。富翁不借给他钱,客气地送走了他。有人问为什么,他说:“这人一定善于赌博,不是正派人。他那套赌钱的本事,无意间就在手上很充分地泄露了。”一打听,还真是的。


\subsection{1.9.20   王 司 马}
\label{\detokenize{p00_u5176_u5b83/_u767d_u8bdd_u804a_u658b_u5fd7_u5f02:id367}}
新城的王大司马名叫霁宇,在镇守北方边关时,曾经叫匠人铸造了把长杆大刀。刀宽超过一尺,重百钧。每次到边防上巡察时,就派四名兵士扛着这把大刀。他的仪仗随从到了那里,就把大刀放在那里的地上,故意让北边的人去提。他们用力去摇撼,大刀却一动不动。王司马暗中用桐木照着铁刀的样子做了一把木刀,宽窄大小一模一样,用银箔贴在刀上,时常骑在马上舞动大刀;北边各个部落的远远见了,没有不吃惊骇怕的。王司马又叫人在防线的外边埋上苇箔作为界墙,横向十余里长。形状好像篱笆墙一样,故意散布说:“这就是我的长城。”北边的敌兵把苇箔全部拔掉放火烧了。王司马又命人设置上篱墙。接连烧了三次以后,他就叫人在苇篱下埋上火药石块设上引信。北方兵又来焚烧苇墙,火药石块猛然爆炸,北兵死伤很多。北兵逃走之后,王大司马又像以前那样设置上苇箔墙,北兵远远地望见就退走了。因此对王司马服服帖帖,敬若神明。

后来王司马退休回家了,边塞上又有敌兵侵犯的警报,朝廷召他再去镇守边塞。王司马这时已经八十三岁了。他极力支撑着病弱的身体进宫向皇帝当面辞行。皇帝劝慰他说:“只是劳你躺着处理边防事务罢了。”于是王司马又到了边塞。每到一处,就躺在军中的营帐之中。北人听说王司马来了,都不相信。因而借着议和的名义。要来验证一下消息的真假。掀开帘子,见王司马神气安闲地躺在床上,就都向着床跪倒拜见,吃惊地退出去了。


\subsection{1.9.21   岳 神}
\label{\detokenize{p00_u5176_u5b83/_u767d_u8bdd_u804a_u658b_u5fd7_u5f02:id368}}
扬州有一位提同知,夜里梦见泰山神召见他,言语、神色很是气愤。抬头看见神旁边有个服侍的人,替他讲情。醒后心里窝囊,于是一大早便到岳庙去祷告。出来后,看见药店里有个人,非常像那个为他讲情的人。一问,才知是医生。回家后,忽然得了重病,专门派人去请那人。那人来了后就开了药方,他傍晚吃下去,半夜就死了。有人说:阎王和岳神天天派出十万八千名服侍他们的人,分布到天下,用迷信方法给人治病,叫“勾魂使者”。所以,吃药的人不可不防备呀。


\subsection{1.9.22   小 梅}
\label{\detokenize{p00_u5176_u5b83/_u767d_u8bdd_u804a_u658b_u5fd7_u5f02:id369}}
蒙阴县王慕贞,是官宦人家的后代。他偶然一次去江浙一带,在路上碰见一个老年妇女坐在路边哭泣。王向前问老妇人为什么哭,老妇人说:“我死去的丈夫只留下一个孩子,现在这孩子犯了死罪,有谁能想办法救救他?”王慕贞素来很慷慨,就记下了她孩子的名字,拿出他带的所有银钱,到处活动,竟把这个孩子保释了出来。

这孩子出了狱,听说是王慕贞救了他的命,心里不明白是怎么回事,就到旅店里去拜访王慕贞,一方面问个明白,一方面表示感谢。到了旅店里问起这件事来,王慕贞说:“没有什么原因,只是可怜你母亲是个老人罢了。”孩子听了大为惊惧,说:“我母亲早已死了多年了!”王也觉得这事奇怪。

到了晚上,老妇人来向王慕贞道谢,王责备她讲了假话。老妇人说:“我实话告诉你,我是东山里的老狐。二十年前曾与这孩子的父亲交好过,所以不忍心他父亲断了后代,没有人给他上坟填土。”王生对老妇人肃然起敬,再想问她几句话时,她已经消失不见了。

当初,王慕贞的妻子很贤惠,又好信佛,素来不吃荤食。在家收拾了一口干净的屋子,供着观音菩萨像。因为没生儿子,天天烧香祷告。而神也很灵,每每托梦给她,叫人躲避开这间房子,因此家中诸事都按神的旨意办。后来王氏病了,病势很重,她就把床搬到这间屋里来,又另安排了被褥在内室,整天关着门,好像在等待什么人。王慕贞很纳闷,但又因为她病得糊糊涂涂的,不忍心伤害她,所以也就未加深究。王妻卧病不起两年,时常吵叫,并撵出别人独自一人睡在屋里。别人偷着听听,似乎有人与她说话;打开门看看,又静静的一个人也没有。她在病中没有别的心事,就是有个女儿才十四岁,没有出嫁,她就天天催着给女儿治办嫁妆,打发女儿出嫁。后来女儿出了嫁,她没有心事了,就叫王慕贞到她床前,握住王的手说:“今天我们要永别了。我刚开始病的时候,菩萨告诉我,我命该早死,因女儿未嫁,心事未了,所以赐了点药,延迟了些时候。去年菩萨要回南海,留下她的侍女小梅侍候我。我今将要死去,我这个薄命人又没给你生个儿子。保儿这孩子,我很喜欢他,担心你将来娶个厉害媳妇,他们母子没有归所。小梅这女子,长得秀气美丽,又很温柔贤惠,我死了你可娶她为继室。”原来王慕贞有一妾,生一男孩,名叫保儿。王慕贞认为妻子说话荒唐,就说:“你素来敬重的是神灵,今说这话,不是侮辱神吗?”妻子说:“小梅侍奉我已经一年多了,互相亲密无间,我已好言求过她了。”王慕贞问: “小梅在哪里?”妻子说:“内室里不是她吗?”王慕贞刚想再问,妻子眼一闭就死了。

王慕贞夜里为妻子守灵,听到内室隐隐有哭泣的声音,大为惊讶,怀疑有鬼。叫了丫鬟使女们来,要开门看时,见屋里有一个二八女子,身穿孝服在哭。大家都认为是神,一起跪下叩拜。女子收了泪扶大家起来。王慕贞凝神看着她,女子只是低着头。王慕贞就对她说:“若是我死去的妻子说的话是真的,请立即上堂,接受儿女们的参拜;如果不是,我也不敢妄想,免得自取罪责。”女子腼腆地走出来。登上北堂屋。王命使女搬来椅子朝着南方。王慕贞先拜,女子也答拜;往下就按长幼卑贱依次跪下叩头,女子端庄地受了礼。唯有王慕贞的妾来拜时,女子下来拉住。王慕贞自从妻子去世后,家中的丫鬟、使女和仆人们又懒又偷,家中长时间不成样子。今天大家参拜以后,都非常肃静地站列两旁。女子说;“我感激夫人的盛意,留在人间,又把家务大事托给我,你们应各自洗心革面。以前的错误,我一概既往不咎,不然的话,不要说没有人管你们!”大家抬起头来向上看,女子真像挂的观音画像一样,时时被风吹动着。大家听了女子的训示,都非常敬畏,一起答应“是”!女子才开始安排丧事,一切都井井有条。从此,大事小事只要她吩咐下来,没有敢懈怠的。女子管理内外事务严谨。就连王慕贞要干什么,也要先告诉她才去干。虽然他俩一天几次见面,王并不敢与她说一句悄悄话。

王氏的丧事办完了,王慕贞想提成亲的事,又不敢自己直接说,就嘱咐小妾稍稍去示意一下。女子说:“我受夫人嘱托,义不容辞。但婚姻大事,不能马虎。年伯黄先生,德高望重,若求他来主持婚礼,我惟命是听。”这时,沂水黄太仆,已辞官在家闲居,他是王慕贞父辈的好朋友,来往很密切。王慕贞就亲自去请,见到黄太仆,把实情告诉了他。黄也觉得奇怪,便与王一同来到王家。女子听说黄太仆来了,急忙出来拜见。黄太仆一见小梅,惊奇地认为是仙女,谦逊地不敢受礼。接着帮助她置办了优厚的嫁妆,举行了结婚大礼就回家去了。小梅又送给他枕头、鞋,像对待公婆一样,从此两家更加亲密。

合婚以后,王慕贞始终把小梅当神看待,亲热时也很严肃,时时追问菩萨的起居情况。小梅笑着说:“你也太傻了,哪有真正的神人下凡与俗人结婚的?”王还是追问小梅的身世。小梅说:“不必苦苦追问了!既然你拿我当神,就早晚供养着,自然就无灾无殃。”

小梅管理仆人非常仁慈宽厚,不带笑容不说话;但是丫鬟使女们打闹时,远远看见小梅,就马上默默地不吱声了。小梅笑着对她们说:“难道你们还拿我当神吗?我哪里是神!实际上是夫人的姨表妹。我们小时就很要好,姐姐病后想我,偷着让南庄王姥姥叫我来的。只是因为天天接近姐夫,男女之间怕有嫌疑,所以假托是神,将我关在屋里,其实哪里是神呀。”大家还是不相信,天天侍奉在她身旁,观察她的一举一动,和平常人并没有两样,从此神的传说才慢慢平息了。但是以前那些顽皮的奴婢,王氏活着时打骂都没有教育好的,现在小梅说一句话,没有不听招呼的。都这样说:“我们自己也不知为什么,说实在的也不是怕她;但只要一见她的脸面,就心里软了,所以不忍心违背她的意旨。”

小梅执掌家务以后,几年的时间,土地连片,仓里存粮一万多石。又过几年,王慕贞的妾生了一女孩,小梅生了一男孩。这男孩生下来,在臂上有一个红点子,因此起个名字叫小红。满月的那天,小梅让王慕贞举行盛筵邀请黄太仆。黄太仆也送了很丰盛的贺礼,但他本人推辞年纪大不能来;小梅又打发两个老妇人再去请,黄太仆才亲自来贺喜。小梅抱着孩子,露出小孩的左臂告诉黄太仆为什么叫小红,并再三请教这名字好不好。黄太仆笑着说:“这个红点是喜红,名字可增加一个字,叫喜红。”小梅很高兴,再一次拜谢。

这一天,鼓乐之声充满了庭院,亲戚富友来往不绝,犹如闹市。黄太仆留住了三天才走。

喜红的生日过后,忽然门外来了一群车马,说是接小梅回去走娘家。过去十几年,小梅并无亲友,怎么忽然有了娘家?大家议论纷纷,而小梅好像什么也没听见。自己梳洗打扮已毕,把孩子抱在怀里,要王慕贞送他,王答应了。送到二三十里处,路上静得没有行人了,小梅停住车,叫王下马,私下对王说:“王郎!王郎!咱们相会的时间短,别离的时间长,不是太悲惨了吗?”王惊慌地问怎么了,小梅说:“你以为我是什么人?”王回答:“不知道。”小梅说:“在江南,你曾救过一个死罪犯人,有没有?”王说:“有这回事。”小梅说:“在路上哭的就是我的母亲。她感激你的义气,想报答你。因为你夫人信佛,让我假托神仙,给你做妾以图报答。现在幸好生下这个孩子,心愿已了。我看你将要有晦运,这个孩子在你那里,恐怕不能养育,所以借着回娘家带走他,以解除儿的危难。你回去记住:家里有人死时,你在早上鸡叫头遍就到西河柳堤上,看见有挑葵花灯的,赶快挡住道路求他,可以免除灾难。”王答应说:“是。”又问小梅什么时候回来,小梅说;“不能肯定,你只要记住我刚才的话,再会时间不会太长。”临别时,握住王的手双泪交流。接着上车风驰电掣般地走了。王远远看不见人影了,才回了家。

经过了六七年,小梅一直没有音信。这一年忽然四乡瘟疫流行,死的人很多,王慕贞家一个丫鬟病了三天就死了。于是王想起小梅临走说的话,就开始关心这个事。这一天他与客人饮酒,不料喝了个大醉睡着了。一觉醒来,听见鸡叫,于是他急忙起来到西河堤上,看见有灯光闪闪烁烁,好像刚刚过去。他就急忙追赶,相距灯光也就百步之远,可是越追越远,渐渐就看不见了,他十分懊悔地回了家。几天的工夫,他便得了急病,接着就死去了。

王家这一家族里有很多无赖之徒,因为王慕贞死了,就仗势欺人。王慕贞家的庄稼、树木,公然去砍伐,王家的日子渐渐衰败。叉隔一年,保儿又死了,一家人更是没有作主的。无赖们也更横行霸道,瓜分了王家的田地,抢走了王家的牛、马;还要瓜分王家宅子。因为王慕贞的妾还住在里面,他们便纠集了几个人硬是把她卖给了别人。妾恋着自己的小女孩不走,母女抱头痛哭,惨不忍睹。正在十分危难的时候,忽然听到大门外有轿子来了。大家一看,见是小梅领着儿子从轿子里出来。小梅四下看了看,见人这么多,就问:“这都是些什么人?”妾哭着告诉了她一切情由。小梅脸色一变,便叫从人来,吩咐把大门锁了。无赖们想要抗拒,可四肢发软一点力气也没有了。小梅叫人把他们一个一个都绑起来,拴在走廊的柱子上,一天给他们三碗稀粥。随即打发老仆人去告诉黄太仆,然后才到屋里痛哭。哭了一会儿,小梅对妾说:“这也是天数!我本来打算上月回来,正碰上母亲生病耽误了几天,所以才有今天的情景。不料转眼之间咱家成了废墟!”又问以前的丫鬟使女们,说是都被无赖们抢去了,小梅更加叹惜!第二天,丫鬟使女们听说小梅回来了,都自己逃了回来,主仆相见,没有不痛哭流泪的。

拴在柱子上的无赖们,都吵着说小梅的儿子不是王慕贞的亲骨肉,小梅也不与他们分辩。随后,黄太仆来到,小梅领儿子出来迎接。黄公见了拉住男孩的臂膀,捋起左臂的袖子,当众叫大家看,见那个朱砂痣清清楚楚,证明这男孩确是王慕贞的后代。然后把丢失的东西,详细检查,登记造册,黄公亲自拿着去找了县官。县官命人逮捕了无赖们,各打了四十大板,又严加追查东西的下落。不几日,田地、牛马等,都归还了王家。

事情料理完了,黄太仆要回家。小梅领着儿子跪下叩头,哭着说:“我并不是世间的人,叔父你是知道的。今把这孩子委托给叔父你了。”黄公说:“只要我有一口气,我一定尽力照顾好他。”

黄公走后,小梅把一切事情安排就绪,把孩子交给妾照管,自己备了酒、祭品到王慕贞坟上去扫墓。半天的工夫没有回来,人们去了一看,光见祭品摆着,而小梅却已不见了。


\subsection{1.9.23   药 僧}
\label{\detokenize{p00_u5176_u5b83/_u767d_u8bdd_u804a_u658b_u5fd7_u5f02:id370}}
济宁有个人,在荒郊某寺院外遇见一个云游四方的和尚,晒着太阳抓僧袍上的虱子,杖上挂着个葫芦,像卖药的。于是这人开玩笑说:“喂,和尚卖不卖男女房事用的药丸儿?”和尚说:“有!治阳痿的,治男人生殖器小的,立刻见效,用不了一个晚上。”这人挺高兴,就向和尚求药。和尚解开旁边的僧袍角,取出药丸,有高粱粒儿大,叫他吞下去。大约半顿饭工夫,他便觉得下部忽然长大。过了一会儿自己一摸,比过去大出三分之一。他还不满足,瞅着和尚去解手的空儿偷偷解开僧袍,捏出两三粒丸子全吞了。立刻觉得皮肤像裂开,像抽筋,脖子在缩短,腰也在弯,而下部还一个劲儿地长。他吓坏了,无计可施。和尚回来见他那样子,吃惊地说: “你一定偷了我的药了!”赶紧给了他另一丸药,才觉得下部不长了。解开衣服自己一看,那里差点长成了第三条腿!这人缩着脖子,一歪一斜地回了家,父母都不认得他,从此成了个废物,天天在街上躺着,有不少人见过呢!


\subsection{1.9.24   于 中 丞}
\label{\detokenize{p00_u5176_u5b83/_u767d_u8bdd_u804a_u658b_u5fd7_u5f02:id371}}
于成龙,是山西永宁州人。他担任中丞时,一次巡视属下的州县,到了江苏高邮,正好遇上一个案子:有个富户人家的女儿将要出嫁,嫁妆很多。夜里被盗贼从墙上打洞进入屋内,全部偷走了。当地知州对这个案子没有办法。于成龙下令把城里其它大门关闭,只留下一个城门让行人出进,派遣捕快看守城门,严格搜查出进行人装载的东西。又在城里到处张贴告示:全城居民都要回到自己家里,等候第二天大搜查,官府一定要找到窝藏赃物的地方。然后又暗地嘱咐捕快:假如有人多次从城门出出进进,就把他捉起来。

第二天下午,捕快捉到两个人,他们除身上穿的之外,并没有携带其它东西。于成龙说:“这两个家伙就是真正的盗贼!”那两个人不停地诡辩,于成龙命令捕快解开他俩的衣服进行搜查。只见两人穿的衣袍内套着两身女人的衣服,都是嫁妆里的服装。原来盗贼看到告示后,恐怕第二天大搜查,就急忙把盗窃的财物往城外转移。只是东西太多,很难一次带出城去,所以就秘密地穿在身上多次出入城门。

还有一次,于成龙在广西罗城县任县令时,因公务到邻县去。清晨,他经过县城郊外,看见两个人用床抬着一个病人,身上蒙着大被子;枕头上露出一缕头发,上面别着一支凤钗,侧着身子躺在床上。有三四个健壮的男人跟在两旁,时常轮换着用手掖掖被子,好像怕风吹进被窝里。走一会儿,就在路边停下来,再换上另外两个人抬。于成龙走过去之后,感到很奇怪,打发衙役过去问问抬的是什么人,他们说妹子病得厉害,快要死了,要把她送回婆家。于成龙走了二三里路,又打发衙役回去,看他们抬进哪个村里去。衙役暗暗跟在后边,那伙人进了一个村庄,在一户人家的门前停下来,从这家出来了两个男人把他们迎了进去。衙役回来告诉了于成龙。于成龙问当地县官: “城里有没有发生抢劫案子?”县官回答:“没有。”当时朝廷对官吏政绩考核很严,官员们往往欺上瞒下。所以百姓即使被盗贼杀了,也要隐瞒起来不敢报案。于成龙到了公馆住处,嘱咐手下的衙役细心打听,看有没有被抢劫的人家。果然有家大户,被盗贼进入家中,烙死了主人,抢走了钱财。于成龙令衙役把他儿子叫来,问他被抢的情况。大户的儿子坚决不承认。于成龙说:“我已经替你把盗贼捉拿到这里了,怎么还说没有呢?”大户儿子这才给于成龙磕头,哭着哀求为他父亲报仇。于成龙又去拜见当地县官,派出强壮的衙役,夜里四更出城,直去那个村庄,当场抓住八个男人。一经审问,都低头认了罪。问他们那个病妇是什么人,盗贼供认说:“那天夜晚住在妓院里,同一个妓女合谋把钱财放在床上,叫她装病躺在床上抱着,抬到窝赃处再分赃。”

案子破获后,大家都钦佩于成龙断案如神。有的人问他怎么知道那些人是盗贼呢?于成龙说:“这事情很容易理解,只是有人不去细心观察罢了。世上哪里有少妇躺在床上,而让男人把手伸到被窝里去呢?而且,不断换人抬着走,看样子就知道抬的东西很重;又一起用手掖被子围护她,就知道里边一定还有其它东西。假若病妇昏迷不醒送到婆家,必定有女人在门口迎接,但仅仅看到两个男人出来,并且见了既不感到惊讶,也不问一声就迎了进去,这是不合乎情理的。以此断定他们是盗贼。”


\subsection{1.9.25   皂 隶}
\label{\detokenize{p00_u5176_u5b83/_u767d_u8bdd_u804a_u658b_u5fd7_u5f02:id372}}
明朝万历年间,历城县令梦见城隍向他要人去服役,他就从自己衙门里挑选了八名皂隶,将他们的姓名写在文牒上,到城隍庙烧了。当天晚上,这八个人就都死了。

城隍庙东有个酒店,店主人原来和其中一个皂隶有交情,碰巧那天晚上那皂隶来买酒,店主人问他:“款待谁呀?”答道:“同事很多,买壶酒一起熟悉熟悉。”天亮后,店主人见了别的皂隶,才听说那人已经死了。去庙里开了门,见酒瓶在那儿,里面酒也没动。主人又回店看付的酒钱,都是纸灰。县令让人给这八个人在城隍庙里塑了像。其他皂隶每逢出差,都要先用酒食酬告了塑像才出发,否则就会受到县令的责打。


\subsection{1.9.26   绩 女}
\label{\detokenize{p00_u5176_u5b83/_u767d_u8bdd_u804a_u658b_u5fd7_u5f02:id373}}
绍兴有个老寡妇,夜里正在纺线,一位少女忽然推门进来,笑着说:“老奶奶不累呀?”老妇一看,少女有十八九岁,长得很俊,一身光彩华丽的长衣。老妇吃惊地问:“你从哪儿来?来干啥?”少女说:“觉得老奶奶一个人住着孤独,所以来跟你作伴。”老妇怀疑她是从官宦人家私跑出来的小姐,便一再追问。少女说:“奶奶别怕,我也像您一样孤身一人。喜欢您的贞洁,才来投奔您。省得咱俩都闷得慌,难道不好吗?”老妇又怀疑她是狐仙,犹豫着不答应。少女竟然上了床替她纺起线来,说:“奶奶别愁,这种活路我最熟悉了,一定不白吃您的饭。”老妇觉得她温柔俊美可爱,也就安心了。

夜深了,少女对老妇说: “我带来的被褥枕头还在门外头,您出去小便的时候请替我提进来。”老妇出了门,果然拿回一个大包袱。少女解开,铺到床上,也不知什么绸缎,只觉得又香又滑溜。老妇也铺开自己的布被子,与少女同睡。少女还未脱完衣服,屋里就充满了浓烈的香味儿。睡下后,老妇暗想:遇见这样的美人,可惜我不是男人。少女在枕头边笑了,说:“奶奶七十多了,还想入非非呀?”老妇说:“没有的事!”少女说:“既然没有,为什么想做男人?”老妇更觉得她是狐仙了,很害怕。少女又笑了,说:“既然想当男人,为什么心里又怕我呀?”老妇吓得全身哆嗦,连床都晃动了。少女说:“唉,这么大个胆,还想当男人!实话告诉您吧:我真是仙人,可对您并无害。但有一件:只要您说话谨慎,就不愁吃穿。”

老婆子早晨起来,拜倒在床下。少女伸臂拉她,那胳膊像油脂一样滑腻,散发着湿热的香气。触到她的肌肉,觉得全身都轻快,老妇又胡思乱想。少女笑话她说:“老婆子,刚不哆嗦了,心又哪儿去了?假如叫你当男人,非为情爱搭上命不可。”老妇说;“假设我真是男人,今夜哪能不死?”从此两人感情融洽,天天一块儿干活。看看那少女纺的麻线,又匀又细又光泽;织出的布,像锦锻那么鲜艳,价钱比平常高出两倍。老妇出门时就把门反锁上。有来找老妇的,老妇就在别的屋子里应酬,所以少女住了半年也没人知道。

后来老妇渐渐地把这事对关系好的人泄露了。邻居中的姊妹们都托她求见少女。少女责备她说:“你说话不谨慎,我在这里住不长了。”老妇为自己的失言懊悔,深深自责。可是求见的一天比一天多,甚至有以势强迫的。老妇哭着对少女自我辩白。少女说:“若是些女伴,见见也没什么。就怕有轻薄男人,会对我无礼。”老妇一再恳求,少女才答应了。过了几天,什么老太太、大姑娘小媳妇,烧着香在大道上排成了队。少女讨厌人多又乱,不论什么身份的,一概不答腔,只静坐着,任人朝拜而已。同乡中的少年听说她的美貌,心都被牵动了。老妇一律拒绝。

有个姓费的少年,是本地有名的文士,倾尽全部财产买通了老妇,老妇答应为他引见。少女早知道了,责备老妇说:“你想卖我呀?”老妇伏在地上承认错误。少女说:“你贪他的贿赂,我被他的痴情感动,可以见见,可就是我们再也没有缘分了。”老妇又叩头。少女定下明天见面。费生知道后,很高兴,带着香烛去了,进门后深深作揖。少女在帘内与他说话,问:“你宁肯倾尽家产也要见我,有什么要跟我说的呢?”费生说:“实在不敢有别的要求,只因为古代美人王嫱、西施仅仅听说但没见过。您若不嫌弃我愚笨凡俗,让我开开眼界,在下就满足了。若说我命中注定不可能,这不是我希望听到的。”说完,隔着布帘忽然看见少女容颜闪现,墨绿色的眉毛,红嘴唇…… 都显露出来,好像并没有帘子挡着。费生神志荡漾痴迷,不觉倒身下拜。拜完站起来,布帘忽然变得又厚又重,什么也看不见了。他又暗恨刚才没见着下半身,这念头刚出现,马上又看见帘下一双穿绣花鞋的小脚,瘦得还不满一把。费生又拜。帘内说话了:“算啦,您回去吧,我累了。”老妇把费生请到另一房间,上茶款待。费生在墙上题了一首《南乡子》词:

“隐约画帘前,三寸凌波玉笋尖;点地分明莲瓣落,纤纤,再着重台更可怜。花衬凤头弯,入握应知软似绵;但愿化为蝴蝶去,裙边,一嗅余香死亦甜。”

题完才走了。少女见了词,不高兴地对老妇说:“我说缘分到头了,这证明我的话不错吧?”老妇又跪下请罪。少女说:“罪不都在你。我偶然掉进情网,把我的美丽显示于人,于是被脏言脏语玷污,这全怪我,跟你没什么关系。倘若不早些搬走,怕在情网中越陷越深,在灾难中脱不了身了。”于是收起行李出门而去。老妇追上去挽留,眨眼间少女已经不见了。


\subsection{1.9.27   红 毛 毡}
\label{\detokenize{p00_u5176_u5b83/_u767d_u8bdd_u804a_u658b_u5fd7_u5f02:id374}}
红毛国,过去许诺与中国互通贸易。边防的元帅见他们来的人太多,就不准许他们登岸。红毛国的人再三请求说:“只要赐给我们一块毛毡那么大的地方就足够了。” 元帅想,一块毛毡能容纳的人没有几个,就答应了。红毛国的人就把毛毡放到岸上,仅能容纳两个人;他们把毛毡拉扯一下,就能容纳四五人;他们一边拉扯毛毡一边从船上登陆,顷刻之间,毛毡大到一亩多,已能容纳数百人了。这些红毛国人一齐抽出短刀,由于出其不意,被他们劫掠了好几里的地方才离去。


\subsection{1.9.28   抽 肠}
\label{\detokenize{p00_u5176_u5b83/_u767d_u8bdd_u804a_u658b_u5fd7_u5f02:id375}}
莱阳有个人,白天在屋里躺着,见一个男人和一个妇女拉着手进来。妇女又黄又胖,腰粗得都快叫她仰面倒下去了,露出一副很愁苦的神色。男的催促说:“来,来!”这人以为是私通的,就假装睡着,看看他们千什么。

进了屋,那男人和妇女好像没看见床上有个人。男的又催妇女:“快点儿!”妇女就自己解衣露出胸膛,肚子大得像鼓。男的拿出一把刀,使劲刺进去,从心下边一直剖到肚脐,还能听见哧哧的声音。这人吓坏了,气也不敢喘。可妇女皱着眉忍着痛,一声不吭。男人用嘴叼住刀,把手伸进妇女的肚子里,拽出肠子挂在胳膊肘上。边抽边挂,一会胳膊上就挂满了,又用刀割断,放在桌上。又抽,桌子又满了,搁在椅子上,椅子又满了。竟然在胳膊上挂了几十盘,像打渔人挂在臂上的网,朝这个人头边上一扔。这人觉得一阵热乎乎的腥味,面上嘴上脖子上被压得连个透气的缝也没了;这人受不了,用手推肠子,大叫着起来往外跑。肠子掉在床前,他的两腿被绊住,扑哒,倒了。家里人听见动静跑去看,只见他缠了一身猪下水。再进屋仔细看,又啥也没有。大家都说他看花了眼,也没害怕。等这人把亲眼见的一说,大家才觉得奇怪,可屋里连点血迹也没有,唯有血腥味儿几天不散。


\subsection{1.9.29   张 鸿 渐}
\label{\detokenize{p00_u5176_u5b83/_u767d_u8bdd_u804a_u658b_u5fd7_u5f02:id376}}
张鸿渐,是永平郡人。年龄才十八岁,是永平郡有名的文土。当时的卢龙县令赵某异常贪婪残暴,百姓们受尽压榨,叫苦连天。有个姓范的秀才被赵县令用杖刑活活打死,全县的秀才们对范生的屈死都忿忿不平,要到省里的巡抚衙门去为范生鸣冤告状,来求张鸿渐起草状词,并约他一起赴省。张鸿渐答应了他们的要求。张的妻子方氏,长得很美,性情贤惠,听到秀才们的主张后,就劝张鸿渐说:“大凡跟秀才们作事,可以共同取胜,而不可以一起失败:若胜了就人人贪天功以为己有,一败了就纷纷瓦解四散,不能再聚合起来。当今是个认钱财看权力的世界,是非曲直很难凭真理判定。您又孤单无兄弟,假若有个三长两短,危难之时谁能来解救您!” 张鸿渐很佩服她说的话,心里后悔了,便去婉言谢绝了秀才们的约请,只为他们写了状词就走了。巡抚衙门对这起案子审理了一下,没有作出结论。赵县令用了巨额金钱贿赂上司,秀才们竟得了个结党的罪名被抓起来,并又追查写状词的人。张鸿渐害怕,只得逃离家乡。

张鸿渐逃到陕西凤翔府境内,钱都花光了。日落西山天将黑了,他还在旷野中徘徊,寻不到住宿的地方。忽然看见附近有个小村庄,就急忙奔了过去。有个老妇人正要出来关门,看见了张鸿渐,就问他要干什么。张鸿渐就对她照实说明了来意。老妇人说:“吃饭睡觉,这都是小事;只是家里没有男人,不便留客。”张鸿渐说: “我也不敢有过高的希望,只要能容我在门里头借宿,躲避一下虎狼就心满意足了。”老妇人这才让他进来,关上门,给了他一捆干草,嘱咐说:“我是同情你没处去,私自答应留宿的。天不明你就得早走,恐怕叫我家姑娘听到,就要怪罪我了。”说完走了。张鸿渐倚着墙打起盹来。突然发现有灯笼闪着亮光,原来是老妇人引着一位女郎出来了。张鸿渐急忙躲到暗处,偷偷看去,那女郎是个二十来岁的俊美人。女郎来到大门口,看见了干草,就问老妇人是怎么回事;老妇人如实说了。女郎生气地说:“咱满门女流之辈,怎能收留非亲非故的男人!”立即又问:“那人在哪里?”张鸿渐害怕,从暗中出来跪在了台阶下。女郎详细问明了他的籍贯族姓,脸色稍微转和,说道:“幸好是位风雅学子,不妨留宿。但老奴竟然不禀报一声,这样潦草简陋,岂能用来招待君子!”便吩咐老妇人领客人进了屋。

不一会儿,摆上酒来,菜肴饭食都精美清洁;饭后又拿进锦缎褥子铺在床上。张鸿渐非常感激女郎,就私下里偷偷打听她的姓氏。老妇人说:“我家主人姓施,老爷和夫人都去世了,只留下了三位姑娘。刚才你见到的那位,是大姑娘舜华。”老妇人说完走了。张鸿渐看见桌上有《南华经》的注释本,便取过来放在床头上,趴在床上翻阅起来。忽然舜华推开门进来了。张鸿渐放下书,要寻找自己的鞋帽。舜华走到床前按他坐下,说:“用不着!用不着!”就靠近床前坐下,很腼腆地说道: “我觉得您是位风流才子,想把自己的终身托付给您,于是不避嫌疑而来。您能不嫌弃我吗?”张鸿渐听了,惊慌得不知怎么回答,只是说道:“不敢相瞒,小生家中已有妻子了。”舜华笑着说:“从这里也能看出您的诚实,不过也不妨碍。既然您不嫌弃,我明天就去请媒人。”说完了,要走。张鸿渐探过身子拉住她,她也就留下来。天还没亮舜华即起床,拿银子送给张鸿渐,说:“您可以拿它作为游玩的费用。临近黑天,应该晚一点来,恐怕被别人看见。”张鸿渐按她的话,早出晚归,这样过了半年也就习以为常了。

有一天,他回来得稍早了点,到了住处,村庄房舍全没有了,感到非常惊讶。正在徘徊的时候,听见老妇人说:“今天怎么回来得这么早哇!”一转眼的功夫,院落又像以前那样,自已原来已经站在屋里了。张鸿渐心里更加惊异。舜华从里屋出来,笑着说:“您怀疑我了吗?实话对你说吧:我,是个狐仙,和您本来就有前世的姻缘。假若你一定要见怪的话,就请你马上走吧。”张鸿渐留恋她的美貌,也就安下心来。夜里张鸿渐对舜华说:“您既然是仙人,千里之遥的路程喘口气的功夫就该到了。小生离家已经三年了,心里惦念着老婆孩子,您能带我回家一趟吗?”舜华听完,好像不高兴地说道:“原以为,我对您的恩爱之情够深厚的了;可您守着我却想着她,看来你对我的这些亲热,都是虚假的啊!”张鸿渐急忙向她道歉说:“您怎么说出这样的话来!俗话说得好:‘一日夫妻,百日恩义。’以后我回家想念您的时候,也会像今天怀念她一样。假若我得新忘旧,您能喜欢我吗?”舜华这才笑着说:“我是有点心窄:对于我,就希望你永远不能忘记;而对于别人,就希望你一定把她忘了。不过您想暂时回家看看,这又有什么难处?你的家就近在咫尺啊!”于是抓着他的衣襟出了门。见道路昏黑,张鸿渐畏缩不前。舜华便拉着他往前走,不多时,她说:“到了。您回家去,我就走了。”

张鸿渐停住脚步仔细认了认,果然见到了自已的家门。他跳墙进了院子,看见屋里仍然亮着灯。便走过去用两个手指头弹敲屋门。屋内问是谁,张鸿渐说明是自己回来了。屋里人拿着蜡烛开开门,真是方氏。两人相见惊喜异常,握着手进了帏帐。张鸿渐看见儿子睡在床上,很感慨地说:“我走的时候儿子才有膝盖那么高,如今却长得这么大了。”夫妇二人互相依偎着,恍惚如在梦中。张鸿渐对妻子历述了自己在外的整个遭遇。当问到那场官司时,才知道秀才们有死在监狱里的,有远离家乡的,张鸿渐更加佩服妻子的远见卓识。方氏纵身投入他的怀抱,说:“您有了漂亮的新娘子,看来不会再想念我这独守空房的落泪人了!”张鸿渐说:“若是不想念,怎么还回来呢?我和她虽说感情好,然而她终究不是人类;只是她的恩义不能忘记罢了。”方氏说:“你以为我是什么人?”张鸿渐仔细一看,眼前哪里是方氏,竟是舜华!伸手去摸儿子,原来是一个“竹夫人”。张鸿渐惭愧得说不出话来,舜华说:“我可知道你的心了!我们的缘分该从此断绝了。幸好你还不忘恩义,多少还能赎罪。”

过了两三天,舜华忽然说:“我想痴心恋着别人,终归没有意味。您天天怨我不送你回家,今天正好要去京城,顺路可和你一同走。”于是从床上拿过“竹夫人”,和张鸿渐都跨上去,叫他闭上两眼。张鸿渐觉得离地不远,耳边响起飕飕的风声。不多时,便落下来,舜华说:“咱们从此别了。”张鸿渐正要和她约定相见日期,舜华早已不见了。

张鸿渐惆怅地站了一会儿,听见村里狗叫,模模糊糊地看见树木房屋,都是家乡的景物,便沿着道路回到家门前。他跳墙进去敲门,还像前一次那个样子。方氏一听惊起,不相信自己的丈夫能回来,再三追问对证确实了,才挑着灯呜咽着开门出来。两人相见,方氏哭得抬不起头来。张鸿渐怀疑这是舜华在变幻花样耍弄他;又看见床上睡着个孩子,和上次一样,就笑着说:“这‘竹夫人’又被你带进来了?”方氏听了大惑不解,变了脸说:“盼着你回来都到了度日如年的地步,枕头上的泪痕还在上边。如今刚刚能相见,竟无一点悲伤依恋之情,哪还有点人性?”张鸿渐见她情真意切,这才上去抓住她的臂膀哽咽起来,把自己的前后遭遇详尽地讲了一遍。问到官司的结果,与上次舜华说的话完全符合。夫妻二人正在相对感慨的时候,忽然听到门外有脚步声,方氏问是谁,却无人应声。

原来村里有个年轻的光棍无赖某甲,早就看上了方氏的美貌。这一夜他从别的村里回来,远远地看见有个人跳进方氏的院墙里面去了,以为这必定是个应方氏之约去私通的,便尾随着进来了。某甲本来不太认得张鸿渐,只是伏在门外偷听他们说话。等到方氏听到脚步声多次问是谁时,某甲竟说道:“屋里是什么人?”方氏假说: “没有人。”某甲说:“我偷听已经很久了,这就要捉奸呢。”方氏不得已,只好说了实话。某甲说:“张鸿渐的大案还没了结,如果是他来家,也应该绑起来送到官府去。”方氏苦苦哀求他,某甲的话却越说越下流,并逼她答应和自己私通。张鸿渐胸中怒火燃烧,拿刀冲出门去,照某甲就是一刀,砍中了他的脑袋。某甲倒在地上,仍在号叫,张鸿渐又连砍数刀,才死了。方氏说:“事情已到了这步田地,罪更加重了。你赶快逃走吧,让我来担这个罪名。”张鸿渐说:“大丈夫该死就死,岂能为活命而辱没老婆、连累孩子呢!你不要管我,只要让孩子能读书成才,我就是死也闭上眼了。”

天明以后,张鸿渐去县衙自首了。赵县令因为他是朝廷审批的案件中的人犯,所以姑且只轻微责罚了他一下。不久张鸿渐就被从府里押往京城,身上的枷锁折磨得他非常难受。路上遇见一位女子骑马而过,有个老妇人为她牵着马,一看原来是舜华。张鸿渐呼喊老妇人想说句话,泪水随着声音淌了下来。舜华掉过马头,用手掀开面纱,惊讶地说:“这不是表哥吗?怎么来到这里?”张鸿渐大略说了一下事情的经过,舜华说:“若依着表兄以往的做法,我就该掉过头去不管;但是我却不忍心这样做。寒舍离这里不远,就邀请差官们一起光临,也可多多资助你点盘缠。”跟着她走了二三里路,看见一座山村,村里楼阁高大整齐。舜华下马进村,吩咐老妇人开门引进客人。不一会儿摆上了丰盛味美的酒菜,就像早准备好了一样。舜华又让老妇人出来对他们说:“家里恰巧没有男主人,请张官人就多劝差官喝几杯,路上依赖他们的地方多着呢。已经派人去筹集几十两银子,一来为官人作盘费,二来也好酬谢两位差官,人到这时还没回来呢。”两个差役心中暗喜,便开怀痛饮,不再说赶路了。天渐渐黑了,两个差役径直喝醉了。舜华出来,用手指了指张鸿渐身上的枷锁,枷锁立刻就从他身上脱落了。她拉着张鸿渐一起跨在那匹马上,像龙一样飞驰而去。不多时,舜华催促他下马,说:“您就留在这儿。我和妹妹约好要到青海去,又为你逗留了半天,让她久等了。”张鸿渐说:“咱们以后何时见面?”舜华没回答;再问她时,她把张鸿渐推落到马下,自己扬长而去。

天亮以后,张鸿渐问人家这是什么地方,原来是山西太原郡。他于是到了郡城,赁了处房子教起书来。并改名换姓叫宫子迁。他在这里一住十年。通过打听知道这几年官府对于追捕他的事已经渐渐松懈,这才又慢慢地朝东往家走。靠近村子时,他没敢急着进,而是等夜深人静后才进去。

张鸿渐到了家门口,一看院墙又高又坚固,没法再跳进去,只得用马鞭敲门。过了好久,妻子才出屋问是谁。张鸿渐小声告诉了她。方氏听说高兴极了,急忙开门叫他进来,并装作斥责的声音,说道:“在京城钱不够用,就该早回来拿,怎么叫你半夜回来?”进了屋,夫妻二人说了说这些年来各人生活的情况,才知道那两个差役也一直逃亡在外没有回来。他俩说话期间,帘子外边有个少妇多次来往,张鸿渐就问她是谁,方氏说:“是儿媳。”张鸿渐又问:“儿子在哪里?”方氏说:“到郡城参加乡试还没回来。”张鸿渐一听流下泪来说:“我在外流落了这些年,儿子已经成人了,没想到他真能读书成才,您的心血可说是全都用尽了!”话没说完,儿媳已烫好了酒做好了饭,摆了满满一桌。张鸿渐真是大喜过望。住了几天,他总是躲在床上不出屋子,惟恐被别人知道。

有天夜里,夫妻二人刚睡下,忽听外面人声鼎沸,捶门的声响非常猛烈。他俩吓坏了,赶紧一同起来。听到外面的人说:“他家有后门吗?”方氏更加害怕了,急忙用一扇门代替梯子,送张鸿渐乘夜色跳墙出去;然后到大门口问是什么事,原来是来家为新科举人报喜的差役。方氏大喜,很后悔让张鸿渐逃走,但是追也没法追了。

张鸿渐这天夜里在野草树丛中连跑带钻,急得顾不上分辨道路;到了天亮,已是困乏到了极点。起初他本想往西走,问了问路上的人,这儿竟离去京城的大路不远了。于是他进了村子,心想拿衣服换顿饭吃。发现有座高大的门楼,墙上贴着报喜的大红纸条,走过去看了看,知道这一家姓许,是新科举人。不一会儿,有位老翁从大门里出来,张鸿渐迎上去行了个礼并说明了来意。许翁见他仪表不凡,知道他不是骗吃喝的人,便请他进家用酒饭招待了他。许翁于是问他要到哪里去,张鸿渐假说道:“在京城设馆教书,回家路上遭了强盗的洗劫。”许翁愿意留下他来教自己的小儿读书。张鸿渐略问了一下许翁的官阶门第,他竟是一位退居林下的京官,新科举人是他的侄子。

过了一个多月,许举人和一位同榜的举人一起来家,这位举人说他家住永平府,姓张,是个十八九岁的年轻人。张鸿渐因为张举人的家乡、姓氏谱系和自己相同,心中怀疑他可能是自己的儿子;但是又一想县里的同姓很多,怕错了就没敢相认。到了晚上解行李时,许举人拿出一册记载同榜举人籍贯、三代的《齿录》,张鸿渐急忙借来翻阅,一看这张举人还真是自己的儿子。张鸿渐看着《齿录》,不觉掉下泪来。大家都惊奇地问他怎么了,他这才指着上面的名字说:“这张鸿渐,就是我呀。”便详尽地叙述了自己的前后遭遇。张举人跑过来抱着父亲大哭起来。经许家叔侄二人安慰劝说,张鸿渐父子才转悲为喜。许翁立即拿出银子和绸缎并写好信,派人送往御史那里,张鸿渐父子于是一同回家。

方氏自从得到儿子中举的喜报以后,天天为张鸿渐逃亡在外感到悲伤;忽然有人说新举人回来了,心里更加悲痛。不多时,张鸿渐父子一起进了家门,方氏大吃一惊,以为丈夫从天而降,当问知事情的经过后,全家人才悲喜交集。

某甲的父亲见张鸿渐的儿子中举显贵了,也不敢再萌发害人之心,张鸿渐却更加厚待他,又历述了当年出事的真实情景。某甲的父亲听了很受感动,并且非常惭愧,于是两家互相和解,成为朋友。


\subsection{1.9.30   太 医}
\label{\detokenize{p00_u5176_u5b83/_u767d_u8bdd_u804a_u658b_u5fd7_u5f02:id377}}
明朝万历年间,有个姓孙的评事官,很小的时候就死了父亲,母亲从十九岁就守寡。待到他考中进士时,母亲也去世了。他曾经对人说:“我必定要博一个‘诰命’称号,使九泉之下的母亲感到荣耀,才不负她老人家守了一辈子苦节!”不想孙评事忽然得了急病,很重。他平日与太医很好,就让人去把太医请来看病。派去的人刚出门,孙评事的病就越发加重了,他眼睁睁地说:“我生不能扬名显亲,死后有什么脸面见老母于地下!”话刚说完就咽了气,两眼还睁得大大的。

一会,太医来了,听到哭声,知道孙评事已去世,进去吊丧。见他死不瞑目的模样,心中很惊异。家中的人向太医说明了原因。太医说:“想得个‘诰命夫人’称号,这也不难。当今皇后马上就要生孩子,只要他再等十几天,诰命是可以得到的。”于是让家人立刻拿了艾条来,在孙评事的尸体上灸了十八处。艾条快要烧尽时,孙评事已在床上呻吟出声,急忙给他灌药,居然又活了过来。太医嘱咐说:“今后切记不要吃熊、虎肉。”家里人都牢牢记住了。但是,因为熊、虎之类的肉平时很少见,所以孙评事也不太在意。过了三天,他一切恢复正常,依旧随大家到朝中进行朝贺。

过了六七天,皇后果然生了太子,皇帝就赐群臣宴饮。宫庭中的侍从,拿出山珍海味遍赐文武大臣,见白片中尖有红丝,甜美无比,孙评事吃着,不知是什么东西。第二天,问他的同僚,人们说:“是熟熊掌。”孙评事大惊失色,继而得病,回到家就死了。


\subsection{1.9.31   牛 飞}
\label{\detokenize{p00_u5176_u5b83/_u767d_u8bdd_u804a_u658b_u5fd7_u5f02:id378}}
县里有个乡下人,买了一头牛,很是健壮。夜里,乡下人梦见牛生了两只翅膀飞走了。他觉得不吉利,怀疑这头牛会走失,第二天便把牛牵到市场上降价卖了。回来路上,乡下人把卖牛的钱用手巾包起来,缠在胳膊上。走到半路,见一只鹰正在吃一只死兔。走近一看,鹰很温顺,乡下人便用包钱的手巾头拴住鹰腿,用胳膊架着它。鹰屡次扑腾挣扎,乡下人稍一分心,鹰带着包钱的手巾腾空飞走了。这虽然是命中注定的事,但如果这乡下人不疑忌自己做的梦,路上也不贪财,那么本只会走的牛怎能飞走呢?


\subsection{1.9.32   王 子 安}
\label{\detokenize{p00_u5176_u5b83/_u767d_u8bdd_u804a_u658b_u5fd7_u5f02:id379}}
王子安,是东昌府的名士,但屡次科考不中。一次,他考过试后,眼巴巴地盼着考中的消息。快临近发榜时,他痛饮一场,喝得酩酊大醉,回家后睡在卧室里。忽然有人喊道:“报马来了!”王子安踉踉跄跄地爬起来说:“赏十千钱!”家里人因为他醉了,骗他安慰他说:“你只管睡下,已经赏了。”王子安才又躺下。一会儿,又有个人进来说:“你考中进士了!”王子安自言自语:“还没去京城殿试,怎么中了进士?”来人说:“你忘了吗?三场已考完了!”王子安大喜,跳起来大叫着说:“赏十千钱!”家人又像上次那样哄着他睡下。

又过了一会儿,一个人急急忙忙跑进来说:“你已点了翰林,跟班在这里伺候!”王子安一看,果然见两个人在床下拜见,衣着都很整洁。王子安又大叫赏给跟班酒饭。家人又骗他,心里暗笑他醉得太厉害。过了很久,王子安自己想,既然做了大官,不可不出去夸耀夸耀,便大叫跟班。叫了几十声,却没人答应。家人笑着说: “你先躺着,我们去找他们。”又过了很久,跟班果然来了。王子安捶床跺脚,大骂跟班:“蠢奴跑哪里去了!”跟班发怒地说:“你这个无赖!刚才不过是跟你玩玩罢了,你倒真的骂起来!”王子安大怒,从床上一跃而起,去打跟班,把他的帽子打落了,王子安也跌倒在地。他妻子走进来,扶起他来说:“怎么醉到这种地步!”王子安说:“跟班可恶,我所以惩罚他,怎么是醉了?”妻子大笑着说:“家里只有我这个老婆子,白天为你做饭,晚上替你暖脚,哪里来的跟班,会伺候你这把穷骨头!”孩子们都笑了起来。王子安这时酒醉也快过去了,忽如大梦方醒,一下子明白了刚才的事都是假的。但还记得跟班的帽子掉了,忙去门后寻找,果然找到了一顶像茶盅那样大小的缨帽。大家都很惊疑,王子安自我解嘲说:“过去有人被鬼揶揄,我现在则是被狐狸戏弄了!”


\subsection{1.9.33   刁 姓}
\label{\detokenize{p00_u5176_u5b83/_u767d_u8bdd_u804a_u658b_u5fd7_u5f02:id380}}
有一个姓刁的,家里没有产业,经常外出给人相面谋生——实际上他并不懂得相术。每次出去都是好几个月才回来一趟,袋子里总是装满了钱和布帛。众人都感到很奇怪。

一次,同村的一个人客居在外,远远地望见一家高门内站着一个人,穿戴打扮道貌岸然,嘴里正在滔滔不绝,四周围了许多妇女。村人走近一看,原来是刁某。他便偷偷地躲在一边,看刁某在干什么。只听围观的妇女中有一个人问道:“我们这些人中有一个贵夫人,你能辨认出来吗?”原来这些人中确有一个贵妇人,穿着普通衣服杂在众人中,要以此检验刁某的相术。村人不禁替刁某发窘。只见刁某从容地望着天空,用手指一划拉,说:“这有什么难辨的!是贵人的头顶上自然有云气环绕!”众人听了,不觉都向其中一人看去,看她头顶上有没有云气。刁某便指着那个妇人说:“这是真正的贵人!”众人非常惊讶,以为他是神仙。

村人回来后,述说了刁某那堪称机智的骗术。才知道这种人尽管操业不雅,但也必有过人的才气;不然,怎么能够骗过那么多人,赚取钱财,没本就能赢大利呢?


\subsection{1.9.34   农 妇}
\label{\detokenize{p00_u5176_u5b83/_u767d_u8bdd_u804a_u658b_u5fd7_u5f02:id381}}
淄川城西的磁窑坞有一位农家妇人,勇猛健壮如同男子一样,常常为乡里排除难题,调解纠纷。她和丈夫分居在两个县里,丈夫家在高苑县,距淄川一百多里;偶然来一趟,住两宿就走。农妇自己到颜山去,贩卖陶器为业。她有了多余的钱,便施舍给讨饭的人。

一天晚上,她正与邻家妇人说话,忽然站起来说:“我肚子稍微有点痛,想必是孩子要离身了。”于是就走了。天明后邻居妇人去看她,却见她肩挑着两个酿酒的巨瓮,正要进门。邻妇随着她进入屋内,看见有一个婴儿包裹着躺在床上。邻妇吃惊地问她,原来她分娩以后已挑着重担走了上百里路了。

农妇过去与村北边庵里的尼姑很要好,拜了干姊妹。后来她听说这尼姑有淫乱的行为,就气愤地抓起一根木棒要去打这个尼姑,众人苦苦劝阻才没有去。有一天,她在路上遇到了这个尼姑,赶上去就打。尼姑问:“我有什么罪过?”农妇也不回答,拳头、石块一齐向尼姑身上打去,直打得尼姑叫不出声了,才停手走了。


\subsection{1.9.35   金 陵 乙}
\label{\detokenize{p00_u5176_u5b83/_u767d_u8bdd_u804a_u658b_u5fd7_u5f02:id382}}
金陵某乙,卖酒为生,每次酿好酒后,都往酒缸里掺水,而且加进一些麻药。即使是很能喝酒的人,喝不上几杯,便烂醉如泥。由此,他的酒得到古时“中山”美酒的好名声,他也以此致富,家资万金。

有一天,某乙早晨起来,看见一只喝醉了的狐狸睡在酒槽边。他用绳子把狐狸的四肢捆起来,刚要去找刀,狐狸醒了,哀求说:“不要杀害我,你有什么要求,我都可以满足你。”某乙就给它解开绳子。狐狸在地上打了个滚,马上就变成了个人。

当时,同一条街上姓孙家的大儿媳妇,被狐狸缠上了,某乙就问狐狸精这件事。狐狸精回答说:“那就是我。”某乙见过大媳妇的弟妹,认为长得比大儿媳更美,便要求狐狸精携带他一同前往,狐狸精很为难。某乙固执地要求,狐狸精只得请某乙跟它一起走。来到一个洞里,狐狸取出一件褐色的衣服给某乙,说:“这是我去世的哥哥留下来的,穿上它就可以去了。”某乙随即穿上褐衣回家,家里人都看不见他。换上平常穿的衣服出来,家里人才看见他。某乙非常高兴,和狐狸一起来到姓孙的家中。见孙家墙上贴着一张巨大的神符,画面上画着蜿蜒曲折的一条龙。狐狸一见害怕地说:“和尚太厉害,我不进去了。”说完匆匆逃走了。某乙试探着走到近前一看,却是一条真龙盘踞在墙壁上,高昂着头跃跃欲飞。某乙大惊失色,也吓得赶紧跑了出来。原来孙家找来一位外地的和尚,为他们家作法驱妖。和尚先给了孙家一张画符带回,贴在墙上,和尚本人还没有到。

第二天,和尚来到,设下神坛,作起法来。邻居们都来观看,某乙也夹杂在里面。忽然他脸色突变,急忙奔跑,那样子就好像被人追赶捉拿。跑到门外,扑倒在地,立刻变成一只狐狸,四肢还穿着人的衣服。和尚要杀死它,某乙的妻子急忙叩头哀求。和尚叫某乙的妻子牵了回去。妻子每日给些吃的喝的,过了几个月,还是死了。


\subsection{1.9.36   郭 安}
\label{\detokenize{p00_u5176_u5b83/_u767d_u8bdd_u804a_u658b_u5fd7_u5f02:id383}}
孙五粒家有一个僮仆独自住在一间屋内,他感到恍惚之间被人提了去。到了一座宫殿,看见阎罗王坐在上面,仔细地看了看他说:“错了,不是这个人。”因此把他遣送回来。

回来以后,他心里十分害怕,就搬到另一间屋里去住了。这家另一个仆人叫郭安,看见床铺空着,于是就在床上睡了。这家还有个仆人叫李禄,与那个僮仆过去就结有怨仇,早就想报复。这天夜里拿着刀进入这间屋子,用手摸了摸,以为是那个僮仆,竟把他杀了。郭安的父亲就告到官府里。这时陈其善担任县令,很不同情郭安的父亲。郭父哀痛哭叫说:“我这半辈子就只有这一个儿子,现在让我依靠谁生活啊!”陈县令就判李禄做郭父的儿子。郭父只好含着冤仇回去了。这件事的奇特不在于僮仆见鬼,而奇特在陈其善的判决。

济南府西边某县有个杀人凶手,被害人的妻子告了他。县令大怒,拍着公案大骂说:“人家好好的夫妻,你竟然叫人家成了寡妇!现在就把你配给她做丈夫,也叫你老婆守寡!”于是就判决两人结成夫妻。这种“英明”的判决,都是进士出身的官员所办的,其它途径出身做官的人是办不出来的;而陈其善也这样断案,谁说官员中没有“人才”呢!


\subsection{1.9.37   折 狱}
\label{\detokenize{p00_u5176_u5b83/_u767d_u8bdd_u804a_u658b_u5fd7_u5f02:id384}}
淄川县的西崖庄,有一个姓贾的被人杀死在路上。隔了一夜,他的妻子也上吊死了。

贾某的弟弟告到了县官那里。当时浙江的费祎祉在淄川做县令,亲自去验尸。他看到死者布包袱里包着五钱多银子还在腰中,知道不是图财害命。传来两村的邻居审问了一遍,没有什么头绪,也没有责打他们,就把他们释放回去种地了。只是命乡约地保仔细侦察,十天向他汇报一次情况。

过了半年,事情渐渐松懈下来。贾某的弟弟埋怨费县令心慈手软,多次上公堂吵闹。费县令生气地说:“你既然不能指出谁是凶手,想叫我用酷刑拷打良民吗?”呵斥一顿,把他赶了出去。贾某的弟弟无处伸诉冤情,气愤地把哥哥嫂子埋葬了。

一天,因为逃税的缘故,县里逮来几个人。其中有一个叫周成的害怕责打,告诉县令说钱粮已经筹办足了。就从腰里取出银袱,交给费县令验视。费县令查看完了,便问他:“你家住在哪里?”回答说:“某村。”又问:“离西崖村几里路?”回答说:“五六里。”“去年被杀的贾某是你什么人?”回答说:“我不认识那个人。”费县令勃然大怒说:“你杀了他,还说不认识?”周成竭力辩解,费县令不听,严刑拷打,他果然认罪了。

原来,贾某的妻子王氏,要走亲戚家,没有首饰觉得羞愧,闹着叫丈夫到邻居家去借。丈夫不肯,妻子自己去借了。她非常珍重,回来的路上,从头上卸下首饰包在包袱里,塞进袖筒中。等回到家,伸手一摸,首饰没有了。王氏不敢告诉丈夫,又没有办法偿还邻居,懊恼得要死。这天,周成正巧拾到了首饰,知道是贾某的妻子丢的。乘贾某外出以后,周成半夜从墙上爬过去,想以首饰要挟和贾妻苟合。当时正是热天,王氏睡在院子里,周成悄悄走近她将她强奸。王氏醒觉,大声喊叫。周成急忙制止,留下包袱把首饰给了她。事情办完了,王氏嘱咐说:“以后不要来了,我家男人很凶,让他知道了,你我都得死!”周成怒冲冲地说:“我给你的东西够到妓院嫖好几宿的!难道只干这一次就能抵偿了吗?”王氏安慰他说:“我并不是不愿与你相交,我男人常常闹病,不如慢慢等他病死就行了。”周成走了,于是就杀了贾某;夜里又到王氏家说:“现在你男人已经被人杀了,请你按说的办!”王氏听了大哭起来。周成害怕惊动邻居,逃走了。天明后王氏也死了。费县令查明实情,将周成抵罪。

大家都佩服费县令断案神明,但不知所以能察明案情的缘故。费县令说;“事情并不难办,只是要随时随地留心罢了。当初验尸的时候,我见包银子的包袱绣着万字文,周成的包袱也一样,是出自一人之手。等审问他时,他又说以前不认识贾某,言词搪塞。神态异常,所以知道他就是真正的凶手了。”

淄川县有个叫胡成的,与冯安同一个村子,两家世代不和。胡家父子很霸道,冯安曲意同他交往,胡家却终不信任他。

一天,他们一块喝酒,略有醉意时,两人说了些心里话。胡成吹嘘:“不要忧愁贫穷,百把两银子的财产不难弄到手!”冯安认为胡成并不富裕,是在吹牛,故意讥笑他。胡成一本正经地说:“实话告诉你,我昨天在路上遇见一个大商人,车上装着很多财物,我把他扔进南山的枯井里了。”冯安又嘲笑他。当时,胡成有个妹夫叫郑伦,托胡成说合购买田产,在胡成家寄存了好几百两银子。这时胡成就全部拿出来在冯安面前炫耀,冯安相信了。散席以后,冯安偷偷地写了状纸告到县衙。费县令拘捕了胡成对质审问,胡成说了实情;费县令又问郑伦和产主,都说是这样。于是就一块去察看南山枯井。一个衙役用绳子吊着下去,竟发现井中果然有一具无头尸体。胡成大吃一惊,无法辩白,只能大喊冤苦。费县令生了气,命人打嘴几十下,说:“证据确凿,还叫冤屈!”用死刑犯的刑具将他锁了起来。却不让弄出尸体来,只是告知各村,让尸主呈报状子。

过了一天,有个妇人持状纸来到公堂,声称自己是死者的妻子,说:“我丈夫何甲,带着数百两银子出门做买卖,被胡成杀死。”费县令说:“井中确实有死人,但未必就是你丈夫。”妇人坚持说是。费县令就命把尸体弄出井来,众人一看,果然是妇人的丈夫。妇人不敢到跟前,站在远处号哭。费县令说:“真正的凶手已经抓住了,但尸体不完整。你暂时回去,等找到死者的头颅,立即公开判决,让胡成偿命。”接着把胡成从狱中唤出来,呵斥说:“明天不将头颅交出来,就打断你的腿!”叫衙役押他出去,找了一天回来,追问他,他只是嚎哭。费县令让衙役把刑具扔在他面前,摆出要用刑的样子,却又不动刑,说:“想必是你那天夜里扛着尸体慌忙急迫,不知将头掉到什么地方了。怎么不仔细寻找呢?”胡成哀求县官准许他再找。县令问妇人:“你有几个子女?”回答说:“没有。”县令问:“何甲有什么亲属?”“只有一个堂叔。”县令感慨地说:“年轻轻就死了丈夫,这样孤苦怜仃以后怎么生活呢?”妇人又哭起来,给县令磕头请求怜悯。县令说:“杀人的罪已经定了。只要寻找全尸,此案就完结了。结案后,你赶快改嫁。你是一个年轻少妇,不要再出入公门。”妇人感动得哭了,叩头下了公堂。县令立即传令村里的人,替官府寻找人头。过了一宿,就有同村的王五,报称已经找到了。县令审问查验清楚,赏给他一千钱。又把何甲的堂叔传到公堂,说:“大案已经查清,但是人命重大,不到一年不能结案。你侄儿既然没有子女,一个年轻轻的寡妇也难以生活,让她早点嫁人吧。以后也没有别的事,只有上司来复核时,你须出面应声。”何甲的堂叔不肯,费公从堂上扔下两根动刑的签子;再申辩,又扔下一签。甲叔害怕了,只好答应后退了下去。妇人听到这个消息,到公堂谢恩。费县令极力安慰她,又传令:“有谁愿买这妇人,当堂报告。”妇人下堂后,就有一个来投婚状的人,原来就是找到人头的王五。县令传唤妇人上堂,说:“真正的杀人凶手,你知道是谁吗?”妇人回答说:“胡成。”县令说:“不是。你与王五才是真正的凶犯!”二人大惊,极力辩白,叫喊冤枉。县令说:“我早已知道其中详情!之所以一直到现在才说明,是怕万一屈枉了好人!尸体没有弄出枯井,你怎么能确信就是你丈夫?这是因为在此以前你就知道你丈夫死在井里了!况且何甲死的时候还穿着破烂衣服,数百两银子是从什么地方弄来的?”又对王五说:“人头在哪里,你怎么知道得那样清楚?你之所以这样急迫,是打算早点娶到这妇人罢了!”两人吓得面如黄土,一句话也说不出来。费县令用刑拷问二人,果然吐露了真情。原来王五与妇人私通已经很久,两人合谋杀了她的丈夫。恰巧碰上胡成开玩笑说杀了人,二人才想嫁祸于胡成。费县令于是释放了胡成。冯安以诬告罪,打了顿板子,判了三年劳役。直到案子结束,费县令没有对一个人乱动刑罚。


\subsection{1.9.38   义 犬}
\label{\detokenize{p00_u5176_u5b83/_u767d_u8bdd_u804a_u658b_u5fd7_u5f02:id385}}
周村有个商人,在芜湖经商,赚了很多钱。他雇了一条船准备回乡,看见河堤上有个屠夫捆住一只狗要杀。这个商人就以加倍的价钱把狗买了下来,养在船上。

船上的船夫本来就是江湖上的惯盗,他暗中观察到商人有这么多钱财,便把船开到芦苇丛中,拿起刀来要杀死商人。商人苦苦哀求船夫赐他一具完整的尸体。于是强盗就用一条毡子把商人裹捆住,扔到江里去了。

那只狗看到商人被抛入江中,哀嚎踵跳下水,用嘴咬住裹捆着商人的毡子,一起在江中沉浮。也不知顺流飘荡了多少里,被一浅滩搁住停了下来。狗浮出水,跑到有人的地方,不停地哀叫。有人觉得其中必有原因,就跟随着这只狗走到了浅滩处,见水中有一捆毡子,于是就拖出来,割断绳子,商人竟还没死,醒过来后把自己遇难的事情讲了一遍。又哀求别的船夫,把他带回芜湖,准备在那里等着强盗的船回去。

商人上了船,发现他的狗不见了。心里非常哀伤痛惜。到达芜湖码头,寻找了三四天,只见经商的船只桅杆如林,就是找不到那只贼船。这时正好有个同乡,打算带着他一块回周村。忽然那条狗自已回来了,朝着商人大声嗥叫。商人忙唤它,它却掉头就走。商人下船去追它,它却奔上另一条船,咬住船上一个人的小腿,任凭怎么打也不松口。商人走上前去呵斥,才发现狗咬住的就是那个劫财害命的惯盗。原来这个强盗把衣服和船都换了,所以商人很难认得出来。商人把惯盗捆绑起来,在船上搜索,结果钱财都还在。唉,一条狗,尚能够如此报恩,世上那些没有心肝的人,应当惭愧自己还不如一条狗呀!


\subsection{1.9.39   杨 大 洪}
\label{\detokenize{p00_u5176_u5b83/_u767d_u8bdd_u804a_u658b_u5fd7_u5f02:id386}}
杨涟,字大洪,是湖北应山人。他在没有做官以前,就颇有名气,自命不凡。有一次科试考完之后,听到报优等的人来了,当时他正吃着饭,嘴里还含着一口,就急忙跑出去问道:“有姓杨的吗?”来人回答说:“没有。”杨大洪灰心丧气,一口饭咽下去,到了胸膈那里搁住了。于是形成了病块,噎阻得很痛苦。大家劝他去省府参加录遗考试;他忧虑没有费用,大家给他凑了十两银子,才勉强上了路。

夜里,梦见一个人对他说:“前面的路上有人能把你的病治好,要苦苦哀求他。”临走时赠给他一首诗,其中有“江边柳下三弄笛,抛向江中莫叹息”两句。到了第二天,杨大洪在住宿的地方,看见一个道士坐在柳树下面,便上前叩拜,请求道士给他治病。道士笑着说:“你找错人了!我哪能会治病呢?为你吹三首曲子倒可以。”说着取出笛子,吹了起来。杨大洪忽然想起梦中的情景,就越发向道士哀求,并且把身上所带的银子都恭敬地递给他。道士接过来就扔到江里去了。因为银子来得不容易,杨大洪心里感到很可惜。道士说:“看样子你是有点心疼,不要紧,银子就在江边,你自己去捡回来吧。”杨大洪走到江边一看,银子果然在那里。心中更加感到奇怪,称呼他是仙人。道士随便用手一指,说道:“我不是仙人,那地方有个仙人来了。”杨大洪回头看时,道士用力拍打了一下他的头颈,说:“你太俗气了!”杨大洪受了意外一击,嘴唇立刻张开,发出咕噜咕噜的声音;接着吐出一块东西,落到地上发出吧嗒的响声。他弯下腰打开它一看,原来是他咽下去的那口饭,血丝包着;他顿觉伤痛好像去掉了。回头再看那个道士,已经不见踪影了。


\subsection{1.9.40   查 牙 山 洞}
\label{\detokenize{p00_u5176_u5b83/_u767d_u8bdd_u804a_u658b_u5fd7_u5f02:id387}}
山东章丘县境内有座查牙山,山上有个像井一样的石窟,深好几尺。石窟北壁上有个洞门,趴在石窟边沿伸下头去就能看见它。

正好附近村里的几个人九月九日重阳节登高,来到这里饮茱萸酒,便共同商议要进石窟探探里面的情况。其中的三个人接过蜡烛来,用绳子缒着下到了石窟底。见北壁洞内高大宽敞,和大屋一样;往里走了几步,变得稍微狭窄了些,再往前走,忽然到了尽头。洞尽头的底部有一个小窟窿,人可以爬进去。用烛光照了照,里面黑糊糊的深不可测。其中的两个人没有勇气再往前走,退了出来;另一个人讥笑他俩胆小,夺过蜡烛,自己缩紧了身体从小窟窿里钻了进去。

幸好狭窄处仅有一堵墙那样厚,钻进里面就忽然又高大又宽敞了。他便站起身来,继续往前走。头顶上的石头参差不齐,非常凶险,像是要坠落下来的样子。两边的洞壁陡峻重迭,就像寺庙里的塑像,都成鸟、兽、人、鬼的形状:鸟像要飞,兽像要走,人有的像坐有的像立,鬼怪显现出忿怒的样子,奇奇怪怪,大都是难看的多,好看的少。他见了心情紧张恐怖起来。好在脚下的路很平坦,没有坑坑洼洼的地方。

向前慢慢地走了几百步,见西边沿壁上开了个石室,门左边有一个怪石鬼,朝他站着,瞪着两眼,嘴像簸箕那样张开着,牙齿和舌头狰狞凶恶地露在外面;它左手攥拳,撑在腰间;右手叉开五指,像要扑人。这人心里非常惊恐,身上的毛发直竖起来。远远地看到石室门内有燃烧过的炭灰,知道有人曾经到过里面,胆子才稍微壮起来,强硬着头皮走了进去。

他见地上摆着些碗和酒盅,里面积存着泥垢;然而都是近今的器物,不是古窑货。旁边放着四把锡酒壶。他想得了这个便宜,便解下根带子拴住酒壶脖子系在自己腰间。接着又向一旁看去,只见一具尸体躺卧在西边角落里,两只胳膊和两条腿向四下里直伸着。他害怕极了。慢慢细看,尸体脚蹬尖头鞋,鞋底上刻的梅花还留存着,知道这是个年轻的妇人。却不知她是哪村的,更不知她死在哪一年。女尸的衣服颜色已经变暗腐败,分辨不出是青还是红来;她的头发蓬松着,就像一筐乱丝,粘附在髑髅骨上;头骨靠下有眼鼻孔各两个;两排牙齿白森森的,知道这是嘴。他琢磨着女尸头顶上一定会有金银珠宝首饰,就用蜡烛靠近她的脑袋。忽然觉得女尸嘴里像有气吹灯,烛光摇晃不定,火焰呈现昏黄色,自己的衣服也被吹得掀动起来。他这时真是吓坏了,手一颤抖摇晃,蜡烛顿时熄灭了。

他在黑暗中凭记忆顺着来时的路急忙往回奔跑,不敢用手去摸洞壁,恐怕碰到鬼物。不料他的头撞到了石头上,一下子跌倒在地。他立即爬了起来,觉得有些又湿又冷的东西顺着脸颊流到下巴颏上,知道是血,也没感到疼痛,克制着不敢呻吟;喘着粗气跑到了那个小窟窿边,刚要趴下,好像突然被人抓住了头发,他一下子就昏死了过去。

众人坐在石窟边上等了很久不见这人出来,怀疑他出了事,便又用绳子把原来那两个人缒了下去。一人把身子探进小窟窿里一看,见这人的头发挂在石头上,满脸血淋淋地倒在那里已经昏迷了。二人大惊失色,又不敢钻进去,只好坐在一边发愁叹气。不一会儿上面又让两个人缒了下来;其中有个大胆的,才很快钻进去,把他拖了出来。

这人被弄出石窟放在山上,过了半天才苏醒过来,他把在洞内见到的情景一条一条很详尽地说给众人听。所遗憾的是未能走到洞的尽头;若能走到尽头的话,一定会有更好的景象。后来章丘县令听说这件事,派人用泥团把石窟洞内的小窟窿封死,不让人再钻进去了。

康熙二十六七年间,养母峪的南石崖崩塌了,出现了一个洞口。人们从一旁观望,见里面的钟乳石林林总总犹如密密麻麻的竹笋。但是洞内又深又险,没有人敢进去。

忽然有个道士来到这里,自称是仙人钟离的弟子,他说:“师父派我先到这里,来清扫洞府。”村人们给他提供了灯火,道士带着它就下去了,没想到他不小心掉在了石笋上,被穿透肚子死去了。人们报告了县令,县令派人封死了洞口。洞内一定会有奇特的境界,可惜道士死了,没听到回音罢了。


\subsection{1.9.41   安 期 岛}
\label{\detokenize{p00_u5176_u5b83/_u767d_u8bdd_u804a_u658b_u5fd7_u5f02:id388}}
长山刘鸿训刘中堂,有一次同一位武官一块出使朝鲜。他们听说朝鲜的安期岛是神仙居住的地方,就想乘船去游览。朝鲜国的大臣们都说不行,让他们等待一个叫小张的人。原来安期岛不与人世间往来,只有岛上的弟子小张,每年来一两次。想到岛上去的人,必须先向小张说明,小张以为可以去,坐上船便可一帆风顺安全到达;否则,船就会被飓风打翻。

过了一两天,朝鲜国王召见刘中堂。中堂上朝后,看见还有一人坐在殿上。这人三十来岁,身佩宝剑,头戴棕斗笠,仪容整洁,神情庄重。刘中堂一问,知道他就是小张。中堂便向他讲述了自己想去安期岛的愿望。小张允许了,但又说:“你的副使不能去。”接着他又出了宫殿把刘中堂的随从看了一遍,说只有两个人可以跟着去。于是,小张备好船,领着刘中堂等人一块去了。

刘中堂坐在船上,也不知道路程有多远。只觉得风声习习。如同腾云驾雾,只过了一个时辰就到了安期岛。当时正是严寒的冬天,可是到了岛上,却是气候温暖如春,鲜花开满山谷。小张领刘中堂进入洞府,见里面有三位老者正盘腿而坐。东西两旁的人看见客人进来,如同没有看见一样,只有中间坐着的老者起身迎客,相互见了礼。坐下后,老者叫小僮上茶。有个小僮拿着盘走了出去,洞外的石壁上有一把铁锥,锥尖插入石头中。小僮拔出铁锥,立刻喷出水来。小憧用杯子接住。接满后,又把铁锥插回原处。小僮把茶端到刘中堂面前。中堂见茶色淡绿,试着吃了一口,凉得牙齿打颤。他怕凉不喝了。老者看看小僮,示意他端走。小僮把茶杯拿去,把剩下的喝了;仍旧来到刚才的石壁前,拔出铁锥,重新接了一杯回来。刘中堂一尝这杯茶,觉得满口芳香,热气扑面,好像刚刚烧出来似的,他暗暗惊异。刘中堂问老人自己以后的命运如何,老者笑着说:“我们世外人连岁月都不知道,怎能预知人世间的事?”刘中堂又问不老之术,老者说:“这可不是你们富贵人所能做到的。”刘中堂起身告辞,小张仍然送他回去。

回到朝鲜后,刘中堂向国王讲述了自已在安期岛的见闻,国王叹息说:“可惜你没有饮那杯凉茶。那是天上的玉液,喝一杯就可以增寿百年。”

刘中堂准备回国了,朝鲜国王赠他一件礼物,用纸帛层层包着,还嘱咐他不要在靠近海的地方打开。刘中堂刚一下船上岸,就急忙拿出来看,一连拆去好几百层纸帛,才看见一面镜子。他仔细地看着镜子,见镜子上出现了海中龙宫景象。里面龙飞蛟舞,历历在目。他正看得出神,忽见海上翻起比楼阁还高的浪潮,气势汹汹地向他扑来。刘中堂怕极了,急忙逃窜。浪潮紧追不放,快得如同狂风暴雨。刘中堂吓慌了,急忙把镜子向海潮扔过去,海潮马上就落了下去。


\subsection{1.9.42   沅 俗}
\label{\detokenize{p00_u5176_u5b83/_u767d_u8bdd_u804a_u658b_u5fd7_u5f02:id389}}
李季霖曾代任沅江县令。刚到任时,见大堂上满是狗、猫,他很惊讶。下属官吏告诉他:“这是乡中的老百姓,来瞻仰大人丰采的。”过了一会儿,已经有一半的猫狗变作人;又过了一会儿,猫狗都复原成了人,纷纷离去了。

有一天,李季霖出门会客,坐着轿子正走在路上,忽然有一个轿夫急乎乎地说:“小人受到伤害了!”就请别人替他抬轿,自己跪下向李季霖请假。李季霖生气地呵斥他,轿夫不听,疾跑而去。李季霖派人跟着他。轿夫跑到集市上,找到一位老头,请他诊治。老头看着他说:“你是受到伤害了。”于是就用手揣按他的皮肉,自上而下地用力推按;推到小腿,见皮下有肉团耸起,用锋利的刀割开,从里面取出一枚石子,说:“好了。”于是轿夫就跑着回来了。后来听说这个地方有个风俗,有的人身子躺在自己的卧室里,手就能飞出去,进入别人家的房门,偷窃财物。假若被主人发觉,拴住他的手不让它回去,那么这个人的一只手就残废不中用了。


\subsection{1.9.43   云 萝 公 主}
\label{\detokenize{p00_u5176_u5b83/_u767d_u8bdd_u804a_u658b_u5fd7_u5f02:id390}}
安大业,是河北卢龙县人。他生下来就会说话,他母亲用狗血灌他,才止住了。长大后,生得很秀美,同辈中没有比得上他的;而且读书很聪慧,名门大家争相向他提亲。他母亲做了个梦,说:“儿子当得公主为妻。”

安大业很相信,直到十五六岁,也没见梦得到验证,慢慢地懊悔了。

一天,安大业独自坐在房间里,忽然闻到一股奇异的香气。接着一个婢女跑了进来,说:“公主来了。”说完用一条长毡铺在地上,从门外一直铺到床前。安大业正在惊疑之际,一位女郎扶着婢子的肩头走了进来。她的容貌与衣服的丽彩,光照四壁。婢子赶快将刺绣的垫子铺在床上,扶着女郎坐下。安大业见此情景,仓皇得不知怎么办才好。施过礼便问:“何方的神仙,光临寒舍?”女郎微笑,用袍袖掩着口。婢女说:“这是圣后府中的云萝公主。圣后看中了你,想把公主嫁给你,因此让公主自己来看看你的住宅。”安大业非常惊喜,不知该说什么话。公主也低着头,相对默默无语。安大业原来就好下棋,围棋经常放在自己座位的旁边。婢女用一条红手巾,拂去棋子上的浮尘,将棋盘拿到桌上,说:“公主平日很喜欢下棋,与驸马一块下,不知谁能胜?”安大业便把座位移到桌边,公主笑吟吟地与他下起来。刚下了三十多着,婢女就将一盘棋搅乱了,说:“驸马已经输了。”把棋子一个一个地收到盒子里,说:“驸马是世间的高手,公主只能让六枚子。”便在棋盘上摆上六枚黑子,公主也依从,与安大业再下。

公主坐着的时候,总是让一位婢女伏在桌下,把脚放在她的背上;左脚着地的时候,便换一个婢女在座位的右边伏着,公主将右脚放上。此外,还有两个丫鬟在左右服侍着。每当安大业凝思考虑时,公主就弯曲着肘靠着丫鬟的肩头。棋局到末尾,还未决出胜负,小丫鬟说:“驸马输了一子。”婢女接着说:“公主疲倦了,该回去了。”公主便倾着身子与婢女说了几句话。婢子出去,不多会儿就回来,把很多钱放在床上,告诉安生说:“刚才公主说,你住的这房子狭窄潮湿,麻烦你用这些钱把宅第修饰修饰。房子修好后,再来相会。”一婢女在一旁说:“这个月是犯天刑的,不宜建造;下个月吉利。”公主起身欲走,安生急忙起身,挡住去路,把门关上。只见婢女取出一件东西,样子很像皮排,就地吹起来,冒出团团云雾。立刻,四处云气合笼,昏暗中什么也看不到;再找时,公主婢女丫鬟已经不见了。

安生的母亲知道后,很疑心是妖怪。安生却夜思梦想,再也舍不得云萝公主。他急于将房舍修葺完好,也没有时间去考虑犯不犯天刑,日夜催促着赶修,限定日期,终于把房子修整一新。

这以前,有个滦州的书生袁大用,侨居在安大业家邻近的巷子里,曾经持名帖来访过。安生平素很少与人交往,便托故他出;又乘袁生不在家时,去回访他。一个月后,二人在门外正好相遇,见袁大用是个二十多岁的少年,穿一身宫绢单衣,扎着丝织的带子,穿着黑色的鞋,看上去意态幽雅。安大业稍稍与他谈了几句,觉得他很温厚而且正派。安生很喜欢他,就很礼貌地请他进屋里坐。二人进了屋,安大业请袁生与他下围棋,二人互有胜负。接着,就设酒相待,谈笑得很欢洽。

第二天,袁大用就请安生到他的寓所,摆出山珍海味,殷勤招待。袁家有个十二三岁的小僮,能拍着手板唱清新的歌,又能跳跃蹦腾,作出各种各样的技艺。安生饮得大醉,袁生就让小僮背着他回去。安生认为小僮身体纤弱,恐怕他背不动,袁生却坚持要这样做。果然,小僮绰绰有余地把他背回了家。安生感到很奇怪。第二天,安大业赠小僮银子,以表示对他的奖励。小僮推辞了几次,才收下。

自这以后,安生与袁生关系越来越密切,三两日就互访一次。袁生为人沉默寡言,但慷慨好施。集市上有因欠债而出卖女孩子的,他解囊代为赎回,一点不吝啬。安生以此就越发尊重他。过了几天,袁生到安生家和他告别,赠给安生象牙筷子、楠木珠等十余件礼物、银子五百两帮助安生修房。安生把五百两银子退给他,并赠送给袁生一些绢帛之类的礼物。

袁大用离别后一个多月,有一位从乐亭县归乡的官宦,袋子装满了搜刮来的钱财。一天夜里,忽然来了一群强盗,把主人捉起来,用烧红的铁钳烫他,将钱财抢劫一空。家中有人认出了袁大用,告到官府,下文追捕。安大业的邻居有位姓屠的,一向与安家关系不好,因为安家大兴土木,起屋修房,他暗地怀有疑心。刚好安大业有一个小仆人偷得主人的象牙筷子,到屠家去卖,屠家得知这是袁大用赠的礼物,就告了官府。县令用兵把安大业家房子围起,正巧安大业与仆人有事外出了,官府就把他的母亲捉去。安大业的母亲年事已高,身体又不好,受惊后,气息奄奄,二三天滴水未进,县令只好将她释放回家。

安大业在外听到母亲被捉的消息,急忙赶回家中。但母亲的病已经很重了,过了一宿,就死去了。安生将母亲刚收殓,就被捉进官府。县令见安生年少又温文尔雅,暗暗地就认为这是诬告,是冤枉的,于是故意大声地恐吓他。安大业把自己与袁大用交往的过程说了一遍。县令问:“你为什么会暴富起来?”安生说:“我母亲自己有一笔积蓄,因我要娶亲,所以拿出来修葺那些结婚用的房子。”县令听信了,就把口供誊录下来,把他解送到府中。那个生屠的邻居,听知安大业无事,就设计贿赂押送的公差,让他在路上把安大业杀死。公差押着安大业进府,路经一座深山,安被公差拖到一峭壁上,准备将他推下去。正在危急的时候,忽然草丛中跳出一只猛虎,把两个公差咬死,口衔安生而去。

到了一个地方,楼阁重重,虎进去,将安生放下。但见云萝公主扶着婢女出来,见了安生,凄切地安慰他说:“我本想把您留在这里,可是母亲的丧葬未毕。现在,你只好拿着押解你的公文,到郡中去自投,保证你无事。”于是就取下安生胸前的带子,打了几个结,并吩咐说:“你见官时,解开这扣结,便可以免祸。”

安生按照云萝公主的吩咐,到郡中自投。太守很喜欢他的忠诚老实,又查了公文,知道他冤枉,就销了他的罪名,让他回家。在回来的路上,遇到了袁大用。安生下马与袁相见,把全部情况都告诉了他。袁听后很气忿,但一言未发。安生说:“以你这样的人才,为什么干这种事情玷染自己的名声?”袁大用说:“我所杀的都是不义之人;所取的也是些非义之财。否则,钱财就是丢弃在路上,我也不取。你的劝告当然是对的,但像你的邻居屠姓这种人,难道还要把他留在人世间!”说完话,就先走了。

安生回到家中,殡葬了母亲,就闭门不出,不再与外界交往。忽然一天夜里,有盗进入邻居屠姓家,把父子十余口全部杀掉了,只留下一个婢女。并且把他家中的财物席卷一空,与一个小僮分拿着。临走时,盗贼用手拿着灯对婢女说:“你要认清,杀人的是我,与别人无关。”他并不从门里走,而是从屋檐下越墙而去。第二天,婢女告到官府,官府怀疑安生知道内情,又把他提了去。县令审问时声色俱厉,安生上公堂,用手握着胸前的带结,边说边解。县令说服不了,又把他放了。

安大业回到家中,更加收敛自己的举止,在家中专心读书,从不外出。家中只留一位跛脚的老婢子为他作饭。他给母亲服孝期已满,每天都打扫台阶、房屋,以等待好消息的到来。一天闻到异香满园,到楼上一看,内外陈设焕然一新。偷偷揭开画帘,见云萝公主已盛妆坐在里面。安生急忙拜见。云萝公主挽着安生的手说:“你不信天数禁忌,建造房屋,酿成灾祸。又因母亲去世,服孝三年,耽误了我们三年。这是越想急于求成,反而越推迟。天下的事,大都是这样啊。”安生要出钱办酒席,公主说:“不再需要了。”婢子从食盒中拿出的菜肴,如同刚出锅的一样。酒也芳洌醉人。二人饮了一会儿酒,天渐渐黑了下来。公主脚下踏着的婢女也渐渐地都走了。公主四肢显出娇懒的体态,脚与腿似无着落。安生亲昵地抱起她,公主说:“你暂放手,现在有两条路由你选择。”安生揽着公主的脖子问她有什么事。公主说:“我们俩假若以棋友而交往,可相聚三十年;假若以床第之欢而交往,只能有六年的相聚时间。你取哪一条?”安生说:“六年以后再说吧。”公主默默无语,二人便共同入寝。公主说:“我本来就知道你是不能免俗的,这也是运数。”

公主让安大业蓄养婢女和佣人,让他们另外居于南院,每天干些做饭、纺织之类的活,以此维持生计。公主所居住的北院从来不见烟火,只有棋盘、酒具一类的东西。门也常关着,安生来推门时,门就自开,其他人是进不去的。然而,南院婢女、佣人作事,谁勤快谁懒惰,公主自己都知道。常常告诉安生去责备她们,没有不服气的。公主说话不多,也从不大声说话,别人和她说话,她只是低头微笑。每当并肩坐着的时候,总喜欢斜着身子靠在别人的身上。安生把她举起放在膝头上,就好像抱着个婴儿一样轻。安生说:“你这样轻,真可在掌上起舞。”公主说:“这有什么难!但那是婢女干的事,我是不屑去作的。赵飞燕原是我九姐姐的侍儿,每每以轻佻而获罪,触怒上界仙人,被贬谪到人世间。她又不肯守女子的贞节,现在已经把她幽禁起来了。”公主住的阁子用锦帛作帷幕围起,冬天不觉寒冷,夏天不觉太热。公主在严冬都带着轻纱。安生给公主做鲜艳的新衣服,强让她穿上。过了一会,公主就把衣服脱了下来,说:“这是尘世间俗浊的东西,让它压得我的骨头几乎得病!”

一天,安生把她抱到膝头上,忽然觉得比往日沉重,感到惊异。公主笑指着肚腹说:“这里头有一个俗子的种了。”过了几天,公主经常皱眉头,不想吃饭,说:“近来胃口不太舒服,很想吃点人间的饮食。”安生于是给她备下很好的饮食。公主从此吃饭,如平常人一样。

一天。公主说;“我的身体单薄瘦弱,不能承受生孩子的劳苦。婢子樊英身体很强壮,可以让她代替我。”于是公主便把她贴身的衣服脱下来,让樊英穿上,关在房子里。不大会儿,听到婴儿的啼哭声,开门进去一看,是个男孩。公主高兴地说:“这个孩子有福相,将来一定是个有出息的人才。”就给他取名叫大器。公主将孩子用被包好,放到安生的怀中,让他送给乳母,在南院中养着。

公主自分娩后,腰细得跟当初一样,又不再食人间烟火。忽然有一天,公主告诉安生,想回家看一看。安生问多长时间回来,回答说:“三天。”于是又像上次那样鼓起皮排,烟气四围,接着就不见公主了。三天之期已到,仍不见公主回来。又等了一年多,公主仍是渺无音信,安大业也就绝望了。

安大业关门读书,不久乡试考中举人。自公主去后,他始终不肯再娶,每每独宿北院,以沐浴公主的余芳。一天夜里,在床上辗转难睡,忽见院里灯火辉煌,映亮了窗口,门也自己开了。只见一群婢女拥着公主进来。安生很高兴,起来责备公主失约。公主说:“我并没有过期,按天上时间算的话,我才过了两天半。”安生很得意地告诉公主,他已中举。公主不高兴地说:“这种无意得来的东西,不能为你增多少光彩,只能减少人的寿命。三天未能见到你,你的俗气又加深一层。”

安生自这以后,再不去争进取了。过了几个月,公主又欲回家探望,安生凄楚地恋恋不舍。公主说:“这次去,一定早日返回,勿须盼望。你也要知道,人生在世,聚散都是有定数的。人的聚散,就好像过日子花钱一样,节制着花得时间长些;不节制恣意乱花,就用的日子短些。”公主去了,一个多月就返回来。从这以后,就一年半载地来一次,往往要住几个月才回去。安生也习惯了,不以此为怪。

不久,又生一个儿子,公主举起来说:“这个孩子是个豺狼。”立刻让安生把他扔掉。安生不忍,就把他留了下来,取名叫“可弃”。可弃才到周岁,公主就急于给他议婚。媒人们一个接一个地上门来。问可弃的生辰八字,都说不合。公主说:“我想为狼子设一深圈,竟然办不到。当该被他败坏六七年,这也是运数。”嘱咐安生说:“要记住,四年后,有个姓侯的生一女,在女孩右胁有个小赘疣,她就是可弃的媳妇,要娶过来,不要管门第如何。”就让安生写下来记住。

后来公主又回家探望,竟再也没回来。

安生常把这件事告知自己的朋友。后来得知,果然有一位侯姓家生了一女,左胁下有一疣赘。这位姓侯的品行下贱,行为不端,众人都看不起他,安生按公主的吩咐给可弃定下了这门亲事。

大器十岁考试及第,娶云氏女为妻,夫妻都孝顺和善,父亲很钟爱他们。可弃渐渐长大,不喜欢读书,而且善偷盗。常与无赖子弟混在一起赌博,常把自家的东西偷出去还债。安生很愤怒,便用棍子打他,可弃也终不改悔。安生告诉家人,都要提防他,不让他得到什么。可弃一天晚上出去,穿墙逾垣,被主人发觉,把他捆起来送到了官府。县官审询他的姓氏家庭,把他送回家中。他父亲与大器把他捆起来,严酷地拷打他,几乎断气。大器代他哀求,安生才把可弃放开。安生从此生气得病,饭食减退。就为两个儿子把家产分开,并写下文书,把楼阁与好的田地,都分给了大器。可弃怨恨,夜里持刀进屋,想把兄长杀死,却误杀了嫂子。先是,公主遗下一条裤子,很轻软,云氏很喜欢它,就改成一件睡衣。可弃用刀一砍火光四射,他大吃一惊,连忙逃走了。安生得知后,病情越加严重,数月就死了。可弃听到他父亲死的消息,才回到家中。大器对他很好,可弃却越加放肆。仅一年多时间,所分的田地全部卖光,于是可弃就到郡中去告大器。郡官很了解他这个人,把他赶了出去。兄弟间的情份从此断绝。

又过了一年,可弃二十三岁,侯氏女十五岁。大器忆起母亲的话,就想快些为可弃完婚。于是将可弃召到家中,把最好的房子腾出打扫于净,给可弃把侯氏迎娶进门。大器又把父亲留下的好田,都造册登记交给了他们,并对侯女说:“几顷薄地,为你死守到现在,今天全都交给你。我弟无德行,若是把一寸草给他,他也会给你卖掉。从此以后,成败如何,全在你这位新妇了。你若能够使他改恶从善,就不会忧虑受冻挨饿。若不然,我也无法填平你们这无底之坑。”侯氏女虽是小家所出,但很聪慧美丽,可弃既怕又爱她,她所说的话,没有敢违背的。每次出去,限时回来;若超过时间,侯氏就辱骂并不让吃饭。可弃因此行为也稍稍有所收敛。一年后,侯氏生了一儿子,说:“我以后无求于别人了。数顷肥沃良田,母子怎么还吃不饱?没有你这个男人,也可以了。”正遇到可弃偷了家中的谷子出去赌博,侯氏知道后,在门口弯弓搭箭,拒绝他进门。可弃很怕,就远避而去。看到侯氏进了门,他才磨蹭着走进屋里。侯氏又持刀出来,可弃掉头就跑,侯氏赶上砍了一刀,把他的衣服砍破,屁股上伤了一刀,血把袜子和鞋子都染红了。可弃气忿地去告诉兄长,大器理也不理。可弃自己只好冤屈惭愧地去了。过了一夜,可弃又到大器家,跪着哀求嫂子,求她给侯氏说情,让他回家。侯氏坚决不同意。可弃很愤怒,说要去把他老婆杀死,大器不说话。可弃忿然起来,手里持着一把刀径直走了出去。嫂子很惊骇,想上去制止他。大器使了个眼色,不要这样做。等到可弃去了,才对她说:“他故意弄个样子给我们看,实际他不敢回家。”使人偷偷地去看一下,可弃已入门。这时大器才变了脸色,想跑去看看,这时可弃正垂头丧气地走进来。原来,可弃进屋后,侯氏正在哄着孩子,望见可弃进来,把儿向床上一扔,到厨房找来一把刀。可弃害怕了,忙向外跑,侯氏将他赶出门才回去。大器得知内情后,还故意问可弃。可弃不说话,只是向着墙角哭泣,两个眼都肿了。大器可怜他,亲自领着他回去,侯氏才让他住下。等到大器出去后,侯氏罚可弃长跪,逼着他发誓,而后让他用瓦盆吃了饭。自此可弃才改邪归正。侯氏井井有条地管理家计,日子越来越富裕,可弃只是坐享其成而已。以后,年近七旬,子孙满堂,侯氏有时还捋着他的白胡子,让他跪着走。


\subsection{1.9.44   鸟 语}
\label{\detokenize{p00_u5176_u5b83/_u767d_u8bdd_u804a_u658b_u5fd7_u5f02:id391}}
中州境内有一个道士,到乡村去募化食物。吃过饭,听到黄鹂叫了一会儿,他告诉主人要谨防火灾。主人问他原因,他回答说:“我听到鸟说‘大火难救,可怕’。”大家都笑他,一点也不防备。

第二天,这家果然失火,火势漫延,烧了好几家,这才醒悟道士的神奇。有好事的人追上他,称他为神仙,道士说:“我不过能听懂鸟语罢了,哪里是什么神仙!”这时正巧有一只皂色的花雀在树上鸣叫,大伙问道士花雀说的什么,道士说:“花雀在说‘初六生的,初六生的,十四、十六就死了’,我想这家可能生了一对双胞胎,今天是初十,不出五六天,两个孩子会一起死掉。”人们到这家一问,果然生了两个儿子,没过多久,便都死了,日期和道士说的一样。

本县县令听说了道士的奇异,便把他召来,奉为上宾。正巧有一群鸭子经过,县令就问道士鸭子说了些什么,道士说:“您的内眷必有争闹的事。鸭子说‘罢罢罢,偏向他!偏向他!’”县令听了大为佩服。原来刚才县令的大老婆和小老婆争吵,县令刚被吵闹出来。于是县令就把道士留在县衙中,非常优待。道士时常辨别鸟语,大都被说中;而道士为人朴实粗鲁,说话直来直去,不知忌讳。县令非常贪婪,一切地方上供给衙门用的物品,他都折算成钱装入自己的腰包。一天,县令和道士正坐着,一群鸭子又过来了,县令又问道士。道士说:“今天它们说的同以前不同,它们在为您算帐呢!”县令问:“算的什么帐?”道士说:“它说‘蜡烛一百八,银珠一千八。’”县令很羞惭,怀疑道士在故意讥讽他。道士要求离开这里,县令不允许。

过了几天,县令设宴招待客人。忽然听到杜鹃的叫声,客人问道士,道士说:“鸟说‘丢官而去!’”客人们听了,愕然失色。县令大怒,立刻把道士赶出门去。时间不长,县令果然因贪污受贿被罢了官。呜呼!这是仙人在警告县令,可惜县令醉心于贪婪,最终也没有醒悟。


\subsection{1.9.45   天 宫}
\label{\detokenize{p00_u5176_u5b83/_u767d_u8bdd_u804a_u658b_u5fd7_u5f02:id392}}
郭生,是京都人,二十来岁,生得秀美潇洒,一表人才。一天傍晚,有个老太婆给他送来一坛酒。郭生奇怪这酒送得不明不自,老太婆笑着说:“不必问!只管喝,自有佳境!”说完便走了。郭生揭开酒坛一闻,香气清冽,便把酒都喝了。忽然大醉,昏沉沉地失去了知觉。等到醒来,觉得像跟一个人同睡在床上。用手摸摸,那人皮肤细腻如脂,芳香四溢,原来是个女子!郭生问她怎么回事,女子不说话;郭生便跟她交合起来。完事后,郭生摸摸墙壁,都是石头,还隐隐有股泥土的气味,极像是墓穴。郭生大惊,怀疑自己被鬼迷住了,便问女子:“你是什么神灵?”女子说:“我不是神,是仙。这里是我的洞府。我跟你有凤缘,你不要惊讶,只管耐心住在这里。往里再进一道门,看见有光亮的地方,那里可以小便。”一会儿,女子起床,关上门走了。

过了很久,郭生觉得肚子饿了。一会儿,来了个女仆,送来了面饼、鸭肉,让郭生摸黑吃饭。洞府里一片昏黑,也不知是白天是夜晚。不一会儿,那女子来睡觉,郭生才知道又到了黑夜了。郭生说:“白天没有太阳,晚上没有灯火,吃饭都找不着嘴。老这样下去,嫦娥跟罗刹鬼有什么区别?天堂跟地狱又有什么两样?”女子笑着说:“因为你是世俗中人,说起话嘴上没把门的,恐怕你泄露我们的事,所以我不愿让你看到我的容貌。况且,即使暗中摸索,美丑也该不同,又何需灯光!”

过了几天,郭生非常烦闷,屡次请求回去。女子说:“明晚我跟您游一游天宫,顺便作别。”第二天,忽然有个小丫鬟打着灯笼进来,对郭生说:“娘子等你很久了!”郭生便跟着她走了出去。只见灿灿的星光下,矗立着无数楼阁。经过好几重曲折的画廊,才来到一个地方:大堂上悬挂着珠帘,点着巨大的蜡烛,照得一片通明,像白天一样。走进去,见一个美人穿着盛装,朝南坐着,大约二十来岁,锦袍耀人眼目,头上的串串明珠,颤颤地四下垂着。地下摆了很多短蜡烛,连美人的裙子里边都照亮了,真是仙人啊!郭生见了,神志恍惚,不由自主地跪下了。美人命丫鬟拉起他来,让他坐下。一会儿,美味佳肴纷纷摆了上来。美人举杯劝酒说: “喝了这杯酒,为您送行。”郭生鞠了一躬说:“过去我见面不识仙人,真是惶恐惭愧!如果能容我赎罪,恳请您收我作您的忠诚奴仆!”美人听了,看着丫鬟笑起来,便命将酒席移到卧室里。卧室中挂着流苏绣帐,被褥又香又软。女子让郭生坐在床上,喝酒之间,屡次说:“你离家很久了,暂时回去一趟也无妨。”酒过数巡,郭生还是不说走。美人便让丫鬟打着灯笼送他,郭生不说话,假装醉了,躺在坐榻上,推也推不动。美人便让几个丫鬟给他脱光了衣服。一个丫鬟拍了下郭生的私处,说:“这男子相貌温雅,这东西怎么这样不老实!”丫鬟们把他抬起来扔到床上,大笑着走了。美人也睡下了,郭生在床上辗转反侧,美人问:“你醉了吗?”郭生说:“小生哪里是醉了?见了仙人,神魂颠倒罢了!”女子说:“这里不是天宫。明早趁天明,你应该早走。你既然嫌洞中幽闷,我们不如早点分别!” 郭生说:“好比现在有人夜间得到一株名花,鼻闻花香,手摸花枝,苦于没有灯光照着看看。这种情景令人怎能忍受!”女子笑了,答应给他灯烛。

直到四更,女子才叫丫鬟打着灯笼,抱着衣服送郭生回洞。进入洞中,在灯光下郭生见墙壁造得很精致,睡觉的地方铺了层一尺厚的皮褥。郭生解开鞋,盖上被子,见那个丫鬟在床边徘徊不走。郭生仔细一看,长得很美,便调戏她说:“说我不老实的,是你吧?”丫鬟笑着用脚踢了下他的枕头,说:“你该挺尸睡觉了,不要再多说!”郭生见她的鞋尖上镶嵌着许多菽粒大小的明珠,便一把捉住她的脚,丫鬟一下子扑倒在他的怀里,两个人便交合起来。丫鬟不断呻吟着,像是忍受不了。郭生问;“你多大了?”丫鬟笑着回答说:“十七岁。”郭生说:“处女也懂得情事吗?”丫鬟说:“我不是处女。但已有三年不跟人办这事了。”郭生又询问那美女的姓名、籍贯和家世,丫鬟说:“别问!这里既不是天上,跟人间也不同。如果你非要弄清楚,怕是死无葬身之地!”郭生听了,不敢再问。

第二晚,那美女来时果然带着蜡烛,二人一块吃饭,然后睡觉,从此习以为常。一天夜晚,女子进来说:“本想我们永远交好,没想到命运不济。马上就要清理天宫了,这里没法再收容你。请让我为你饯行。”郭生流下了眼泪,请求女子给些自用的梳妆品作为纪念。女子不答应,赠给他黄金一斤,明珠百颗。郭生三杯酒喝完,忽然昏睡过去。一觉醒来,觉得四肢像被捆上了,绳索密密麻麻、捆扎得十分紧密。腿也伸不开,头也转不动,极力挣扎,头一晕,摔倒在地下。伸手一摸,自已被用细绳捆在一个锦被做成的袋子里。他坐起身极力回想,看见屋里的东西,才知道是在自己的书房中。当时,他离家已经三个月了,家里人都以为他已经死了。郭生起初不敢说这件事,怕被仙人责罚,但心里却感到奇怪。后来他偷偷地讲给知己朋友听,没有一个能猜透是怎么回事的。那个用锦被做的袋子还放在郭生的床头上,散发出的香气充满了整个屋子。拆开一看,被套是用湖绵掺着香料做成的,郭生便珍藏了起来。后来,一个大官听说这件事,问了郭生经过,笑着说:“这是晋朝那个好淫的贾皇后曾经使过的伎俩,仙人怎会这样?虽然如此,这件事你一定要保守秘密,不能泄露。否则,会被夷灭三族的!”

有个巫婆曾经出入当时的显贵人家,说是郭生在“仙人”那里见过的那些楼阁形状,极像是严嵩的次子严世蕃家。郭生听说,恐惧万分,携家逃走了。不久,严嵩一家被诛,郭生才回家。


\subsection{1.9.46   乔 女}
\label{\detokenize{p00_u5176_u5b83/_u767d_u8bdd_u804a_u658b_u5fd7_u5f02:id393}}
平原县的乔生,有个女儿长得又黑又丑:豁鼻子,还瘸着一条腿,二十五六岁了,也没有来提亲的。同县有个穆生,四十多岁,妻子死了,家里很穷,无钱再娶,就出了一份微薄的彩礼,娶了乔女。三年后,生了一个儿子。不久,穆生死了。乔女家里更穷了,生活十分困难,就去乞求母亲接济。母亲很不耐烦,乔女生气,再不去娘家,只靠纺织维持生活。

有一个孟生,死了妻子,撇下个儿子叫乌头,刚满周岁,没人抚养,所以孟生急着再娶一房媳妇;可是媒人一连提了好几个,孟生都不中意。一天孟生偶然看见乔女,十分喜欢她,就找人暗中传信给乔女,愿意娶她。乔女推辞说:“我现在如此忍冻挨饿,嫁给官人可以得到温饱,怎能不愿意呢?但是我又瘸又丑,和别人不一样。我所能自信的是品德。再嫁第二个丈夫,官人图我什么呢?”孟生敬佩她是一位贤良女子,对她更加爱慕。便叫媒人带上封好的钱去找乔女的母亲商量。乔母很高兴,亲自到女儿家里,执意要女儿改嫁孟生。乔女坚决不答应。乔母很惭愧,向孟生表示,愿意把小女儿嫁给他。孟生的家人都很喜欢,孟生却不愿意。

过了不久,孟生突然得急病死了。乔女前去祭奠,哭得很悲哀。孟生本没有亲戚,他一死,村里的无赖都来欺负他家。家里的东西被拿光了,又谋划瓜分他的田产。家中的仆人也各自乘机偷了东西走了。只有一个老妈妈抱着孟生的儿子在灵堂帐幕中哭泣。乔女问明了原委,心中忿忿布平。听说林生同孟家很要好,乔女就登门对林生说:“夫妇、朋友是人间大伦。我因为很丑,被人看不起,只有孟生能了解我。以前我虽然拒绝了他的求婚,可我的心却早已许给他了。如今他死了,儿子又小,我当然应该报答知已。但是抚养孤儿容易,抵御坏人的欺侮就难了。如果因为孟生没有父母兄弟,就坐视他的儿子饿死,家产被抢光也不相救,那么五伦之中就可以不要朋友这一伦了!我所期待你的并不多,只要你写张状子告到县官那里。孤儿我来抚养。”林生说:“可以。”乔女便告辞回家。

林生按乔女的嘱托,准备写状子。那些无赖火了,要和林生动刀子。林生非常害怕,关上大门不敢出来了。乔女等了几天,不见动静,连忙去问,孟家的田产已经被分光了。乔女气极了,挺身而出,亲自去找县官告状。县官问乔女是孟生的什么人,乔女说:“你是一县之主,断案凭的是理。如果我告的不是真情,就是他的亲戚也逃脱不了罪过;如果是真的,就是过路人说了也可以听。”县官气她说话难听,训斥了一通把她赶出去了。乔女的冤屈无法伸述,就到一个乡绅家里哭诉。那乡绅听了,觉得乔女很义气,就替她到县官那里剖明是非。县官查明实情后,惩治了那些无赖,将孟家被抢走的东西又全要了回来。

有人提议,想留乔女住在孟家,就便抚养他的孤儿。乔女不肯,把孟家的房门锁起来,让老妈妈抱着乌头跟她一块回去,住在自家另一间屋里。凡是乌头的日常所需,乔女都是和老妈妈一块去孟家打开房门拿出粮食,替乌头置办,自己从不沾孟家一点光,依然抱着儿子过穷日子,和从前一样。

过了几年,乌头慢慢长大了。乔女给他请了老师,教他读书;自己的儿子则叫他学着干活。老妈妈劝她让儿子和乌头一块读书,乔女说:“乌头的费用是他自已的。我如耗费人家的钱教自己的孩子,我的心意怎么能说明呢?”又过了几年,乔女为乌头积攒了几百石粮食,给他娶了大户人家的女儿为妻。又整修了房屋,让乌头回自己家里生活。乌头哭着再三要求她一同去自己家住,乔女才依从了。但仍然自己纺线织布度日。乌头夫妇夺去纺织的工具,乔女说:“我们母子俩光吃不干活,怎么能安心呢?”就早起晚睡给乌头管理家务。让他的儿子去巡查庄稼,如同一个佣人。乌头夫妻有点小过错,乔女总是训斥责备,从不宽容。稍有不改,乔女就生气地要回去。直到夫妻俩跪下认错,悔过了,才罢休。不久,乌头考中了秀才。乔女又要告辞回家,乌头不答应,出钱为乔女的儿子娶了媳妇。乔女就把儿子分出去回家过。乌头留不住他,就暗地让人从附近村子里买了一百亩好地,送乔女的儿子走了。

后来,乔女得了病,要回去,乌头仍然不答应。看看病情越来越重,乔女嘱咐乌头说:“一定要把我葬在穆家!”乌头答应了。乔女死了以后,乌头用金钱买通了穆生的儿子,让她母亲同自己的父亲孟生葬在一起。到了下葬那天,只觉棺材特别沉,三十个人也抬不动。穆生的儿子忽然倒在地上,七窍流血,自己说:“不孝的儿子怎么能卖掉自己的母亲!”乌头害怕了,连忙跪下磕头祷告,乔女的儿子才好了。灵柩又停了几天,等把穆生的坟墓修好,乌头才把乔女同穆生合葬了。


\subsection{1.9.47   蛤}
\label{\detokenize{p00_u5176_u5b83/_u767d_u8bdd_u804a_u658b_u5fd7_u5f02:id394}}
东海里有一种蛤,饿了时,就游到岸边,两壳张开,从里边爬出一只小蟹。蟹身上系着一根很细的红线,能离开蛤几尺远寻找食物,吃饱后爬回去,蛤的两壳才闭起来。

有人如偷偷地把小蟹身上的红线剪断,蛤和小蟹就会一块死去。这也是自然界中的奇事。


\subsection{1.9.48   刘 夫 人}
\label{\detokenize{p00_u5176_u5b83/_u767d_u8bdd_u804a_u658b_u5fd7_u5f02:id395}}
河南彰德府有一位姓廉的书生,从小勤奋好学,可是很早就失去了父亲,家里十分贫穷。

有一天廉生外出,傍晚回家的时候迷了路。他走进一个村子,有一位老太太走过来问道:“廉公子到哪里去呀?夜不是很深了吗?”廉生正在惊慌害怕的时候,也来不及问这位老太太是谁,就请求借宿。老太太就领着他走去,进入了一所高大的宅第中。有个丫鬟挑着灯笼,引导着一位妇人出来了,年纪约有四十余岁,举止有大家风度。老太太迎上前去说:“廉公子到了。”廉生连忙上前拜见,妇人高兴地说:“公子清秀英俊,岂只是做个富家翁!”随即摆设酒宴,妇人在一侧陪坐,很殷勤地频频劝饮,而她自己虽举杯却未曾饮过酒,举起筷子也未曾吃过菜。廉生感到惶恐疑惑,屡屡打听她的家世。妇人笑着说:“我故去的丈夫姓刘,客居江西,因为遭到意外变故突然去世。我这未亡人,独自住在这荒僻的地方,家境也日益败落。虽然有两个孙子,不是像鸱鸮一样凶顽不驯,就是像驽骀一样愚钝无能。公子虽然和我们不同姓,但也是隔了一代的骨肉至亲。而且你生性忠厚诚朴,所以我很冒昧地和你相见。也没有别的事情麻烦你,我稍微存有几两银子,想请你拿去到江湖上做买卖,分得一部分利润,也比像案头萤那样,只知苦读清贫而死好多了。”廉生推辞说自己年轻,又是个书呆子,恐怕辜负了她的重托。刘夫人说:“你要打算好好读书,首先要解决生活问题。公子很聪明,到哪里去不可以?”于是命婢女取出银子来,当面交付八百多两。廉生十分惶恐,再三推辞。刘夫人说:“我也知道你不习惯作买卖,但是试着干一干,我想不会不顺利。”廉生顾虑这么多钱自己一人不能胜任,打算找一个同伙合作经商。刘夫人说:“不必这样,只找一个朴实谨慎、懂得商务的仆人,为公子跑腿办事就足够了。”于是她伸出纤长的手指掐算了一卦说:“找一个姓伍的吉利。”就叫仆人备马,装上银子送廉生出发,说:“到了腊月底,我洗干净杯盘,恭候给公子洗尘。”又转头对仆人说:“这匹马调理得很驯良了,可以乘骑,就送给公子吧,不要牵回来了。”

廉生回到家,才四更多天,仆人拴好了马就自己回去了。第二天,廉生多方寻找伙计,果然找到一个姓伍的人,于是用高价雇用了他。姓伍的曾多年出门经商,又为人耿直,办事认真。于是廉生把钱财全托付给他。两人来往跋涉于荆襄一带,年底才回来,计算一下,获得了三倍的利润。廉生因为得到姓伍的伙计的帮助很多,在工钱之外,另给了他一些赏赐。并商议着把这些赏钱分加在其它帐目内,不让主人知道。

他们刚刚回到家,刘夫人已经派人来迎请了,于是他们就与来接的人一起去了刘夫人家。只见堂上已经摆好了丰盛的筵席。刘夫人出来了,再三慰问他的劳苦。廉生交纳了钱财之后,就把帐簿呈交出来,刘夫人放在一边不看。一会儿大家入了席。还伴有歌舞音乐。在外屋也给姓伍的伙计摆了酒席,让他尽量喝醉了才回去。因为廉生没有家室,便留在刘夫人家守岁。

第二天,廉生又要求检查帐目,盘点财物,刘夫人笑着说:“以后不必这样,我早已计算好了。”于是拿出一本帐簿给廉生看,登记得十分详尽,连他赠给仆人的赏钱,也记载在上面。廉生惊愕地说:“夫人真是位神人啊!”

廉生住了几天,刘夫人对他的食宿照顾得十分丰盛,好像对待自己的子侄一样亲切。有一天,刘夫人在堂上设了酒席,一桌朝东,一桌朝南,堂下一桌朝西。刘夫人对廉生说:“明天财星照临,最适于远行。今天为你们主仆设宴饯行,使你们远行更有气派。”过了一会儿,也把姓伍的伙计叫来了,让他坐在堂下。一时之间,锣鼓齐鸣,一名女艺人呈上曲目单,廉生点唱了一出《陶朱富》。刘夫人笑着说:“这是一个好兆头,你一定能得到像西施一样贤惠的妻子。”宴会结束以后,仍把全部资财交给廉生,说:“这一次出门,不可受时间限制,不获得数以万计的巨利不要回来。我与公子凭借的是福气和命运,所信托的是心腹之人,你们也不必花费心思去计算了,你们在远方的盈亏,我自然会知道。”廉生答应着告辞出来。

他们俩到两淮一带作买卖,当了盐商。过了一年,又获得了数倍的利润。然而廉生爱好读书,做生意也不忘记书本,他结交的朋友也都是读书人。获得的利润已经很多了,廉生就想不干了。渐渐地把经商的重任全交给了姓伍的伙计。

桃源县一个姓薛的书生与廉生交情最好。有一次,廉生到桃源县去拜访他,可薛家全家都到别墅去了。天黑了他又不能再到别的地方去,看门人就把他请进去,扫床做饭招待他。廉生详细询问他主人的情况,原来这时正谣传朝廷要选良家女子,送到边疆去犒赏军人,民间便骚动起来。只要听说有没娶亲的年轻人,便也不请媒人,不订婚约,直接就把女儿送到家里去,甚至有人一晚上就得到两个媳妇。薛生也在最近和某大姓人家的女儿结了婚,恐怕事情喧哗轰动,被县令知道,所以暂时迁居到乡下去了。

初更将尽的时候,廉生扫扫床铺正要睡觉,忽然听见有好几个人推开大门直接进来了。守门的人不知说了句什么话,只听见一个人说:“相公既然不在家,那么屋里点着灯的是谁?”守门人回答说:“是廉公子,一位远方来的客人。”一会儿,问话的人进屋来了,这人穿戴整洁华丽,向廉生略一举手致礼,就打听他的家世。廉生告诉了他,他高兴地说:“我们是同乡呢,你岳父家姓什么?”廉生回答说:“还没有娶妻。”这人越发高兴,跑出去急忙招呼了另一位少年一同进来,很恭敬地与廉生见礼,突然说道:“实话告诉你:我们姓慕。今天晚上来,是把我妹妹送来嫁给薛官人,到了这里才知道这件事办不成了。正进退两难的时候,恰巧遇见了公子,这难道不是天意吗?”廉生因为不了解这两个人,所以踌躇着不敢答应。慕生竟然不听他说什么,就急忙招呼送亲的人。一会儿,两个老妇人扶着一位女郎进来,坐在廉生床上。廉生斜着眼睛一看,女郎年约十五六岁,美丽无比。廉生十分高兴,这才整整衣帽向慕生道谢,又嘱咐守门人去买酒,稍微表示一点殷勤款待的心意。慕生说:“我们的祖先也是彰德府人;母亲一族也是世代官宦人家,现在衰落了。听说外祖父留有两个孙子,不知道家境情况怎么样了。”廉生问:“你外祖父是谁?”慕生说:“外祖父姓刘字晖若,听说住在城北三十里之处。”廉生说:“我是府城东南人,离城北比较远,我的年龄又小,交游不广。郡中姓刘的人最多,只知城北有个刘荆卿,也是一位读书人,不知道是不是你外祖父的后人,但是他家已经很穷了。”慕生说:“我家的祖坟还在彰德府,常常想把父母的棺木送回故乡安葬,因为路费没有筹措足,固而迟迟未办成。现在妹子嫁给了你,我们回去的心意就决定了。”廉生听了,很爽快地答应帮助他们办好这件事。慕家兄弟都非常高兴,喝了几巡酒以后,就告辞走了。廉生打发走了仆人,移走了灯火,新婚夫妻恩爱缠绵,就无法用语言表达了。

第二天,薛生已经知道了这件事,就赶到城里来,收抬出另一个院落让廉生居住。廉生回到两淮,移交盘点完了之后,留下姓伍的伙计住在店铺里,自己装上财物返回桃源县,同慕家兄弟起出岳父母的遗骨,带着两家的妻儿,一起回到了彰德。

回家安置好了之后,廉生便装好银子去见主人。以前送他的那个仆人已经在路上等侯他了。廉生跟着他到了刘家,刘夫人迎出来相见,满面喜色地说道:“陶朱公载着西施回来了。以前是客人,今天是我的外甥女婿了。”摆下酒宴为他接风洗尘,对廉生倍加亲爱。廉生佩服刘夫人有先见之明,就问道:“夫人与我岳母关系远近?”刘夫人说:“不必问这事,时间长了你就知道了。”于是刘夫人就把银子堆在案子上,分为五份,自己拿了两份,说:“我要银子没什么用处,只不过是送给我的大孙子。”廉生因为太多,推辞不肯接受。刘夫人很难过地说:“我们家败落了,院子中的树木被人砍去当柴烧了,孙子离这儿挺远,门庭破败,麻烦公子经营操办一下。”廉生答应了,而银子只肯收一半。刘夫人强使廉生都收下,送他出门,流着泪回去了。廉生正感到迷惑怪异的时候,回头一看,宅第成了一片坟地,这才明白刘夫人就是妻子的外祖母。

回去以后,廉生拿出银子买了坟墓周围一顷地作为墓田,封土植树,修饰得壮观幽美。刘夫人有两个孙子,长孙就是刘荆卿;次孙名为玉卿,酗酒赌博,不务正业。弟兄俩都很贫穷。弟兄俩到廉生家感谢他为他们整修祖坟,廉生赠给他们一大笔银子。从此互相往来,最为密切。

一次,廉生对他们详细说了经商的情由。玉卿暗想坟墓中一定有许多银子,就在一天晚上,纠合了几个赌徒,掘开坟墓,搜索银子。剖开棺木露出了尸体,竟然一点银子也没得到,很失望地散去了。廉生知道坟墓被掘,就告知了荆卿。荆卿和廉生一起到墓地查验。进入墓室,就看见案上堆得满满的,以前所分的两份银子都在那里。荆卿要和廉生两人分了银子,廉生说:“夫人原来就是留在这儿等待赠给你的。”荆卿把银子装运回家,然后向官府告发了掘墓之事。官府查访缉拿得很严。后来有一个人出卖坟中玉簪,被抓获了,官府审讯追问他的同党,才知道是玉卿为首。县令要把玉卿处以极刑,荆卿代他哀求,仅仅免予处死。两家一起出力修缮,坟墓内外修饰得比以前更为坚固幽美。从此,廉生和荆卿家都富裕了,只有玉卿仍然像以前一样贫困。廉生和荆卿常常周济他,然而到底不够他赌博挥霍的。

有一天晚上,有几个强盗闯入了廉生家,抓住廉生追要银子。廉生收藏的银子,都按一千五百两铸成银锭,就挖出来给他们看,强盗们拿了两个。这时只有以前刘夫人赠送给廉生的那匹马在马厩里,强盗用它驮着银子走了,就逼廉生把他们送到村外野地里,才释放了他。村里众人望见强盗的火把离得不远,就呐喊着追上去,强盗吓跑了。大家追到那里一看,银子扔在路边,那匹马已经倒地变为灰烬。廉生这才知道马也是鬼物。这天晚上只丢失了金钏一枚。原来,强盗抓住了廉生的妻子,喜爱她美貌,就要奸污她,有一个带着面具的强盗大声呵斥阻止了他们,声音好似玉卿。强盗们就放开了廉生的妻子,只褪下她腕上的金钏而去。廉生因此怀疑是玉卿,然而心里又暗暗感激他。后来有一个强盗用金钏作为赌注,被捕役抓获,追问他的同党,果然有玉卿。县令大怒,把五种酷刑全用上了。玉卿的哥哥与廉生商议,想用重金贿赂官府使他免于死罪,他们还没有办成而玉卿就已经死了。廉生还经常照顾周济玉卿的妻儿。

廉生后来乡试考中了举人,几代都是富贵人家。唉!“贪”这个字的点、划、形象,十分接近“贫”字。像玉卿这样的人,可以作为前车之鉴。


\subsection{1.9.49   陵 县 狐}
\label{\detokenize{p00_u5176_u5b83/_u767d_u8bdd_u804a_u658b_u5fd7_u5f02:id396}}
陵县李太史家,经常看见瓶呀鼎的古玩摆设等物品不知怎么就挪到桌子边沿上,要掉下去的样子。他怀疑是下人们干的,常愤恐地责备他们。仆人说冤枉,可也不知原因。于是将物品放归原处,把门锁严了。可天明后又那样了,知是怪事,便暗中观察。

一天夜里,屋里忽然亮得很,还以为来了贼,很惊讶,两个仆人走进去看究竟。见一只狐狸躺在木柜上,两眼冒光,把四周照得亮亮的。怕它跑了,赶快去捉。狐狸咬仆人手腕想逃,仆人抓得更紧了。于是一齐动手绑了,抬起来看,见四条腿都没有骨头,手一碰,荡悠荡悠像带子垂着。李太史怜惜它的通灵,不忍杀掉。于是用柳筐盖住狐狸,狐狸出不来,只能顶着筐走。李太史数落了它的过错,把它放了。过去那种怪事就绝迹了。


\section{1.10   卷 十}
\label{\detokenize{p00_u5176_u5b83/_u767d_u8bdd_u804a_u658b_u5fd7_u5f02:id397}}

\subsection{1.10.1   王 货 郎}
\label{\detokenize{p00_u5176_u5b83/_u767d_u8bdd_u804a_u658b_u5fd7_u5f02:id398}}
济南有一个卖酒为生的老翁,一天,支使他的儿子小二去齐河讨酒债。小二刚出西门,忽然看见哥哥阿大——当时阿大已死了很久了。小二惊讶地问:“哥哥怎么来了?”阿大答道:“阴间有件疑案,要弟弟去作证。”小二闻听变了脸色,怨骂哥哥。阿大指着身后一个像皂隶模样的人说:“现有官差在这里,我也是身不由己啊!”便向小二招手,小二不知不觉地跟着他们狂奔起来。

跑了一夜,他们来到泰山脚下,忽然看见一座衙门,刚要进去,里边很多人一涌而出。那个像皂隶模样的人问:“事情怎么样了?”其中一人回答:“不用再进去了,已经结了。”皂隶听说便释放了小二,让他回家。阿大担心弟弟没有盘缠,皂隶考虑了很久,就领着小二走了。走出二三十里路,进入一座村庄,来到一家屋檐下,皂隶嘱咐小二说:“这家如有人出来,你就让他送你回家。如果不肯,就说是王货郎说的。”说完便走了。小二立即人事不知,僵死在地上。

天亮后,这家主人出来,见有个人死在门外,十分惊骇。守候了一会儿,小二逐渐苏醒过来,主人把他扶进家中,又喂了点饮食,小二方才说出自己的家乡,要求主人送他回家。主人为难,小二便按皂隶交待的那样说了。主人一听,惊吓万分,急忙赁了匹毛驴送小二回去。到家后,小二拿钱给他,他不接受;问他原因,也不说。道别后,自己走了。


\subsection{1.10.2   疲 龙}
\label{\detokenize{p00_u5176_u5b83/_u767d_u8bdd_u804a_u658b_u5fd7_u5f02:id399}}
胶州的王侍御,奉命出使琉球国。船行海中,忽然从天上云间掉下一条巨龙,激起了数丈高的海浪。龙半浮半沉,高高地昂着头,把下巴支在船上,眼睛半闭着,一副筋疲力尽的样子。船上的人都十分恐慌,停下船桨,一动也不敢动。船家说:“这是在天上行过雨的疲龙。”王侍御忙将皇诏悬在龙头上,和众人一块烧香祷告。过了一会儿,巨龙方悠然游去。船刚行驶,又从天上掉下条龙,像上次一样;一天内先后掉下三四条。

又隔一天,船家叫人多备一些白米,告戒众人说:“离清水潭不远了,大家如看见什么,只管往水里撒米,要肃静,不能喧哗!”一会儿船来到一个地方,海水清澈见底,底下盘踞着一群巨龙,五种颜色,像盆、瓮那样,一条条地伏在海底。有的正在蜿蜒爬行,龙身上的麟、鬣、瓜、牙历历可数。船上的人见了,魂飞魄散,屏住呼吸,闭着眼睛,不仅不敢看,动也不敢动,只有船家不断抓米撒到海水里。过了很久,看到海水的颜色逐渐转为深黑色,才有人敢出声,便问船家撒米的缘故。船家回答说:“龙害怕蛆,怕蛆钻入它的鳞甲内。白米像蛆,所以龙见了往往伏在海底,船行驶在上面,可保安全。”


\subsection{1.10.3   真 生}
\label{\detokenize{p00_u5176_u5b83/_u767d_u8bdd_u804a_u658b_u5fd7_u5f02:id400}}
长安有一个读书人叫贾子龙,有一天,他偶然经过邻近的一条小巷,看见一个外地人,风度潇洒自如。贾生便问他,得知他姓真,是成阳人,在长安赁屋居住。贾生心里很敬慕他。

第二天,贾子龙就到真生住处投递名片拜访,正巧赶上真生不在家。前后拜访了三次,都没有遇到。贾生就暗中派人看准他在家而后去拜访,真生躲避着不肯出来,贾生闯进去搜他,他才出来。两个人促膝谈心,彼此都感到相见恨晚,因而非常高兴。贾生就在真生住处派个小僮去打酒。真生又善于饮酒,又能说风雅的笑话,两个人非常快活。酒快喝没了,真生翻了翻自己的箱子,拿出一个饮酒的器皿来,是一个大白玉杯子,却没有底,把一小杯酒倒在里面,就满满的了;用小杯舀酒倒入壶中,大玉杯中的酒并不减少一点。贾生觉得很神奇,执意要求真生传授这种法术。真生说:“我为什么不愿意和你相见?你没有其它短处,只是贪心未净罢了。这是仙家秘不传人的法术,怎么能传授给你呢?”贾生说:“真是冤枉啊!我哪里是贪心,偶尔产生一些奢望,只是因为贫穷啊。”笑了笑,二人就分别了。

从此,两人往来亲密无间,不分彼此。每当贾生窘困缺钱的时候,真生就拿出一块黑石头,吹上口气,再念些咒语,用它去磨瓦块碎石,瓦块碎石立刻就变成银子。便拿出赠给贾生;每次仅仅够贾生用,从没有多余的。贾生每次要求多变一些银子,真生就说:“我说你贪心,怎么样?怎么样?”贾生心想,明着要求必定得不到,便打算趁真生醉后睡觉时,偷了黑石头要挟他。一天,两人喝完了酒睡下以后,贾生偷偷起来,到真生衣服里搜摸。真生发觉了,说道:“你真没良心,不能再和你相处了!”于是就辞别贾生,移居到别处去了。

以后过了一年多,贾生在河边游玩,看见有一块石头晶莹光洁,很像是真生的那一块。贾生就拾起来,珍藏着像宝贝一样。过了几天,真生忽然来了,精神恍惚若有所失。贾生安慰他,并询问原因。真生说:“你以前所见的那块石头,是仙人的点金石。我从前跟随抱真子云游,他喜欢我性格耿直,把这块石头赠给了我。不料喝醉以后丢失了。暗中占卜应该在你这儿,如果你对我有‘还带之恩’,我一定不敢忘记报答你。”贾生笑道:“我生平从不敢欺骗朋友,的确和你占卜的那样,石头在我这儿。但是了解管仲贫穷的,莫过于鲍叔,你准备怎么办呢?”真生便答应送给他一百两银子。贾生说:“一百两银子不少了,但请你传授给我口诀,我亲自试一试,就没有遗憾了。”真生恐怕他不讲信用。贾生说:“你本是个仙人,怎么不了解贾某,我难道是失信于朋友的人吗?”真生就传授给贾生口诀。贾生回头看到台阶上有一块巨石,就要在上边试一试。真生拉住他的胳膊,不让他上前去磨。贾生就弯腰拾起半块砖头,放在石砧子上说:“像这么大,不多了吧?”真生就让他试了。贾生不去磨那半块砖头而磨那石砧子,真生变了脸色要和贾生争夺,而石砧已经化为一整块金子。贾生把石头还给真生。真生叹着气说:“已经这样了,还能说什么呢?但是,我随便把福禄加给别人,必然要遭受上天的惩罚。如果要挽回我的罪过,请你做善事施舍棺木一百口、棉衣一百件。你肯这样做吗?”贾生说:“我所以要得到金子,本来就不是为了窖藏起来。你还把我看成个守财奴吗?真生高兴地走了。

贾生得到金子后,一边施舍,一边做买卖;不到三年,施舍的数量已经够了。真生忽然来了,握着贾生的手说:“你真是个讲信用有义气的人啊!我们分别后,福神就报告了玉皇大帝,削去了我的仙籍;承蒙你广为施舍,现在用功德抵消了罪过。希望你勉励自己,不要停止做善事。”贾生问他是天上哪一部的神仙,真生说:“我是一只有道业的狐狸,出身很低微,承受不了罪孽的牵累,所以生平很自爱,一丝一毫也不敢胡作非为。”贾生摆下酒宴,真生就和他像从前一样对饮起来。贾生活到九十多岁,狐仙还时常到他家里去。

长山县某人,卖能解除信石(砒霜)之毒的药。即使是中毒垂危的病人,灌下他的药去没有救不活的;但是对他的药方保密,即使是亲戚好友也不传授。有一天,他因为被一件案子牵连被逮捕。他的内弟到狱中给他送饭,暗中就把信石放在饭菜里。守着他等他吃完了以后才告诉他。这人不信,一会儿腹中乱搅动起来,才大吃一惊,骂道:“畜牲养的,快去!家中虽然还有药末,恐怕路远来不及了;赶快在城里找到薜荔研成末,清水一杯,赶快拿来。”妻弟按他所说的去办了。等到拿回来,他已经连呕带泻快要死了,急忙给他灌下药去,立刻就好了。这个药方从此才传出。这也像那位狐仙秘其石不传于人一样。


\subsection{1.10.4   布 商}
\label{\detokenize{p00_u5176_u5b83/_u767d_u8bdd_u804a_u658b_u5fd7_u5f02:id401}}
某布商,到青州境内,偶然进入一座废寺之中,看见庙宇荒芜颓败,感叹哀伤不已。寺僧在一边说:“现在如有善人信士,帮助暂起一座山门,也是佛面的光彩啊。” 布商慷慨答应自己出资。寺僧大喜,将他请进方丈中,殷勤款待。既而寺僧又要求布商连同里里外外的殿阁也一并修复。布商感到很为难,便加以推辞。寺僧坚持要求这样做,言词神色逐渐变得凶横无赖。布商害怕,只得将自己的财物倾囊倒出,全部交给了寺僧。刚要离开,寺僧一把扯住,恶狠狠地说:“你献出全部财物,并非出于本心,以后怎能和我善罢甘休?倒不如先让你死!”持刀逼近布商。布商哀求饶命,寺僧不听;又恳求让自己吊死,寺僧才同意;将他逼入一间暗室中,催逼自尽。恰在此时,有一防海将军经过寺外,从墙缺处远远望见一红衣女子进入僧舍,心中大疑。于是下马进入寺中,前前后后仔细搜索,竟无影无踪。来到那间暗室,只见双门紧锁,寺僧不肯开门,假说内有妖邪。将军大怒,破门而入,发现布商已自缢在房梁上。急忙救下来,片刻便苏醒过来。问明实情后,又拷打寺僧,究问红衣女子的去向。实际上并无此人,才明白大概是神佛化身,指引将军救人而已。将军杀了寺僧,财物仍归还旧主。布商感激神佛救命,重新募资修庙,由此香火大盛。这件事孝廉赵丰原讲得最详细。


\subsection{1.10.5   彭 二 挣}
\label{\detokenize{p00_u5176_u5b83/_u767d_u8bdd_u804a_u658b_u5fd7_u5f02:id402}}
禹城人韩公甫讲:“一次我和同乡彭二挣一块走在路上,忽然回头不见了他,只有他骑的驴子跟在后面。又听到有急切的呼救声传来,细听声音发自驴背上的行李袋中。近前一看,袋子内有东西鼓起,虽然偏向一头,却掉不下来。想打开看看,袋口又被缝得结结实实;忙用刀割开,才发现彭二挣像狗一样卧在里面。出来后,问他怎么进去的,他自己茫然不知。原来他家有狐狸作祟,像这样的事经常发生。”


\subsection{1.10.6   何 仙}
\label{\detokenize{p00_u5176_u5b83/_u767d_u8bdd_u804a_u658b_u5fd7_u5f02:id403}}
长山县公子王瑞亭,能扶乩算卦。请下的乩神自称何仙,是吕洞宾的弟子。有人说实际上是吕洞宾骑坐的仙鹤。何仙每次降临,都喜好和人们谈文作诗。太史李质君拜他为师,何仙为他批改文章,条理分明,准确恰当。李质君能考中进士,多亏何仙帮助。因此很多文人学士都依附何仙。但何仙为人决断疑难事时,往往分析事物的道理,不多说吉凶祸福。

辛未年,文宗朱轼驾临济南,进行岁试。考完后,王瑞亭的朋友们请何仙判别等第。何仙索要他们的文章,一一评阅。座中有人和乐陵县的李忭关系很好,李忭本是好学善思之士,大家对他期望很高,于是拿出李忭的文章,请何仙判别。何仙批道:“一等。”不一会儿,又写道:“刚才评李生一等,是依据他写的文章评判的。但该生运气太坏,只能得四等。奇怪啊!文章和运数不相符,难道文宗取士不论文章好坏吗?你们稍等,我去看看。”过了一会儿,写道:“我刚才到提学官衙中,见文宗公事繁忙。他所焦虑的事根本不在评阅考卷上,一切都委托给六七个幕宾处理,廪生和例监都在其中。这些幕宾前世没有一点根气,大都是饿鬼道上的游魂,到处讨饭吃的。曾在黑暗狱中蹲了八百年,损坏了眼睛的精气,就像人久在洞中一样,乍出洞,天昏地暗,没有个正色,所以评起文章来只会是好坏不分。其中还有一两个是人托生的,但阅卷分曹,恐不能正好赶上啊!”大家便问挽回的办法,何仙批道:“办法是有,大家都知道,何必再问?”众人明白了何仙的意思,便告诉了李忭。李忭害怕,忙带了自己的文章去征求太史孙子未的意见,并告诉他文章、运数不符的预兆。孙子未看了文章后,大加赞赏,认为凭李忭的文章绝没有不考一等的道理。李忭因孙子未是文学大家,听了他的话便放心了,再不把何仙的预言放在心上。

后来放榜,李忭果然仅是四等。孙子未十分惊骇,又拿来李忭的文章反复审阅,还是找不出一点毛病,无可奈何地说:“文宗朱公一向有文名,肯定不会荒谬到这种程度。这一定是他幕宾中那些醉汉、不懂文章的人干的!”于是,大家越发佩服何仙的神异,一块焚香祝谢他。何仙又批道:“李生不要因为暂时的委屈,便感到羞愧。应当将判错的试卷多多抄写,广为传送,让大家都看看,明年即可得优等。”李忭按照何仙说的去做了,时间一长,文宗衙门中也听说了这件事,便安慰李忭。第二年考试时果然名列前茅。何仙就是如此神灵。


\subsection{1.10.7   牛 同 人}
\label{\detokenize{p00_u5176_u5b83/_u767d_u8bdd_u804a_u658b_u5fd7_u5f02:id404}}
(本篇残缺)牛同人到父亲的卧室,见父亲睡在床上没醒,以此知道定是狐狸作祟,不禁大怒,骂道:“狐狸本可容忍,怎能乱我家人伦?关公号称‘伏魔大帝’,现在哪里,怎能听任这种东西横行!”于是作表向玉帝上诉,内中说了些关公失职的话。过了很久,忽听到空中呐喊嘶叫,原来是关帝降临。关帝怒斥牛同人:“书生怎敢对我无礼!我难道是专为你家捉狐的吗?你并没有向我禀诉,有什么理由埋怨责怪我?”命将牛同人杖打二十,打得腿上皮开肉绽。一会儿,有个黑面将军捆来一只狐狸,牵走了。怪异方才灭绝。

三年后,济南游击将军的女儿被狐狸迷住,什么办法也驱赶不走。狐狸告诉女的:“我平生所怕的只有牛同人而已。”游击将军不知牛同人家住哪里,所以无从寻找。正值提学驾临济南,牛同人前去赴试,在省衙偶然被一营兵侮辱,他便忿忿不平地到游击将军府告状。将军一听牛同人的名字,惊喜万分,恭敬接待,将那个营兵抓来,责打了一顿。处理完毕,将军便将女儿被狐狸迷住的事告诉牛同人,央求他驱狐。牛同人没法推辞,只得替他呈告关帝。一会儿,一个金甲神自天而降,正在室内的狐狸见了面色突变,现出原形,像一只狗,嗥叫着绕屋子乱窜。接着便出屋自己跪到阶下,一动不动。金甲神说:“前次关帝没忍心诛杀你,这次又犯,再难饶恕了!”捆绑起来拴在马脖子上走了。


\subsection{1.10.8   神 女}
\label{\detokenize{p00_u5176_u5b83/_u767d_u8bdd_u804a_u658b_u5fd7_u5f02:id405}}
有一个姓米的书生,是福建人,写这篇故事的人忘记了他的名字和籍贯,姑且称之为米生吧。

米生有次偶然到郡城去,喝醉了酒经过一处市场,听到一高门大户内传出雷鸣般的箫鼓乐声,他感到奇怪,便问当地人,回答说这家正在开庆寿宴会。但门外、院内却十分清静。再听听,笙歌繁响,嘹亮动听。米生醉中十分向往,也不问这是什么人家,就在街头买了份贺寿礼物,向门内投了晚生的名帖。有个人见他衣着简朴寒伧,便问:“你是这家老翁的什么亲戚?”米生告诉他:“不是亲戚。”那人说:“这家是暂住在这里的过路人家,不知是什么高官,十分富贵显赫。既不是他家的亲属,你图个什么呢?”米生听说,心中后悔,但名帖已经投进去了。没过多久,两个少年人出门来迎接客人,都穿着耀眼华美的衣服,生得雍容俊雅,恭敬地请米生进家。米生来到室内,见一老翁面南坐着,东西两边摆列着几桌酒席,客人有六七个,都像是富贵子弟;看见米生,都站起来行礼,老翁也扶着拐杖站了起来。米生站了好一会儿,想和老翁寒暄,老翁却不离开座位。那两个少年人客气地说:“家父年老力衰,难以行礼,我们弟兄二人代家父感谢您的盛情!”米生谦逊地谢过,于是就在老翁边上又加了一桌酒席。不一会,有女子在下面奏乐。酒席座位后摆设着琉璃屏风,用以遮挡内眷。这时,击鼓的,吹笙的,乐声大作,使客人没法再交谈。宴席快结束的时候,两个少年站起来,每人拿一个足能盛三斗酒的酒杯劝客。米生一看,面有难色,但见其他客人都喝了,也只得跟着喝了;一会儿便连劝四杯,主人客人都一饮而尽。米生迫不得已,勉强喝干。少年又给斟上,米生感到酒力难当,站起来告辞,少年硬拉着衣服不让走。米生不觉大醉,颓然倒地。醉中感到有人在用冷水喷自己的脸,迷迷糊糊地醒了过来,站起来一看,客人都已散了,只有少年人扶着胳膊送他,于是告辞回家。后来,再经过那家门口,老翁一家已迁走了。

从郡城回来后,米生有次到街市上去,有个人从酒铺中出来,招呼他喝酒。米生看那人,又不认识,心想,姑且进去看看吧。进入店内,见同村的鲍庄已先坐在那里。米生问那个人的姓名,知道姓诸,是市中磨铜镜的,不禁奇怪地问:“你怎么认识我呢?”姓诸的反问:“前几天做寿的那人,您认识吗?”米生答道;“不认识。”姓诸的解释说:“我经常出入他家,最熟悉。那老翁姓傅,但不知是哪省人、做什么官。先生你去上寿时,我正好在那里,所以认识你。”三人一直喝到傍晚才散。当夜,鲍庄忽然死在路上。鲍庄的父亲不认识姓诸的,一口咬定是米生杀了儿子;又检查到鲍庄身有重伤,米生便以谋杀罪被官府拟判死刑,饱尝了严刑拷打的滋味。因为姓诸的一直没有抓获,官府无法证实米生确实杀人,便将他下在狱中。过了一年多,直指巡视地方,访察得知米生冤枉,才从狱中释放了他。

回家后,米生的家产已荡然无存,功名也被革除了。米生想到自己冤枉,希望能谋求辨复功名。于是带上行李往郡城赶来。天快黑的时候,米生疲惫不堪,再也走不动了,坐在路边休息。远远望见一辆小车驶来,两个青衣丫鬟两边跟随着。车子已经过去了,忽听有人叫停车,车中不知说些什么。一青衣丫鬟接着过来问米生道: “您是不是姓米啊?”米生吃惊地站起来答应。丫鬟叹道:“怎么穷困潦倒到这种程度!”米生告诉她缘故。丫鬟又问他要去哪里,米生也告诉了。丫鬟回去向车中说了几句话,又返回来,请米生到车子前。只见车中伸出一双纤纤小手,拉开车帷帘;米生偷偷地斜了一眼,见里面坐着一个绝色女郎。女郎对米生说:“您不幸遭受这么大的冤枉,听说后令人叹息!现在的学使衙门中,不是空着手就能出出进进的。路上也没什么东西送你,”说着从发髻上摘下一朵珠花,递给米生:“这东西能卖百金,请收起来藏好。”米生下拜,刚要问女郎的家族门第,车子飞快地离去,已经跑远了,终于不知她是什么人。米生拿着珠花,苦苦思索,见上面缀饰着明珠,不像是凡间的东西,便珍重地藏起来,继续往前赶路。到了郡城,投上诉状,衙门内上上下下勒索财物。米生拿出珠花看看,不忍心送掉,只好又回来了。

从此后,米生依附哥嫂生活。所幸哥哥比较贤良,替他经营料理生计,虽然贫困,也还能读书。

转过年来,米生又赴郡城去考童子试,误入深山之中。正值清明佳节,游山的人很多。有几个女子骑着马走过来,其中一个正是去年车子里的那个绝色女郎。女郎看见米生,便停马问他到哪里去,米生细说原委,女郎惊问:“你的功名还没恢复吗?”米生凄然地从衣服里拿出那朵珠花:“不忍心丢掉它,所以现在仍是童生。”女郎的脸不禁红了。之后,嘱咐米生坐在路边等等,自己骑马慢慢走了。过了很久,一个丫鬟驰马奔来,将一个包裹送给米生,说:“娘子有话:现在学使门内就像那做买卖的市场,公行贿赂。特赠二百两白银,作为你求取功名的资本。”米生推辞说:“娘子给我的恩惠太多了,我觉得以我的才能考个秀才不是难事。如此多的金钱,我不敢接受。只求告知娘子的姓名,绘一幅肖像,烧香供奉,便知足了。”丫鬟不听,将包裹放到地上,自己走了。

米生从此用度充足,但终不屑为了功名去攀附巴结权贵。后终于以第一名的成绩考进县学。他便将女郎赠送的白银送给哥哥。哥擅长聚财,三年时间,全部恢复了原来的家业。正好当时的闽中巡抚是米生先祖的门人,对米生十分照顾,兄弟二人俨然成为富贵大家了。但米生一向耿直清廉,虽是大官的通家世好,却从没有为了功名富贵去请见过巡抚。

一天,有个客人着裘衣、骑肥马来到米生门前,家人没有一个认识的。米生出来一看,原来是傅公子。行礼请入,各诉离情,米生便准备酒肴款待。客人以太忙推辞,但也不说就走。酒菜摆上,傅公子请求和米生单独谈谈,有事要说。进入内室,傅公子拜倒在地,米生惊问:“什么事?”傅公子悲伤地说:“家父刚遭受大祸,想求助于抚台大人,非兄不能办到这事。”米生推辞说:“他虽然与我是世代交情,但用私事麻烦别人,是我平生最不愿做的!”傅公子伏在地上哭着哀求,米生放下脸来,说:“我和公子只是一场酒的交情罢了,怎么拿丧失名节的事勉强别人呢?”傅公子又惭又忿,起来自己走了。

隔天,米生正在家中独坐,一个青衣丫鬟走进来。一看,正是深山中代女郎赠白银的那个。米生刚惊异地站起来,丫鬟责备道:“您难道忘了那朵珠花吗?”米生连忙说:“怎敢怎敢,实在不敢忘!”丫鬟又说:“昨天来的傅公子,就是娘子的亲哥哥。”米生闻言,心中暗喜,佯说:“这难以叫我相信。如果娘子亲自来说句话,油锅我也愿跳;否则,不敢奉命。”丫鬟听后,出门驰马而去。天将明,丫鬟又返了回来,敲门进来说:“娘子来了!”话没说完,女郎已进入室内,面壁哭泣,一句话不说。米生下拜说:“如果不是娘子,哪有我的今天?有什么吩咐,怎敢不惟命是听!”女郎哭道:“受人求的人常看不起人,求人办事的人常敬畏人。我半夜里到处奔波,平生没受过这般苦楚,只因为求人畏人的缘故啊,还有什么话说!”米生安慰说:“我所以没立即答应,是恐怕错过这个机会再难见你一面。使你深夜遭受奔波之苦,这是我的罪过啊。”说着拉住女郎的袖子,却暗地里捏摸她。女郎大怒,骂道:“你真不是个正派人!不念过去给你的恩惠,却想乘人之危,我看错人了,我看错人了!”忿忿出门,登上车就要离去。米生忙追出去赔礼道歉,长跪在地拦挡她,丫鬟也在一边讲情,女郎才稍微缓和点怒气,在车上对米生说:“实话告诉你:我不是凡人,是神女。家父是南岳都理司,偶然得罪了地官,马上就要上诉到玉帝那里惩处,没有本地巡抚大人的官印,没法解救。你如不忘我过去的恩义,就用张黄纸,为我求取印信!”说完,车子便走了。

米生回屋,吓得出了身玲汗。于是假借驱邪,向巡抚借官印用。巡抚觉得驱邪一事类似蛊惑人的巫术,不同意借印。米生又用重金贿赂巡抚的心腹,心腹答应给用印,却一直找不到机会。米生回来后,丫鬟已等在家门口,米生将事情详细告诉了她,丫鬟默默地走了,像是埋怨米生没有尽力。米生追上送她说:“回去告诉娘子:如事情办不成,我愿牺牲掉自己的这条性命!”回家后,米生彻夜辗转,不知如何办好。碰巧,巡抚有个宠幸的小妾要买珠子,米生便将那朵珠花献上。小妾非常喜欢,偷出印来为米生用了印。米生忙将盖了印的黄纸揣到怀里,返回家中,丫鬟刚好来到。米生洋洋得意地说:“万幸没辜负使命。但多年来我贫贱讨饭时都没舍得卖的东西,现在还是为了它的主人而丢弃了。”于是告诉丫鬟用珠花换印信的过程,又说:“扔掉黄金我都不可惜。麻烦你捎话给娘子,珠花可是要再赔我!”

过了几天,傅公子登门表示谢意,顺送黄金百两。米生不高兴地说:“我所以这么做,是因为令妹曾无私地帮助过我。否则,即使拿来万两黄金,又怎能换得一个人的名誉和气节呢!”傅公子再三要求收下,米生动怒,傅公子只好走了,说:“这事不能就这样算了。”第二天,青衣丫鬟又奉神女命,赠米生明珠三百颗,说:“这些足可以赔偿你的珠花了吧?”米生道:“我看重的是那朵珠花,不是这些珍贵的明珠。假使当时赠给我的是价值万金的宝物,也只能卖了当富翁罢了。我把珠花珍重地藏起来而甘于贫贱,为了什么?娘子是神仙,我怎敢有别的奢望,所幸能报答娘子给我恩惠的万分之一,我死无遗撼了。”丫鬟把明珠放到案几上,米生向明珠拜了拜,又退给了丫鬟。几天后,傅公子又来到。米生叫人准备酒肴款待,傅公子让同来的仆人下厨房,自己做菜。二人对面坐下,开怀畅饮,欢欢乐乐的,就像一家人。有个客人曾给米生一种米酒,傅公子尝了尝,觉得味道很好,连喝了上百杯,脸色微微变红,对米生说:“您是一个梗直正派的人,我们弟兄没能及早了解您,还不如我家小妹有眼光呢!家父感激您的大恩大德,无法报答,想将小妹许配给您,又担心您因人神隔绝而嫌恶。”米生又惊又喜,不知说什么好。傅公子告辞,说:“明晚是七月初九,新月和钩辰星同时出现,织女星有少女下嫁,正是良辰吉期,可准备青庐。”第二晚,果然将神女送了来,婚礼如仪,一切和常人一样。

三天后,神女对米生的哥嫂及家里的奴仆、丫鬟每人都有赏赐;性情又最贤惠,侍奉嫂嫂像对待婆母一般。只是几年不生育,劝米生另娶妾,米生不肯。正好米生的哥哥在江浙经商,替米生买了个妾回来。这个小妾姓顾,名叫博士,相貌清秀婉丽,米生夫妇都很喜欢。神女看见妾头上插着朵珠花,很像是当年那朵旧珠花,摘下来仔细一看,果然不错。便惊奇地追问珠花的来历,小妾回答说:“从前有个巡抚的爱妾死后,她的奴婢盗出这枝珠花出卖,先父觉得价格便宜,便买了下来,我见了非常喜爱。先父没有儿子,只生下我一个女儿,我想要的东西没有得不到的。后来父亲去世,家道衰落,我被寄养在一个姓顾的老太太家里。顾老太是我姨母辈的,见了珠花,屡次想卖掉,我投井觅死,坚决不同意,才得以保存到现在。”米生夫妇感叹地说:“十年前的东西,仍旧归还旧主,这岂不是天意!”神女拿出另一枝珠花,说:“这东西很久没对偶了!”把两枝珠花都赠给了小妾,并亲自给她插到发髻上。小妾退下,跟人详细打听神女的家世,家里的人都避讳谈起。小妾暗对米生说:“我看娘子不是凡人,她的眼眉间透着股仙气。昨天给我戴花时,我从近处看,觉得她那种美与生俱来,发自肌里,不像普通人只是眉眼长得匀称好看而已。”米生笑笑,不置可否。妾又说:“你不要说,我要试试她:如果她真是神仙,凡人有什么要求,在没人的地方烧香求她,她就知道。”神女绣的袜子十分精美,妾很喜欢,但不敢说。于是就在闺房中烧香祷告。神女早晨起来,忽然翻起针线箱子,捡出一双绣袜,让丫鬟送给小妾。米生看见,不禁失笑。神女询问缘故,米生便将妾的计划说了。神女也笑了,骂道:“好狡猾的婢子!”但因为妾的聪明,也越发爱怜她。妾侍奉神女也越恭敬,天不明,便沐浴熏香,收拾整齐,前去拜见神女。

后来妾一胎生下两个儿子,米生夫妇俩分别给起了名字。米生活到八十岁时,神女还年轻得像少女一样。后来米生卧病不起,神女找来木匠做棺材,让做得比普通棺材大一倍。米生死后,神女也不哭。家人外出,回来发现神女也躺在棺中死了,于是合葬了他们。至今还流传着“大材冢”的说法。


\subsection{1.10.9   湘 裙}
\label{\detokenize{p00_u5176_u5b83/_u767d_u8bdd_u804a_u658b_u5fd7_u5f02:id406}}
晏仲,是陕西延安人,他跟哥哥晏伯生活在一起,兄弟二人非常友爱。晏伯三十岁时就死了,没有子嗣;不久,他妻子又相继去世。晏仲十分悲痛,常常想自己如能生两个儿子,就把一个过继给去世的兄嫂作为子嗣。但刚生下一个儿子,自己的妻子也死了。晏仲担心续弦后,新妻子会虐待儿子,便不想再娶,只想买一个妾。正好邻村有卖奴婢的,晏仲去相看了相看,一点也不中意,很感沮丧无聊。又碰上一个朋友请他喝酒,喝完后,便醉醺醺地往回赶来。

路上,晏仲忽然碰到已经死去的同学梁生,见了晏仲热情地握手问好,请晏仲到自己家里坐坐。晏仲醉得稀里糊涂,也忘记他已经死了,跟着他走了。进入家门,一看不像是梁生原来的家,心中疑惑,便问他,回答说:“最近才搬来。”到屋里坐下,要喝酒时,一看酒却没了。梁生嘱咐晏仲稍等等,自己拿着酒瓶出去买酒去了。

晏仲站在门口等着粱生,见一个妇人骑着匹毛驴经过,后面还跟着个小孩,大约八九岁的样子,相貌神态极像哥哥晏伯。晏仲怦然心动,急忙赶上,问那小孩姓什么。小孩回答说:“姓晏。”晏仲更加惊疑,又问:“你父亲叫什么名字?”回答说:“不知。”正说着话,已经到了小孩的家门口,妇人下驴走了进去。晏仲拉住小孩,问:“你父亲在家吗?”小孩点点头,也走了进去。一会儿,又有个妇女出来看了看果然是晏仲的嫂嫂。见了晏仲,惊讶地问他是怎么来的。晏仲大为悲伤,跟着嫂子进入家门,见房屋院落,整洁一新,便问:“哥哥在哪里?”嫂子回答说:“出去讨债还没回来。”晏仲又问:“那骑驴进来的是谁?”嫂子回答说:“是你哥哥的妾甘氏。她已经生了两个男孩了。大的叫阿大,到市上去还没回来。你看见的那个是阿小。”

晏仲坐了很久,酒渐渐醒了过来,心里一下子明白了自己看见的这些人全是鬼。但因为跟哥哥感情深厚,所以也不害怕。这时,嫂子开始热酒做饭,晏仲急于见到哥哥,催促阿小去寻找。过了很久,阿小哭着回来,说:“李家赖债不还,还和父亲打架!”晏仲听说,急忙跟阿小奔跑了去,见两个人正把哥哥摔到地上。晏仲大怒,挥舞着拳头,径直冲了过去,一连打翻了几个人,将哥哥救了起来。李家的人四处逃散,晏仲追上一个,按到地上痛打一顿,解恨后才起来。拉着哥哥的手,跺着脚伤心地哭泣,晏伯也哭了。

回来后,全家人都来慰问。晏伯于是备下酒菜,兄弟二人举杯相庆。不一会儿,一个少年走了进来,约十六七岁的年纪,晏伯叫他阿大,让他拜见叔叔。晏仲忙将阿大拉起来,哭着跟哥哥说:“大哥在地下已有了两个儿子,但大哥阳间的坟墓却无人祭扫。我孩子小,妻子又死了,这可怎么办好呢?”晏伯也辛酸悲伤起来。嫂子在一边跟晏怕说:“要不的话,就让阿小跟他叔叔去吧!”阿小听了,依偎在叔叔的怀里,恋恋着不想离开。晏仲抚摸着他,越发感到难过,问阿小:“愿意跟我走吗?”阿小忙答:“愿意。”晏仲心想:阿小虽然是鬼不是活人,但有总比没有好,心里便高兴起来。晏伯嘱咐弟弟说:“让他去,不要太娇惯了他。要让他多吃血肉,每天在太阳底下暴晒,一直到过午。他才六七岁,此后历尽寒暑,再生骨肉,仍可娶妻生子,只是恐怕寿命不会长了。”正说着话,门外有个少女在偷听,模样很是温柔文静。晏仲以为是哥哥的女儿,便询问晏伯。晏伯说:“她叫湘裙。是我的妾甘氏的妹妹。因为父母双亡,孤独无靠,寄养在我这里也有十年了。”晏仲又问:“嫁人了吗?” “还没有。最近有媒人给介绍东村田家的孩子。”少女在窗外小声嘟囔:“我不嫁田家那放牛郎!”晏仲对她不觉心动,但不便直说。接着,晏伯离座,在书房中摆下床榻,让弟弟住宿。晏仲本不想住下,但心中惦念着湘裙,正想设法摸摸哥哥的意思,于是,便告辞哥哥去睡了。

当时,正是初春,气候还很寒冷。书房中没有炉火,像在冰窖里一样。晏仲不觉毛骨悚然,浑身起了层鸡皮疙瘩。突然想喝点酒。一会儿,阿小推门进来,把一碗肉羹、一斗酒放到桌子上。晏仲大喜,问阿小谁让他来的,阿小回答说:“是湘姨。”酒刚喝完,阿小又端了盆炭火来,用灰盖着,放到床下。晏仲问:“你爹娘都睡了吗?”阿小说:“已睡下很久了。”“你睡在什么地方?”“我跟湘姨一块睡。”阿小直等到叔叔睡下,才闭上门走了。晏仲觉得湘裙既聪明,又会体贴人,心里更加爱慕。又因为她能抚养阿小、越发坚定了娶她的念头。辗转床头,一夜没睡。

第二天早早起来,晏仲告诉哥哥说:“我孤单一人,没有配偶,麻烦大哥多多费心。”晏伯说:“我们家不是穷家,自然会有人替你物色。阴间虽然有漂亮女子,恐怕对你没有好处。”晏仲说:“古人也有娶鬼妻的,有什么害处呢?”晏伯像是明白了他的意思,便说:“湘裙倒是不错。但须拿大针刺‘人迎’穴后血流不止的鬼,才能做活人的妻子。这事怎能草率呢?”晏仲说:“娶了湘裙也能照顾阿小。”晏伯只是摇头。晏仲哀恳不已。嫂子说:“不妨捉住湘裙,强刺一针检验一下,不行的话就算了。”于是握着针出去,到门外正碰上湘裙,急忙攥住她的手腕,只见她手上有血迹,还是湿的!原来,湘裙在门外愉听到晏伯的话,已经自己试过了。嫂子放开她的手,笑着回去告诉晏伯说:“她早就对小叔有意了,你还为她忧虑什么?”妾甘氏听说后大怒,奔到湘裙跟前,用手指戳着眼骂道:“淫婢好不害臊!想跟着小叔私奔吗?我偏不让你如愿!”湘裙又羞又气,号哭着要寻死,闹得一家人沸反盈天。晏仲十分惭愧,告辞兄嫂,带着阿小出门走了。哥哥说:“你暂且回去吧。不要让阿小再来,以免减损他的阳气。”晏冲答应了。

回家后,晏仲故意夸大了阿小的年龄,跟人假说是哥哥先前所卖奴婢生的遗腹子。众人因为阿小相貌极像晏伯,也就相信了他是晏伯的儿子。

晏仲教阿小读书时,总是让他抱着本书坐在日头底下朗读,阿小起初还觉得苦,时间长了也就习惯了。六月酷暑天气,桌子被烤得烫人,但阿小边玩耍边读书,一点也不抱怨。又最聪慧,每天读半卷书。夜晚就和叔极一块睡,还常常把学会的文章背给叔叔听。晏仲很感欣慰。但心中一直念念不忘湘裙,所以也不想再娶别的女人了。

一天,有两个媒人来为阿小提亲。因为没个女人操持招待,晏仲十分焦躁。忽然甘氏从外面走了进来,对晏仲说:“小叔别怪,我把湘裙送来了!前次因为她太不害羞,要自己跟人,我所以故意羞辱她一番。其实小叔一表人才,不让她跟你跟谁呢?”晏仲见湘裙果然站在甘氏身后,非常高兴。恭敬地请嫂子坐下,说还有客人在堂屋里,自已便出去了。一会儿又回来,见甘氏已走了。湘裙卸妆进了厨房,只听叮叮当当一片刀板声传来,瞬间,美味的菜肴便纷纷摆了上来。客人走后,晏仲进屋,见湘裙盛装端坐着,于是和她交拜成了亲。到晚上,湘裙仍想跟阿小一块睡,晏仲说:“我要用自已的阳气温暖他,他不能离开我。”让湘裙到别的屋子住下了,只是每晚过去和她喝几杯酒、欢会一次罢了。湘裙待晏仲前妻生的儿子犹如亲生一般,晏仲更加喜欢她,觉得她非常贤惠。

一晚,夫妻二人谈得非常融洽欢乐。晏仲开玩笑般地问湘裙:“阴间里也有美人吗?”湘裙想了很久,回答说:“我没见过。只有邻居家的女儿葳灵仙,大家都说漂亮。其实她相貌平常,不过会打扮罢了,和我来往最久了,但我心中一直鄙视她太浪荡风骚。你如想见她,我可以马上叫她来。只是这种人招惹不起!”晏仲听说,立刻就要见见她。湘裙提起笔来像要写信,却又扔下笔说:“不行不行!”晏仲再三恳求,湘裙才说:“你可不要被她迷住了!”晏仲答应。湘裙便在纸上画了几笔,像是一道符咒,拿到门外烧了。一会儿,便听见门帘微动、帘钩作响,有吃吃的女子笑声传来。湘裙起身,出去将一个女子拉进来。只见她高高的发髻,前面翘起,真像画上的美人一样。湘裙拉她到床头坐下,二人喝着酒诉说离情。那女子初见晏仲时,还害羞得用红袖子捂着嘴,不怎么说话。几杯酒下肚,便露了本相,跟晏仲嬉笑打闹,毫无顾忌,渐渐伸过一只脚去压到晏仲的衣服上。晏仲心迷神摇,魂都不知飞到哪里去了。只是眼前碍着湘裙在座,湘裙也防范着葳灵仙,在旁边一刻也不离开。一会儿,葳灵仙忽然起身拉开门帘走了出去,湘裙忙跟着,晏仲也随后出屋。葳灵仙竟拉着晏仲的手,二人跑进了别的屋子。湘裙十分愤恨,但又无可奈何,只得愤愤地回屋,听任他们为所欲为了。不长时间,晏仲回来了,湘裙责备他说:“不听我的话,恐怕你日后赶也赶不走她!”晏仲怀疑湘裙是在嫉妒葳灵仙,二人不欢而散。

第二晚,葳灵仙不叫自来。湘裙极为厌烦,也不答理她。葳灵仙竟又和晏仲手拉着手走了。这样一连过了好几晚,湘裙再也忍耐不住,看见葳灵仙来,就百般斥骂,却苦于赶不走她。又过了一个多月,晏仲便一病不起,才开始后悔,叫来湘裙一块睡,想以此躲避葳灵仙的纠缠。湘裙也日夜防范,但稍一疏忽,晏仲又被葳灵仙勾去了!湘裙怒不可遏,操起擀面杖,往外赶葳灵仙,葳灵仙也愤怒地和她争执,二人打了起来。湘裙体弱,手脚都被葳灵仙打伤。晏仲见此情景,病势更加沉重。湘裙哭着说:“我怎么去见我姐姐啊?”又过了几天,晏仲便死了。

先是晏仲见两个皂隶手持文牒走了进来,自己不知不觉地跟他们走了。途中担心没有路费,便邀请皂隶顺便到哥哥住的地方去坐坐。到了晏伯家,哥哥一看见他,惊骇失色,问他:“弟弟最近干了些什么?”晏仲回答说:“没什么,只是得了鬼病。”于是告诉了哥哥实情。晏伯听了说道:“这就是了!”拿出一包银子,递给两个皂隶说:“请你们收下吧!我弟弟罪不至死,请你们放了他,我让我儿子跟你们去,不会出什么事的!”便叫过阿大来陪着皂隶喝酒。自已返身进屋,将情形告诉家里人,立命甘氏去隔壁叫葳灵仙来。不一会儿,葳灵仙进来,看见晏仲,返身就逃。晏伯一把揪住,拽回来骂道:“好个骚奴婢!活着时是荡妇,死了还是贱鬼,早就不齿于人了,还敢来害我弟弟!”摔手就是几耳光,打得她头发四散,容貌减色。过了很久,一个老婆婆走进来,跪在地上哀恳晏伯饶了葳灵仙。晏伯斥责老婆婆纵女淫荡,又痛骂了一会儿,才让她领着女儿走了。

晏伯送晏仲回来,飘飘忽忽的,不觉到了自家门外,径直走进卧室。晏仲一下子醒了过来,才知道自己刚才已经死了。晏伯见了湘裙,责怪她说:“我和你姐姐以为你贤慧能干,所以让你跟了我弟弟。没想到你反而促他早死!若不是碍于名分,我非重打一顿不可!”湘裙又惭愧,又惧怕,低声哭泣着,跪在晏伯面前谢罪。晏伯看见阿小,喜欢地说: “我儿子竟然像活人了!”湘裙要出去做饭,晏伯推辞说:“弟弟的事还没办妥,我没功夫吃饭。”阿小这时已经十三岁了,渐渐留恋父亲,见父亲要走,流着泪跟着。父亲安慰他说:“跟着叔叔最快乐。我走后还会再来的。”说完,一转身便无影无踪了。从此后,再没通过音讯。

后来,阿小娶了媳妇,生了一个儿子。阿小也是到三十岁时死了。晏仲抚养着他的独子,就跟侄子活着时一样。晏仲八十岁时,阿小的儿子已经二十多了,便让他分家另过。湘裙则始终没有生育。

一天,湘裙对晏仲说:“我先到地下准备好居住的地方。”说完,便盛装上床去世了。晏仲也不悲伤,半年后也死了。


\subsection{1.10.10   三 生}
\label{\detokenize{p00_u5176_u5b83/_u767d_u8bdd_u804a_u658b_u5fd7_u5f02:id407}}
湖南有个人,姑且称之为湖某,能记得前生三世的事。第一世是县令,乡试中作同考官,负责阅卷。有个叫兴于唐的名士,在考试中落第,冤愤而死,拿着自己的考卷到阴司里状告湖某。兴于唐的诉状一投,和他患同一种病死去的冤鬼,成千上万,共同推兴于唐为首领,结成同伙以作响应。湖某便被摄到阴司中,和众鬼对质。阎王问他道:“你既然负责评阅文章,为什么革除名士而录取平庸的人?”湖某叫屈说:“我上面还有主试官,我不过是奉命行事罢了。”阎王便发签,命小鬼去拘拿主考官。过了很久,将主考官拘来,阎王告诉他湖某的辩解,主考官说:“我不过最后汇总,即使有好文章,簾官不推荐,我又怎么知道呢?”阎王道:“这件事你们不能互相推卸责任,都算失职,按律应受笞刑。”刚要施刑,兴于唐不满意,大声鸣起冤来,两阶下的众鬼,万声响应。阎王问兴于唐缘故,兴于唐大声说:“笞刑太轻,应该挖出他们的双眼,以作为不识文章优劣的报应!”阎王不同意,群鬼号叫越发猛烈。阎王说:“他们不是不想得到好文章,只是见识太鄙陋罢了。”众鬼又请求剖出他们的心,阎王迫不得已,只得命小鬼剥去考官的衣服,用刀剖胸剜心。两人滴着鲜血,嘶呀痛叫,众鬼方才高兴。纷纷说:“我们终日在阴间里气愤烦闷,没有一个能出这口气的人。现在多亏兴先生,才消了这口怨气!”于是哄然散去。

湖某受刑毕,被押投到陕西,托生为普通百姓的儿子。长到二十多岁,正赶上家乡闹土匪,他被掳入贼寇中。官兵前去剿捕,俘虏了很多人,湖某也夹在里边。心里还想自己不是贼,希望官府能辨认出来释放。等看到大堂上坐着的审判官,年龄也是二十多岁,仔细一看,却是兴于唐。湖某大惊道:“我合该死了!”不长时间,被俘虏的人全部释放了;最后是湖某,审判官不容他申辩,立命杀了。

湖某冤魂到阴司中,状告兴于唐。阎王没有立即拘拿兴于唐,等到他官禄享尽,迟至三十年后才勾来阴司,两人当面对质。兴于唐因乱杀人命,被法托生为畜牲;考察湖某生前的行为,曾打过父母,罪行和兴于唐均等,也罚作畜牲。湖某恐怕来世再遭报应,请求托生个大畜牲。阎王便判他托生为大狗,兴于唐为小狗。大狗生在顺天府的一个市场中。一天,大狗卧在街头,有个南方来客牵着一条金毛狗,只有狐狸那样大。大狗仔细一看,正是兴于唐。心里轻视它小,一口咬住了它。没想到小狗反咬住了大狗的喉咙,吊在大狗的脖子底下,像个铃铛一样。大狗嗥叫着翻滚扑腾,市场上的人怎么也分解不开,不一会,两条狗都死了,又一块到阴司打官司,各说各理。阎王说: “像这样冤冤相报,何时算了!现在我为你们和解。”于是判兴于唐来世做湖某的女婿。

此后,湖某又托生到庆云。二十八岁时,考中举人,生了一个女儿,长得十分文静漂亮。世族大家争着提亲,湖某一概不答应。一次他偶然经过邻郡,正赶上学使定等公布岁试考卷,一个姓李的列一等卷第一名,就是兴于唐。湖某将李生请到旅舍,殷勤招待。打听他的家庭情况,知道还没成亲,便将女儿许给了他。人们都说湖某爱才,却不知这是前世的姻缘。不久,李生将湖某的女儿娶了去,两个人感情很好。但李生常常依仗着自己的才气,慢待老丈人,经常一年都不到丈人门上。湖某也忍了下来。后来,李生中年失意,屡考不中,湖某千方百计替他夤缘,才使他科考得志,如愿以偿。从此以后,翁婿和好亲如父子一般。


\subsection{1.10.11   长 亭}
\label{\detokenize{p00_u5176_u5b83/_u767d_u8bdd_u804a_u658b_u5fd7_u5f02:id408}}
石太璞是泰山人,喜爱画符念咒,祈神驱鬼的法术。有一个道士遇见了他,很赏识他的聪明,就收他做弟子,打开一个书套的牙签,拿出两卷书来,上卷驱狐,下卷驱鬼。道士把下卷传授给他,说:“虔诚地学好这部书,衣食和美人就都有了。”石生问道士姓名,他说:“我是汴州城北村玄帝观的王赤城。”道士留下住了几天,把下卷的口诀都传授了给他才走了。

石生从此精于驱鬼镇邪之术,带着财礼到他家求他驱鬼镇邪的人接连不断。

一天,来了一位老叟,自称姓翁,把带来的银子绸缎炫耀地摆列出来,对石生说,他的女儿得了鬼病已经很危险了,求他务必亲自去一趟。石生听说病人已经很危险了,就推辞不接受他的财物,答应和他一起去试一试。

走了十几里路,进入了一个山村,到了翁叟的家,只见房舍华丽美好。进入内室,看到一个少女躺在薄纱帐子里,婢女用钩子把帐子挂起来。石生一看,姑娘约有十四五岁,气息微弱地躺在床上,脸色枯黄干瘦。石生走近前,姑娘忽然睁开了眼睛说;“良医来了。”翁叟全家都非常高兴,说这姑娘已经好几天不能说话了。石生便退出内室,详细询问了病情。翁叟道:“白天常见一个少年进来,跟她睡在一起,去捉他的时候,又看不见了;一会儿又来了。我想他一定是个鬼。”石生说:“如果他是个鬼,驱走他并不难;我担心他是个狐狸,那么我就不知驱赶它的办法了。”翁叟说:“一定不是狐狸,一定不是!”石生就画了一张符给他,这天晚上就住在他家里。

半夜里,有一个少年进入石生房里,穿戴整洁。石生怀疑是主人的亲属,就站起来问他。少年说:“我是个鬼。老翁家都是狐狸。我偶然喜爱上他家的女儿红亭,才暂时住在这里。鬼作祟迷惑狐狸,并不损伤阴德。你何必护着他家而拆散别人的姻缘呢?姑娘的姐姐叫长亭,容貌艳丽绝伦,我特地保留下她清白的身子,让她完好无瑕,以便等待你来。他们如果答应把她许配给你,你才可以给红亭治病;到那时候,我一定自己离去。”石生答应了他。

这天晚上,少年没再来,姑娘顿时就清醒了。天明以后,翁叟非常高兴,把这件事告诉了石生,请石生进去看看。石生焚烧了旧符,坐下来诊视病人。只见绣花帷幕边有—个女郎,美丽得如同天上的神仙,心里知道她一定是长亭。诊视完了以后,石生要一碗水洒洒帐子,这位女郎急忙端了一碗水给他。她走动之间,眼波流转,神韵动人。石生此时心动神摇,心里早已不在鬼身上了。他出了内室后辞别老翁,托词说要回去制药就走了,好几天没回来。

此后,翁家那个鬼越发肆无忌惮了,除了长亭之外,媳妇、婢女都被他迷惑淫乱。翁叟又派仆人牵着马去请石生,石生推托有病不去。第二天,翁叟亲自来了,石生故意装出腿有病的样子,拄着拐杖出来。翁叟向他行了礼,问他得病的缘故。他说:“这是单身的难处啊!昨日晚间婢女上床给我换汤壶,跌了一跤,失手把汤壶掉下来,把我的两脚烫起了泡。”翁叟问:“为什么这么久了不再续娶呢?”石生说:“只恨找不到像您一样的清白人家。”翁叟默默无言地走了,石生走着送他说:“病好了我一定去,不用麻烦你亲自来了。”又过了几天,翁叟又来了,石生一跛一拐地见他。翁叟安慰问候了几句话,就说:“刚才我跟老伴商议过了,你如果能把鬼驱走,使我全家安宁,我的女儿长亭,已经十七岁了,我就情愿把她嫁给你。”石生大喜,跪下磕了头,对翁叟说:“你既然有这样的美意,我怎么还能珍惜我这有病的身体呢?”立刻就走出门去和翁叟一起骑马去了。

到了翁叟家,给患鬼病的人看完了病,石生恐怕他们背约反悔,就要求和老太太见面订婚约。老太太急忙出来说:“先生怎么怀疑我们呢?”就把长亭头上所插的金簪交给石生作为凭证。石生磕头拜见了岳母,于是把全家人都召集起来,一个个都给他们把鬼患驱除了。只有长亭一个人藏在内室没有见到,石生就画了一张佩在身上的符,叫人拿去给他。这一天晚上安安静静,鬼影都消失了。唯有红亭还在呻吟,向她身上洒了一些洁水,她所患的病好像立刻消失了。石生想告辞回去,翁叟殷勤诚恳地挽留他。到了晚上,请石生喝酒,珍肴美味罗列,劝酒布菜十分亲切。一直喝到二更天,主人才向石生告辞走了。

石生刚躺下要睡,听见敲门声很急,起身开门一看,长亭闪身进来了,神色语气惊慌地说:“我们家的人要拿刀来杀你,赶快逃走吧。”石生胆颤心惊,面无人色,越过墙头,急忙逃窜了。他远远望见前面有火光,就急忙向那里奔去,原来是村里的人夜间在打猎。等到他们打完了猎,石生就跟他们一起回去了。

石生心里又怨恨又愤怒,没有地方可以申诉,想要到汴城寻找师父王赤城;而家里有个老父亲,病卧在床很久了,放心不下。石生日夜筹思谋划这件事,不能决定去还是不去。忽然有一天,两辆车子来到门前,原来是翁家老太太送长亭来了,她对石生说:“那天晚上你就回来了,为什么不再商议一下婚事?”石生见了长亭,怨恨都烟消云散了,所以对那天夜里的事也就隐瞒不说了。翁老太太督促两人在庭院里拜完了天地。石生要设酒席招待岳母,她推辞说:“我不是闲人,没有时间坐下来品尝美味佳肴。我家老头子年老糊涂了,有什么对不住你的地方,姑爷你肯为了长亭而念及到老身,就深感庆幸了。”于是上车走了。原来翁叟杀女婿的预谋,老太太并不知道,等到没有赶上石生返回来,老太太才知道,心里颇为生气,和老头子整天吵骂。长亭也哭泣不肯吃饭。老太太硬作主张把长亭送来,不是老头子的本意。长亭过了门,石生问她,才知道了其中的缘故。

过了两三个月,翁家来接女儿回家探亲,石生估计她不能回来了,就不许她回去。长亭从此就时常啼哭。过了一年多,生了一个儿子,起名叫慧儿,雇了一个奶妈哺育他。然而儿子好哭,晚上必定要回到母亲那儿。一天,翁家又派车来,说老太太非常思念女儿,长亭越发悲伤,石生不忍心再留她了。长亭要抱着孩子去,石生不允许,长亭就自己回娘家了。临别时,约定以一个月为期;可是过了半年多仍然没有消息。石生派人去探看,翁家从前租赁居住的院子已空了很久,没人住了。又过了两年多,一切希望都断绝了。儿子整夜啼哭,石生心如刀割。

不久,石生的父亲病死了,石生倍加哀伤,因而病倒了。父丧期问病势沉重,不能接受宾客朋友的吊唁。正在昏昏沉沉之际,忽然听见一个妇人哭着进来了。一看,原来披麻戴孝的人是长亭。石生心中十分悲痛,一阵难受就断了气。婢女惊慌呼叫,长亭才停止了哭泣,过来抚摸石生身体。过了好一会儿,石才渐渐苏醒过来,自已疑心已经死了,以为是在阴间与长亭相聚。长亭说:“不是在阴间。我不孝顺,不能得到严父的欢心,受到阻挠,三年不能回来,实在对不住你的一片心。正好我的家人由东海经过这里,得知公公去世的凶信。我遵严父之命断绝了与你的儿女之情,却不敢遵从他的乱命而违背翁媳之间的礼制。我来的时候母亲知道,父亲却不知道。”说着话儿子扑到她怀里。说完了话,她才抚摸着儿子哭着说:“我有了父亲,孩子你没了母亲了!”慧儿也嚎啕大哭,满屋的人都掩面哭泣。长亭站起身来,着手料理家务,灵柩前供的祭品器具齐全而干净,石生心里大感安慰。但是因为得病时间久了,急切间不能起床。长亭就请石生的表兄接待来吊唁的宾客。吊唁的礼仪结束以后,石生才能柱着拐杖站起来,与长亭一起商议安排殡葬的事。安葬完毕,长亭要辞别回去接受违背父命的谴责。可是丈夫拉着手臂,儿子大声哭泣,于是就忍住暂时不走了。

过了不多日子,翁家有人来告诉长亭的母亲病了。长亭就对石生说:“我是为了郎君的父亲来的,郎君就不为了我的母亲放我回去吗?”石生答应了。长亭叫乳母抱着儿子到别处去,自己流着泪出门走了。一去之后,好几年没有回来,石家父子也渐渐忘记她了。

一天,天刚亮时打开大门,长亭竟飘然进来了。石生正惊骇地询问,长亭满面愁容地坐到床上叹息着说:“从小在闺阁中长大,把走一里路都看作很远;现在一天一夜奔波千里,累坏了!”石生仔细问她,长亭想说又住口了。石生执意请她说,她才哭着说:“现在就对你说,恐怕我感到悲痛的事,正是郎君感到快乐的事。近几年,我家迁居到山西境内,租赁了赵乡绅家的宅第居住。主客交情十分密切,父母就把红亭许配给赵公子为妻。赵公子经常嫖赌放荡,家庭生活很不和睦。妹妹回来告诉了父亲,父亲留下她,半年不叫她回去。赵公子十分愤恨,不知从哪里聘了一个恶人来,派遣神将拿着铁索,把老父亲绑去了。一家人十分惊恐,顷刻间就四处逃散了。”石生听说后,禁不住笑了起来。长亭气愤地说:“他虽然不讲仁义,可也是我的父亲。我与你夫妻几年,只有相好而没有相怨之处。今天我家人亡家败,上百人流离失所,你即使不为我父亲伤心,难道也不为我伤心吗?听说之后反而手舞足蹈,更没有一言半语安慰我,为什么这么无情义啊!”一甩袖子就走了。石生追着向她道歉,长亭已经不见了。石生心里惆怅悔恨不已,只好打算彻底决裂了。

过了两三天,翁老太太和长亭一起来了,石生非常高兴地安慰问候。老太太与长亭二人都跪下了,石生吃惊地问他们,母女二人都哭了。长亭说:“我赌气走了,现在自己不能坚持,又要来求人,还有什么脸面呢?”石生说:“岳父固然不是人,但是岳母对我的恩惠,你对我的情义,都是我永远不会忘记的。然而那天我听见岳父遭祸事而感到高兴,也是人之常情,你为何不能暂时忍耐一下呢?”长亭道:“刚才在途中遇到母亲,才知道捉去我父亲的人,原来是你师父。”石生说:“真是这样,也很容易办。但是岳父不回来,你们父女离散;恐怕岳父回来了,那么你的丈夫就要哭,儿子就要悲了。”老太太立誓表明自己的心意。长亭也立誓报答丈夫的恩情。

石生准备了行装到汴州去,打听着找到了玄帝观,原来王赤城也刚回来不久。石生进去参拜了师父,师父便问他:“你来为了什么事?”石生看见厨房里有一只老狐狸,在它的前股上穿了一个孔用绳索拴着,就笑着说:“弟子这次来,就是为了这只老狐精。”王赤城追问他,石生说:“它是我岳父。”就把实情告诉了师父。王道士说这老狐太狡诈,不肯轻易释放。石生再三请求,王道士才答应了。石生就详细地述说了这老狐狸的种种狡诈行为,老狐狸听见了,把身体挤进灶膛里,好像惭愧的样子。王道士笑道:“他羞耻之心还未完全丧失。”石生站起来,牵着他出去,用刀割断了绳子从伤口里抽出来。狐狸痛极了,咬得牙直响。石生不一下子抽出来,而是一顿一挫地往外抽,笑着问老狐狸:“岳父感到痛,不抽绳子可以吧?”老狐狸眼睛凶光闪闪,好像有恼怒的神色。石生放了它以后,它便摇着尾巴出了道观跑了。

石生辞别了师父回家。三日前已经有人来石家报告翁叟回来的消息。老太太先回去了,留下女儿等候石生。石生到了家,长亭迎上前跪在地上,石生把她扶起来说:“你如果能不忘夫妻的感情,我倒不在乎感激不感激。”长亭说:“现在我父母家已经迁回故居了,村子离这儿邻近,可以互通音信了。我想回娘家探望父亲,三天就可以回来,郎君相信不相信?”石生说:“儿子生下来以后就没有母亲,可是也并没有夭折。我天天过着光棍的生活,已经习惯了。现在我不像赵公子那样,反而以德来报答你父亲,我已为你尽到了情义。如果你真的不回来,在你来说是辜负了我的情义。两家相距虽然很近,我一定不再过问了,还有什么不相信的?”长亭第二天回了娘家,过了两天就返回来了。石生问道:“为什么这么快就回来了?”长亭说:“父亲因为郎君在汴州曾经戏弄过他,心里老忘不了,絮絮叨叨地老说这件事。我不想再听了,所以早回来了。”从此以后,长亭和她母亲、妹妹之间的往来很密切,而岳父和女婿之间还是不相往来。


\subsection{1.10.12   席 方 平}
\label{\detokenize{p00_u5176_u5b83/_u767d_u8bdd_u804a_u658b_u5fd7_u5f02:id409}}
席方平,是东安县人。他的父亲名叫席廉,非常憨厚老实,和村里一个姓羊的富户结下了怨仇。姓羊的先死了;几年后,席廉得了重病,快要死时,告诉家人说:“现在姓羊的买通了阴间鬼吏,要拷打我。”接着身上又红又肿,号叫着死了。席方平想着父亲临死时的悲惨样子,难过得吃不下饭去,说:“我父亲一生老实巴交。不会说巧话,如今竟被恶鬼诬告,遭人欺凌;我要到阴间去,替父伸冤报仇!”从此,席方平不再说话,时而坐着,时而站着,就像傻了一样。原来,他的灵魂已经离开他的躯体了。

席方平觉得刚一出门,不知道往什么地方去,见路上有行人,便上前询问去城里的路。一会儿,进了城。他父亲已被关在监狱里。席方平来到监狱门口,老远就看见父亲躺在屋檐下,样子很狼狈。父亲抬头看见儿子,伤心地哭起来,告诉儿子说:“狱吏们都受了羊某的贿赂,日夜拷打我,两腿都被打烂了。”席方平气愤极了,大骂狱吏:“我父亲要是有罪,自有王法处置,怎么能由你们这些死鬼随意摧残呢!”于是出了狱门,写下状子。正赶上城隍早上坐堂问事,席方平大喊冤枉投了状纸。姓羊的害怕了,里里外外贿赂串通遍了,才出来对质。城隍说席方平没有证据,断他无理。席方平满腹冤气,无法伸述,只好又在阴间走了一百多里路,来到郡衙,把城隍营私舞弊的情况申诉给郡司。郡司拖延了半月,才给审理,却把席方平痛打了一顿,仍然批回城隍复审。席方平回到县衙,受尽了酷刑,悲惨的冤情无处可伸。城隍怕他再上告,就派衙役押送他回家。衙役到龙口就回去了,席方平不肯进门,又偷偷地逃到阎王殿,控告郡司和城隍贪财受贿,残害无辜。阎王立刻派人押郡司、城隍来对质。郡司与城隍害怕,秘密派心腹找席方平说情,答应给席方平一千两银子。席方平不听。过了几天,旅店的主人告诉他说:“你太意气用事了!官府跟你求和你都不听,如今听说城隍与郡司都给阎王送了信去,你的案子恐怕糟了。”席方平以为这是道听途说,不太相信。

不久,有个穿黑衣服的衙役传席方平去见阎王。升堂后,席方平见阎王满脸怒气,不容他申辩,就命衙役打他三十大板。席方平厉声责问:“小人犯了什么罪?”阎王就像没听见一样。席方平挨着打,大喊:“我该打!我该打!谁叫我没有钱啊!”阎王更火了,命小鬼准备火床。两个小鬼把席方平揪下堂,见东台阶上有张铁床,床下燃着熊熊烈火,床面烧得通红。小鬼剥掉席方平的衣服,把他按到火床上,反复揉搓。席方平疼痛至极,骨肉都烙得焦黑,苦于死不了。大约过了一个时辰,小鬼说:“可以了。”就把他扶起来,叫他下床穿上衣服。虽然一瘸一拐的,幸而还能走路。又来到堂上,阎王问:“还敢再告吗?”席方平说:“大冤未伸,决不死心!如果说不告了,是欺骗大王,一定要告!”阎王说:“你告什么?”席方平说:“我所遭受的一切冤苦,全都要告!”阎王大怒,命小鬼用锯锯了他。两个小鬼把席方平拉过去,只见一根八九尺高的木柱竖在地上,旁边有两块木板,木板上下糊满血迹。小鬼刚要绑上他,忽然听到堂上大喊“席方平”,两个小鬼又把他押回去。阎王又问:“你还敢告吗?”席方平回答:“非告不可!”阎王命小鬼捉下去快锯。下堂后,小鬼用两块木板把席方平夹住,绑在木柱上。刚下锯,席方平觉得头渐渐成为两半,疼不可忍,却咬着牙一声不吭。听见一小鬼说:“好一条硬汉子!”锯声隆隆地响着,快锯到席方平胸间了,又听见一个小鬼说:“这人没有什么错,是个大孝子,锯稍偏一点,别损坏了他的心。”就觉得锯锋曲折着锯下来,席方平疼得更厉害了。一会儿,身体锯成两半。小鬼解下板子,席方平的两半身子都倒在地上。小鬼上堂大声回报,堂上传呼,叫把两半身子合起来去见阎王。两个小鬼就把两半身子又合到一块,拖着席方平走。席方平觉得中间锯缝疼痛得像又裂开了,走半步,就跌倒了。一个小鬼从腰间拿出一条红丝带给他,说:“送你这条带子,报答你的孝心。”席方平接过来捆在身上,立刻觉得身体健壮,没有一点痛苦。于是来到堂上跪下,阎王还像前面那样问他,席方平怕再受酷刑,便回答:“不告了。”阎王立刻命小鬼送他回人间去。

鬼差领席方平出了北门,指给他回家的路,转身回去了。席方平想,阴间的暗无天日,比阳间还要厉害,怎奈没有办法到上帝那里。世上传说灌口二郎神是玉帝的亲戚,为神聪明正直,向他告状,一定灵验。暗自高兴那两个鬼差已经走了,就转身往南跑去。正跑着,有两个鬼追上了他,说:“阎王怀疑你不回去,果然如此。”拖他回去再见阎王。席方平暗想,阎王一定非常恼怒,这次遭的罪更惨了。但是阎王没有一点生气的样子,对席方平说:“我知道你确实是个孝子,你父亲的冤案,我已为他昭雪了,如今他已经到富贵人家投生去了,哪里用得着你去喊冤?现在送你回去,给你千金家业,百岁之寿,该满足了吧?”接着就注在生死簿上,还盖上巨大的官印,叫席方平亲眼看了。席方平谢了恩退下,小鬼同他一起出了殿门。走在路上,小鬼驱赶着骂他:“你这奸滑的贼人!一次次地反复,叫我们跑来跑去,快累死了。要是再犯,就把你扔到大磨里,研成细末!”席方平瞪着眼怒骂:“你们这些小鬼想干什么?我这性子耐得住刀锯,可受不了抽打!请回去见阎王,阎王如果叫我自己回去,也用不着你们送我!”说完,就往回跑,两个小鬼害怕了,好言好语劝他回来。席方平故意慢慢地走,走几步就坐路旁歇一会儿,小鬼憋着一肚子气不敢说。走了半天,来到一个村子,一家大门半开着,小鬼拉席方平一块坐下歇歇,席方平就坐在门坎上。两个小鬼趁他没有防备,把他推进门去。席方平吃了一惊,再一看自己,身体已成了婴孩。他愤怒地哭着不吃奶,三天后就夭亡了。

席方平的灵魂飘瓢摇摇,没忘去灌口找二郎神。大约飘荡了几十里,忽然看见一队用鸟羽装饰的仪仗队走过来,旌旗戟铖摆满道路。席方平赶紧想跑开躲避,不想,冲撞了仪仗队,被前面的人逮住,用绳子捆着送到车前。席方平抬头见车中坐着一位年轻人,气度非凡,问席方平:“你是什么人?”席方平冤恨正无处发泄,猜想这一定是个大官,或许能利用他的权威为自己作主,就把自己的悲惨遭遇说了一遍。少年叫人给他松绑,让他跟着车子一块走。一会儿,来到一个地方,有十多个官员在路旁迎接。车中的少年向每个人打了招呼,然后指着席方平对一位官员说:“这是下界人,正想去找你诉冤。你最好立即察明案情,进行判决。”席方平向随从一打听,才知道车中坐着的少年是玉帝的九王子,他所嘱咐的人正是二郎神。席方平端详端详,见二郎神身材高大,胡须很多,不像世间传说的样子。

九王子走了以后,席方平跟着二郎神来到一座官署。看见他父亲与姓羊的及衙役都已经在那里。不一会儿,从囚车中押出来几个犯人,却是阎王、郡司和城隍。二郎神当堂审问,查明席方平所控告的全部属实。那三位官吓得战战兢兢,像老鼠一样趴在地上。二郎神立刻提笔写判决书。接着,传下判决书,叫案中所涉及到的人都看过。判决书写道:“查得阎王:职任王爵,身受帝恩,本当廉身自法,给臣僚作表率;不该贪赃枉法,招人责骂。而你却结党营私,夸耀你官阶的尊严;狠毒贪婪,玷污了臣子的气节。斧敲凿、凿入木一般,妇孺的骨髓为之一空;鲸吞鱼、鱼吃虾一样,蝼蚁般小命实在可怜。应当捧西江之水,给你洗肠;烧红东壁下的火床,请君入瓮。城隍、郡司:身为百姓的父母官,替上帝管理好百姓,虽然官位不高,也应尽心竭力,不辞辛苦;就是大官僚以权势相逼,有志气的也会刚正不阿。而你们却上下勾结,枉法作弊,早已忘了百姓的疾苦;任意施展你狡猾的奸谋,更不嫌乎鬼瘦。只知贪赃枉法,真是人面兽心!应该剔去你们的骨髓,剥去你们的皮毛,暂处以阴间死刑,罚你们转世投胎变作牛马。差役:既然身在鬼府就不是人类,只应在衙门里修身行善,或许还可转世为人;怎能在苦海中兴风作浪再造弥天大罪?飞扬跋扈,狗脸布满杀气;横行霸道,阻断九衢大路。在阴间滥施淫威,人人都知道狱吏惹不起;帮昏官干尽坏事,百姓们都惧怕你们屠夫般的凶残。应当在刑场上,剁去你们的四肢,再扔进油锅里,捞出你们的筋骨。姓羊的:富而不仁,狡诈多端。金光遮地,致使阎王殿上昏暗不明;铜臭熏天,直教枉死城中不见天日。铜钱能役使鬼卒,金银能买通神灵。应当没收羊家的家产,奖赏席生的孝行。上述人犯,立即押赴东岳大帝执行。”二郎神又对席廉说:“念你儿子孝义,你又生性善良懦弱,再赐给你三十六年阳寿。”说完就派两个差役送他们回家。

席方平抄写了判决书,路上父子两人一块读着。回到家,席方平先苏醒过来,叫家人启开父亲的棺材,见父亲僵硬的尸体仍然冷冰冰的。等到天黑了,才渐渐温暖复活了。再找那抄来的判词,已经不见了。从此,席家慢慢富裕起来,三年的功夫,良田成片。而羊家子孙则败落下去,楼阁田产,全部归了席家。村里有人想买羊家田产的,夜里都梦见神人叱责说:“这是席家的东西,你不能占有它!”起初还不相信,等种上庄稼,一年下来连一升半斗的粮食都收不到,于是,只得又卖给席家。席方平的父亲一直活到九十多岁才死。


\subsection{1.10.13   素 秋}
\label{\detokenize{p00_u5176_u5b83/_u767d_u8bdd_u804a_u658b_u5fd7_u5f02:id410}}
俞慎,字谨庵,出身于顺天一个官宦世家。他进京赶考时,住在郊区一所房子里,经常看见对门有一个少年,生得美如冠玉,心中很喜欢他。使渐渐接近他,同他交谈。少年谈吐尤其风雅,俞慎更加喜爱,拉着他的胳膊来到自己的住处,设酒宴款待。问他的姓氏,少年自称是金陵人,姓俞名士忱,字恂九。俞慎听到与自己同姓,更觉亲近,就同他结拜为兄弟。少年便将自己的名字减去“士”字,改为俞忱。

第二天,俞慎来到俞恂九家,见书房、住处明亮整洁,但门庭冷落,也没有仆人、书僮。俞恂九领着俞慎进入室内,招呼妹妹出来拜见,他妹妹年约十三四岁,肌肤晶莹明澈,就是粉玉也不如她白。一会儿,俞恂九的妹妹又端来茶敬客,好像家里也没有丫鬟、女佣。俞慎感到奇怪,说了几句话出来了。从此他们二人就像亲兄弟一样友爱。俞恂九没有一天不来俞慎的住所,有时留他住下,他就以妹妹弱小无伴而推辞。俞慎说:“弟弟离家千里,也没有个应门的书僮;兄妹俩又纤弱,怎么生活啊!我想,你们不如跟我去,一同住在我那儿,怎样?”俞恂九很高兴,约定考完试后随他回去。

考试完毕,俞恂九把俞慎请到家,说:“今天是中秋佳节,月明如昼。妹妹素秋准备了酒菜,希望不要辜负了她的一番心意。”说完,拉着俞慎的手来到内室。素秋出来,说了几句客套话,就又进屋,放下帘子准备饭菜。不多时,素秋亲手端出菜肴来。俞慎站起来说:“妹妹来回奔波,让我怎么过意得去。”素秋笑着进去了。一会儿,就有一个穿青衣的丫鬟捧着酒壶,还有一个老妈妈端着一盘烧好的鱼出来。俞慎惊讶地说:“她们是从哪里来的?为什么不早点出来侍候,却麻烦妹妹呢?”俞恂九微笑着说:“素秋又作怪了。”只听到帘内吃吃的笑声传来,俞慎不解其中的缘故。到了散席的时侯,老妈妈同丫鬟出来收抬碗筷。俞慎正在咳嗽,不小心将唾沫吐到丫鬟衣服上,丫鬟应声摔倒,碗筷菜汤撒了一地。再看那丫鬟,原来是个用布剪的小人,只有四寸大小。俞恂九大笑起来,素秋也笑着出来,捡起布人走了。不一会儿,丫鬟又出来,像刚才一样奔忙。俞慎更加惊异,俞恂九说:“这不过是妹妹小时候学的一点小魔术罢了。”俞慎于是又问:“弟弟妹妹都已长大成人,为什么还没成亲呢?” 回答说:“父母已经去世,我们是留是走还没有拿定主意,所以拖了下来。”接着两人商定了动身的日子,俞恂九将房子卖了,带着妹妹同俞慎一块西去。

回到家后,俞慎教人打扫出一所房子,让俞恂九兄妹住下,又派了个丫鬟侍候他们。俞慎的妻子,是韩侍郎的侄女,非常爱怜素秋,每顿饭都在一块吃。俞慎同俞恂九也是这样。俞恂九非常聪明,读书时一目十行,试着作了一篇文章,就是那些名望的老儒也比不上他。俞慎劝他去考秀才,俞恂九说:“我暂时读点书,不过是想替你分担些辛苦罢了。我自知福分浅薄,不能做官;况且一旦走上这条路,就不能不时时担忧,患得患失,所以不想去考试。”

生了三年,俞慎考试又落了榜。俞恂九为他抱不平,奋然说:“榜上挂个名字,怎么就艰难到这种地步!我当初为成败所迷惑,所以宁愿清清静静地生活。如今看到大哥不能施展文才,不觉心热。我这十九岁的老童生,也要像马驹一样去奔驰一番了。”俞慎听了很高兴,到了考试的日期,送他去考场,结果他县、郡、道三场都考了第一名。从此俞恂九与俞慎一块更加刻苦攻读。过了一年参加科试,两人并列为郡、县冠军。俞恂九从此名声大噪,远近的人都争着来提亲,俞恂九都拒绝了。俞慎竭力劝说他,他就推说等参加乡试以后再说。不久,乡试完毕,倾慕俞恂九的人都争着抄录他的文章,互相传诵。俞恂九自己也觉得必定名列榜首了。等到放榜,兄弟两人却都榜上无名。当时他们正在对坐饮酒,听到这消息,俞慎还能强作笑语;俞恂九却大惊失色,酒杯掉在地上,一头扑倒在桌子下面。俞慎急忙把他扶到床上,恂九的病情却已经十分危险了。急忙喊妹妹来,俞恂九睁开眼对俞慎说:“我们俩交情虽如同胞,其实不是同族。小弟自己感到快要死了,受你的恩惠无法报答。素秋已长大成人,既蒙嫂嫂抚爱,你就纳她为妾吧。”俞慎生气地说:“兄弟这是胡说什么呀!你以为我是那种衣冠禽兽吗?”俞恂九感动地流下眼泪。俞慎用重金为他购置了上等棺材,俞恂九让人把棺材抬到跟前,竭力支撑着爬进去,嘱咐妹妹说:“我死以后,立即盖好棺盖,不要让任何人打开看。”俞慎还有话想说,俞恂九已经闭上眼睛死了。俞慎十分哀伤,如同死了亲兄弟。可是私下里却怀疑俞恂九的遗嘱奇怪。趁素秋外出,他偷偷打开棺材一看,见里面的袍服像蝉蛇褪下的皮。揭开衣服,有一条一尺多长的蠹鱼,僵卧在里面。俞慎正在惊讶,素秋急匆匆地进来了,惨痛地说:“你们兄弟之间有什么隔阂?我们所以这样做,并不是避讳兄长;只是怕传播声扬出去,我也不能在这里久住了!”俞慎说:“礼法因人情而判定,只要感情真挚,不是同类又有什么区别呢?妹妹难道还不知道我的心吗?就是你嫂嫂我也不会泄漏一句的,请你不要忧虑。”于是很快选定了下葬的日期,把俞恂九厚葬了。

当初,俞慎想把素秋嫁给官宦世家,俞恂九不同意。俞恂九死后,俞慎又同素秋商量这事,素秋不肯。俞慎说:“妹妹已经二十岁了,再不嫁人,人家会说我什么呢?”素秋回答说:“如果是这样,我就听兄长的。但是我自觉得没福分,不愿嫁给富贵人家。要嫁,就嫁给一个穷书生吧。”俞慎说:“可以。”不几天,媒人一个接一个的来,但素秋都不中意。先前,俞慎的妻弟韩荃来吊丧,见到过素秋,心里非常喜爱她,想买她作妾,同他姐姐商量。姐姐急忙告诫他不要再说了,怕俞慎知道生气。韩荃回家后,始终不死心,又托媒人传信给俞慎,许诺为姐夫买通关节,保证他乡试得中。俞慎听了后,勃然大怒,将捎信的臭骂了一顿,打出门去。从此与韩荃断绝了交往。后来又有个已故尚书的孙子某甲,将要娶亲时,没过门的媳妇忽然死了,也派媒人来俞慎家提亲。某甲家宅第高大,家财万贯,俞慎平素就知道,但想亲眼见一见某甲本人,就同媒人约定日期,让某甲亲自来见。到了那天,俞慎让素秋隔着帘子,在里边自己相看。某甲来了,身穿皮袍骑着骏马,带领一大帮随从,向街坊四邻炫耀自己的富有。再看他人长得秀雅漂亮,像个姑娘,俞慎非常喜欢。看见的人也都纷纷称赞,但素秋却很不乐意。俞慎没听索秋的,竟许了这门亲事,给素秋准备了丰厚的嫁妆,花钱毫不计较。素秋再三制止,说是只带一个大丫头侍候就行了,俞慎也不听,终究还是陪送得很丰厚。

素秋出嫁以后,夫妻感情很好,但是兄嫂常挂念她,每月总要回来一趟。来时,妆盒中的首饰,总要拿回几件,交给嫂子收存。嫂嫂不知她的意思,姑且依从她。某甲从小父亲就没了,守寡的母亲对他过分溺爱。他经常和坏人接触,渐渐被引诱去嫖妓、赌博,家传书画、珍贵的古玩,都让他卖掉还债了。韩荃与他相识,便请他喝酒,暗中试探他,说愿用两个小妾加上五百两银子交换素秋。某甲开始不同意,韩荃再三请求,某甲的心有些动了,但又怕俞慎知道不答应。韩荃说:“我与他是至亲,况且素秋又不是他家中的人,如果事情办成了,他也拿我没办法。万一有什么事,由我出面承担。有我父亲在,一个俞慎有什么可怕的!”接着让两个侍妾打扮得漂漂亮亮出来劝酒,并且说:“如按我说的办,这两个女子就是你家的人了。”某甲被韩荃迷惑住了,约定好交换日期,就回去了。到了那天,某甲怕韩荃欺诈他,半夜就在路上等着,看到果然有辆车子前来。掀开帘子,见里面的人果然不假,就领她们回家,暂且安置在书房里。韩荃的仆人又拿出五百两银子,当面交清。某甲便跑入内室,骗素秋说,俞慎得了急病,叫她赶快回家。素秋来不及梳妆,急匆匆地出来,上车就走了。夜里看不清方向,车子迷了路,走了很远,也没有到韩荃家。忽然看见两支巨大的蜡烛迎面而来,大家暗暗高兴,以为可以问路了。不一会,走到跟前,原来是一条巨蟒,瞪着两只像灯一样的大眼睛。众人害怕极了,人和马都逃窜了,把车子丢在路旁。天明了才会齐回去一看,只剩下一辆空车子了。他们认为素秋一定是被大蟒吃了,回去告诉主人,韩荃只有垂头丧气而已。

几天后,俞慎派人来看望妹妹,才知道素秋被坏人骗走了。当时也没有怀疑是素秋的女婿搞鬼。直到陪嫁的丫头回来,细说了事情的经过,俞慎才觉出其中有变故,不禁气愤至极,跑遍了县府到处告状。某甲很害怕,向韩荃求救。韩荃因为人财两空,正在懊丧,拒绝了他的要求,不肯帮忙。某甲傻了眼,没有一点办法。府、县拘票来到。他只好贿赂衙役,才暂时没被带走。过了一个多月,金银珠宝连同服饰全被他典卖一空。俞慎在省衙追究得很急,县官也接到上司严加追查的命令。某甲知道再也不能躲藏了,才出来到公堂说出全部实情。省衙又下传票,拘捕韩荃对质。韩荃害怕,把事情的经过告诉了父亲。他父亲当时已退职在家,恼怒儿子的作为不守法,把他绑起来交给了衙役。韩荃到了各官府,说到遇见大蟒的变故,官府以为他是胡编乱造,将他的仆人严刑拷打,某甲也挨了好几次板子。幸亏他母亲整日变卖田产,上下贿赂营救,韩荃才受刑不重,免于一死,韩荃的仆人却已经病死在狱中。韩荃长期囚禁牢中,愿意帮助某甲送一千两银子给俞慎,哀求撤销这桩案子,俞慎不答应。某甲的母亲又请求再加上两个侍妾,只求暂作疑案搁一搁,等他们去寻访素秋。俞慎的妻子又受了叔母的嘱托,天天劝解,俞慎才应允不再去催。某甲家中已经很贫穷,想卖掉宅子凑点银两,宅子一时又卖不出去,就先送了侍妾来,乞求俞慎暂缓交银日期。

过了几天,俞慎夜里正坐在书房中,素秋同一个老妈妈忽然进来了。俞慎惊奇地问:“妹妹原来平安无事啊?”素秋笑着说:“那条大蟒是妹妹施的小法术。那天夜里我逃到一个秀才家里,和他母亲住在一起。他说认识哥哥,现在门外,请他进来吧。”俞慎急得倒穿鞋子迎出去,拿灯一照,不是别人,原来是周生,是宛平县的名士,两人一向意气相投。俞慎拉着周生的手进了书房,摆酒宴招待,亲热地谈了很久,才知道事情的原委。

原来,素秋天将明时,去敲周生的门,周母接她进去,仔细询问,知道是俞公子的妹妹,就要派人通知俞慎,素秋制止了,就和周母住在一起。周母看她聪慧,善解人意,很喜欢她。因为周生还没有娶亲,就想把她娶来给儿子作媳妇,还含蓄地透露了这个意思。素秋以没有得到哥哥的同意而推辞。周生也因为与俞公子交情很好,不肯没有媒人就成亲。只是经常打听消息,得知官司已经说情调解,素秋便告诉周母想回家。周母让周生带一老妈妈送她,并嘱咐老妈妈作媒提亲。

俞慎因为素秋在周家住了很长时间,也有意把素秋嫁给周生。听说老妈妈是来作媒的,非常喜欢,就同周生当面订好了这门亲事。原先,素秋夜里回来,是想让俞慎得了银子后再告诉别人,俞慎不肯这么办,说:“以前是因为气愤无处发泄,所以想借索取钱财让他们尝尝败家的苦头。如今又见到妹妹,一万两银子也换不来啊!”马上派人告诉那两家,官司算了结了。又想到周生家不太富裕,路途遥远,迎亲很艰难,就让周生母子搬来,住在俞恂九原来住过的房子里。周生也备了彩礼,请了鼓乐队,举行了婚礼。

一天,嫂嫂同素秋开玩笑:“你如今有了新女婿,从前和某甲的枕席之爱还记得吗?”素秋笑着问丫头说:“还记得吗?”嫂嫂感到疑惑,就追问她。原来素秋在某甲家三年,枕席之事都是让丫头代替的。每到晚上,素秋用笔给大丫头画好双眉,让她过去陪某甲。即便是对着蜡烛坐着,某甲也分辨不出来。嫂嫂更加惊奇,请求素秋教给她法术,索秋只笑不说话。

第二年,是三年一次的大会考。周生准备同俞慎一块去赶考,素秋说:“不必去。”俞慎强拉着周生去了,结果俞慎考中了,周生落了榜。回来后便打算不再去应考了。过了年,周母去世,周生再也不提赶考应试的事了。

一天,素秋告诉嫂嫂说: “以前你问我法术,我本不肯用这些来惹别人惊骇。现在要离别远去,让我秘密传授给你,也可以躲避兵火。”嫂嫂吃惊地问她缘故,素秋回答说:“三年后,这里就没有人烟了。我身体弱,受不住惊吓,要去海滨隐居。大哥是富贵中的人,不能一起去,所以说要离别了!”就将法术全部教给嫂嫂。几天后,素秋又告诉俞慎。俞慎留不住她,难过得流泪,问:“到什么地方去?”素秋也不说。鸡一叫就早早起来,带一个白胡须的老仆,骑着两头驴走了。俞慎叫人暗暗跟在后边送她,到了胶州、莱州一带,尘雾遮天,晴天后已经不知道她们往哪里去了。

三年后,李自成率众造反,村里的房屋变成了一片废墟。韩夫人剪个布人放在大门里面,义兵来了,看到院子里云雾围绕着一丈多高的天神守着,就吓跑了。因此,全家得以安然无恙。

后来,村中有一个商人到海上,遇见一个老头,像是素秋的老仆。但是老头的胡子头发全是黑的,不敢贸然相认。老头停下笑着说;“我家公子还安康吧?请你捎个口信,素秋姑娘也很安乐。”商人问他住在什么地方,老头说:“很远,很远,”就急忙走了。俞慎听说后,派人到海边四处寻访,竟没有一点踪迹。


\subsection{1.10.14   贾 奉 雉}
\label{\detokenize{p00_u5176_u5b83/_u767d_u8bdd_u804a_u658b_u5fd7_u5f02:id411}}
贾奉雉,是甘肃平凉人。他的才名冠绝一时,但是科举考试却总是不中。

有一天,他在道上遇见一位秀才,自称姓郎,风度很潇洒,言谈也很有学问。贾奉雉就邀他一起回到家里,拿出自己的八股文习作向他请教。郎生读完后,不很赞许,说道:“您的文章,科试得个第一名肯定有余,然而乡试考场想取个榜尾恐怕也不够格。”贾奉雉说:“那怎么办呢?”郎生说:“天下之事,仰着头踮起脚去高攀倒很难办到;而低下头去俯就却容易得多,这些道理还用得着我来说吗!”于是指出了一两个人和他们的一两篇文章作为标准,大致都是贾奉雉最看不起而不屑一提的。贾奉雉听完后,笑着说:“学者作的文章,贵在能历久不朽,即使把它列入八珍美味之中,也应当使天下人不认为过分才是。像你所说的这两个人,用那样低劣的文章来猎取功名,虽然登上显贵的台阁高位,他们仍然是低贱的。”郎生说:“并非这样。有的人文章虽然写得好,但是由于他的地位低贱却不能流传。您要想死抱着自己的卷子一直到老那就罢了;否则,连那些主考官们,都是靠这等劣质货色爬上去的,恐怕不会因为看了你的好文章,就会另外换上一副眼睛和肝肺肠子的。”贾奉雉最终不说话了。郎生起身笑着说:“你还是年轻气盛啊!”于是告辞走了。

这一年乡试的时候,贾奉雉赶考又落榜了。他心情郁闷很不得志,渐渐想起郎生说过的话,就拿出以前他所指出的那一两个人的文章来勉强阅读。可是还没读完,就先昏昏欲睡,心里疑惑,拿不定主意是不是按郎生说的办。

又过了三年,乡试的日期将近,郎生忽然来到,两人相见非常高兴。郎生于是拿出自己所拟好的七篇八股文的题目,让贾奉雉来作。过了一天,他就索要文章来看,认为写得不行,再让贾奉雉重作;作完了再看,又说不好。贾奉雉便开玩笑地把以往自己参加乡试未中的卷子找出来,将里面那些芜杂冗长、空洞浮泛难以见人的词句集中起来,胡乱拼凑成文,等郎生来了又让他看。郎生一看高兴地说:“这一回可以了!”就让他熟记,一再叮嘱不要忘了。贾奉雉笑着说:“和您实说吧:这些东西都不是我心里想写的,转眼就忘了,即便受责打,也不可能再记起它了。”郎生坐在书桌旁边,硬逼着贾奉雉朗诵了一遍;又叫他脱去上衣露出脊背,用笔在上面写上了一道符,临走出门时说:“仅有这些就足够了,可以把其它的书都束之高阁了。”贾奉雉检查了一下自己背上的符,想洗也洗不掉,已经渗透到皮肉里面了。

贾奉雉进了乡试考场中,一看发下来的试卷题目,郎秀才所拟的七道题一道也没漏下。回想自己以前几次所作的文章,心中一片茫然,怎么也记不起来了。惟有那篇开玩笑拼凑的文章,仍历历在心。但他手握毛笔,始终感到写那样的文章太丢人;想稍作一下更改,但反复苦想,竟然不能改换一字。太阳偏西了,贾奉雉只得按着记忆照直抄录下来交卷出场。郎生等候他已经很久了,见面就问道:“怎么回来得这样晚?”贾奉雉如实相告,并立即求他擦去自己脊背上的符;可是脱衣一看,符已经消失了。再回忆考场中的作文,竟如隔世的事情一样没了印象。贾奉雉大为惊异,就问郎生说;“您为啥不用这种办法自己参加考试呢?”郎生笑着说:“我只因没有这种念头,所以就能不理会这些文章。”于是约贾奉雉明天到他家里去,贾奉雉答应了。郎生走了以后,贾奉雉拿出那七篇文稿自己阅读,大非本意,怏怏不乐,也不再践约去访郎生,便垂头丧气地回了家。

过了不久,乡试榜张了出来,贾奉雉竟然考中了第一名。他又最新阅读那七篇旧文稿,真是一读一身汗,读到最后,好几层衣服全湿透了。他自言自语地说:“这样的文章一公布出来,怎么有脸去见天下的读书人呢!”正在羞愧难当之际,郎生忽然来到,说:“你希望考中就中了,怎么还这样闷闷不乐呢?”贾奉雉说:“我恰好在想,这是用金盆玉碗盛狗屎,真无脸再出去见同人。我将要离家隐居到山林之中,与尘世永绝了。”郎生说:“这样做也确实很高明,只是怕你办不到。果真能行的话,我就能为你引见一个人,可以学得长生不老,连同千年的盛名,也都不值得留恋了,何况是无意得来的富贵呢!”贾奉雉听了很高兴,便留下他和自己同宿,说:“容我再想想这事。”到了天明,贾奉雉对郎生说:“我的主意已经定了!”他也不告诉老婆孩子一声,竟飘然离家出去了。

他俩渐渐地进了深山,到了一处洞府,里面别有一番天地。有个老人坐在堂上,郎生叫贾奉雉过来参拜老人,称呼他师父。老人说:“怎么来早了?”郎生说:“他修道的决心已经下定了,盼望师父能收录他。”老人向贾奉雉说道:“你既然来了,必须把自己的身子一并置之度外,这样才能得道。”贾奉雉很礼貌地连连答应着。郎生把他送到另一处院子里,给他安排好睡觉的地方,又为他拿来吃的糕饼,这才走了。贾奉雉见这房子也还精致清洁;只是门上没有门扇,窗上没有窗棂,里面仅有一张小书桌和一张床铺。他脱下鞋子上了床,月光已经从门窗中射进来了。他感到肚子稍微有点饿,就拿过糕饼吃起来。糕饼的味道很甜美,只吃了一点就饱了。心里暗想郎生一定还再回来,但是坐了很久却静悄悄的,一点声音也没有。只觉得屋子里充满了清香味,自已的脏腑也竟然清晰透明起来,里面的脉络都能历历可数。忽然听见屋外有很尖厉的声响,就像猫抓痒的动静。贾奉雉从窗子向外一看,原为是只老虎蹲在屋檐下面。乍一见老虎,他吓了一大跳;转而想起了师父说的话,就又收回了心神,端坐在那里。老虎好像知道里面有人,随即进屋走近床铺,使劲用鼻子吸气,把贾奉雉的脚和腿闻了个遍。不一会儿,听到院子里有东西鸣叫乱扑楞,像是鸡被绑住了,老虎立即迅速奔出屋去。

贾奉雉又坐了一会儿,一个美女进了屋,兰麝熏香扑面而来,她悄然无声地登上了床,趴在贾奉雉的耳朵上小声地说道:“我来了。”一说话,散发出一阵口脂的香气。贾奉雉紧闭双眼,一点也不动心。美女又低声说:“睡着了吗?”声音很像他的妻子。贾奉雉的心略微动了一下,可又一想:“这都是师父为了试探我耍弄的幻术罢了。”依然闭着眼睛。美女笑着说:“老鼠动了!”当初,贾奉雉夫妻和丫鬟同住在一屋,做爱时恐丫鬟听见。就背后约好一句暗语说:“只要说‘老鼠动了’,就开始亲热。”如今贾奉雉忽听这句话,不觉大为动心,睁开眼仔细一看,真是他的妻子无疑。就问她道:“你怎么会来到这里?”妻子回答说:“郎秀才怕您自己寂寞想回家,派去了一位老太婆领我来的。”说话之间,两人偎靠在一起,妻子对他离家出走时没说一声非常怨恨。贾奉雉安慰地好长时间,她才高兴地和他亲热起来。过后,天也快亮了,听见师父怒斥的声音,离院子越来越近。贾妻急忙起来,见无处藏身,就跑过矮墙走了。

不一会儿,郎生跟在师父身后进了门。师父当着贾奉雉的面用棍子打了郎生,随后便叫他把贾奉雉赶出去。郎生也只好领着贾奉雉从矮墙上出去了,说道:“我对您的期望有点过分,未免太激进了;没想到你的情缘未断,连累我也挨了责打。你从这里暂且回去,我们以后再见的日子不远了。”说完为他指明了回家的路,两人于是拱手而别。

贾奉雉在山上俯视自己的村子,原来就在眼前。心想妻子步小走得慢,一定还在半路上。他疾奔了一里多路,已经到了家门口。只见房屋院墙破败不堪,不是原来的老样子了;村里的老人小孩,竟然没有一个认识的。心里这才感到惊异,忽然想起汉朝的刘晨、阮肇二人遇仙后从天台上返回家园时,所见情景和今天的模样非常相似。他没敢再进家门,就在对门坐下休息。过了很久,有个老翁拖着根拐杖从里面出来。贾奉雉向他拱手行礼,问道:“贾奉雉家在哪儿?”老翁指着贾宅说:“这就是。莫非您要问那桩奇事吗?我全都知道。相传这位贾公当时听说自己考中了举人就逃走了;走的时侯,他的儿子才七八岁;后来到了十四五岁的时候,贾夫人忽然又大睡不醒。儿子在世的时候,冷了热了还能够为母亲换换衣服;等到儿子死了,两个孙子很穷,房子也拆毁了,只好用木头扎了架子,盖上点草苫子给贾夫人遮蔽风雨。一个月前,贾夫人忽然醒过来,屈指一算已经一百多年了。远近的人听说这件奇事,都来寻访观看,近几天的人才少了点。”贾奉雉听说恍然犬悟,说:“老翁有所不知,贾奉雉就是我呀。”老翁大惊,急忙走去告诉贾家的人。

此时贾奉雉的长孙早死了;他的次孙贾祥也已经到了五十多岁。孙子认为他长得年轻,怀疑他是冒充伪装。不多时,贾夫人出来,才认出他来。顿时泪流不止,叫着他一块进了家门。夫妻二人苦于没有房子,只好暂时进了孙子的屋里。一时男女老幼,跑来挤满了一屋,都是贾奉雉的曾孙、玄孙辈,大都粗俗无知。长孙媳妇吴氏,买酒并准备了粗茶淡饭招待他们;又叫小儿子贾杲和媳妇,同自己共住一屋,倒出房子清理干净让祖父母去住。贾奉雉住进了为他准备的房子,里面烟熏火燎的气味再加上小孩子的尿味,实在难闻。住了几天,他悔恨得不得了。两个孙子家分别轮换着供给他们吃喝,饭菜做得很不对口味。村里人因为贾奉雉百年新归,天天请他去喝酒;然而贾夫人却经常吃不上饱饭。长孙媳妇吴氏是读书人家的女儿,很懂闺训家规,对祖父母一真很孝顺。而次孙贾祥家里送的饭菜越来越少,有时得呼喊着才给他们送一点来。贾奉雉很生气,就带着夫人离开这里,到东村设帐教学去了。他常对夫人说:“这次回家我非常后悔,但是已经来不及了。不得已,只好再重操旧业,倘若心里不再感到羞愧的话,要想富贵也并不是难事。”

过了一年多,长孙媳妇吴氏还时时给他们送些东西来;而次孙贾祥父子竟然和他们断了来往了。这一年,贾奉雉考中了秀才。县令很看重他的文才,便厚厚地赠送钱财资助他,从此家里稍微富裕了起来。贾祥也渐渐地来近乎他。贾奉雉把他叫进来,算了算过去用他的饭钱,拿出银子偿还了他,并喝斥他离开,永不来往。于是贾奉雉买了一处新宅子,让长孙媳妇吴氏搬过去同住在一起。吴氏有两个儿子,大儿在家留守旧业;小儿贾杲很聪明,贾奉雄便叫他和自己的学生们在一起读书。

贾奉雉从深山回来以后,脑子更加清晰好用。不久,他参加乡试、会试连中,成了进士。又过了几年,贾奉雉以监察御史的职衔巡按浙江。他声名显赫,家中楼台歌舞,称盛一时。但是贾奉雉为人刚正,不媚权贵,朝中的大官们都想陷害他。他曾屡次上疏请求辞官回乡,一直没得到皇帝的准许,不久祸患就发生了。

原先,贾祥的六个儿子都是些无赖之辈,贾奉雉虽然不理睬并拒绝他们进门,但是他们都利用贾奉雉的势力作威作福,蛮横地强占别人的田宅,乡邻们都认为他们是些祸害。有个某乙才新娶了个媳妇,被贾祥的次子夺去当了妾。某乙本来就诡诈,乡邻们又凑钱帮助他去告状,因此这件事就传到京城。当权的那些大官于是都纷纷奏章攻击贾奉雉。贾奉雉毫无办法来为自己辨白,被关进牢狱一年多。贾祥和他的次子都病死在狱中。后来贾奉雉奉旨充军辽阳。当时贾杲考中秀才已经很久了,他为人非常仁义厚道,名声很好。贾奉雉夫人后来生的一个儿子,年已十六岁了,就把他托付给了贾杲。贾奉雉夫妻二人这才带着一个男仆和一个女仆上路赴辽阳。贾奉雉说道:“这十几年的富贵,还不如一场梦的时间长。如今才知道荣华的官场,都是地狱的境界,悔比刘晨和阮肇多造了一重罪孽呀!”

他们走了几天,抵达海边,远远地看见有一艘巨大的船向这边驶来,上面鼓乐齐鸣,侍卫们都像些天神。大船靠近岸边后,从里面走出一个人来,笑着请贾御史上船休息一下。贾奉雉一见那人惊喜异常,一纵身就跃了过去,押解他的官差也不敢阻挡。贾夫人急忙想跟过去,但大船已经驶去很远了,于是她气愤地投入海中。才漂泊了几步。就见有个人从船上垂下一条白缎子来,把她引救到船上而去。

押解的官差赶紧登上小船,叫划桨的快划,一边追一边大喊。只听到大船上鼓声如雷,和轰鸣的浪涛声交相呼应,转眼间就不见了。贾奉雉的仆人认识大船上的那个人,原来他就是郎生。


\subsection{1.10.15   胭 脂}
\label{\detokenize{p00_u5176_u5b83/_u767d_u8bdd_u804a_u658b_u5fd7_u5f02:id412}}
山东东昌府,有个姓卞的,以医牛为业。他有个女儿,名叫胭脂,从小生长得聪明伶俐,卞医生很喜欢她,一心想给她找一门读书人家的子弟作女婿。而当地大户人家却因为他家出身寒贱,没有愿意同他家结亲的,因此,胭脂虽已经长大,但还没找到称心的婆家。

卞家对门,是一家姓龚的,他的妻子王氏,为人很轻浮,爱开玩笑,平日常到胭脂闺房中闲谈,是胭脂的好友。一天,胭脂送王氏到门口,见到一位少年从门前走过,穿戴一身白色衣帽,生长得风度翩翩,相貌出众。胭脂对他产生了好感,有点动心,两只水灵灵的大眼睛直瞅着他。那青年含羞地低下头,快步走了过去。青年已经去了很远,胭脂还在注目远望。王氏看透了胭脂的心意,开玩笑地说:“姑娘以你的才貌,若匹配那位少年,才算是终生无遗憾了。”胭脂两颊红若桃花,含情脉脉,也不出声。王氏又问;“认识这位青年吗?”胭脂回答说:“不认识。”王氏说:“这就是南边巷子里的鄂秀才,名叫秋隼,那位已死去的孝廉的儿子。我与他就住在一条巷子里,所以认识他。人世间的男子,没有比他再温情的,没有比他更会体贴人的。今天他穿一身素白的衣服,是因为妻子刚死去不久,服丧期未满。姑娘您若对他有意的话,我代您给他传个信,叫他托媒人来提亲。”胭脂没有出声,王氏戏笑地走了。

几天过去了,没见回信,胭脂心中怀疑王氏没有马上告诉鄂秋隼;又怀疑他是乡绅的后代,不肯降低身份与她结亲。心中闷闷不乐,犹豫不决,苦苦地思念,渐渐地不吃不喝,病倒在床上,只感非常劳累。王氏正好来看望她,追问她的病因。胭脂回答说:“我自己也说不清楚,只是那天分别后,就觉精神恍惚,心中不快。现在这样气息奄奄,只怕是命在朝夕了。”王氏小声说:“我家的男人出去作买卖还没回来,还找不到人告诉鄂秋隼。你现在身体病成这样,是否就是为的这个?”胭脂脸羞红了很长时间。王氏戏笑地说:“果真为了这件事,身子已经病成这步田地,还有什么可顾忌的!假若先叫他夜晚来与你相会,他还会不同意吗?”胭脂叹口气说:“事情已经这样了,不能再顾面子了。只要他不嫌我出身贫寒,就赶快让他找媒人来,我的病就好了。若是私下约会,是万万不可的。”王氏点点头,就走了。

王氏在小的时候,就同邻居的一个书生宿介私通,即使出嫁以后,宿介只要打听到她的丈夫外出,就来找她寻旧相好。这天夜里,宿介正好来到王氏家中,王氏就把胭脂的痴情当作笑话向他述说,并戏笑地告诉宿介,给鄂生传个话。宿介很早就知道胭脂的美丽,听说后心中暗自高兴,庆幸自己有机可乘。本打算让王氏帮助他,但又怕王氏嫉妒。于是,就说了些漫不在意的话,但他对胭脂家的情况,问得很详细。

第二天夜里,宿介越墙进了胭脂家的院子,径直来到胭脂的住房,用指头叩她的窗户。胭脂在里边问:“是谁?”宿介回答说:“鄂秋隼。”胭脂说:“我所以思念你,为的是百年之好,不是为这一晚上的欢快。你如果真的爱我,就应当快请媒人;假若想私会,我是无法答应的。”宿介假装答应,却苦苦哀求握一下胭脂纤细的手表示诚意。胭脂也不忍心过于拒绝他,就用力支撑着身子去开门。宿介很快地闪入,抱着胭脂求欢。胭脂无力支撑,倒在地上,喘不上气来。宿介急忙去拉她。胭脂说:“哪来的恶棍少年,你必定不是鄂公子!如果是鄂公子,他为人温存、驯良,知道我是为他生的病,应当很体恤我,哪里会这样粗暴!假若你再这样,我就大声叫喊,你的品行也全完了,这对我们俩都没有好处!”宿介恐怕假装鄂秋隼的马脚败露,不敢再强求,但请求她说定再会的日期。胭脂说以迎娶的那一天作为见面之期。宿介认为这太远了,又让她再定个日期。胭脂实在讨厌他的纠缠,便约定等她病好。宿介又向她要件凭信的东西,胭脂不允许。宿介就捉住胭脂的脚,把她的绣鞋脱下来。胭脂喊他回来,说:“我的身子都许给你了,再还有什么可吝惜的,只恐怕‘画虎不成反类狗’,以致给别人遗留唾骂的笑料。现在我的绣鞋已经到了你的手,料想你也不会给我。若你背信弃义,我只有一死。”宿介出了胭脂的家,又到王氏家中投宿去了。宿介躺下后,心里仍然挂念着那只鞋,暗暗地摸摸衣袖,竟然已经没有了。急忙起来点灯,抖搂着衣服寻找。王氏问他,也不答应。宿介怀疑是王氏藏起来了,王氏故意地戏笑着让他怀疑。宿介感到不能再隐瞒了,就将实情告诉了王氏。说完,两人点起灯火,找遍门外,就是没有找到绣鞋,只好懊丧地回去睡了。心里还暗暗庆幸,深夜无行人,丢了也应在路上。但一早起来去寻找,仍然毫无踪影。

在此之前,同街有个游手好闲的二流子叫毛大,曾经勾引王氏遭到拒绝。他知道宿介和王氏有私情,就想用捉奸的方式来要挟她。这天夜里,毛大经过王氏门前,推了推门,没有关,便偷偷地摸了进去。刚走到窗户外面,就踏着一件像丝绵样软软的东西。拿起来一看,原来是用一条汗巾包着的一只绣鞋。毛大趴在窗户上细听,正好听到宿介在详细讲述事情的经过,他高兴极了,赶快悄悄溜出了王氏的家。

过了几夜,毛大爬墙来到胭脂家。由于门户不熟悉,竟误走到卞老汉房门前来了。卞老汉隔窗看到一个男人的影子,细看他的行踪,知道是为女儿而来。顿时,心中怒火上冲,拿起一把砍刀,奔了出来。毛大一看,吃了一惊,拔腿就跑。刚要爬上墙头,卞老汉已追上。急得毛大走投无路,转过身来夺老汉的刀。这时卞老婆也起来大声喊叫,毛大眼看无法逃脱,就势杀了老汉,夺路逃走了。这时胭脂的病已稍有好转,听到喧闹的声音,也急忙赶了来。母女俩点灯一照,老汉脑袋已被劈开,不能说话,不一会儿就断了气。在墙脚下拣到一只绣鞋,老太婆一看,是胭脂的,在母亲的追问下,胭脂哭着把那晚上的情形告诉了母亲,但不忍心连累王氏,只说鄂生自己来的。

天亮以后,到县里告了状,县令逮捕了鄂生。鄂生为人谨慎,又不善说话,当时十九岁,见到客人就像小孩子那样腼腆。他突然被捕,害怕极了。上了公堂不知说什么好,只有浑身颤抖。于是县令更加相信他就是凶手,对他重刑拷打。鄂生忍受不了皮肉之苦,屈打成招。押到府里,也同样受尽了刑罚。鄂生一肚子冤气,无处诉说。每次都想与胭脂对质,但一见面,胭脂就破口大骂,因而有口难辩,最后被定为死罪。以后,虽经许多官吏,反复审讯,也没有不同的口供。

后来,案子交给济南府复审,太守是吴南岱。他一见鄂生,觉得他不像杀人犯。就暗中派人细细盘问,让鄂生把心里话都说出来。吴太守也就更加明白了鄂生的冤情。谋划了好几天,才开庭审理。他先问胭脂:“你们订约后有人知道吗?”回答说:“没有。”“你遇上鄂生时,有人在场吗?”胭脂回答说:“没有。”于是,吴太守传鄂生上堂,好言安慰他一番。鄂生主动说道:“我曾从她家门前走过,只看到老邻居王氏和一个姑娘走出来,我就快步走开了,连一句话都没说。”吴太守吓唬胭脂说:“刚才你说没有别人在场,为什么有个邻居妇女?”说着就要动刑。胭脂害怕了,说:“虽然有王氏在场,和她实在没有牵连。”吴太守暂停审问,命令拘留王氏,隔离关押,不让她和胭脂通气,然后立即开庭审讯。问王氏:“谁是杀人犯?”王氏回答:“不知道。”吴太守骗她说:“胭脂已经招供了杀人的事你完全了解,怎么能隐瞒得了?”王氏大喊:“冤枉啊!那臭婊子自己想找男人,我虽说要给她做媒人,但纯粹是开玩笑。她自己勾引奸夫到家里,我怎么知道呢?”吴太守慢慢地追问,王氏才讲出了原来与胭脂开玩笑的话。吴太守传胭脂上堂怒斥道:“你说她不知情,现在为什么她自己供认做媒人的事?”胭脂流泪说:“我自己不成器,害得父亲惨死。官司又不知哪年才能了结,再连累别人,实在不忍心。”吴太守又问王氏:“开玩笑后,你曾跟谁说过?”王氏供称:“没有。”吴太守发怒说:“夫妻同床而眠,该是无话不说,怎能说没有?”王氏连忙解释:“丈夫外出,好久没有回来了。”太守说:“即使是这样,凡捉弄别人的人,都以取笑别人的愚蠢来炫耀自己的聪明,你说没对一个人讲,骗得了谁?”随即命令左右夹她的十个指头。王氏不得已,如实招供:“曾对宿介说过。”于是吴太守释放了鄂秋隼,逮捕了宿介。宿介被传到堂,供说;“不知道。”太守说:“偷女人的一定不是好男子!”加以严刑拷打。宿介被迫招供说:“我曾冒充鄂生骗过胭脂是真,但丢了鞋子后,就没敢再去,杀人的事,实在不知道。”太守发怒说:“爬墙偷女人的人,什么坏事干不出来!”又加重刑罚折磨,宿介实在受不住了,就屈招是自已杀的。供词上报以后,无不称赞吴太守断案如神。这样,铁案如山,宿介只等着秋天被杀头了。

但是,宿介虽说生性放荡,品行不端,毕竟是山东有名的才子。他听说山东学使施愚山最有贤德才能,而且爱惜人才,就写了一张状子来申诉冤情,言词十分凄惨悲伤。于是,施学使调阅宿介的供词,反复分析研究,拍着桌子说:“这书生冤枉了。”接着请示上司,要求将案件交他来重新审理。施学使问宿介:“你的鞋丢在什么地方?”回答说: “我已记不清楚了。只记得去王氏家敲门时,还在袖中。”又转问王氏:“宿介之外,你的奸夫还有几个?”王氏供称:“没有了。”施学使喝道:“淫乱的人,怎能只与一人私通?”王氏解释说:“我与宿介年轻时就相好,因此,关系无法割断。后来并非没有勾引我的,但实在与他们没有来往。”施学使让她指出姓名来证实。王氏说;“只有同街的毛大,屡次勾引,都遭到我的拒绝。”施学使说:“你怎么忽然变得这样贞洁了?分明不老实。”喝令左右重刑伺候,王氏慌忙磕头,都磕出了血,并极力申辩确实没有了。施学使停止用刑,又问王氏:“你丈夫远出在外,难道就没有借故到你家来的吗?”回答说:“有的。某甲、某乙,都以借钱或送东西为名,曾来过我家一二次。”原来,某甲、某乙,都是村里有名的二流子,都曾打过王氏的主意,但没表现出来。施学使一一查考了他们的姓名,并将他们拘捕。等到拘齐了,就把他们押到城隍庙里,让他们跪在神案前,对他们说:“我梦见一个神仙告诉我,杀人犯就在你们四五个人之中。现在你们面对神灵,不能讲假话,如能坦白交代,还可从宽处理。说假话的,那就严惩不饶。”这伙人都齐声说没有杀人。施学使让把刑具摆在地上,准备用刑。刚把他们的头发束起来,脱光了衣服,他们就齐声大喊冤枉。施学使下令,暂免受刑,对他们说:“你们既然不肯自己招供,就让鬼神指明谁是凶手。”就派人用毡褥把大殿的窗子完全遮住,不留一点空隙;又让他们光着脊背,把他们赶进黑暗之中。开始给他们一盆水,让他们洗净手,然后用绳子把他们拴在墙壁下,警告说:“面对墙壁,不许乱动。是杀人凶手的,一定有神灵在他背上写字。”一会儿,把他们叫出来,施学使便挨个观察检验了一遍,最后指着毛大说:“这才是真正的杀人凶手!”原来,施学使先让人用白灰涂了墙壁,又用烟煤水让他们洗手,杀人凶手恐怕神灵在他背上写字,因此暗中将背紧贴墙壁,使脊背沾上了白灰;临走出暗殿时,又用手去护着背,因此脊背上沾上了黑烟色。施学使本来就怀疑是毛大,这样就更确实了。再对毛大动用重刑,他就全部如实交代了。最后,施学使判道:

“宿介:走了盆成括耍小聪明而招致杀身之祸的老路,得了个像登徒子那样好色的名声。就因为他与王氏两小无猜,竟然像夫妻一样同床而眠;又因王氏泄露了胭脂的心事,他竟占有了王氏还不满足,又打胭脂的主意。他学将仲子翻墙越园,就像飞鸟轻轻落地;他冒充鄂生来到闺房,竟然骗得胭脂开门;动手动脚,竟然不要一点脸皮;攀花折柳,伤风败俗,丢尽了读书人的品行。幸而听到胭脂病中的微弱的呻吟,还能顾惜;能够可怜姑娘憔悴的病体,还没有过份狂暴。从罗网里放出美丽的小鸟,还有点文人的味道;但脱去人家的绣鞋作为信物,岂不是无赖透顶!像蝴蝶飞过墙头,被人隔窗听到了私房话;如同莲花落瓣,绣鞋落地后,就无影无踪。假中之假因此而生,冤枉了鄂生之外,又冤枉了宿介有谁相信?天降大祸,酷刑之下差点丧命;自作自受,几乎要身首分离。翻墙越穴,本来就玷污了读书人的名声;而替人受罪,实在难消胸中的冤气。因此暂缓鞭打,以此抵消他先前受的折磨。姑且降为青衣,留一条自新之路。

像毛大这样的人,刁诈狡猾,游手好闲,是街坊里的流氓无赖,勾引邻家女人遭拒绝,还淫心不死;等着宿介进了王氏家中,鬼主意就顿时产生。推开王氏的家门,高兴地随着宿介的足迹进入院内,本想捉奸,却听到了胭脂的消息,妄想骗取美丽的姑娘。哪里想到魂魄都被鬼神勾去,本想进胭脂闺房,却误入卞老汉之门,致使情火熄灭,欲海起风波。卞老汉横刀在前,无所顾忌;毛大却走投无路,转而夺刀杀人。本来想冒充他人骗奸胭脂,谁知却夺刀丢鞋,自己逃脱却使宿介遭殃。风流场上生出这样一个恶鬼,温柔乡哪能有这样的害人精?必须立即砍掉他的脑袋,以快人心。

胭脂:还未定亲,已到成年,以嫦娥般的美貌,自然会配上容貌如玉的郎君。本来就是霓裳舞队里天仙中的一员,又何必担心金屋藏娇?然而她却有感到《关睢》的成双成对,而思念好的郎君;以至于春梦萦绕,感叹年华易逝,对鄂生一见倾心,结想成病。只因一线情思缠绕,招来群魔乱舞。为了贪恋姑娘的美貌,宿介、毛大都恐怕得不到胭脂,好像恶鸟纷飞,来冒充鄂秋隼。结果绣鞋脱去,差点难保住少女的清名,棍棒打来,几乎使鄂生丧了命。相思之情很苦,但相思入骨就会成为祸端;结果使父亲丧命于刀下,可爱的人竟成了祸水。能清正自守,幸好还能保持白玉无瑕;在狱中苦争,终于使案件真相大白。应该表扬她曾拒绝宿介入门;还是清洁的有情之人;应该成全她对鄂生的一片爱慕之情,这也是风流雅事。便让你们的县令,做你的媒人。”这个案子一结,远近都流传开了。

自从吴太守审讯以后,胭脂才知道自己冤枉了鄂生。在公堂下相遇时,满面羞愧,热泪盈眶,像有一肚子痛悔、爱恋的话而无法说出口。鄂生为她的爱恋之情所感动,爱慕之心也特别深。但又考虑到她出身贫贱,而且天天出入公堂,为千人指万人看,怕娶她被人耻笑。想来想去,拿不定主意。判词宣布后,才定下心来。县官为他送了聘礼,并派吹鼓乐队迎娶胭脂到了鄂家。


\subsection{1.10.16   阿 纤}
\label{\detokenize{p00_u5176_u5b83/_u767d_u8bdd_u804a_u658b_u5fd7_u5f02:id413}}
奚山,是山东高密县人,以行商为业,常常客居于蒙阴、沂水之间。

有一天,他在途中遇上了大雨,等他赶到他经常住宿的地方时,夜已经很深了。敲遍了旅店的门,没有开门的。他只好徘徊在一户人家的房檐下。忽然两扇门打开了,一个老头儿出来,请他进去。奚山很高兴地跟着他走进去。拴好了毛驴来到堂屋里,屋里并没有床榻几桌。老头儿说:“我是可怜客人你没有住处,所以才请你进来。我家其实并不是卖酒卖饭的人家。家中没有多余的人手,只有老妻弱女,已经睡熟了。虽然有点隔夜剩下的饭菜,苦于缺少炊具无法再热,请不要嫌弃,吃点冷饭吧。”说完了就进入里边。一会儿,拿了一张矮凳来,放在地上,催促客人坐下。又进去拿了一张短腿茶几出来。跑来跑去,忙忙碌碌,十分劳累。奚山一会儿站起,一会儿坐下,心里很不安,就拉住老头儿请他休息。过了一会儿,一位女郎出来给他们斟酒。老头说:“我家阿纤起来了。”奚山一看这姑娘,有十六七岁,身材苗条,容颜秀丽,举止风度优美动人。奚山有一个小弟弟还未结婚,心里暗暗看中了这位姑娘,因而就请问老头的籍贯和门第。老头儿回答说:“我姓古,名叫士虚。儿子、孙子都早死了,只剩下这个女儿。刚才不忍心打搅她的酣睡,想必是老伴儿把她叫起来的。”奚山问:“女婿是谁家?”老头儿回答说:“还没有许配人家。”奚山心里暗暗高兴。接着各种菜肴摆上了许多,好像早就有准备似的。奚山吃完了以后,恭恭敬敬地表示道谢,说道:“我这萍水相逢之人,受到你热情的接待,终生不敢忘记。因为老先生是盛德之人,我才敢冒昧地提一件事。我有一个弟弟叫三郎,十七岁了,正在读书学习,还不算愚笨顽劣,我想要高攀老先生结一门亲事,您不会嫌我家穷贱吧!”老头儿高兴地说:“老夫住在这里,也是寄居。倘若能得到你们这样的人家相依托,便请借给我一间屋子,我们全家都搬去,以免悬念。”奚山都答应了,就站起来表示感谢。老头儿很殷勤地安排他住下,才出去。鸡叫以后,老头已经出来了,请奚山去漱洗。奚山收拾完行装,拿出饭钱给他,老头儿坚决推辞说:“留客人吃一顿饭,万万没有收钱的道理。何况我们还依附你结为亲家了呢。”

分别以后,奚山在外客居行商一个多月,才返回来。离这个村子一里多路,遇见一位老太太领着一位姑娘,衣帽都是白色的。走近以后看了看,觉着那姑娘好像阿纤,姑娘也一再转过脸来看他,并拉着老太太的衣袖附在老太太耳边说了些什么。老太太便停下脚步问奚山说:“先生姓奚吗?”奚山连声说是。老太太神色凄惨地说: “老头子不幸被倒坍的墙压死了,现在我们要去上坟,家里空了没有人。请你在路边稍等一会儿,我们马上就回来。”于是进入树林里去了。过了一段时间才回来。这时,路上已经昏暗了,于是就和奚山一块儿走。老太太诉说自己和女儿的孤苦,不知不觉伤心啼哭,奚山也心酸难受。老太太说:“这个地方的人情很不善良,我们孤儿寡妇很难过口子。阿纤既已经是你家的媳妇,错过了这个机会恐怕就要推迟许多日子,不如今天晚上,就同你一起回去吧。”奚山也同意了。

回到了家以后,老太太点上灯伺候客人吃完了饭,对奚山说:“我们估计你快回来了,所以把家里存的粮食都已经卖出去了;还有二十多石,因为路远还没有送去。往北去四五里路,村中第一个门,有一个叫谈二泉的,是我们的买主。你不要怕辛苦,先用您的驴运一袋去,敲开门后告诉他,只说南村古姥姥有几石粮食,想卖了当作路费,麻烦他赶着牲口来运去。”就把一口袋粮食交给奚山。奚山赶着驴到了那儿,敲了敲门,一个大腹便便的男人出来了。奚山把事情对他说明了,放下粮食先回来了。一会儿有两个仆人赶着五头骡子来了。老太太领着奚山到藏粮食的地方,原来是在地窖中。奚山下去给他们用斗装粮食,老太太在上面发放,阿纤验收签码。顷刻装足了,打发他们走了。共计来回四次才把粮食装运完,接着就把钱交给老太太。老太太留下他们一个人和两头骡子,收拾行装就起身东去。

走了二十里,天才亮。到了一个集镇,在市场边上租赁了牲口,谈家的仆人才回去。

回到家里以后,奚山把经过情由告诉了父母。双方相见都很高兴。奚家就收拾了另一所房子,让老太太住了,占卜选择了好日子替三郎完了婚。老太太给女儿置办的嫁妆很齐全。

阿纤寡言少语,性情温和,有人和她说话,她也只是微笑,白天晚上纺线织布,一停不停。因此,全家上下都爱惜喜欢她。阿纤嘱咐三郎说:“你对大哥说,再从西边经过的时候,不要向外人提起我们母女。”过了三四年,奚家越发富裕了,三郎也入了县学。有一天,奚山投宿到古家原先的邻居家中,偶尔谈到往日有一次没有地方住宿,投宿到隔壁老头老太太家的事。主人说:“客人你记错了。我的东邻是我伯父家的别墅,三年前,住在这里的人经常见到怪异的事,所以空废了很久了,哪会有什么老头老太太留你住宿?”奚山很感到惊讶,但没有再往深处说。主人说:“这座宅子一向空着,有十年了,没有人敢进去住。有一天后墙倒坍了,我大伯去察看,看见石块底下压着一头大老鼠,有猫儿那么大,尾巴还在外边摇摆。大伯急忙回来,招呼了不少人一块去,老鼠已经不见了。大伙怀疑那东西是个妖物。十几天以后,又进去试探,很安静,什么东西也没有了。又过了一年多,才有人居住。”奚山越发感到奇怪。回到家中私下里和家里人谈论,都怀疑新媳妇不是人,暗暗地为三郎担心,而三郎和阿纤恩爱如常。时间久了,家中人纷纷议论猜测这件事,阿纤多少有些觉察了。半夜里对三郎说:“我嫁给你好几年了,从没有失做媳妇的品德的行为,现在却把我不当人看。请赐给我一份离婚书,任郎君自己去选一个好媳妇。”说着眼泪就流下来了。三郎说:“我的心意你应该早就了解。自从你进入我家门,我家日益富裕,都认为这福气应归功于你,怎么会有别的坏话?”阿纤说:“郎君没有二心,我难道不知道?但是众人纷纷议论,恐怕难免有抛弃我的时候,就像秋天抛弃扇子那样。”三郎再三安慰解释,阿纤才不再提离婚的事。

奚山心里始终放不下这件事,就天天寻求善于捕鼠的猫,以观察阿纤的态度。阿纤虽然不怕,然而总是愁眉不展。一天晚上她对三郎说母亲有点病,辞别三郎去探望母亲。天明后,三郎过去问候,只见屋子里已经空了。三郎吓坏了,派人四方寻访她们的踪迹,都没有消息。三郎心中萦绕着思念之情,吃不下饭也睡不着觉。而三郎的父亲和哥哥却都感到庆幸,轮流不断地安慰劝说他,打算给他续婚,而三郎的心情非常郁闷不欢。等待了有一年多,音信都断绝了,父亲和哥哥时常讥笑责备他。三郎不得已花重金买了一个妾,然而思念阿纤的心情始终不减。又过了好几年,奚家的日子一天天贫困了,因此又都思念起阿纤来。

三郎有一个叔伯堂弟阿岚,因为有事到胶州去,途中拐了个弯去看望表亲陆生,并住在了他家。晚上阿岚听见邻居家有人哭得很哀痛,未来得及询问这件事。到胶州办完了事回到陆生家,又听到了哭声,因而就询问主人。主人回答说:“数年以前有寡母孤女二人,赁屋居住在这儿。上个月老太太死了,姑娘独自居住,没有一个亲人,所以这样悲伤。”阿岚问:“她姓什么?”主人说:“姓古。她家经常关门闭户不跟邻里往来,所以不了解她的家世。”阿岚吃惊地说:“是我嫂子啊!”于是就去敲门。有人一边哭一边出来,隔着门答应说:“你是谁呀?我家从来没有男人。”阿岚从门缝里窥视,远远仔细一看,果然是嫂嫂,便说:“嫂嫂开门,我是你叔叔家的阿岚。”阿纤听了,就拨开门栓让他进去,对阿岚诉说孤苦之情,心情凄惨悲伤。阿岚说:“我三哥思念你很痛苦,夫妻之间即使有点不和,何致于远远地逃避到这儿来!”阿岚就要赁一辆车载她一起回去。阿纤面色凄苦地说:“我因为人家不把我当人看待,才跟母亲一块隐居到这里。现在又自己回去依靠别人,谁不用白眼看我?如果想要我再回去,必须与大哥分开过日子,不然的话,我就吃毒药寻死算了!”

阿岚回去之后,把这件事告诉了三郎,三郎连夜跑了去。夫妻相见,都伤心流泪。第二天,告诉了房子的主人。房主谢监生见阿纤长得美貌,早已暗中打算把阿纤纳为妾,所以好几年不收她家的房租,而且多次放风向阿纤的母亲暗示,老太太都拒绝了他。老太太一死,谢监生私下庆幸可以谋取到手了,而三郎忽然来了。于是就把几年的房租一起计算,借以刁难他们。三郎家本来就不富裕,听说要这么多银子,显出很忧愁的神色。阿纤说:“这不要紧。”领着三郎去看粮仓,大约有三十石粮食,偿还租金绰绰有余。三郎高兴了,就去告诉谢监生。谢监生不要粮食,故意只要银子。阿纤叹息说:“这都是因为我引起的麻烦啊!”于是就把谢监生图谋纳她为妾的事告诉了三郎。三郎大怒,就要到县里去告他。陆生阻止了他。替他把粮食卖给了乡邻,收起钱来还给了谢监生,并用车把两人送回家去。三郎如实地把情况告诉了父母,和哥哥分了家过日子。

阿纤拿出她自已的钱,连日建造仓房,而家中连一石粮食还没有,大家都感到奇怪。过了一年多再去看,只见仓中粮食已装满了。过了没有几年,三郎家中十分富有了,而奚山家却很贫苦。阿纤把公婆接过来供养,经常拿银子和粮食周济大哥,逐渐习以为常了。三郎高兴地说:“你真可谓是不念旧恶啊。”阿纤说:“他也是出于爱护弟弟啊,而且如果不是他,我哪有机会结识三郎呢?”以后也没有什么怪异的事情出现。


\subsection{1.10.17   瑞 云}
\label{\detokenize{p00_u5176_u5b83/_u767d_u8bdd_u804a_u658b_u5fd7_u5f02:id414}}
瑞云,是杭州的名妓,容貌才艺举世无双。十四岁时,妓院的蔡妈妈要让她接客,瑞云说:“这是我一生的开端,不能草率。价钱由你定,客人由我自已选择。”蔡妈妈说:“可以。”就定身价为十两银子。从这天起瑞云开始接客,求见的客人必须有见面礼。礼厚的,瑞云就陪他下盘棋,酬谢一幅画;礼少的只留喝杯茶就打发走了。瑞云的名字早已远近闻名,从此,登门求见的富商及贵家子弟,天天不断。

余杭县有个贺生,是个很有名气的才子,只是家中不太富裕。他一直仰慕瑞云,虽然不敢打算和瑞云同床共枕,也竭力准备了一点礼物,希望能看到瑞云的芳容;但又暗自担心瑞云交往的人多,不会把他这个穷书生放在眼里。等到相见时一交谈,瑞云却招待得十分殷勤,坐在一起谈了很久。瑞云眉目含情,作了首诗赠贺生:“何事求浆者,蓝桥叩晓关?有心寻玉杵,端只在人间。”贺生得到这首诗喜欢极了,有许多话正想说,忽然小丫鬟进来说:“来客人了。”贺生只好匆匆告别。

回家以后,贺生反复品味赠诗,睡梦里也思念着瑞云。这样过了一两天,情不自禁,又带了礼物去见瑞云。瑞云见到他很高兴,把坐位移到贺生跟前,小声对他说: “你能想法和我欢聚一夜吗?”贺生说:“穷书生,只有一片痴情可献知己,就这一点薄礼,已经竭尽了微薄的力量。能够见到你的芳容,我就心满意足了,至于肌肤之亲,我是想也不敢想的。”瑞云听了闷闷不乐,两人对面坐着谁也不说话。贺生坐了很长时间没出来,蔡妈妈三番五次叫瑞云,以催促他走,贺生只得走了。他心里非常愁闷,想卖掉所有家产,换得一夜之欢,但是天亮还得分别,那时的情景更难以忍受。想到这些,贺生心灰意冷,从此,就和瑞云断了音信。

瑞云选择初夜女婿,选了好几个,没有一个合适的。蔡妈妈很生气,要强迫她改变原来的打算,但还没说出来。一天,有一个秀才带着赠礼来,坐着说了一会话,便起来用手指头按了一下瑞云的额头说:“可惜!可惜!”就走了。瑞云送客回来,大伙见她额头上有一个像墨一样的指印。瑞云洗了洗,越洗越清楚。不几天,墨痕渐渐扩大;过了一年多,已漫延到左右颧骨及上下鼻梁。见到她的人无不嗤笑,从此再没有来访她的客人了。蔡妈妈夺了瑞云的妆饰,叫她和婢女们一块干活。瑞云身体很弱,干不了重活,一天比一天憔悴起来。贺生听说后,来看望她,只见瑞云蓬头垢面,正在厨房里干活,丑得像鬼一样。瑞云抬头看见贺生,忙面向墙壁,遮掩面容。贺生怜惜她,便同蔡妈妈说,愿意赎瑞云作妻子。蔡妈妈同意了。贺生变卖田地,拿出所有家产,把瑞云买了回来。进了家门,瑞云拉着贺生的衣服哭泣着,说不敢做他的夫人,愿意给他当侍妾,等待贺生另娶正妻。贺生说:“人生所注重的是知己。你走运的时候能拿我当知己,我怎能因为你失意了就忘了你呢?”终于没有另娶。听到这件事的人都讥笑贺生,但贺生对瑞云的感情却更加深厚了。

过了一年多,贺生偶然到苏州,有一个姓和的书生与他同住在一家客店。和生忽然问他:“杭州有个叫瑞云的名妓,近来怎样了?”贺生回答说嫁人了。和生又问: “嫁给谁了?”贺生说:“那人和我差不多。”和生说:“如果能像你,可算嫁了合适的人了,不知身价是多少?”贺生说:“因为她得了种奇怪的病,所以贱卖了。不然,那个像我这样的穷书生,怎么能从妓院中买到那样一个漂亮的女子呢?”和生又问:“那人果真和你一样吗?”贺生觉得他问得特别,就反问他为什么这样说。和生笑着说:“实不相瞒,我曾见到她的芳容,很可惜她绝世佳人,流落在那种地方,就使了点小法术,遮掩了她的光彩,以保护她的纯真,留着等待真正爱怜她的人去赏识她。”贺生急忙问他说:“你既然能点上墨痕,能不能再洗掉它呢?”和生笑着说:“怎么不能!但须要那个人诚心诚意来请求!”贺生起身施礼说:“瑞云的丈夫就是我呀。”和生高兴地说:“天下只有真正的才子才能懂得真情,不因为丑陋而改变心意。我跟你回去,送还你一个美人。”于是就同贺生一块往回走来。

到了家里,贺生要准备酒宴,和生止住他说:“先让我施行法术,好让准备酒菜的人高兴!”就让贺生用盆盛上水,和生用手指在水上画了几下,说:“洗洗马上就好了。可是必须让她亲自出来谢医生。”贺生笑着捧了盆进去,等瑞云自己洗脸。瑞云一洗,脸上的墨痕果然随手而落,光洁艳丽,就同当年一样。夫妇两人非常感激,一块出来拜谢,但是客人已经不见了。找遍了也没找到,心想大概是个仙人吧!


\subsection{1.10.18   仇 大 娘}
\label{\detokenize{p00_u5176_u5b83/_u767d_u8bdd_u804a_u658b_u5fd7_u5f02:id415}}
仇仲,是山西人,忘记了他是哪个郡哪个县的了。有一年,正赶上兵荒马乱,他被强寇俘掳了去。家中两个儿子仇福、仇禄都还年小,他续娶的妻子邵氏抚养着两个孤儿,艰难度日。所幸他留下的一点家业,还能使母子三人维持温饱。但那时的年景,天灾人祸不断,收成又不好,加上村里的豪门大户,仗势欺人,使得孤儿寡母衣食不保,苦苦煎熬。

仇仲有个叔叔叫仇尚廉,企图吞并仇仲的那点家产,多次劝邵氏改嫁,邵氏坚决不肯。仇尚廉便将她暗地里卖给了一个大户人家,想强行赶走她。仇尚廉跟大户人家讲妥后,邵氏还蒙在鼓里,别的人也都不知道这个阴谋。同村有个叫魏名的,为人奸滑狡诈,跟仇家多年有仇,事事都想造谣中伤。因为邵氏在家守寡,魏名便到处散布谣言,败坏邵氏名声,以此来污辱诋毁仇家。这些谣言正好被那个大户人家听到了,厌恶邵氏不贞洁,便告诉仇尚廉,不愿再买邵氏。时问一长,仇尚廉的阴谋和外面的流言蜚语,都传到了邵氏耳朵里,邵氏冤愤不已,天天哭泣,渐渐地四肢不适,一病不起了。当时,仇福才十六岁。家里无人缝补衣裳,便匆匆忙忙地为仇福娶了媳妇。新媳妇姓姜,是秀才姜屺瞻的女儿,为人贤惠能干。从此后,一切家务事都依靠姜氏料理,家境竟也渐渐好过起来,便又让仇禄拜了先生,开始读书。

魏名见仇家日子好起来,非常忌恨,一计不成,另施一计。假装和仇福套近乎,常常叫了他去喝酒。仇福受骗,把魏名看作是心腹之交。魏名乘机挑拨他说:“你母亲卧床不起,已成了废人,不能再料理家业;你弟弟又坐吃闲饭,什么事都不干。就你们这对贤惠的夫妇,整天给人作牛作马!况且日后为你弟弟娶媳妇,必定花费不少。我为你着想:不如早点分家,那么贫困的是你弟弟,而富裕的是你啊!”仇福回家,便和妻子商量跟弟弟分家,被姜氏斥骂了一顿。无奈魏名天天引诱离间仇福,仇福完全上了圈套,径直去告诉母亲,要分家另过。邵氏大怒,又痛骂了他一场。仇福更加忿怒,从此便把家里的银两和粮食都看作是别人的东西,尽情挥霍。魏名又乘机引他赌博,渐渐把家里的粮囤都快输空了。姜氏知道后,没敢和婆母说。不久,家里忽然断了粮,邵氏吃惊地询问,才得知仇福赌博的事,虽然极为愤怒,但又无可奈何,只得分了家,让仇福另过。所幸姜氏很贤惠,天天给婆母做饭吃,仍像以前一样侍奉。

仇福分家后,更加没了顾忌,大肆赌博。只几个月的时间,便将全部田产输了个净光。母亲和妻子还都不知道。仇福没了本钱,无法再赌,竟想拿妻子作抵押,借债再赌,但一直没找到个愿意借债的。本县有个赵阎王,本是漏网的大盗,横行一方,无人敢惹,是当地一霸。所以他不怕仇福会食言,慷慨地借给他钱。仇福拿到钱,仅仅几天,又输光了。心中犹豫,想跟赵阎王反悔。赵阎王发怒起来,仇福害怕,只得将妻子骗到了赵家,把她交给了赵阎王。魏名昕说后,非常高兴,忙跑去告诉了姜家,巴不得姜、仇两家为此打个不亦乐乎。姜家听到消息,果然大怒,立即打起官司。仇福十分恐惧,连忙远远地逃走了。

姜氏被丈夫骗到赵阎王家后,才知道自己被丈夫卖了。真是万箭钻心,只想寻死。赵阎王起初还好言安慰她,姜氏不听。赵阎王又威逼她,姜氏索性破口大骂。赵阎王大怒,用鞭子毒打姜氏,还是不服。乘人不备,姜氏拔下头上的簪子,直向自己的咽喉刺去。众人急忙将她救下时,簪子已穿透喉管,鲜血涌出。赵阎王忙用布帛包住她的脖颈,还盼望着以后再慢慢地说服她,让她顺从自己。

第二天,官府的拘牒便到了,要捉赵阎王去会审。赵阎王毫不在乎,大大咧咧地赶到县衙。县官查验到姜氏脖子上有重伤,便命衙役拉下赵阎王去痛打。衙役却面面相觑,不敢动手。县官早就听说赵阎王横行残暴,这时更加相信了,不禁怒火中烧,将衙役喝退,命家仆们一涌齐上,将赵阎王即刻打死了。姜家才将女儿抬回家中。自姜家打起官司后,邵氏才知道仇福犯下的种种罪恶,痛哭一场,昏厥过去,渐渐露出要下世的景象。仇禄这年才十五岁,孤孤单单的,失去了依靠。

先前,仇仲的前妻生了个女儿,叫大娘,嫁到了远郡。性情刚猛。每次回娘家探亲,只要父母送给的东西太少,她不满意,就使性子顶撞父母。仇仲因此很生气厌恶这个女儿;又因为她嫁得远,所以常常几年不来往。邵氏病得快死的时候,魏名便不安好心地想叫了她来,以挑起仇家更大的家务纠纷。正好有个小商贩,跟仇大娘是同村的,魏名便托他捎话给大娘,说她继母快要死了,而且暗示大娘娘家有利可图。过了几天,大娘果然带了一个小儿子来了。进入家门,见只有二弟侍奉着病在床上的继母,那情景很是惨淡,大娘不觉悲伤起来,便问大弟仇福哪去了。仇禄便把家里的变故一五一十地告诉了她。大娘听说后,气得一句话也说不出,过了会儿,才说:“家里没个成年男子掌家,就任人欺凌到这种程度!我们家的田产,那些贼徒怎敢骗赚了去!”说完,走进厨房,烧火做饭,先让母亲吃了,才招呼二弟、小儿子一块吃。吃完,忿忿地出了家门,径直到县衙去投了诉状,告那些赌徒们引诱仇福赌博,把家产都骗了去。赌徒们听说,都害怕起来,一块凑了银子,贿赂大娘撒诉。大娘将银子收下,照样打官司。县官便将几个赌徒捉到县衙,分别打了顿板子了事,田产一事竟不过问。大娘愤愤不平,又带着儿子告到郡里。郡守最痛恨赌博,加上大娘极力诉说孤儿病母的痛苦艰难,以及那些赌徒骗赚田产的种种情形;讲得慷慨激昂,声泪俱下。郡守也被打动了,便判令县官将田产追还仇家,仍将仇福从重惩罚,以警戒那些不肖之子。大娘回家后,县官已奉郡守令,重新拘拿赌徒,严加追究,终于又把仇福输掉的田产全部夺了回来。

大娘这时已守寡很久了,便让小儿子回去,而且嘱咐他回家后跟着哥哥好好干活,不要再来了。大娘从此后便住在娘家,奉养继母,教诲二弟,里里外外,料理得井井有条。继母大为欣慰,病情也逐渐好转,家务大事全委托给大娘掌管。村里那些地痞无赖,有时稍微欺负到仇家头上,大娘就持刀找上门去,理直气壮地讲理,那些地痞无赖没有不屈服的。过了一年多,家产便一天天多起来。大娘还时常买些药品和食物给姜氏送去。又见仇禄渐渐长大,便频频嘱托媒人给他提亲。魏名枉费心机,仍不罢休,又跟人说:“仇家产业,全都归了大娘了。恐怕将来要不回来了。”人们都相信了,所以没人肯把女儿嫁给仇禄。

有个叫范子文的公子,家里有座有名的花园,是山西首屈一指的。花园里,众多的名贵花草,种满了路两边,一直通到范家内室。曾有个人不知这是范家的花园,误顺路一直走到内室,正好碰上范公子开家宴。范家便愤怒地将这个人抓起来,说他是强盗,差点把他打死。清明节那天,仇禄从私塾里回来,正碰上魏名。魏名假装和他玩耍,渐渐把他引到范家花园附近。魏名本来跟花园的园丁有交情,所以园丁将他们放了进去。二人把园里的楼台亭榭逛了个遍。一会儿来到一个地方,一条小溪,远远流去,水势汹涌。溪上横跨着一座画桥,两边有朱红栏杆,通向一个红漆大门。远远望见大门内花团锦簇,原来这就是范公子的内室。魏名欺骗仇禄说:“你先进去吧,我要去上厕所。”仇禄信以为真,从桥上过去,进入红漆大门,来到一个院子,听见有女子的说笑声。正停步惊疑间,一个丫鬟出来,看见仇禄,转身便跑。仇禄才恍然大悟:自己误入了人家的内室,惊骇地拔脚就逃。刹时,范公子也出屋来,喊叫家人拿着绳索追赶仇禄。仇禄大为窘迫,一急之下,自己跳进了溪中。范公子见了,忽然破怒为笑,命仆人们把他救上来。见仇禄容貌衣著俊雅华丽,便叫仆人替他换下湿了的衣服、鞋子,拉他走进一个亭子,询问他的姓名。看范公子的神态,脸色和蔼,话语温和,样子很亲近。谈了一会儿,范公子走进内室,接着又出来,笑着握住仇禄的手,拉他走过桥去,渐渐走近刚才的院子。仇禄不解其意,犹豫着不敢进去。范公子强拉着他进了院子,见花蓠笆内隐约有个漂亮女子往这边窥视。二人坐下后,丫鬟们摆上酒来。仇禄推辞说:“我年幼无知,误进了你家内室,承蒙你原谅了我,已出我所望。只愿你早点放我回家,我将感恩不浅!”范公子不听。不长时间,菜肴已摆满了桌子。仇禄又推辞说已经酒足饭饱了。范公子强按他坐下,笑着说:“我有一个乐拍名,你若能对上,我就放你走!”仇禄连忙答应,请他说。范公子说道:“拍名‘浑不似’,”仇禄默默想了很久,才对上,回答说:“银成 ‘没奈何’。”范公子大笑着说:“真是石崇来了!”仇禄听了,更加迷惑不解。

原来,范公子有个女儿叫蕙娘,既美丽又懂诗书。范公子天天想为她选个好丈夫。头天夜里,蕙娘梦见一个人告诉自己说:“石崇,是你女婿!”蕙娘问:“在哪里?”回答说:“明天就要落水了。”早上起来,蕙娘告诉父母,都感到奇异。仇禄正好符合了蕙娘的梦兆,所以范公子才将他请进内室,让夫人、蕙娘和丫鬟们相看相看。此时,范公子听了仇禄这样巧合的联对,喜欢地说:“这拍名是我女儿拟的,想了很久也没想出对句。现在你对上了,这也是天定缘分。我想把女儿嫁给你,我家里不缺房子,不用麻烦你家来迎亲了,你就入赘到我家来吧!”仇禄惶恐地谢绝,说母亲正生病卧床,自己实在不敢入赘到别家。范公子便让他先回去,跟家里商量一下。于是派仆人拿着仇禄的湿衣服,让他骑马回去。

仇禄回到家中,把这事告诉了母亲。母亲很惊讶,认为这事不吉祥。邵氏从这件事上才看出魏名此人十分险恶。但因祸得福,也就不想跟他为仇,只是告诫儿子不要再和他来往。过了几天,范公子又让人传话给仇禄母亲。母亲还是不敢答应,大娘却作主应下了,随即就派两个媒人送去了彩礼。不久,仇禄便入赘到了范公子家。一年多他就考中了秀才,很有才名。后来,妻弟长大后,对仇禄很怠慢。仇禄一怒之下,带着妻子返回了自己家。此时,母亲已能扶着拐杖走路。年年依赖大娘料理,家里的房子倒也很宽敞完好。仇禄的妻子搬来后,奴婢仆人也带来了不少,仇家于是俨然成了高门大户了。

魏名又没有得逞,更加嫉妒仇家。只恨抓不到仇家的把柄,便收买了一个从旗下逃出来的汉奴,让他诬告仇禄代为窝赃。大清刚立国的时候,惩治旗籍逃奴的法律最为严苛。仇禄于是依律被判流刑,发配到关外。范公子上下贿赂活动,仅仅保住了蕙娘不被流放,凡仇禄的田产全部投入官库。幸亏大娘拿着原来的分家文书,挺身而出,跟官府申辩:新增的若干顷良田,都挂在仇福名上,不属仇禄的田产,才没被没收,母女二人得以有个地方居住。仇禄自料这次被发配可能永远回不来了,便写下离婚文书,送给岳父家,自己孤单一人去了关外。

仇禄走了不几天,来到都北,在一个客店里吃饭。偶然看见一个乞丐在窗外正愣愣地盯着自己,模样极像是哥哥仇福。仇禄忙上前询问,果然是仇福。仇禄便述说了自己的遭遇,兄弟二人十分凄恻悲伤。仇禄解开内衣,拿出几两银子,交给哥哥,嘱咐他回家去。仇福哭泣着接受下,二人便分别了。

仇禄到了关外,被安排在一个将军的帐下做奴仆。因为他生得文弱,将军便让他掌管文书籍簿,和其他奴仆们一块吃住。奴仆们询问他的家世,仇禄详细讲了。其中一人忽然惊讶地说;“你是我的儿子!”

原米,仇仲被强寇掳去后,最初是给他们牧马。后来这股强寇向官军投降,就又把仇仲卖给了旗人为奴,这时他正跟着主人屯扎在关外。仇仲向仇渌回忆了往事,大家才知道二人真是父子。仇仲、仇禄不禁抱头痛哭,一屋的人也为之心酸落泪。既而,仇仲又愤怒地说:“哪里来的这个逃奴,诬告诈骗我的儿子!”便去哭着跟将军诉说了经过。将军听说后,就让仇禄做了书记官,又给朝廷中一个亲王写了封信,让仇仲拿着去京城上告。

仇仲进入京城,等候亲王的车驾出来,便大喊冤枉,并递上将军的信。亲王得知事情经过,很是为仇禄叹惜。便责令地方官为他申冤昭雪,将没入官库的家产归还仇家,并判仇禄无罪,释放回家。

仇仲返回关外,父子二人都很喜欢。仇禄又细问父亲这些年有没有再成家,以便替父赎身返回老家。得知仇仲后来结过两次婚,但都没孩子,这时仍是孤身一人。仇禄便治办下行装,自己先返回家乡去了。

起初,仇福告别弟弟返回老家,进入家门,跪着叩见母亲。大娘侍奉着母亲高坐在堂屋里,自己操起根棍子站在一边,问仇福,“你如愿意挨打受罚,可以先留在家里;否则,你的家产早已没了,这里也没你吃饭的地方,你请走人!”仇福跪在地上哭着说愿意受罚。大娘听了,把棍子扔到地上,说:“卖老婆的人,打都不值得打!但你犯下的旧案还没消,如果再犯,就到官府自首去吧!”便派人去告诉姜氏仇福回来的消息。姜氏大骂道:“我是仇某的什么人?用得着来告诉我!”大娘便将姜氏的话告诉仇福,故意羞辱他。仇福非常惭愧,大气不敢出。

过了半年,大娘虽然供给仇福吃喝穿戴,十分周到,但一直拿他当仆人对待。仇福也整天操劳,毫不抱怨。有时给他银子,让他去办事,仇福也变得一丝不苟,花多少,剩多少,一清二楚。大娘观察到他确实变了,便告诉母亲,去哀求姜氏回来。母亲觉得恐怕不好挽回。大娘说:“不会的。她当初如肯嫁别人,就不会自己受那样大的罪了!她实在是不能不气愤啊!”于是,大娘亲自领着弟弟,前去姜家负荆请罪。岳父母见了仇福,骂了又骂。大娘喝令他跪在岳父母面前谢罪,然后,才请姜氏出来见面。连请了三四次,姜氏躲了起来,坚决不出来。大娘搜寻到她,强将她拉到仇福面前,姜氏才指着仇福的鼻子大骂一通。仇福汗如雨下,无地自容。姜母才命拉他起来。大娘便乘机询问姜氏什么时候回去,姜氏说:“过去我受姐姐的恩惠太多了,现在你叫我回去,我怎敢说别的?但恐怕不能保证我不会再被卖掉!况且,我与他情义已绝,还有什么脸面与这个黑心无赖的豺狼一块生活?请姐姐另准备一间屋子,我回去侍奉母亲,稍胜过削发出家当尼姑,我就满足了。”大娘忙替仇福说明他已很悔恨,约定第二天来接她回去,便告别走了。

第二天,大娘准备了华丽的车子,将姜氏接回来。母亲已早早等在门口,见了姜氏,跪拜在地。姜氏也急忙跪在地上,放声大哭起来。大娘忙在一边劝解。又准备下酒宴,欢庆姜氏回来,命仇福坐在桌子一侧。过了会儿,大娘端起酒杯说:“过去我苦苦为仇家挣下这份家业,不是为了自己得到什么好处!现在,大弟已经悔过,贞妇也已回来,我马上将全家帐册如数交出。我空着身子来,仍然空着身子回去!”仇福夫妇听说,忙离席站起来,跪拜在一边哭着哀求她别走,大娘才作罢。

不长时间,官府为仇禄昭雪的命令下达。仅几天,原来没入官库的田产全都退了回来。魏名大惊,不知是什么缘故。恨得牙痒痒的,但又无计可施。正好碰上仇家的西邻遭了火灾,魏名假装救火,却暗地里用把草束点着火引燃了仇禄的房子。当时又刮大风,火势迅速蔓延,将仇家的房屋几乎烧了个净光,只剩下仇福住的两三间屋子。全家人只得都搬到这几间屋子去住。

不久,仇禄返回家来,一家人团聚,又悲又喜。起初,范公子收到仇禄的离婚文书,拿了去跟蕙娘商量。蕙娘痛哭着,将文书撕碎了扔到地上。父亲便顺从了女儿的意思,不勉强她改嫁。仇禄回来后,打听到蕙娘没有嫁人,喜出望外,急忙赶到岳父家。范公子知道他家遭了火灾,便想留住他,仇禄不肯,告辞回家。所幸大娘平日积攒下了些银子,这时便全都拿出来整修破房。仇福拿着锨干活时,意外挖出一个金窑。到了夜晚,便和弟弟一块打开,只见石砌的金窑足有一丈见方,里面放满了白银。得到这些银子后,仇家于是召集工匠,大兴土木,建了一片楼房,壮观华丽得不亚于富贵大家。

仇禄回来后,感激将军在危难中帮助,便备下一千两银子,要去拜见将军,顺便赎回父亲。仇福愿意代替弟弟前去,于是便派了几个健壮的仆人,跟随着他去了关外。仇禄又接回了蕙娘。不久,仇福便将父亲接了回来,全家一片欢腾。

大娘自从住在娘家,禁止儿子来看望自己,是恐怕有人议论她企图侵吞仇家家产。现在父亲已经回来,便坚决告辞,要回去。兄弟们不忍心,父亲便将家产分成三份:儿子得两份,女儿得一份。大娘苦苦推辞,兄弟二人都哭着说:“我们若不是姐姐,哪里有今天!”大娘只得安心收下,派人去叫儿子搬了家来,跟父母住在了一起。

后来,有人问大娘:“仇福、仇禄是你异母兄弟,你怎么如此关心?”大娘回答说:“只知有母亲,不知有父亲,只有禽兽才会这样!人哪能效仿呢?”仇福、仇禄听到这话后,都感激得热泪滚流。让工匠整修大娘的房屋,建得跟自己的一样。

此后,魏名自己反思:十几年里,越是祸害仇家,却越是给仇家招福,也不禁渐渐后悔起来。又仰慕仇家富裕,便想和他家交好。于是他便以庆贺仇仲回家为由,备下礼物到了仇家。仇福要赶走他,仇仲不忍心拂了人家的好意,便接受了他送来的活鸡和酒等礼物。鸡本是用布条绑着脚的,却跑进了厨房,被火烧着了布条;鸡又钻到柴禾堆里栖息,奴婢仆人们见了都没在意。一会儿,厨房的柴禾燃烧起来,引着了厨房。一家人惊慌失措,幸亏人手多,不一会儿就把火扑灭了,但厨房中所有的东西都已变成了灰烬。仇福兄弟二人都觉得魏名送来的东西不吉利。后来,又赶上父亲做寿,魏名又牵来一只羊作贺礼。仇家推辞不了,只得暂时将羊拴在院子中一棵树上。到了夜晚,家里有个童仆因为被别的仆人殴打了一顿,便忿忿地走到树下,解开拴羊的绳子,自己吊死了!仇福、仇禄兄弟感叹地说:“他好好地对待我们家,倒不如坑害咱们家呢!”从此后,魏名虽然很殷勤,但仇家兄弟再也不敢接受他一丝一缕的东西了,宁恳反过去厚厚地酬谢他。后来,魏名老了后,家里非常贫困,只好去作乞丐,仇家仍时常拿些布匹、粮食去周济他。


\subsection{1.10.19   曹 操 冢}
\label{\detokenize{p00_u5176_u5b83/_u767d_u8bdd_u804a_u658b_u5fd7_u5f02:id416}}
许昌城外有一条河,水流湍急,波涛汹涌。临近一处崖岸的地方,河水的颜色变成深黑色。盛夏天,有人从这里跳进河中洗澡,忽然像被刀斧斩过一样,尸体断为两截,浮出水面。后来又有一人也如此这般。人们深感惊奇。县令听说这件事后,派人截断河的上流,排尽余水,见崖岸下有一个深洞,洞中安装着一个转轮,轮上排列着锋利的刀刃。拆掉转轮,深入洞中,发现一块小石碑,碑上的字都是汉篆字,细细辨认,原来是曹操墓。于是打破棺材,散掉腐骨,将殉葬的金银财宝全部取了出来。


\subsection{1.10.20   龙 飞 相 公}
\label{\detokenize{p00_u5176_u5b83/_u767d_u8bdd_u804a_u658b_u5fd7_u5f02:id417}}
安庆有一个姓戴的书生,年纪轻轻,却行为不检,品行不端。

有一天,戴生从外面喝酒回来,路上碰到已经死去的表兄季生。戴生喝醉了酒,两眼昏花,忘记表兄已经死了,问候道:“你一向在哪里做事?”季生答道:“我早已死了,难道你忘记了吗?”戴生恍然醒悟,知道碰上了鬼,但乘着酒意,也不害怕,又问道:“你在阴间干些什么?”季生说:“现在转轮王殿下掌管轮回生死簿。”“那么人世的祸福,你一定都知道了?”“那是我的职责,怎能不知!”季生道:“只是簿子中记录太烦琐,不是关系密切的人,我也记不清楚。前些天偶然翻检簿子,还看见你的大名。”戴生听说,急忙问上面都写了些什么。季生答道:“实在不敢瞒你,你的名字列在黑暗狱中!”戴生大吃一惊,连酒也吓醒了,苦苦哀求表兄拯救他。季生叹道:“这是我无能为力的事。人生在世,行善才有好报。你恶贯满盈,不积大善怎能挽回呢?但你一个穷书生,又没有力量去行大善;即使你从现在起每天都做善事,没有一年多的时间也抵消不了你的罪恶,所以现在太晚了!只希望你从此后洗心改过,努力行善,地狱之中或许还有出头之日。”戴生哭着拜伏在地上,哀恳表兄救他。一会儿抬头一看,季生已经无影无踪,只好闷闷不乐地返回了家。从此以后,戴生尽心改过,不敢再稍有差迟。

在此以前,戴生曾与邻居的妻子私通。邻居察觉后,隐忍下来,没有发作,指望有朝一日捉奸捉双。没想到戴生洗心革面,永远断绝了与他妻子的私情。邻居抓不到把柄,怀恨在心。一天,两人在田野里相遇,邻居假装要和戴生说话,引他望一眼枯井里看,却在背后将戴生推落下井。井深数丈,以为这下戴生必死无疑了。半夜,戴生苏醒过来,坐在井中大声呼救,没有一个人听见。第二天一早,邻居恐怕戴生再活过来,又到井边察看动静,正好听到了戴生的呼救声,他急忙往井里投掷石块。戴生藏身到井下的地洞里,大气不敢出。邻居知道他没死,于是便挖土填井,几乎将井都填满了。戴生蹲在洞中,漆黑一团,真是与地狱没什么两样。洞中没有食物,自料这下子是死定了。匍匐着往洞深处爬了爬,只见三步以外都是积水,没有可去的地方,只好回到原处坐下。起初还感到肚饿,时间一长,连饥饿也忘了。又想到人在地洞之中,没什么善事可做,只好高颂佛号而已。既而看见鬼火点点,在洞中游荡,便祷告道:“听说磷火都是冤鬼所化。我虽然暂时还活着,但难以再返回人世。如果我们能聚谈聚谈,也聊以解除寂寞。”祷告毕,便看见鬼火都从水面上漂浮过来,每点鬼火中都有一个人,身高只及活人的一半大小。戴生问他们的来历,鬼火们答道:“这是一座古煤井,煤井主人挖煤时,震动了边上的古墓,被墓里的龙飞相公决海水淹了煤井,溺死四十三人,我们都是这些淹死的冤鬼。”戴生惊奇地问:“龙飞相公是什么人?”回答说:“不知道。只知相公是文学士,现为城隍的幕宾。相公也怜悯我们无辜而死,所以每隔三五天便施舍水粥给我们充饥。只是我们被冷水浸骨,难再超度苦海;您若能再返人世,请您打捞我们的残骨造一座义坟,那您的恩惠就遍及九泉之下的人了。”戴生叹道:“假使有万分之一生还的希望,这事又有什么难的呢?但身在九泉之下,怎指望还能重见天日!”便叫众鬼念佛,将泥块捻成佛珠,以记下诵佛遍数。也不知天黑天明:疲倦了就睡,醒过来就坐着念佛。

不知过了多长时间,忽见洞深处有灯笼出现,众鬼喜欢地说:“龙飞相公来施舍食物了!”邀请戴生一同前去。戴生担心前面水深过不去,众鬼强拉着他前行,飘飘忽忽地像凌空行走。曲曲折折地走了约半里路,来到一处地方,众鬼才放开他让他自己走。越往上走越高,像在爬几丈高的台阶。戴生登上阶梯,看见一座房廊,大堂上点着一支明亮的蜡烛,像小孩胳膊一样粗。戴生很久没见灯光了,乍见之下,十分兴奋,急忙跑了过去。堂上坐着一个老翁,文人打扮。戴生一见,不敢再往前走。老翁看见他,惊讶地问:“我不认识你,从哪里来的?”戴生走上前去,跪在地上叙述了经过。老翁说:“原来是我的远代孙子!”让他起来,赐座坐下,自已说:“我叫戴潜,字龙飞。过去因为不肖孙子戴堂,勾结土匪,靠近我的墓打井,让我不得安宁,所以用地海水淹了他们。现在他的后代怎样了?”原来,戴氏近宗共有五支,戴堂居长。以前,本县有个大户贿赂戴堂,在他祖坟边探井采煤。戴氏子弟畏惧戴堂,不敢不从。挖了不长时间,地下水突然汹涌而出,将采煤的人全部淹死在井中。死者亲属,联合打官司,戴堂及那个大户因此破产贫困下来,戴堂的子孙以至于无立锥之地。戴生是戴堂弟弟的后代,曾听老人们说过这件事,便告诉了老翁。老翁说: “这种不肖子孙,他的后代怎会兴旺!你既然来到这里,还应别忘了读书。”于是,让戴生吃饱喝足后,拿一本书放到桌子上,都是八股文,让戴生研读。又命题考查他的文章,就和塾师教学生一样。大堂上的蜡烛,不用剪,也不灭,长久亮着。疲倦的时候就睡,也分不清哪是早晨哪是晚上。老翁有时外出,便派一个童仆供他使唤。戴生觉得像过了数年之久,所幸没受什么苦难。只是没别的书可读,惟作百篇八股文,每篇写了四千多遍。一天,老翁对他说:“你孽报已满,马上就要再回到人间。我的坟邻近煤洞,阴风刺骨,你得志以后,要把坟迁到村东地里去。”戴生恭敬地答应下。老翁便将群鬼都叫上来,让他们把戴生仍然送到原来的地方。回到原处,众鬼又再三行礼嘱咐,戴生也不知怎么才能出去。

戴生突然失踪以后,家里多方搜寻打听,一直没有踪影。他母亲便告了官,逮了许多人审讯,还是没有一点线索。过了三四年,原来的官离任,搜查也就松了下来。戴生的妻子也改嫁走了。正好村中有人重新整治原来的煤井,进入洞中,发现了戴生,摸摸竟还没死,惊骇万分,连忙告诉他家,抬了回去。一天后,戴生才会说话,详细述说了经过。

自从戴生被推落井以后,那个邻居又打死了他老婆,被他岳父告了,逮到狱中,一年多才出来,瘦得只剩皮骨了。听说戴生又活了过来,他十分害怕,连忙逃走了。戴生的族人商量拿住那邻居治罪,戴生不许,说过去是自找的,是阴间的处罚,与邻居没有关系。邻居察觉到他没有恶意,又犹豫着回来了。

煤井里的水干了以后,戴生便出钱雇人进洞捡拾群鬼的遗骨,买了口棺材,找个地方一块葬了。又稽查宗谱上果有一个戴潜,字叫龙飞,便备下祭品,到祖坟上祭祀了一番。学使听说了戴生的奇异遭遇后,又欣赏他写的文章,让他以优等参加了乡试,中了举人。戴生便在村东地里造坟,将龙飞的墓迁来厚葬。此后春秋上坟,年年不断。


\subsection{1.10.21   珊 瑚}
\label{\detokenize{p00_u5176_u5b83/_u767d_u8bdd_u804a_u658b_u5fd7_u5f02:id418}}
秀才安大成,四川重庆府人。父亲是个举人,早已去世。弟弟名叫二成,年纪还小。大成娶了个媳妇,小名叫珊瑚,她知礼孝顺又很漂亮。但是大成的母亲沈氏,蛮横无理不讲仁爱,处处虐待珊瑚,但珊瑚脸上毫无怨色。每天早晨,珊瑚都梳洗得干干净净去伺候婆母。一次,正好遇上大成有病,婆母说都是珊瑚打扮得漂亮引诱的,为此叱骂责备她。珊瑚回到自己房里,卸下华饰再去见婆母;婆母反而更加愤怒,自己碰头打脸地哭闹起来。大成向来很孝顺,见闹到这样就用鞭子打了媳妇,母亲的气才略微消了点。从此沈氏更加厌恶儿媳妇。珊瑚虽然侍奉得更加周到谨慎,沈氏却始终不和她说一句话。大成知道母亲生妻子的气,就躲到别处去睡,表示和妻子断绝关系。过了很长时问,沈氏到底也不痛快,成天地指桑骂槐,意思都是在骂珊瑚。大成说:“娶媳妇是为了伺候公婆,像现在这个样,还要媳妇做什么!”于是写了休书,叫了个老妇人把珊瑚送回娘家。

刚刚出了村子不远,珊瑚哭着说:“当个女人做不好媳妇,被人休回家有啥脸去见爹娘?还不如死了算了!”说着从袖子里抽出一把剪刀刺向自己的咽喉。送她的老妇人急忙抢救她,鲜血从伤口冒出来染红了衣襟。老妇人把珊瑚扶到了大成的一个同族婶子家。大成的这个婶子王氏,守寡独居,就把珊瑚收留了。老妇人回到家,大成叮嘱她要瞒着这事,但心里总是怕被母亲知道。

过了几天,大成探听到珊瑚的创伤渐渐好了,就来到王氏门上,让她不要收留珊瑚。王氏叫他进屋,大成不肯进去,只是很气盛地要赶珊瑚走。不一会儿,王氏领着珊瑚出来,见了大成,就问他说:“珊瑚有什么过错?”大成责备她不能伺候婆婆。珊瑚默默地一句话也不说,只是低着头呜呜哭泣,泪水都成了红色,白衣衫也染红了。大成见状心酸,话没说完就扭头走了。

又过了几天,大成母亲已经听说这件事,气冲冲地跑到王氏门上,说了很多难听的话谴责她。王氏傲然相对,反过来数落她的恶行;并且说:“媳妇已经被你休出家门,还是你安家什么人?我自愿收留陈家的女儿,不是留你安家的媳妇,何用你来多管别人家的事!”沈氏真气极了,但却理屈词穷,又见王氏气势汹汹,只得羞惭沮丧地大哭着跑回了家。

珊瑚觉得在这里给王氏找麻烦,自己心里很不安,就想再到别处去。原先,大成有个姨母于老太婆,就是沈氏的姐姐,她年纪六十多岁,儿子已经死了,家里只有一个孙子和守寡的儿媳,她曾很好地待过珊瑚。于是珊瑚辞别了王氏投奔到于大姨那里。于大姨问出了根由,直说自己的妹妹无理暴虐,立即要送珊瑚回婆家。珊瑚再三说不能这样做,又叮嘱她不要对人说。从此珊瑚就和于大姨住在一起,跟婆媳一个样。

珊瑚有两个哥哥,听到妹妹的遭遇很同情她,想把她接回家再另嫁人。珊瑚拿定主意不嫁,只是跟着于大姨纺纱织布用来自已生活。

大成自从休了珊瑚以后,他母亲多次设法为儿子谋划婚事。但是她的凶狠名声到处传遍了,无论远近都没有愿意把女儿嫁给她家做媳妇的。过了三四年,大成的弟弟二成渐渐长大,于是先为二成完婚。二成的媳妇叫臧姑,性情骄横凶暴,言语尖刻不讲情理,比她婆婆沈氏还厉害几倍。婆母有时怒气刚刚表现在脸上,臧姑马上就怒骂出声相还。二成又生性懦弱,不敢袒护自己的母亲。于是沈氏的威风顿减,再不敢冒犯臧姑,反而看着脸色笑着逢迎她,就是这样也还得不到臧姑的欢心。臧姑使唤婆母像奴婢一样;大成又不敢出声,只好自己代替母亲干活,洗碗扫地之类的事都自己干。母子二人常在无人处,面对面地偷偷掉泪。

过了不久,沈氏积郁成疾,身体虚弱得下不了床,大小便翻身都须大成伺候;大成白天黑夜不能睡觉,两只眼睛都熬红了。他弟弟二成来替他伺候一霎,可二成刚进门,臧姑就把他叫了回去。

大成于是跑去找于大姨,希望她能来看望陪伴母亲。进了姨家的门,大成对着姨母边哭边诉苦。他苦还没诉完,珊瑚掀开帘子出来了。大成羞愧极了,停住声就想走。珊瑚用两只手叉住了门口。大成窘急了,从珊瑚腋下冲出去跑回了家,也没敢把这事告诉母亲。

不久,于大姨来到大成家,沈氏高兴地不再让她回去。从这以后于大姨家没有一天不派人来,给她送些好吃的东西。于大姨让来人捎话给寡妇儿媳说:“这里饿不着,以后不要再这样送东西了。”但是她家里仍然按时送好吃的来,从没间断过。于大姨不肯自己吃,全都留着给了生病的妹妹。沈氏在姐姐的照料下身体也渐渐好起来。于大姨的小孙子又按母亲的吩咐拿着好吃的礼物来慰问病人。沈氏叹息着说:“真是个贤孝的媳妇啊!姐姐是怎么修的呀!”于大姨说:“妹妹觉得你休了的媳妇是个怎么样的人呢?”沈氏说:“哎!她的确不像二儿媳那么坏!但却不如外甥媳妇这样贤孝!”于大姨说:“珊瑚在你家的时候,你不知道什么是劳累;你发怒的时候,珊瑚也没有怨言,怎么还说不如我的儿媳呢?”沈氏听说这才掉下泪来,并告诉她自己已经后悔了,又问道:“不知珊瑚改嫁了没有?”于大姨回答说:“不知道,等我打听打听。”

又过了几天,沈氏的病好了。于大姨要回家去。沈氏哭着说:“只怕姐姐回去了,我还是个死!”于大姨于是和大成商议,把二成分出去。二成把意思告诉了臧姑。臧姑听了很不高兴,说了许多难听的话责备大成,并连大姨也牵扯进去。大成情愿把好地全给二成,臧姑这才转怒为喜。分家产的文书写好以后,于大姨才回了家。

第二天,于大姨用马车来接沈氏。沈氏到了姐姐家,先求见外甥媳妇,极力称道甥媳贤孝。于大姨说:“年轻媳妇有百样好,难道就没有一点过失?我不过一向都能容忍她。就是你的儿媳能像我的儿媳一样,恐怕你也不会享受得了。”沈氏说:“哎呀冤枉啊!你把我说成是木头石块山鹿野猪了!都有鼻子有嘴的,难道还能有闻不出香臭来的?”于大姨说道:“就说被你休出门去的珊瑚吧,不知道她现在想起你来会怎么说?”沈氏说:“无非是骂我罢了。”于大姨说:“你若确实做到了无啥可骂的地步,那她还能骂你什么呢?”沈氏说:“过失是人所常有的,惟独她不贤孝,因此知道她会骂我的。”于大姨说:“应当怨恨而不怨,以此可知她对你的贤孝之心;应当离去而不离,以此可知她对你的体谅抚慰之情。以前送东西孝敬你的,本来不是我的儿媳,而是你的儿媳!”沈氏惊讶地问道:“怎么着?”于大姨说:“珊瑚寄居在这里很久了。以前所送的东西,都是她靠夜里纺织赚钱买的。”沈氏听说,老泪纵横地说:“我怎么有脸见我那儿媳啊!”于大姨这才去呼唤珊瑚。珊瑚含着眼泪出来,跪在地上。沈氏惭愧悲痛地自己打开了自己,于大姨极力劝说她才住手,于是婆媳二人和好如初。

十几天以后珊瑚和婆婆一同回到家。家里仅有几亩薄田,已经不够生活开销,只有依赖大成去代人抄抄写写,珊瑚去做针线活来维持生计。二成家倒是很富足,但是哥哥不来求借,弟弟也不去照顾。臧姑因为嫂子曾被休出过家门而看不起她;嫂子也厌恶臧姑的凶悍不讲理,从不和她来往。兄弟两家隔上院墙各住各的院子。臧姑时常发威骂给邻院听,大成一家人都捂上自己的耳朵全当听不见。臧姑没处使厉害,就虐待丈夫和丫鬟。丫鬟有一天受不了虐待,自己上吊死了。她的父亲到衙门告了臧姑,二成代替媳妇去对质说理,挨了一顿责打,最后仍把臧姑传拘了去。大成上上下下为她疏通关节、谋划解脱,终究未能免罪。臧姑受了拶指的酷刑,夹得十个手指头上的肉都脱落了。县官贪婪暴戾,勒索的胃口很大。二成拿良田作抵押借来了钱,如数缴上,两口子这才被释放回家。但是债主催逼还债一天急于一天。没有办法,二成只好全把良田卖给了本村的任翁。任翁因为这些良田半数是大成让给二成的,就叫大成在文书上签字。大成到了任家,任翁见了他忽然自己说:“我是安举人。任某是什么人,敢买我的家产!”又看着大成说:“冥府感念你夫妻俩孝顺,因此叫我暂且回来见你一面。”大成流着眼泪说:“父亲有灵,请赶紧救我弟弟吧!”只听父亲的声音说:“这逆子悍妇两口子,不值得怜惜!你快回家治办银子,赎回我的血汗家产。”大成说:“我们母子仅能糊口活命,怎能得到那么多银子?”父亲的声音回答说:“咱家的紫薇树下藏有银子,可以取出来用。”大成想再问他,任翁已不说话了;不一会儿他醒过来,茫然不知自己刚才都说了些什么。

大成回到家如实对母亲说了,母亲也不怎么相信。臧姑一听说这事,先早已领着好几个人前去挖银窖了。可挖下去四五尺深,只见到些砖瓦石块,并无所谓的藏银,便失望地回去了。大成听说臧姑已去挖银窖,就告诉母亲和妻子不要去看。后来知道她没挖到,沈氏便偷偷到那里去看,只见一些砖瓦石块掺杂在土里,也就回来了。珊瑚接着也到了那里,却见土里全是些白花花的银锭;她喊大成去验证,果然是银子。大成认为这是父亲遗留的财富,不忍心私自独吞,就招呼二成来平半分了它。拣出来银锭数量恰好能平均分成两份,兄弟俩各装了一袋带回家去。

二成和臧姑一同检验银子数量,打开袋子一看,里面竟然装了满满一下子砖头瓦块,两人大惊。臧姑怀疑二成是被大成愚弄了,让二成去看大成的。二成见大成把银子堆放在桌子上,和母亲共同庆贺,便把实情说给哥哥听。大成也十分吃惊,心里很同情弟弟,就把桌子上的银子全都送给了他。二成于是高兴起来,拿着银子去还清了欠债,很感激哥哥的仁义。可臧姑却说:“就这件事越发知道大成的奸诈。若不是他自己心里有愧,谁肯把已经分到手的银子再让给人家呢?”二成对臧姑说的话半信半疑。第二天,债主派仆人来到二成家,说他昨天偿还的全是假银子,将要拿着去告官。二成夫妻听说大惊失色,臧姑说:“怎么样啊!我本来就说你哥哥绝不会好到这步天地,他这是来害你呀!”二成害怕,就去哀求债主;债主的怒气就是不消。二成把地契给了债主,任凭他点卖,这才把原来的银子拿回来。仔细看了看,见银子中有两锭被剪断,表面上仅裹着一韭菜叶厚的银皮,而中间全是铜。

臧姑于是为二成出谋:留下两锭被剪断了的,其余的银子送还给大成,看他怎么办。并交给二成去这么说:“承蒙哥哥的好意屡次让我,实在是不忍心。我只留下了两锭,以见哥哥的后意。眼下我那边所有的财产,仍和哥哥的相等。我也不需要更多的田地,既然已经放弃了,赎不赎的就在哥哥了。”大成不知他的真意,还一再让二成。二成很坚决的推辞,大成这才收下了银子。大成把银子称了称,比原来少了五两多。就叫珊瑚典当了首饰凑足了原数,带去交付了债主。债主怀疑还像是先前的那些假银子,可是用剪刀把银子剪断验证了一下,全是足色的纹银,没有一点差错,就收下银子,把地契还绐了大成。二成给大成送回银子后,以为他必定会惹出事端来的;可随后听说地契已经赎回来了,大为惊奇。臧姑怀疑是当初挖掘时,大成先藏起了真银子,就气急败坏地到了哥哥家里,声色俱厉地数落诟骂。大成这才明白了二成送还银子的缘故。珊瑚迎上前去笑着说:“地契本来在这里,何用生那么大的气!”叫大成拿出地契交给了臧姑。

二成有天夜里梦见父亲谴责他说:“你不孝顺母亲不尊敬兄长,阴间的期限已近在眼前,寸土都不是自己的,你还赖着占用将作何用?”他醒来把梦告诉了臧姑,想把地还给哥哥。臧姑反而讥笑他愚蠢。这时二成已有了两个男孩,大的七岁,小的三岁。不久,大儿子生水痘死了。臧姑这才害怕了,叫二成把地契退给哥哥。可二成去了再三说,大成就是不收。没过几天,小儿子又死了。臧姑愈加害怕,便自己把地契送去放到了嫂子屋里。春季就要过去了,归还的地里还都荒着没耕,大成不得已,只好自己去耕种。

臧姑从此改变了以前的恶行,早晚都去给婆母请安,犹如孝子;对嫂子也极尊敬。不到半年,婆母因病去世了。臧姑哭得很恸,竟到了食水不进的程度。她对人说道:“婆母早死,叫我不能尽孝心,是老天不许我自己赎罪啊!”后来臧姑生了十胎,但一个孩子也没活,最后只得过继了哥哥的儿子为子。夫妻二人都长寿而终。大成和珊瑚夫妇共生了三个儿子,有两个考中了进士。人们都说这是他俩孝敬父母友爱兄弟的好报。


\subsection{1.10.22   五 通}
\label{\detokenize{p00_u5176_u5b83/_u767d_u8bdd_u804a_u658b_u5fd7_u5f02:id419}}
南方有五通神,犹如北方有狐狸精一样。但北方狐狸怍祟,还能想方设法驱赶;江浙一带的五通神,则是随意霸占老百姓家漂亮的妇女,父母兄弟,没有一个敢吭气的。因此,为害尤其厉害。

有一个叫赵弘的,是吴中的典当商人,妻子姓阎,长得很有姿色。一天夜晚,一个男子从外面昂然走了进来,手按宝剑,四下环顾。丫鬟、婆子吓得尽都逃走。阎氏刚要出来,男子蛮横地拦住她,说:“不用害怕,我是五通神中的四郎。我喜欢你,不祸害你。”便拦腰抱起她,像举个婴儿一般,放到床上。妇人的衣服、腰带自动解开。四郎粗暴异常,阎氏不能忍受,迷惘中痛声呻吟。事毕下床,四郎说:“五天后我还来。”于是走了。赵弘在城门外开典当铺,晚上没有回家,丫鬟奔跑了去告诉他。赵弘知道是五通神,问都不敢问。天将明,赵弘回家见妻子疲惫不堪,卧在床上起不来,心里很感羞耻,告戒家里人不要传出去。

阎氏三四天后才恢复过来,又害怕四郎再来。到了第五天,丫鬟婆子都不敢睡在阎氏卧室内,全都避到外间里,只有阎氏孤身一人面对着蜡烛,愁闷地等着五通神的降临。不长时间,四郎带着两个人来,都是年轻人,一副风流潇洒的样子。童仆摆上酒肴,三人与阎氏一块喝酒。阎氏又羞又怕,低头无语,强让她喝也不喝,心里惴惴不安,恐怕他们三人轮番奸淫,那命就没了。三人互相劝酒,有的喊大哥,有的叫三弟。直喝到半夜,上座上的两个客人才一块站起来说:“今天四郎因喜得美人而款待我们,应该告诉二郎、五郎,大家凑资买酒庆贺。”于是告辞走了。四郎拉着阎氏进入床帐,阎氏哀恳饶过,四郎不听,直至阎氏昏迷过去不省人事,四郎才离去。阎氏奄奄一息,躺在床上,羞气交加,便想自尽。但一上吊绳子就断,试了好几次都是这样,苦于死不了。所幸四郎不常来,大约阎氏身体痊愈后才来一次。这样熬了两三个月,一家人都无法生活。

会稽有一个万生,是赵弘的表弟,为人刚强勇猛,精于箭术。一天,万生来拜访赵家,天已晚,赵弘因为客房都被家人占用,便让万生到内院去住。万生翻来覆去睡不着,过了很久,忽然听到院子里有脚步声;趴在窗子上偷偷往外看看,见一个陌生男人进入表嫂的卧室,心中大疑,便持刀暗暗尾随。来到卧室往屋里一瞅,只见那男人和阎氏并肩坐着,桌子上摆放着酒肴。万生怒火升腾,持刀奔入室内,男子惊诧地站起来,急忙找剑,万生已挥刀砍中他的头颅,脑袋裂开,死在地上。仔细一看,原来是一匹小马,像驴那样大小。万生惊愕万分,询问表嫂,阎氏详细地告诉了他,又焦急地说:“那些五通神马上就要来了,怎么办?”万生摇手示意,叫别出声,自己吹灭蜡烛,取出弓箭,埋伏在暗处。不一会,大约有四五个人从空中飞下,刚落到地面,万生急忙射出一箭,为首的中箭倒地;剩下的三个怒吼着,拔出宝剑,搜索射箭人。万生抽出刀,藏在门后,不出声,也不动。一会儿,有一个走进来,万生突然跃出,挥刀砍去,正中那人脖颈。也死了;仍藏在门后,很久很久,没有动静。于是出来,敲门告诉赵弘。赵弘大惊,一块点亮蜡烛察看,见一匹马、两头猪死在室内。全家庆贺。恐怕剩下的两个会来报仇,就留万生在家,烤猪肉、烹马肉供奉他,味道很美,不同于平常的菜肴。万生从此后名声大振,住了一月多,五通神绝无踪影,便想告辞回去。

有个木材商人苦苦恳求万生去他家住住。原来,木商有个女儿还没嫁人,忽然五通神白天降临,是一个二十来岁的美男子,说要聘他女儿为妻,送黄金百两,约定吉日便走了。计算着日子已经临近,全家人惊惶不安。听到万生的大名后,执意请万生到家里来捉怪。恐怕万生不愿来,先是隐瞒了实情不说。将万生请到家,盛宴款待后,命女儿盛妆而出,拜见客人。那女子大约十六七岁,生得十分漂亮。万生很惊讶,不明白是什么缘故,忙离座鞠躬行礼。木商把他按在座位上,将实情告诉了他。万生刚听说还有点紧张,但平生豪爽意气,所以也不推辞。

到了约定的那天,木商依旧在门口张灯结彩,却让万生坐在室内;一直等到日头西斜,五通神还没来。木商暗喜那五通神新郎是注定要被杀死了。不一会,忽见房檐上有东西像鸟一样飞落下来,落地则是一青年人,穿着华丽的衣服,来到室内。看见万生,返身便逃。万生急追出门外,但见一道黑气刚要飞起,万生跃起一刀砍去,断掉一只脚,怪物嗥叫着逃走了。俯身仔细一看,巨大的爪子,像手一样,不知是什么东西。循着血迹找寻,怪物已逃入江中。木商大喜,听说万生没娶妻,这晚便在已准备好的新房里,让万生和女儿成了亲。

于是,原来常遭五通神祸害的人家,都拜请万生住到家中。共住了一年多,万生才带着妻子离去。从此后,吴中“五通”只剩下“一通”,再也不敢公然为害了。

又:金生,字王孙,是苏州人。在淮水一带设馆教书,住在一官宦人家的花园里。花园中房屋不多,花草树木,丛杂茂密。每当夜深以后,童仆都走了,只剩金生一个人在灯下闷坐,形单影只,心情很是寂寞、惘怅。

一天晚上,三更将尽,忽然有人用指头叩门。金生忙问是谁,门外答道:“借个火,”像是童仆的声音。开门让进来,却是一个年轻漂亮的女子,后面还跟着个丫鬟。金生十分惊异,怀疑是妖物,穷根究底地询问来历。女郎说:“我觉得你是个高雅潇洒的文士,可怜你孤单寂寞,所以不怕人说闲话,来和你共度良宵。恐说明我的来历,我不敢来,你也不敢收留。”金生又怀疑是邻居家私奔的女子,害怕毁了自己的操行,请她离开。女郎眼波一送,勾魂摄魄。金生不觉心醉神迷,再也控制不住自己。丫鬟见此情景,便说;“霞姑,我先走了。”女郎点头,又接着呵斥道:“走就走吧,什么霞姑云姑的!”丫鬟离开后,女郎笑着说:“正好家里没人,便带她一块来,却这样无知,把我的小名泄露给了你。”金生不安地说:“你这样精细,我怕这里头埋藏着什么祸患。”女郎安慰道,“时间长了你就知道了,保证不会有损你的品行,不用担心。”上床后,金生解开女郎的衣服,见她手腕上戴着副手镯,用细金条穿连宝石做成,还镶嵌着两颗明珠。蜡烛熄灭后,宝石、明珠的光芒照亮了整个屋子。金生越发惊怕,到底也猜不透女郎是从哪里来的。事完,丫鬟来敲窗子,女郎起来,用手镯照着路,进入树丛中走了。从此后,女郎每晚都来。

一次,金生等女郎回去时,远远地尾随着,想看个究竟,女郎似乎已察觉,忽然掩蔽了手镯的光芒。树丛深处,黑得伸手不见五指,金生只好返回。

隔天,金生骑马到淮北去,头上斗笠的带子断了,风一吹,就要刮下来,只好不时地用手按按。来到淮河,乘一叶小舟渡河,忽然一阵风来,将斗笠吹落河中,随着水流漂走了,金生怅然若失。过河后,一阵大风,又将斗笠刮了回来,飘在空中,团团旋转着,渐渐落下来。金生用手接住,一看,带子已经接好了,心中大感惊异,回到学馆,金生向女郎讲述这件怪事,女郎也不说话,只是微笑而已。金生怀疑是女郎干的,假装生气地说:“你若真是个神人,应当明白告诉我,免得我烦恼疑惑!”女郎说:“你冷清寂寞的时候,有我这样一个痴情女子为你解忧驱闷,我自觉自己并不是坏人。即使我能做那件事,也是爱护你啊!现在你这样苦苦盘问我,想和我绝情吗?”金生听了,不敢再问。

在此以前,金生有个外甥女儿,已经嫁人,被五通神迷住。金生日夜忧心,但从没告诉别人。因为和女郎亲昵久了,无话不说,便把自己的这件心事告诉了她。女郎沉吟道:“这种东西,我父亲驱赶得了。只是怎么拿情人的私事和父亲说呢?”金生哀求想个办法,女郎思索了会儿,说:“倒也不难除掉,但得我亲自前去。那些怪物都是我家的奴仆,假设争斗间被他们一个指头戳到身上,那这耻辱是跳进大江也洗不清的。”金生哀恳不已,女郎答应说:“马上替你想办法。”第二晚,女郎来告诉金生:“已经派丫鬟南下了。丫鬟力量弱,恐不能立即杀死那怪。”次日晚上,二人方才睡下,丫鬟叩门。金生急忙起床,开门请进。女郎便问:“怎么样?”丫鬟回答:“我擒拿不住,已经把他阉了!”二人笑着询问经过,丫鬟讲述道:“起初我以为在金郎家,去了后,才知不是。等赶到外甥女婿家,已到了掌灯时分。娘子正在灯下靠着几案打盹。我把娘子的魂魄敛在一个瓦罐中,自已躺在床上等着。一会儿,怪物来了,刚进门又急忙退出,说:‘怎么有生人气味?’仔细看看,没有别人,复又进屋,掀开被子钻进来,又惊说:‘怎么有兵器的气味?’我本不想脏了自己的手,但怕迟则生变,急忙捉住那脏东西一刀割掉,怪物嗥叫着逃走了。打开瓦罐,放出魂魄,娘子像醒了过来,我就回来了。”金生大喜,再三致谢。女郎和丫鬟一块走了。

此后,一连半个多月,女郎一次没来,金生慢慢彻底绝望了。到了年底,想辞馆回家,女郎忽然来了。金生惊喜万分,出门迎接,说:“你躲了我这么长时间,我以为什么地方得罪了你,原来没有和我绝情啊?”女郎说:“相好了一年,分手时不说句话,终是遗憾的事。听说你要撤馆回家,我特来送别。”金生请她一块回去,女郎叹息道:“叫我怎么说呢!现在马上就要长别,我也不忍再瞒你:我是河神金龙大王的女儿。因为和你有夙缘,所以来投奔你。我不该派丫鬟下江南,以致江湖上到处都在传言我替你阉割五通怪。父亲听说后,认为是家门的奇耻大辱,十分震怒,要赐我自尽。多亏丫鬟一力承当,把事情都揽了过去,父亲才稍减怒气,将丫鬟杖打一百。现在,我每行一步,都有保姆跟随。抽机会来看看你,也不能尽诉衷肠,有什么办法呢?”说完,便要告别,金生哭着拉住不放。女郎凄然地说:“你不要这样,我们三十年后能再相会。”金生说:“我现在已三十岁了,再过三十年,成了白头老翁了,有什么脸再相见。”女郎道:“不是的,龙宫里无老人。况且人活着是长寿是短命,也不在容貌。如果仅求容貌不老,那太容易了。”于是写了张药方子给金生,自己走了。

金生返回家乡后,外甥女谈起那件怪事,说:“那天晚上,我像做了个梦,觉得有人捉住我塞进了瓦罐中。等醒过来,见鲜血沾满床褥,怪物从此灭绝了。”金生解释说:“是我祈祷的河神捉怪。”一家人方才打消疑虑。后来,金生六十多岁时,容貌还像是三十来岁的人。一天金生乘船渡河时,远远望见上游漂来一片荷叶,像席子那样大,一个美丽的女子坐在上面。近处一看,正是神女霞姑。金生一跃跳到荷花上,一会儿,人与荷花渐渐漂远了,越来越小,最后像铜钱那样大,终于看不见了。

这件事与赵弘那件事,都发生在明朝末年,只不知谁在前,谁在后。如果这事发生在万生诛杀五通神之后,那么吴中“五通”就只剩下“半通”,越发不足为害了。


\subsection{1.10.23   申 氏}
\label{\detokenize{p00_u5176_u5b83/_u767d_u8bdd_u804a_u658b_u5fd7_u5f02:id420}}
泾河边上,有个读书人的儿子,姓申,家里非常贫穷,甚至于整天没米下锅。夫妻二人相对愁闷,想不出一点办法。妻子说:“要不,你去偷吧?”申某生气地说: “读书人的后代,不能光宗耀祖倒也罢了,怎能去败坏门户、羞辱祖宗呢?与其做强盗活着,还不如饿死!”妻子也气愤地说:“你想活着又怕丢脸吗?世上没有田产又能过日子的,只有两条路:你既不能去做强盗,我只好去当娼妓了!”申某大怒,跟妻子吵骂起来,妻子含忿去睡下了。

申某想:一个男子汉,连两顿饭都挣不来,竟使妻子要去当娼妓,真不如死了。便悄悄地下床,来到院子里,在一棵树上上吊了。忽见他已死去的父亲走来,劝告他说:“傻孩子!何至于这样呢!”说着,弄断了他上吊的绳子,说:“强盗不妨去做一次,但须拣庄稼茂密的地方藏身。这次去可以发家,勿须第二次了。”妻子听到院子里有什么东西落地的声音,一下子惊醒过来。叫叫丈夫,没见答应,便点上灯寻找。走到院子里,见树上有根断绳,申某死在地下,妻子十分惊骇。急忙替他按摩,过了会儿申某才苏醒过来。妻子把他扶到床上躺下,心中的气也消了。

天明后,妻子假托丈夫病了,去邻居家借了点稀饭给丈夫吃。申某吃完,出门走了。到中午,背了一袋米回来。妻子问米是哪来的,申某说:“我父亲的朋友都是有钱人家,过去我认为向人摇尾乞怜太羞耻,所以不屑于求人。现在我马上就去做强盗了,还顾什么脸面?赶快做饭,我马上就听你的话,去抢劫去!”妻子以为他还在生自己的气,隐忍着没还嘴。淘了米做了饭,申某饱吃一顿,急急忙忙地找来根结实棒子,用斧子砍成木棍,拿在手里就走。妻子见他不像是开玩笑,急忙拉住了他。申某说:“这是你让我去的,如果我被抓住连累了你,你不要后悔!”挣脱妻子的手,径直出门走了。

天黑后,申某摸到邻村,在离村子一里多远的地方藏了起来。这时,天上忽然下起暴雨,申某浑身上下全被淋湿了。远远望见有片浓密的树林,他便走过去避雨。闪电一照,申某发现自己已靠近村庄。远处像有行人,他恐怕被人看到,见一堵墙下庄稼茂密,急忙躲了进去。不一会儿,一个男人走过来,身躯健壮魁伟,也钻进了庄稼地里。申某害怕,一动不敢动。幸亏那人从一边斜插了过去。申某在暗处见那人翻过墙去,心中暗想:墙那边是本村富户亢家的住宅,这人必定也是强盗!等他偷了东西出来,我应分他一份。又想:这人太强壮,跟他客客气气地要,他如不给,必然动武,我不是他的对手;不如等他跳墙出来时,冷不防打翻他。计划已定,藏着等了很久。直听到鸡叫时,才见那人跳墙而出,脚还没落地,申某突然跃出,一棍扫去,正打中那人腰部,一下子跌倒在地。申某仔细一看,那人竟是一只大乌龟,张着盆一般的大嘴。申某大吃一惊,又连连打去,终于打死了它。

原来,亢家有个女儿,非常聪明美丽,父母待如掌上明珠。有天夜晚,一个男人忽然进来,逼她交欢,女儿正要喊叫,那人的舌头已伸入她嘴里,她立即昏迷过去,不醒人事,听任那人糟踏后走了。亢家羞于把这事告诉别人,只是派了很多奴婢婆子护守女儿,并严锁门户而已。但到了夜晚,门又不知不觉地开了。那人一进屋,众人都立即昏迷过去,奴婢婆子们被挨个奸淫了一遍。于是,大家都害怕起来,去告诉亢老翁。亢老翁让家人手持利刃,把女儿的卧室团团围起来,又让室内的人点着灯坐着守护。大约到了半夜,里里外外护守的人忽然都像睡着了。过了一会儿猛然惊醒,见亢老翁的女儿光着身子躺着,像痴了一样,过了很久才清醒过来。亢老翁十分愤恨,但又想不出办法。过了几个月,女儿病得皮包骨头,奄奄一息了。亢老翁告诉众人,有能驱除怪物的,酬谢三百两银子。申某过去也听说过这件事,这晚打死了乌龟,才醒悟祸害亢家女儿的,必定是这个东西,于是就去亢家讨赏。亢老翁大喜,将他请到上座,又让人把死龟抬到院子里,一片片地零割掉。留申某过了一夜,这夜果然怪物绝迹了。亢老翁便如数赠了他三百两银子,申某背了银子返回家中。

申妻因为丈夫出去连续两夜没回家,正在担忧盼望,丈夫突然回来了。妻子急忙问他去哪了,申某默默不语,把背着的银子全倒在床上。妻子一见,几乎吓死过去,说:“你真做了强盗了?”申某说:“是你逼我去干的,现在又说这种话!”妻子哭泣着说;“我那是和你开玩笑啊!现在你犯下杀身之罪,我不能受贼人的连累,让我先死吧!”说着,跑出门去。申某急忙追出去,笑着把她拖了回来,详细讲了打龟讨赏的经过,妻子才高兴起来。此后,申某夫妻用这些银子辛勤经营,渐渐成了富裕人家。

写这个故事的人认为:人值得忧虑的事情不是贫穷,而是没有品行。品行端正的人,虽然挨饿,但终究不会死;即使不被人可怜,也会有鬼神保佑。世上那些贫穷的人,往往见利忘义,为了得到一口饭便不顾羞耻,人还不屑于给他一文钱,又凭什么能得到鬼神的同情呢?

我们县有个穷人某乙,腊月都快过去了,身上还没件囫囵衣服。自己想:怎么才能熬过年去呢?也不敢和妻子商量,悄悄地拿了根白本棍,出去埋伏在一座墓地里,希望能碰上孤身走道的,好劫人家的东西。苦苦地盼望了很久,还是渺无人影。墓地里寒风刺骨,他再也忍受不住了,正感到绝望,忽见一人怄偻着身子走过来。某乙心中暗喜,等那人走近,他手持木棍突然跳了出去,见是一个老头子背着个口袋。老头吃了一惊,等看清是劫道的,连连向某乙哀求说:“身上没别的东西,家里断了顿了,才从女婿家借来这五升米!”某乙夺过米袋子,还想剥下老头的衣服。老头苦苦哀求,某乙念他年老,放了他,只把米拿回了家。妻子问他米哪来的,他假称是别人还的赌债。

经过这件事后,某乙暗想抢劫倒是个好办法,第二晚他又去了那片墓地。呆了不一会儿,见一个人拿着根棍子走来,也藏在了墓地里。某乙蹲着远远地望了他一眼,心里明白也是劫道的,便犹豫着走了过去。那人看见他,吃惊地问:“你是谁?”某乙回答说:“走路的。”那人又问:“怎么不走呢?”某乙说:“等着你啊!”那人失声而笑,明白了来人也是同道。两人各自诉说了一番饥饿寒冷的苦楚。夜深后,没碰上一个人,某乙便想回去。那人说:“你虽然干了这一行,但看来是个雏儿。前村有家嫁女儿,一直操办到半夜,这时全家人肯定都劳累地睡着了。你跟我去一趟,得到财物我们对半分。”某乙大喜,跟着他走了。

他俩来到一家门口,听到隔壁传出打饼的声音,知道那人家还没睡下,二人便藏起来等着。不一会,有个人开门挑着桶出来打水,二人乘机溜了进去。见北屋有灯光,别的屋一片昏黑。这时,听见一个老婆婆的声音说:“大姐儿去东屋看看,你的嫁妆都在柜子里,看忘记上锁了没有?”又听见一个少女娇懒的回答声。二人暗喜,又摸到东屋里。在暗中摸到柜子,掀开柜盖一探,深不见底。那人便对某乙说:“你进去!”某乙跳入柜子,把包裹一一递出来。那人问:“完了吗?”某乙说:“完了。”那人骗他说:“你再摸摸!”乘某乙不备,一下子合上柜盖,上了锁走了。某乙被锁在柜子里,又窘又急,想不出办法。不一会儿,见屋里有灯的光亮。有人用灯光照了照柜子,听见那个老婆婆的声音说:“谁已锁上了!”于是母女二人灭灯上床睡下了。某乙焦急不堪,模仿起老鼠啃东西的声音。只听那少女说:“柜子里有老鼠!”老婆婆说:“别咬坏了你的衣服。我累极了,你自己去看看吧。”少女穿衣下床,打开锁掀开柜子,某乙突然跳了出来。少女吓得跌倒在地,某乙乘机打开门逃了出去。虽然没得到什么东西,但侥幸没被捉住。这家人家夜间被盗的事,立即传遍了四方。有人怀疑起某乙,某乙害怕,往东逃了一百多里,被一个旅店的主人雇佣了。过了一年多,流言平息了,他才回来把妻子接了去,从此再不干偷盗了。

这件事是某乙自己讲述的,因为类似申某的故事,所以一并记下它。


\subsection{1.10.24   恒 娘}
\label{\detokenize{p00_u5176_u5b83/_u767d_u8bdd_u804a_u658b_u5fd7_u5f02:id421}}
京都人洪大业的妻子姓朱,长得美丽标致,夫妻二人感情很好。后来,洪大业又纳了个婢女为小妾,名叫宝带,姿色远不如朱氏,但洪大业却偏偏宠爱她。朱氏不平,经常为了这事和洪大业吵闹不休。洪大业虽然不敢公开睡在小妾房里,但从此后越发宠幸宝带,疏远朱氏了。

不久后,洪大业迁家,和一个姓狄的布商作邻居。狄的妻子名叫恒娘,先过院来拜会朱氏。恒娘约三十多岁年纪,姿色平平,但言谈巧妙动人,朱氏十分喜欢。第二天,朱氏去回访,见狄家也有一个小妾,二十多岁年纪,相貌非常漂亮。两家相邻近半年,从没听到恒娘骂过小妾一次,但布商却独独宠爱恒娘,妾房仅是虚设而已。朱氏很感奇异,一天见恒娘询问缘故,说:“我原以为男人爱妾,不过因为她是‘妾’罢了,常想把‘妻子’的名目换成‘妾’。现在才知道不是这样。你用的什么法术?如能传授,我愿给你当弟子!”恒娘笑着说:“唉!是你自己疏远了他,怎能怨男人呢?整天从早到晚絮絮叨叨,这不是为丛驱雀、为渊驱鱼吗?只能是愈加疏离了二人的关系。回去后,你应该放纵他,别再干涉他的行动,如果他和你套近乎,也不要理他。一个月后,我再替你想办法。”

朱氏听从了恒娘的建议,回家后,越发打扮宝带,让她和丈夫一块睡,一块吃。洪大业偶而应付应付朱氏,朱氏总是严加拒绝。于是,一家人都夸朱氏贤惠。这样过了一个多月,朱氏去见恒娘。恒娘喜悦地说:“好了!你回去后,别再打扮,不穿华丽衣服,不要施脂抹粉,让自己污面破衣,和家里仆役们一起劳作,一月后再来。”朱氏听了后,回家便穿起破衣服,故意让自已浑身肮脏,除了纺线织布,别的事一概不管。洪大业可怜她,有时让宝带帮她干点活,朱氏不让,总是将宝带喝开。这样过了一个月,又去见恒娘,恒娘夸奖说:“孺子真可教也!后天是上已节,我想约你一块逛春园,你要丢掉破衣,精心梳妆,浑身上下焕然一新,早早过来见我!”朱氏答应道:“好吧。”到了那天,朱氏照着镜子涂脂抹粉,按照恒娘的吩咐,精心梳妆。打扮完,去见恒娘,恒娘喜欢地说:“可以了。”又替朱氏挽头发,光可鉴影;衣服不时髦的地方,拆了重做;又说她的鞋样式太拙,从针线筐中翻出一双正在做着的鞋,赶完后让朱氏换上。……两人临分别,让朱氏喝了点酒,嘱咐说:“回去后见过丈夫,就早点关门睡觉。他若是叫门,不要听。叫三次门,才可让他进去一次。他想和你亲热,也不要太迁就他。半个月后,你再来。”

朱氏回家,盛妆去见丈夫。洪大业一见,露出非常惊异的样子,上上下下地凝目打量,有说有笑,不像平时。朱氏略微讲了讲游园的情况,便手托香腮,作出一副疲惰的样了。天还没黑,就起身回房中睡觉。不长时间,洪大业果然来敲门,朱氏高卧不起,洪大业只得离去。第二晚洪大业又来叫门,同样吃了闭门羹。天明,洪大业责备朱氏,朱氏说:“我一个人睡惯了,受不了别人的打扰。”日头刚一偏西,洪大业就赖在朱氏房中不走。天黑,二人灭烛上床,极尽欢爱,犹如新婚。又约下夜再相会,朱氏觉得不能太频繁,和洪大业约定三天相会一次。

大约过了半月,朱氏又去见恒娘,恒娘关上房门对她说:“从此后你丈夫只会喜欢你一个人了。但你虽然很美,却不妖媚。以你这样的姿色,再媚一点能胜过西施,更何况还不如西施的人呢!”于是让朱氏飞了个媚眼,恒娘纠正说:“不对,毛病出在眼眶上。”让朱氏笑了一下,又说:“不对,毛病在左腮上,”于是恒娘自己秋波送情,又嫣然媚笑,让朱氏模仿。朱氏一连学了几十次,才大致模仿得和恒娘一样。恒娘说:“你可以回去了,照着镜子仔细演习。我的方法就是这些了。至于床上功夫,关键在随机应变,投其所好,这不是言词所能表达的。”朱氏回去,完全按照恒娘教的去做,洪大业果然被迷得神魂颠倒,唯恐遭到朱氏拒绝,每天天不黑,便和朱氏调笑,不离开朱氏的房子半步,赶也赶不走。朱氏却更加善待宝带,每次在卧室中饮宴,都招呼宝带同榻而坐。但洪大业却觉得宝带越来越丑陋,越来越看不顺眼,经常是酒还没喝完,就让宝带走开。朱氏把丈夫骗到宝带房中,再锁上门,洪大业也是一夜不理宝带。从此后,宝带开始恨洪大业,常常对人怨骂,洪大业听说后更讨厌她,渐渐地就打骂起宝带来。宝带羞愤不堪,索性破罐子破摔,整天拖着双破鞋。头发乱蓬蓬的像柴草一样,再不成人了!

一天,恒娘问朱氏:“我的法术怎么样?”朱氏说:“妙倒是很妙,但弟子我却解不透其中奥妙。先是要放纵男人,这是为什么?”恒娘道:“你没听说过吗,人都是喜新厌旧,重难轻易?男人宠爱小妾,不一定是因为她生得美,而是刚娶进门觉得新鲜,又难得同床一次,就更增加了这种新鲜感。现在放纵他,让他尽情享受,山珍海味也有吃厌的时候,更何况还是野菜羹呢?”朱氏又问:“先毁了盛妆,又再盛妆炫耀,这又是为什么?”恒娘回答:“让他不注意你一段时间,乍见之下,则如久别重逢;忽然又见你艳妆浓抹,就像刚娶的新妇,这好比穷人突然得到肉食美味,那么再看看粗米就难以下咽了。但又不马上满足他,让他觉得那个已陈旧而我新鲜;那个容易得到而我难以得到。这就是你变妻为妾的办法了。”朱氏十分喜欢,和恒娘结成闺中密友。

过了几年,恒娘忽然对朱氏说:“我们两个人好得像一个人一样,应当不对你隐瞒我的生平。过去我一直想跟你说,是怕你疑虑。现在马上要分别了,我也就实话告诉你吧:我实是狐狸,幼年时被继母卖到京都中。我丈夫对我很好,所以不忍心和他立即决别,留恋至于今天。明天我父亲仙逝,我回去省亲,再不会回来了。”朱氏听说,拉着恒娘的手唏嘘落泪。第二天一早去看恒娘,见狄氏全家惊慌,原来恒娘突然无影无踪了!


\subsection{1.10.25   葛 巾}
\label{\detokenize{p00_u5176_u5b83/_u767d_u8bdd_u804a_u658b_u5fd7_u5f02:id422}}
常大用,是洛阳人,特别喜爱牡丹,听说曹州牡丹甲齐鲁,就一心想去看看。恰好因为有别的事来到曹州,常大用就借住在一家官宦人家的花园里。当时是二月天,牡丹还没开放。他整天在园中徘徊,注视着那幼芽,希望它早日开花,并作了一百首怀牡丹诗。不久,牡丹渐渐含苞待放,而他的盘缠也快用完了。他就找了些春天的衣服典当点钱生活,整日流连于牡丹园中,忘了回家。

一天凌晨,常大用来到牡丹花园,看见一位女郎和一位老婆婆已经先在那里。他怀疑是富贵人家的家眷,就赶紧回来了。黄昏时候,他又去,又看见她们,就从容地躲在一旁。远远地偷看她们,只见那女郎穿着十分艳丽的宫装,令人眼花缭乱。常大用迷惑不解,转念一想:这一定是位仙人,人间哪有这么美丽的女子!急忙返回去寻找她们。他转过假山,正好遇到老婆婆,那女子正坐在石头上,他们相互看见都吃了一惊。老婆婆用身子挡住女郎,呵叱常大用说:“大胆狂生,你想干什么?”常生直挺挺地跪着说:“娘子必定是神仙!”老婆婆斥责他说:“如此妄言,就该捆起来送到县官那里!”常生非常害怕。女郎微笑着说:“我们走吧!”就转过假山走了。常生往回走,连脚也迈不动了。心想那女郎回家告诉父母,必定有人来辱骂他。他仰面躺在床上,后悔自己卤莽冒失;又暗自庆幸女郎脸上没有怒容,也许没把这事放在心上。他一会儿后悔,一会儿害怕,折腾了一夜晚病倒了。第二天太阳老高了,不见有来兴师问罪的,常大用心情才慢慢平静下来。他回想起女郎的音容笑貌,又转害怕为想念了。这样过了三天,憔悴得快要死了。

这天深夜,仆人已经睡熟了,常生还点着蜡烛没睡。忽然上次见过的那个老婆婆走进来,手中捧着个杯子说:“我家葛巾娘子亲手调和了毒药,要你赶快喝了。”常生听了非常害怕,随后就说:“我与娘子从来没有什么怨仇,何至于赐我死呢!既然是娘子亲手调和的,与其相思得病,不如服毒死了好!”于是接过杯子就喝了下去。老婆婆笑着接过杯子走了。常生觉得药味又凉又香,不像是毒药。一会儿,觉得胸中宽松舒畅,头脑清爽,酣然入睡。一觉醒来,红日满窗。常大用试着起来,病全好了,心中更加相信她们是神仙。没有机会巴结她们,只能在没人的时候到她站过、坐过的地方,虔诚地跪拜,默默地祷告。

一天,他正在园中散步,忽然在树林深处,迎面遇见那女郎。幸好没有别人,常生高兴极了,立即跪在地上。女郎过来拉他,常大用忽然闻到女郎身上有种奇异的香气,就手握着女郎雪白的手腕站起来,只觉女郎皮肤柔软细腻,令人骨节欲酥。正想说话,老婆婆忽然来了。女郎叫常大用藏到石头后面,指着南边说:“夜里你用梯子翻过墙去,见四面红窗的房子,就是我住的地方。”说完匆匆走了。常生怅然若失,像掉了魂,不知道女郎到什么地方去了。到了夜里,他搬了梯子登上南边的墙头,看见墙里边已经有个梯子放在那儿。常生高兴地踩着梯子下去,果然看见有座四面红窗的房子。听到屋里有下棋的声音,不敢往前走。在外面站了很长时间,只好翻墙头回去。一会儿,再过来,棋子的声音仍然频频作响。常生慢慢靠近窗户偷看,见女郎同一个素色衣服的美人正在下棋,老婆婆也坐在那儿,有一个丫鬟在旁边侍候。他只好又返回去。往返了三次,已经三更天了。常生伏在梯子上,听到老婆婆出来说:“梯子!谁放在这里的?”叫丫鬟来一起把梯子搬走了。常生爬上墙头,想下去没有梯子,只好闷闷不乐地回去。

第二天夜里常生又去,梯子已经放在那儿。幸亏寂静无人,常生进去,看见女郎独自坐着,似乎在想什么事。看见常生,女郎吃惊地站起来,羞羞答答地斜过身子站着。常生作了个揖说: “我自以为福分浅薄,恐怕同仙人没有缘份,想不到会有今天!”说着就亲热地拥抱她,只觉得她腰身纤细只有一把,呼出的气息像兰花那么清香。女郎撑拒着说: “你怎么这样性急?”常生说:“好事多磨,慢了怕鬼嫉妒!”话没说完,就听见远处有人说话。女郎急忙说:“玉版妹子来了,你可暂时藏到床底下!”常生听从了。不一会儿,一个女子进来,笑着说:“败军之将,还敢和我再战一场吗?我已经烹好了茶,特来邀你痛痛快快地玩一夜。”女郎借口困倦推辞。玉版再三请求,女郎坐着坚决不去。玉版说:“如此恋恋不舍,是不是有男人藏在房里?”强拉着她出门走了。常生爬出来,恨死了玉版。就搜索女郎的枕头席子,希望得到一件女郎遗留的东西。可是房中并没有香奁等物,只有床头上放着一个水晶如意,上边系着条紫巾,芳香洁净,十分可爱。常生揣到怀里,翻墙回到住处。整理自己的衣衫时,闻到沾染的香味依然浓郁,使他对女郎的倾心爱慕更强烈了。可是想到趴在床底下的恐惧心情,又怕被人发觉受到惩罚,想来想去不敢再去了。只有把如意珍藏起来,希望她能来寻找。

隔了一夜,女郎果然来了,笑着说:“我向来以为你是个正人君子,想不到你竟是个小偷!”常生说:“确实如此!之所以偶然一次不做君子,是希望能得到如意。”说着就把她揽在怀里,替她解掉衣裙。女郎洁白的肌肤刚露出来,温热的香气便四处流散。偎抱之间,觉得她鼻息汗气,无处不香,常生就说:“我本来就认为你是仙人,如今更知道不是假的。有幸得到你的赏识,真是三生有缘!只是怕像杜兰香的下嫁,终成离恨!”女郎笑着说:“你过虑了。我不过是钟情的少女,偶然为情爱动了心。这件事你一定要谨慎秘密,怕那些爱搬弄是非的人,捏造黑白;那样你不能插翅飞走,我也不能乘风驾云,遭受灾祸而分离比好离好散就更惨了!”常生认为她说得很对;但始终认为她是仙人,就再三询问她的姓氏。女郎说:“你既然认为我是神仙,仙人何必留姓传名呢?”常生问:“那老婆婆是什么人?”女郎说;“她是桑姥姥,我小时受过她的照顾,所以待她与别的仆人不同。”接着就起来想走,说:“我那里耳目多,在外面不能待得时间太长,有空我还会再来。”临别的时候,向常生讨还如意,说:“这不是我的东西,是玉版遗留在我那儿的。”常生问:“玉版是谁?”女郎说:“是我的叔伯妹妹。”常生把如意还给她,她就走了。

女郎走后,常生的被子枕头都沾染了异香。从此女郎每隔两三晚上就来一趟。常生迷恋她,不再想回家;但是盘缠全花光了,就想卖马。女郎知道以后,说:“你为了我的缘故,才花光了盘缠,又典当了衣服,我实在过意不去。现在你又要卖马,一千多里路你怎么回去?我有点积蓄,可以帮你一点忙,”常生推辞说:“感谢你对我的真情,无论怎样我也无法报答你。如再贪心花费你的钱财,我还怎么做人呢?”女郎坚决要给他,说:“就算是暂时借给你吧!”接着拉着常生的胳膊,来到一株桑树下,指着一块石头说:“搬了它。”常生就把石头搬了。女郎拔下头上的簪子,在土上刺了几十下,又说:“把土扒开。”常生照做了,已经能看见瓮口了。女郎把手伸进瓮里,取出将近五十两银子。常生拉住她的胳膊制止,她不听,又拿出十几锭银。常生强迫着放回去一半,把瓮掩埋好了。一天夜里,女郎告诉常生说:“近几天有些流言,看情景我们不能长聚了。这事我们不能不先商量一下。”常生吃惊地说:“这可怎么办?我一向小心谨慎。如今为了你的缘故,就像寡妇丧失了平日的操守,不能也作不了自己的主。我一切听你的,刀锯斧铖也顾不得了!”女郎出主意说一块逃走,叫常生先回家,约定到洛阳相会。常生收拾行装回家,准备回家后再来迎她。他刚到家门口,女郎的车子也到了,于是一同进门拜见家人。街坊四邻都惊奇地来祝贺,并不知道他们是偷着逃出来的。常生暗暗担心,女郎却很坦然,告诉常生说: “不要说在千里之外寻访不到,就是知道了,我是世代显贵人家的女儿,家里也不敢把我怎样!”

常生的弟弟常大器,这年十七岁。女郎看到他,对常生说:“弟弟本质聪明,前途比你强多了。”大器已定下了完婚的日期,未婚妻忽然死了。女郎说:“我妹妹玉版,你曾经偷偷地见过,相貌很不错,跟弟弟年龄相仿,结为夫妇可称是天生的一对。”常生一听就笑了,用开玩笑的口气请她说媒。女郎说:“如一定要娶她,并不很难。”常生高兴地说:“有什么办法?”女郎说:“妹子同我最要好。只要两匹马驾一辆轻车,派个老婆子跑个来回就行了。”常生害怕他们自己过去的事会暴露,不敢听从她的主意。女郎一再说:“没有妨碍。”就让驾车,打发桑姥姥去接。几天后,来到曹州,快到门口时,桑姥姥下了车,叫车夫在路上等着,自己乘黑夜进了院子。过了很久,才同一个女子一块出来,上车就往回走。夜里就睡在车里,五更天再走。女郎计算她们归来的日子,叫大器身穿盛装去迎接。大约迎出五十里,才相遇。大器上车同她们一块回到家中,鼓乐齐奏,洞房花烛,拜堂成亲。从此兄弟俩都娶了个漂亮媳妇,家境也一天天富裕起来。

一天,几十个骑马的强盗突然闯进常生的家。常大用知道有了变故,带领全家登上楼顶。强盗进来,把楼围住。常大用俯下身子问:“我们可有仇?”回答说:“没仇!但有两件事相求:一是听说两位夫人的美貌是世上没有的,请让我们见一见;再有一件是我们五十八个人,每人向你们讨五百两银子。”说完,强盗们把柴草堆在楼下,摆出放火烧楼的架势来威胁。常大用只答应了给他们每人五百两银子,强盗不满意,要放火烧楼,家人吓得要命。女郎要同玉版下楼,常大用劝说她们,不听。二人穿着华丽的衣服下了楼,站在离地面只差三层的台阶上,对强盗说:“我姐妹都是仙女,暂时来到尘世间,还怕什么强盗!我就是赐给你们万两黄金,恐怕你们也不敢接受!”强盗们一齐仰拜,连声说:“不敢。”姐妹二人正想回楼上,一个强盗说:“这是欺骗我们!”女郎听了,返回身站住说:“你想干什么?趁早说出来,还不算晚!”强盗们你看我,我看你,没有一个说话的。姐妹俩从容地上楼去了。强盗们抬头看不见她俩了才一哄而散。

两年以后,姐妹俩各生了个儿子,这才自己透露说:“我家姓魏,母亲被封为曹国夫人。”常大用怀疑曹州没有姓魏的官宦家。而且如果是大户人家丢失了女儿,怎么能耽搁到现在也不闻不问呢?不敢追根问底,心里却很奇怪,就借口有事又去了曹州。进曹州境内察访,官宦世族根本没有姓魏的。于是,常大用仍旧借住在旧主人家。忽然看见墙壁上有赠曹国夫人的诗,他感到很奇怪,就向主人打听。主人笑了,请他去看看曹国夫人。到那儿一看,却是一棵牡丹,和房檐一样高。常大用问主人花名的由来,主人说这棵牡丹在曹州名列第一,所以同人开玩笑,封它为曹国夫人。常大用问它属什么品种,主人说:“葛巾紫。”常大用心中更惊奇,怀疑女郎是花妖。

回到家后,不敢质问,只是述说那首赠曹国夫人的诗,观察女郎的表情。女郎听了立刻皱起眉头,变了脸色,猛然走出房门,呼喊玉版把儿子抱来,对常大用说:“三年前,我感激你对我的思念,才嫁给你报答你!如今你既然猜疑我,怎么能够再在一起生活!”就和玉版举起孩子远远地抛出去,孩子落在地上一下子不见了。常大用吃惊地看着,两个女子也忽然不见了。常大用悔恨不已。

几天后,孩子落地的地方长出两棵牡丹,一夜间就长到一尺多高。当年就开了花,一棵紫的,一棵白的,花朵大得像盘子,比平常的葛巾、玉版花瓣更加繁茂细碎。几年后,枝繁叶茂,各长成一大片花丛。把花移栽到别的地方,又变成了别的品种,谁也叫不出名字。从此牡丹的繁荣茂盛,洛阳可算是天下无双了。


\section{1.11   卷 十 一}
\label{\detokenize{p00_u5176_u5b83/_u767d_u8bdd_u804a_u658b_u5fd7_u5f02:id423}}

\subsection{1.11.1   冯 木 匠}
\label{\detokenize{p00_u5176_u5b83/_u767d_u8bdd_u804a_u658b_u5fd7_u5f02:id424}}
抚军周有德,要将一座旧王邸改建为部院衙门。工匠们招齐以后,有个叫冯明寰的木匠在里面住宿。

一夜,他刚刚就寝,忽见窗子半开,窗外月光通明,像白天一样。远远望见一堵短墙上立着一只红鸡,正凝目注视间,红鸡已从墙上飞掠下地。一会儿,便有个美丽的少女,从窗子外露出半个身子往屋里窥视。冯木匠怀疑是哪个同行私通的女人,便假装睡着,竖起耳朵听着动静。这时,屋里的人都已睡熟了,冯木匠一下子起了私心,心也怦怦地跳起来,暗暗希望少女误走到自己睡的地方来。不常时间,少女果然从窗子跳进来,径直投入冯木匠的怀里。冯木匠大喜,默不作声,一会事毕,少女自己走了。

从此后,少女每夜必到。冯木匠起初还隐瞒着,后来便问少女是不是找错了人,少女说:“不是的,我敬慕你的为人,所以以身相许。”不久,工程完毕,冯木匠要回去,少女已在旷野中等候。冯居住的村子本来离郡城不远,少女便跟他回到家中。进入家门,家里的人都看不见少女,冯木匠才知道她不是人类。

过了几个月,冯木匠精神疲顿,憔悴不堪。心里越发害怕起来,请来法师镇邪驱赶,还是一点用也没有。一夜,少女盛装来到,对冯木匠说:“缘分都有天数,该来的推也推不走,该去的留也留不住。从此后我和你永别了。”说完便走了。


\subsection{1.11.2   黄 英}
\label{\detokenize{p00_u5176_u5b83/_u767d_u8bdd_u804a_u658b_u5fd7_u5f02:id425}}
顺天人马子才,家里世世代代喜好菊花,到了马子才这辈爱得更深了;只要听说有好品种就一定想法买到它,不怕路远。

一天,有位金陵客人住在他家,说自己的一位表亲有一两种菊花,是北方没有的品种。马子才高兴地动了心,立刻准备行装跟客人到了金陵。客人千方百计为他谋求,才得到两棵幼芽。马子才像得了珍宝似地裹藏起来。

回家路上,子才遇见一个少年,骑着小毛驴,跟随在一辆华丽的车子后面,生得英俊潇洒,落落大方。马子才慢慢来到少年跟前攀谈起来,少年自己说:“姓陶。”言谈文雅。又问起马子才从什么地方来,马子才如实告诉了他。少年说:“菊花品种没有不好的,全在人栽培灌溉。”就同他谈论起种植菊花的技艺来,马子才十分高兴,问:“你要到什么地方去?”少年回答说:“姐姐在金陵住厌了,想到黄河以北找个地方住。”马子才很高兴地说:“我家虽然很穷,但有茅草房可以居住。如果你们不嫌荒陋,就不要再找别的地方了。”陶生快步走到车前同姐姐商量,车里的人掀开帘子说话,原来是个二十来岁的绝世美人,她看着弟弟说:“房屋好坏不在乎,但院子一定要宽敞。”马子才忙替陶生答应了,于是三人一块儿回家。

马家宅子南边有一个荒芜的园子,只有三四间小房,陶生看中了,就在那里住下来。每天到北院,为马子才管理菊花。那些已经枯了的菊花一经他拨出来再种上,没有不活的。陶生家里贫穷,每天和马子才一块吃饭饮酒,而他家似乎从来不烧火做饭。马子才的妻子吕氏,也很喜爱陶生的姐姐,时常拿出一升半升的粮食接济他们。陶生的姐蛆小名叫黄英,很会说话,也常到吕氏的房里同她一块做针线活。

一天,陶生对马子才说:“你家生活本来就不富裕,又添我们两张嘴拖累你们,哪能是长久法子呢?为今之计,卖菊花也足以谋生。”马子才一向耿直,听了陶生的话,很鄙视地说:“我以为你是一个风流高士,能够安于贫困,今天竟说出这样的话,把种菊花的地方作为市场,那是对菊花的侮辱。”陶生笑着说:“自食其力不是贪心,卖花为业不是庸俗;一个人固然不能用不正当的手段来谋利,但也不必去追求贫穷啊。”马子才没有说话,陶生站起来走了。

从这天起,马子才扔掉的残枝劣种,陶生都拾掇回去,也不再到马家吃饭。马子才叫他,他才去一次。不久,菊花将要开放了,马子才听到陶生门前吵吵嚷嚷像市场一样,感到很奇怪,便偷偷地过去瞧,见来陶家买花的人,用车载的、用肩挑的,络绎不绝。所买的花全是奇异的品种,从来没有见过的。马子才心里讨厌陶生贪财,想与他绝交,又恨他私藏良种不让自己知道,就走到他门前叫门,要责备他一顿。陶生出来,拉着他的手进了门,马子才见原来的半亩荒地全种上了菊花,除了那几间房子没有一块空地。挖去菊花的地方,又折下别的枝条插补上了,畦里那些含苞待放的菊花没有一棵不是奇特的品种,仔细辨认一下,全是自己以前拨出来扔掉的。陶生进屋,端出酒菜摆在菊花畦旁边,说:“我因贫穷,不能守清规,连续几天幸而得到一点钱,足够我们醉一通的。”不大一会儿,听房中连连喊叫“三郎”,陶生答应着去了;很快又端来一些好菜,烹饪手艺很高。马子才问:“你姐姐为什么还不嫁人?”陶生回答说:“没到时候。”马子才问:“要到什么时候?”陶生说:“四十三个月。”马子才又追问:“这是什么意思?”陶生光笑,没有说话,直到酒足饭饱,两人才高兴地散了。

过了一宿,马子才又去陶家,看到新插的菊花已经长到一尺多高,非常惊奇,苦苦请求陶生传授种植的技术。陶生说:“这本来就不是能言传的,况且你也不用它谋生,何必学它?”又过了几天,门庭稍微清静些了,陶生就用蒲席把菊花包起来捆好,装载了好几车拉走了。过了年,春天过去一半了,陶生才用车子拉着一些南方的珍奇花卉回来,在城里开了间花店,十天就卖光了,仍旧回来培植菊花。去年从陶生家买菊花保留了花根的,第二年都变成了劣种,就又来找陶生购买。陶生从此一天天富裕起来。头一年增盖了房舍,第二年又建起了高房大屋,他想建什么就建什么,从不和主人商量。慢慢的旧日的花畦,全都盖起了房舍。陶生便在墙外买了一块地,在四周垒起土墙,全部种上菊花。到了秋天,用车拉着花走了,第二年春天过去了也没回来。这时,马子才的妻子生病死了。马子才看中了黄英,就托人向黄英露了点口风,黄英微笑着,看意思好像应允了,只是专等陶生回来罢了。

过了一年多,陶生仍然没有回来,黄英指导仆人栽种菊花,同陶生在家时一样。卖花得的钱就和商人合股做买卖,还在村外买了二十顷良田,宅院修造得更加壮观。

一天,忽然从广东来了一位客人,捎来陶生的一封书信。马子才打开一看,是陶生嘱咐姐姐嫁给马子才。看了看信的日期,正是他妻子死的那天。又回忆起那次在园中饮酒时,到现在正好四十三个月,马子才非常惊奇。便把信给黄英看,询问她聘礼送到什么地方。黄英推辞不收彩礼,又因为马子才的老房太简陋,想让他住进自己的宅子,像招赘女婿一样。马子才不同意,选了个吉庆日子把黄英娶到家里。

黄英嫁给马子才以后,在墙壁上开了个便门通南宅,每天过去督促仆人做活。马子才觉得依靠妻子的财富生活不光彩,常嘱咐黄英南北宅子各立帐目,以防混淆。然而家中所需要的东西,黄英总是从南宅拿来使用。不过半年,家中所有的便全都是陶家的物品了。马子才立刻派人一件一件送回去,并且告诫仆人,不要再拿南宅的东西过来。可不到十天,又混杂了。这样拿来送去好几次,马子才烦恼得很。黄英笑着说:“你如此追求廉洁,不觉太劳心吗?”马子才感到惭愧,便不再过问,一切听黄英的。

黄英于是召集工匠,置备建筑材料,大兴土木。马子才制止不住,只几个月,楼舍连成一片,两座宅子合成一体,再也分不出界线来了。但黄英也听从了马子才的意见,关起门不再培育、出卖菊花,生活享用却超过了富贵大家。马子才心里不安,说:“我清廉自守三十年,被你牵累坏了。如今生活在世上,靠老婆吃饭真是没有一点男子汉大丈夫的气慨,别人都祈祷富有,我却祈求咱们快穷了吧!”黄英说:“我不是贪婪卑鄙的人,只是没有点财富,会让后代人说爱菊花的陶渊明是穷骨头,一百年也不能发迹,所以才给我们的陶公争这口气。但由穷变富很难,由富变穷却容易得很。床头的金钱任凭你挥霍,我决不吝惜。”马子才说:“花费别人的钱财也是很丢人的。”黄英说:“你不愿意富,我又不能穷,没有别的办法,只好同你分开住。这样清高的自己清高。浑浊的自己浑浊,对谁也没有妨害。”就在园子里盖了间茅草屋让马子才住,选了个漂亮的奴婢去侍候他,马子才住得很安心。可是过了几天,就苦苦想念起黄英,叫人去叫她,她不肯来,没有办法只好回去找她。隔一宿去一趟,习以为常了。黄英笑着说:“你东边吃饭西边睡觉,清廉的人不应当是这样的。”马子才自已也笑了,没有话回答,只得又搬回来,同当初一样住到一块了。

一次,马子才因为有事到了金陵,正是菊花盛开的秋天。一天早晨他路过花市,见花市中摆着很多盆菊花,品种奇异美丽。马子才心中一动,怀疑是陶生培育的。不大会儿,花的主人出来,马子才一看果然是陶生。马子才高兴极了,述说起久别后的思念心情,晚上就住在陶生的花铺里。他要陶生一块回家,陶生说:“金陵是我的故土,我要在这里结婚生子。我积攒了一点钱麻烦你捎给我姐姐,我到年底会去你家住几天的。”马子才不听,苦苦地请求他回去,并且说:“家中有幸富裕了,只管在家中坐享清福,不需要再做买卖了。”说过,马子才便坐在花铺里,叫仆人替陶生论花价贱卖,几天就全卖完了,立刻逼着陶生准备行装,租了一条船一块北上了。一进门,见黄英已打扫了一间房子,床榻被褥都准备好了,好像预先知道弟弟回来似的。

陶生回来以后,放下行李就指挥仆人大修亭园。只每天同马子才一块下棋饮酒,再不结交一个朋友。马子才要为他择偶娶妻,陶生推辞不愿意。黄英就派了两个婢女服侍他起居,过了三四年生了一个女孩儿。

陶生一向很能饮酒,从来没有见他喝醉过。马子才有个朋友曾生,酒量也大得没有对手。有一天曾生来到马家,马子才就让他和陶生比赛酒量,两个人放量痛饮,喝得非常痛快,只恨认识太晚。从辰时一直喝到夜里四更天,每人各喝了一百壶,曾生喝得烂醉如泥,沉睡在座位上;陶生起身回房去睡,刚出门踩到菊畦上,一个跟头摔倒,衣服散落一旁,身子立即变成了一株菊花,有一人那么高,开着十几朵花,朵朵都比拳头大。马子才吓坏了,忙去告诉黄英。黄英急忙赶到菊畦。拔出那株菊花放在地上说:“怎么醉成这样了!”她把衣服盖在那株菊花上,让马子才和她一块回去,告诉他不要再来看。天亮以后,马子才和黄英一道来到菊畦,见陶生睡在一旁,马子才这才知道陶家姐弟都是菊精,于是更加敬爱他们。

陶生自从暴露真相以后,饮酒更加豪放,常常亲自写请柬叫曾生来,两人结为莫逆之交。二月十五花节,曾生带着两个仆人,抬着一坛用药浸过的白酒来拜访陶生,约定两人一块把它喝完。一坛酒快喝完了,两人还没多少醉意,马子才又偷偷地拿了一瓶酒倒入坛中。两人喝光后,曾生醉得不醒人事,两个仆人把他背回去了。陶生躺在地上,又变成了菊花。马子才见得多了也不惊慌,就用黄英的办法把他拔出来,守在旁边观察他的变化。待了很长时间,见花叶越来越枯萎,马子才害怕起来,这才去告诉黄英。黄英听了十分吃惊,说:“你杀了我弟弟了!”急忙跑去看那菊花,根株已经干枯了。黄英悲痛欲绝,掐了它的梗,埋在盆中,带回自已房里,每天浇灌它。马子才悔恨欲绝,怨恨曾生。

过了几天,听说曾生已经醉死了。盆中的花梗渐渐萌发,九月就开了花,枝干很短,花是粉色的。嗅它有酒香,起名叫“醉陶”。用酒浇它,就长得更茂盛。后来陶生的女儿长大成人,嫁给了官宦世家。黄英一直到老,也没有什么异常的事情。


\subsection{1.11.3   书 痴}
\label{\detokenize{p00_u5176_u5b83/_u767d_u8bdd_u804a_u658b_u5fd7_u5f02:id426}}
郎玉柱,是彭城人。他的父亲曾做过太守,为官清廉,得到俸禄后,不置田产,酷爱买书,积攒了满满一屋子。到了玉柱,尤其痴:家里非常贫困,东西都卖光了,只有父亲的藏书,一本也不忍卖掉。父亲在时,曾抄录《劝学篇》贴在郎玉柱书桌的右边。玉柱每天都要读上几遍,还罩上层白纱,恐怕磨坏了。玉柱读书倒不是为了做官,而是真的相信书中自有“千钟粟”“黄金屋”,因此昼夜苦读,四季不断。二十多岁了,也不知娶妻,盼望着书中那“颜如玉”的美人自己会来找他。有时亲戚朋友来到家里,他也不知问寒道暖。略说几句话,便又旁若无人地高声读起书来。客人无味,自己坐一会儿就走了。每次科考,学使总是首先选他参加,但却一直考不中。

一天,玉柱又在读书,忽然一阵大风吹来,将书刮跑了。玉柱急忙追赶,一脚踏空,双脚陷进地里。低头一看,见是一个坑,上头盖着层烂草。往下挖了挖,才知原来是古人窖藏粮食的地窖,里面的粮食已经腐烂成粪土了。虽然粮食没法吃,但玉柱更加相信“书中自有千钟粟”的说法确实不错。因此,读书也更加用功。又一天,玉柱爬梯子上书架高处找书,在一堆乱书中发现一个尺把长的小金车,惊喜万分。以为“书中自有黄金屋”的话又应验了。拿出去给人家看了看,原来是镀金的,并不是真金。玉柱沮丧不堪,暗地里埋怨古人欺骗自己。过了不几天,有个跟父亲同榜考中的人,做了本道的观察,此人信佛。有人便劝玉柱将金车献给他作佛龛。观察非常高兴,赐给玉柱三百两银子、两匹马。玉柱大喜,以为“书中车马多如簇、书中自有黄金屋”都应验了,越发刻苦攻读。

玉柱到了三十多岁,有人劝他该娶妻子了。玉柱说:“‘书中自有颜如玉’,我还愁没有漂亮的妻子吗?”又过了两三年,书里仍没出来个美女找他,大家都嘲讽他。这时,民间谣传天上的织女星私奔到了人间。有人和玉柱开玩笑:“织女私逃,大概是为了你吧?”玉柱知道他们是在戏弄自己,也不答理。一晚,读《汉书》读到第八卷,刚到一半的时候,见一个用纱剪成的美人夹在书页中。玉柱大惊道:“书中自有颜如玉,难道就是这个吗?”心里怅然若失。他再细看看那纱剪的美人,眼睛眉毛栩栩如生,脊背上隐隐约约有行小字:“织女。”玉柱十分惊异,天天把美人放到书上,反复观赏,至于废寝忘食。

一天,正在凝视着那纱美人,美人忽然弯弯腰起来了,坐在书上向他微笑。玉柱惊骇万分,忙拜倒在桌下。美人坐起身,已变得有一尺多高。玉柱更加惊疑,连连叩头。美人走下桌子,亭亭玉立,真是艳美无双。玉柱边拜边问:“你是什么神仙?”美人笑着说:“我姓颜,叫如玉,你早就知道我了。承蒙你天天盼着我,我如不来一次,恐怕千年之后没人再相信古人的话了!”玉柱十分高兴,便和她一块睡了;但枕席上虽然亲爱非常,玉柱并不懂男女间那事儿。

此后,玉柱每读书,一定要那女子坐在一边陪着。女子劝他不要再读了,玉柱不听。女子说:“你所以不能飞黄腾达,就是因为只会死读书罢了!试看那些科考中榜的人,有几个是像你这样读书的?你不听我的话,我就走了!”玉柱只得暂时听她的。刚过一会儿,又忘了,照读如旧。过了一霎,再找女子,已经不见。玉柱丧魂失魄,忙跪下祈祷,还是没有踪影。忽然想起女子隐藏的地方,忙拿过《汉书》仔细翻检,果然在原来的地方找到了她。叫也叫不动,便跪下恳求,女子才下来说:“你若再不听,我就永不和你来往了!”于是,让玉柱买来棋盘、纸牌,天天和他游戏。但玉柱的心思一点也不在玩上,瞧见女子不在,就偷来书赶紧浏览几页。恐怕她发觉后再走了,暗将她藏身的《汉书》第八卷混杂在其它书中,让她迷失归路。一天,玉柱又读入了迷,女子进来,他竟一点也没发觉。忽抬头看见她,急忙合上书,女子已消失了。玉柱大为恐慌,搜遍了藏书,也没找到她。最后,还是从《汉书》第八卷中找了出来,连页数都丝毫不错。于是,玉柱再次哀求,发誓决不再读了,女子才从书上下来,跟他下棋,说:“三天内棋还下得不好,我还走!”到了第三天,二人下棋时,玉柱竟然赢了两子,女子才高兴起来。又给他一架琴,限五天弹会一支曲子。玉柱手里弹着,眼睛看着,再也顾不上别的。时间一长,竟也弹得得心应手,自己不觉也兴奋起来。女子天天跟他喝酒、玩耍,玉柱高兴地忘了读书。女子又让他走出家门,多交朋友,从此郎玉柱风流潇洒、多才多艺的名声就远远传开了。女子说:“这下你可以去考试了!”

一天晚上,玉柱对女子说:“凡男女同居到一起,就会生孩子。我和你住了这么长时间,怎么不生呢?”女子笑着说:“你天天读书,我本来就说没用处。现仅夫妇这一章,你就还没明白。枕席之上有功夫!”玉柱惊奇地问:“什么功夫?。女子只是笑,也不说话。过了会儿,暗暗地凑上去,教给玉柱。玉柱快乐至极,说:“没想到夫妇之间还有这种不可言传的快乐!”于是逢人便说,引得人无不掩口而笑。女子知道后责备他,他还不解地说;“钻墙越院偷东西,才不能告诉人;天伦之乐,人人都有,有什么可忌讳的呢?”过了八九个月,女子果然生下个男孩,玉柱便雇了个老妇人抚养着婴儿。

一天,女子突然对玉柱说:“我跟了你两年,已经生了儿子,我们可以分手了。耽搁时间久了,恐怕会给你招祸,那时后悔就晚了!”玉柱听说,流着泪拜倒在地上: “你就不念我们的孩子吗?”女子也十分凄伤。过了很久,说:“你一定要我留下来,就把书架上这些书全扔了。”玉柱不肯,说:“这些书是你的故乡,我的生命,怎么说这种话!”女子不再勉强,说:“我也知道一切都是运数,不得不预先告诉你罢了!”

先前,玉柱的亲属中有人发现了女子,无不惊骇万分。但又没听说他和哪家姑娘结婚,便一起询问他。玉柱不会说假话,只是默默不语,大家更加怀疑。结果这事传遍了各地,也传到了县令史某的耳朵里。史某,是福建人,少年时就考中了进士。听到玉柱家有个美人的消息,动了坏念头,想瞧瞧那女子是什么模样,立即派衙役去捉拿玉柱和女子。女子听说,逃得无影无踪。史县令大怒,将玉柱逮捕下狱,革去功名,严刑拷打,定要他交待出女子的去向。玉柱被打得死去活来,还是不说。县令又拷打丫鬟,丫鬟知道得不多,只说了个大概。史县令便认为那女子是妖怪,骑着马亲自赶到玉柱家捉拿。见满屋子都是书,多得无法搜查,县令便命放火烧书。浓烟滚滚,凝聚在院子上方,像乌云一样,久久不散。玉柱被释放后,到远方去求了一个父亲的门人帮忙,才得以恢复了功名。这年考中了举人,第二年又中了进士。玉柱对史县令恨入骨髓,立起了颜如玉的牌位,天天祷告说:“你如有灵,就保佑我到福建做官!”后来他果然被朝廷任命为巡按,到福建视察。过了三个月,访查到史县令在老家的劣迹,便抄了他全家。当时,玉柱有个表兄弟是法官,逼着他娶了个妾,假说是买的婢女,寄居在玉柱的官衙里。这件案子一了结,玉柱于当天就辞职,带着爱妾返回了老家。


\subsection{1.11.4   齐 天 大 圣}
\label{\detokenize{p00_u5176_u5b83/_u767d_u8bdd_u804a_u658b_u5fd7_u5f02:id427}}
许盛,是兖州人,跟着哥哥许成在福建做买卖,货物一直没有购全。有个人说大圣最灵验,要去圣庙祈祷。许盛不知大圣是什么神灵,便也和哥哥一起前往。到了大圣庙,只见殿台楼阁,连绵不断,极其弘大壮丽。来到大殿中瞻仰神像,见是猴头人身,原来是齐天大圣孙悟空。大家肃然起敬,没有一个敢怠慢的。许盛一向刚直,脾气倔强,见此情景,心里暗笑世风习俗竟如此鄙陋!别人都在焚香奠酒,叩头祷告,他却偷偷地溜了。

从圣庙回来后,哥哥责备许盛怠慢神灵,许盛不屑地说:“孙悟空不过是丘处机笔下的寓言人物,怎么就这样虔诚地信奉他?如果他真有神灵,刀劈雷打,我自己承担了!”旅店主人听他出言不逊,直呼大圣姓名,一个个都脸上变色,一个劲地摆手,像是恐怕大圣听到。许盛见此情景,越发大声嚷起来,吓得人们赶紧捂着耳朵跑开了。到了夜晚,许盛果然得病,头疼得要死。有人劝他快去大圣庙祷告,许盛不听。不一会儿,头疼好了,大腿又疼,竟然当夜生了一个大疮,连脚都肿了,疼得没法吃饭睡觉。哥哥替他祷告,也没有一点效验。有人说:“这是神灵责罚,要自己祷告才行。”许盛还是不信。过了一个多月,腿上的疮渐渐好了;却又生了个疮,比前番加倍痛苦。请来医生,用刀割掉烂肉,鲜血直流,淌了满满一碗。许盛恐怕人们将所谓神灵责罚一事传得神乎其神,故意咬牙忍住疼痛,一声不吭。又过了一月多,自己的疮刚开始好转,哥哥又大病。许盛说:“怎么样?你这敬过神的人也这样,足以说明我的病不是因为孙悟空而起的。”哥哥听他这样说话,更加气愤,说这是神灵迁怒到自已身上,责骂弟弟不替他祈祷。许盛拧着脖子说:“兄弟之间犹如手足。前些天我自己身上肉都烂了,我还不祈祷;现在怎能因为‘手足’病了,就让我改变操守呢?”坚决不同意向大圣祷告,只是请来医生,为哥哥开了付药。没想到药一吃下,哥哥突然死了。许盛悲哀痛苦,愤不欲生。买来棺材,将哥哥的尸体敛好后,直奔到大圣庙,指着神像斥责道:“我哥哥生病,说是你迁怒于他,让我有口难言。假使你真有神灵,就让我死去的哥哥再活过来,我就心甘情愿给你当弟子,不敢再说别的。否则,别怪我拿你处置‘三清’的办法处治你,也消除我哥哥在九泉之下的疑惑!”

到了夜晚,许盛梦见一人招呼他跟着走,进入大圣庙中,仰头看见大圣脸上有怒色,责备许盛说:“我因为你对我无礼,用菩萨刀扎穿你的大腿以示惩罚,你还不悔悟,仍在胡言乱语!本应当把你送到拔舌狱中,念你一生刚正梗直,姑且先饶了你。你哥哥的病,是你请庸医害死的,跟别人有什么关系?现在我若不稍施法力让他活过来,更使你们这些狂妄之徒有话说了。”于是,命一青衣使者前去通知阎王。使者说:“人死三天后,鬼名籍已报送天庭,恐怕不好办了。”大圣便取出一块方板,提起笔来不知写了些什么,命使者拿着前往。过了很久使者才返回,许成在后面跟着,一块跪到大堂上。大圣问道:“为什么这样迟?”青衣使者回答说:“阎王不敢做主,又拿大圣的旨意请示了南、北斗星,所以来迟。”许盛见哥哥果真回来,赶紧快步走上前去,叩谢太圣神恩。大圣说:“快和你哥哥回去吧。今后如能回心向善,我就替你降福。”兄弟二人悲喜交集,互相搀扶着往回赶来。

许盛梦中忽然惊醒,想想梦中的经历,深感惊异。急忙打开棺材看看,哥哥果然已经苏醒,便扶了出来,心中十分感激大圣神力。从此后,许盛诚心诚意地信奉大圣,比其他人还要虔诚。

兄弟二人分别生了那场病,经商的资本已耗去了一半。加以许成身体还没有完全康复,二人相对长愁。一天,许盛偶然在城外走走,忽然一穿褐色衣服的人端详端详他说:“你有什么愁事啊?”许盛正没个诉说的地方,便对那人详细讲述了自己的遭遇。褐衣人说:“有处风景很美的地方,我们去游览游览,能够解忧驱闷。”许盛问:“什么地方?” 那人只是说不远。许盛跟着他,出城约半里路,那人说:“我有个小小的法术,能让我们片刻就到。”让许盛抱住他的腰,褐衣人微微点了点头,许盛只觉脚下涌起了云彩,身子腾空而起,瞬间便不知飞到了哪里。许盛十分害怕,紧闭着双眼。刚一会儿,那人就说:“到了。”许盛睁眼一看,一片琉璃世界,光华万丈,色彩斑斓。惊讶地问:“这是哪里?”回答道:“是天宫。”两人信步而行,越往上走越高。远远望见一个老翁走来,褐衣人喜悦地说:“正碰上这个老头,真是你的福气!”便与老翁互相作揖拜见。老翁请二人到他的住所,煮茶献客,只斟两盏。褐衣人说:“这位是我的弟子,千里跋涉做买卖的,现在来到仙府,恳求多少表示表示。”老翁便命童儿捧出一盘白色的石子,形状像鸟蛋,晶莹透澜,清澈如冰,让许盛自己拿。许盛想,这玩意倒可以拿回去作酒筹子,于是取了六枚。褐衣人觉得许盛太小气,又拿了六枚,交给许盛一块包好,嘱咐收到钱袋中。向老翁拱拱手说:“足够了。”便告辞出来,仍让许盛抱着腰,从天上飞下来,片刻便到了地面。许盛拜问仙号,褐衣人笑着说:“刚才我的小法术,就是所谓的筋斗云。”许盛恍然大悟,明白是齐天大圣,忙恳求保佑自己。大圣说:“我们刚才碰到的是财星,他已赐你十二分利钱,你还求什么呢。”许盛赶紧叩拜,起身一看,大圣已渺无人影了。

回来后,许盛欢喜地把事情告诉哥哥,解开钱袋一块探视,石子已经融在里面了。后来运货物回去,赚了数倍的利钱。从此后,许盛每到福建,必定前去祈祷大圣;别人的祷告,有时还不灵,许盛的祈祷则是有求必应。


\subsection{1.11.5   青 蛙 神}
\label{\detokenize{p00_u5176_u5b83/_u767d_u8bdd_u804a_u658b_u5fd7_u5f02:id428}}
南方长江、汉水一带,民间信奉青蛙神最虔诚。蛙神祠中的青蛙不知有几千几百万,其中有像蒸笼那样大的。有人如触犯了神,家里就会出现奇异的征兆:青蛙在桌子、床上爬来槌去,甚至爬到滑溜溜的墙壁上而不掉下来,种种不一。一旦出现这种征兆,就预示着这家要有凶事。人们便会十分恐惧,赶忙宰杀牲畜,到神祠里祷告,神一喜就没事了。

湖北有个叫薛昆生的,自幼聪明,容貌俊美。六七岁时,有个穿青衣的老太太来到他家,自称是青蛙神的使者,来传达蛙神的旨意:愿意把女儿下嫁给昆生。薛昆生的父亲为人朴实厚道,心里很不乐意,便推辞说儿子还太小。但是,虽然拒绝了蛙神的许亲,却也没敢立即给儿子提别的亲事。又过了几年,昆生渐渐长大了,薛翁便与姜家订了亲。蛙神告诉姜家说:“薛昆生是我的女婿,你们怎敢染指!”姜家害怕,忙退回了薛家的彩礼。薛翁非常担忧,备下祭品,到蛙神祠中祈祷,自己说实在不敢和神灵做亲家。刚祷告完,就见酒菜中浮出一层巨蛆,在杯盘里蠢蠢蠕动着。薛翁忙倒掉酒肴,谢罪后返回家中,内心更加恐惧,只好听之任之。

一天,昆生外出,路上迎面来了一个使者,向他宣读神旨,苦苦邀请他去一趟。昆生迫不得已,只得跟那使者前去。进入一座红漆大门,只见楼阁华美。有个老翁坐在堂屋里,像有七八十岁的样子。昆生拜伏在地,老翁命扶他起来,在桌旁赐座坐下。一会儿,奴婢、婆子都跑了来看昆生,乱纷纷地挤满了堂屋两侧。老翁对她们说:“进去说一声薛郎来了!”几个奴婢忙奔了去。不长时间,便见一个老太太领着个少女出来,约十六七岁,艳丽无比。老翁指着少女对昆生说:“这是我女儿十娘。我觉得她和你可称得上是很美满的一对,你父亲却因她不是同类而拒绝。这是你的百年大事,你父母只能做一半主,主要还是看你的意思。”昆生目不转睛地盯着十娘,心里非常喜爱,话也忘说了。老太太跟他说:“我本来就知道薛郎很愿意。你暂且先回去,我随后就把十娘送去。”昆生答应说:“好吧。”告辞出来,急忙跑回家,告诉了父亲。薛翁仓猝间想不出别的办法,便教给儿子话,让儿子快回去谢绝。昆生不愿意,父子正在争执时,送亲的车辆已到了门口,成群的青衣丫鬟簇拥着十娘走了进来。十娘走进堂屋拜见公婆。薛翁夫妇见十娘十分漂亮,不觉都喜欢上了她。当晚,昆生、十娘便成了亲,小夫妻恩恩爱爱,感情密切。

从此后,神女的父母时常降临昆生家。看他们的衣着,只要穿的是红色衣服,就预示薛家将有喜事;穿白色衣服,薛家就会发财,非常灵验。因此,薛家日渐兴旺起来。只是自与神女结婚后,家里门口、堂屋、篱笆、厕所,到处都是青蛙。家里的人没一个敢骂或用脚踏的。昆生年轻任性,高兴的时候对青蛙还有所爱惜,发怒时则随意践踏,毫无顾忌。十娘虽然谦谨温顺,但生性好怒,很不满意昆生的这些所作所为,昆生仍不看在十娘的份上有所收敛。一次十娘忍耐不住,骂了他两句,昆生发怒,说:“你仗着你爹娘能祸害人吗?大丈夫岂能怕青蛙!”十娘最忌讳说“蛙”字,听了昆生的话,非常气愤,说:“自从我进了你家的家门,使你们地里多产粮食,买卖多挣银子,也不少了。现在老老少少都吃得饱穿得暖,就要猫头鹰长翅膀,要吃母亲的眼睛吗!”昆生愈怒,骂道:“我正厌恶你带来的这些东西太肮脏,不好意思传给子孙!我们不如早点分手!”将十娘赶了出去。昆生的父母听说后,急忙跑来,十娘已走了。便斥骂昆生,让他快去追回十娘。昆生正在气头上,坚决不去。到了夜晚,昆生和母亲突然生病,烦闷闷地不想吃饭。薛翁害怕,到神祠中负荆请罪,言词恳切。过了三天,母子的病便好了。十娘也自已回来了。从此夫妻和好,跟以前一样。

十娘不好操持女红,天天盛妆端坐,昆生的衣服鞋帽,全都推给婆母做。一天,昆生母亲生气地说:“儿子已娶了媳妇,还来累他妈!人家都是媳妇伺候婆婆,咱家却是婆婆伺候媳妇!”这话正好让十娘听见了,便赌气走进堂屋。质问婆母:“媳妇早上伺候您吃饭,晚上伺候您睡觉,还有哪些侍奉婆婆的事没做到?所缺的,是不能省下雇佣人的钱,自己找苦受罢了!”母亲哑然无言,既惭愧又伤心,禁不住哭了起来。昆生进来,见母亲脸上有泪痕,问知缘故,愤怒地去责骂十娘,十娘也毫不相让地争辩。昆生怒不可遏,说:“娶了妻子不能伺候母亲高兴,不如没有!拚上触怒那老青蛙,也不过遭横祸一死罢了!”又赶十娘走。十娘也动了怒,出门径自走了。

第二天,薛家便遭了火灾,烧了好几间屋子,桌子床榻,全成灰烬。昆生大怒,跑到神祠斥责说:“养的女儿不侍奉公婆,一点家教都没有,还一味护短!神灵都是最公正的,有教人怕老婆的吗?况且,吵架打骂,都是我一人干的,跟父母有什么关系!刀砍斧剁,我一人承担,如不然,我也烧了你的老窝,作为报答!”说完,搬来柴禾堆到大殿下,就要点火。村里的人忙都跑来哀求他,昆生才愤愤地回了家。父母听说后,大惊失色。到了夜晚,蛙神给邻村里的人托梦,让他们为女婿家重盖房子。天明后,邻村的人拉来木材,找来工匠,一起为昆生造屋,昆生一家怎么也推辞不了。每天有数百人络绎不绝地前来帮忙,不几天,全家房屋便焕然一新,连床榻、帷帐等器具都给准备下了。刚整理完毕,十娘也回来了。到堂屋里给婆母赔不是,言辞十分温顺。转身又朝昆生陪了个笑脸,于是全家化怨为喜。此后,十娘更加和气,连续两年没再闹别扭。

十娘生性最厌恶蛇。一次,昆生开玩笑般地把一条小蛇装到一只木匣里,骗十娘打开看看。十娘打开一看,吓得脸上失色,斥骂昆生。昆生也转笑为怒,恶语相加。十娘说:“这次用不着你赶我了!从此后我们一刀两断!”径直出门走了。薛翁大为恐惧,将昆生怒打一顿,到神祠里请罪。幸而这次没什么灾祸,十娘也寂然没有音讯。

过了一年多,昆生想念十娘,很是后悔。偷偷跑到神祠里哀恳她回来,还是没有回音。不长时间,听说蛙神又将十娘改嫁给了袁家,昆生大失所望,便也向别的人家提亲。但连相看了好几家,没有一个能比得上十娘的,于是更加想念她。去袁家看了看,见房屋一新,就等着十娘来了。昆生越发悔恨不已,不吃不喝,生起病来。父母忧虑着急,不知怎么办才好。昆生正在昏迷中,听有人抚摸着自己说:“大丈夫常要和我决裂,怎么又作出这种样子!”睁眼一看,竟是十娘!昆生大喜,一跃而起,说:“你怎么来了?”十娘说:“要按你以前对待我的那样,我就应该听从父命,改嫁他人。本来很早就接受了袁家的彩礼,但我千思万想不忍心舍下你。婚期就在今晚,父亲没脸跟袁家反悔,我只好自己拿着彩礼退给了袁家。刚才从家里来,父亲送我说:‘痴丫头!不听我的话,今后再受薛家欺凌虐待,死了也别回来了!’”昆生感激她的情义,不禁痛哭流涕。家里人都高兴万分,赶紧跑了去告诉薛翁。婆母听说后,等不及十娘去拜见她,忙跑到儿子屋里,拉着十娘的手哭泣起来。

从此后,昆生变得老成起来,再也不恶作剧了。夫妻二人感情更加深厚。一天,十娘对昆生说:“我过去因为你太轻薄,担心我们未必能白头到老,所以不敢生下个后代留在人世。现在可以了,我马上要生儿子了!”不长时间,十娘父母穿着红袍降临薛家。第二天,十娘临产,一胎生下两个儿子。此后便跟蛙神家来往不断。居民有时触犯了蛙神,总是先求昆生;再让妇女们穿着盛装进入卧室,朝拜十娘。只要十娘一笑,灾祸就化解了。薛家的后裔非常多,人们给起名叫“薛蛙子家”。附近的人不敢叫,远方的人才这样称呼。

又:青蛙神,往往借巫的嘴说话。巫能察知神的喜怒。巫如告诉信士们说:“神喜欢了!”那么福气就来了;如说:“神发怒了!”那么一家人都呆呆地坐着,忧愁叹息,至于有吃不下饭去的。是习俗就是如此呢,还是青蛙神确实神灵,并非完全虚妄呢?

有个姓周的富裕商人,生性吝啬。正赶上本地的人募资修建关圣祠,不论穷人富人,都乐意出钱出力,唯独周某一毛不拔。过了很久。因为募的钱不够用,关圣祠仍没建好,领头的人一筹莫展。一次,众人正祭祀青蛙神,神忽然附在巫身上说话了:“关圣驾前的周仓将军命小神掌管募资事宜,快给我取帐簿来!”众人忙把帐簿递上去。巫说:“已捐资的人,不再勉强;还没有捐的,自己量力注明要捐的数目!”众人唯唯听命,分别写上了自己要捐的银两数。最后,巫看着众人问:“周某在这里吗?”周某正混在人群后面,恐怕蛙神知道自己来了。这时听到巫的问话,大惊失色,不敢不答应,极不情愿地挪动着脚步走到前面。巫指着帐簿说:“你写上捐一百两!”周某不肯。巫发怒地说:“淫债你都付出二百两,况且这是好事呢!”原来,周某曾跟一个妇人私通,被她丈夫当场抓住,他便交出了二百两银子赎罪。所以蛙神现在故意揭他这件丑事。周某既羞惭又恐惧,迫不得已,只得注上了捐一百两银子。

周某同家后,把这事告诉了妻子。妻子说:“这是巫在敲诈你!”此后,巫多次登门索要银两,周某总是不给。一天,周某正白天躺着休息,忽听门外传来牛喘一样的声音。抬头一看,是一只巨大的青蛙,房门刚好容得下它的身子,蠢蠢地爬动着,从两扇门当中硬挤进了屋里。然后转过身去,把下巴颏搁到门槛上。周某一家人都惊恐不安。周某说:“这定是来讨募金。”便烧上香祷告,愿先交三十两,余下的以后再送上,青蛙一动没动。周某又说先交五十两,青蛙身子忽然一缩,小了一尺多;周某又加上二十,青蛙再次缩得跟斗一样大。周某说愿全部交上,青蛙才缩得跟一只拳头那么大,慢慢腾腾地爬出去,钻进墙缝走了。周某急忙拿了五十两银子,送到监造关圣祠的地方。人们见铁公鸡竟拔了毛,都感到惊异,周某也不说原因。

过了几天,巫又说:“周某还欠五十两银子,为什么不赶快催他交齐!”周某听说后害怕,只得又送了十两,想就此完结。一天,周某夫妇正吃着饭,那只大青蛙又来了,跟前次一样爬到屋里,眼睛瞪得大大的,像在发怒。一会儿,巨蛙又爬到床上去,把床摇晃得像要翻了一样,把嘴巴搁在枕头上睡起觉来。肚子高高地鼓起,像头卧牛,把四个墙角都塞满了。周某十分恐惧,只得又拿出四十两银子,凑足一百之数。但看看床上的青蛙,一动没动。没出半天,小青蛙群渐渐聚集而来。第二天,青蛙更多,粮仓、床上到处都是。比碗还大的青蛙,跳到炉灶上吃苍蝇。死苍蝇纷纷落到饭锅里,然后靡烂,把饭搞得污秽不堪,没法再吃。到第三天,连院子里都挤满了青蛙,一点空隙都没有了。周某一家人惊慌失措,迫不得已,去请教巫。巫说:“这肯定是嫌银子少。”周某听说,便烧上香祷告,愿在一百两之外,再加二十两,床上的巨蛙才抬起了头;又加了些,巨蛙抬起一只脚;直至又增到一百两,巨蛙才挪动四脚,下床爬出门去。但刚笨拙得爬了几步。又返回来卧在门内。周某害怕,问巫是怎么回事。巫揣摩它的意思,是要周某现在就交钱。周某无可奈何,如数拿出银子交给了巫,巨蛙才走了。几步之外,巨蛙的身子忽地猛缩,杂在蛙群中,再也辨认不出来。蛙群也乱纷纷地渐渐散了。

关圣祠建成后,举行落成仪式,又需要费用。巫忽然指着一个领头的说:“你应该出若干两银子!”领头的共十五人,除两人之外,都被巫点了名捐银。这些领头的指了指那两个没被点名的人说:“我们和他们二人都已捐过了。”巫说:“我并不是因为你们比他们二人富有,才再让你们捐钱;而是按你们侵吞的银两数来决定捐钱多少的。这些银子都是从众人身上募集来的,你们不能贪污自肥,恐怕以后会有横灾。念你们领头建祠,十分辛苦,所以让你们捐出私吞的银两,以替你们消灾。除他们二人廉洁正直,没有参与,可以免了外,就是我的家巫,我也不包庇他。就让他先拿出银两,给大家带个头!”巫说完,飞跑进家,翻箱倒柜。妻子问他,也不回答,把家里的银子尽数拿了来,告诉众人说:“我这个家巫私自克扣银子八两,现在让他倾囊赔偿。”大家把银子称了称,只有六两多,巫便让人记下欠数。大家见此情景非常惊愕,再不敢争辩,全部如数交清了银两。巫醒过来后,自己茫然不知这件事。有人告诉他经过。巫十分羞惭,忙当了衣服凑足了应交的数目。其中只有两个人没有交齐,结果一个病了一个多月,另一人生了个大疮。花的医药费用远远超过了他们欠下没交的钱。人们都说这是侵吞捐银的报应。


\subsection{1.11.6   任 秀}
\label{\detokenize{p00_u5176_u5b83/_u767d_u8bdd_u804a_u658b_u5fd7_u5f02:id429}}
山东鱼台人任建之,以贩毛毡和皮大衣为生。他把所有的本钱都带上到陕西去。路上遇到一个人,自称申竹亭,江苏省宿迁县人。二人谈得挺投机,拜了把兄弟,好得一步也不离。

到了陕西,任建之病倒了,申竹亭细心照顾他。十多天后,病情加重,任建之对他说:“我家没多少财产,八口人的生活来源全靠我跑外做买卖,如今我不幸得了这个病,这把骨头怕是要扔在异乡了。在这离家两千多里的地方,除了你,我的亲兄弟,我还依靠谁?包袱里二百多两银子,你拿一半,除了给我置办棺材什么的,剩下的做你的路费;另一半烦你寄给我妻子,好叫她雇辆车把我运回去。若是兄弟你肯亲自把我送回家,那么所需的费用全在我那一份里出就是了。”说完就在枕头上写了给妻子的信,交给申竹亭,晚上就死了。

申竹亭只用了五六两银子买了口薄皮棺材装殓任建之。店主人催他赶紧运走,他借口去找和尚道士来给亡友做道场,一去不回。任家一年后才得到确信。任建之的儿子叫任秀,十七岁,正念书呢,听到父亲的死讯,要去陕西找回父亲的灵柩。母亲因他年纪太小,不舍得叫他去,他哭得死去活来,母亲这才同意。变卖了东西给他准备路费,派老仆人和他一块儿去,半年才回来。出殡后,家里一贫如洗。幸亏任秀聪明,满了服,考中了本县的秀才。可惜这孩子性情放荡,又爱赌博,母亲虽然严加管教,只是不改。一次主考官前来主考科试,他只考了四等,母亲气得哭,饭也吃不下。他又惭愧又害怕,发誓好好念书。闭门读了一年,终于考了优等,并开始享受国家供给的衣物食品。母亲劝他收几个学生,教学,可是人们了解他过去的行为,不相信他,讥讽他,书也没教成。

任秀有个表叔,姓张,在北京经商,愿意带他进京,并且不要他的路费,任秀很高兴,就跟表叔坐船上了路。到了临清地界,船停泊在城西关。正值好多运盐的船也停在那里,帆呀樯呀像树林。睡下以后,水声人声闹得他睡不着。更深夜静,忽然听见邻船上有掷骰子声,叮叮当当,清脆悦耳,牵动人心,任秀的手不禁痒痒起来。听听同船人都睡熟了,他摸摸包中的一千文钱,很想过船玩一玩。便轻轻起来解开包袱,拿起钱,但想起母亲的教导又犹豫了!便把钱包好睡下,心里终究不安定,还是睡不着。又起来,又解包袱。这样折腾了三次,终于忍不住了,带着钱上了邻船,见两个人正对赌,赌注很大。他把钱放在桌上,要求入局,那两人表示欢迎,就一起掷起骰子来。一会儿,任秀大胜。两人中的一个钱输光了,便把大块银子给船主人做抵押,换来零钱,又赌。后来又下了十几贯钱的注,想孤注一掷。正赌得起劲,又来了一个人,看了半天,也拿出所有的钱入了赌局。任秀的表叔半夜醒来,发觉任秀不在船上,听见骰子声,知道他准去赌博了,就到了邻船上,打算阻止他,一看任秀腿边上的钱堆积如山,就不说什么,背了好几千钱回船,把同船的几位客人都喊起来和他一块儿去运钱,运了好几趟,还剩下十几千钱没运完。一会儿,邻船的三个客人全败了,那船上再也没有钱了,三个客人要赌银子;可是任秀已经没了赌兴,借口只赌钱不赌银子,表叔又一个劲地催他别赌了,回船睡觉。三个客人输急了眼,船主人又贪恋赌客给小费,希望继续赌下去,就主动地到别的船上借来了很多钱。三个客人有了钱,赌得更欢了,不一会儿,又都成了任秀的。这时天已亮了,临清码头放早班开船了,任秀和表叔以及同船客人一起把赢的钱运到自己船上,三个客人也散去了。

邻船主人看看做抵押的二百多两银子,全是上坟的纸锭烧的灰,大惊,找到任秀船上,打算叫任秀赔偿他的损失。一问姓名、住处,才知是任建之的儿子,只好缩起脖,红着脸退回去了。原来这位船主人就是申竹亭。任秀当年去陕西找父亲灵柩时,也听说过;今天,鬼已经给了他报应,也就不再追究他以往的过错了。任秀跟表叔合资到北边做生意,到年底赚了几倍的利。不久,根据常例,被擢为监生,任秀也更会算经济帐了,十年间,成了那一方的首富。


\subsection{1.11.7   晚 霞}
\label{\detokenize{p00_u5176_u5b83/_u767d_u8bdd_u804a_u658b_u5fd7_u5f02:id430}}
五月五日端午节,吴越之地有斗龙舟的民间游戏。人们砍伐树木,把船做成龙的样子,龙身绘上鳞甲,装饰得金碧辉煌,上部有雕栋朱槛,所挂的船帆旌旗全部使用锦绣。船的末端是龙尾,高达丈余,上空悬一木板,用布绳牢牢系住。游戏时,一个男孩在木板上翻滚倒立,表演各种技巧。木板下是滚滚江水,稍不小心,便有掉落水中的危险。男孩是买来的,买时便告知了他父母,然后预先调教训练,如果堕落水中淹死,莫要后悔。而吴门一带,则是使用美女表演的。

镇江有个姓蒋的男孩叫阿端,刚七岁,聪明伶俐,敏捷灵活,同岁儿童中,没有能超过他的,于是,他身价倍增。十六岁了还操此艺,船到金山脚下失足掉下江中溺死了。蒋母就阿端一个儿子,听说儿子死讯,哭得死去活来。

阿端并不知道自己已死,觉得有两个人引着他走去,见水中别有天地;回头一瞧,身后波流回旋,像石壁直立。一会儿走进一座宫殿。见一人戴头盔坐着,这时,一旁走出两个人,对阿端说:“这位便是龙窝君。”就催着阿端下拜。龙窝君面色和蔼,吩咐那两个说:“阿端的技巧不错,可让他到柳条部去。”二人将阿端引到一个处所,内里殿堂宽广,庭院方正。走上东廊后,出来几个少年,向阿端行礼,看上去大都是十三四岁。不一刻,走出一位老婆婆,众少年见了,忙呼“解姥姥”。解姥姥应了,令阿端当场献技。阿端便使出浑身解数,为解姥姥表演了一场。完了,解姥姥又教给阿端钱塘飞霆之舞,洞庭和风之乐。只听见鼓钲声聒耳,各院均响。随后各院都平静了。但解姥姥怕阿端不能很快熟悉舞乐,又絮絮叨叨地调教阿端;而阿端只需一遍,就清楚明白了。解姥姥高兴地说:“这孩子性灵,决不在晚霞以下!”

第二天,龙窝君巡视各部,各部群集在大殿前。龙窝君首先巡视夜叉部,均是鬼脸,穿鱼服。这时,鼓钲敲响,那大钲周长足有四尺多;鼓也可四个人合抱,声音就像是巨雷轰响。接着,部属又跳起舞来,人动水动,霎时,波涛光涌,横流星空,那浪竟击落了一颗天星,坠下地陨灭了。龙窝君见了,忙命停住,命乳莺部进见。

乳莺部一色年轻貌美的丽人,只听见笙乐之声奏起,清风习习,适才还喧嚣无比的河底,顿时波平声息,水渐渐地凝成水晶般的世界,上上下下一片明亮。一曲舞毕,燕子部依次进来——原来尽是少年儿女。其中有一位十四五岁模样的姑娘,拂袖低头,跳散花舞。她舞步轻盈,翩翩如飞,袖中衣下抖出五色花朵,随风扬下,飘洒了一庭院。乐声住后,姑娘跟着她的燕子部立在西边丹墀。阿端忍不住斜视了姑娘一眼,心中不禁生出喜爱之情,他悄悄向燕子部的人打听姑娘姓名,知道她就是解姥姥说的晚霞。不一会儿,又叫柳条部上前。龙窝君要特地试试阿端的舞艺。阿端上前拜过,大大方方地跳了起来,他忽为柳条沐风,舞姿柔软多变;忽如金刚力鼎,身架力量贯注,节奏有序,舞步合折。龙窝君大喜,极力夸奖阿端聪慧灵悟,赐给他诸多宝物。阿端谢过,和众部下堂来到西边丹墀,阿端在人群中远远地去看晚霞,却见晚霞也在往他这边瞄。停了一会儿,阿端徘徊着向部北端靠,晚霞也渐渐地趸出来向南挨近,尽管相隔咫尺,却因法度威严而不敢走出部伍一步,两人只是四目传神,暗送秋波而已。待蛱蝶部巡察完毕后,各部鱼贯而出。柳条部跟在燕子部后,阿端急忙走到部前,而晚霞也有意落在部后。她回头脉脉含情地看了眼阿端,故意丢下一支珊瑚钗。阿端手疾眼快,俯身拾起藏在袖中。回去后,他想念晚霞,竟然患了病,不思茶饭,夜难成寐。解姥姥心疼他,派人送来好吃的,她自己也每天来看望三四次,殷切安抚,阿端的病仍不见好转。解姥姥深深为阿端忧虑,却无任何办法,只好叹道:“眼看吴江王寿辰已近,阿端病未痊愈,这可怎么办好?”

到天将黑时,一个男孩子前来,坐在阿端床上和他搭讪。那孩子说他是蛱蝶部的人,又直截了当地问阿端道:“你是为晚霞生的病吧?”阿端不由惊问:“你怎么知道的?”男孩笑说:“晚霞也和你一个样子噢!”阿端听了,神色凄然地撑起身来,问男孩自己该怎么办好。那男孩问阿端:“你现在能走路么?”阿端说:“勉强能支撑着走。”

男孩便搀扶着他出来,向南打开一扇门,进去后,又折向西,再进一门。只见眼前豁然开朗,面前有好几十亩莲花,奇怪的是这些莲花竟长在平地上,瓣叶像床席一般大,花大如盖,地上堆的花瓣有一尺厚。男孩将阿端引进来后,对他说了声:“你先在这儿等着。”就走了。没多久,一位美人拨开莲花进来,阿端凝神一看,正是晚霞。两人相见,分外惊喜,彼此倾诉了相思之情,各自又叙述了家世。末了,他们用石头压住硕大的荷叶,以作遮蔽,又将荷花瓣铺在地上,然后躺在其中亲热睡觉。离别时,两人约定每天黄昏时相见,这才依依不舍地告别而去。阿端回来后,病也好了。从那以后,两人每天一次在荷花地里相会。

几天后,各部随同龙窝君去吴江王处祝寿。寿庆完毕,各部全都返回,只留下晚霞和乳莺部的一个人在宫中教舞,几个月没有一点消息。阿端不禁怅然若失,整天无精打彩的。一天他偶然得知解姥姥每天来往于吴江府,不由一阵狂喜,便去见解姥姥,假说晚霞是他的表妹,请求解姥姥带他去见见晚霞。解姥姥答应了。到吴江府后,因宫禁森严,晚霞无法出来与阿端见面,阿端只好闷闷不乐地回来。这样又过了一个多月,阿端只觉得度日如年,想晚霞几乎到了痴狂的程度。

一天,解姥姥来了,哭着对阿端说:“真可惜啊!晚霞投江死了!”阿端大惊,眼泪唰唰流了下来。他踩坏了冠帽,又撕破了衣服,将金珠藏在怀中冲出门,想要随晚霞一道去死。但是那江水如石壁般坚硬,凭他怎么用头去撞也进不去。他正想再回来,又怕人再问起官服,增重罪责。正在通身大汗,彷徨犹豫间,忽然看见壁下面有一株大树,便灵机一动,攀援而上,快到树梢时,他使出全身力气,猛地跳下,连衣服也没有沾湿,就已浮到了水面之上。在这一瞬间,阿端恍恍惚惚就如看见了人世,随即顺水漂流向岸边游去。不一会儿,阿端终于游到岸边。在江边坐着休息了一会儿,突然想念起家中老母,便乘着一叶小船前往家乡。抵达乡间时,他四面打量村中房舍,恍然有隔世之感。到家后,忽然听见窗中有女子说话的声音:“你儿子来了!”那声音听上去格外耳熟,极像晚霞。片刻,一女子与阿端母亲一同迎了出来,阿端定晴一看果然是晚霞。两个有情人见面,高兴得忘了悲哀,而阿端母亲却是又悲又疑又惊又喜,均合作一处了。

当初,晚霞在吴江,突然觉得肚子里有了动静。宫中法规森严,她担心生下孩子,会被狠狠鞭笞,再加上与阿端见不得面,便只求一死,便投了江。投江后不久,她的身体浮出水面,被一只客船的人救起。人家问她是哪里人,家在何方。晚霞原是吴地的名妓,投水没有死,妓院又不能再去,便告诉人家说镇江蒋氏是他的夫婿,那人便掏钱为她租了条船,将她送到蒋家。阿端母亲怀疑她认错了人;晚霞却一口咬定没有说错,并将详情细细告诉了阿端母亲。老婆婆爱晚霞丰艳美丽,待她极好,只是担心她年纪轻,未必肯终身寡居。晚霞却孝顺谨慎,见家中贫穷,便将所戴珍奇宝饰变卖了,得了几万钱。阿端母亲看她并无二念,这才放下心来。阿端母亲担心儿子不在,儿媳一旦产下孩子,会被乡邻笑话。晚霞说:“只要得到真孙子,何必怕人知道?”阿端母亲听了,想想也是,便安下心来。这时恰逢阿端回家,晚霞怎能不高兴?阿端母亲却怀疑儿子并没有死,趁夜间偷偷地挖开儿子的坟冢,见骨骸仍在;回去又细问儿子。阿端才恍然知道自己已经死了。怕晚霞知道自己不是人后会厌恶,遂叮嘱母亲别再说了。阿端母亲又告知邻里,说当年得到的并不是儿子的尸体。她始终忧虑儿子会不会生育。没过多长时间,晚霞又生下一子,和普通人家孩子一样,阿端母亲这才转忧为喜。

时间一长,晚霞渐渐感觉到阿端不是活人,责备他说:“为什么不早说!凡是鬼穿了龙宫衣裳,经过四十九天,魂魄坚固凝聚,与活人一样的。如果得到宫中的龙角胶,可以续骨节、生肌肤,只可惜当初没有早早买下来!”阿端取出身上带的夜明珠出卖,被一位西域商人用百万金买走。从此以后,蒋家变成巨富。

一次,阿端为母亲作寿,阿端夫妻俩双双起舞,消息传到王府,王爷想将晚霞夺过来。阿端吓慌了,忙去面见王爷,对王爷说他夫妻二人全是鬼。王爷不相信,让人检验阿端,果然没有影子,这才作罢。王爷又命晚霞在宫中别院教宫女舞技。晚霞用龟尿毁了自己的容貌然后去见王爷。晚霞在宫中教了三个月舞,宫女们到底不能全部学会,她后来也就离去了。


\subsection{1.11.8   白 秋 练}
\label{\detokenize{p00_u5176_u5b83/_u767d_u8bdd_u804a_u658b_u5fd7_u5f02:id431}}
直隶有慕生,小字蟾宫,商人慕小寰之子。聪惠喜读。年十六,翁以文业迂,使去而学贾,从父至楚。每舟中无事,辄便吟诵。抵武昌,父留居逆旅,守其居积。生乘父出,执卷哦诗,音节铿镪。辄见窗影憧憧,似有人窃听之,而亦未之异也。

一夕翁赴饮,久不归,生吟益苦。有人徘徊窗外,月映甚悉。怪之,遽出窥觇,则十五六倾城之姝。望见生,急避去。又二三日,载货北旋,暮泊湖滨。父适他出,有媪入曰:“郎君杀吾女矣!”生惊问之,答云:“妾白姓。有息女秋练,颇解文字。言在郡城,得听清吟,于今结念,至绝眠餐。意欲附为婚姻,不得复拒。”生心实爱好,第虑父嗔,因直以情告。媪不实信,务要盟约。生不肯,媪怒曰:“人世姻好,有求委禽而不得者。今老身自媒,反不见纳,耻孰甚焉!请勿想北渡矣!” 遂去。少间父归,善其词以告之,隐冀垂纳。而父以涉远,又薄女子之怀春也,笑置之。

泊舟处水深没棹;夜忽沙碛拥起,舟滞不得动。湖中每岁客舟必有留住守洲者,至次年桃花水溢,他货未至,舟中物当百倍于原直也,以故翁未甚忧怪。独计明岁南来,尚须揭资,于是留子自归。生窃喜,悔不诘媪居里。日既暮,媪与一婢扶女郎至,展衣卧诸榻上,向生曰:“人病至此,莫高枕作无事者!”遂去。生初闻而惊;移灯视女,则病态含娇,秋波自流。略致讯诘,嫣然微笑。生强其一语,曰:“‘为郎憔悴却羞郎’,可为妾咏。”生狂喜,欲近就之,而怜其荏弱。探手于怀,接为戏。女不觉欢然展谑,乃曰:“君为妾三吟王建‘罗衣叶叶’之作,病当愈。”生从其言。甫两过,女揽衣起曰:“妾愈矣!”再读,则娇颤相和。生神志益飞,遂灭烛共寝。女未曙已起,曰:“老母将至矣。”未几媪果至。见女凝妆欢坐,不觉欣慰;邀女去,女俯首不语。媪即自去,曰:“汝乐与郎君戏,亦自任也。”于是生始研问居止。女曰:“妾与君不过倾盖之交,婚嫁尚未可必,何须令知家门。”然两人互相爱悦,要誓良坚。

女一夜早起挑灯,忽开卷凄然泪莹,生起急问之。女曰:“阿翁行且至。我两人事,妾适以卷卜,展之得李益《江南曲》,词意非祥。”生慰解之,曰:“首句‘嫁得翟塘贾’,即已大吉,何不祥之与有!”女乃少欢,起身作别曰:“暂请分手,天明则千人指视矣。”生把臂哽咽,问:“好事如谐,何处可以相报?”曰:“妾常使人侦探之,谐否无不闻也。”生将下舟送之,女力辞而去。无何慕果至。生渐吐其情,父疑其招妓,怒加诟厉。细审舟中财物,并无亏损,谯呵乃已。一夕翁不在舟,女忽至,相见依依,莫知决策。女曰:“低昂有数,且图目前。姑留君两月,再商行止。”临别,以吟声作为相会之约。由此值翁他出,遂高吟,则女自至。四月行尽,物价失时,诸贾无策,敛资祷湖神之庙。端阳后,雨水大至,舟始通。

生既归,凝思成疾。慕忧之,巫医并进。生私告母曰:“病非药禳可痊,惟有秋练至耳。”翁初怒之;久之支离益惫,始惧,赁车载子复入楚,泊舟故处。访居人,并无知白媪者。会有媪操柁湖滨,即出自任。翁登其舟,窥见秋练,心窃喜,而审诘邦族,则浮家泛宅而已。因实告子病由,冀女登舟,姑以解其沉痼。媪以婚无成约,弗许。女露半面,殷殷窥听,闻两人言,眦泪欲望。媪视女面,因翁哀请,即亦许之。至夜翁出,女果至,就榻呜泣曰:“昔年妾状今到君耶!此中况味,要不可不使君知。然羸顿如此,急切何能便瘳?妾请为君一吟。”生亦喜。女亦吟王建前作。生曰:“此卿心事,医二人何得效?然闻卿声,神已爽矣。试为我吟‘杨柳千条尽向西’。”女从之。生赞曰:“快哉!卿昔诵诗余,有《采莲子》云:‘菡萏香莲十顷陡。’心尚未忘,烦一曼声度之。”女又从之。甫阕,生跃起曰:“小生何尝病哉!”遂相狎抱,沉疴若失。既而问:“父见媪何词?事得谐否?”女已察知翁意,直对“不谐”。

既而女去,父来,见生已起,喜甚,但慰勉之。因曰:“女子良佳。然自总角时把柁棹歌,无论微贱,抑亦不贞。”生不语。翁既出,女复来,生述父意。女曰:“妾窥之审矣:天下事,愈急则愈远,愈迎则愈拒。当使意自转,反相求。”生问计,女曰:“凡商贾之志在于利耳。妾有术知物价。适视舟中物,并无少息。为我告翁:居某物利三之;某物十之。归家,妾言验,则妾为佳妇矣。再来时君十八,妾十七,相欢有日,何忧为!”生以所言物价告父。父颇不信,姑以余资半从其教。既归,所自买货,资本大亏;幸少从女言,得厚息,略相准。以是服秋练之神。生益夸张之,谓女自夸,能使己富。翁于是益揭资而南。至湖,数日不见白媪;过数日,始见其泊舟柳下,因委禽焉。媪悉不受,但涓吉送女过舟。翁另赁一舟,为子合卺。

女乃使翁益南,所应居货,悉籍付之。媪乃邀婿去,家于其舟。翁三月而返。物至楚,价已倍蓰。将归,女求载湖水;既归,每食必加少许,如用醯酱焉。由是每南行,必为致数坛而归。后三四年,举一子。

一日涕泣思归。翁乃偕子及妇俱入楚。至湖,不知媪之所在。女扣舷呼母,神形丧失。促生沿湖问讯。会有钓鲟鳇者,得白骥。生近视之,巨物也,形全类人,乳阴毕具。奇之,归以告女。女大骇,谓夙有放生愿,嘱生赎放之。生往商钓者,钓者索直昂。女曰:“妾在君家,谋金不下巨万,区区者何遂靳直也!如必不从,妾即投湖水死耳!”生惧,不敢告父,盗金赎放之。既返不见女。搜之不得,更尽始至。问:“何往?”曰:“适至母所。”问:“母何在?”腆然曰:“今不得不实告矣:适所赎,即妾母也。向在洞庭,龙君命司行旅。近宫中欲选嫔妃,妾被浮言者所称道,遂敕妾母,坐相索。妾母实奏之。龙君不听,放母于南滨,饿欲死,故罹前难。今难虽免,而罚未释。君如爱妾,代祷真君可免。如以异类见憎,请以儿掷还君。妾自去,龙宫之奉,未必不百倍君家也。”生大惊,虑真君不可得见。女曰:“明日未刻,真君当至。见有跛道士,急拜之,入水亦从之。真君喜文士,必合怜允。”乃出鱼腹绫一方,曰:“如问所求,即出此,求书一‘免’字。”生如言候之。果有道士蹩躠而至,生伏拜之。道士急走,生从其后。道士以杖投水,跃登其上。生竟从之而登,则非杖也,舟也。又拜之,道士问:“何求?”生出罗求书。道士展视曰:“此白骥翼也,子何遇之?”蟾宫不敢隐,详陈始末。道士笑曰:“此物殊风流,老龙何得荒淫!”遂出笔草书“免”字如符形,返舟令下。则见道士踏杖浮行,顷刻已渺。归舟女喜,但嘱勿泄于父母。

归后二三年,翁南游,数月不归。湖水俱罄,久待不至。女遂病,日夜喘急,嘱曰:“如妾死,勿瘗,当于卯、午、酉三时,一吟杜甫《梦李白》诗,死当不朽。待水至,倾注盆内,闭门缓妾衣,抱入浸之,宜得活。”喘息数日,奄然遂毙。后半月,慕翁至,生急如其教,浸一时许,渐苏。自是每思南旋。后翁死,生从其意,迁于楚。


\subsection{1.11.9   王 者}
\label{\detokenize{p00_u5176_u5b83/_u767d_u8bdd_u804a_u658b_u5fd7_u5f02:id432}}
湖南巡抚某公,派遣一名州佐押解六十万两饷银进京。途中,遇到大雨,耽搁到天晚,误了行程,找不到住宿的地方。远远望见有座古庙,州佐便驱赶着役夫,去古庙投宿。住了一晚,天明起来一看,押解的银子已荡然无存。众人都大惊失色,极为奇怪。到处找寻不到,州佐只得返回,禀报了巡抚。巡抚认为他在说谎,要惩办他。等到审讯役夫们时,也都是众口一词。巡抚便责令州佐,仍回古庙去缉查头绪。

州佐返回古庙,见庙前有个瞎子,相貌非常奇异,标榜说:“能知人心事。”州佐便求他给算算卦。瞎子说:“你必定是为了丢失银子的事。”州佐回答说:“是的。”便告诉瞎子自己因丢失饷银被巡抚重责的情形。瞎子让他找一顶二人抬的小轿,说:“只管跟着我走,到时你就知道了。”州佐听了,便找来顶轿子抬着瞎子,自己和差役们跟着他走。瞎子说:“往东,”众人便都往东走;瞎子又说:“往北,”大家便又往北走。一连走了五天,进入一座深山中,忽见一座城市,街上车水马龙,行人川流不息。进城后,又走了一会儿,瞎子说:“停下,”从轿子上下来,用手往南指了指,说:“往前走,见有个朝西开的大门,你就敲门询问,自然会知道。”说完,拱拱手自己走了。

州佐按照瞎子说的,又往前走了走,果然见有座大门。走进门内,一个人迎出来。看那人的穿戴衣著,都是古时装束,见了州佐,也不通报自己的姓名。州佐告诉他自己是从哪来的及来的缘由,那人说:“请你暂住几天,我和你去见主事的。”便领着州佐来到一间屋子,让他住下,按时供给饮食。州佐闲得没事,走出屋子蹓跶着闲逛。来到屋后,见有个花园,便进去游览。花园里,高大的古松遮天蔽日;地上细草茵茵,像铺着层绿色的毡被。穿过几处画廊亭阁,迎面见一个高亭,州佐信步登上石阶,走了进去。忽然发现墙上挂着几张人皮,脸上的五官样样不缺,腥气熏鼻。州佐毛骨悚然,急忙退出,回到了自己的屋子。自己想:看来这次得将皮留在这异域他乡了,已没有生还的希望。又想反正是死,听之任之吧。

第二天,早先的那人,来叫他走,说:“今天就可以见到主事的了。”州佐连声答应。那人骑着一匹高头大马,跑得很快,州佐徒步跑着跟在后面。不一会儿,到了一个辕门,很像是总督衙门。众多的皂隶排列在两边,气象十分威严。那人下马,领着州佐进去。又进了一重门,才看见一个大王戴着珠冠,穿着王服,面南坐着。州佐急忙走上前,跪地拜见。大王问:“你就是湖南巡抚的押解官吗?”州佐答应。大王说:“银子都在这里。这么一点点东西,你们巡抚就慷慨地送给我,也未尝不可。”州佐哭着诉说:“巡抚大人给我的期限已满了,回去后交不出银子,我就要被处死了。大王留下银子,我回去后空口无凭,怎么向巡抚大人交待呢?”大王说:“这也不难,”交给州佐一个大信封:“拿这个回去向巡抚交差,可保你无事!”说完,派了几个力士送州佐回去。州佐大气不敢喘,哪里还敢申辩!只得接下信,退出返回。力士送他走的山川道路,完全不是来时走过的。出山后,送的人才回去了。

州佐几天后才赶回长沙,去禀报巡抚事情的经过。巡抚听了,越发认为州佐在说谎欺骗自己,愤怒地命左右将他捆起来。州佐忙解开包袱,拿出那封信呈给巡抚。巡抚拆开信还没看完,已是脸色如土。又命放开州佐,只说了句:“银子也是小事,你先出去吧!”于是,巡抚重新急令属下各地,设法补齐原来的银两数,押解进京,这事才算完结。不几天后,巡抚便一病不起,不久就死了。

在此以前,巡抚有一晚跟他的一个爱妾睡觉。醒来后,发现爱妾成了光头,头发全没了。整个官衙的人无不惊骇,谁也猜不到其中缘由。原来州佐带回来的大信封中,装的就是巡抚爱妾的头发,还附着一封信,内容是:“你从当一个小县令起家,如今做到这么大的官职,贪婪地收受贿赂,赃银不计其数。上次的六十万两银子,我已查收入库,你应该从自己的私囊中补齐原数。这事与押解官无关,不得惩办他。前次特取来你爱妾的头发,以略示警告。如再不遵命令,早晚就取你项上人头!附去你爱妾的头发,以作证明!”巡抚死后,家里人才传开这封奇怪的信。

后来,巡抚的属下派人寻找深山中那座城市,只见一片崇山峻岭、悬崖峭壁,根本没有进山的路。


\subsection{1.11.10   某 甲}
\label{\detokenize{p00_u5176_u5b83/_u767d_u8bdd_u804a_u658b_u5fd7_u5f02:id433}}
某甲,和自己仆人的老婆私通,后来,他便杀了仆人,夺了他老婆,生了两男一女。过了十九年,有巨寇攻破城池,将城市抢劫一空。一个少年强盗,持刀进入某甲家。某甲见强盗长得酷似被杀死的仆人,叹息说:“我今天死定了!”献出了全部财物,想赎条命。强盗却始终不屑一顾,也不说话,只是搜出人来便杀,共杀了某甲一家二十七口人,才扬长而去。某甲被砍了一刀,但脑袋没掉下来,强寇们走后,又微微苏醒过来,还能向人们讲述这件事,三天后便死了。唉!因果报应,丝毫不错,真是怕人啊!


\subsection{1.11.11   衢 州 三 怪}
\label{\detokenize{p00_u5176_u5b83/_u767d_u8bdd_u804a_u658b_u5fd7_u5f02:id434}}
张握仲曾从军在衢州驻防,说:“衢州夜深人静后,没人敢在街上独自行走。传言钟楼上有鬼,头上长角,相貌狰狞凶恶。听到人的走路声,就从钟楼上飞扑而下。行人惊骇地逃走后,鬼也随着离开。但见鬼的人往往得病而且很多都死了。

又:城中有个水塘,夜里会从水中悄悄伸出一匹白布,像白练一样横在地上。行人如果捡抬,就会被白布卷入水中。塘中还有鸭子鬼,夜深后,水塘边什么东西也没有,一片死寂。行人如听到鸭子叫。就会得病。”


\subsection{1.11.12   拆 楼 人}
\label{\detokenize{p00_u5176_u5b83/_u767d_u8bdd_u804a_u658b_u5fd7_u5f02:id435}}
平阴人何冏卿,刚到秦中做县令时,一个卖油的犯了轻罪。但言语冲撞,何冏卿一怒之下,将他打死了。后来何到吏部做官,家里十分富有,便建了一座楼。上梁那天,召集亲戚朋友。开宴庆贺。忽见一个卖油的走了进来,何冏卿暗暗惊疑。一会儿,人报小妾生了儿子。何冏卿忧虑地说:“楼还没建成,拆楼的人先来了!”人们以为他在说笑话,不知道他实际上是有所指的。后来,何冏卿的儿子长大后,很不成人,将家产踢腾得一干二净,自己被人雇佣为役夫,每得到几文钱,就买香油吃。


\subsection{1.11.13   大 蝎}
\label{\detokenize{p00_u5176_u5b83/_u767d_u8bdd_u804a_u658b_u5fd7_u5f02:id436}}
明代时,彭宏将军率军队征伐流寇,进入四川。到一深山中,发现一座大寺院,据说已经一百多年没僧人居住。询问当地人,回答说:“寺里有妖怪,人进去就死。” 彭宏恐怕里边埋伏强盗,便率兵披荆斩棘,进入寺院中。到前殿,一只黑雕夺门飞了出去;中殿没有异常情况;又继续往前走,则是佛阁。到阁中四下一看,什么也没有,但凡是进去的人便头疼不止;彭宏自己进去,也是这佯。不一会儿,只见一个像琵琶那样大的蝎子从天花板上蠢蠢爬下,士卒们惊得一哄而散。彭宏便命令放火烧了那座寺院。


\subsection{1.11.14   陈 云 栖}
\label{\detokenize{p00_u5176_u5b83/_u767d_u8bdd_u804a_u658b_u5fd7_u5f02:id437}}
真毓生,是湖北夷陵人,举人的儿子。他文章写得好,长得又俊雅潇洒,少年时就出了名。还是孩子时,有个相面的见了他说:“以后当娶女道士为妻。”真生的父母听了都以为是笑谈。但真生长大后,虽多方提亲,却高不成,低不就,一直找不到合适的。

真生的母亲臧夫人,娘家是黄冈的。这天,真生因为有事去拜见外祖母。到了黄冈,听人都在传说“黄州‘四云’,少者无伦”。原来,本郡有座吕祖庵,庵中的女道士们都长得很美,所以有这种说法。吕祖庵距臧家村仅十几里路,真生便偷偷跑了去想见识见识。到了吕祖庵,敲敲门,果然有三四个女道士出来迎接,都很整洁漂亮。其中一个最年轻的,真是绝代佳人,无与伦比。真生一见钟情,目不转睛地盯着她。那少女手托香腮,只是看着别处。女道士们都去煮茶、找茶碗去了。真生乘机问少女的姓名,少女回答说:“叫云栖,姓陈。”真生开玩笑说:“太巧了!我正好姓潘。”云栖听了,羞红了脸颊,低下头默默不语,接着起身走了。不一会儿,女道士们煮了茶来,又端上水果,各自介绍了自己的姓名。一个叫白云深,三十多岁;一个叫盛云眠,二十来岁;另一个叫梁云栋,二十四五,却是妹妹。只是陈云栖没来。真生心中怅惘,便问云栖哪去了。白云深说:“这丫头怕生人。”真生便起身告辞。白云深极力挽留,真生不听,走出门去。白云深说:“如想见云栖,明天可再来。”

真生回去后,非常想念陈云栖。第二天,又去吕祖庵拜访。女道士们都在,惟独不见陈云栖,真生也不好马上便问。女道士们摆下饭菜,留真生吃饭。真生极力推辞,道士们不听。白云深掰开一块饼,又塞给他一双筷子,殷勤地劝着。吃完饭,真生说:“云栖在哪里?”回答说:“她自己会来的。”过了很久,天已晚了,真生想回去。白云深拉住他的胳膊,说:“再待会儿,我去把那丫头捉来见你!”真生便不走了。一会儿,白云深挑着灯笼,摆上酒菜,这时盛云眠也走了。酒过数巡,真生推辞说醉了。白云深说:“喝三杯,云栖就出来了。”真生便喝了三杯。梁云栋也以此要挟,真生又喝了三杯。喝完,倒扣过酒杯,告辞要走。白云深看着梁云栋说:“咱俩的面子小,不能劝客人多喝点。你去拖陈丫头来,就说潘郎等妙常已经很久了!”,梁云栋离去,不一会儿又回来了,说:“云栖不来!”真生想走,但夜已深,便假装醉了,仰面睡下。白、梁二人替他脱光了衣服,轮番凑上去行淫。真生终夜不堪骚扰,天刚亮,便立即走了。此后,一连好几天,不敢再去吕祖庵。但心里仍念念不忘云栖,只好不时在吕祖庵附近探视云栖的行踪。

一天,天已黑了。真生见白云深跟着一个少年男子走了,非常高兴。他不太怕梁云栋,便急忙去敲门;盛云眠答应着出来开了门,真生一问,梁云栋也出去没回来,便问云栖在不在。盛云眠领着他又进入一个小院,呼唤说:“云栖,来客人了!”只见云栖的房门“砰”地一声关上了。盛云眠笑着说:“关门了!”真生站在窗外,像有话要说,盛云眠便走了。云栖隔着窗对真生说:“她们拿我作钓饵,在钓你上钩呢!你再来,性命难保!我虽然不能守一辈子清规,可也不敢丧尽廉耻。我想得到一个真正像潘郎那样的人侍奉他!”真生发誓要跟她白头到老,云栖说:“我师傅抚养我很不容易,你如果真的爱我,就用二十两银子赎我出去。我等你三年。如指望跟我幽会偷情,绝对办不到!”真生答应了。正想再倾吐心曲,盛云眠又来了。真生只得跟着她出去,告辞回去了。心中惆怅,想再想方设法,亲眼看看云栖,正巧老家来人,告诉他父亲病危。真生连夜奔回。不久,真举人便去世了。臧夫人家教很严,真生不敢让母亲知道自己的心事,只是减扣自己的花销,天天攒钱。有来拉亲的,真生就以给父亲服孝为由推辞。母亲不听,真生婉转地告诉母亲说:“上次在黄冈,外祖母想给我提一个姓陈的姑娘,我很愿意。因为家中遭了这次变故,跟黄冈久不通音讯,很久没再去问这事了。等我再去一趟,如这事不成,再听凭母亲吩咐!”藏夫人答应了。真生便携带着自己的积蓄上了路。

到了黄冈,真生径直去了吕祖庵。只见院宇颓败,一片荒凉,跟原先大不相同。真生慢慢走进去,见只有一个老尼姑正在做饭,真生便上前询问。老尼姑说:“前年老道士死了,‘四云’早已散了。”真生问:“到哪里去了?”回答说:“云深、云栋跟恶少走了;云栖听说寄住在郡北;云眠不知下落。”真生听了,悲叹不已。便又赶到郡北,碰到庙观就打听,却没有一点云栖的踪迹。真生只得惆怅地返回家,骗母亲说:“舅父说:陈老翁到岳州去了,等他回来,就派仆人来告知。”半年后臧夫人回娘家探亲,问母亲这件事,她母亲却茫然不知。臧夫人大怒,知道儿子在撒谎。臧老太太却怀疑外甥孙子跟他舅父有商量,只是没告诉自己。幸亏真生的舅父出了远门,没法对证。

臧夫人到莲峰烧香还愿,在山下住宿。睡下后,店主人又来敲门,送进来一个女道士,同宿一屋。女道士自称叫“陈云栖”,听臧夫人说家是夷陵的,云栖就搬过座位,挨着夫人讲诉起自己的坎坷遭遇,言词神情悲伤凄恻。最后又说:“我有个姓潘的表兄,跟夫人是同一个地方的。麻烦夫人托您的子侄们去告诉他,就说我现在暂住在栖鹤观师叔王道成处,天天受苦,度日如年,让他早点来看看我。不然恐怕错过这个机会,以后就难以见面了。”臧夫人询问潘生的名字,云栖却不知道,只是说:“他既然在学宫读书,那些秀才们一定听说过他。”第二天,天还没亮,云栖早早告辞,又再三嘱咐臧夫人不要忘了。

臧夫人回家,跟儿子提起这事。真生跪在地上说:“实话告诉母亲:那个潘生,就是儿子!”臧夫人问知缘故,大怒地说:“不肖之子!在尼姑观行淫,以女道士为妻,传出去还有什么脸见亲戚朋友!”真生耷拉着脑袋,一句话不敢说。正好真生要到郡城考试,便偷偷地租了船去访王道成。赶到栖鹤观,得知云栖已于半月前出游,一去不回。真生回到家中,郁郁不乐,接着便病了。

正赶上真生的外祖母去世了。臧夫人回去奔丧。出殡后回家的路上迷了路,来到一个姓京的人家,一打听,还是自己的族妹家。京家请臧夫人进屋。臧夫人见到堂屋内有个少女,约十八九岁,长得秀雅无比,真是从没见过这样漂亮的少女。臧夫人常想找个美丽的儿媳,好安慰儿子,见了这个少女,不禁心动,便打听她的情况。族妹说:“这是王家的女儿,京家的外甥女。双亲都已去世,暂时寄居在这里。”臧夫人问:“婆家是哪里?”族妹回答说:“还没有。”臧夫人握着那少女的手跟她说了几句话,见她神情娇婉,心中更加高兴。便在京家住了一晚,私下把自己的意思告诉了族妹。族妹说:“这事很好。只是这姑娘自视很高;不然,怎会拖到现在还没婆家。容我慢慢和她商量。”臧夫人叫过少女同床而睡,二人又说又笑,十分高兴。少女自愿认臧夫人为母,夫人欢喜,请她同去荆州。少女更加高兴。

第二天,臧夫人带着少女同船返回。到家后,真生仍然卧病在床。母亲想安慰安慰他,让丫鬟悄悄地去告诉他说:“夫人给公子带了个美人来!”真生不信,趴在窗子上往外瞅了瞅,果然见一个少女,生得比云栖还要美丽十分。心中想道:三年之约已经过去,既然出游一去不返,肯定有了新意中人。现在得到这样一个美人,倒也足慰平生。于是喜笑颜开,病也好像一下子好了。母亲招呼真生和少女见过面,真生便出去了。臧夫人对少女说:“你知道我让你一同来的意思吗?”少女微笑着说:“我已经知道了。但我之所以愿意一同来的本意,母亲却不知道。我小的时候和夷陵人潘生订了亲。后来音讯隔绝,想必他早已另娶。如果真是这样,那我们就做婆媳;不然,我们仍然做母女。”臧夫人说:“既然早有婚约,当然不能勉强。只是前些年我在五祖山时,就有个女道士打听潘生;现在又是潘生,可夷陵的世族大家并没有姓潘的。”少女惊讶地问:“那次在莲峰下住宿的,是母亲吗?打听潘生的那个女道士就是我啊!”臧夫人恍然大悟,笑着说:“如是这样,那么潘生早就在这里了!,少女问:“在哪里?”夫人命丫鬟领着她去问真生。真生大惊,问:“你是云栖?”少女问:“你怎么知道的?”真生讲了实情,说当初冒姓潘是跟她开了个玩笑。少妇知道“潘生”就是真生,害羞地不说话了,忙回去告诉了夫人。夫人问道:“你怎么又姓了王呢?”云栖回答说:“我本姓王。我的师傅很喜欢我,认了我作女儿,我便改姓了师傅的姓。”臧夫人也很高兴,择了吉日为儿子和云栖成了亲。

原来,云栖和云眠当初都去投奔了王道成。因为王道成住处狭窄,云眠便又去了汉口。云栖娇弱,不能劳作,又害羞再去当道士,王道成很不耐烦。正好碰上亲戚京氏去黄冈,云栖哭着讲了自己的遭遇,京氏便带着她一同回了家,让她换下道士的服装还了俗。因为要给她向大户人家提亲,所以忌讳提起她当过道士。但是有来提亲的,云栖都不愿意。舅父、舅母摸不透她的心思,心里十分厌烦她。由于这次偶然的机会,云栖得以跟臧夫人回到夷陵,最终找到自己的归宿,她如释重负。成亲后,真生和云栖各自述说了自己的遭遇,都欢喜得流下了眼泪。云栖为人孝顺勤谨,臧夫人非常爱怜她。但云栖喜好的是弹琴下棋,不会料理家务,臧夫人很感忧愁。

一个多月后.臧夫人让真生夫妻俩去京氏家拜访。两人住了几天才往回走。船行江中,见另一只船很快地驶过,船上有个女道士。靠近一看,原来是云眠!云眠惟独和云栖要好。云栖见了她非常高兴,让她到自己船上来,二人相对心酸。云栖问:“你要到哪里去?”盛云眠说:“很久以来,我一直想着你,特地去栖鹤观寻找;听说你又去投奔京氏舅舅了,我所以要去黄冈,想去探望你,竟不知你跟意中人已经团聚!现在看你像仙女一样,只剩我一人到处漂泊,真不知何时算了?”说着,泪流不止。云栖想出一个主意:让云眠换下道士装,假称是自己的姐姐,将她先带回家中陪伴夫人,再慢慢寻找个好丈夫。盛云眠听从了。

回家后,云栖先去禀报过夫人自己的姐姐来了,盛云眠才进家。只见她举止端庄,有大家风度,言谈笑语,老练世故。臧夫人守寡已很久,很感苦寂,见了盛云眠,非常高兴,惟恐她马上就走了。第二天,云眠早早就起来,替夫人操劳,不把自己看作是客人。母亲更加欢喜,心中便暗想再为儿子娶了盛云眠,以掩饰儿媳的道士身份——她却不知道云眠也是道士。臧夫人尽管有了这心思,但还没敢直接说。一天,臧夫人忽然想起忘了一件事没做,急忙问时。云眠早已给办妥了。夫人便对云栖说:“即使长得像画上的人,但不会治家,又有什么用?新媳妇能像你姐姐这佯,我就不用担忧了。”夫人不知云栖也早就有这个心思了,只是怕母亲嗔怪,没敢说。听了母亲这样说,便笑着回答说:“母亲既然喜爱她,我想效法女英、娥皇二女同侍大舜的故事,怎么样?”母亲没说活,笑了笑。云栖退下,告诉真生说: “老母已经点头了!”于是另准备了一间干净屋子,云栖又去对云眠说:“过去我们在观中同床共宿时,姐姐曾说:‘只要能得到一个亲爱知己的人,我们两人共同服侍他。’你还记得吗?”云眠听了,不觉双眼蒙上了泪光,说:“我所谓的亲爱之人,不指别的:过去我天天劳作,并无一人知道我的甘苦;几天来,我不过稍操劳了一下,就烦老母挂念体恤,这一冷一暖,我怎能不明白!如果不下逐客令撵我走,能让我长伴老母,我便很满足了,并不敢希望能实现过去说过的话。”云栖告诉了母亲,母亲便命姐妹俩焚香发誓,永不后悔。接着就让真生和云眠行了夫妇礼。同床时,云眠告诉真生说:“我是二十三岁的老处女。”真生还不太相信。既而下红沾湿了褥子,真生才大感惊奇。盛云眠说:“我之所以想找个丈夫,并不是耐不得女尼观中的寂寞;实在是因为拿自己的清白身子,像妓女一样应酬客人,令人不能忍受!我借和你这一次欢会,以明确我是属于你的人。今后我只愿代你服侍老母,料理家务。像那闺房之乐,请你跟别的人一块去探讨。”三天后,云眠便抱着被子去找老母,让她回去也不回。云栖便早早地到母亲处占了她的床,云眠迫不得已,只得跟真生去睡。从此,隔两三天,两人就更换一次。

臧夫人本来很会下棋,自从守了寡,便没心思再下了。盛云眠来了后,一切家务都料理得井井有条。夫人白天没事,常常和云栖下棋;晚上就挑灯品茶,听两个儿媳妇弹弹琴,到半夜才散。常常对人说:“孩子的父亲活着时,我都没现在这么快活!”盛云眠掌管帐簿和钱财,每次记完帐,都要报告老母。老母怀疑地说:“你们姐妹俩都说自小就成了孤儿,那么记帐、弹琴都是跟准学的?”云眠实说了自己的道士身分,母亲也笑着说:“起初我不想给儿子娶个女道士,现在竟娶了两个!”忽然想起儿子小时算的卦,才相信命中注定,运数难逃。

后来,真生又去考了次试,仍没考中。夫人说:“我们家虽不富裕,也有薄田三百亩。多亏云眠经营料理,生活越来越好过。儿只管在我膝下,领着两个媳妇跟我共乐,不愿意你去求什么富贵!”真生听从了。后来,云眠生了一个儿子,一个女儿;云栖生了三男一女。母亲八十多岁时才去世,这时孙子都成了秀才,其中长孙是云眠生的,已经考中了举人。


\subsection{1.11.15   司 札 吏}
\label{\detokenize{p00_u5176_u5b83/_u767d_u8bdd_u804a_u658b_u5fd7_u5f02:id438}}
某游击官,妻妾很多。最忌别人提他的小名。不光名字,还有别的好多忌讳。“年”讳作“岁”,“生”喊作“硬”,“马”叫作“大驴”,还忌讳“败”字,叫做 “胜”,“安”叫做“放”。虽然在公文书信来往中,不怎么避忌,但家里的人如犯了忌,他便要发怒。一天,司札吏禀报公事时,误犯了忌讳,游击官大发雷霆,飞过石砚来,将他砸死了。三天后,游击官喝醉了酒卧在床上,忽见司札吏拿着一个名帖走进来,便问:“什么事?”司札吏禀报说:“‘马子安’来拜。”游击官忽然醒悟是鬼,急忙跃起,拔刀砍去。司札吏微微一笑,将名帖掷到案几上,忽然不见了人影。游击官取过名帖来看看,见上面写着“岁家眷硬大驴子放胜”几个字。(这是避游击官讳所写的拜帖。应写作“年家眷生马子安拜”。科举时代,同年登甲者,互称“年家”;旧时,两家姻亲,对幼辈门称“眷生”。胜:山东土俗称驴马的阳物为“胜”)。残暴荒谬的武夫,竟遭鬼揶揄嘲讽,太可笑了。

牛首山有一个僧人,自己起名叫“铁汉”,又名“铁屎”。有诗四十首,见过他的诗的人无不笑得前仰后合,秀才王司直将他的诗刊行,题名作“牛山四十屁”,署名“混帐行子、老实泼皮放”。不必看他的诗,光看这书名就足以让人开颜而笑了。


\subsection{1.11.16   蛐 蜒}
\label{\detokenize{p00_u5176_u5b83/_u767d_u8bdd_u804a_u658b_u5fd7_u5f02:id439}}
学使朱矞三家门槛下,有条蚰蜒,长好几尺。每遇到刮风下雨天气,蚰蜒就会钻出来,盘旋在地上,很像是一团白绢。据说:蚰蜒形状像蜈蚣,白天看不见,晚上才出来。闻到腥味就聚到一起。有的人说:蜈蚣没有眼睛,性贪。


\subsection{1.11.17   司 训}
\label{\detokenize{p00_u5176_u5b83/_u767d_u8bdd_u804a_u658b_u5fd7_u5f02:id440}}
有个掌管学校的教官,耳朵聋,但和一个狐狸很友好。狐狸对着他耳朵说话,就能听见。每当拜见上司时,他便让狐狸跟随,因此,人们都不知道他耳朵背。过了五六年,狐狸辞别他离去,临走前嘱咐说:“你现在的样子就像一个木偶。木偶不舞弄它,脸上的五官便都没有用。与其将来因为耳聋获罪,不如自求清高,现在就辞职回家。”但教官留恋官禄,不听狐狸的劝告。此后,他答对上司的提问时,常常驴唇不对马嘴。学使要赶他走,教官哀求大官们给讲情,才留了下来。

一天,这个教官在考场中任事。学使点完名,退下和教官们闲坐。教官们乘机纷纷从靴子里摸出要走后门的考生名籍,呈给学使,请求录取。过了会儿,学使笑着问他:“贵学怎么没有要呈进的?”教官茫然不懂。靠近他坐的人忙用胳膊肘捅捅他,把手伸到靴子里示意。教官正好在为亲戚代卖房事中用的淫具,总是藏在靴子里,到处求卖。看到学使笑着问他,怀疑是索要这种东西,站起来鞠个躬说:“有个价值八钱的最好,只是卑职不敢呈进。”满座人听了都暗笑起来。学使生气地将他赶了出去,于是被免官。

朱子青写的《耳录》一书中记载:东莱有一个老贡生,脑袋迟钝。在沂水县官学中任司训,性情颠狂痴呆。每当同行们聚会时,老贡生总是默默地坐着,不发一语。坐一会儿,不知不觉地五官都动起来;又哭又笑,旁若无人。如听到别人的笑声,就会立即止住。平时十分贪吝,积攒了一百多两银子,埋在书房里,连老婆孩子都不让知道。一天,老贡生独自坐着,忽然手脚自已动起来;一会儿,自言自语道:“一辈子做恶结怨,挨饿受冻,好不容易积蓄下的银子,都埋在书房里,如果有人知道了,怎么办呢?”像这样一连说了好几次,连官学中的一个仆役正在旁边,他也没察觉。第二天,老贡生外出,仆役进去,将银子全部挖了出来盗走了。又过了两三天,老贡生不放心,挖开藏银子的洞看看,已空空如也,他顿脚捶胸,悔恨地直想死去。

可见,教职中的人可算是千姿百态了。


\subsection{1.11.18   黑 鬼}
\label{\detokenize{p00_u5176_u5b83/_u767d_u8bdd_u804a_u658b_u5fd7_u5f02:id441}}
胶州的李总镇,曾买过两个黑鬼。黑鬼黑得跟漆一样,脚上的皮又粗又厚,把刀子竖起来摆成条路,黑鬼能在上面来回行走,丝毫不受伤。李总镇给黑鬼配了个妓女,生下的儿子却是白的。总镇的同僚和仆人跟黑鬼开玩笑,说儿子不是他的种。黑鬼也怀疑,便杀死了儿子,发现骨头是黑的,才感到后悔。总镇常常命两个黑鬼对舞,舞姿倒还值得一看。


\subsection{1.11.19   织 成}
\label{\detokenize{p00_u5176_u5b83/_u767d_u8bdd_u804a_u658b_u5fd7_u5f02:id442}}
洞庭湖中,常常有水神借船游湖。有时,一只空船系在那里,缆绳会忽然自己解开,随水飘然行驶起来。这时,只听到空中歌吹并作,乐声渺渺。船家蹲伏在船的一角,闭着眼凝神谛听,不敢抬头看上一眼,听任空船自由行驶。游完,船会仍然回到原来的地方泊住。

有一个姓柳的书生,科考落第后返回家乡,喝醉了酒卧在船上。忽然空中传来笙乐声,船家急忙摇晃柳生,要他躲避。柳生却醉得醒不过来,船家只好自己躲到船舱里。不一会儿,有人过来揪柳生,柳生醉得一塌糊涂,揪起来一放手,又瘫在船板上照旧大睡,那人便不再管他。片刻,乐声大作。柳生迷迷糊糊地醒了过来,闻到一种浓浓的兰麝香气;斜眼偷看,只见满满一船美丽女子,心里知道是神人,又闭上眼睛,假装睡着。又一会儿,听到传叫“织成”,便有个侍女走过来,正好站在柳生脸旁。柳生看侍女的脚,绿袜紫鞋,小脚又细又瘦,像手指一样,心里很喜欢,偷偷地用牙齿咬住了她的袜子,恰好侍女要走动,一下子被绊倒在船上。上座坐着的一个人奇怪地询问,侍女禀报了缘故。那人大怒,命武士将柳生拉去杀了。接着一个武士过来,将柳生按住捆绑起来,拖着便走。柳生见上座朝南坐着一人,头戴像王冠一样的东西,便边走边说:“听说洞庭君姓柳,我也姓柳;过去洞庭君考举人落第,现在我也落第;洞庭君遇到仙女而成了神,现在我醉中调戏一个侍女却要被处死,为什么幸运和不幸之间相差竟如此悬殊呢?”那个像王者的人听了,便命将柳生带回来,问道:“你是落第的秀才吗?”柳生答应。大王便给他笔和纸,命他以“风鬟雾鬓”为题作一篇赋。柳生本是襄阳名士,但得到题目后却构思了很长时间,久久没有下笔。大王讥讽地说:“名士怎么会这样?”柳生放下笔,辩解道:“过去左思作《三都赋》,十年才完成。可见文章可贵的是精妙,不是写得快。”大王笑着点了点头。又过了两个时辰,柳生才脱稿。大王阅览毕,十分高兴,称赞道:“真不愧是名士!”于是命坐赐宴,片刻之间,珍馐美味,纷纷摆了上来。柳生和大王正答对间,一个官员捧着个本子过来禀报:“溺死人的名册已经造好。”大王问:“多少人?”回答说:“共该溺死一百二十八人。”又问:“差谁去办了?”回答道;“派毛、南二尉去了。”柳生起身告辞。大王赐黄金十斤,又赠一根水晶界尺,说:“湖中将有场小劫难,拿这个可以保身。”忽见人马仪仗。纷纷列在水面上,大王下船登车,便看不见了。又过了很久,湖面上终于寂静下来。

船家等神人都消失以后,才从船舱里爬出来,驾船往北行驶。正遇逆风,船行得十分吃力。忽然有个铁锚浮出水面,船家惊骇地喊道:“毛将军出来了!”各船上的商人立刻全都伏在船里。又不长时间,湖中出现一根木头,直立在水中,忽上忽下,摇动不已。船家更加恐惧,大喊:“南将军又出来了!”话音刚落,狂风大起。湖中万丈波涛,遮天蔽日,四周的船只全部倾覆。柳生见状,急忙高举起水晶界尺,正襟危坐在船上。说也奇怪,滔天的波浪压到柳生的船前便一下子没了。由此柳生的船得以保全。

柳生回来后,常向人们谈起这件奇异的事。说船上那个侍女,虽没看见她的容貌,但只裙子下那双小脚,便是人间所没有的。后来,柳生有事到武昌,有个姓崔的老太太卖女儿,却又千金不售,家里藏着一根水晶界尺,声称有能配上这根界尺的人,便将女儿嫁给他。柳生很奇怪,便怀揣着自己的那根界尺前去看个究竟。老太太一见柳生,高兴地迎接,忙叫女儿出来拜见。她女儿有十五六岁的样子,生得娇媚温柔,风流俊雅,无与伦比。略一施礼,便返身退入帐内。柳生神魂颠倒,急忙说: “我也藏着件东西,不知能否与老太太的相匹配。”于是双方各取界尺来对照比较,样式、长短分毫不差。老太太大喜,便问柳生住在哪里,请柳生赶快回去租辆车来,界尺留下作为信物。柳生不肯,老太太笑着说:“你也太小心了!老身我怎会为了根界尺抽身逃走呢?”柳生迫不得已,只好将界尺留下,出来租辆车子,急忙返回去,老太太却已经无影无踪了。柳生大惊,问遍了住在附近的人,没有一个知道去向的。太阳已经西斜,柳生懊恼不堪,垂头丧气地往回走。走到半路上,正好一辆车子经过,忽然一人掀起车帘问道:“柳郎为什么来得这样迟?”抬头一看,正是崔老太太。柳生惊喜万分,问道:“要到哪里去?”老太太笑着说:“你一定在怀疑我是骗子。你走了以后,我突然想起你也是客居在外,要操办这些事也有困难,正好有一辆便车,便想将女儿送到你船上。”柳生便请回车一块走,老太太不肯,自己走了。柳生惶恐不安,不敢十分相信,急忙奔到船上,少女和一个丫鬟果然已经先在了,看见柳生,含笑迎接。柳生见少女绿袜紫鞋,与原来船中那个侍女没一点差别,心里很感惊异,犹豫地凝目注视。少女笑着说:“看你虎视眈眈的样子,原来没见过?”柳生听说,索性俯下身子偷偷察看,见袜子上的齿痕宛然还在,大惊说:“你是织成?”少女捂着嘴浅笑不止。柳生拜揖道:“你如真是神人,请早明白告诉我,以消除我的烦恼迷惑。”少女说:“实话告诉你吧:上次你在船中碰到的就是洞庭君。他仰慕你的才华,想把我赠给你。因为我是王妃很喜欢的侍女,所以须回去和王妃商量。我现在回来,就是奉了王妃之命的。”柳生大喜,洗手焚香,望洞庭湖中朝拜。于是,便带着织成回来了。

后来,柳生又到武昌去,织成要求同去,就便回去探亲。到了洞庭湖中,织成从头上拔下一根头钗,掷到水中。忽然从湖中冒出一只小船,织成轻轻地一跃而上,如小鸟飞上树梢,转瞬便无影无踪了。柳生紧盯着织成消失的地方,盼着她快回来。远远望见一艘楼船驶过来,来到近前,船上的一扇窗子打开,一只彩色的鸟飞掠过来,落地则是织成。接着又有人从窗子里递下许多金器明珠之类的珍贵东西,都是王妃所赐。从此后,织成每年都要回到湖中一两次去探亲,柳生也因此非常富有,金银珠宝,每拿出一件,富贵大家也不认识。

相传唐代柳毅曾为龙女传书,洞庭龙君把他招为女婿,后来,又传位给他。柳毅相貌文雅,龙君恐怕他不能威服水怪,便给他一副鬼脸面具,白天戴上,晚上摘下,时间一长,柳毅也就渐渐习惯了。一次,晚上忘了摘下来,面具便长在了脸上。照照镜子,十分自惭。所以,此后人们在湖上行船,只要用手指指某件东西,柳毅就怀疑是在指自己的脸;人们用手遮遮额头,也以为是在窥视自己,便往往必风作浪,将船只打翻。因此,凡初次在洞庭湖乘船的人,船家都要先告诉这些忌讳,或者摆上供品祭祀一番,才能安全渡湖。一次,许真君偶然来到洞庭湖,为风浪所阻,不能渡过,大怒,将柳毅逮捕,送到人间的郡狱中。郡狱狱吏点检犯人时,老是多一个人,不解何故。一晚,柳毅托梦给郡知府。哀求拯救。知府觉得人神隔绝,恐无能为力,便加以推辞。柳毅说:“许真君于某天某时来此地,只要代为恳求,一定管用。”不久,许真君果然降临。知府便替柳毅求情,才得以获释。从此,湖上行船才稍平安些了。


\subsection{1.11.20   竹 青}
\label{\detokenize{p00_u5176_u5b83/_u767d_u8bdd_u804a_u658b_u5fd7_u5f02:id443}}
鱼客,是湖南人,但不知他是哪府、哪县。他家中贫穷,科举落榜回来的路上,盘缠用光了。鱼客不好意思去讨饭,饿极了,就暂时到吴王庙中歇息,跪拜神像祈求保佑。

鱼客拜祷完出来躺在廊下,忽然有一个人带他去见吴王。那人跪下报告说:“黑衣队还缺一名士兵,可以让这个人补缺。”吴王说:“可以。”就给了鱼客一身黑衣服。鱼客穿上后,变成了乌鸦,振动着翅膀飞出去。见乌鸦们都聚集在一起,鱼客就跟着它们一块飞走了。它们三三两两分落在各条船的帆和桅杆上,船上的旅客,争着把肉抛向空中,乌鸦们都飞起来在空中接着吃。鱼客也学着这样做,一会儿就吃饱了。他飞到树梢上,觉得很得意。过了两三天,吴王可怜他没有配偶,许配他一只雌乌鸦.叫作“竹青”,它们相处得很恩爱。鱼客每次去接食物吃,总是不够机警。竹青常劝他不要去,他不听。一天,有队清兵经过,用弹子射中了鱼客的胸膛。幸亏竹青衔着它飞走了,才没被捉去。乌鸦们被激怒了,鼓动起双翅扇起波涛,浪滔汹涌,船全被掀翻了。竹青带了食物来喂鱼客,但鱼客伤得很重,到了晚上就死了。鱼客忽然像从梦中醒来,见自己仍然躺在庙中。起初,居住在这里的人看见鱼客死了,不知他是谁,摸摸他的身体还没有冷,就让人不时来照看他。这时,人们向鱼客询问了缘故,凑了些钱送他回家。

三年后,鱼客又经过这个地方,到庙中参拜了吴王,摆设了食物,唤乌鸦们下来一齐吃;并说:“竹青如果在的话,请留下来别走。”吃完以后,乌鸦们都飞走了。后来,鱼客中举回来,又来参拜吴王庙,献上猪、羊供拜。供完以后,就准备了丰盛的食物宴请乌鸦们,又祝愿竹青留下。这天晚上,鱼客在湖村住宿,点上蜡烛正坐着,忽然桌子前面像有只飞鸟飘落。鱼客一看,原来是个二十来岁的美人。这女子微笑着说:“别来无恙吧?”鱼客惊奇地问她是谁,女子说:“你不认识竹青了吗?”鱼客很高兴,问她从哪里来。竹青说:“我如今是汉江神女,很少回故乡。在这之前,乌鸦使者两次跟我说起你邀请的情谊,所以特地来与你相会。”鱼客更加兴奋感动,二人就像久别的夫妻,非常爱恋。鱼客要竹青一同到南方去,竹青想叫鱼客一块到西边去,最后也没定下去哪里。第二天,鱼客刚刚睡醒,见竹青已起来了。他睁开眼,只见高堂中巨大的蜡烛发出一片光亮,竟然不是在船上!他吃惊地起身问:“这是什么地方?”竹青笑着说:“这是汉阳啊。我家就是你家,何必一定要到南方去呢!”天色渐渐亮了,丫鬟婆子们纷纷过来侍候,酒菜也已端进来。就在大床上放一矮桌,夫妇两人对饮。鱼客问:“我的仆人在哪里?”竹青回答说:“在船上。”鱼客担心船主不能久等,竹青说:“不要紧,我会替你酬报他的!”于是二人日夜吃喝谈笑,鱼客高兴地忘了回家。

船主从梦中醒来,忽见是在汉阳,十分惊奇。仆人寻访鱼客,没有一点音信。船主想去别的地方,缆绳又解不开,两人只好一同守在船上。过了两个多月,鱼客忽然想起回家,对竹青说:“我在这里,不能与亲戚来往。况且你与我名义上是夫妻,可是连我家都没去过,怎么行呢?”竹青说:“不要说我不能去;就是去,你家里有妻子,又怎么安置我呢?不如让我住在这里,作为你的另外一个家!”鱼客恨路途太远,不能常来常往。竹青便拿出一件黑衣服来,说:“你原来穿过的旧衣服还在。如果想我时,穿上这件衣服就来了。到了这里,我再为你把衣服脱下。”于是,竹青摆下了美味佳肴,给鱼客饯别。鱼客喝得大醉,不禁睡着了。醒来后身子已经在船上,一看,船停在洞庭湖原先停泊的地方,船主和仆人都在。他们相互一看,十分震惊,都问鱼客到哪里去了。鱼客也觉得很惊奇,怅然若失。他见枕边有一个包袱,打开一看,里面是竹青赠的新衣服和鞋袜,那件黑衣也折叠在里面。又有一个绣制的口袋系在腰上,伸手一摸,里面装满了银子。于是他们开船南行,到了岸,鱼客付给船主一大笔钱,自己就回家了。

回家几个月后,鱼客苦苦思念汉水,就偷偷拿出黑衣穿上,两肋立刻长出翅膀,迅速飞向空中。过了两个时辰,已经到了汉水。鱼客盘旋飞翔着往下看,见孤屿中有一片楼舍,就飞下来落在地上。有个婢女已经看到他,呼喊说:“官人来了!”不一会儿,竹青出来,命仆人们给鱼客脱了黑衣,鱼客觉得身上的羽毛立即随之脱落下来。竹青握着他的手进了房中,说:“你来得正好,我马上就要分娩了。”鱼客开玩笑地问她说:“是胎生还是卵生?”竹青说:“我如今成了神了,皮肤和骨头已经硬了,与过去不同了。”过了几天,竹青果然生产了。孩子被厚厚的胎衣包裹着,像一个大卵。破开一看,是个男孩。鱼客非常高兴,取名叫“汉产”。三天后,汉水的神女们都来祝贺,送来了衣服食物和珍宝作为贺礼。神女们个个都非常美丽,岁数在三十以下,都走近床前,用拇指按按小孩的鼻子,说是“增寿”。神女们走后,鱼客问:“刚才来的都是谁啊?”竹青说:“她们也是汉水的神女。走在后面那个穿藕白色衣服的,就是传说中郑交甫路过汉皋台下遇见的那个解佩相赠的仙女。”过了几个月,竹青用船送鱼客回家。船上没有帆和桨,飘然自行。到了陆地上,已经有人牵着马在路旁等候,鱼客就回家了。从此,两人不断来往。

过了几年后,汉产长得更加秀美,鱼客十分疼爱他。鱼客的妻子和氏不能生育,常常想见一见汉产。鱼客就把事告诉了竹青。竹青准备了行装,送儿子跟随父亲回去,约定三个月就回来。和氏喜爱汉产,胜过自己亲生的孩子。过了十个多月,还舍不得让他回去。一天,汉产忽然暴病而死。和氏哭得死去活来。鱼客就去汉水告诉竹青。一进门,见汉产光着脚躺在床上,高兴地问竹青。竹青说:“你长时间背约,我想儿子,所以就把他招来了。”鱼客就说这是因为和氏太喜爱孩子的缘故。竹青说:“等我再生个孩子,就让汉产回去。”又过了一年多,竹青生了对双胞胎,一男一女,男孩取名“汉生”,女孩取名“玉佩”。鱼客就带着汉产回了家。然而鱼客一年总要到汉水三四次。后来觉得路远不方便。鱼客就把家迁移到汉阳。汉产十二岁时,进了郡学学习。竹青认为人间没有美貌的女子,就把汉产叫走了,给他娶了妻子后,才让他回来。汉产的妻子名叫“卮娘”,也是神女生的。后来和氏死了,汉生和妹妹都来举哀送葬,安葬完了,汉生就留在这里。鱼客带着玉佩走了,从此再没回来。


\subsection{1.11.21   段 氏}
\label{\detokenize{p00_u5176_u5b83/_u767d_u8bdd_u804a_u658b_u5fd7_u5f02:id444}}
段瑞环,是大名县的富翁,四十多岁了还没有儿子。妻子连氏,为人非常妒忌,段瑞环想买妾又不敢,便和一个奴婢私通。连氏察觉后,将奴婢痛打一顿,卖给了河间县一个姓栾的人家。

后来,段瑞环渐渐衰老,侄子们天天登门借钱借物,一句话不中意,就个个脸色难看,话也带气。段瑞环觉得不能听任他们贪得无厌,便想过继一个侄子作儿子,其他侄子们却都阻挠。连氏再凶悍,此时也无可奈何,愤怒地说:“老头子年纪才六十多岁,怎见得就不能再生儿子!”连买了两个妾,听凭丈夫所为,也不过问。过了一年多,两个妾居然都怀上了身孕,全家人欢喜万分。连氏心胸舒畅,腰杆也硬了起来,侄子们再登门强借东西,就恶声恶气地拒绝。不长时间,一个妾生了个女儿;另一个妾生了个儿子。生下不久却死了,夫妻二人大失所望。又过了一年多,段瑞环中风,一病不起。侄子们更加放肆起来,牛、马、财物只管往自家拿,连氏又哭又骂,他们却反唇相讥。连氏无计可施,只有整天哭叫罢了。段瑞环的病经过这番折腾,更加厉害,不久就死了。还没送葬,侄子们便在灵柩前商议起瓜分段瑞环的家产来。连氏痛心无比,但又无法阻止。只求给留下一所肥沃的田庄,以养活老小。侄子们不肯,连氏怒骂道:“你们寸土都不给我留下,要让我一家老少都饿死吗?”愤恨地大哭着,捶胸顿足。

忽然有个客人来吊丧,径直走到灵前,号泣尽哀,哭完,便跪到居丧的地方。众人都很惊疑,忙问是谁,来客说:“死者是我父亲!”众人大惊。客人从容地讲述了其中原委。原来,连氏卖给栾家的那个奴婢,过了五六个月,就生个儿子,取名叫怀。栾氏把栾怀跟其他儿子一样看待,抚养成人,十八岁时考中了秀才。后来栾氏去世,儿子们分家,却没有栾怀的份。栾怀询问母亲,才知道自己是段家的血脉,就说:“既然跟栾家是两姓,各人有各人的祖庙,何必在这里争人家那百亩田?”便骑马来到段家,段瑞环却已经死了。来客说得有根有据确凿无疑。连氏正在恼怒,听说后大喜,径直出来高声说道:“我现在又有儿子了!你们各人强拿去的牛马财物,可好好给我送回来,不然,咱就打官司!”侄子们面面相觑,脸上失色,一个个借故溜了。栾怀便更名为段怀,将家眷接了来,一块为父亲居丧。

段氏子侄们对段怀的来到,很感不平,一块密谋要赶走他。段怀知道后,忿怒地说:“栾家不认我姓栾,段家又不承认我姓段,要让我到哪里去!”忿忿地要向官府告状。亲戚邻居为他们排解,段家子侄才打消了念头。但连氏因牛马等物都没要回来,不肯罢休。段怀劝她算了,连氏不听,说:“我不是为了几匹牛马,心中这口气出不来。你父亲被他们气死,我所以忍气吞声,全因为没有儿子。现在有儿子了,我还怕什么!以前的事你不了解,等我自己去和他们打官司。”段怀再三劝阻,连氏不听,写下状子,径直到县衙去告了。县令便拘拿了段氏子侄们,审理起这件案子。连氏在大堂上陈述时,理直气壮,言词哀伤,滔滔不绝,县令也被感动了,将段家子侄们重打一顿,追回财物,还给了连氏。

连氏回家后,将那些没有参与瓜分自己家产的侄子们叫了来,把追回的财产全分给了他们。连氏七十多岁,将要去世时,把女儿、孙媳叫到跟前,说:“你们记着:如果三十岁还不生育,就要典当家产,给丈夫娶妾。没有儿子的滋味不好受啊!”

济南人蒋稼的妻子毛氏,不会生育,但十分嫉妒。嫂子屡次劝她给丈夫纳妾,毛氏不听,说:“宁绝了后,也不让那送眼流眉的小狐狸精在我跟前气人!”快到四十岁时,毛氏开始经常忧虑没有子嗣,想过继哥哥家的儿子,兄嫂都答应下,但却故意拖着。孩子每到叔家,蒋稼夫妇都给他好吃的,再问:“愿意来我们家吗?”孩子就说愿意。哥哥得知以后,暗地里嘱咐儿子说:“倘若她再问你,就说不愿意。问你为什么,就说‘等你死了后,不愁你们家的田产不归我所有’。”

一天,蒋稼去远方做买卖,孩子又来了。毛氏再问他,孩子就把父亲教的话学了一遍。毛氏大怒,说:“我一家老少还活着,就天天算计我家的田产吗?打错主意了!”将孩子赶了出去,立即叫来媒婆,为丈夫买妾。正好有个卖奴婢的,但价钱昂贵,毛氏拿出全部的钱也不够,眼看买不成了。蒋稼的哥哥恐怕一拖毛氏要反悔,便将媒婆叫了去,给她银子,让她假称是自己借的,再转借给毛氏帮她办成这件好事。毛氏大喜,将奴婢买回了家。等到蒋稼回来,毛氏告诉他哥哥家孩子的话,蒋稼大怒,跟哥哥断绝了来往。

过了一年多,妾便生了个儿子,蒋稼夫妻二人十分喜欢。毛氏说:“媒婆也不知从谁那里借的钱,一年多了也不来要,这恩情不能忘。现在儿子都有了,应该偿还他母亲的身价了。”蒋稼便带上钱去拜访媒婆。媒婆笑着说:“你应该感谢你哥哥。我一贫如洗,怎敢借债呢?”便详细告诉了当初买妾的经过。蒋稼醒悟,十分感动。回家来告诉了毛氏,夫妻二人感激涕零,备下酒宴,邀请哥嫂来家,二人跪着迎接,拿出银子还给哥哥,哥哥不要,尽情欢喜后走了。后来,蒋稼连续生了三个儿子。


\subsection{1.11.22   狐 女}
\label{\detokenize{p00_u5176_u5b83/_u767d_u8bdd_u804a_u658b_u5fd7_u5f02:id445}}
伊袞,是九江人。一天夜晚,他正在独坐,有个女子忽然进来。伊袞心知是狐狸,但爱怜她相貌美丽,便留住她一块睡了,也不告诉别人,父母都不知道。时间一长,伊袞变得骨瘦如柴,憔悴不堪。父母细细究问,才得知实情,非常忧虑。便让人晚上和伊袞做伴,又画咒贴符驱赶狐狸,还是阻止不了。但伊的父亲和儿子一块睡时,狐狸就不来;换个人,又来了。伊袞奇怪地询问狐,狐女回答说:“一般符咒,怎能奈何得了我?但我们狐女也讲伦理,对着父亲怎能行淫哟!”伊翁听说,此后就和儿子作伴睡觉,狐狸才走了。

后来,赶上贼寇作乱,全村人尽都逃窜。伊袞一家走散了,他自己跑进了昆仑山中,四下一看,一片荒凉。天黑后,伊袞心里更加害怕。忽然远远看见一个女子走来。等走近一看,正是那个狐女。离乱之中,两人意外相逢,都感欣慰。狐女说:“太阳已经落山了。你先在这里等等,我找一个好地方,暂时建座房子,以躲避虎狼。”说完往北走了几步,蹲在树丛中,不知干些什么。一会儿过来,拉着伊袞又往南走;约十几步,又拽着他返回来。忽然见上千棵大树,围绕着一座高大的亭子,四周有墙壁,是铜的,柱子是铁的,亭顶蒙着像金箔样的东西。近前一看,墙壁只跟肩一样高,四周围也没有门窗,墙上密密麻麻地排满了坑窝。狐女踏着这些坑翻墙进入亭内,伊袞也跟着进去。在里面一看,怀疑这座金屋不是人力造的,便问来历。狐女笑着说:“你只管住着,明天便把它赠给你。金子、铁各有千万两,够你吃半辈子的了!”说完便要告辞。伊袞苦苦挽留,孤女才留下来,说:“自己是被人厌烦抛弃了的,已决意永不再来往,现在又让我毁誓了。”第二天醒来,狐女已不知什么时候走了。天明后,伊翻墙出来,再回头看看睡觉的地方,金屋一下子消失了。只有四枚针插在一个顶针指环上,上面扣着个胭脂盒子。那千棵大树,不过是老荆棘丛而已。


\subsection{1.11.23   张 氏 妇}
\label{\detokenize{p00_u5176_u5b83/_u767d_u8bdd_u804a_u658b_u5fd7_u5f02:id446}}
凡是过大队士兵的时候,灾难比盗贼还厉害。因为盗贼人们还可以治他;兵,人们可不敢得罪。兵不同于盗贼的一点,只是不敢随便杀人而已。

甲寅年,三藩造反。去南方平叛的军队,在兖州府歇马休养,抢掠财物,奸污妇女。正赶上连阴天,田里积水成湖,老百姓没处跑,便跳墙躲到高粮地里。兵知道了,光着身子骑马进水找妇女奸污,很少有幸免的。只有张氏妇不怕,硬是不离家。家里有间不大的房,夜里同丈夫把那里挖出一个深坑,坑底竖上尖尖的竹矛,坑口盖上秫秸箔,箔上再铺上席,像睡觉的地铺。张氏妇从容地在灶房做饭。来了兵,就出门给点吃的。这时,有两个蒙占兵蛮横地要奸污她,她说:“这号事哪能当着人干?你两个人,难道叫一个看着吗?”其中一个微笑着,咕哝着,招呼她出去。她和那兵进了那间屋,指指席叫他先上去。结果箔被压断,兵就陷进了坑里。她又另找出箔和席盖上,故意站在门边引诱。一会儿,另一个兵进来了,听见有人嚎叫,不知是哪里。妇人笑着向他招手说:“这儿这儿!”这个兵踏上席也掉进去了。妇人就往坑里扔柴禾,又扔进火点着,火大起来,连屋子都烧了,妇人还人喊救火。火灭以后,尸体的焦臭味弥漫开来,人们问是什么味儿,他说:“我那两口猪怕叫兵给抢了去,藏在地窖里烧死了。”

此后,张氏妇又拿上针线活儿,找离村几里路连棵树也没有的大路旁边,在烈日下坐着。村子离城远,来的兵差不多都是骑着马,一会儿过好几拨。兵士们怪腔怪调地笑,虽然听不大懂,但妇人知道是调戏自己的下流话。但因为紧靠大路,没有遮身的东西,常常是调笑两句就过去了。这样,几天没事。

这一天,来了一个兵。这兵极无耻,大毒日头下就要强奸她。她笑笑,也不拒绝,只是偷偷地用针刺他的马。马连嘶带跳,兵就把马缰拴在自己腿上,然后去抱住妇人。妇人忽然拿绱鞋的锥子狠刺马脖子,马痛得狂奔起来。缰绳又一下子解不开,拖着兵跑了几十里,才被别的兵拦住。这位兵的头和身子不知哪去了,缰绳上的一条腿还很完整。


\subsection{1.11.24   于 子 游}
\label{\detokenize{p00_u5176_u5b83/_u767d_u8bdd_u804a_u658b_u5fd7_u5f02:id447}}
一个住在海边的人说:一天,大海中忽然冒出一座高山,人们十分惊骇。有个秀才正寄住在一条渔船上,买酒来一个人独酌。夜深后,一个少年来到船上,一副文士打扮。自称是“于子游”,谈吐文雅诙谐。秀才很高兴,请他坐下,二人便对喝起来。喝到半夜,少年起身告辞。秀才问道:“你家住哪里?黑夜茫茫,也太苦了自己了!”少年回答说:“我不是本地人。因为临近清明节,随大王去扫墓,家眷先走了,大王暂留此处歇息。明天辰时就要动身。我要先回去,打点行装。”秀才也不知大王是什么人,便送他到船头,少年一下子跳入海中,划着水远去了,秀才才醒悟是鱼妖。第二天,只见大海中的高山浮动起来,一会儿便消失了。人们才知道那座山是条大鱼,也就是所说的“大王”。

人们传说清明节前,海里有大鱼携儿带女去扫墓,难道真有这回事吗?

康熙初年,莱郡海滨,被海潮冲出一条大鱼,号叫了好几天,声音像牛叫。鱼死后,挑着担子去割鱼肉的人,一路络绎不绝。鱼足有一亩地大,鱼翅、尾巴完好无损,惟独没有眼珠子。眼眶像井一样深,里面积满了水。割肉的人误掉到里面,就被淹死了。有人说:“海龙王贬大鱼,总是先挖出眼珠子。因为眼珠是夜明珠。”


\subsection{1.11.25   男 妾}
\label{\detokenize{p00_u5176_u5b83/_u767d_u8bdd_u804a_u658b_u5fd7_u5f02:id448}}
一个官绅在扬州买妾,连相看了好几家,都不满意。有个寄住此地的老太太卖女儿,才十四五岁,相貌身材都十分标致,又能歌善舞,官绅一眼看中,用重金买了去。到夜晚睡觉时,摸摸少女的身上,皮肤光滑细腻,心中大喜;又往下一摸,大吃一惊——原来是个男子!官绅极为惊骇,细细究问。原来那老婆婆先买了男童后,再精心修饰成女子,设下骗局,出售骗钱。黎明,官绅派家人去寻找那老婆婆,早已不知去向了,心中十分懊丧。是留下“她”还是让“她”走,踌躇不决。正好浙中有个朋友来拜访。官绅告诉他这件事后,这个朋友便要看看那假女子。一见之下,很是中意,便用原价赎走了。


\subsection{1.11.26   汪 可 受}
\label{\detokenize{p00_u5176_u5b83/_u767d_u8bdd_u804a_u658b_u5fd7_u5f02:id449}}
湖广黄梅县人汪可受,能记得前生三世的事。第一世是秀才,在一座寺庙里读书。寺僧有匹母马,生了头小骡驹,秀才见了很喜欢,从寺僧手里强夺了去。死后,阎王稽查生死簿,愤怒他贪婪暴虐,罚他托生为骡子,赔给寺僧。骡子生下后,寺僧十分爱护,想求死又没有机会。稍大点,骡子常想跳到山涧里自尽,又怕辜负了寺僧的豢养之恩,到阴间里处罚会更重,只得安心活着。

几年过后,骡子孽满,自己死了。托生到一个农人家里,刚出生就会说话,父母以为是妖怪,弄死了他,才又投生到汪秀才家。汪秀才年近五十,意外得子十分喜欢。汪可受一生下就很聪明,但想起前生是因为说话过早被弄死的,便不再说话;直到三四岁,人们还都以为他是哑巴。

一天,他父亲正在写文章,有客人来访,便放下笔出去会客。汪可受进去看见父亲的文章,不觉手瘁,提笔续完。父亲回来后见了,问:“什么人来过?”家人说; “没人来。”父亲十分惊疑。第二天,故意写了个题目放到桌子上,便出去了;一会儿又返回,蹑手蹑脚地进来,见儿子正伏案握笔,纸上已写了好几行。汪可受忽抬头看见父亲,吓得叫出了声,跪在地上哀求饶命。父亲很高兴,握住他的手说:“我们家就你一个儿子,既然会写文章,这是全家的荣耀啊,为什么要隐瞒呢?” 从此后,更加教他读书。汪可受少年考中进士,后来官至大同巡抚。


\subsection{1.11.27   牛 犊}
\label{\detokenize{p00_u5176_u5b83/_u767d_u8bdd_u804a_u658b_u5fd7_u5f02:id450}}
湖北有一个农民,赶集归来,在路上略事休息。有个相面的自后面过来,停住与农人交谈,忽然端详端详农人说:“你气色不吉利,三天内当破财,受官府刑罚。”农人说:“我官粮已经交完,平生不懂得和人家争斗,刑罚从何而来?”相面的说:“我也不知道。但从你气色上看是这样,不能不谨慎啊!”农人不太相信,拱拱手,二人分别。第二天,农人在田野里放牧牛犊,有一匹驿马经过。牛犊望见,误以为是老虎,直奔过去,用脑袋猛顶,竟将马顶死。赶驿马的忙报了官,官府倒没怎么惩罚农人,只命他赔匹马。

原来,水牛见虎必斗,所以牛贩露天住宿时,往往用牛自卫;远远望见有马匹经过,就急忙将牛驱赶开躲避,就是恐怕牛误顶了马。


\subsection{1.11.28   王 大}
\label{\detokenize{p00_u5176_u5b83/_u767d_u8bdd_u804a_u658b_u5fd7_u5f02:id451}}
李信,是个赌徒。一天,他正躺着休息,忽然看见已经故去的赌友王大、冯九进来,邀请他去赌博。李信此时也忘记了二人是鬼,高高兴兴地跟他们走了。出了家门,王大要再去邀请同村的周子明。冯九领着李信先走一步,来到村东庙中。不一会儿,周子明果然跟着王大来了。冯九便拿出纸牌,四人约定赌钱。李信说:“来得太匆忙,没带本钱来,辜负了诸位的邀请,怎么办?”周子明也说没带钱。王大道:“燕子谷的黄八是放利贷的,我们一块去跟他借贷,肯定能借给你们。”于是四人又一块去借钱。飘飘忽忽地走着,瞬间便到了一个大村中。只见高门大户,连绵不断。王大指着一个大门说:“这就是黄公子家。”正要进去,一个老仆从门内出来,王大便告诉他来意。老仆回去禀报,一会儿又出来说奉公子命请王大、李信二人相见。进去一看,黄公子大约十八九岁,言谈和气,满面笑容,拿出一串钱给李信说:“我知道你一向忠厚诚实,可以借给你钱。周子明这人我信不过。”王大委婉地替周子明讲情,黄公子才同意借,但必须由李信署名担保。李信不肯,王大在一边怂恿劝说,李信无可奈何,勉强同意,黄公子才又拿出一串钱给了他。出来后,李信把钱交给周子明,又将黄公子的话说了一遍,意思是激他日后一定偿还。

四人出了燕子谷,迎面看见一个妇人走过来。原来是同村中赵家媳妇。这个妇人一向凶悍,平时好争善骂。冯九说:“这里没人,我们捉弄捉弄这个悍妇。”于是和王大逮住妇人,拉入谷中。妇人惊惧地大哭大叫,冯九抓起把土塞进她嘴中。周子明赞同地说:“这种悍妇,就应当在她阴户中打个橛子!”冯九便剥下她的裤子,用根长条石强捅了进去。妇人就像死了一样,不出声了。四人见状,赶紧散了。又一块聚到庙中,开始赌博。从中午一直赌到晚上,李信大胜,冯九、周子明却输了个净光。李信把赢的钱加利息后全部给了王大,让他代还给黄公子。王大又匀给冯九、周子明一些,赌博才继续进行。刚赌了不长时间,听见庙外人声嘈杂,一片呐喊,一个人飞跑进来,喊道:“城隍老爷亲自捉拿赌徒,已到了门外了!”四人脸上失色。李信见机不好,扔下钱,翻墙逃走了。剩下三人顾钱,都被拿住,捆了起来。从庙里出来,果然见一个神仙骑在马上,马后拴着一串赌徒,足有二十多人。天还没亮,已走到一座城池,开了城门进去,来到官衙中,城隍面南坐下,将犯人叫上大堂,手中拿本花名册,一一点名毕,命将所有赌徒的中指用斧子剁下来;又命将赌徒的两眼分别涂成红色和黑色,游街三圈。游完街,押送的差役向赌徒们索贿,答应替他们抹去眼上的颜色。众人都争着送钱,惟独周子明不肯,说没钱。差役要把他送回家去取,周子明也不肯。差役指着他骂道:“你真是粒炒不爆的铁豆子!”拱拱手走了。周子明一人出城回家,路上用唾沫沾湿袖子,边走边擦眼睛。走到河边往水里一照,颜色依然还在;又捧水猛洗,却怎么也洗不掉,只得悔恨地回家。

在此以前,赵家媳妇有事回娘家,天黑后还没返回。丈夫去接,走到谷口,见老婆横躺在路边。看样子,知道是遇上了鬼。忙把嘴里的泥巴抠出来,背回家中。妇人渐渐醒了过来,丈夫才知道阴中还有东西,便将那根长条石慢慢转着拔出来。妇人述说了路上的遭遇,赵氏大怒,急忙去县衙,状告李信和周子明。衙役来到李、周二人家中逮人。见李信刚睡觉醒来,周子明却还在昏睡,像死了一样,不可能是他们干的。县令一听以诬告罪将赵氏夫妻重打一顿,夫妻二人无法申辩。

第二天,周子明醒过来,两眼眶子忽然一个成了红色,一个成了黑色;又大叫手指痛,仔细一看,中指的骨头已经断了,只有皮连着,几天后,半截手指便掉了下来。眼睛上的颜色,深入皮肉之中,看见的人无不掩口而笑。一天,又见王大来索债,周子明只是大声说没钱,王大忿恨地走了。家里人询问后,才知道缘故,都劝他神鬼无情,还是还钱为好。周子明执意不肯,说:“现在当官的,都袒护赖债不还的。阴间和阳间应该没什么两样,更何况还是赌债呢?”第二天,有两个鬼役来,说黄公子已向城隍投了诉状,告了周子明赖债不还,要拘拿他去审讯;李信在家中也见有鬼役来到,捉去作证——于是周、李二人突然死了。到村外会面,见王大、冯九都在。李信对周子明说:“你还是红黑眼,怎敢去见官呢?”周子明仍是说没钱行贿。李信知道他一向贪吝,便说:“你既然想赖,我只好请见黄公子,替你还钱了!”又一块到黄公子家,李信先说明了缘故,黄公子不同意,说:“欠债的是谁,却让你还钱?”李信便出来告诉周子明,跟他商量自已出钱,让他拿去还债。周子明恼羞成怒,连黄公子也攻击起来。鬼役便将公子家仆一块拘拿。不长时间,来到官衙,进去看见城隍,城隍怒斥周子明说:“无赖贼!眼上的颜色还在,又赖债吗?”周子明招供说:“是黄公子放的利债,引诱我去赌博,才被老爷处罚。”城隍便叫公子家的老仆上来,发怒说:“你家主人开场聚赌,还敢讨债吗?”老仆分辩说:“借钱时公子不知道他们是去赌博。公子家住燕子谷,他们的赌场在观音庙,两地相距十几里路。公子也从无开设赌场之事。”城隍听说,看着周子明道: “借人钱赖帐不还,还给人捏造罪名,你可算是人当中最不是东西的了!”喝命痛打。周子明忙又诉说黄公子放的贷利钱太重,城隍问道:“你还了多少了?”老仆说:“一文钱也没还。”城隍怒道:“本钱都还没还,谈什么利息!”命重打三十,立即押回家取钱还债。鬼役索贿,将他押回家中,不让立即复活,却将他绑在厕所里,托梦给他的家人。家人忙烧了二十串纸钱,火一灭,化成二两金子,两千钱。周子明用金子还赌债,用钱贿赂鬼役,才被释放回家。醒过来后,屁股上被打伤的地方都鼓了起来,脓血淋漓,几个月时间才好了。

后来,赵家媳妇不敢再骂大街;而周子明尽管少了个手指,又是红黑眼,却照赌如故。由此可知赌徒们真不是人啊!


\subsection{1.11.29   乐 仲}
\label{\detokenize{p00_u5176_u5b83/_u767d_u8bdd_u804a_u658b_u5fd7_u5f02:id452}}
乐仲,是西安人,还没出生时父亲就去世了,是遗腹子。母亲信佛,一辈子不吃荤酒。乐仲长大后,能吃好喝,嘴上虽不敢说,心里却讥笑母亲太愚,常常拿甘甜肥美的东西劝母亲享用,总遭母亲呵斥、拒绝。后来,母亲一病不起,弥留之际,忽然苦苦想肉吃。乐仲急切间找不到,便从自己左腿上割下块肉献给了母亲。母亲吃了后,病稍好了点,却又后悔破了戒,竞不吃不喝,绝食而死。乐仲痛不欲生,心想母亲是吃了自己的肉才悔恨死的,不禁气愤地用刀猛刺自己的右腿,以至于露出了骨头。家里的人急忙将他救下。又敷药包扎起来,所幸不长时间便好了。心里惦念着母亲一辈子守节受苦,又哀痛母亲太愚,一气之下,烧了母亲生前供奉的佛象,立起母亲的牌位,早晚祭祀。常常是酒醉后,便对着牌位痛哭上一场。

后来,乐仲长到二十岁,结婚娶妻,此时还是个童男。婚后三天,便对人说:“男女共居一室,真是天下最污秽的事情!我实在没感到有什么快乐的!”将妻子休回了娘家。岳父顾文渊,央求亲戚讲情,跑了三四趟,乐仲执意不允。延迟了半年,顾文渊只得让女儿改嫁。乐仲打了二十年光棍,行为更加狂荡不羁。不管是奴仆皂隶,还是戏子乐工,他都愿和他们一块喝酒。亲戚邻居上门求借,他毫不吝惜。有个人说嫁女儿还缺口铁锅,他便从自家灶上揭下锅奉送,自己此后只得借邻居家的锅做饭。那些无赖之徒摸准了他的性情,经常来骗他的东西。有个赌徒,赌博没有本钱,便跑去对着他挤下几滴眼泪,说家里没钱交税,官府催逼又紧,没办法打算将儿子卖了。乐仲听说,果然倾囊出资,将“税金”如数送给了他。等到官役催税到了自己家门,便只好典卖家产筹办了。因此,乐仲日益穷困下来。先前,乐仲还很富裕的时候,同族子弟们都争着侍奉他;凡是家里有的,任他们取拿,乐仲毫不计较。等到家境困苦败落,子侄们便再也不登门了。乐仲性情旷达,也没放在心上。有次,赶上母亲忌日,乐仲正好病了,不能上坟,打算让一个侄子代他去祭奠,那些人却都找借口拒绝,没一个愿去的。乐仲无可奈何,只得在室内祭了一番,对着母亲的牌位痛哭了一场。没有子嗣的忧伤,萦绕心头,使得病势越发沉重。正在昏迷中,觉得有人在抚摸自已,微微睁眼一看,竟是母亲!乐仲惊诧地问:“母亲怎么来了?”母亲回答说:“没人给我上坟,所以来家里享祭,顺便看看你的病。”乐仲又问:“母亲一直住在哪里?”回答是“南海”。等母亲抚摸完,乐仲只觉遍体凉爽,舒畅无比,睁眼一看,室内已渺无人影,病却好了。

乐仲痊愈后,立志要去朝拜南海。正好邻村有结香社去南海的,乐仲便卖了十亩地,带着钱去恳求加入香社。香社的人嫌他不洁净,都加以拒绝。乐仲只得尾随着他们上路了。一路上他酒肉韭蒜照吃不误,大家更加厌恶他,乘他醉酒大睡时,众人不告而别,乐仲落了个踽踽独行。走到福建,碰上个朋友邀请他喝酒,有个叫琼华的名妓也在座。乐仲谈起要去南海,琼华愿意一块前去,乐仲大喜,整治行装,和她一块继续南下。二人虽然吃住在一起,却从未有染。到了南海,香社里的人见他竟然带了个妓女来,越发讥笑他,鄙夷地不屑和他们一块朝拜。乐仲和琼华明白众人的意思,听任他们先拜完,自己才拜。众人拜时,海里没有一点显示,十分恼恨。等二人拜时,刚跪到地上,忽然遍海莲花座,座上垂着串串璎珞。琼华看见上面坐着的都是菩萨,乐仲看到的却是每个佛座上都坐着母亲,急忙大喊大叫着跳到海中,向母亲奔去。众人只见万朵莲花,突然都变成了绚丽彩霞,像彩锦一样铺满了整个海面。不一会儿,云静波平,一切都消失了,乐仲仍然还在海岸上,自己也不知是怎么从海里出来的,衣服鞋子没一点沾湿的地方。乐仲望海大哭,声震岛屿。琼华扶着他百般劝解,自己也不禁流下了眼泪。

二人朝拜完毕,驾船北返。路上有个豪门大户将琼华叫了去。乐仲自己住在旅店里,见有个小孩,大约八九岁,在店铺中行乞,看样子又不像是个乞丐。乐仲上前细细询问,得知是被继母赶出家门的流浪儿,心里十分可怜。小孩依傍着他,苦苦哀求拯救,乐仲便带着他返回家中。询问小孩的姓氏,回答说:“叫阿辛,姓雍,母亲姓顾。曾听母亲说,嫁给姓雍的六个月,便生下了我,我本姓乐。”乐仲大惊,怀疑自己平生只和原来的妻子顾氏同居过一次,不可能有儿子,因此又问孩子的老家在哪里,小孩回答道:“不知道。但母亲去世时,留给我一封书信,嘱咐不要丢了。”乐仲急忙索信,一看,原来是自己写给顾家的休妻文书。大惊道:“真是我的儿子!”又问明孩子出生的年月时间都相符,心中顿感十分欣慰。只是从此后家计日渐艰难,过了两年,田地便卖净了。再也不能雇奴仆。

一天,父子二人正在做饭,忽然有个美丽的女人走进家门,一看,原来是琼华。乐仲惊问:“你怎么来了?”琼华笑着说:“我们已经做了假夫妻,又问什么?先前没有跟你来,是因为家里还有个老太太。现在她已死去,自己考虑着不跟了人,没法保护自己;跟了人又没法守身,两全齐美的办法,只能是跟你,所以不远千里赶了来。”说完,放下行装,代阿辛做饭。乐仲十分高兴。到了夜晚,父子仍像往常一样一块睡觉,另打扫一间屋子让琼华住下,阿辛也认了她为母亲,琼华待他如亲生儿子一般。亲戚朋友听说后,都按照婚仪礼节馈赠给乐仲和琼华食物,二人都高兴地收下。有客人来家,琼华总是治办下丰盛的酒宴招待,乐仲也从不问酒菜是哪来的。渐渐地,琼华拿出金、珠之类。赎回原来的家产,又广置牲畜、奴仆,日子一天天富裕热闹起来。乐仲常对琼华说:“我酒醉时,你要避开,不要让我看见。” 琼华笑着答应。一天,乐仲大醉,急切地呼唤着琼华,琼华盛装迎出。乐仲斜着醉眼看了很久,忽然高兴地手舞足蹈,说:“我明白了!”顿时清醒过来,只觉世界一片光明,所住的茅屋全变成琼楼玉宇,过了会儿才恢复原样。从此后,乐仲再不外出喝酒,只是天天面对着琼华喝.琼华吃素,也用茶水陪着。

一次,乐仲微醉,让琼华按摩大腿,见腿上疤痕,变成了两朵红荷花,隐隐突出肉际,琼华非常惊奇。乐仲笑着说:“当这两朵荷花盛开的时候,你我二十年的假夫妻就该分手了!”琼华深信不疑。为阿辛完婚后,琼华逐渐把家务事托付给儿媳管理,自己和乐仲另住一座院子。儿子、媳妇三天拜见一次,家中没有疑难大事不告诉二人,只用着两个奴婢,一个管温酒,一个管煮茶而已。有天,琼华到儿子处,儿媳禀报请示了很多家务事,又一块去见父亲。进入屋门,见父亲赤着脚坐在坐榻上,听见声音,睁开眼微笑着说:“你们都来了,很好!”说完便合上了眼。琼华大惊,问:“你要干什么?”看看他的腿上,只见莲花大开;再用手试试嘴边,已经气绝了。琼华急忙将荷花捻合住,祷告说:“我不远千里跟了你,太不容易了。又为你教子训妇,也有点功劳。就差个两三年,为什么不稍等等呢?”过了会儿,乐仲忽然又睁开了眼,笑道:“你有你自己的事,何必拉扯着别人作伴呢?没办法,姑且为了你先留下来吧!”琼华听说才放开手,见莲花又合上了。于是二人言笑如初。

又过了三年多,琼华已年近四旬,还像是二十来岁的人。一天,忽然对乐仲说:“人死了后,被别人捉头抬脚,太不雅观,也不洁净。”便找来木匠做两口棺材。阿辛惊骇地询问缘故,琼华答道:“这不是你能知道的事。”棺材做成,琼华沐浴梳妆,将儿子、媳妇叫到跟前,说:“我要死了!”阿辛大哭着说:“这些年多亏母亲料理生计,全家人才不至挨饿受冻。母亲还没享几天清福,怎么竟撇下儿子要去呢?”琼华道:“父亲种福,儿子享受。咱们家的奴仆牛马,都是那些骗债的偿还你父亲的,我没有功劳。我本是散花天女,偶然思凡,被贬谪到人间三十年,现在期限已满了。”说完,自己进入棺内躺下,再叫时,双眼已经闭上了。阿辛大哭着去告诉父亲,只见父亲不知什么时候也死了,依然穿戴整齐!阿辛悲恸欲绝,将父亲收敛到另一口棺中,和母亲并排停放在堂屋里。连续几天没有发丧,期望着父亲能活过来。此时,只见一片光明从父亲双腿上发出来,照彻了整个屋子;琼华的棺内则是香雾喷溢,连邻居家都闻到了。棺材合盖后,香气和光明才渐渐消失。

葬了二人后,乐家子弟们觊觎乐仲的家产,阴谋要赶走阿辛。告了官府,打起官司,说阿辛不是乐家的人。官府也分辨不清,打算将乐仲的家产分一半给乐氏子弟们。阿辛不服,又把官司打到郡里,仍然久久不能判决。起初,顾文渊将女儿改嫁给了姓雍的,过了一年多,雍某流落到福建,音讯也就断绝了。顾文渊老了没有儿子,十分想念女儿,便到女婿家探望,才得知女儿已死,外甥被赶出了家门,不知流落到了什么地方。顾文渊大怒,写下状子,告了官府。雍某害怕,用财物贿赂顾文渊,顾文渊不要,非要找回外甥不可。雍某到处搜寻,还是没有下落。一天,顾文渊偶然走在路上,看见一辆彩车过来,便躲让到一边。车中一个美女喊道:“你不是顾老翁吗?”顾文渊忙答应,女子说:“你外甥已成为我的儿子,现在乐家,别再打官司了,外甥正有灾难,你要赶紧前去!”顾文渊刚要仔细问问,彩车已经跑远了。顾文渊便接受了雍某的财物,急忙赶到西安。此时,乐家的官司正打得热闹,顾文渊自投到官府,说出了女儿被休回娘家的日子和改嫁的日子,以及生儿子的确切时间,十分确凿清楚。于是真相大白,乐氏子弟们都被痛打一顿,赶出大堂,案子终于了结。回家后,顾文渊讲述起看到美人的那天,正是琼华去世的那天。阿辛便让顾文渊搬到自己家,又给他房子和奴仆。直到六十多岁,顾文渊还又生下一子,阿辛也一直十分优待这个小阿舅。


\subsection{1.11.30   香 玉}
\label{\detokenize{p00_u5176_u5b83/_u767d_u8bdd_u804a_u658b_u5fd7_u5f02:id453}}
崂山下清宫里,有一株两丈高的耐冬树,树干粗壮得几个人合抱才能围过来;还有一株牡丹,也有一丈多高,花开时节,绚丽夺目,宛如一团锦绣。胶州黄生爱上这个道观的清幽雅静,便借住一个房间作了书斋。

一天,黄生正在书斋中读书,偶然抬头向窗外一望,远远看见一个白衣女郎的身影在花丛中若隐若现。黄生想,道士修炼之地哪来的女子呢?便急走出书斋看个究竟,女郎却早已无踪无影了。但此后又有几次看见女郎出来,黄生便预先藏在树丛里,等候女郎再来。不一会儿,女郎果然来了,身旁还有一个红衣女郎陪伴着。黄生望去,两个妙龄女郎,红白相映,光彩照人,真是艳丽双绝。女郎愈走愈近,突然,红衣女郎停住脚步,一边后退一边小声说:“这里有生人!”黄生不肯错过机会,猛扑过去,两个女郎吓得扭头便跑,裙衫长袖飘舞起来,传来一阵浓郁的香气。黄生追过短墙,女郎们倩影又消失了。黄生爱慕极了,便提笔在树上写了一首绝句:无限相思苦,含情对短窗。恐归沙吒利,何处觅无双?

他边想边走进书斋,白衣女郎忽然笑盈盈地走了进来。黄生又惊又喜,起身相迎。女郎笑着说:“瞧你刚才气势汹汹像个强盗,怪吓人的;没承想原来是个风流儒雅的诗人呢,那就不妨会见会见了。”黄生问起她的身世,女郎说:“我叫香玉,本是妓院中人,被道士幽闭在这山中,实在并非心甘情愿的。”黄生忙问:“道士叫什么名字?我一定替您洗雪耻辱。”香玉说:“不必了。他也没敢逼我。我趁此机会跟您这位风流文士常来幽会,倒也不错呢。”黄生又问那位红衣女郎是谁,香玉说:“她叫绛雪,是我的义姊。”两人愈谈愈亲密,当夜香玉便留宿在黄生的书斋里。第二天醒来,已是红日临窗。香玉急忙起身,说:“这真是贪玩忘天晓了!” 一边穿衣,一边高兴地对黄生说:“我也凑了一首诗,算是对昨天您的大作的酬和吧,请勿见笑:良夜更易尽,朝暾已上窗。愿如梁上燕,栖处自成双。”

黄生一听,喜出望外,握住香玉的手说:“您原来秀外慧中,漂亮而又聪明,真叫人爱死!离了你一天,真如千里之别。您务必抽空就来,不必等到晚上啊!”香玉答应着。从此二人每夜必会。黄生还常求香玉邀绛雪来,绛雪却总是不来,黄生颇觉遗憾。香玉只好安慰他:“绛姐的性情落落寡合,不像我这么痴情。你得容我慢慢劝他,不要性急呀!”

一天晚上,香玉突然闯进书斋,满面凄惨地对黄生说:“你连‘陇’都守不住,还望‘蜀’呢。咱俩永别的日子到了!”黄生大惊:“这是怎么说?你要到哪里去?” 香玉用衣袖擦着泪,泣不成声地说:“这是天意,很难给你说清的。反正当初的诗句今日应验了。‘佳人已属沙吒利,义士今无古押衙’,可以说是为我而吟的了。”黄生一再追问究竟是怎么回事,香玉就是不肯明言,只是呜呜咽咽,哭个不止。这一夜两人通宵未眠,天刚透亮香玉就走了。黄生感到十分奇怪,惴惴不安。第二天,一个姓蓝的即墨县人到下清宫来游览,见到那株白牡丹,十分喜爱,便把它挖走了。黄生这才恍然大悟,原来香玉是牡丹花妖,于是感到怅惘,而又十分惋惜。

过了一些天,黄生听说那位姓蓝的把牡丹花移植到家中,牡丹花却一天天枯萎了。黄生痛恨极了,写了哭花诗五十首,天天跑到白牡丹原来的坑穴边上痛哭凭吊。一天,凭吊完毕,正在返回书斋,远远望见红衣女郎绛雪也在牡丹穴边凭吊。黄生便慢慢走过去,绛雪也不躲避;黄生近前拉住她的衣袖,两人相对流泪。站了一会儿,黄生邀绛雪到书斋一叙,绛雪便跟着来了。绛雪长叹一声,说:“从小要好的姐妹,竟然一旦断绝了。听到你的哭声,我更悲痛。你的眼泪流到九泉之下,也许她会为你的诚心感动而复生呢。可是死者精魂开始消散,短时间内怎么能跟我们一块儿谈笑啊?”黄生也叹息说:“都怪小生命薄,妨碍了情人,当然更无福气消受双美了。从前我多次托香玉转达我的热忱,为什么您不来见我呢?”绛雪回答说:“我以为年轻书生,十有八九是薄情儿,不知你原来是个至性至情的人。不过你我相交,只在友情而不在淫乐。如果一天到晚总是卿卿我我,那我是办不到的。”说罢就要告辞,黄生赶紧上前拦住,说:“香玉长别已使我废寝忘食。全靠您陪我一会儿,我才得到一些安慰,您怎么能如此绝情呢?”绛雪无奈,只好留宿一夜,走后还是多日不见回来。黄生独自面对窗外凄冷的雨丝,苦苦思念着香玉,夜里辗转反侧,眼泪洒满了枕席。凄苦难奈之际,便披衣起床,挑亮灯烛,按照前首诗的韵脚又写起来:山院黄昏雨,垂帘坐小窗。相思人不见,中夜泪双双。

写成之后,正在低吟,忽听窗外有人说:“有作诗者便应有和诗者呀!”一听就知道是绛雪,黄生急忙开门迎接。绛雪看看书案上的诗,顺手提笔在后面续了一首:连袂人何处?孤灯照晚窗。空山人一个,对影自成双。

黄生读了和诗,又流下泪来,也更埋怨与绛雪相见的次数太少了。绛雪劝解说:“我不能像香玉妹子那么热情,只不过多少安慰一点儿您的寂寞罢了。”黄生想同她亲热,绛雪不同意,说:“聚首的欢乐,何必这样呢?”从此,每当黄生孤独难奈时,绛雪便来一次,来了也不过是与黄生饮酒作诗,有时不过夜便走了。黄生也只好由她,因此常常对她解嘲说:“香玉是我的爱妻,绛雪您是我的良友啊。”黄生总想问绛雪:“您是院中第几株?希望早告诉我,我要把您移植到我老家去,免得像香玉似的又被恶人抢去,让我遗恨一辈子。”绛雪说:“花木像人一样,故土难离,告诉你也无益。你跟爱人还不能白头偕老,何况朋友呢?”黄生不听,拉着她的臂膀来到院中,每到一株牡丹花下,就问:“这是您吧?”绛雪掩口笑笑,不作声。

不久,腊月将尽,黄生回胶州老家过年。到了二月间的一个晚上,忽然梦见绛雪来了,愁容满面对他说:“我要遭大难了!您赶紧来,还能见上一面,晚了就来不及了!”黄生惊醒后,诧异万分,急忙命仆人备马,星夜赶到崂山下清宫,看见道士要盖房屋,地基上有株耐冬树妨碍动工,工匠们正要刨树呢。黄生急忙上前阻止。到了夜间,绛雪到书斋来表示谢意。黄生笑了说:“谁叫你从前不告诉实情来着!就该遭这场灾难!现在我算知道你的底细了。如若你再不来,我一定点一把艾草烤你。”绛雪叹息一声说:“就因为知道您要这样,所以我以前才不敢实说呢。”两人对坐一会儿,黄生又想念起香玉,对绛雪说:“目下面对良友,就更思念艳妻了。这一回家,很久没去凭吊香玉了。您能陪我去哭她一场吗?”于是二人一同走到牡丹穴边,流泪悼念了好长时间。大约一更过后,先是绛雪收泪劝慰,黄生才止住悲痛。

又过了几天的一个晚上,黄生正在书斋中寂然独坐,绛雪忽然笑着快步走进来,说:“报告您个好消息:花神为您的至情所感动,要让香玉再次降生到这下清宫中来啦!”黄生一下站起来,又惊又喜地问:“什么时候?”绛雪说:“那可不知道。大约总不会太久吧!”第二天清早,绛雪临走时,黄生拉住绛雪嘱咐说:“我这一回可是为你才回下清宫来的,你可别老让人孤零零煎熬啊!”绛雪笑笑,答应着走了。

过了两天,绛雪并没有来。黄生便跑到耐冬树下,拥抱着树,摇动着,抚摩着,低声呼唤绛雪的名字。但是没有回声。黄生便跑回书斋,抓起一把艾草,在灯下捆扎起来,准备去烤灼耐冬,逼绛雪出来。正捆扎间,绛雪突然闯进来,夺过艾草一扔,生气地说:“你要恶作剧,给人烙个疮吗?我要跟你绝交了!”黄生笑了,上前拥抱住她。两人刚坐下,香玉忽然笑盈盈,悄没声息地进来了。黄生抬头一见,登时热泪盈眶,急起身拉住她的手。香玉一手拉着黄生,一手拉着绛雪,三人相对悲泣一阵。就坐之后——真奇怪,黄生觉得自己的手掌空空的,好像并没有握着什么一样,便惊奇地问香玉,香玉流泪回答说:“过去我是花中的神,所以凝实,有形体;现在成了花中的鬼,所以虚若无物了。今天我们虽然能够会面,你不必以为是真的,只当作梦中相会吧。”倒是绛雪在一旁说:“妹子来得太好了,我快要被你家男人纠缠死了!”说罢告辞而去。香玉和黄生继续谈笑叙情,黄生觉得她像从前一样亲切可爱,可是亲近偎倚之间,总像影子一般虚幻缥渺,因此闷闷不乐,香玉也深感遗憾,就告诉他:“你用白蔹碎末掺些硫磺再兑上水,每天往我原先的穴坑里浇一杯,明年今日便可报答你的恩情了。”说罢,也告辞而去。

第二天,黄生到白牡丹穴边一看,果然冒出牡丹嫩芽来了。黄生便按照香玉的嘱咐,天天浇水、培土,还在四周修起一圈雕栏护着它。香玉晚间来时,对黄生十分感激。黄生打算干脆把牡丹移栽到老家去,香玉劝阻说:“不,我现在体质太嫩弱,经不起折腾损伤了。况且,万物生长,各有定所。我本来是不该生长在你胶州老家的,违背了反而促短寿命。只要咱俩相亲相爱,合好的日子自然会到来的。”黄生又埋怨绛雪不来,香玉说:“你一定要她来,我有妙法。”说着便领黄生举着蜡烛来到耐冬树下,她先捡起一根细草,张开手沿树身自下而上量到四尺六寸,按捺住这个部位,让黄生双手一齐给树挠痒,很快就见绛雪从树后绕出来,笑骂着说: “死妮子真坏,刚回来就助纣为虐吗?”说着三人手挽手来到书斋。香玉赶忙道歉:“姐姐切莫见怪,求姐姐暂且陪伴一下黄郎,一年后就决不敢麻烦打扰了!”从此绛雪也常来陪伴黄生。

黄生眼看着牡丹嫩芽一天天长大起来,茁壮而又旺盛,到暮春时已长到二尺多高了。他回老家时,便给道士一些钱,请他一定天天浇灌护理。第二年四月,黄生回到下清宫。牡丹恰好有一朵含苞欲放呢。黄生站在花旁,流连忘返,注视着,只见它微微摇动,开张,一会儿开得圆盘一样大,一个三四指高的小小玉美人儿端坐在花蕊中央,转瞬间飘然而下,落地就像人一般高,亭亭玉立,流光素雅,竟是香玉,笑容可掬地说:“我忍着风吹雨淋等待您来,您怎么来得这么晚哪!”两人来到书斋里,绛雪也闻讯赶来,开玩笑说:“天天代人作妇,现在好了,我可退而为友了。”三人饮酒叙谈,言笑尽欢,直到半夜,绛雪才告辞。黄生、香玉夫妻二人又恩爱美满,一如当初了。

后来,黄生的妻子去世,黄生便长住在下清宫里,不再同家。这时,牡丹已很高大,树干像人的胳膊一样粗壮。黄生常指着白牡丹说:“将来我要把灵魂寄留在这里,就在你的左边!”香玉、绛雪接茬儿笑他,说:“可别到时忘了你的诺言!”

过了十多年,黄生忽然病危,他的儿子从老家赶来探望,不禁哭泣起来。黄生自己倒很坦然,笑着说:“这是我的生期,又不是死期,你哭什么呢!”又转向道士说: “将来牡丹花下有一个红芽冒出来,一长五片嫩叶,那就是我。”说罢便不再作声。他儿子用车把他拉回家去,他便溘然长逝了。第二年,牡丹花下果然冒出一根又肥又旺的红嫩芽,果然是五片小叶。道士觉得神奇灵验,更加注意浇水护理。仅仅三年.这株牡丹就长到几尺高,主干有两只手合围那么粗,格外茂盛,只是不开花。老道士死后,弟子不知道爱惜,竟把它砍掉了。不久,白牡丹也枯死,耐冬树也死了。


\subsection{1.11.31   三 仙}
\label{\detokenize{p00_u5176_u5b83/_u767d_u8bdd_u804a_u658b_u5fd7_u5f02:id454}}
有个书生去金陵赶考,经过宿迁县时,遇到三个秀才,言谈超逸旷达。书生便买来酒,请他们聚谈。三个各自介绍自己的姓名,一个叫介秋衡,一个叫常丰林,另一个叫麻西池。四人开怀痛饮,十分快乐。一直喝到天黑,介秋衡说:“我们还没尽东道主之谊,先叨扰客人一顿丰盛的酒宴,实在于理不当。我们住的地方距此不远,请客人前去住宿。”常麻二人也站起身,拉着书生,叫上仆人一块前去。

到了县城北山,忽然看见一座院落,门口绕着一道清溪。进入家门,见房屋甚是整洁。三秀才喊小童掌上灯,又叫人安排下书生的随从。麻西池说:“过去都是以文会友。现在考期临近,不能虚度了今夜。我有个主意,咱们拟四道题目,用抓阉的办法,每人抓一个,文章完成后方可喝酒。”大家都同意,分别拟个题目。写下放到案几上,每人抓一个后就在案几上构思写作。二更没完,四人都已脱稿,互相传换着品评。书生读了三秀才写的文章,佩服至极,草草抄下藏到怀里。这时,主人拿出好酒,用大杯劝客。书生不觉大醉。主人便领他到另一座院子里住下。书生醉得来不及脱鞋,穿着衣服倒头便睡下了。

第二天,书生一觉醒来,红日高照,四下一看并没有房屋,自己和仆人睡在山谷里,心中大惊。见旁边有个深洞,水从洞里缓缓流出,惊讶得不知怎么办好。看看怀里,三篇文章都在。下山询问当地人,才知道那洞叫“三仙洞”。洞中有蟹、蛇和蛤蟆三种仙物,最灵验,经常出洞游逛,人们往往会碰到他们。书生进了考场,三个题目都是三仙写的文章,书生因此高中了解元。


\subsection{1.11.32   鬼 隶}
\label{\detokenize{p00_u5176_u5b83/_u767d_u8bdd_u804a_u658b_u5fd7_u5f02:id455}}
济南历城县的两个衙役,奉县令韩承宣之命,去别的郡办事,年底才返回。路上碰到两个人,衣着打扮也像是公差,便一块同行。交谈中,二人自称是郡里的捕快。衙役说:“济南城的捕快,我们认识十之八九,你们两位却从没见过。”二人说:“实话告诉你们:我们是城隍庙的鬼隶,要去泰山向东岳大帝投送公文。”衙役便问:“有什么公事?”回答说:“济南将有大劫,报送的公文就是应死人的姓名和数目。”衙役惊骇地询问死人的数目,鬼隶说:“我们也不太清楚,大约将近一百万人。”衙役又问时间,回答是“大年初一”。二衙役惊得面面相觑,计算着赶到济南时正是除夕。回去恐怕遭难,拖延返回又怕受县令责罚,鬼隶说:“违了期限是小罪,把命丢了却是大祸,应该赶快躲到别的地方,先不要回去。”衙役听从了鬼隶的劝告。

不长时间,清兵大举南下,屠戮了济南城,杀了一百万人,死尸堆积如山。二衙役因逃避得以幸免。


\subsection{1.11.33   王 十}
\label{\detokenize{p00_u5176_u5b83/_u767d_u8bdd_u804a_u658b_u5fd7_u5f02:id456}}
高苑人王十,在博兴县贩盐,夜里被两个人抓获。王十以为是当地大盐商的巡逻士卒,抛了盐想逃走,脚却怎么也迈不动,于是被捆住。王十衷恳不已,二人说:“我们不是盐铺中的人,是阴间鬼卒。”王十大,更加惧怕,乞求让自己先回家,同妻儿告别。鬼卒不让,说:“这次捉你去不是让你死。不过是暂时使唤使唤罢了。” 王十便问:“什么事?”鬼卒答道:“阴司中新阎王上任,见‘奈河’已淤平,‘十八狱’中的茅坑都满了,叫捉人世间的小偷、贩私盐的和铸私钱的这三种人去淘河,捉乐户去刷厕所。”王十只得跟着鬼卒走了。

进入一座城市,来到一个官衙中,见阎王端坐在上面,正在稽查生死簿。鬼卒禀报说:“捉了一个贩私盐的,叫王十。”阎王往下一看,发怒说:“贩私盐的是指那些上漏国税、下坑百姓的大盐商,像世上贪官奸商所说的贩私盐的,都是天下的好老百姓。穷人竭尽微少的资本,去挣点赖以糊口的利钱,怎么算‘私’呢?”罚两个鬼卒再去买四斗盐,连同王十原来的那些,一起代送到王十家中。又留住王十,给一根蒺藜骨朵,让他和鬼卒一起监督河工。

鬼卒领着王十来到“奈河”,只见淘河的人夫,都用布子遮体,川流不息像蚂蚁一样多。又见河水又浑又红,臭不可闻。淘河的人都赤裸着身子,手持竹筐和铁锹,在河水里出没,打捞朽骨烂尸,满满地装在筐子里,再背上岸边。水深的地方,就沉下水去打捞。动作稍慢点,鬼卒们就用蒺藜骨朵痛打脊背或大腿。一块监工的鬼卒给王十一颗像豆粒大小的香丸,让他含在嘴里,才领着他走到河边。王十发现高苑的那个大盐商也杂在人夫中,就特别“照顾”他,进河时打背,上岸就敲腿,吓得那个盐商常常机在水里不敢出来,王十才作罢。

过了三昼夜,人夫死了一半,河才淘完。以前的那个鬼卒仍然送王十回去。一到家,王十豁然醒来。起初,王十贩盐一直没有返回,天亮后,王十的妻子打开门,见两袋盐放在院子里,却不见王十。让人到处寻找,发现王十已死在路上。抬回家中,还微微有气,众人都不解是什么缘故。等到醒了过来,王十才说明了缘由。高苑的那个盐商在前天也死了,到此时也苏醒过来,被蒺藜骨朵打过的地方,都成了大疮,全身腐烂化脓。臭得让人不敢靠近。王十故意去拜访他,盐商看见他,还把脑袋缩到被子里,像在“奈河”中一样。过了一年,盐商才好了。从此后,再不经商了。


\subsection{1.11.34   大 男}
\label{\detokenize{p00_u5176_u5b83/_u767d_u8bdd_u804a_u658b_u5fd7_u5f02:id457}}
成都书生奚成列,娶了一妻一妾。妾姓何,小名叫昭容。原配妻子很早就去世了,又续娶了一个姓申的,特别嫉妒凶悍,经常虐待何氏,连同奚成列也受连累,整天吵闹不休,搅得一家人没法过日子,奚成列一怒之下,离家出走了。

奚成列走后,何氏生了个儿子,取名叫大男。丈失一去不返,申氏更加排斥何氏,让她分家另住,计算着日子供给口粮。大男渐渐长大,粮不够吃,何氏只得靠纺线织布挣钱来贴补家用。一次,大男路过私塾,见学童们吟诵文章,琅琅上口,非常羡慕,也想读书。母亲觉得孩子还太小,姑且先送到私塾中长些见识。大男十分聪慧,读会的文章超过其他学童一倍。塾师很惊奇,情愿不要酬金教他读书,何氏便让儿子正式拜师,入了私塾,自己略微给塾师一点学费。过了两三年,大男就精通了全部经书。

一天,大男从私塾回来,对母亲说:“私塾里有五六个同学,都跟父亲要钱买饼吃,惟独我为什么没有父亲呢?”母亲说:“等你长大了。再告诉你。”大男着急地说:“我才七八岁,什么时候能长大呀?”母亲哄他说:“你上私塾路过关帝庙时,就进去叩拜,让关老爷保佑你快快长大。”大男信以为真。此后每经过关帝庙必定进去叩拜。母亲知道后,便问:“你都祝愿些什么呀?”大男笑着说:“只祝愿关老爷明年便让我像十六七岁那样大!”母亲笑儿子太纯真。但说也奇怪,从此后大男的身量和学问都长进迅速,到十岁,看上去已像是十三四岁的样子了。下笔能成文章,连塾师也改不动一个字。一天,他又对母亲说:“过去你说等我长大了,就告诉父亲的去向,现在可以说了吧?”母亲摇头说:“还不行,还不行!”又过了一年多,大男俨然是成年人了,益加询问父亲的下落,何氏迫不得已,便将往事一一告诉了儿子。大男悲痛不已,想要去寻找父亲。母亲说: “儿还太年幼,你父亲是死是活还不知道,匆忙之中哪里就找得到呢?”大男一语不发,自己走了。到了中午也没回家,何氏急忙去私垫询问塾师,说是早饭后就没来。何氏大惊,出钱雇了人,到处搜寻,却杳无踪影。

大男从家里出走后,毫无目标地沿路奔跑,自己也不知要到哪里去。路上恰巧碰到一个人,要到夔州去,自称姓钱,大男便一路讨着饭,跟着他前往。钱某嫌他走得慢,替他租了匹驴骑着,不久便花光了全部盘缠。到了夔州,二人吃饭时,钱某暗在饭中下了迷药。大男吃了后,昏迷过去,不醒人事。钱某将他驮到一座寺庙中,假称是自己的儿子,路上得了病,又花光了路费,情愿卖给僧人挣点盘缠。寺僧们见大男长相不俗,都争着买。钱某拿到钱后,扬长而去了。寺僧给大男灌了些水,才把他稍微弄醒了过来。庙里的长老听说这件事后,就去探望大男,很惊奇他的长相,详细询问后,才得知事情的经过。十分可怜他,赠给路费,让大男走了。

有个泸州的秀才,姓蒋,考试落第归来,途中碰见大男,问知缘故,非常赞许他对父亲的孝敬,便带着他一块同行。到泸州,让大男住在自已家里,一个多月里,多方打听访查。有人说福建有个商人姓奚,大男便辞别蒋秀才,要去福建。蒋秀才赠给他衣服鞋帽,同村的人也凑钱资助他,大男便又上路了。路上碰到两个布商,也要去福建,邀请大男一块走。走了几程路,布商窥探到大男钱袋里有银子,便将他引到一处无人的地方,捆住手脚,将钱袋子抢走了。正好有个福建永福县的陈姓老翁,经过这里,发现了大男,替他解开绳索,用车子运到家中。陈老翁极为富有,各地的商人,大都出自他的门下,老翁嘱托南来北往的商人代为寻访奚成列。又留住大男,让他陪伴自己的儿子读书。从此后,大男就住在陈老翁家,再不外出流浪了,但此地离成都太远,跟老家越发难通音讯了。

何昭容失去儿子后,一个人生活了三四年。申氏日益减少给她的费用,想以此逼她改嫁。何氏却矢志不嫁。申氏便将她强卖给一个重庆商人。商人将何氏劫到家中,到了夜晚,何氏用刀自伤。商人不敢再逼,等她伤好后,又将她转卖给一个盐亭地方的商人。到盐亭县后,何氏仍然宁死不从,又用刀自刺心窝,至于从伤口里看见了内脏。商人非常恐惧,只得替她敷药疗伤。伤好后,何氏请求商人让自己出家做尼姑。商人说:“我有个同行,天生不能行房事,一直想找个女人理理家务。这跟做尼姑也没两样,还可以让我稍挽回些本钱。”何氏答应了。商人便用车子将她送了去。进入大门,商人的同行迎出门来,何氏一看,竟是奚成列!原来,奚成列从家里出走后,早已弃文从商。盐亭商人因为他没有妻室,所以想将何氏赠给他。二人相见,悲喜交集,各自述说分别后的经历和苦难。奚成列才知道还有个儿子一直在寻找自己,没有回家。便嘱托客商同行们,代为访查大男。何氏从此后由奚成列的妾变成嫡妻了。只是何氏以前倍尝艰辛,染上多种疾病,再不能劳作,便劝奚成列纳妾。奚成列有了前番的教训,不愿再娶。何氏便说:“你放心。我如想和别人争床第之欢,几年来,早已跟了别人生儿育女了,还能和你有今天吗?况且,以前别人强加给我的苦难,至今心有余痛;我又怎能再把苦难强加给别人呢?”奚成列于是嘱咐一个同行,为自己买个三十多岁的老妾。过了半年多,同行果然买了老妾回来。进入家门,奚成列一看,买来的老妾竟是原来的嫡妻申氏!申氏也认出了奚成列,两人都惊骇不已。

原来,申氏自丈夫出走,又卖了何氏后,独居了一年多。哥哥申苞让她改嫁,申氏顺从了哥哥。但田产却被奚家的子侄们占住,不允许申氏出售。申氏只得卖了自己的东西,换了数百两银子带到哥哥家。有个保宁地方的商人,听说申氏嫁妆丰厚,就用重金引诱申苞,将申氏娶了去。没想到商人已经年老无用,不能再有床第之欢。申氏怨恨哥哥,从此不安于家,又是上吊,又是投井,将商人闹得无法忍受。商人一怒之下,将她的财物搜掠一空,要卖了她给人作妾。没想到人家都嫌申氏太老,没有要的。后来,商人要到夔州去,便带着申氏一起前往,正好碰上奚成列的同行要买老妾,二人一谈即妥,商人便将申氏卖了后自己走了。申氏见了奚成列,又惭愧,又惧怕,一语不发。奚成列询问同行,才知道了事情的经过。便对申氏说:“假设你在保宁嫁的是壮年男子,我们就再也没有相见之日了,这也是天数啊!但我今天买的是妾,不是娶妻,你可先拜见昭容,行嫡庶之礼!”申氏认为这是自己的耻辱,不愿行礼。奚成列骂道:“过去你作正房,是怎样的来?”何氏忙劝免了,奚成列不让,抄起棍棒,逼着申氏行礼。申氏迫不得已,只得向何氏行了拜见礼。但此后却始终不屑于奉承何氏,自已在别的屋子里劳作。何氏全部宽容下来,也不忍心去检查她是勤是懒。奚成列每次和何氏饮酒谈天,往往让申氏在一边支使,何氏总是让丫鬟代替,不让她在前面侍奉。

一次,正值陈嗣宗做了盐亭县的县令。奚成列和同村一人发生了小争执,那人便到县衙告他“逼妻作妾”。陈县令不准诉状,将那人赶出了大堂。奚成列很高兴,晚上正在私下和何氏颂扬县令的恩德,忽然有小童叫着敲门,进来说:“县令陈公来了!”奚成列十分惊骇,急忙寻找衣服鞋子,县令已到了卧室门口。奚成列越发惊疑,不知怎么办才好。何氏仔细看了看县令,急忙出门,说:“这是我的儿子大男!”说着便大哭起来。陈县令也伏在地上悲痛哽咽。原来,大男改随了陈老翁的姓,起名嗣宗,已经做了官了。

起初,陈嗣宗自京都科考返回,绕路赶到老家,才知道两个母亲都已改嫁,内心极度哀痛。同族的人得知昔日的大男已经显贵,便将他家的田产房舍全部退回。陈嗣宗留下仆人经营,希望有朝一日父亲能回来,自己则返回了福建陈老翁家。不久,陈嗣宗被任命为盐亭县令。但他一心要再去寻找双亲,想辞官不做,陈老翁苦苦劝阻,才作罢。正好来了个算卦的,陈嗣宗便让他给算算。算卦的算了算说:“小者居大,少者为长,求雄得雌,求一得两。去做官大吉大利。”陈嗣宗听说,便去盐亭上了任。因为找不到父母,立誓居宫不吃荤腥。这天,有个村人告状,看到状子中写着奚成列的名字,陈嗣宗暗自惊疑,秘派心腹人细细访查,果然是父亲!便乘深夜微服私访,竟意外地连母亲也一块找到了,心中更加相信算卦的算得神。临走时,嘱咐不要宣扬,拿出二百两银子,让父亲治办行装,返回成都。

奚成列赶回老家,只见房屋全新,家里仆役、马匹众多,已经成了高门大户了。申氏见大男已经富贵,也就越发收敛了些。她哥哥申苞认为不合理,又打官司,为妹妹争嫡妻的位子。官府查知实情,大怒,说:“你贪图钱财,让你妹妹改嫁,已经换了两个丈夫,还有什么脸争过去嫡妻的位子!”将申苞狠狠地鞭打了一顿。从此后,何氏、申氏的名分益加明确了下来。申氏把何氏当作妹妹看待,何氏也乐意把她当作姐姐,衣服饮食,从不独占。申氏起初还怕她会报复,到现在更加愧悔。奚成列也原谅了申氏过去的过错,让内外家人都称她“太母”,只是不能像嫡妻那样封“诰命”罢了。


\subsection{1.11.35   外 国 人}
\label{\detokenize{p00_u5176_u5b83/_u767d_u8bdd_u804a_u658b_u5fd7_u5f02:id458}}
己巳年秋天,岭南从外洋漂来一艘大船,上面有十一个人,都穿着用鸟羽毛做成的衣服,华丽多彩,自己说:“我们是吕宋国人。在海上航行时,遭遇大风,船被打翻,死了好几十人。只剩下我们十一个,抱着巨木,随波漂流到一个大岛上,才幸免于难。在岛上待了五年,每天捉虫逮鸟吃,夜晚就藏在山洞里,编织羽毛当衣服穿。一天,忽然又飘来一只船,船橹和船帆都没了。可能也是被大风打翻的船,我们便爬上这只船想返回去,风却把我们送到了澳门。”巡抚便上疏奏闻皇帝,送他们返回祖国。


\subsection{1.11.36   韦 公 子}
\label{\detokenize{p00_u5176_u5b83/_u767d_u8bdd_u804a_u658b_u5fd7_u5f02:id459}}
韦公子,是咸阳官宦人家的子弟,为人放荡好色。家中凡有点姿色的奴婢、仆妇无不被他奸污过。他曾携带数千黄金发誓要找遍天下名妓。凡是繁华热闹有妓女的地方,他都要去看看。那些不怎么出众的妓女,他睡上两晚就离开了;而特别中意的名妓,则往往要逗留上好几个月。

韦公子的叔父韦公,也是名宦。年老辞官回家,痛恨韦公子的德行,请了个有名的塾师,逼迫他和弟兄们一块闭门读书。韦公子本性难移,夜晚等塾师睡熟后,跳墙逃走,去嫖妓女,天明才返回,习以为常。一夜,跳墙时摔折了胳膊,塾师才知道这事,便告诉了韦公。韦公大怒,将韦公子臭揍一顿,直打得他爬不起来才用药治伤。伤好后,给他订下戒约:读的书能比其他弟兄多一倍,文章也写得好,就不禁止他外出游荡;否则,再私自外出,仍如前次一样痛打。但韦公子最聪慧,读书经常超过塾师规定进度,仅几年,考中了举人,便想破戒。韦公却约束得更紧,公子到京都去,韦公给随行的老仆一个日记簿,让他记下公子每天的一言一行。因此,连续数年,韦公子一直不敢干出格的事。后来又考中进士,韦公对他的约束才稍微放松了一点。此后韦公子每去嫖妓时,还惟恐叔父知道,一进入妓女居住的偏僻小巷,便假称姓魏。

一天,韦公子路过西安,见到一个戏子,名叫罗惠卿,十六七岁,生得非常秀丽,犹如漂亮的女子。韦公子很喜欢,晚上留住他鬼混,赠送了许多财物。听说罗惠卿新娶的媳妇很有韵昧,私下暗示他带了来。罗惠卿面无难色,痛快答应,夜晚果然带了妻子前来,三人同床而睡。韦公子十分眷恋罗惠卿,一直留了好几天,商量着要带他回家,便询问他的家口。罗答道:“母亲早已去世,父亲还在。我原不姓罗,母亲年轻时是咸阳韦家的奴婢,后被卖到罗家,四个月就生了我。倘若能跟公子回去,也可察访韦家的情况。”韦公子大惊,忙问他母亲的姓,回答说“姓吕”。韦公子惊骇万分,出了一身冷汗。原来他母亲正是被韦公子私通后才卖给罗家的婢女。韦公子哑然无言,挨到天明,送给他许多财物,劝他改行,自己假称还要到别的地方去,回来时再叫着他同行,脱身走了。

后来,韦公子做了苏州县令。有个乐妓叫沈韦娘,生得娴椎美丽,韦县令十分喜爱,留住她奸宿,调戏她道:“你小名莫非是取自‘春风一曲杜韦娘’吗?沈韦娘回答说:“不是。我母亲十七岁时是苏州名妓,有一咸阳来的公子,和您同姓,在我母亲处逗留了三个月,两人订下了婚誓。公子离去后,八个月我母亲生下了我。因此取名叫韦,实际是我的姓。公子临别时,曾赠一枝金鸳鸯,现在还在。没想到公子一去再无音讯,我母亲愤恨忧郁而死。我三岁时,被一个姓沈的老太太抚养成人,所以改姓了她的姓。”韦县令闻言,既恼羞,又惭愧,无地自容。沉默了一会儿,顿生一条毒计。忽然从床上起来,点上灯,招呼韦娘一块喝酒,却暗在杯中下了剧毒。韦娘酒才下咽,即倒地呻吟,众人急忙看时,已气绝身亡。韦县令叫来戏子乐工们,把韦娘的尸体交给他们,又重重赏赐财物。但韦娘平生交好的都是些有钱有势的人家,听说韦娘暴死,都鸣不平,收买戏子们,激他们向韦县令的上司告状。韦县令惊慌失措,只得倾囊行贿,到底还是被以浮躁为由罢了官。返回老家时,才三十八岁。

从此后,韦公子闭门思过,很后悔以前的丑行。但妻妾们,却都没有子女,想过继叔父的孙子为嗣。韦公因为他家满门无品行,恐怕自己的子孙也染上恶习,虽然同意过继,但须等他老了以后。韦公子听说,大怒,想收留罗惠卿作儿子,家里人都说不可,才作罢,又过了几年,韦公子忽然大病,常常拍打着自己的心口说:“淫婢嫖妓的,不是人啊!”他叔父听说后叹息道:“这大概是要死了!”便将自己次子的儿子送到他家,让孙子早晚问安。过了一月多,韦公子果然一命呜呼了。


\subsection{1.11.37   石 清 虚}
\label{\detokenize{p00_u5176_u5b83/_u767d_u8bdd_u804a_u658b_u5fd7_u5f02:id460}}
邢云飞是河北顺天府人,喜欢玩赏石头,见到形态特别优美的玩石,自己从不惜代价收买。一次,偶然到河中打鱼,水中有一块东西把网挂住了。他潜伏到水底把它捞上来,一看,是块尺把长的方石头,四面玲珑剔透,峰峦叠起,秀美异常。邢云飞非常高兴,如同得到了无价的珍宝。带到家中,用紫檀木雕了一个底座,把石头安在上边。陈设在桌子上。每当天将下雨的时候,石头的每一个细孔中,都有云烟生出,远处观望,如同在上面塞了白色的棉絮。

一个有权势的土豪,来到邢云飞家中,要求观看一下石头;他一见到石头,就把石头递到健仆手中,策马扬鞭而去。邢云飞无可奈何,只有顿足悲愤罢了。那仆人扛着石头走到河边,将石头放在桥上休息,忽然失手,石头掉到河中。土豪愤怒地用鞭子抽打仆人,马上出钱雇佣善长水性的人,到水中打捞。但是,他们想尽一切办法,到处搜寻,始终没有见到。最后,只好贴了一个愿出重金悬赏打捞石头的约书,就走了。自此以后,到水中打捞石头的人,每天都把河挤满了,最终仍然没有一个人得到。后来,邢云飞来到石头掉落的地方,望着滔滔的河水呜咽悲泣。只见河水清澈见底,而石头仍然还在水中。邢云飞大喜,脱去衣服跃入水中,抱着石头从河底浮出,把它带回家,再不敢将石头摆放在客厅中,就另打扫一间清洁的屋子,把石头陈设在那里。

一天,忽然来了一位老头敲门,要求看一看石头,邢云飞假托说石头已经丢失了。老头笑着说:“客厅里陈设着的不是吗?”邢云飞便把他请到客厅里,以证实确是丢失了。等到老头子与邢云飞走到客厅里,那块石头果然陈设在客厅的几案上。邢云飞惊愕地说不上话来。老头子用手抚摸着石头说:“这是我家的旧物,丢失了已经很久了,今天才知道它原来在这里。我既然找到了,那就请你归还我吧!”邢云飞很窘迫,便与老头子争论谁是石头的真正主人。老头子笑道:“既然是你家的东西,有什么验证?”邢云飞回答不上来。老头笑说:“我本来就识得,此石前后共有九十二个孔窍。那个大孔中有五个字,是:‘清虚天石供’”邢云飞细细审视,果然如同老头说的,孔窍中刻有小字,细如粟粒。只有仔细看,才能辨认清楚;又数它的孔窍,也像老头子说的那样。邢云飞没有话说,只是执意不给。老头笑着说:“谁家的东西,凭你来作主啊。”拱拱手便走了。邢云飞把他送出门去;回到屋里一看,石头不见了,大吃一惊。心疑是老头干的,急忙去追赶老头。而老头在慢慢地走着,还未走远。跑上前去拉着他的袖子苦苦哀求。老头说:“奇怪啊!那么一块大石头,能用手握着藏到袖子里吗?”邢云飞知道老头子是神人,强拉着他回来,跪在他面前请求还给他。老头就问:“这块石头,究竟是你家的,还是我家的?”邢云飞说:“确实是属先生你的,但我求先生割爱啊!”老头说:“既然是这样,那么石头本来就在这里。”邢云飞进到内室,则石头仍然放在那里。老头说:“天下的宝物,应该给与那些真正爱惜的人。这块石头能自己选择主人,老汉我也高兴啊。然而这块石头急于自我显露,他出世过早,而恶劫的运气还未消除。我确实是想把它带走,等三年后再奉赠你。但是,既然你一定想留下,应当减少你三年的寿数,这样这块石头才能自始至终与你相伴,你愿意吗?”邢云飞说:“愿意。”老头于是用手指捏石头的孔窍,石窍像泥一样软,随手闭塞。老头捏闭三窍,说:“石头上的孔窍数,也就是你的寿数。”老头子欲去,邢云飞苦苦地挽留他,但老头坚决辞别;邢云飞问他姓字,他也不说,就去了。

时间过了一年,邢云飞因为有事出门去,夜间有小偷到他的房问,别的什么东西也没丢失,惟独把那块石头偷了去。邢云飞回来,见石头丢失,悲痛伤心得要死。他到处访察、购求,但没有一点踪影和头绪。又过了几年,偶然到报国寺去,见到有卖石头的,走近一看,就是自己丢失的那块石头,邢云飞准备认走自己的石头。但卖石头的很不服,因而扛着石头告到官府。官府问:“你有什么证据啊?”卖石头的能说清楚石头上的孔窍数。邢又问卖石头的其它特征,却茫茫然说不出来了。邢云飞寻于是说明这石头窍中的五个字及三个孔窍被捏闭的指痕。邢云飞的情理,得到伸张。当官的于是想以棍杖责打卖石头的人,卖石头的人解释说,这是他用二十两银子在集市上买来的,这才把他放了。邢云飞得到石头后,用锦帛把石头包裹起来,藏在木柜子中,偶尔观赏,必先烧香以后才拿出来。

有一位尚书,想花百两银子的价格购买邢云飞的石头。但邢云飞回答:“就是万两黄金也不卖。”尚书很生气,就借故其它的事,陷害邢云飞,把他关在监狱里。为了把邢云飞救出来,家里人便典卖田产。尚书于是托人传话给邢云飞的儿子,儿子又把情况告诉了邢云飞,邢云飞宁愿以死殉石,也决不给这个尚书。邢云飞的妻子于是与他的儿子偷偷商量,把石头献给了这个尚书。邢云飞出狱以后才知道这件事,他骂妻子打儿子,屡次想自杀,都被家人觉察而未死成。一天,夜间,邢云飞梦见一伟丈夫来,自称是:“石清虚。”告诉邢说:“不要难过。我只不过与君分别年余。明年八月二十日清晨,可到崇文门,用两贯钱就可赎回来。”邢云飞得到这个梦示后,很高兴,特别记住这个日子。再说那块石头到尚书家以后,再也没有孔窍出烟雾的灵异,时间一久,尚书也就不以此石为珍贵。第二年,这位尚书以获罪而被削职,接着就死了。邢云飞按期到崇文门,尚书家中人把石头偷出来,正在寻找买主,邢云飞见了用两贯钱买了回来。

后来,邢云飞八十九岁了,就自己准备好送葬的寿材、寿衣,又嘱咐自己的儿子,必定用这块石头殉葬。不久,邢云飞果然死了,他儿子就遵照他生前遗嘱,把石头给埋到坟里。大概过了半年时间,小偷把坟墓挖开,把石头盗走了。邢云飞的儿子知道了,但也无法去搜查寻找。过了二三天.邢云飞的儿子携带着仆人走在路上,忽然见到两个人,跌跌撞撞,满头大汗,对着天空自已认罪说:“邢先生,不要相逼,我二人把石头拿去,不过卖了四两银子罢了。”邢云飞的儿子便捉住两个偷石者,送到官府,一审讯就伏罪了。问石头哪里去了,说已经卖给一位姓宫的了。把石头取回来,长官玩弄着爱不释手,竟想得到这块石头,命令把石头寄存在官库中。差役把石头举起,忽然石头掉在地上,碎成几十片,众人无不失色。长官于是就用重刑处死了两个偷盗石头的贼。邢云飞的儿子将石头的碎片拾起,回去后,仍然埋到邢云飞的坟里。


\subsection{1.11.38   曾 友 于}
\label{\detokenize{p00_u5176_u5b83/_u767d_u8bdd_u804a_u658b_u5fd7_u5f02:id461}}
曾老翁,是昆阳的世家大族。老翁刚死去还没入敛时,两眼中忽然泪出如汁,老翁的六个儿子都不解是什么缘故。次子曾悌,字友于,是县中名士,见此情景,认为不吉利,告戒弟兄们各自谨慎,不要让父亲死了后还感到痛心。但弟兄们却有一半讥笑他迂腐。

原来,老翁原配妻子生了长子曾成,长到七八岁时,母子二人都被强盗掳去。续娶后,生了三个儿子:曾孝、曾忠、曾信,妾又生了三个儿子:曾悌、曾仁、曾义。曾孝因为曾悌等三人都是庶出,十分鄙视,不和他们来往,拉拢曾忠、曾信,结成帮派。有时和客人喝酒,曾悌等经过堂下,也傲不为礼。曾仁、曾义都很气愤,和曾友于商量,也跟他们为仇,曾友于不听,百般宽慰。曾仁、曾义年龄还小,哥哥既不同意,也就罢了。

曾孝有个女儿,嫁给了本县一姓周的人家,后来病死了。曾孝便叫上曾友于,要去周家问罪。友于不愿去,曾孝很生气,命曾忠、曾信纠集本族中的无赖子弟,去捉了周妻,横加毒打,抛粮摔碗,盆盆罐罐砸了个一干二净。周家告了官府,县令大怒,将曾孝等拘拿了去,下在狱中,要申报郡府,革去功名。友于为弟兄们担心,自己去见县令投案。对友于的品行,县令一向器重,看在他的面上,诸兄弟们才没受多少苦。友于又到周家,代表弟兄们负荆请罪,周家也看重友于,官司才算了结。但曾孝回家后,并不感激友于。不长时间,友于的母亲张夫人去世。曾孝等三弟兄也不穿丧服,照旧欢宴喝酒。曾仁、曾义气愤不过,友于说:“这是他们无礼,对我们有什么损害?”等入葬时,曾孝等又守住父亲的墓门,不让张夫人合葬。友于没办法,只得将母亲暂时葬在墓道中。又过了不长时间,曾孝的妻子亡故。友于招呼曾仁、曾义过去赴葬,二人说:“老一辈的丧礼他都不讲,还讲什么小一辈的丧礼!”友于再三劝告,二人不听。友于只得自已前去,到选葬时,哭得十分伤心。此时,却隔墙听见曾仁、曾义又是敲鼓,又是奏乐,曾孝大怒,纠合诸弟兄去殴打二人,友于操起根棍子跑在前面。曾仁先觉到不好,立即逃走了。曾义刚要跳墙,被友于从后面一棍打下来。曾孝等人上前拳棍交加,往死里殴打。友于见状,忙用身子护住弟弟。曾孝大怒,责骂友于。友于说:“责打曾义,是因为他太无礼,但他罪不至死。我不偏袒弟弟的过错,也不帮助哥哥的凶暴。你如还没出气,就打我吧!”曾孝掉过棍来就打友于,曾忠、曾信也跟着,打骂声、痛叫声震惊了邻居。大家忙都跑过来劝解,曾孝才悻悻地走了。友于挨了打,毫不怨恨,扶着拐杖到哥哥曾孝家请罪。曾孝却将他赶了出去,不让居丧。曾义也被打得遍体鳞伤,水米不进。曾仁悲愤不已,写下诉状,告了曾孝等不为庶母出丧。县令接状发签,捉拿了曾孝、曾忠、曾信,让友于陈述事情经过。友于因为脸被打伤,无法去县衙,呈文禀报县令,哀求息事宁人。县令便销了此案,不再过问。曾义不久伤也好了。从此后,双方仇怨日深。曾仁、曾义都年小体弱,常遭毒打,抱怨友于说;“人家都有弟兄,就我们没有!”友于生气地说:“这话是我应该说的,你们说什么?”又苦劝两个弟弟忍耐,二人始终不听。友于便锁好门窗,携带妻子儿女借住到别的地方,离家五十多里路,希望从此后耳根清静,再不管闲事了。友于在家时,虽然并不帮着弟弟们,但曾孝等也有顾忌。友于走后,曾孝等稍不如意,就跑到曾仁、曾义的家门口高声辱骂,连去世的母亲也跟着受辱。二人估量着敌不过,只有关门锁户,一心想找个机会杀了他们,拚个你死我活。每出门,怀里都揣着利刃。

一天,被强盗掳去的长兄曾成,忽然带着家眷逃了回来。曾孝等三兄弟因为分家已久,一块商量了三天,竟无处安置他。曾仁、曾义暗喜,将长兄叫到自己家中养着,又去告诉了友于。友于十分高兴,忙回家来,三弟兄共同匀出田产、房屋,让长兄住下。曾孝等却认为友于三人是买好送人情,又愤怒地跑上门来叫骂。曾成长期沦落在贼寇中,养成了勇武刚猛的脾气,此时不禁大怒,骂道:“我回家来,你们没有一个人肯倒出间屋子,幸亏三个弟弟念手足之情。现在你们上门叫骂,是想赶我走吗?”冲出家门,用石头将曾孝打翻居地。曾仁、曾义见机,各持棍棒、一涌而上,捉住曾忠、曾信痛打一顿。曾成又到县衙告状,县令命人询问友于,友于只得去拜见县令,低头无语,只是流泪。县令征求他的意见,友于说:“求大人给个公断!”县令便判曾孝等都拿出财产,曾老翁的家业由七人平均分配。从此后,曾仁、曾义与曾成更加互相爱护尊敬,谈及葬母一事,三人都伤心地落了泪。曾成怨恨地说:“如此不仁义,真是禽兽!”便想开坟,将庶母与父亲合葬。曾仁跑了去告诉友于,友于匆忙回家,劝阻长兄。曾成不听,订下日子,开墓改葬。到了那天,曾成在墓前摆上祭品,又一刀砍去了墓边一棵树的树皮,对六个弟弟说:“谁不披麻戴孝,就如同此树!”大家唯唯听命。一家人痛哭着重新为张夫人发丧,一切按礼仪进行毕。此后,弟兄们相安无事。但曾成性子暴烈,动不动就打骂弟弟们,对曾孝尤其严苛。惟有看重友于,即使是盛怒之下,只要友于来到,一句话就烟消云散。曾孝行事,曾成总是看不顺眼。曾孝因此无一天不去友于家,暗地里对着友于咒骂长兄。友于委婉地劝解,还是不听。友于受不了他的骚扰,只得又将家迁到三泊,离家越发远了,也就渐渐地很少通音讯了。弟兄们虽都害怕曾成,时间长了也就习惯了。

又过了三年,曾孝已是四十六岁的人了。生了五个儿子,长子继业、三子继德,是嫡妻生的;次子继功、四子继绩是妾生的;一个奴婢还生了个儿子,叫继祖,都已长大成人。也效仿父亲过去的做法,分别结成帮派,整天争斗不休,连曾孝也制止不了。曾继祖没有亲兄弟,年龄又最小,兄长们谁都对他又打又骂。继祖的岳父家距三泊不远,一次,去拜见岳父时,绕道看望叔父友于。进入家门,见叔家两个哥哥一个弟弟,正在弦歌诵读,那种和睦亲近的样子,令继祖感慨万千,便住在叔家,一连几天不说回去。叔父催促,就哀求叔父同意自己住在这里。友于说:“你住在这里,你父母都不知道,所以让你快回去。我岂是吝惜那一碗饭吗?”继祖只得返回。过了几个月,继祖带着妻子去给岳母拜寿,告诉父亲说:“我这次去就不回来了。”父亲询问缘故,继祖流露了要借住到叔父家的意思。父亲担心和他家夙有嫌隙,恐难以久住。继祖说:“父亲太过虑了,我二叔可是圣贤之人!”于是携妻去了三泊。友于为他打扫了房子,让他住下,当儿子一般看待,让他和长子继善一块读书。继祖最聪慧,寄居叔家一年多,就考进云南郡学。此后,更是与继善关门苦读,十分勤奋,友于非常喜爱他。

自从继祖去了三泊后,家中弟兄们更加不睦。一天,为了点小事,继业又辱骂庶母。继功大怒,将继业一刀刺死。官府拘捕了继功,严刑拷打,不几天便死在狱中。继业的妻子冯氏还整天以骂带哭,继功妻刘氏听见,恼怒无比,骂道:“你家男人死了,我家男人就活着吗?”持刀进入继业家,将冯氏又杀死了,自己投井而亡。冯氏的父亲冯大立,痛愤女儿惨死,率领自家子弟,衣服里暗藏兵器,去捉拿住曾孝的妻子,剥下衣服,在路上痛打。曾成大怒说:“我家死人如麻,冯家怎敢又如此?”吼叫着冲出家门,曾家子弟随后,将冯家打了个落花流水。曾成首先抓住冯大立,割下了两个耳朵,冯大立的儿子见状急救,被继绩用铁棍一下横扫,打断了双腿。冯家人都被打伤,一哄而散。只剩下冯大立的儿子躺在路边呻吟,曾成用胳膊夹着他,扔到冯家村外,自己回来了。又让继绩去县里自首,冯家的状子正好也到了县里,于是曾家子弟尽被拘拿,只有曾忠逃走了。跑到三泊,在友于家门外徘徊不敢进。正好友于领着儿子继善和侄子继祖科考归来,看见曾忠十分吃惊,问到:“弟弟怎么来了?”曾忠还没说话,已经涕泪交流,长跪在路边。友于忙拉住手,把他拽进家内,询问后才得知家里发生的变故,大惊说:“这可怎么办!一家人都凶横暴戾,我早预料到大祸不远了!否则我怎会躲到这里?但我离家已久,与县令久不通声气,现在就是一路跪着去哀求,也只会受辱罢了。只希望冯家父子重伤不死,我们爷三个侥幸有考中举人的,这场大祸倒还能消解。”于是,留住曾忠,白天一块吃,晚上一起睡,曾忠很是感激,又十分惭愧。住了十几天,见他们叔侄亲如父子,兄弟如同胞手足,不禁凄然落泪,说:“现在才知道自己从前不是人啊!”友于很高兴他能悔悟过来,两人相对不禁心酸悲伤。不长时间,人报友于父子同榜考中举人,继祖也中了副榜,全家大喜。也不去赴“鹿鸣宴”,先赶回老家省视祖坟。明代末年,科甲最重,冯家听说友于高中,也自收敛了些。友于又托亲友送给冯家许多财物、粮食,让他们买药治伤,官司才算了结了。曾家全家人都哭泣着感激友于,恳求他搬回老家来。友于和弟兄们焚香立誓,以让他们都改过自新,然后将自己家迁了回来。继祖仍想跟着叔父,不愿回自己家。曾孝对友于说:“我没有德行,不该有光宗耀祖的儿子。弟又善于教诲,就让他做你的儿子吧。等他有了长进,再请赐还给我。”友于答应了。

又过了三年,继祖果然中了举人。友于便让他搬到父亲家住,夫妻二人痛哭着离去。不几天,继祖有个儿子才三岁,又逃到友于家,藏在继善屋里,不肯回去。捉了回去就逃回来,曾孝只得叫继祖分家另过,和友于作邻居。继祖把自己家开了个门,通向叔父家,一早一晚跟往常一样问安。此时,曾成已经老了,家里的事都由友于作主。从此后,全家和睦,兄弟友爱,孝敬父母,家风一天天好起来了。


\subsection{1.11.39   嘉 平 公 子}
\label{\detokenize{p00_u5176_u5b83/_u767d_u8bdd_u804a_u658b_u5fd7_u5f02:id462}}
嘉平县某公子,生得容貌俊秀,风流潇洒,才十七八岁年纪。一次,他到郡里去考秀才,偶然路过一家姓许的鸨母开的妓院,见门内有个年轻的美貌女子,便禁不住呆呆地凝视着她。那女子微微一笑,点了点头,公子便凑上前去跟她搭话。女子问:“你住在哪里?”公子告诉她住宿的地方。又问:“住处有别的人吗?”公子回答说:“没有。”“我晚上去拜访你,不要让人知道。”

公子回到住所,到了晚上,将童仆都支走了。那女子果然来到,自己介绍说:“我小名叫温姬,”又说:“我爱慕你俊美风流,所以背着鸨母来了。我愿意和你订下终身!”公子也十分高兴。从此后,女子两三夜就来一次。一天晚上,女子冒着大雨来了,进门后脱下湿衣服,扔到衣架上;又脱下脚上的靴子,让公子替她擦去泥巴,自己上床去钻到被窝里。公子看那双小靴子,是用绣有五色花纹的新锦做的,全被泥水涂脏了,感到很可惜。女子说:“我不是故意让你干这种下贱的活,我是要你知道我对你的一片痴情。”听到窗外雨声不止,女子信口吟了句诗道:“凄风冷雨满江城,”让公子续下句。公子推辞说不懂诗,女子说:“公子这样一表人才,怎么会不懂诗呢?真扫我的兴!”便劝他好好学习。公子答应了。

两人来往得越来越频繁,仆人们都知道了。公子的姐夫宋某,也是世家子弟,听说后,暗地里恳求公子让自己见见温姬。公子便告诉了女子,女子坚决不同意。宋某便藏到仆人房里,等到女子进来时,趴在窗子上往外偷看,看见女子的模样,神魂颠倒,不能自持,急忙推门出来,女子已起身,翻过墙去走了。宋某非常想念,于是备下厚礼去见许鸨母,指名要会温姬。鸨母说:“是有个温姬,但已死多年了。”宋某愕然,回来后告诉了公子,公子才知道那女子是鬼。到了夜晚,公子告诉女子宋某的话,女子承认说:“是的。但你想得到美人,我想得到美丈夫,我们两人各遂所愿就足够了,管它是人是鬼干吗?”公子认为很对。

公子考完试,便返回家来,温姬也跟着。别的人都看不见她,只有公子能看得见。到家后,便让女子在书房住下。公子一个人睡在书房里。也不回卧室,父母都很奇怪。温姬回去探亲,公子才偷偷地告诉了母亲。母亲大惊,告诫公子跟她断绝关系,公子不听。父母十分担忧,想尽了办法也驱赶不走那女子。

一天,公子有事要交待仆人,写了张简帖,放到桌子上。帖子上有许多错别字:花椒的“椒”字错成了“菽”,生姜的“姜”写成了“江”,“可恨”写成了“可浪”。温姬见到了这张帖子,翻过来在后面写到:“何事‘可浪’?花菽生江。有婿如此,不如为娼!”于是告诉公子说:“我当初以为你是世家文人,所以不怕羞耻,自己找上门来。没想到你虚有其表!我只凭外貌取人,怎不被天下人耻笑呢!”说完,一下子就不见了。公子听了温姬的话,虽然很惭愧、悔恨,还是看不懂她的话题,仍然把帖子交给了仆人。结果,传出去后成了笑话!


\section{1.12   卷 十 二}
\label{\detokenize{p00_u5176_u5b83/_u767d_u8bdd_u804a_u658b_u5fd7_u5f02:id463}}

\subsection{1.12.1   二 班}
\label{\detokenize{p00_u5176_u5b83/_u767d_u8bdd_u804a_u658b_u5fd7_u5f02:id464}}
殷元礼,是云南人,善长用针灸治病。在一次战乱中,他逃到深山里。这时,天快黑了,离村庄又很远,他怕遭遇虎狼,远远看见前面路上有两个人,就快步赶了上去。

到了跟前,那两人问他从哪里来,殷元礼便讲了自己的姓氏籍贯,那两人拱手尊敬地说;“原来是良医殷先生啊,久仰先生大名!”殷元礼反问他们的姓氏,那两人自称姓班,一个叫班爪,一个叫班牙。他们又对殷元礼说:“先生,我们也是避难的。幸好有间石屋可以暂住,敢求先生屈尊前去;我们对先生还另有所求。”殷元礼高兴地跟他们走了。一会儿来到一个地方,靠近岩谷处有间石室。那两人点着木柴代替蜡烛,殷元礼这才看清他们的面容:相貌凶恶,身躯威猛,好像不是善良的人。又一想没别的地方可去,也只好听天由命了。这时他听到床上有呻吟声,仔细一看,是一个老妇人直挺挺地躺着,好像有什么痛苦。殷元礼问:“得了什么病?”班牙说:“就因为这个原因,敬请先生来。”于是拿着根火把照着床,请殷元礼到近前看看。殷元礼见老妇人鼻下口角有两个瘤子,碗那么大,并且说痛得很厉害,妨碍饮食。殷元礼说:“容易治。”就拿出艾团,为老妇人灸了几十壮,说:“过一夜就好了。”二班很高兴,烤鹿肉给客人吃;并没有酒和别的饭食,只有鹿肉。班爪说:“太仓促,不知道客人来,希望不要怪罪招待不周。”殷元礼吃饱后,就枕着石块睡下了。二班虽然很诚朴,但粗鲁莽撞,让人害怕。殷元礼翻来复去不敢睡熟,天不亮,就招呼老妇人,问她的病情。老妇人刚醒,自己一摸,瘤子已经破了,只留下两个疮口。殷元礼催促二班起来,用火照着,给老妇人敷上药末,说:“好了!”然后拱手告别,二班又拿出一条熟鹿腿送给他。

三年以后,殷元礼一次有事进山,路上遇到两只狼挡道,不能过去。这时太阳快要落山了,又来了一大群狼,殷元礼前后受敌。他被一条狼扑翻在地,好几只狼争抢着来咬他,衣服全被撕碎了。殷元礼想,这回是死定了。这时忽然窜过来两只老虎,群狼一见,四散逃跑。老虎大怒,一声怒吼,群狼害怕地都趴在地上,一动不敢动。老虎扑过去把它们全杀死,才走了。殷元礼侥幸逃生,狼狈地继续赶路,正在担心无处投宿,迎面走来一个老妇人。老妇人看到他的样子,连忙说:“殷先生吃苦了!”殷元礼悲伤地诉说了刚才的情景,问她如何认识自己。老妇人说:“我就是石屋中那个让你灸瘤子的病老太婆啊!”殷元礼才恍然大悟,便请求在她家借宿,老妇人领着他去了。

走进一所院落,里面已点起了灯火。老妇人说:“老身已等先生很久了。”接着拿出衣裤,让殷元礼换下破衣服;又摆上酒菜,热情招待。老妇人也用陶碗自斟自饮,她既健谈,又能饮酒,不像是一般女人。殷元礼问:“前些日子那两个男子,是老人家的什么人?怎么没看见他们?”老妇人说:“我那两个儿子去迎接先生,还没有回来,一定是迷路了。”殷元礼感激他们的情义,开怀痛饮,不觉大醉,酣睡在座位上。醒过来时,天已经亮了。四面一看,并没有房舍,他自己一个人正坐在岩石上。这时听到岩下发出牛吼一般的喘息声,走近一看,是只老虎正睡着没醒。老虎的嘴间有两块瘢痕,都大得像拳头。殷元礼害怕极了,恐怕老虎醒来,偷偷地逃跑了。这时才醒悟到救自己命的那两只老虎,就是二班。


\subsection{1.12.2   车 夫}
\label{\detokenize{p00_u5176_u5b83/_u767d_u8bdd_u804a_u658b_u5fd7_u5f02:id465}}
有一个车夫,推着辆沉重的车子正在爬坡。当到最吃力的时候,一条狼窜来咬住了他的屁股。车夫想放下车子,又担心翻车毁了货物,把自已也压在下面,只好忍住疼继续推车。等上了坡,狼已经从车夫屁股上啃下片肉逃走了。乘车夫无能为力的时候,偷尝他一片肉,这条狼也算是狡猾可笑了。


\subsection{1.12.3   乩 仙}
\label{\detokenize{p00_u5176_u5b83/_u767d_u8bdd_u804a_u658b_u5fd7_u5f02:id466}}
章丘的米步云,擅长扶乩算卦,每同人聚会。便召乩仙互相唱和。一天,有个朋友见天上微有云彩,忽得一联,请乩仙对下联。这一联是:“羊脂白玉天,”乩仙批字:“下联问城南老董。”大家怀疑乩仙对不上,所以乱说一气。

后来有事到城南去,发现一处地方,土红得像丹砂一样,很感奇怪。正好看见有个放猎的老翁在一边,便问他是什么土,老翁说:“这是猪血红泥地。”忽然想起乩仙的批词,十分惊骇;又问老翁的姓,老翁说:“我是老董。”能联对倒不奇,奇的是预先知道城南老董,这也太神了!


\subsection{1.12.4   苗 生}
\label{\detokenize{p00_u5176_u5b83/_u767d_u8bdd_u804a_u658b_u5fd7_u5f02:id467}}
龚生,是四川泯州的书生。到西安去参加科举考试,在旅社中休息,买了一些酒菜自斟自饮。一位身材高大的男子进来,坐下来和他攀谈。龚生举起杯劝他共饮,客人也不推辞,自称姓苗,谈笑粗俗豪放。龚生因他不甚文雅,以傲慢的态度冷遇他,酒喝完了,也不再去买。苗生于是说:“与穷读书人喝酒,真叫人闷死!”便起身到酒店买酒,提着一个很大的酒坛子进来。龚生推辞说不能再喝了,苗生捉住他的胳膊,劝他干杯;龚的胳膊被捉得疼痛欲折,迫不得已,干了数杯。苗生以盛汤的大碗自饮,笑着说:“我不善于劝别人喝酒,去留随你的便吧。”龚生马上收拾行李起程。大约走了几里路,马病了,躺在路上,龚生在路旁坐着。他正在为行李繁重所累,无计可施的时候,苗生赶到。问清楚了原因,把马背上的行李卸下来,交给仆人,自己用肩托着马肚子,把马扛起来,急速地走了二十多里,才到旅店。他把马放下,让马就槽吃草。过了一段时间,龚生和他的仆人才到达旅店。龚生感到很惊讶,认为他是神人,优厚地款待他。打酒买饭,让苗生一同吃。苗生说:“我的饭量很大,不是你能管饱的,共同饱饮一顿也就可以了。”喝完一坛子酒,苗生起身告别说:“您给马治病,还需要些日子,我不能等待了,我先走了。”于是就离开了。

后来,龚生参加考试完毕,与三四位朋友,共邀登华山游玩,大家在地上摆上酒菜作筵。正在欢宴时,苗生忽然到来。左手拿着一只大酒怀,右手提着个猪肘子,向地上一扔说:“听说诸位登临,我特意来与大家助兴。”大家起来,以礼相待,邀苗生一快坐下。酒喝得很痛快,都很高兴。大家想以联句为戏,苗生争辩说:“这样无拘束地喝酒,很高兴,何必去苦苦构思使自己苦恼。”大家不听,立下金谷酒令,对不上的罚酒三大杯。苗生说:“联句不佳者,当以军法论处。”大家笑着说:“罪不至于到这种程度吧!”苗生说:“若不被杀头,我这武夫也能凑几句。”坐在首席的靳生说:“绝巘凭临眼界空。”苗生便信口续道:“唾壶击缺剑光红。”下座的沉吟好久,也没续上,苗生拿起酒壶就自己斟酒。过了一会,以次序向下联下去,渐渐地越联越俗。

苗生大声喊道:“这就够了。如果你们还想让我活下去,就不要再联下去了。”大家不听。苗生再也不能忍耐了,就学龙吟声,一声长啸,震得群山轰鸣;又后仰前合地学狮子舞。诗兴被打乱了,大家才停止了联句,又举杯酌酒畅饮。酒喝得半醉时,众人又各自得意地朗诵起在考场上所作的文章,不断地相互赞扬,相互吹捧。苗生委实不想听这酸腐的腔调,就拉着龚生的手豁拳。二人各已胜负数次,但那些诵文章、吹捧的还没个完。苗生严厉地说:“你们的文章,我已经听得熟悉了。像这样的文章,只能在床头读给自己的老婆听,大庭广众之中,喋喋不休,叫人听了厌烦。”众人听了感到惭愧,更讨厌苗生的粗鲁莽撞,于是,就提高了声音大声朗读起来。苗生愤怒了,趴在地下大吼,立刻变成一只老虎,扑上去把众客人吃掉,然后咆哮一声,跳过山梁就跑了。所幸存者,只有龚生和靳生两人。靳生是这次乡试的第一名。

过了三年,靳生再从华阴经过的时候,忽然在路上见到嵇生,他也是当年在华山上被虎咬死的一位。靳生大为吃惊,欲策马扬鞭疾驰。嵇生捉住马笼头,使马走不得。靳生下马,问他想做什么?嵇生说:“我现今已成了苗生的伥鬼,干活很苦,必须再扑杀一位读书的人,才能把我替换出来。三天后,必有一个穿儒服戴儒冠的书生被虎咬死,但地点必须是在苍龙岭下,那才是我的替身。请君在那一天,多邀几位书生到那里,也就是替老朋友帮点忙啦。”靳生不敢申辩,答应下就分手了。靳生回到寓所,思考了一夜,但总也想不出个办法来。最后决定,豁上了背弃与嵇生的约定,听凭这伥鬼的处置吧。就在这时,恰巧自己的表亲戚蒋生来探望他,他就把自已遇到鬼怪的事讲述了一遍。蒋是出名的劣等生员,同县的秀才尤生考在他的前面,心中很妒忌。听到靳生所讲的事,暗地里即有谋害尤生的念头。马上写了一封信,邀请尤生共同到苍龙岭游览,自己穿上一身平民的衣服,尤生见了也不知有什么用意。来到苍龙岭的半山腰,便摆下酒菜,恭恭敬敬地请尤生饮酒。这天,恰巧知府也上了苍龙岭,知府是蒋生父亲的好朋友,听说蒋生在岭下,就派人去叫他。蒋生不敢着平民衣服去见知府,便与尤生把衣服帽子换过来。衣服还没有换完,老虎猛然扑来,把蒋生叼着就走了。


\subsection{1.12.5   蝎 客}
\label{\detokenize{p00_u5176_u5b83/_u767d_u8bdd_u804a_u658b_u5fd7_u5f02:id468}}
有一个贩蝎子的南方商人,每年都到临朐县收购很多蝎子。当地人拿着木钳子进入山中,掀开石块,寻找洞穴,到处搜捉蝎子出售。

一年,商人又来了,住在客店中。忽然感到心跳,毛骨悚然,急忙告诉店主人说:“我杀生太多,现在蝎子鬼发怒,要来杀我了!请快救救我!”店主人环顾室中,见有口大瓮,便让商人蹲伏着,拿瓮将他扣了起来。一会儿,有个人奔了进来,黄色头发,相貌狰狞丑恶。问店主人:“那南方商人哪里去了?”主人回答:“出去了。”那人到室内四下里看了看,又像闻什么东西一样抽动了好几次鼻子,便出门走了。店主人松了口气,说:“侥幸没事了!”忙打开瓮看看,那商人却已经化成血水了!


\subsection{1.12.6   杜 小 雷}
\label{\detokenize{p00_u5176_u5b83/_u767d_u8bdd_u804a_u658b_u5fd7_u5f02:id469}}
杜小雷,是益都县西山人,母亲双目失明。杜小雷十分孝敬老母,家里虽然贫穷,但母亲从不缺可口味美的东西。

一天,杜小雷要外出,买了肉给妻子,要她给母亲做水饺吃。妻子最忤逆不道,切肉时,将屎克螂杂在肉里。母亲吃水饺时,觉得味道恶臭,不能下咽,便藏起水饺来,等儿子回家。杜小雷回来后,问母亲:“水饺好吃吗?”母亲摇摇头,拿出水饺来给儿子。杜小雷掰开水饺一看,见馅里有屎克螂,大怒,想责打妻子,又怕母亲听见,便上床想办法。妻子问他,也不说话。妻子心中有愧,在床下徘徊,不敢上床。过了很久,听到很粗的喘气声。杜小雷躺在床上叱骂道:“还不睡觉,想挨揍吗?”床下却没有一点动静。起来点亮蜡烛察看,见一头猪在床下;仔细看看,猪的两脚还是人脚,才知道妻子变成了猪。

县令听说了这件事,命将猪拴了去,在城四门游街,以告诫众人。谭薇臣亲眼见过这事。


\subsection{1.12.7   毛 大 福}
\label{\detokenize{p00_u5176_u5b83/_u767d_u8bdd_u804a_u658b_u5fd7_u5f02:id470}}
太行人毛大福,是专治疮伤的医生。一天,他行医归来,路上碰到一匹狼,嘴上叼着个小包裹,见到毛大福,便将小包吐在地上,蹲在路边。毛大福拾起来一看,见里面包着几件金首饰。正感到惊异,狼又跃上前欢跳着,用嘴巴轻轻拉了拉毛大福的衣角就走开了;毛大福刚要离开,狼又回来拽住了衣服,像是要他跟它走。毛觉察到狼没有恶意,便跟着它去了。不一会儿,来到一个洞穴,见一匹狼正生病躺在地上。仔细一看,狼头顶上长了个大疮,已腐烂生蛆。毛大福立即明白了狼的意思,便为病狼仔细剔净蛆虫,又敷上药,才往回走来。此时,天已经晚了,狼远远地跟着送他。走了三四里路,又碰上几匹狼,咆哮着要围攻毛大福。毛非常恐惧,正在危急的时候,后面跟着的狼急忙跑来,到狼群中似乎说了些什么,群狼便都散去了,毛大福才得以安全返回。

在此以前,毛大福所在的县里有个叫宁泰的银商,被人杀死在路上,凶手一直没有抓获。正好毛大福出售从狼那儿得来的金首饰,被宁家的人认出是宁泰之物,将他扭送到了县衙。毛大福诉说了首饰的由来,县官不信,将他严刑拷打。毛大福冤枉至极,无法申辩,只得恳求县官让他去问问那匹狼。县官便派两个衙役,押着毛大福,进入山中,径直去那个洞穴找狼。狼却没回来,等到天黑也不见踪影,三人只得返回。走到半路,迎面碰上两匹狼,其中一匹头上的疮疤还在,毛大福一下子认了出来,便向它作揖说:“上次承蒙您赠我礼物,现在我因为那些礼物蒙冤受屈,您若不为我申辩昭雪,同去后我就被打死了!”狼见毛大福被绑着,愤怒地冲向衙役,衙役忙拔出刀抵挡。狼见状,便用嘴巴拱着地,长声嗥叫起来。刚叫了两三声,只见从山中窜出了上百匹狼,转着圈将衙役团团包围起来。衙役受困,大为窘迫。有疮疤的狼一跃上前,去咬捆着毛大幅的绳索。衙役明白了狼的用意,无可奈何地松开了毛大福,狼群才一起离去了。

回来后,衙役讲述了经过,县官深感惊异,但也没有立即释放毛夫福。过了几天,县官出巡,见一匹狼叼着只破鞋,放在道上。县官走了过去,狼又叼起鞋跑到前头,重新放到地上。县官很奇怪,命收起鞋子,狼才走了。返回后,县官命人秘密访查鞋子的失主。有人说某村有个打柴的,在山中被两匹狼穷追不舍,将他的鞋子叼跑了。县官将打柴的拘拿了来认鞋子,果然是他的。于是怀疑杀银商宁泰的凶手定是此人,一审问,果然不错。原来这个打柴的杀死了宁泰,抢劫了巨金,还没来得及搜出宁泰藏在衣服里的金首饰,便逃走了。结果首饰被狼叼了去,才发生了这件奇事。

从前,有个接生婆出门归来,碰到一匹狼等在路上,拉住她的衣服,像要她跟着走。接生婆跟着狼走到一处地方,见一匹母狼正难产。接生婆为它用力按摩,直到小狼生下,狼才放她返回。第二天,狼叼来鹿肉放到接生婆家的院子里以示报答。可见这类事是自古以来就多有发生的。


\subsection{1.12.8   雹 神}
\label{\detokenize{p00_u5176_u5b83/_u767d_u8bdd_u804a_u658b_u5fd7_u5f02:id471}}
太史唐济武,到日照去为一姓安的人送葬。路经雹神“李左车祠”,便进去游览眺望。祠前有个水池,池水清澈见底,里面有几条红鱼正安详地游动;其中一条斜尾巴的游上水面吃食,见人也不害怕。唐济武便拾起块小石子,要打它玩,一个道士急忙阻止。唐济武洵问缘故,道士说:“池里的鱼都是龙类,打它会招致风雹。”唐济武讥笑道士太穿凿附会,不听他的话,还是打了鱼。

从祠里出来后,唐济武继续坐车往东赶去。不一会儿,一块黑压压的云彩,像盖子一样,罩在唐济武头顶上,随他一块前行,棉子大小的冰雹簌簌地落下来。又走了一里多路,天才放晴。唐济武的弟弟唐凉武走在后面,追上哥哥询问,唐济武竟不知下过冰雹;又问走在前面的人,都说不知。唐济武笑着说:“这难道是广武君在作怪吗!”心中还没感到有多奇怪。

日照安家村外有座关圣祠,一个小商贩正在祠门外放下担子休息,忽然抛了两个篓子,直奔入祠中,拔下架子上的大刀挥舞起来,口里说道:“我是李左车,明天将陪同淄川的唐济武前来帮助安家送葬,先敬告主人一声。”说完,便清醒过来,并不知道自己说了些什么,也不认识唐济武是什么人。安家闻知,十分恐惧,村里离关圣祠四十多里路,急忙恭敬地备下祭品,到祠里哀恳祈祷,只求雹神怜悯,千万别屈驾前来。

唐济武赶到后,奇怪安家如此敬奉李左车,询问主人,主人说:“雹神一向最灵,常借活人的口说话,没一次不灵验的。如不虔减祷告阻止他来,那明天这里就要有大风雹了。”


\subsection{1.12.9   李 八 缸}
\label{\detokenize{p00_u5176_u5b83/_u767d_u8bdd_u804a_u658b_u5fd7_u5f02:id472}}
太学生李月生,是李升宇老翁的第二个儿子。李老翁非常富有,金子多得用缸贮藏,乡里人称他是“李八缸”。李老翁到了晚年,一病不起,便叫过两个儿子,给他们分金子。哥哥得十分之八,弟弟得十分之二。李月生不满,埋怨老父亲太偏向哥哥。老翁说:“我不是偏向谁,也不是喜欢谁不喜欢谁。家里还窖藏着金子,必须等到人不多的时候才能给你,你不要着急。”

过了几天,老翁病势沉重,李月生担心父亲一旦去世,就不知道藏金的下落了,瞅没人的时候,在床头偷偷询问老翁。老翁说:“人一生的祸福苦乐,都是命中注定的。你现在正享受着妻子贤惠的福气,不应当再多给你金子,免得再增加你的罪过。”原来。李月生的妻子姓车,为人十分贤惠,真有桓少君、孟光的美德,所以李老翁这样说。李月生苦苦哀求,老翁发怒说:“你还有二十年的磨难没受,即使现在给你千两黄金,也马上就完了。不到你山穷水尽的时候,别指望得到金子!”李月生性格忠厚孝敬,听老父亲这样说,便不敢再问。不常时间,老翁病危,接着就去世了。所幸哥哥贤良,营葬之事,也不和弟弟计较。

李月生为人一片天真,不吝惜财物,又十分好客,能喝酒。每天都要催妻子做三四次饭,治办酒席,招待客人,却不懂得治家理业。同村中那些无赖地痞,见他懦弱,经常欺凌他,因此家业逐渐衰落下来。幸亏生活困难时,哥哥多少接济一点,倒还不至于十分贫困。不久,哥哥年老病故,李月生失去了依靠,经常绝粮断顿。春天借贷,秋天偿还,地里种的粮食,刚上场就分完了,只好卖地为生,家业越发不可收拾。又过了几年,妻子和大儿子相继死去。李月生悲哀无聊,便又买了个贩羊人的老婆为妻,指望有了她生活能稍好点。新妻子姓徐,性子刚烈,每天都欺负他,以至于李月生都不敢和亲戚朋友有婚丧嫁娶一类的来往。一夜,梦见父亲对他说:“你现在的境况,算是山穷水尽了。过去曾答应给你金子,现在可以了。”李月生便问:“金子在哪里?”老翁说:“明天给你。”醒过来后,感到奇怪,还觉得可能是自己穷极了发生的幻想。

第二天,李月生挖土修墙,忽然挖出了巨金。至此才醒悟老翁过去说的“人不多的时候”是指全家人死亡近半的时候。


\subsection{1.12.10   老 龙 舡 户}
\label{\detokenize{p00_u5176_u5b83/_u767d_u8bdd_u804a_u658b_u5fd7_u5f02:id473}}
朱徽荫初任广东巡抚时,客商游人很多告无头冤状的。千里行人,忽然死不见尸;几人同行,也全都神秘地失了踪,像这样的案子积下了很多,没法究查。起初告状的时候,官府还行文追辑;状子一多,又没头绪,官府竟再不过问。

朱公到任后,一一翻阅旧案,见状子中称人已死的就不下一百多份;那些远离家乡,无人寻找的死者更不知有多少。朱公十分惊异哀伤,苦苦思索,废寝忘食,又走访遍了同僚和部属,还是没有一丝线索。于是,朱公便沐浴熏香,给城隍发去檄文,请求神灵帮助。既而睡下后,恍惚中见一个官员,穿着公服走进来,朱公便问:“你是什么官?”来人回答说:“城隍神刘某。”朱公又问:“有什么要说的吗?”城隍答道:“鬓边垂雪,天际生云,水中漂木,壁上安门。”说完就退下了。朱公豁然醒来,梦中的话还记得清清楚楚,但不解是什么意思。辗转反侧,思索了一晚,忽然大悟道:“鬓边垂雪是‘老’。生云是‘龙’,水上木当是‘舡’,壁上门是 ‘户’,合起来岂不是‘老龙舡户’吗?”原来本省东北地区,有两条河叫“小岭”和“蓝关”,都自老龙津发源,一直流到南海,岭外巨商大都从老龙津乘船进入广东。朱公便派遣武官,密授机谋,捉拿老龙津驾船的船户,陆续抓住了五十余人,全都不经上刑便招供了罪行。

原来,这些贼寇以舟渡为名,将客商骗上船去,或者下迷药,或者烧闷香,将客商弄得昏迷不醒,再剖腹放入石头,将尸体沉到水底。图财害命,极为狠毒惨酷。为冤死者昭雪后,远近欢腾,人们都编成民谣颂扬朱公的英明。


\subsection{1.12.11   青 城 妇}
\label{\detokenize{p00_u5176_u5b83/_u767d_u8bdd_u804a_u658b_u5fd7_u5f02:id474}}
费县高梦说做成都太守时,发生了一件奇案。有个从西边来的商人,客居在成都,娶了青城山的一个寡妇。不久,商人有事回老家去,过了一年多才返回来,夫妻刚一团聚,商人突然死了。同行们觉得事情蹊跷,告了官府。官府也怀疑寡妇与人私通,谋害亲夫,将她严刑拷打,苦苦逼讯,寡妇却始终不承认。押解到郡府后,也因为缺乏实证,只好将寡妇下在狱中,案子拖了很久没有解决。

后来,高梦说的官衙中有人生病,请来一个老医生诊治。交谈中,高梦说提起这件奇案,老医生突然问道:“那寡妇是尖嘴吗?”高梦说一楞,反问:“有什么说法吗?”老医生起初不肯说,再三询问,才道:“本地青城山周围有几个村落,村中妇女多被蛇交配过,生下的女儿都是尖尖的嘴巴,阴户中长着像蛇舌一样的东西。和男子淫荡时,蛇舌有时会伸出来,一插入男子阴管,男子便会立即脱阳死去。”高梦说听说,大感惊骇,但还不太相信。老医生又说:“这地方有巫婆,能用药使寡妇产生淫意,蛇舌自然会出来。是与不是,到时一看便知。”高梦说便按照老医生讲的,找来一个巫婆如法炮制,果然有蛇舌样的东西伸出来,才案情大白。高梦说便拟公文呈告上司,上司又重新检验过,才免了寡妇的罪,将她释放回家。


\subsection{1.12.12   鸮 鸟}
\label{\detokenize{p00_u5176_u5b83/_u767d_u8bdd_u804a_u658b_u5fd7_u5f02:id475}}
长山县有个姓杨的县令,为官极其贪婪。康熙乙亥年间,朝廷往西部边疆用兵,购买民间的骡马运送军粮。杨县令以此为借口,大肆搜刮,将地方上老百姓的牲畜抢了个干净。

周村是商人云集的地方,每逢集日,车水马龙。杨县令率领手下走卒,明火执仗,在集上抢夺了不下数百头牲畜,各地商人,无处控告。当时山东各县县令都因有公务全在省城里。正好益都县的董县令、莱芜县的范县令和新城县的孙县令在旅店里会聚到一起。有两个山西商人在门外大声喊冤。原来,两位商人有四头健壮的骡子,被杨县令抢了,出门在外,又远离家乡,丢失了财产,没法回家,恳求各位老爷给讲讲情。三县令觉得他们可怜,答应下来,于是便一块去拜访杨县令。杨县令置酒款待。酒席上,三人说明来意,请杨县令还给商人骡子,杨县令不听。三人苦劝,杨县令忙举杯劝酒,不让他们说下去,说:“我有一个酒令,不能对的罚酒。这个酒令必须是说一个天上的东西,一个地下的东西,还要说个古人。左问手拿什么东西,右问嘴里说什么话,随问随答。”自己先行令,说:“天上有个月轮,地下有个昆仑,有个古人叫刘伯伦。左问手拿什么东西,回答是‘手持酒杯’,右问嘴里说什么话,说是‘酒杯之外的事不要提’。”范县令接着说:“天上有广寒宫,地下有乾清官,有个古人叫姜太公。手持钓鱼杆,嘴里说是‘愿者上钩’。”第三个是孙县令,也说道:“天上有条天河,地下有条黄河,有个古人名叫萧何,手拿一本《大清律》,嘴里说是‘赃官赃吏’。”杨县令一听,脸上不自在,沉吟了一会儿。说:“我又有了一个:天上有座灵山,地下有座泰山,有个古人叫寒山。手里拿把扫帚,说是‘各人自扫门前雪’。”三人互相看看,脸上都有惭色。

忽然,一个少年从门外昂然进来,衣着华丽整洁,对四人举手行礼。大家一块请他坐下,拿大杯让他喝酒。少年笑着说:“酒先别喝。听见各位大人正行酒令。我也凑上一个。”大家便请他说,少年说道:“天上有玉帝,地下有皇帝,有个古人是洪武朱皇帝,他手持三尺剑,说是‘赃官应该剥皮’。”大家大笑。杨县令愤怒地骂道:“哪里来的狂徒竟敢如此!”命羞役抓起来。少年一跃,跳到桌子上,变成了一只鸮鸟,冲帘飞出,落到院子中的树梢上,回顾室中。口里作人笑声。主人忙拿东西打它,鸮鸟笑着飞走了。


\subsection{1.12.13   古 瓶}
\label{\detokenize{p00_u5176_u5b83/_u767d_u8bdd_u804a_u658b_u5fd7_u5f02:id476}}
淄川县城北村中,有口水井干了。村中有甲乙两人缒着绳子下到井中淘井。挖了一尺深,发现一具骷髅,不小心,将头打破了,嘴里含着块黄金,两人十分喜欢,将金子收到腰包里。继续往下挖,又挖出六七具骷髅。两人贪心不足,希望还有金子,便把这些骷髅头全都打碎,却再也没有。只发现旁边有两个瓷瓶、一件铜器。铜器有一抱大小,几十斤重,两侧有双环,不知是干什么用的,锈迹斑斑。瓷瓶也很古老,不是近时的式样。甲乙两人出井后,突然死了。一会儿,乙又苏醒过来,开口说道:“我是汉代人,遭逢王莽之乱,全家人投井而死,正好有点黄金,因此含在口中,并不是含敛之物,每个人都有。为什么把头颅全都打碎了?实在可恨!”大家听了,赶忙焚香烧纸祷告,并许愿重新殡葬,乙才好了。甲却再也没有活过来。

颜镇孙生听说了这件奇异的事,将铜器买了去。孝廉袁宣四得到一个瓷瓶。这个瓷瓶能预报阴晴天气。只要见瓷瓶上开始有一点湿润的地方,最初像米粒大小。越来越大,不长时间天便会下雨;湿润的地方逐渐消退,就云开天晴。另一个瓷瓶被张秀才家得到。这个瓷瓶能显示日期。每月初一,瓶上便起一个黑点,与日俱增,到了十五,黑点便布满了整个瓶身;过了十五,黑点又逐渐减少,到了月底最后一天,黑点全部消失,恢复为原来的样子。因为埋在地下久了,瓶口处粘上了一个小石粒,怎么刷也剔不下来,便用东西敲打,结果石粒下来了,把瓶口也打了个小缺口,也算是一件遗憾的事。据说将花泡在瓶中,花开花落,结的果实和树上结的没有两样。


\subsection{1.12.14   元 少 先 生}
\label{\detokenize{p00_u5176_u5b83/_u767d_u8bdd_u804a_u658b_u5fd7_u5f02:id477}}
韩元少先生还是生员时,一天在家,突然来了个官差,禀报说主人想延请他作塾师,但竟没主人的名帖;问他主人家的家族门第,回答也是含含糊糊。先生见官差带来的聘礼十分优厚,便答应下,约定了来接的日子,官差才走了。

到了那天,果然有车子来接先生。出门后,一路曲折绵延前行,走的路都是以前从没走过的。忽见前面有楼台殿阁,先生下了车进去,见像是藩王的官邸。到了学馆,有仆人纷纷摆上了丰盛的酒菜,劝客人自饮,却没主人陪同,先生十分不解。撤宴后,过来一个十五六岁的公子拜见先生,生得秀雅不俗,施完礼,又去了别的屋子,请教学业时才来到老师的住所。公子绝顶聪明,上课只要先生讲讲大意,便自己明白了。先生因不知公子的家世,心中十分疑惑。学馆里有两个童仆供先生使唤,先生私下问他们,都不回答。问主人在哪里,回答说主人太忙。先生又要求他们领着偷偷去见见主人,二人都不愿去。恳求了好几次,才勉强同意,领着先生来到一座大殿,听见里面传出审讯拷打声。先生忙从门缝里往里一瞅,见一个大王高坐在殿上,两阶下剑树刀山,都是传说中阎王殿的景象。先生惊骇万分,转身要走开,里面已经知道外头有人。阎王停下公事,将众鬼喝退,厉声呼叫小童。童仆脸上一下子变了色,恐惧地说:“我为了先生,惹祸上身了。”战战兢兢地急忙跑了进去,阎王发怒说:“你怎敢领人到这里来偷看?”用巨鞭重打小童。打完,叫先生进去,说:“我所以不见你,是因为阴阳两世路途隔绝。现在你既然已经知道了,不好再在这里。”便赠给酬金。让先生回家,说:“你将来能中状元,只是还有些磨难没受完罢了。”命一个青衣仆人牵着驴送先生回去。先生怀疑自己已经死了,仆人说;“怎么可能呢!先生吃的用的都来自人世,不是阴间的东西。”

回来后,先生又坎坷数年,后来连中会元、状元,阎王的话都应验了。


\subsection{1.12.15   薛 蔚 娘}
\label{\detokenize{p00_u5176_u5b83/_u767d_u8bdd_u804a_u658b_u5fd7_u5f02:id478}}
丰玉桂是山东聊城的一位书生,家里很贫穷,没有谋生的职业。明代万历年问,有一年发生了大灾荒,丰玉桂孑然一身到南边去逃荒。等到回家的时候,到了沂州就病了。他极力撑持着有病的身体走了几里路,来到了城南的一片乱葬岗子,越发疲累无力了,因此只好倚着一座坟墓躺下来休息。

忽然,他好像做梦似地来到了一个村庄里。有一位老翁从一家大门中出来,邀请他进去。这老翁家只有两间简陋的房屋,屋里有一位女子,年龄有十六七岁,面貌神态俊秀文雅。老翁叫她煮柏枝汤,用陶器盛了招待客人;又询问起丰生的籍贯、年龄,问完了,就说: “我姓李名洪都,祖籍山西平阳,流落居住在这里已经三十二年了。请你记住这里的门户,我家的子孙如来寻访,就麻烦你指给他们。老夫不敢忘你的恩德。我的干女儿慰娘,也不算丑,可以许配给你,等我的三儿子到来的时候,就叫他给你们主持婚事。”丰生大喜,拜谢说:“我今年二十二岁了,还没有婚配,承蒙你把女儿许给我为妻,固然很好;但什么地方能找到您的家人告诉他们呢?”老翁说:“你只要到北边的村子里去,等一个多月,自然就有人来,只求你不要怕麻烦啊。”丰生恐怕老翁说话没有信用,就要求说:“实话告诉您:我本来就穷得家徒四壁,恐怕日后不能像您所期望的那样。如果半路把我抛弃了,那是人所难以忍受的事。即使您没有许配婚姻的情义,我也不会不遵守答应您的诺言,您不如直接了当地说要我为您办点事好了。”李翁笑着说:“你要叫老夫信誓旦旦地向你发誓吗?我早就知道你家贫。这次订立婚约并非专门为了你。慰娘孤独无靠,依托在我这儿已经很久了。我不忍心听任她流离失所,所以把她许配给你。你何必疑心呢?”于是就拉着他的胳膊把他送出门去,拱了拱手关上门回去了。

丰玉桂一觉醒来,原来仍在坟墓边躺着,看看太阳,已经将近正午了。他慢慢地站起来,一步一步地走到村中。村里的人见了他都吃了一惊,认为他已经死在道旁一天多了。于是丰生顿时明白了李翁就是坟墓中的人,他把这事隐瞒起来没说,只求借间屋子住下。村里的人恐怕他再死了,没有人敢留下他。这村里有一位秀才和丰生同姓。听说了这件事,就跑去询问丰生的家世,原来丰秀才还是丰生的远房叔叔。丰秀才高兴地把丰生领到自己家中,给他吃喝治病。过了几天丰生就痊愈了。丰生讲述了梦中所遇见的情景,他叔父也很惊异,于是就让丰生住下等待着,静观事情的发展。住了不久,果然有位官人来到村中,访问他父亲坟墓的地点。自称是山西平阳县的进士叫李叔向。从前他父亲与同乡某甲一起出外经商,他父亲死在沂州,某甲就把他葬在一处乱葬岗中。回到家乡后,某甲也死了。那时李翁的三个儿子年龄都还小,长子叫伯仁,考中了进士,在淮南当县令,多次派人寻找父亲的坟墓,始终没有知道的人。次子叫仲道,考中了举人。叔向最小,也考中了进士,于是就自己出来寻求父亲的遗骨,来到沂州,到处寻访。这天来到了这个村子里,村里人都不知道,丰生就把他引到墓地,指给他看。李叔向不敢相信,丰生对他具体述说了自己所遇到的情景。叔向感到很惊奇,仔细看了看,两座坟墓紧紧靠在一起。有人告诉他说:三年前有一个做官的人,把他的小妾葬在这里。李叔向恐怕错挖了别人家的坟墓,丰生就把自己躺过的地方指给他看。李叔向便吩咐抬一口棺材放在坟旁,才开始挖掘。坟墓掘开,见到一具女尸,衣服装饰都腐朽了,而面色像活人一样。李叔向知道是挖错了,非常骇怕,不知道该怎么办。不料棺材中的女子顿时坐了起来,向门外看了看说:“三哥来了吗?”李叔向吃了一惊,走过去问她,原来她就是薛慰娘。李叔向脱下自己的外衣给她盖好,派人抬着她回到旅店。又急忙命人发掘旁边的坟墓,希望父亲也能复活。挖开以后,尸体皮肉尚存,用手一摸已僵硬干燥了。叔向悲哀不止,把尸体装殓入棺木中,作了七天法事超度亡灵。慰娘也穿一身孝服,像女儿一样祭奠。

一天,慰娘忽然对叔向说:“从前阿爹有黄金两锭,曾经分给我一锭作嫁妆。我因为孤弱一人,没有收藏的地方,便用丝线拴在腰里,而没带来,三哥你得到了没有?”李叔向不知道此事,就叫丰生返回去在墓穴中寻找,果然找到了一锭,和慰娘说的一样。叔向仍把那锭有丝线作为标志的黄金赠给慰娘。闲暇的时候,叔向就详细询问了她的家世。原来慰娘的父亲薛寅侯没有儿子,只有慰娘一个女儿,十分疼爱她。慰娘有一天从金陵舅父家回来,带着一个老婆子去雇船,驾船的是南京的一个专门保媒的人。当时有一个做官的人,任期满了要到北京去,托这个保媒的人给找一个美貌的侍妾。媒人跑了几家,没有一个中意的。为了这事驾着小船到广陵去物色,忽然遇上了薛慰娘,暗中就产生了一个害人的诡计,急忙招呼她们搭船过江。薛慰娘带的老婆子本来就认识他,因此就和慰娘一起上船渡江。中途,这个人把迷药放到食物中,慰娘和老婆子都中毒昏迷了。他就把老婆子推到江中,载着慰娘又返回了南京,用重金卖给了那个做官的人。慰娘入了门,这家的大老婆才知道,非常愤怒。慰娘又因中毒后头脑尚不清楚,不知道向她行礼,于是大老婆就鞭打她,并把她囚禁起来。北上三日以后,薛慰娘才完全清醒过来,婢女把事情经过告诉了她,慰娘大哭。一天夜晚,在沂州住宿,慰娘就上吊死了,他们就把她葬在乱葬岗中。慰娘在坟墓中,被群鬼欺凌,李翁时常保护着她,慰娘便拜李翁为义父。李翁说: “你命不该死,我一定给你挑选一个好女婿。”前些日子,丰生见了面后走了,李翁回来后对慰娘说道:“这个读书人品行可以信赖。等你三哥来了,替你主婚。” 有一天,李翁对她说:“你可以回去等侯着,你三哥快来了。”原来这就是李叔向发掘坟墓的那天。慰娘在服丧期间,对叔向追述了往事,叔向叹息了很久,就把慰娘当作妹妹,让她改姓李。略微置办了一些衣服物品,安排慰娘和丰生结了婚。叔向说:“我带的盘费不多,不能给妹妹办嫁妆,我的意思是带着你们一起回去,以安慰老母之心,怎么样?”慰娘也非常高兴。于是夫妻二人随着叔向,用车载着灵柩一起出发了。

到家以后,李母询问明白了事情的经过,喜爱慰娘胜过了自己的亲生女儿,安排她们在另一座院中居住。在办丧事过程中,慰娘对李翁的哀悼之情超过了他的儿孙。李母越发喜爱她,不让他们回聊城了,嘱咐儿子给他们买一座宅子。正巧有一个姓冯的要卖宅子,要价六百两银子。李家仓促之间未能凑足银子,暂时先把房契收下,约定日子交兑银两。到了日期,姓冯的早一步来了,正巧慰娘也从别院中来探望母亲,突然看到了冯某,觉得非常像那个驾船的人。冯某见到慰娘也很吃惊。慰娘急忙越过他走了进去。两位哥哥也因为母亲有点小病,都集合在这里,慰娘问:“厅前度步的那个人是谁?”李仲道说:“几乎忘了这件事,这人一定是前几天卖房子的人。”就站起来准备出去。慰娘阻止了他,把自已的怀疑告诉了他,叫他去仔细盘问这姓冯的。仲道答应着出去,冯某已经离开了,而巷子南边教私塾的薛先生却在那儿。仲道就问:“先生来有什么事?”薛先生说:“昨天晚上冯某请求我今天早些到府上来,给他写个文契并作保人。刚才在路上遇见他,说偶然忘记了一件事,暂时回去一趟,立刻就回来,叫我来这儿坐着等他。”少停了一会儿,李叔向和丰生都来了,于是互相攀谈起来。慰娘因为冯某的缘故,悄悄地来到屏风后偷看客人。仔细地看了看薛先生,原来是她的父亲,就突然从屏风后跑出,抱着父亲失声大哭。薛翁惊喜地流着泪说:“我儿从哪里来?”众人才知他就是薛寅侯,仲道虽然在路上常常遇见他,当初并不知道他的名字。到了这时候大家都非常高兴,对他讲述了慰娘前前后后的经历,设下酒席庆贺他父女团圆,因而留下他住了两晚。薛先生谈了自己的经历,原来他丢失了女儿后,妻子因为悲伤过度死了,他光棍一人无依无靠,就游学到了这里。丰生和他约定,购买了宅子后就把他接来同住。薛翁第二天去探看,冯某全家都逃走了,才知道杀害老婆子、卖了女儿的,就是这个人。冯某刚到平阳,做买卖发了家,但连年来赌博,日子一天天穷困,所以就卖他的住宅。卖薛慰娘的钱,也快花尽了。慰娘得到了好的归宿,也就不十分仇恨冯某了,只是选了个好日子迁入新居,也不再追究他逃到哪儿去了。李母经常馈赠慰娘财物,一切日用所需都供给他们。丰生于是就在平阳安了家,但需要按期回原籍参加各种考试,十分辛苦,幸而这一科乡试他考中了举人。慰娘富贵了以后,常常想念那老婆子是为了自己而死,想报答一下他的儿子。老婆子的丈夫姓殷,有一个儿子叫殷富,喜欢赌博,穷得没有立锥之地。有一天殷富在赌场上为赌注发生了争执,打死了人,就逃亡到了平阳,老远地来投奔慰娘。丰生把他收留在自己家中,询问他杀的那人的姓名,原来就是驾船的冯某。丰生惊骇感叹了好一会儿,就向殷富说明了情况,殷富才知冯某就是杀母的仇人,越发高兴,就在丰生家当了仆人。薛寅侯就在女婿家养老,女婿给他买了一个妾,生了一个儿子,一个女儿。


\subsection{1.12.16   田 子 成}
\label{\detokenize{p00_u5176_u5b83/_u767d_u8bdd_u804a_u658b_u5fd7_u5f02:id479}}
江宁人田子成,在一次过洞庭湖时,翻船淹死了。儿子田良耜,是明末进士,当时还在怀抱中。田子成的妻子杜氏,听到丈夫的噩耗,痛不欲生,服毒自尽。田良耜被庶祖母抚养成人,后考中进士,被派到湖北做官。过了一年多,改调湖南。走到洞庭湖,他想起了被淹死的父亲,痛哭而返,向上司禀报财力不及,请求辞官。上司不许,只将他降职为县丞,隶属汉阳府。田良耜推辞不去,院司再三督促,才勉强上任。到任后,他放荡不羁,常常遨游于江湖之间,不理政事。

一天,他乘小船出去游览。夜晚,船泊江边,听到岸上传来悠扬动听的洞箫声。兴致所来,便弃船上岸,乘着月光,望箫声传来的方向走去。大约走了半里路,见一片旷野中孤立着几间茅屋,隐隐透出灯光。近前从窗子里往里偷看,见里边有三个人正坐着喝酒。上座是一个秀才,三十多岁年纪;下座是一个老翁,打横坐着个吹洞箫的,是个少年人。少年一曲吹完,老翁击节赞赏。秀才却面对着墙壁,嘴里念念有词,像是在吟咏诗句,少年的箫声、老翁的赞赏声,仿佛充耳不闻。老翁忽道: “芦十兄一定有了佳作,请朗诵朗诵,让我们也欣赏欣赏。”秀才便长声吟道:“满江风月冷凄凄,瘦草零花化作泥。千里云山飞不到,梦魂夜夜竹桥西。”声音凄恻哀伤。老翁笑着说:“芦十兄又故态发作了!”顺手拿过一个大酒杯,斟满酒说:“老夫不会对诗,就唱首歌劝酒吧。”于是便大声唱道:“兰陵美酒郁金香,玉碗盛来琥珀光。但使主人能醉客,不知何处是故乡。”唱完三人都面现喜色,气氛才轻松起来。少年站起来说:“我看看月亮斜到哪里了?”从窗子里探出身子,忽然发现了田良耜,拍着手说:“窗外有人,我们的狂态都让人看到了!”便请田良耜进屋,大家彼此行礼,老翁请良耜坐到少年对面。良耜试试酒杯,是冷酒,推辞不喝。少年复又站起来,点着把苇草把酒温热,良耜也命随从出去买酒,老翁执意不许。又问良耜家世,田良耜详细说了。老翁恭敬地说:“原来是家乡的父母官。我妻子姓江,是本地人。”指着少年说:“这位是江西的杜野侯!”又指着秀才:“这是芦十兄,和您是同乡。”芦十自见了田良耜后,很是傲慢,也不行礼。田良耜问他道:“家住哪里?有这样高的才华,怎么以前没听说过?”芦十回答道:“我客居在外已经很久了,连亲属都不认识,真令人叹息啊!”话音十分哀伤。老翁忙摇手打断,说:“与佳客相聚,不赶紧喝酒,只管罗罗嗦嗦,叫人听厌烦了。”自己端起杯一饮而尽;又说;“我有个酒令,大家共行,不能的罚酒。一次掷三个骰子,其一的点数须与另两个点数之和相等,还要暗合一个典故。”自己先掷,是幺、二、三点,便唱道:“三加幺二点相同,鸡黍三年约范公,(《后汉书·范式传》:范式,字巨卿,山阳金乡人。与汝南张劭为友,两人同时归里,约定二年后的某日范式去张劭家看望。至期,张劭于家中准备鸡黍。范式果至。)朋友喜相逢。”第二个是少年,掷了两个两点,一个四点,客气道:“我学识浅薄,知道的都是俗典,请不要见笑。四加双二点相同,四人聚义古城中,(《三国演义》,刘备、关羽、张飞战乱中失散,后在古城相会。)兄弟喜相逢。”接下来是芦十,掷了两个幺点,一个两点,说道:“二加双幺点相同,吕向两手抱老翁,(《陕西通志》:吕向,字子回,唐朝人,少托于外祖母家。父亲长期远游在外,存亡未卜,多方找寻不到。后吕向官至翰林,一天自朝中回,路上碰到一位老人,恻然心动,问之正是他父亲,吕向抱父痛哭,将父亲迎回家中。)父子喜相逢。”最后是田良耜,掷的点数却与芦十一样,也唱道:“二加双幺点相同,茅容二簋款林宗。(《后汉书·茅容传》:茅容,字季伟,东汉陈留人。耕于野,与人避雨树下,众皆夷踞相对,容独危坐愈恭,郭林宗见而奇之,遂与共言,寄宿其家。次日茅容杀鸡为馔,林宗谓为己设,既而以供其母,自己以菜疏与客共饭。林宗深为感动,称为贤孝。簋,音鬼,古代盛食品用的器具。)主客喜相逢。”酒令行完,田良耜起身告辞,芦十站起来挽留道:“老乡情谊还没来得及倾吐,怎么就忙着走呢?我还想打听个事,请再坐会儿。”田良耜只得重新坐下,问,“要问什么?”芦十说:“我有个老朋友,在洞庭湖淹死了,和你是一个家族吧?”田良耜回答说:“是我父亲。不知你们是怎么认识的?”芦十解释说:“我们年轻的时候很好。他淹死那天,只有我在场,是我收敛了他的尸体,埋葬在江边。”田良耜闻听,流泪下拜,恳求指示父亲的墓所。芦十说:“你明天还到这里来,再和你说。也很好找,离这里有几步路,见坟头上长着十棵芦草的便是。”田良耜哭着拱手告别。

回到船上,田良耜一夜没睡。回想芦十的话,像句句都有深意。天刚明,便迫不及待地赶去,只见昨晚上的茅屋全没有了,十分惊骇。又到芦十指点的地方,果然有座坟墓,坟头上长着一丛芦草,数了数正好十棵,才恍然大悟:“芦十”原来是指十棵芦草。昨晚遇到的,是父亲的鬼魂。又详细打听当地人这座坟墓的来历。原来,二十年前,本地有个姓高的富翁,乐善好施,凡淹死的人,他都一一打捞上来,为他们修建坟墓,所以在这地方有几座坟。于是,田良耜便挖开坟,敛好父亲的遗骨,弃官返回了家乡。

回家后,询问祖母父亲的相貌,与“芦十”一模一样。江西的杜野侯,原来是田良耜的表兄,十九岁的时候在江中淹死了,后来他父亲流落到了江西。又醒悟母亲杜夫人死后,葬在竹桥之西,所以芦十的诗中有“梦魂夜夜竹桥西”的句子。只是不知那老翁是什么人,也无从打听了。


\subsection{1.12.17   王 桂 庵}
\label{\detokenize{p00_u5176_u5b83/_u767d_u8bdd_u804a_u658b_u5fd7_u5f02:id480}}
王樨,字桂庵,是河北大名府的世家子弟。有一年,他到江南游历,停船在长江边上。附近船上有个船家少女,漂亮极了,正坐在船头低着头绣鞋。王桂庵瞅了她好半天,那女子像是毫无察觉。王桂庵便高声吟诵王维的“洛阳女儿对门居”一诗,故意让她听见。

女子好像也懂了是为她吟诵的,但也不过略一抬头,瞥了一眼,又低头刺绣起来。王桂庵更加情思飞驰,忘情地把一锭金子扔了过去,恰好落在女子的衣襟上;女子依旧不抬头,顺手拾起,扔到岸上去了。王桂庵只好讪讪地把金子拣回来。他又拿出一副金镯扔过去,落在女子的脚旁,女子仍旧绣鞋,毫不理睬。不一会儿,船家从外边回来,王桂庵怕他发现金镯,正急得抓耳挠腮,却见女子从容地用脚把金镯勾来,遮掩过去了。船家上船后就催女子收拾活计,一边自己解开缆绳,开船顺流而去。

王桂庵望着远去的帆影,呆呆地坐在那里,心情十分怅惘。当时他刚丧妻不久,很后悔没有立即请媒人去和船家定下婚事。再到周围船上打听,都不知刚才那位船家的姓名。王桂庵赶紧让自己船上的艄公开船去赶,哪里还有那船的踪影!

不得已,王桂庵只好先过江办事。北返时再沿江查访,却依然不见消息。回家后,吃饭睡觉,都难以忘却那个船家女子的倩影。

第二年,王桂庵又到南方去,专门买了一条小船,住在江边,天天察看往来的船只。半年功夫,对这一带活动的船都熟悉了,惟独不见去年那条小船的踪影,而腰中的钱袋却渐渐空了,只好又回家来。这一回,王桂庵的思念之情更加急切,无论白天走路还是黑夜梦中,漂亮船家女的影子总是浮动在他的心头。

一天夜里,王桂庵做了一个梦:他忽然到了江边一个小村落里,刚走过几家门口,就见一家柴门朝南开着,院内稀疏的翠竹编成篱笆,花木繁茂,像是一座亭园。王桂庵迳直进去,不远处忽见一株高大的合欢树,满树红丝低垂,浓荫诱人。他不禁默念:元代虞集诗“门前一树马缨花”,写的大概就是这种景致吧?再走几步,眼前忽然出现了一处围着芦苇篱笆的光洁素朴的小院,院内北房三间,门关掩着;回头看见南墙边有一间小屋,一株红蕉掩映在窗前。王桂庵探身窥望,见屋内迎门一个衣架,上挂一条彩裙,知道是女子的闺房,急忙退了回来;但屋里人似乎已经发觉有人来了,就迎了出来。王桂庵一看,那俊俏的面庞,正是去年那个船家少女。王桂庵喜出望外。大叫道:“这不是也有相见的一天吗?”二人正要亲昵,女子的父亲突然回来了。王桂庵一惊,醒了过来,才知是一个梦。回想梦中景物,历历如在眼前。王桂庵便把这个美梦珍藏在心里,恐怕同别人说了,会破坏这美好的意念。

又过了一年多,王桂庵再次到江南镇江去。城南徐太仆,是王家的世交,请王桂庵去喝酒。王桂庵赴宴途中迷了路,误入一个小村,忽觉村中景物好像在哪儿见过似的。一家院门里,正有一株高大的合欢树,宛然是梦中曾见的情景。他惊喜极了,投鞭翻身下马,闯了进去。院内景物,果然与美梦无异。再往院内走,房舍格局也全符合。梦境既然应验,王桂庵不再犹豫,直奔后院小南屋而去,船家女果然正在屋中。她远远看见闯来的王桂庵,吃了一惊,急忙站起身用门扇遮住自己,呵斥道:“哪儿闯来的男子?”王桂庵进也不是,退也不是,似乎仍在梦中。女子见他已经站在房门边,便砰地一声把门关上了。王桂庵急得大叫起来:“您难道不记得那个扔镯人了吗?”接着,便倾诉了几年来的相思之苦,并且述说了梦中的预兆。女子隔窗询问了王桂庵的出身家世,王桂庵都如实告诉了她。女子说:“您既是宦门之后,家中自然早有美妻了,还要我去干什么呢?”王桂庵着急地说:“如果不是因为思念您的话,我早就娶妻了!”女子说:“如果真像你说的这样,足见你的诚心。我的心事虽然难向父母表白,却也已经违命回绝过好几家的婚聘了。那副金镯,我至今保存着,料想钟情者终究会有信息来的。今天不巧,父母到外婆家去了,眼看就要回来,您暂且回去,然后请媒人前来正式提婚,我看一定如愿以偿。可是假如你想非礼成亲,那就打错算盘了!”

王桂庵正要匆匆退出,女子又望着他的背影远远地喊道:“王郎!我叫芸娘,姓孟,父亲的表字是江篱。你可别忘了呵!”

王桂庵一边答应“记住啦”!一边跑出院门。

王桂庵到徐太仆家赴宴,因为心里有事,便早早结束筵饮,告辞回来,赶紧到小村里去拜见孟江篱。孟江篱很有礼貌地接待了王桂庵,在院中篱笆墙边设了桌凳请他就座。王桂庵谢座后,先作了一番自我介绍,然后便说明来意,恭恭敬敬地奉上一百两银子作为聘礼。不料老人摆摆手,说道:“对不起,我的女儿已经许配人家了。”王桂庵急得嚷道:“我打听得确确实实,明明是尚在待聘中,您为什么这样拒绝呢?”江篱老人平静地说:“我刚才说的,全是实话,不敢有半点撒谎。”王桂庵听了,立刻失魂落魄,垂头丧气地告辞出来。

王桂庵回到住处,左思右想,无处找媒人。一夜辗转反侧,不能成寐。回想在徐太仆家,本来是想把这事告诉他的,只因为害怕他耻笑自己娶一个船家女,没好意思开口;现在情急无奈。只得前去求他。于是天一明,便跑到太仆家,把情况告诉他。太仆笑了,说:“原来如此。不用急。这老头儿,我还与他有点瓜葛,他是我祖母的内侄孙。你为什么不早说呢?”王桂庵这才鼓起勇气,倾吐了几年来藏在心中的隐情。太仆一听,却诧异道:“江篱本是个贫苦农民,从来不以撑船为业,你是不是弄错了呢?”于是,他打发儿子大郎到孟家去询问。江篱老人解释说:“我家虽然贫穷,却决不是卖婚的人。昨天王公子以金银作媒,我觉得他大概以为我们穷人家见钱眼开,见利而动,所以不敢高攀这门亲事。今天你来,既是太仆公的意思,必定没错。但我这女儿很任性娇惯,好人家不合她的心意也往往拒绝,我得跟她商量商量,免得日后落下埋怨。”说着,起身走进内室。一会儿出来,拱手笑道:“现在完全可以照太仆公的意思办了。”于是二人约定吉日,大郎告辞回家,向太仆报告了喜讯。王桂庵治办了丰盛的嫁妆,到孟家交了彩礼,借徐太仆家的房子,举行了婚礼。

完婚后,住了三天,王桂庵带着芸娘,辞别岳父北返。途中夜间住在船上,新婚夫妻闲谈起来。王桂庵问芸娘道:“那一年就是在这一带遇见你的,当时就疑心你不像船家女。你那是乘船到哪里去呢?”芸娘说:“我叔父家在江北,那是我们借船去看望他。我家虽然贫寒,只够吃穿,没有积蓄,可是这种意外而来的财物,我们不贪恋。我笑你当时两眼瞪得圆圆的,一次又一次地想用金钱打动人心。听你吟诵古人诗句,知道是个风雅文士,可又疑心是轻薄浮浪子弟把人家当作荡妇挑逗呢。哼,假如让我爹发现了你那金镯,你就死无葬身之地啦!你看我爱财心切吗?”王桂庵听了笑道:“你固然聪明,却还是掉进我的圈套里了!”芸娘吃了一惊,忙问:“怎么回事?”王桂庵故意停住,笑而不言。芸娘一个劲地追问,王桂庵才说:“离家一天天近了,这事也不能再瞒你。实话告诉你吧,我家里是有妻子的,是吴尚书的女儿。”芸娘不信,王桂庵故意又郑重地说了一遍。芸娘听了,一声不响,突然起身跑出船舱,王桂庵急忙拖着鞋子往外赶,芸娘却已经跳进江中去了。王桂庵大喊救人,周围船只一阵骚动。然而但见江上夜色茫茫,星光点点,哪儿去找芸娘的影子呢?王桂庵号啕大哭,撕心裂肺,痛不欲生,悔恨莫及。他沿江出高价雇水手打捞芸娘的遗体,丝毫不见踪影。最后只好返回大名府家中,又是悲痛,又是忧愁,害怕岳父来探望闺女,那时如何交代?

恰巧,他的姐夫在河南做官,于是王桂庵到那里去住了一年多才回来。归途中遇上大雨,王桂庵到村子里一个农家去避雨,见那院中房舍整洁,一个老太太抱着一个婴孩在房厦下面逗弄玩耍。婴孩看见王桂庵进来,就扑过来叫他抱。王桂庵觉得有点怪,又见婴孩长得秀气可爱,便抱过来搁在膝上。老太太唤他,他也不去。一会儿,雨过天晴,王桂庵把这小家伙举起来交给老太太,走下堂阶,让仆人整装动身。哪知婴孩却哭闹喊叫起来:“爸爸走了!”老太太笑这孩子喊陌生人为爸爸,连忙呵斥制止,抱起他回室内去了。王桂庵正在等待仆人整治行装,忽见一个美丽少妇抱着那婴孩从室内屏风后面走出来。王桂庵愈看愈像芸娘,正在疑惑,芸娘已经骂起来:“负心汉!你自己丢下的这块肉,叫人怎么处置?”王桂庵这才明白,原来这个婴孩是他的儿子,不禁一阵心酸,也来不及问他们母子如何来到这里,先把当初的戏言解释表白了一番,芸娘方才转怒为喜,二人相对流下泪来。接着芸娘述说了事情的经过:这家主人莫翁,因为六旬无子。领着老婆到浙江南海普陀寺去进香。归途中在长江岸边停船时,正好芸娘顺流而下,恰好碰到莫翁的船舷。莫翁急忙让人把她打捞上来,抢救了一夜,芸娘才苏醒过来。莫翁见芸娘是良家女子,便高兴地认作干女儿,带回家来。过了几个月,想给她聘个夫家,芸娘表示不愿改嫁。到十个月上,芸娘生下一个男孩,取名寄生。事有凑巧,王桂庵到这家来避雨,这时寄生已经快满周岁了。王桂庵听了,大喜过望,于是重新卸车,入室拜见了老翁老太太,同样以岳父岳母相称。住了几天,夫妻携带儿子及莫老夫妇一起返回大名府。

一进家门,孟翁已来王家等待两月了。孟翁刚来时,王家人们情辞恍惚,使孟翁感到奇怪;现在相见,真相大白,大家高兴起来,孟翁才知道以前的闪烁支吾是事出有因的了。


\subsection{1.12.18   寄 生}
\label{\detokenize{p00_u5176_u5b83/_u767d_u8bdd_u804a_u658b_u5fd7_u5f02:id481}}
王寄生,字叫王孙,是郡中名士。小时,父母因为他在襁褓中就能识得父亲,认为他天生聪慧,所以十分钟爱。长大以后。出落得越发秀美,八九岁能写文章,十四岁考入郡学,立志要自己选择对偶。父亲王桂庵有个妹妹叫二娘,嫁给了秀才郑子侨,生了个非常聪明漂亮的女儿,起名叫闺秀。王孙见了闺秀后,十分爱慕,日思夜想,渐渐地就不吃不喝,生起病来。父母忧虑伤心,苦苦询问缘故。王孙便将心事讲了。父亲无可奈何,只得请媒人去妹妹家提亲。郑子侨为人古板严谨,觉得中表亲上再做亲于理不合,便推辞了这门亲事。王孙得知,病势更加沉重。母亲芸娘无计可施,只好暗地里恳求二娘,要闺秀来家安慰安慰王孙。郑子侨得知,怒不可遏,说的话也难听起来。于是,桂庵夫妇彻底绝望,只好听任王孙死活了。

本郡有一姓张的大户人家,五个女儿都很漂亮。最小的一个叫五可,尤其美丽,是姊妹中最出类拔萃的,一直还没订亲。一天,五可在去扫墓的路上,碰到王孙,从车子中偷偷看了一眼,一见钟情,回家后告诉了母亲。母亲探知她的心事,便叫来一个姓于的媒婆,向她流露了许亲给王孙的意思。于氏会意,立即到王家来。这时,王孙还在病中,于氏得知,笑着说:“公子的病我能治好。”芸娘询问缘故,于氏便说明来意,又把五可夸赞了一番。芸娘非常喜欢,让于氏快去跟王孙说说。于氏走进内室,抚摸着王孙告诉他这件事,王孙摇着头说:“你请的医生不对我的病症,有什么办法!”于氏笑着说:“治病,要问是不是好医生。如果是好的,即使求的是医和而来由是医缓,也可以啊!何必非求那个人,死了也要等她,这不是太傻了吗?”王孙流着泪叹息道:“但普天下的医生,却再也没有好过医和的!”于氏讥笑道:“公子怎么这样见识不广呢?”于是又把五可的容貌神情、体态衣著,连说带比划,极力描述了一番。王孙还是摇着头说:“算了吧!这人并不是我心中所想的人!”于是便回过头去,面对墙壁,再也不听。于氏见他心意不变,只好起身离去。

又一天,王孙昏昏沉沉中,忽见一个丫鬟进来说:“你想念的人来了!”王孙惊喜万分,从床上一跃而起,急忙出门,只见一个漂亮的美人已站在庭院中。仔细一看,却不是闺秀,穿着身松花色细褶绣裙,微微露出一双小脚,美丽绝伦,真是不亚于天仙!王孙忙施礼,询问姓名。美人回答说:“我就是五可。您是一个痴情的人,却把情意都倾注到闺秀身上,叫人不平!”王孙道歉说:“我平生没见过漂亮女子,所以心中只有一个闺秀,现在我知罪了!”两人便订下婚誓,正在手握着手依依不舍时,芸娘来探病,用手抚摸王孙,王孙一下子醒了过来,却是一个梦。回想梦中五可的音容笑貌,还历历在目。暗想:五可如真是像梦见的那样漂亮,何必非求那难以相遇的人呢!便把刚才的梦告诉了母亲。芸娘很喜欢他心思转变,立即就要请媒人去提亲。王孙恐怕梦见的不确实,便托邻居一个熟悉张家的老太太。借故去张家暗地里相看五可。老太太来到张家,五可正在病中,靠着枕头,手托着腮,婀娜多姿,无与伦比。老太太便上前问:“姑娘得了什么病?”五可玩弄着腰带,默默不语。她母亲代答道:“哪里有什么病!连续几天和爹娘呕气呢!”老太太又问缘故,五可母亲诉说道:“好几家提亲的,都不愿意,非像王家寄生一样的不嫁。是我这个做娘的劝了两句,就使性子好几天不吃不喝了!”老太太笑着说:“姑娘和王郎相配,倒真是一对玉人!他如果见了姑娘,恐怕也想念得憔悴要死。我回去后,就让他家来提亲,怎么样?”五可忙阻止说:“您千万别!如果不成,越发成了笑料了!”老太太赌咒发誓,担保必定能成,五可才露出了笑容。回去后,老太太向王孙讲了五可的相貌,和于媒婆对五可的描述一模一样。王孙又详细询问五可的衣著,也与梦中见的一样,心中大喜。心情虽然稍舒畅了些,但还是不敢太相信别人说的。

又过了几天,王孙病渐渐好了,把于媒婆找了来,请她想办法让自己亲眼见见五可。于氏为难,姑且答应下走了。过了很久,没有回音。王孙焦躁不堪,正要打发人去询问,于氏突然笑眯眯地来了,说:“幸亏有个好机会,五娘最近身体有病,每天都让奴婢们扶着到对院去散步。公子可去她家附近藏起来等着,五娘走路迟缓,到时就可以仔细相看相看了。”王孙大喜。第二天,早早骑马前去,于氏已先等在那里。让王孙把马拴在树上,领他进入临街的一处房子,为他取了座位,闭上门走了。不一会儿,五可果然扶着丫鬟走出家门来。王孙忙从门缝里凝目注视着。五可经过门外时,于氏故意指指天上的云,又指指路边的树,让五可看,以使她走慢点。王孙看了个仔细,心里惊喜得差点控制不住自己。不一会儿,于氏进门来笑着说:“可以代替闺秀吗?”王孙欢欢喜喜,再三致谢。返回家后,要父母立即托媒人去提亲。媒人赶到张家,张家却回答说五可已许了别人了!王孙闻听,悔恨忧闷,又立刻生起病来。父母既忧虑,又伤心,责备他自己耽误了好事。王孙也不说话,只是每天喝一小碗米汁度日。不几天,便瘦骨嶙峋,病得比前次更厉害了。

几天后,于媒婆忽然来到王家。见了王孙,惊讶地问:“怎么病成这个样子?”王孙流着泪,将五可已许人的事告诉了于氏。于氏笑道:“痴公子!起初人家主动许亲,你不答应;今天你求着人家,哪里就能那么爽快呢?即使她真许了人家,也还能再想办法。若早点和我商量,就是许给了京城皇帝老爷的儿子,我也能再夺回来!”王孙欢喜非常,求于氏给想个办法。于氏便叫他赶快写下书信庚帖,派人送去,约定第二天在张家会齐。王桂庵担心这样太唐突,会遭人家拒绝。于氏说: “前些天我已和张公说好,才过了几天又突然翻悔?况且他真把女儿许给了人家,也还没有书信庚帖。俗话说‘先做饭的先吃’,还怀疑什么?”王桂庵只好依从。第二天,便派了两个仆人送信去。张家也没说别的,收下书信,重重地赏了两个仆人回来。王孙的病一下子就好了,从此后,再不把闺秀放在心上了。

先前,郑子侨拒绝王家提亲时,闺秀便不高兴。后来听说王孙已与张家姑娘订亲,心里越发忧郁烦闷,也病了起来,身体逐渐衰弱。父母究问,也不说话。丫鬟窥知她的心事,悄悄地告诉了二娘。郑子侨听说后,非常生气,也不请医生诊治,听之任之。二娘怨怪地说:“我侄子也没什么不好的,你怎么这样迂腐固执,要害死我的女儿!”郑子侨大怒,骂道:“你生的好女儿!不如早点死了,也免得让人家笑话!”从此夫妻反目。二娘便和女儿商量,可以仍然嫁给王孙,只是只好做二房了。闺秀低着头,样子像是十分愿意。二娘又和丈夫商量,郑子侨更加恼怒,一切事都推给二娘,权当自己没有这个女儿,再也不闻不问了。二娘爱女心切,便想按照自己答应的去做,闺秀才喜欢,病也渐渐好了。

二娘暗地里打听,知道还有几天王孙就要娶亲了。到了那天,天刚明,二娘便以侄子要结婚,需要回娘家探亲为理由,打发人去向哥哥王桂庵借仆人和车马。王桂庵很爱护妹妹,觉得妹妹是邻村,路又不远,便让迎亲的车马先去接回二娘。车子一到,二娘便将女儿梳妆打扮好,命车子拉着,让两个仆人、两个婆子护送着往王家赶来。到了王家门口,用红毡铺地,走了进去。这时,鼓乐手早已会齐准备好,跟来的仆人便喝命奏乐,一时吹擂大作,人声鼎沸。王孙急忙跑过来一看,见一女子头蒙红帕,大吃一惊,刚想跑开,郑家仆人过来捉住,让两个人交拜。王孙稀里糊涂地拜完,两个婆子扶着女子径直到新房坐下,王孙才醒悟过来是闺秀。全家一片惊惶,不知如何办才好。这时,天渐渐黑了,王孙不敢再去张家迎亲。无可奈何中,王桂庵只得派仆人去张家说明情况。张公大怒,便想退亲。五可不肯,说:“她虽然先到,但并没正式订婚,不如让王家快来迎娶。”事已至此,张公只得照此办理,让王家的仆人赶快回去禀报。王桂庵还是不敢去。父子二人相对谋划,真是喜也不是,怒也不是,无计可施。张家等了很久,没见王家来人,便自己备车,将五可送到了王家门上。王桂庵只得另设一新房,让五可住下。王孙来回奔跑于两座新房中,疲于应付。芸娘便给二女调停,让她们按年龄大小确定名分,两人都答应。可等五可听说闺秀比自己还稍大点,得称“姐姐”,便面有难色。芸娘很是担心。婚后“三朝”那天,二人同去拜见婆母,五可见闺秀风姿绰约,似乎比自己还略胜一筹,便甘心居次,二人的名分才终于定下来。但王桂庵夫妇始终担心二人时间长了会互不相容。没想到两人却是言语投机、相敬相爱,连衣服也换着穿,真像亲姊妹一般。

后来,王孙问五可当初为什么拒绝提亲,五可笑着说:“没别的!当初你拒绝于媒婆来许亲,我只是想报复报复你。你还没见过我,心中只有闺秀;既然见了我,我也稍微傲慢点,看你对待我比对待闺秀如何!假若你为了她生病,而不为了我生病,我也就不强求了!”王孙笑到:“这报应也太毒了!但不是于媒婆,我怎能够见你一面呢?”五可说:“是我自己想让你看看的,媒婆有什么能为?经过那座房门时,我岂不知里面有个人正虎视耽耽?我们梦中已订下誓约,你怎么还不相信,非去看看不可呢?”王孙惊问:“你是怎么知道梦中订下婚誓的事的?”五可说:“我病中梦见到了你家,醒来后觉得太荒诞。后来听说你也梦见了我,我才知道我的魂魄真来过这里。”王孙极为惊异,详细讲述了自己当时梦中的情景,二人做梦的日期时辰都完全相符。

王桂庵父子两人的姻缘都从梦中来,也算是奇事了,所以一并记下这两件事。


\subsection{1.12.19   周 生}
\label{\detokenize{p00_u5176_u5b83/_u767d_u8bdd_u804a_u658b_u5fd7_u5f02:id482}}
周生,是淄川县县令的幕宾,县令因公外出,夫人徐氏,很早就有去泰山朝拜碧霞元君的心愿。因为路途遥远,想派仆人带着祭礼替自己前往,便让周生写一篇祷词。周生作了篇骈体文,历述徐氏生平,文中很多地方颇为淫荡不恭。其中有句话说:“栽般阳满县之花,偏怜断袖;置夹谷弥山之草,惟爱余桃。”(般阳:即淄川县。断袖:指男宠。余桃,典出《韩非子·说难》。这两句意思是讽谕县令偏爱男色。)这是诉说徐夫人的愤慨,像这样的句子还有很多。脱稿后,拿给同事凌生看,凌生见文章太轻侮亵渎神灵,劝告他不要用,周生不听,还是将祷词交给仆人带去了泰山。

此后,不长时间,周生在县衙中暴毙身亡。仆人也跟着死了。徐夫人生产后,也得病去世。人们还没感到有什么奇怪的。周生的儿子自京城中赶来接回父亲的棺木,夜晚与凌生同宿,梦见父亲告诫他说:“写文章不可不慎重啊!我没有听从凌君的劝告,以淫荡之词,招致神灵发怒,不光丢了自己的命,还连累了徐夫人和焚烧祷词的仆人。恐怕我在阴间里还免不了受罚。”醒后,他惊异地告诉了凌生,凌生也作了同样的梦,便将他父亲写祷词一事告诉了他。周生的儿子不禁为之悚然戒惧。


\subsection{1.12.20   褚 遂 良}
\label{\detokenize{p00_u5176_u5b83/_u767d_u8bdd_u804a_u658b_u5fd7_u5f02:id483}}
山东长山县有个赵某,从一个大姓人家租了一间屋居住。他得了一种腹中长肿块的病,又孤苦贫困,病得奄奄一息,眼看就要死了。

有一天,他极力支撑着病重的身体寻找凉爽的地方,移到屋檐下就躺下了。醒来以后,看见一位绝代佳人坐在自己身旁,就询问她。女郎说:“我是特地来给你做媳妇的。”赵某吃惊地说:“且不说穷人不敢有这种妄想;如今我已奄奄一息,有妻子又有什么用?”姑娘说:“我能治你的病。”赵某说:“我的病不是短时间能够治好的。纵然有良方,没有钱买药又有什么办法!”姑娘说:“我治病不用药。”于是就用手按着赵某的肚子,用力按摩,赵某觉得她的手掌像火一样热。过了一会儿,赵某腹中的结块,隐隐约约地发出拆解分裂的声音。又过了一会儿,赵某就想上厕所。他急忙爬起来,走出几步,解开衣裤就大泻起来。粘液倾泻,结块都排出来了,只觉得浑身十分爽快。他回来躺在原来的地方,对姑娘说:“娘子是什么人?请你告诉我姓氏,以便立个牌位祭祀。”姑娘说:“我是狐仙。你前世原是唐朝的褚遂良,曾经对我家有恩,我经常铭记在心,想要报答。天天寻找你,今天才见到了你。长久以来回报的愿望算可以实现了。”赵某因自己貌丑感到惭愧,又顾虑茅屋被灶烟薰得很黑,会弄脏了姑娘华丽的衣服。姑娘只是请求跟他一起去,赵某就领着她进入自己家中。土炕上铺着碎草,连席子也没有。灶膛是冷的,多日不曾烧火做饭了。赵某说:“且不论家境如此贫寒,不忍心屈辱你;即使你能心甘情愿地留下。你看瓮底空空,又用什么来养活老婆孩子?”姑娘只说:“不要担忧。” 她说话的功夫,赵某回头一看。只见床上毛毡被褥都已铺设好了。赵某正要询问,又一转眼间,满屋已用银光闪闪的纸裱糊得像镜子似的,各种东西也都变换了。几案精致光洁,上面已经摆好了酒菜,于是两人就欢快地对饮起来。天晚了就和姑娘一同睡下,和夫妻一样。

赵某的房主人听说了这件怪异的事,就请求见一见姑娘。姑娘就出来相见,并没有为难的神色。从此,这件事四方传播,登门求见姑娘的人很多,姑娘并不拒绝。有的人设筵招待他们,姑娘也一定和丈夫一起去。

有一天,酒筵中有一位孝廉,暗中产生了淫恶的念头。姑娘已经知道了,突然对他斥骂起来,立即用手推他的头,孝廉的头就伸出窗棂之外,而身子还在屋里。出不去,进不来,也不能转动。大家都请求宽恕他,姑娘才把他拽出来。过了一年多。登门拜访的人越发多了,姑娘十分厌烦。被拒绝的人就骂赵某。过端阳节的那一天,赵家请来了许多朋友饮酒,忽然一只白兔跑了进来。姑娘站起来说:“捣药翁来召见我们了。”对兔子说:“请你先走一步。”兔子跑出去,迳直走了。姑娘叫赵某拿了一架梯子来,有数丈高。院子里有一棵大树,便把梯子倚在树上,梯子还高过树梢。姑娘先爬上去,赵某也跟着她。姑娘回过头来说:“亲戚朋友有愿意跟着去的,请立即登梯。”众人互相看着,没有人敢上去。只有屋主人家一个家童,踊跃地跟在他们后面。越上越高,梯子到头,就进入云彩,看不见了。大家一看那架梯子,原来是多年的一扇破门,去掉了镶板罢了。大家一齐进入他家一看,依然是原来的灰壁破灶,其它空无一物。还寻思着家童回来时可以问问情况,但竟然始终杳无踪迹。


\subsection{1.12.21   刘 全}
\label{\detokenize{p00_u5176_u5b83/_u767d_u8bdd_u804a_u658b_u5fd7_u5f02:id484}}
邹平县有个姓侯的牛医,一次,提着饭去招待为自己耕地的人。来到田野里,有股旋风在他前面转来转去,侯某就用勺子舀起饭汤祭奠到地上,连舀了几勺,旋风才离去。又一天,侯某到城隍庙闲逛,见里面有座“刘全献瓜”塑像,刘全被鸟雀粪糊住了眼睛。侯某慨叹地说:“刘大哥怎么竞受如此玷污!”便进去用手指甲将鸟粪仔细剔下来。

几年过后,侯某卧病在床,梦见被两个鬼隶摄了去。来到官衙前,鬼隶逼索财物,侯某无计可施。忽然从官衙内走出一个穿绿衣服的人,看见侯某,惊讶地说:“侯老翁怎么来了?”侯某便诉说了经过。绿衣人斥责两个鬼隶说:“这位是你们侯大爷,怎敢无礼!”两鬼隶忙道歉说不知。一会儿,听见官衙内传出雷鸣般的鼓声,绿衣人说:“升堂了!”便和侯某一块进去,叫他站在台阶下,说:“先在这等会儿,我替你问问是怎么回事。”走上前去,向一个官吏模样的人招手示意。那人从大堂上下来,两个人简单谈了几句。官吏模样的人看见侯某,拱手施礼说:“侯大哥来了!你也没什么大事,有一匹马告了你,两下里一对质就可以回去。”说完便走了。不长时间,听到大堂上叫侯某的名字,侯某忙上前跪倒在地,一匹马也跪在那里。判官问侯某:“这匹马告你将他药死了,有这回事吗?”侯某申诉道:“它得了瘟病,我用治瘟病的药方治疗,吃了药后不见好转,隔了一天就死了。这与我有什么关系呢?”马竟开口说起人话来,两下里各说各的理,互不相让。判官便命查生死簿,簿上注明了这匹马年龄多大,应死于某年某月某日,与马死的时间相符。判官便呵斥马说:“这是你寿数已尽,怎能无妄指控别人!”将马赶了出去,又对侯某说:“你原本是想救活它的,与你无关。这次你可以不死。”仍然命那两个鬼隶将侯某送回。绿衣人和官吏模样的人也和他们一块出来,嘱咐鬼隶路上好好对待侯某。侯某感激地说:“今天承蒙二位保护,但我并不认识你们,请告诉姓名,以便将来报答。”绿衣人说:“三年前,我从泰山回来,路上又热又渴,难受得要死。经过你们村外时,承蒙你舀饭汤给我喝,这恩情现在也不敢忘。”像官吏模样的人说:“我就是刘全。以前遭受鸟粪玷污,被糊住了双眼,闷得不能忍受。你亲手将鸟粪剔除,这恩情我也不敢忘怀!只是阴间里的酒肴,没法招待客人,我们就此分别。”侯某恍然大悟,于是便往回赶来。到了家,要款留两个鬼隶,鬼隶却连一杯水也不敢喝。过了会儿,侯某苏醒过来,此时他已死了两天了。

从此后,侯某更加好善。逢年过节,他定用酒祭奠刘全。八十多岁,身体还很强健,能跃马奔驰。一天,路上见刘全骑马走来,像要出远门。两人拱手行礼,寒暄毕,刘全说:“大哥寿数已尽,勾魂牒已经发出了。差役要来勾你,被我阻止住了。大哥可回家料理料理后事,三天后我来叫你一块走。地下我已经替你买了个小官职,不会受什么苦的。”说完就走了。侯某回家告诉了妻子儿女,和亲戚朋友一一告别,又准备好了棺材和寿衣。第四天天刚黑,侯某对众人说:“刘大哥来叫我了!”进入棺材,便去世了。


\subsection{1.12.22   土 化 兔}
\label{\detokenize{p00_u5176_u5b83/_u767d_u8bdd_u804a_u658b_u5fd7_u5f02:id485}}
靖逆侯张勇镇守兰州时,一次外出打猎,打到很多野兔。其中有的兔子半身或两条大腿还是土质的。一时,秦中地方的人争相传说土能变兔。这也是自然界中不好理解的事情。


\subsection{1.12.23   鸟 使}
\label{\detokenize{p00_u5176_u5b83/_u767d_u8bdd_u804a_u658b_u5fd7_u5f02:id486}}
苑城人史乌程正在家中,忽见一群鸟飞集到屋顶上,颜色和叫声像是乌鸦。史乌程告诉人家说:“夫人派鸟使来叫我了。赶快准备后事,在某一天我就要死了。”到了那天,史乌程果然去世。出丧的那天,乌鸦群又来了,跟随着棺材慢慢飞着,从苑城一直跟到新城。葬后,鸦群才不见。这件事是长山吴木欣亲眼看见的。


\subsection{1.12.24   姬 生}
\label{\detokenize{p00_u5176_u5b83/_u767d_u8bdd_u804a_u658b_u5fd7_u5f02:id487}}
南阳有一家人,姓鄂,家里遭受狐患,金钱、杂物经常被狐盗去。如触犯他们。作祟便更加厉害。

姓鄂的有一个外甥姓姬,是一个放荡不羁的书生。姬生来到鄂家向狐祷告免灾,狐不听;又祷告狐舍了外祖家到自己家去,狐还是不听,众人都笑起来。姬生说:“它们既能变幻,必通人情。我持之以恒地引导它们,早晚就能引入正果。”于是,不几天就到鄂家祷告一次。虽不大见效,但姬生一到鄂家,狐就不扰乱了。因此,鄂家就常叫姬生住在他家。姬生住在外祖家,夜里就对空祷告,执意要求见狐一面。

一天,姬生回到家,独自坐在书房里,忽然见房门慢慢地自己开了。他忙站起来恭敬地说:“狐兄来了?”可是寂静无声。一夜,门又自开,姬生说:“倘若是狐兄光临,是我祷告求来的,何妨叫我见一面?”仍是寂静无声。这夜,姬生桌子上有二百钱,到了天明就不见了。第二天晚上,姬生又增加了几百钱放在桌子上。半夜时分,听见布帘子响,姬生忙说:“狐兄来啦?我已准备下几百铜钱给你使用。我虽家里不富裕,但也绝不吝啬。若有急用,不妨明说,何必盗窃呢?”稍等了一会儿,去看了看钱,只拿了二百去。姬生把余下的钱仍放在原处。但是一连几夜没有再丢失一文。姬生有只熟鸡,打算请客用,可是忽然丢失了;到了晚上,姬生又准备了酒给狐喝。从此,狐就不再来了。

鄂家的狐患仍和以前一样,姬生又去祷告,对狐说:“我准备下钱你不要,准备了酒你不喝,我外祖年纪老了,身体又不好,哪能受得起你们长久骚扰?我特备下不太丰盛的一点礼物,到夜里任凭你自己拿,愿拿什么就拿什么。”于是把十几千钱。一坛子酒,两只切好的鸡,摆在桌子上。到了晚上,姬生躺在一旁守着,可是整夜没有一点动静,钱与物一点也不少。然而,狐却从此绝迹了。

一天,姬生晚上回家,一开门,见他的桌子上放着一壶酒和满满一盘熟鸡蛋,四百钱,还用红线串着,就是前几天丢失的那些东西。心里明白这是狐来报答他。走向前去闻了闻酒,酒味很香,倒出来看了看,酒色碧绿,喝了一口,味道很醇。及至一壶酒喝完了,觉得也半醉了。这时,心里顿时产生了贪财的念头,忽然想作贼偷东西。于是便开门出去。村中有一富户人家,他就去跳墙当贼,这家人家墙虽高,但姬生一上一下,犹如长了翅膀一样。进到财主的房子里,偷了貂皮衣服、金鼎等物,拿回自己家里,放在床头上,这才上床睡觉。到了天明,他便带着这些东西到里屋给他妻子看。他妻子问他,姬生吞吞吐吐地告诉了她,并现出很高兴的样子。他妻子惊骇地说:“郎君素来刚正,为什么做起贼来?”姬生仍恬不知耻,向妻子述说狐通人情。他妻子才恍然大悟地说:“你这是喝了狐酒,中了毒。”想起丹砂可以祛邪,妻子就研细了丹砂加到酒里,叫姬生喝了。稍待一会,姬生忽然失声大叫:“我为什么做贼!”他妻子急忙代他解释了其中的缘故,姬生懊悔不已。

财主被盗后村里村外到处都在传说。姬生听说,终日吃不下饭,睡不着觉,不知怎么办好。他妻子给他想办法,叫他趁夜里把偷来的东西隔墙抛进财主家的院子。姬生照办了,财主家又得到了丢失的东西,这事也就作罢。

姬生参加岁试,考了第一名,又被推举为品行优等生,加倍受赏。到了受赏的那一天,官府的梁上贴了一张帖子,上写:“姬某作贼,偷某家裘、鼎,何为行优?”房梁很高,不是跷跷脚就能贴上的事。考试官心里纳闷,就拿着帖子来问姬生。姬生愕然不知所措,心里想,这个事除了我妻子知道外,别人没有知道的,况且官署中戒备森严,哪里能进来贴这个帖子?恍然大悟地说:“这事必然是狐办的。”便详细地毫不隐瞒地说了以前的经过。考试官还是加倍奖赏了他。

后来姬生常想:我并未得罪狐狸,它所以屡次要陷害我,恐怕是因为小人不甘心独自为小人,一心要拉别人下水吧!


\subsection{1.12.25   果 报}
\label{\detokenize{p00_u5176_u5b83/_u767d_u8bdd_u804a_u658b_u5fd7_u5f02:id488}}
安丘县某书生。为人邪恶放荡,行为不检。通晓占卦术,每当要翻墙越院偷盗人家的财物时,就先算算卦。一天,他忽然大病,吃药也不好,自己说:“我早知会这样。冥司愤怒我亵渎天数,将要重重惩罚我,药怎能治好!”不长时间,书生的眼睛突然失明,两手也无缘无故地折断了。

某甲的伯父没有子女,某甲贪图他家的财产,愿意作子嗣。伯父死后,某甲侵吞了全部田产,却背弃了前约。又有一个叔父,家里很富裕,也没有儿子。某甲又认作父亲。叔父死后,他同样反悔,背弃了前约。某甲吞并了三家的田产,成为乡里的首富。一天,他突然暴病,像疯子一样,自言自语地说:“你想要占有富厚的财产而活吗?”随后就用利刀往自己身上猛割,一片片地切下肉来扔到地上。又说:“你绝了人家的后代,自己还想有后吗?”接着就剖开肚子,肠子也流了出来,人就一命呜呼了!不长时间,他的儿子也死了!如此因果报应,真是怕人啊!


\subsection{1.12.26   公 孙 夏}
\label{\detokenize{p00_u5176_u5b83/_u767d_u8bdd_u804a_u658b_u5fd7_u5f02:id489}}
保定有个监生,打算到京城去花钱买个县官做。刚整理行装要出发,就生病了。他一病就是一个多月不能起床。一天,忽然书童来向他禀报:“外面来了一个客人。” 他一听有客,忘记了自己的病就出去迎接。一出门,见来人衣服华贵,像个贵人,就连连拱手请客人进屋。客人坐下后,他便问贵客来意。客人说:“我叫公孙夏,是十一皇子家的座上客。听说你整理行装要去活动个县官做,我认为你既然有志气,何不活动个太守当,那不更好吗?”监生谦逊地表示感谢,说:“我的钱太少,不敢有更高的想法。”客人听了,表示愿意帮忙,帮他出一半钱,并约好时间叫人到他住所去拿。监生很高兴,要求给以引荐。客人对他说:“总督、抚台都是我的好朋友。只要有五千贯钱,事就能办成。眼下真定地方缺额,可快一点办。”监生认为真定是本省内的地方,在当地做官不好。客人说:“你真傻!只要有空子可钻,管它本省不本省的!”监生心里不踏实,仍犹犹豫豫,总怀疑这事有点荒唐。客人进一步说:“不用怀疑,我实话告诉你吧!这个官是阴间的一个城隍缺职,你寿限已经尽了,注了死名册,趁此机会办理办理,到阴间还可荣华富贵。”说完就要告别而去,临走还再嘱咐:“你自己先准备着,三日内再见。”骑上马就走了。

监生忽然睁眼一看,想了想,原来是个梦,但他相信梦里的一切是真的,就与妻子说了永别的话。并拿出所藏的银子,买了纸元宝一万多提,一时郡中的这类东西全被他买光了。把纸元宝堆在院子里,加上纸扎的童男童女、纸马、纸牛等物。一起点上火,日夜焚烧,烧的灰有小山那么高。到了第三天,那个客人果然来了,监生便拿出钱交给他兑现。客人收了钱,就领他到了部院。见一个贵官坐在殿上,监生便跪拜在殿下。贵官略问了问他的姓名后,便勉励他为官要清正等,拿任命书给他,监生便叩头谢恩而去。

监生当了太守,自认为出身监生,地位卑贱,如果没有大队车马,没有好的服饰加以炫耀,不足以震服部下。于是他买上很好的车马,还打发鬼役用彩车接来了美妾,各项准备工作刚刚就绪,真定郡的仪仗队就来接他。他跟着仪仗队,一路走着,人们夹道欢呼,他十分自得。大队人马正走着,忽然前面领路的鼓乐停住了。旌旗也放倒了。他正惊疑问,又见前面骑马的人都下了马,一起跪倒在路旁,并且渐渐缩小,人缩到一尺高,马缩到如猫大。他车前的人报告说:“关帝神来了!”监生一听,也害怕了,急忙下车跪在地上。一抬头,远远看见关帝骑着大马,后面跟着四五匹坐骑,慢慢向他走来。神长的是络腮胡子,不大像人世间所画的肖像那样。然而种态威严,两只眼很长,一直长到耳朵边。关帝走进前来问:“这是什么官?”随从回答:“真定太守。”关帝说:“小小的一个太守。怎么这么威风!”监生听了,吓得毛骨悚然,身子觉得一下缩小了许多。他看了一下自己,像个六七岁的小孩子。关帝叫他起来,一块跟在马后走。

走了不多时,道旁有一座宫殿,关帝下马进了殿,朝南坐下。命人取纸、笔给监生,先叫他自己写出籍贯、姓名。监生写完呈上,关帝一看大怒,对他说:“看你写的错别字这么多,字也不成样子,真是个市侩小人.哪里能当民官!”又命人查他的德行录,有一人跪奏,没有听到说什么。关帝严厉地说:“你投机钻营罪还小些,买爵讨官罪恶太大!”于是就有两个金甲神人拿了锁链出去,又有两个小神捉住监生,脱去官服,摘去官帽,推倒在地打了他五十大板,直打得腚上的肉都几乎掉了下来。最后把他撵了出去。

监生出门后,四下一看,车马都没有了,觉得浑身疼得不能走路,便趴在草丛里休息。仔细辨认了一下周围,这地方离家并不远。幸好觉得身子很轻,轻得走起路来像树叶一样。他走了一天一夜,才到了家。忽然觉着像做了个梦一样,睁眼一看,自己还是躺在床上呻吟。全家人都来问他,他啥也不说,直喊腚疼。在此以前,他一直闭着眼像死了一样,已有七天了。到现在,他才明白了一切,便问家人:“阿怜为什么没来?”——原来阿怜是他爱妾的小名。先是有一天,阿怜正与人说话,忽然说:“我丈夫当了真定太守,派人接我来了。”说罢就进屋梳妆打扮,打扮完后就死了。这事到今天才隔了一夜。家人说完,认为这事很奇怪,监生却完全明白。只有悔恨而已。他叫人把阿怜的尸体留下,不要埋葬,等她苏醒过来,可是一直等了几天仍没还阳,才埋葬了。

监生的病渐渐好了,可腚疮却更厉害了,半年后才能起来走路。自己常对人说:“我官和钱都没有了,而且还受到阴间的刑罚,这些我都能忍受;但不能忍受的是我的爱妾不知道哪里去了,一到夜晚便不知如何消磨了。”


\subsection{1.12.27   韩 方}
\label{\detokenize{p00_u5176_u5b83/_u767d_u8bdd_u804a_u658b_u5fd7_u5f02:id490}}
明代末年,济南郡以北好几个州县,盛行瘟疫。家家都有病人。齐东有个叫韩方的农民,父母都染上了疫病,韩方对老人十分孝敬,急得没法,便备好祭品,到孤石大夫庙中痛哭着为父母祈祷。回去的路上,还在伤心地落泪。

忽然碰见一个人,衣着整洁,问韩方:“什么事这样悲伤?”韩方详细地告诉了他。那人说:“孤石大夫是很神验,但不在治疫病上,向他祈祷有什么用?我有个小办法,倒可以试试。”韩方大喜,询问那人的姓名,那人说:“我不求报答,何必告诉你姓名籍贯?”韩方又恳求去自己家看病,那人摇摇头,说:“不必。你回家后,拿张黄纸放到床上,厉声说‘我明天去鬼都告诉东岳大帝’,你父母的病就好了。”韩方恐怕不灵验,执意请那人去家里看看。那人说:“实话告诉你,我不是凡人。巡环使者见我忠厚诚实,让我做了南乡土地。我为你的一片孝心所感动,所以教给你这个方法。现在,东岳大帝正在从枉死鬼中选拔那些对老百姓有功、或一生正直、不作邪祟的,用作城隍、土地。这些行瘟疫殃害人的,都是郡城中被清兵杀死的冤鬼,急着要去鬼都向岳帝自荐,所以沿途索贿,借此糊口。你说要告诉岳帝,他们一定害怕,病就好了。”韩方听罢,又惊又敬.忙伏地叩头谢恩,起来一看,土地已渺无人影了。叹息着回到家中,按照土地说的去做,父母果然好了。又把这方法传到邻村,无不灵验。


\subsection{1.12.28   纫 针}
\label{\detokenize{p00_u5176_u5b83/_u767d_u8bdd_u804a_u658b_u5fd7_u5f02:id491}}
东昌人虞小思,经商为业。妻子夏氏,一天从娘家回来,走到自家门口,见一老太太和一个少女正哭得十分悲伤。夏氏好奇地询问缘故,老太太挥泪诉说了一番。原来,老太太的丈夫叫王心斋,本是官宦后代。后来家道衰落,无法谋生,便央求人担保,借了富户黄某家的银子去做买卖。途中碰上强盗,钱财全被抢光,侥幸保住条命逃回家来。黄某天天索债,连本带息共有三十多两银子,家里实在没东西抵债。黄某窥视到王心斋的女儿纫针生得很美,便想弄到手做妾。让保人去告诉王心斋:如果愿意拿女儿顶债,除原来的欠帐一笔勾销外,另外再给二十两银子。王心斋拿不定主意,去跟妻子商量。妻子哭着说:“我们虽然贫困,也是官宦人家的后裔。那黄某靠做奴仆发家,怎敢让我的女儿去给他做小老婆!况且,纫针早就有了女婿,你可不能擅自作主啊!”先前,本县傅举人的儿子,和王心斋很投机,生了个儿子叫阿卯,两家便订下了“娃娃亲”。后来。傅举人去了福建做官,一年多就死在任上。妻子儿女回不来老家,与王家也就断绝音讯了。因此,纫针长到十五岁,还没嫁人。妻子提到这件事,王心斋无话可说,长吁短叹,合计着如何才能还上黄某的债。妻子说:“实在没法的话,我回娘家跟我的两个弟弟商量商量,请他们帮助。”王心斋的妻子姓范,她祖父曾在京城做过官,有两个孙子,家里富有田产。第二天,范氏带着女儿纫针回了娘家,央求两个弟弟接济,两个弟弟却任凭她伤心地流泪,连一句想帮忙的话都没有。范氏无法,大哭着返了回来,正好碰上夏氏询问,便又连说带哭地诉说了一番。

夏氏听后,非常怜悯母女二人。见那少女生得柔媚可爱,心里更感到酸楚。便将她们母女请到自己家,用酒饭招待。安慰她们说:“你们娘俩不要难过,我一定尽力帮助你们!”范氏还没来得及致谢,女儿纫针已哭着跪倒在地。夏氏更加怜惜她,筹思着说:“我虽然略有点积蓄,但要拿出三十两银子也很困难,只得靠典当东西凑钱了。”母女再三拜谢。夏氏和她们约定三天后来取钱。范氏母女走后,夏氏想方设法筹钱,也没敢告诉丈夫。三天后,仍没凑齐三十两银子,便打发人回娘家去跟母亲借钱。这时,范氏母女却已来了。夏氏告诉她们实情,让她们第二天再来。傍晚,银子借来了,夏氏便将银子连同原来的那些一块包起来,放在床头上。到了夜晚,有个强盗钻透墙壁,举着灯进入屋内。夏氏惊醒过来,微微睁眼一看,见一个人胳膊上挎着短刀,相貌十分凶恶。夏氏非常害怕,假装睡着,大气不敢出。强盗走近箱子,像是要撬锁,一回头,发现夏氏枕头边上有个小包裹,一探身抓了去,在灯下解开看了看,便放进了腰包里。也不再开箱子,径自走了。夏氏连忙起身呼救。家里只有一个小丫头,听见喊声,忙隔墙去招呼邻居。等邻居们都跑过来,强盗早已无影无踪了。夏氏丢失了银子,对着灯哭泣着,觉得没法向范氏母女交待。见小丫头已经睡熟,便在窗棂上上吊自杀了。

天刚亮,丫头发现了吊着的夏氏,惊惧地喊人解救。救下来一看,四肢早已冰凉了。虞小思听到消息,忙赶回家来,询问小丫头,才得知事情经过,痛哭着办理丧事。当时正是夏天,夏氏的尸体既不僵,也不腐烂。过了七天,才入了敛。埋葬后,纫针偷偷地从家里跑出来,到夏氏的坟墓上痛哭。正哭着,忽然雷电大作,暴雨倾盆,霹雳一声,将夏氏的坟炸开,纫针也被震死了。虞小思听说,奔到妻子坟上察看,只见棺材已打开,妻子正在里面呻吟,忙抱了出来。见旁边还有具女尸,不认识是谁。醒过来的夏氏仔细看了看,才认出是纫针。二人大感惊骇奇怪。不一会儿,范氏跑了来,发现女儿已死,哭着说:“我本来就怀疑她在这里,果然没错!听到夏夫人的死讯后,她就日夜啼哭。今晚告诉我,想来坟上哭祭,我没答应.她就自己跑来了。”夏氏为纫针的情谊所感动,跟丈夫说了说,就用葬自己的棺材和墓穴葬了纫针。范氏拜谢。

虞小思背着妻子回了家,范氏也回去告诉丈夫经过。这时,听人说村北有个人被雷劈死在路上,身上还写着行字:“偷夏氏银子的贼!”一会儿听到邻居的妻子号哭。才知雷打死的强盗就是她的丈夫马大。村里有人忙告了官府,官府将马大的老婆捉了去询问,才得知其中原委。原来,范氏因为夏氏答应替她出钱赎女儿,感激地哭着对别人说了。马大本是个赌徒无赖,听说后便萌生了偷盗夏氏银子的念头。官府便押着这个妇人去她家搜寻赃物。只搜出二十两银子。又从马大尸体上搜出了四两。官府判决将马妻卖了,凑齐失盗的银子数,还给了虞小思。夏氏更加喜欢,仍将银子如数给了范氏,让她还给债主。

葬了纫针三天后,夜晚忽然狂风呼啸,电闪雷鸣,将坟墓再次震开,纫针也活了过来。她也不回家,径直去敲夏氏的门。原来纫针认出了葬自己的地方本是夏氏的坟,所以怀疑她已经复活了。夏氏听到敲门声,惊醒过来,隔着门问是谁。纫针说:“夫人果然活了吗?我是纫针啊!”夏氏听了,惊骇不已,以为是鬼。忙招呼邻居家的老太太一块询问,才知道纫针真的又活了,忙高兴地让她进屋。纫针对夏氏说:“我愿意留下来服侍夫人,不想再回家了。”夏氏说:“你莫不是怀疑我出钱是为了买奴婢吗?你葬了后,我已经替你家还了债。请你不要猜疑。”纫针越发感激,哭泣着,要认夏氏为母亲。夏氏不答应,纫针哀求说:“女儿能操劳家务,不会吃闲饭的!”天明后,夏氏去告诉范氏纫针复活的事。范氏大喜,急忙赶来,也顺从了女儿的意思,让女儿拜夏氏为母。范氏回家后,夏氏又把纫针强送回了家。纫针啼哭着思念夏氏。王心斋便背着女儿来到夏氏家,把她放到门内自己走了。夏氏看见纫针,惊讶地问她怎么来的,得知缘故后便放了心,收下了这个女儿。纫针看见虞小思过来,急忙下拜,称他父亲。虞小思本来就没有子女,又见纫针楚楚动人,心里很是高兴。从此后:纫针纺线织布,缝补衣服,十分勤苦。一次,夏氏偶然生病,纫针昼夜侍奉。见夏氏不吃饭,自己也不吃,脸上常常带着泪痕,跟人说:“母亲万一有个好歹,我也决不再活了!”夏氏的病好转后,纫针才露出了笑容。夏氏听说后,哭着说:“我四十多了没有孩子,能生个女儿像纫针一样,我也就满足了!”夏氏从没生育,一年后,忽然生了个儿子。人们都说这是行善的报答。

又过了两年,纫针越发大了。虞小思跟王心斋商量,不能死守过去跟傅家的婚约。王心斋说:“女儿在你家,婚姻大事一切由你作主!”纫针这年十七岁,贤惠美丽,举世无双,要嫁人的消息一传出,来提亲的人踢破了门槛。虞小思夫妻挑来拣去,极力要选个称心如意的女婿。富户黄某也派了媒人来提亲,虞小思厌恶他为富不仁,坚决拒绝,而是选中了冯家的儿子。冯某,本是县里的名士,儿子既聪明,文章又写得好。虞小思想把自己选择的结果告诉王心斋,王却外出做买卖没回来。虞小思便自己作主,跟冯家订下了亲事。黄某没有得逞,便也假托外出做买卖,找到了王心斋,摆下酒宴请他喝酒,还资助了他一些本钱。二人渐渐融洽起来,黄某便吹嘘自己的儿子如何如何聪明,要自己做媒给儿子提亲。王心斋感激黄某的资助,又仰慕他的富有,便答应了。回家后,去告诉虞小思。虞却已在昨天接受了冯家的婚书,听了王心斋的话,很不高兴,让女儿出来,告诉她情形。女儿生气地说:“黄债主是我们的仇家!让我侍奉他们,我只有一死!”王心斋很惭愧,托人去告诉黄某虞小思已答应了冯家的婚事。黄某大怒,说:“那女子姓王,不姓虞!我有约在先,他提亲在后,怎么能背弃盟约!”于是,向县衙告了状。县令因为黄某有约在先,要将纫针判给黄家。冯家不服,说:“王心斋把女儿托忖给虞家,亲口说婚姻大事由虞家作主;况且,我有订婚书,而黄某跟王某不过是几句酒话罢了!”县令听了,一时不能判决,便想听凭纫针所愿。黄某忙用重金贿赂县令,求他偏袒自己。因此,这事一直拖了一个多月也没最后判定。

一天,有个举人北上进京赶考,坐着公车路过东昌,派入打听王心斋,正好问到虞家。虞小思反问来人,得知那举人姓傅,就是当年的阿卯。他已经入了福建籍,十八岁时就乡试考中了举人。因为以前跟王家有婚约,所以一直没有娶亲。这次北上赶考,他的母亲特意嘱咐他顺便访查王家的下落,打听一下纫针是不是已经嫁了人。虞小思听说后大喜。把傅举人请到家中。详细讲述了纫针这些年来的遭遇。但女婿自千里以外的地方赶来,苦于没有凭证。傅阿卯便打开一个箱子,拿出了当年王家给的许婚书。虞小思忙叫了王心斋来,检验了检验,果然不错。大家都很高兴。这天,县令复审纫针一案,阿卯投进名帖,拜见县令,说明了情况,县令便撤销了这个案子。阿卯跟王家约下婚期,又继续北上了。

不久,阿卯参加会试回来。买了很多礼品,在他原来的家住下,跟纫针成了亲。这时,阿卯考中进士的喜报已经报到了福建,接着又报来东昌,会试又考中了,接着是入京观政。从京城回来后。纫针不愿到南方去。阿卯也因为旧宅祖坟都在这里,于是自己南下迎回父亲的棺木,用车载着母亲一同迁来老家。

又过了几年,虞小思去世了,儿子才七八岁。纫针抚养着他,比待自己的亲弟弟还好,让他读书,进了县学。家里也很富有,这一切都是靠阿卯的力量啊!


\subsection{1.12.29   桓 侯}
\label{\detokenize{p00_u5176_u5b83/_u767d_u8bdd_u804a_u658b_u5fd7_u5f02:id492}}
荆州人彭好士,从朋友家喝酒回来,下马小便,马在路旁啃草。有一丛细草毛茸茸的,小黄花刚开放,鲜艳夺目,可已被吃了大半了;彭好士看见了,赶忙把剩下的草茎拔下来,闻闻有特殊香昧,就揣在怀里,上马再走。

马一会儿快一会儿慢,他觉得很惬意,竟忘了看看是否到家了,由着马随便走。走着走着,忽然发觉太阳要落山了,这才想起该拉住马往回返了。只见满眼都是山,不知到了什么地方。这时来了一个穿青衣服的人,见马连嘶带跳,就替他拉住马嚼环,说:“天快黑了,我家主人请你去住一宿。”彭问:“这是什么地方?”青衣人答:“这是四川阆中县。”彭吓了一跳:半天功夫出来一千多里路了!便问:“你家主人是谁?”答道:“到了你就知道了。”彭又问:“在哪儿?”回答说:“近在咫尺。”说完就替他拉着马,人与马都飞一样走起来。

过了个山头,见半山中一层层房屋殿阁,其中夹杂着帐篷,远远地见一伙人穿着礼服,好像在等什么人。彭到了近前下马,与那些人互相打拱问候。一会儿,主人出来了,气度不凡,一副刚猛的样子,穿戴也很特别,向彭拱拱手说:“今天来的客人没有比彭君更远的了。”又礼貌地请彭走在前头。彭谦让地不肯冒然先走,主人拉了他的胳膊走,彭觉得被拉的地方像钳子夹住一样,痛得要折了,不敢再谦让,便顺从地走了。他之后的人还再谦让。主人就连推带拉,客人们有的喊痛,有的摔倒,好像受不了的样子,全依着主人的安排走进了厅堂。里面陈设华丽,两个客人一桌筵席。彭悄悄问同座的人:“主人是谁?”回答说:“是张桓侯啊。”彭很惊愕,连咳嗽都不敢,各座都鸦雀无声,开始喝酒。桓侯说:“我年年叨扰诸位亲朋,现在略备薄酒,表示我一点心意。又逢远来的彭君光临,很是幸运。彭君,在下对您有点小要求,可是你若舍不得,我也不勉强。”彭站起来问:“您指的是什么?”桓侯说:“您骑的马有仙骨,尘世的人不能够驭使它。我打算买匹马跟您换换,怎么样?”彭说:“我哪敢跟您换,赠给您吧。”桓侯说:“我一定还你一匹好马,而且外加一万两银子。”彭听了,离开座位伏在地上叩谢,桓侯命人拉他起来。一会儿,酒菜一起上来了。太阳落下后,桓侯吩咐点了蜡烛,大家起身告辞,彭也要走。桓侯说:“你远道而来,到哪里住?”彭指着同席的人说:“我已经求他给我安排住处了。”桓侯又用大杯挨个儿向客人们敬了酒,对彭说;“你怀里的香草,鲜嫩的,人或畜吃了可以成仙,干枯的也能点金,七根草茎能点一万两黄金。”命令童仆把点金秘方传给他,彭又拜谢。桓侯说:“明天到集市去,你可以在马市里随便挑,不要讲价钱,不管多少我都给他。”又对大家说:“远来的客回家,我可以帮路费。”大家都唯唯答应着。大杯饮尽,都辞别出来。路上才互相询问姓名,彭好士的同席叫刘子翚。同行了二三里路,过了一座小山,就看见村庄了,众客人陪着彭好士都到了刘家,才谈起山上的事很怪。

过去,村里年年有个习惯,宰猪杀羊在桓侯庙前搞些比赛、唱戏一类的活动,叫“赛社”,刘某是领头而且出钱最多的。三天前。赛社才结束。中午,村中每家都有一人被一位外来人邀请到山那边去一趟。问去干什么,谁作东道主,来人言语含混,只是催促得紧。人们过了山,看见了房舍,都觉奇怪。快到门口时,来人才以实相告,大家虽然有些害怕,也不敢退回去。来人说:“先在门口停一停。桓侯还请了一位远方客人,马上就到。”远方客人就是指彭好士。大家在刘家回想起来,又惊又怕。人们中间凡是被主人用手抓过的,都喊胳膊痛,脱下衣服点了蜡烛一照,肉都发黑了。彭看看自己,也一样。大家散后,刘某就收拾床铺叫彭休息。次日一早,村里人争着请他,又陪他赶集去选马,十几天也没挑着好马,彭打算好歹买一匹凑合算了。这天又去马市,见一匹马骨架外形像是良马,骑上一试,其快无比,竞骑回村来了。再到马市找卖马人,卖马人已经走了。于是告诉村人想回家,村人都赠他钱财,他就动身回家了。

买的那匹马,日行五百里。到了家,彭好士说明了马是从千里外骑回来的,人们认为不可能。他拿出从四川带来的东西,大家才信了,都觉得是怪事。那些草茎呢,因为日子久了,都干巴了,数了数正好剩下七根。按照张桓侯教的法子点金,彭家因而骤然富起来。他又到了老地方,专门祭祀桓侯祠,雇戏班,唱了三天戏才回来。


\subsection{1.12.30   粉 蝶}
\label{\detokenize{p00_u5176_u5b83/_u767d_u8bdd_u804a_u658b_u5fd7_u5f02:id493}}
阳曰旦,是琼州的文士。有一次,他偶然从外地回家,乘船在海上行驶,遭遇飓风。船眼看就要被浪打翻,忽然飘来一条空船,他急忙跳上去,回头一看,同船的人都被淹没了。风越来越狂,他闭着眼任风吹船行驶。

过了一会儿,风停了,阳曰旦睁开眼,忽见一个岛屿,房舍成片。他划着船靠近岸边,直到村口。村中寂静无声,阳曰旦走一会坐一会,很长时间,连鸡狗的叫声都听不到。阳曰旦看见一个朝北的大门,松竹掩映。这时已是初冬,墙内不知是什么花,满树蓓蕾。他心中很喜欢这种花,就慢慢地走进去。远远地听见弹琴声,就稍稍停下步子。这时一个婢女从里边出来,大约十四五岁,长得十分艳丽,看见阳曰旦,反身又进了屋。接着,听到琴声歇止,一个少年走出来,惊讶的问阳曰旦从什么地方来。阳曰旦详细地告诉了他。又问阳曰旦的家事,阳曰旦又告诉了他,少年高兴地说:“我们是姻亲啊!”接着就客气地请阳曰旦进院子里来。

院中房舍华丽,又传来琴声。走进房中,见一个少妇端坐着,正在调琴弦,年龄大约在十八九岁,光采照人。少妇看见客人进来,推开琴想避开,少年止住她说:“不用走,这个人是你家亲戚。”就替阳曰旦说了根由。少妇说:“原来是我侄子。”问阳曰旦:“祖母还健在吗?父母多大岁数了?”阳曰旦说: “父母四十多岁,都很安好。只是祖母年已六旬,得了重病,久治不愈,连走路都要人扶。侄儿实在不知道姑姑是哪一房的?请明白告诉我,以便回去告诉家人。” 少妇说:“路途遥远,和你家早就断了音信了。你回去只要告诉你父亲十姑问候他,他自然就知道了。”阳曰旦问:“姑丈是哪族?”少年说:“海屿姓晏。这岛叫神仙岛,离琼州三千里路,我流寓这里时间也不长。”十娘进去,让婢女备办了酒食招待客人。新鲜的菜肴香美可口,也不知道叫什么名字。吃完饭,晏生带着阳曰旦四处游览。阳曰旦见园中桃花、杏花含苞待放,非常奇怪。晏生说:“这里夏天无酷暑,冬天无大寒,四季花开不断。”阳曰旦高兴地说:“真是仙乡啊!回去告诉父母,搬家来和你们作邻居。”晏生只是微笑。

回到书斋,点起蜡烛,见琴横放在桌案上,阳曰旦请求聆听一下晏生演奏的琴曲。晏生就抚琴调弦,这时十娘从里面出来,晏生说:“来,来!你为你的侄子弹一曲吧。”十娘坐下,问侄子:“愿意听什么曲子!”阳曰旦说:“侄儿从来没读过《琴操》,实在说不出愿听什么曲子。”十娘说:“只要随意出个题目,都可以弹出曲调。”阳曰旦笑着说:“海风引舟,也可以作一支曲子吗?”十娘说:“可以。”于是拨弦弹奏起来。像早有曲谱,意调激昂,奔腾入耳。阳曰旦静静地领会,好像自身仍在船上,被飓风吹得随波颠荡。阳曰旦惊叹至极,说:“我可以学学吗?”十娘把琴给他,让他试着勾拨琴弦,说:“可以教你。想学什么曲调?”阳曰旦说:“刚才弹奏的‘飓风操’,不知道几天能学会?请先把曲写下来。我读熟它。”十娘说:“这个曲子没有文字,我是按自己的意想谱曲的。”就另拿了一张琴,作勾剔的动作,让阳曰旦照着做。阳曰旦练习到起更后。音节大略能合得上,晏生夫妻二人才告辞离去。阳曰旦专心一意,对着蜡烛自己弹奏,时间一长,就领悟到了其中的奥妙,不禁高兴得手舞足蹈。一抬头,忽见一个婢女站在灯下,阳曰旦吃惊地说:“你还没有走啊?”婢女笑着说:“十姑命我等你睡下后,关好门把灯移开。”阳曰旦仔细看婢女,见她眼睛明亮,姿态媚人,怦然心动。微微地挑逗她,婢女只是低头笑。阳曰旦更加迷了,猛地站起来搂住她的脖子,婢女急说:“不要这样!夜已经四更了,主人要起来了。如果我们有意,明天晚上也不晚。”正在戏弄拥抱时,听到晏生呼唤:“粉蝶!”婢女变了脸色说:“坏了!”急忙跑出去了。阳曰旦偷偷地跟过去听着,只听晏生说:“我本来就说这个婢女尘缘未灭,你一定要把她收下来,现在怎么样?应该打她三百鞭子!”十娘说:“这丫头有了这种心思,不能再使唤了,不如干脆给我侄子算了。”阳曰旦听了既惭愧又害怕,回到书斋灭了灯睡下了。天亮后,有个童子来侍候他盥洗,没有再看见粉蝶。阳曰旦心中惴惴不安,恐怕受到谴责被赶走。不多会,晏生与十姑一块出来,好像没把那件事放在心上,就考他的琴技。阳曰旦弹了一曲,十娘说:“虽然还没达到出神入化的境界,但已经学到十之八九了。练熟了就可以达到神妙的地步。”阳曰旦又请求教别的曲子。晏生教了他“天女谪降”之曲。这支曲子指法很难,阳曰旦练习了三天,仅能成调。晏生说:“已经学了个大概,以后只须熟就行了。只要练熟这两首曲子,就再没难弹的曲调了。”

阳曰旦很想家,告诉十娘说:“我住在这里,承蒙姑姑抚养。十分快乐,只是担心家中人悬念。这里离家三千里。不知什么时候才能回到家里!”十娘说:“这并不难,你原来坐的船还在,我助你一帆风。侄子没有成家,我已让粉蝶先去了。”又赠送他一张琴,并给他些药说:“回去给祖母医病。这药不但能治好病,还可以延年益寿。”说完就把阳曰旦送到海岸,让他上船。阳曰旦找船浆,十娘说:“不需要这东西。”说完,解下裙子当作船帆,系到船上。阳曰旦担心会迷路,十娘说: “不要担忧,只管听凭风帆飘荡。”系好了帆,阳曰旦上了船,心情凄然,正想拜谢告别,忽然刮起南风,离岸边已经很远了。阳曰旦见船上已经准备了干粮,但是只够吃一天的。心中埋怨十娘吝啬。肚子饿了,又不敢多吃,怕一下子吃光,只吃了一块胡饼,觉得胡饼里外又甜又香。剩下的六七块,阳曰旦珍重地保存起来,也不觉得肚饿了。夕阳要下山了。阳曰旦正后悔来时没有要灯烛,转瞬间,远远看见有人烟。仔细一看,原来是琼州。阳曰旦高兴极了,一会儿就到了岸边。他解下裙子,裹好胡饼,就回家了。

进了门,全家人十分惊奇,原来阳曰旦离家已经十六年了。这时阳曰旦才知道他遇到了神仙。看到祖母病重,阳曰旦便拿出药让祖母吃了,多年的重病立刻好了。家里人都奇怪地问他,阳曰旦就把见到的事情都讲了。祖母伤心地说:“那是你姑姑啊。”当年,老夫人有个小女儿,名叫十娘,生来就有仙姿,许配给晏家。女婿十六岁时,进山没有回来,十娘等到二十多岁,忽然没病死了,埋葬了已经三十多年了。听了阳曰旦的话,大家都怀疑十娘没死。阳曰旦拿出裙子,正是十娘当年在家里穿的那条。阳曰旦又把胡饼分给家人吃,只吃一块,一天都不饿,而且精神倍增。老夫人命人打开十娘的棺墓验视,原来只是一具空棺材。

阳曰旦起初聘了吴家女儿,因为他出去几年没有回来,吴家女儿就嫁了别人。大家都相信十娘的话,等着粉蝶到来。过了一年多也没有音信,才商议另外娶亲。临邑的钱秀才,有个女儿叫荷生,远近都知道她长得漂亮,年已十六岁。还没嫁人,就死了三个未成亲的女婿。阳家就托媒人和钱家订了亲,选好日子成亲拜堂。娶到家后,果然非常艳丽漂亮。阳曰旦仔细一看,原来是粉蝶!惊奇地问她过去的事,钱女茫然不知。原来粉蝶被赶走的日子。正是钱女降生的时辰。阳曰旦每次为她弹奏《天女谪降》,钱女总是手支下巴凝思,好像有所心领神会。


\subsection{1.12.31   李 檀 斯}
\label{\detokenize{p00_u5176_u5b83/_u767d_u8bdd_u804a_u658b_u5fd7_u5f02:id494}}
长山人李檀斯,是国学生。有次,他村中有个老太太“走无常”(迷信说法:阴间如同人世,有时吏不足,就从人世勾生人帮忙,完事后仍将人放回,称“走无常”),对人说:“今夜和一个人共同抬着李檀斯,去投生淄川县柏家庄一家大门崭新的人家,他身躯太重,差点没被他压死!”当时李檀斯正在与客人高兴地喝酒,听到老太太的话,以为是胡说八道。到了夜晚,李檀斯突然无病死去。天明以后,有人赶紧去老太太说的投生地点打听,果然那人家夜晚生了个女孩。


\subsection{1.12.32   锦 瑟}
\label{\detokenize{p00_u5176_u5b83/_u767d_u8bdd_u804a_u658b_u5fd7_u5f02:id495}}
沂水县有个姓王的书生,少年时父母就死了,家里十分贫困。但王生却是一个高雅修洁、清奇洒脱的美少年。当地有个姓兰的富翁,见了王生很喜欢,将自己的女儿嫁给了他,还答应为他盖房子、置田产。王生刚娶妻不久,兰富翁就去世了。妻子的弟兄们都鄙视王生,从不和他来往。特别是妻子兰氏,更是傲慢凶悍,常把丈夫当作奴仆使唤。自己吃美味佳肴,让丈夫吃粗茶淡饭,吃饭时给折两根草杆当筷子,这些王生都忍了下来。

王生十九岁时,去郡里考秀才,结果名落孙山,心里很是懊丧。回到家中,正好妻子不在,锅里熬着羊肉羹。王生便舀起一碗吃起来。一会儿,兰氏走了进来,也不说话,劈手就把锅子端走了。王生十分羞惭,把筷子抛到地上,说:“这种境遇,倒不如死了!”兰氏怨恨的问王生什么时候去死,扔过一盘绳子让他去上吊。王生大怒,将饭碗抛到了兰氏身上,把头打破了,自己离家出走。路上仔细想想,万念俱灰,活着实在是不如死了,便揣着根带子进入一条深谷中。来到树丛里,正要选根树枝系带子,忽见土崖间微微露出条裙子。瞬间,一个小丫鬟冒出来,看见王生,急忙缩了回去。像影子一样消失了,土崖上却没有一点裂痕。王生心知是妖物,但正要寻死,所以也不害怕,将带子放下,一屁股坐在地上,察看动静。一会儿,丫鬟又露出半张脸,往外看了一眼,立即缩了回去。王生心想,如能跟着这些鬼物去,倒能享受到死的乐趣,便抓起块石头,敲打着土崖说:“地下如能进去,请指条路。我不是寻欢的,是求死的!”很久,没有动静。王生又敲着说了一遍。只听土崖内有人说道:“想寻死 先回去吧,晚上再来!”话音虽细得像蜂子鸣叫,却清晰刺耳。王生答道“好吧!”往回走了走,坐等天黑。

不长时间,夕阳落山,天空繁星闪烁。土崖间忽然冒出一片高大的府第,两扇大门静静地敞开着。王生一步步登上台阶,走了进去。才几步,见前面横着一条河,波浪汹涌,热气蒸腾,像是温泉。用手试试,水热得像沸水,不知河有多深。王生怀疑这就是鬼指给他寻死的地方,便一头扎了进去。热水浸透几层衣服,皮肤灼烫得像要烂了一样。幸亏是浮在水面上,没有沉下去。在水中游了很久,渐渐能忍受水的热度,极力挣扎着才爬上河的南岸,所幸全身并没有烫伤。又往前走了会儿,远远望见一座大屋子内透出灯光,便朝着屋子走去。突然一条大狗窜出来,向王生猛扑,一口含住了他的衣服,将袜子撕破了。王生急忙摸起块石头打去,狗才稍往后退了退。接着又有一群狗拦路狂叫起来,都像牛犊一样大。正在危急时刻,一个丫鬟出来将群狗喝退,看着王生说:“寻死的人来?我家娘子可怜你遭受迫害,处境艰难。让我送你去‘安乐窝’。从此后再没有苦难了。”便挑着灯,领着王生开了后门,在昏黑的夜幕下往前走去。

不一会,来到一家,明亮的蜡烛照射着窗户。丫鬟说:“你自己进去,我回去了!”王生进门四下一看,原来是自己家!急忙返身跑了出来,正碰上兰氏使唤的一个老妈子,见了王生说:“找了你一整天,又要往哪里去?”拉着王生进入屋内。只见兰氏用手帕包着脑门上的伤,笑咪咪地从床上下来迎接,说:“我们做夫妻一年多了,连和你开个玩笑都不知道吗?我已经知罪了。你只是受了我一点责备,我可是实打实让你给打伤了,你也可以稍出口气了!”从床头上拿出两锭金子,塞到王生怀里,说:“以后全家的吃穿,你说了算,行了吧?”王生一语不发,将金子扔到地上,夺门跑了出去,仍想去深谷敲那座府第的大门。来到田野里,那个丫鬟行走缓慢,远远地挑着灯还能看得见,王生忙喊叫着追赶,灯停住了。等赶上,丫鬟说:“你又来了!宰负了娘子一片苦心。”王生说:“我想寻死。没和你商量再求活。娘子是大户人家,地下也需要人手,我愿意做苦役。实在感不到活着有什么快乐!”丫鬟劝道:“好死不如赖活,你的想法怎么这样荒谬啊!我家也没别的活,只有淘河、洒扫、喂狗、搬尸,做不到规定数量,就要削下耳朵、割掉鼻子、敲断小腿、剁去脚趾。你能行吗?”王生忙回答说:“能行!”又进入后门。王生问道:“刚才说的那些差事都干些什么?还要搬尸,哪来那么多死尸?”丫鬟说:“我家娘子以慈悲为怀,开了座‘给孤园’,专门收养地下极深处那些无家可归的冤鬼游魂。鬼魂多得成百上千,每天都有死去的,所以需要背了去埋了。请你去看看。”

不一会儿,走近一座门,上写着“给孤园”。进去一看,只见房屋又多又乱,十分污秽,臭气薰天。园里的鬼魂看见灯光,纷纷聚集过来,都是些没脑袋或缺胳膊少腿的,令人不堪入目。王生回过头来想走开。见一具鬼尸横躺在墙下,近前看看,血肉狼藉。丫鬟说:“才半天没搬,就被狗啃成这样。”让王生把鬼尸背走。王生面有难色,丫鬟见状,说:“你若办不到,请仍回你的‘安乐窝’享福。”王生迫不得已,只得将鬼尸背起来,放到偏僻的地方。王生请丫鬟向娘子求情干点别的,以免遭受尸污,丫鬟答应。走近一间屋子,丫鬟说:“先坐在这里等着,我进去替你说说。喂狗的活较轻,我替你谋求这个差事,今后可要报答我!”去了刚一会儿,又跑出来,招呼王 生说:“快来,快来!娘子出来了!”王生急忙跟她进去,见大堂上四下里挂着灯笼,一个女郎靠窗坐着,是一个二十来岁的仙女。王生拜伏在阶下,女郎命丫鬟扶起来,说:“这是个书生,不能养狗。就让他住到西屋里,主管簿籍吧!”王生大喜,忙跪下谢恩。女郎又说:“你看上去是个诚朴的人,可好好做事。如有差错。罪过不小。”王生 连声答应。

丫鬟领着他来到西屋,见屋子非常整洁,王生心中很高兴,感谢丫鬟,又询问娘子的家世。丫鬟回答道:“娘子小名叫锦瑟,是东海薛侯的女儿。我叫春燕,早晚有什么事,就说一声。”说完便离开了;不一会儿,又抱来衣服和被褥,放到床上。王生兴奋终于有了个落脚的地方,天刚明,便起来开始工作,抄录鬼魂名册。属下的仆役,都来参见王生,送了很多酒肉。王生为了避嫌。将酒肉全部退回。每天两餐,都是吃的供应饭。锦瑟娘子察知王生廉洁谨慎,特别赐给他儒生巾和漂亮的新衣服。凡有赏赐,都命春燕送去。春燕生得很标致,跟王生熟了后,常常眉目送情。王生假装糊涂,谨慎地躲避,以免招致罪责。又过了两年多,锦瑟娘子赏给王生的东西超过日常薪俸一倍,但王生谦谨自守,一如既往。

一天夜晚,王生刚睡下,听到内院人声吵嚷。忙起床提刀出门,见内院一片火光,映红了天际。跑到院中暗处一看,一群强盗正在抢劫,仆役们惊骇得四散逃窜。一个仆人发现了王生,催促他快跟他逃。王生不肯,将脸上涂黑,紧了紧腰,杂在强盗中高呼:“不要惊吓了薛娘子!只搜掠财物,不要漏下!”这时,强盗们正到处搜不到锦瑟。王生得知锦瑟还没被捉到,便暗暗潜入府第后面,一个人寻找。碰到个藏着的老妇人,询问后才知道锦瑟和春燕都已翻墙逃走,便也跳过墙去,发现锦瑟二人藏在一个黑暗的角落里。王生说:“这地方怎能藏住人呢?”锦瑟回答道:“我实在走不动了!”王生扔下刀,背起她便跑起来。一直跑了二三里路,累得汗流浃背,才逃进深谷中。将锦瑟放下,让她坐在地上歇息歇息。忽然,一头猛虎挟着疾风窜了过来。王生大惊,急忙要拦住它,猛虎已一口叼住了锦瑟。王生紧紧地揪住虎耳朵,极力将自己的胳膊塞到虎口中,以代替锦瑟。老虎发怒,扔下锦瑟,咔吱一声咬断了王生的胳膊,断臂掉在地上,虎才离去了。锦瑟大哭着说:“苦了你了。苦了你了!”王生在急忙中还没感到疼痛,让丫鬟从衣服上撕下片布子裹住伤口。锦瑟忙阻止,俯下身子找到那根断臂,安到断茬上接好,又包扎起来。东方渐渐发白,天要亮了,三人才慢慢地往回赶来。到家中一看,一片废墟。天亮后,仆人和婆子们才渐渐会集起来。锦瑟亲自到西屋去,探视王生的伤臂,解开绷带一看,断臂已经接好,又拿出药敷到伤口上,才离开了。从此后,锦瑟越发看重王生。让他享用的所有东西都和自己的一样。

王生臂伤痊愈以后,锦瑟在室内摆下酒宴慰劳他。王生来到,赐他坐下;王生再三谦让,才在一角落坐。锦瑟举杯劝酒,犹如对待贵宾。过了会儿,锦瑟说:“我的身子已让你背过,我想效仿过去楚王女和钟建的故事,但没有媒人,羞于自荐。”王生恐慌地说:“娘子对我恩重如山,即使舍上这条命也难以报答。刚才娘子讲的对我是非分之事,我怕遭雷打,实在不敢从命。如果娘子可怜我没有妻室,赐一个丫鬟就已经太过了。”锦瑟默然无语。

一天,锦瑟的大姐瑶台忽然来了。是个四十多岁的美人。到了晚上,瑶台叫进王生,让他坐下,说:“我不远千里赶来,为我妹妹主婚,今晚就把她嫁给你。”王生急忙站起来推辞,瑶台立命拿酒来,命两人喝交杯酒。王生苦苦推谢,瑶台夺过他的酒杯,为他们二人换盏。王生才伏到地上谢罪,接过锦瑟的酒喝了。瑶台出去后,锦瑟对王生说:“实话告诉你吧:我本是仙姬,因为有罪被谪。我自愿来到地下,收养冤鬼,以将功赎罪。赶上遭天魔劫难,才和你有附体姻缘。所以从远方请大姐来,一是为我们主婚;二是让她代理家务,以便我跟你回去。”王生恭敬地说:“在地下最快乐!我家有悍妇,而且屋子狭窄简陋。恐怕你受不了委屈。”锦瑟笑着说:“不妨事,两人欢饮一场,便上床成就了好事。过了几天,锦瑟对王生说:“阴间聚会时间不可太长,你先回去,料理一下家务,我随后自己便去。”于是给王生一匹马骑着,开门出去后,土壁又合上了。

王生骑马回到村中,村里的人见了都大为惊骇。来到家门口,只见高房大屋,焕然一新。原来,王生打伤兰氏离家出走后,兰氏叫来两个哥哥,想等王生回来痛打一顿报仇。等到天亮也没回家,两个哥哥才走了。有人在沟里找到王生的鞋子,怀疑他已经死去。既而一年多没有音讯。有个姓贾的陕中人,请媒人说通了兰氏,就在王生的家里娶了她。半年中又修建了好多房子。后来姓贾的外出经商,又买了一个小老婆回来,从此后兰氏便经常在家吵闹,贾某也常常是几个月不回家。王生询问了实情,大怒,将马拴住,直奔入内。看见原来的那个老妈子,老妈子惊得忙伏在地上叩头。王生痛骂一顿,又让她领着去找兰氏,兰氏却已经跑了。不久,在屋子后面找到了她,她已经上吊自杀了。王生便让人将尸体送回她的娘家。将那个小妾叫来,见十几岁年纪,生的还算标致,晚上便收留了她。贾某托村里的人传话,恳求还他的小老婆,小妾却哀号着不肯去。王生便写下诉状,要告贾某霸占家产,夺人妻子。贾某不敢再要,连忙收了店铺走了。王生怀疑锦瑟负约,一晚正在和妾喝酒,听见外面有车马声,接着有人敲门,原来是锦瑟来了。锦瑟只留下春燕,其他人都让回去了。进入室内,妾行拜见礼,锦瑟端详了端详说:“这人有生男之相,可以代我受苦了。”便赐给她华丽的衣服和明珠首饰,妾拜了后收下,立在一边侍奉。锦瑟拉她坐下,尽情谈笑。过了很久,锦瑟说:“我醉了,想睡觉!”妾便辞出,王生也脱鞋上床。妾出门一进入自己的卧室,却见王生躺在床上,大吃一惊;忙返回原来的屋子窥视,屋里的灯已灭了。此后王生没有一晚不睡在妾处。一夜,妾起来,偷偷的到锦瑟卧室看看,见锦瑟和王生二人正在谈笑,十分惊骇,,忙跑回去告诉王生,床上却没人了!天明后,暗暗告诉王生这些奇怪的事,王生自己也不知道,只觉得有时候睡在锦瑟处,有时又睡在妾处。王生嘱咐妾不要宣扬这事。时间长了后,丫鬟也跟王生私通起来,锦瑟仿佛不知道一般。后来,丫鬟忽然难产,嘴里只叫“娘子”。锦瑟一进去,胎儿马上就下来了,还是一个男孩。锦瑟接生毕,把孩子递到丫鬟怀里,笑着说:“奴婢不可再生啦!业障一多,割爱可就难了!”从此后。丫鬟没再生产。妾生了五个男孩,两个女孩。锦瑟在王生家住了三十年。其间常常返回老家。来来往往都是在黑夜。一天,带着丫鬟走了,从此没再来。锦瑟走后,王生活到八十岁时,忽然带着一个老仆夜间外出,也一去不复返了。


\subsection{1.12.33   太 原 狱}
\label{\detokenize{p00_u5176_u5b83/_u767d_u8bdd_u804a_u658b_u5fd7_u5f02:id496}}
太原有户人家,婆、媳都是寡妇。婆婆方到中年,不能自守。村里一个无赖常常跑到她家里去跟她私通。媳妇看不惯,暗暗地在门口、墙头下阻挡那个无赖,不让进门。婆婆十分羞惭恼恨,找了个茬要休了媳妇。媳妇不愿走,因此婆媳二人天天吵架。婆婆更加愤怒,便反咬一口,向官府诬告媳妇有奸情。官府问她奸夫的姓名,婆婆既:“那人黑夜来天明就走,谁知道是谁?拷打那淫妇,就会知道!”于是,又传唤媳妇。媳妇果然知道奸夫的姓名,但却说是婆婆跟那人私通,不是自己。二人争执不休。官府便将那个无赖拘拿了来,无赖又申辩说:“她们两个我谁都没有私通,是她们婆媳合不来,所以胡说八道冤枉我!”官府说:“一村上百人,怎么单单冤枉你!”将他重打一顿。无赖只得招供说是跟媳妇私通。官府拷打媳妇,她却始终不承认。官府便判决婆婆可以将媳妇赶出家门。媳妇不服。忿忿地又告到了省里。像上次一样,省里也不能判决。

当时,正好淄川县的进士孙柳下做临晋县令,以善断案而闻名。省里便把这个案子下到临晋,让孙县令审理。人犯带到后,孙县令略略审讯了一遍。就将犯人暂且下到狱中。让衙役准备砖头、行块、刀子、尖锥等东西,等黎明时使用。衙役们都困惑不解,说:“要上酷刑,自有板子大棍,怎么拿这些不是刑具的东西审案呢?”不明白是什么意思,姑且准备下再说。

第二天,孙县令升堂。问知吩咐预备的东西都已备好,便命都摆到大堂上。将犯人带上来,又挨个大略审问了一遍,才对婆媳二人说:“这件事也没必要搞得多么清楚明白。淫妇是谁虽然定不下来,但奸夫已经明确。你们家本是清白人家,不过是被坏人一时诱骗了罢了,罪责全在那奸夫身上。大堂上现有刀子、石块,你们自己拿去给我将那奸夫杀了!”婆媳听说,害怕一旦失手会偿命,孙县令说:“不用担心,由我作主!”于是,婆媳二人一同起身,拾起石块砸那个无赖。媳妇早对那无赖恨入骨髓,两手搬起块大石头,恨不能立即砸死他!婆婆则只是拿些小石子往无赖的屁股、大腿上砸。孙县令又命用刀子,媳妇拿起刀来,一刀往那无赖的胸膛上捅去;婆婆则犹豫着不敢下手。孙县令见状,忙阻止说:“行了!我已知道淫妇是谁了!”命将婆婆拿下,严刑拷打,果然讯知实情。痛打了那无赖三十大板,才了结了这个案子。


\subsection{1.12.34   新 郑 讼}
\label{\detokenize{p00_u5176_u5b83/_u767d_u8bdd_u804a_u658b_u5fd7_u5f02:id497}}
长山县石宗玉,是个进士出身,在新郑县当县令。有一个远客张某人,在外经商,因生了病,既不能步行,又不能骑马,便雇了一辆人力车回家。身上带着做买卖赚的五千贯钱。由两个车夫载着他在路上走。

到了新郑县城,两个车夫把车子放在路边,去买东西吃,张某自己一人守着钱躺在车上。有个某甲从这车旁经过,偷眼一看,见车上没有别人,就去抢张的钱。张某有病不能抵挡,被某甲把钱抢了去。张某不顾有病,用尽全身力气爬起来,远远跟在某甲身后。走了不多时,见某甲进了一个村子,他仍紧跟不放,也进了村子。随后又见某甲进了一家门里,张某不敢进人家的宅子,便从短墙上向里张望。恰好某甲放下钱回头一看,也看见了张某,就跑出来抓住张某喊抓贼,并将张某绑起来送到县署去见石县令,恶人先告状,诬告张某做贼。石公问张某,他详细说明了经过,喊冤叫屈。石公因没有什么证据,就责令他们先回去。

张某与某甲下了大堂,都说县官没有青红皂白,石公只当没听见。下堂后,石公回忆起某甲很早就欠赋税,就派人去某甲处追交,结果,第二天,某甲就拿了三两银子来纳税。石公问他银子是哪里来的,某甲说:“卖了衣物换来的。”并且说得有名有姓。石公命人去问纳税人中有没有与某甲一个村的,正好某甲的邻居也来了。石公就传来问他说:“你是某甲的邻居,他的银子是哪里来的,你当然知道。”邻人说:“不知道。”石公又对某甲说:“邻居都不知道你的钱是哪里来的,一定来路不明。”某甲一听便害怕,使眼色给邻居说:“我卖了某东西、某家具,你岂不知道?”邻人急忙说:“对!对!是有这个事。”石公生气地说:“你必定与某甲一同偷过,不动大刑你不会说实话。”邻人一听要动大刑,就赶忙说:“因为我们是邻居,没有敢说实话。现在大刑眼看就到了我身上了,还隐瞒什么?他实在是抢的张某的钱。”石公问出了真情,便放了邻人。

这时,张某因丢了钱,还在城里未走。石公就命某甲把钱还给张某。从这件事,可以看出石公为官是真心为民办事。


\subsection{1.12.35   李 象 先}
\label{\detokenize{p00_u5176_u5b83/_u767d_u8bdd_u804a_u658b_u5fd7_u5f02:id498}}
李象先,是寿光县的名人。他的前世是一座寺庙的火头僧,无病而死。灵魂离体后,游荡在外,栖息在一个牌坊上。往下看见往来行人,头顶上都冒出火光,大概是身体内的阳气。夜晚,想到牌坊上不能久居,但下面房屋一片昏黑,不知应该到哪里去。只有一家灯火通明,灵魂便飘荡着赶去,进入家门,一下子变成了一个婴儿。母亲喂他,看见乳汁十分恐惧;但肚子饥饿不堪,只得闭住眼睛勉强吮吸起来。过了三个多月,就不再吃奶。再喂他,便惊叫得啼哭不休。母亲只得用米汁掺和着枣栗喂养他,才得以长大成人,这就是李象先。李象先儿童时到那座寺庙,看见寺僧,还能一一叫出他们的名字。只是到老都害怕奶水。


\subsection{1.12.36   房 文 淑}
\label{\detokenize{p00_u5176_u5b83/_u767d_u8bdd_u804a_u658b_u5fd7_u5f02:id499}}
开封人邓成德,游学来到兖州,住在一座破庙中,受雇为一个专造户口簿的人抄抄写写。到了年底,同事和仆役们都回家了,只剩下邓成德一个人,在庙里做点饭吃。

一天,天刚明,有个少妇敲门进来,十分艳丽,到佛像前烧上香,叩拜后走了。第二天,少妇又来拜佛。晚上,夜深后,邓生起床掌上灯,刚想做点什么,少妇却早早地来了。邓生便问:“怎么来得这样早?”少妇说:“天明后人太杂,所以不如黑夜来;又担心来得太早会打扰你睡觉休息。刚才望见灯光,知道你已起床,所以来了。”邓生调戏道:“庙里没人,住在这里可免来回奔波之苦。”少妇讥笑道:“庙里没人,难道你是鬼吗?”邓生见能和她亲近,等她拜完佛,就拉她坐下求欢。少妇说:“在佛面前怎能做那种事!你身无片瓦,还敢妄想吗?”邓生执意恳求,少妇才说:“离这里三十里地,有个村庄,村里有六七名儿童还没请到塾师。你可前去找一个叫李前川的人,请求这个差事,就说要带家眷去,让他另准备一间屋子.我就可以跟你过了,这是长久之计。”邓生担心拐人家妇女事发后会获罪,少妇说:“不要紧。我姓房,小名叫文淑。没有亲属,常年寄居在舅父家里,不会有人知道的。”邓生大喜。辞别文淑,去那个村庄拜会李前川,果然被雇为塾师,又约定年前就带家眷来。返回后,告诉文淑经过。文淑先走一步,约定在路上等着他。邓生随后即告别同事,借了匹马往村庄赶去,文淑果然在半路等候。邓生下马,让她骑上,继续赶路。到了学馆,两个人便成了好事,生活在一起。一直过了六七年,竟然像夫妻一样,感情和好,安安稳稳,也没有追捕逃妇的。

后来,文淑忽然生了个儿子。邓生因为家里的妻子不生育,意外得子十分高兴,起名叫“兖生”。文淑却说:“假婚配终究不会变成真的。我马上就要辞别你离去,又生下这么个累人的东西干什么!”邓生惊异地说:“我正想倘若我命好,挣下点钱。和你一块逃回老家,怎么说这种话?”文淑忙笑着说:“多谢,多谢!我可不会献媚谄笑,去仰大婆子的鼻息!给人作奶妈,让孩子难堪。”邓生忙替妻子辩白不妒嫉,文淑默然无语。一个多月后,邓生辞馆,计划和李前川的儿子一同外出经商,告诉文淑说:“我想,指望做塾师度日,难有宽裕的时候。不如学着做点买卖,倒还有赚些钱返回老家的希望。”文淑也不说话。到了夜晚,文淑忽然抱着孩子起来,邓生忙问:“干什么?”文淑说:“我要走了!”邓生急急起床,刚要追问,但门没开,文淑却无影无踪了。邓生惊骇之下,才醒悟文淑不是凡人。因为文淑形迹可疑,走了后也不敢告诉别人,只推说是回娘家去了。

在此以前,邓生离家远游时,曾与妻子娄氏约定,年底一定回来。没想到一去好几年没有消息。有人传言邓生已死,娄氏的哥哥因为娄氏并无子女,便劝她改嫁。娄氏不同意,和哥哥约下再等三年,每天靠纺线织布来维持生活。一天,天黑后,娄氏出去关大门,一个少妇忽然从门外挤进来,怀中抱着一个婴儿,说:“从娘家回来。正好天黑了。知道姐姐一个人住,所以来借宿一晚。”娄氏便让她进屋。到房中仔细一看,是一个二十来岁的美人。娄氏便高兴地和她同床而睡,两人一块逗弄着婴儿。娄氏见婴儿自得像瓠瓜一样,十分可爱,伤感地说:“我怎么就没有这么个东西!”少妇便说:“我正嫌他累人。就把他过继给姐姐作儿子,怎么样?”娄氏说:“别说娘子不舍得,就是舍得,我也没有奶水养活他啊!”少妇道:“这不难。这孩子刚出生时,我也没乳水,喝了半剂药就好了。剩下的药还在这里,就送给你吧。”说着,拿出一个小包放到窗台上。娄氏以为少妇在开玩笑,漫不经心地答应下,也没感到有什么奇怪的。第二天醒来,呼唤少妇,没人答应。一看,孩子在,少女却已开门走了。娄氏十分惊骇,直等到辰时,婴儿饿得号哭起来,娄氏不得已,只得将那包药喝了,一会儿便有乳汁流出来,就喂婴儿。这样过了一年多,孩子长得又白又胖,渐渐会学人说话,娄氏喜爱他不亚于自己亲生的。从此后,便打消了改嫁的念头。只是每天早起后便抱孩子,再不能干活赚钱,家里越发困难起来。

一天,少妇忽然来了。娄氏大吃一惊,害怕是来要孩子的,便先发制人,先责怪她当初不辞而别,接着又喋喋不休地讲起抚养孩子的艰难。少妇笑着说:“姐姐诉说诉说难处,我就扔了儿子不要了吗?”便用手招呼小孩,孩子却哭着扑到了娄氏怀里。少妇骂道:“小犊子不认得亲娘了!”又对娄氏说:“这孩子可是百金不换。拿钱来吧,我们立下买卖字据!”娄氏信以为真,却又拿不出一文钱,脸不禁红了。少妇忙笑着说:“姐姐别怕。我这次来正是为了孩子。自分别后,我一直担心姐姐没有养儿的资本,所以多方求借,凑了十多两银子拿来了。”于是拿出银子递给娄氏。娄氏又担心接受了银子,人家再要孩子自己就没话说了,死活不要。少妇放到床上,自己出门走了。娄氏忙抱着儿子追出门外,人已走远了,喊也不顾。娄氏怀疑少妇负气走了,心里惴惴不安。但自从得到银子,放债生息,家境富裕了不少。

又过了三年,邓生做买卖赚了钱,治办行装,返回家来。夫妻二人久别重逢,欣喜万分。邓生忽然看见了孩子,便问是谁家的,娄氏详细地讲了经过。邓生又问:“叫什么名字?”娄氏说:“他妈喊他兖生。”邓生大吃一惊,说:“这真是我的儿子!”忙问少妇带着孩子来的时间,正是他和文淑分别的那晚。邓生便向妻子讲了和房文淑的悲欢离合,两人因终有一子,倍觉欣慰,期望着文淑还来,却再也没有音讯了。


\subsection{1.12.37   秦 桧}
\label{\detokenize{p00_u5176_u5b83/_u767d_u8bdd_u804a_u658b_u5fd7_u5f02:id500}}
青州的冯中堂家,杀了一头猪。拔去猪毛,见肉上写着一行字:“秦桧七世身”。将猪肉烹了一尝,味道恶臭,不能下咽,只得扔给狗吃了。唉!秦桧的臭肉,恐怕狗也不愿吃啊!

听益都人说,冯中堂的祖先,是在宋朝时被秦桧害死的,所以后代最敬岳飞。在青州城北大街旁建了座“岳王殿”,又塑了秦桧、万俟卨两人,跪在岳飞像前。来往行人每去瞻拜岳王殿时,都用石块投打秦、万二人,殿内香火不绝。后来,朝廷大军征伐于七,冯家子弟毁了岳王像。将秦桧、万俟卨二人的塑像搬到几里外的“子孙娘娘庙”中,让他们跪起娘娘来。恐怕百年以后,必定又有“杜十姨”“伍髭须”之类的讹误出现,(伍髭须、杜十姨:浙西有伍子胥庙,老百姓不知伍为何人,讹传为“伍髭须”,为他塑的像上有五溜长须。又有“杜拾遗祠”。即杜甫祠。又讹传为“杜十姨祠”,还一块商量将“杜十姨”嫁给“伍髭须”)真是太可笑了。

又:青州城内,原有座“澹台子羽祠”,当魏忠贤显赫时,有个世家中人谄媚他。将子羽的塑像毁掉了帽子,打落了胡子,改成魏忠贤的模样,这也算是骇人听闻的一件丑事了!


\subsection{1.12.38   浙 东 生}
\label{\detokenize{p00_u5176_u5b83/_u767d_u8bdd_u804a_u658b_u5fd7_u5f02:id501}}
浙东有个书生房某,到陕西设馆教书,常常对人吹嘘自己的胆大,啥也不怕。

一夜,房某赤身躺在床上睡觉,正睡间,忽然觉得有一个怪物从空中掉下来,浑身毛茸茸的,一下打在他的胸膛上,还有声响。他觉得这怪物有狗那么大,气喘嘘嘘,四只爪子不断地挠动。房生一时大为害怕,想爬起身来,这怪物就用两爪扑倒他;不一会,房生就被吓昏了。

呆了一个时辰,房生朦胧中觉得有人用尖东西刺他的鼻子,他立即打了个喷嚏,便苏醒了过来。睁眼一看,见屋里灯光荧荧。床边还坐着一个美人,这美人笑着对他说:“好一个男子汉。胆子就这么大吗?”房生马上意识到这女子一定是狐,不觉更害怕起来。女子渐渐走近房生,与他戏耍挑逗,房生这才慢慢胆子大了些,便相欢好。待了半年,他俩感情更深。

一天,女子正睡在床上,房生起了不良之心,偷着用打猎的网子蒙住了她。女子醒来,不能动身,就请求房生放开她,但房生只是对她笑而不去放她。女子生气,忽然化作一团白气,从床下冒了出来,非常气愤地:“你终究不是好朋友,赶快送我走。”说着就一手拉着房生出了门。房生就觉得身不由己地跟女子走起来,一霎又腾空而起,在天空飞行。约走了一顿饭的时间,女子忽然一放手,房生便晕头晕脑地从空中掉了下来,正好落在一个财主的园子里。这园子里有一口陷阱,上面盖着用绳子结的网子,房生落在网子上,肚子压在网上,把网也压偏了,半个身子悬在阱口上。他定了定神向阱下一看,一只老虎正蹲在阱底,仰起头来虎视眈眈地看他。老虎想吃他,就猛地跳起来咬人,只差不到一尺就咬着房生。房生这时吓得心胆都碎了。主人的园丁来喂虎,见了房生很觉奇怪,便把他扶上来,这时房生已被吓昏。待了一会儿,才慢慢醒来。园丁问他,他详细说了经过。这个地方是浙江地界,距离房生老家已有四百多里。主人知道这件事后,便赠给他路费,叫他回家。

房生回到家里,告诉别人说:“我虽然两次死过,都是狐所为。但没有狐,我还穷得回不了家呢。”


\subsection{1.12.39   博 兴 女}
\label{\detokenize{p00_u5176_u5b83/_u767d_u8bdd_u804a_u658b_u5fd7_u5f02:id502}}
博兴人王某,有个女儿刚满十五岁。当地一恶霸看中女子的姿色。趁她外出时,将她强抢了去,没有一个人知道。到家后,恶霸逼奸女子,女子不从,号哭着抗拒撕打。恶霸不遂,勒死了她,将尸体用石头缒着,沉在了家门外一口深水塘中。王某到处找不到女儿,正无计可施。天上忽然下起暴雨。雷电绕着恶霸家闪来闪去,突然霹雳一声,一条龙飞腾而下.将恶霸的脑袋拧下来抓走了。不一会儿,雨过天晴,塘中女尸浮了上来,一只手中还抓着个人头,仔细一看,正是恶霸的首级。官府得知,将恶霸的家人逮了去讯问,才知道实情。


\subsection{1.12.40   一 员 官}
\label{\detokenize{p00_u5176_u5b83/_u767d_u8bdd_u804a_u658b_u5fd7_u5f02:id503}}
济南府吴同知,性格刚强,清正廉洁。当时官府里有一条不成文的规矩:哪个官犯了贪污罪,上司总是加以庇护,不但不处罚,反而把他贪污的钱分摊在其他同事身上,没有人敢阻挠或违抗。只有这位吴同知不怕,上司强迫他为赃官垫钱,他不干;上司气得骂他,他回骂说:“我官虽小,也是朝廷任命的。你可以参奏处分我,但不可以咒骂我!要死便死,我绝不会损朝廷之禄,代赃官偿还赃钱!”他这么一说,上司拿他没办法,只得好言劝慰。人们都说那年头不兴走正道,叫我说,不能怪年头不好,是有些混帐人自己不走正道罢了。

跟吴同知同时的有个叫穆情怀的,博兴县高苑镇人,被狐狸精附了体。常常慷慨激昂地谈论世道。外人只能听见座上的说话声,看不见跟他对谈的人。这天他到了济南,朋友们谈话间有人问他:“你既是狐仙附体,该没有不知道的事儿,请问济南府有多少官员?”穆情怀马上答道:“只有一个。”大家听了,都笑他说得不对;又问他为什么那么说,他说:“合济南府虽然有七十二名官员,其实,真够格的只有吴同知一个。”

那时候,泰安张知州,因为脾气倔,人们送外号“橛子”。过去大官僚等有地位的人来游览,登山的人工、牲畜、车舆等一切费用都向当地老百姓摊派。可是张知州到任后就把这个陋规废除了。若是大官跟他要猪啊羊的一类物品,他就说:“我就是一只猪、一只羊,请把我宰了犒劳你的仆人去吧!”大官也无可奈何。张知州自从远离家乡到泰安做官,与妻子儿女分别已十二年。刚到任时,夫人领着儿子从都城来看望他,开头一两天挺喜欢;六七天后。夫人不慌不忙地说:“你做官这么多年,穷困得连蒸饭的甑子上都蒙上了尘土。你难道老糊涂了,不顾子孙了吗?”张公一听这话气坏了,把夫人大骂一通,还令人拿棍子来,逼着夫人跪下挨打。儿子伏在娘身上大哭,恳求代母受过。张公狠狠地打了儿子一顿,才算消了气。夫人难过加失望,领着儿子回了家,发誓说:“老东西就是死在泰安我也不再来了!”过了一年,张知州死在了任上。


\subsection{1.12.41   丐 仙}
\label{\detokenize{p00_u5176_u5b83/_u767d_u8bdd_u804a_u658b_u5fd7_u5f02:id504}}
高玉成是大户人家的子弟,住在金城的广里。他擅长针灸,不论病人穷富都给治。

有一天来了个乞丐,小腿上长着烂疮,躺在路边上,腿上又是脓又是血,臭不可闻。居民们怕他死了,每天给他送点吃的。高玉成见了,可怜他,派人把他扶到家来,安顿在偏房里。仆人们嫌他臭,捂着鼻子远远地站着。高玉成拿出艾草点着,亲自给他针灸,天天供他饭菜。过了几天,乞丐馋了,要汤喝要饼吃,仆人怒骂了他一顿。高玉成知道了,就打发仆人给他汤和饼。没过多久,乞丐又要酒肉,仆人跑来说:“这个要饭的太可笑了!原先在路上躺着的时候,一天连一顿饭也吃不上。现在可好,一天三顿吃着,还嫌孬;给了汤饼又要酒肉,这么贪吃,就该把他扔回大路上去!”高玉成问仆人,他的疮怎么样了,仆人说:“痂快掉了,好像可以走路了。我看他是故意呻吟,装着喊痛。”高玉成说:“唉,给他酒肉能花几个钱?等他恢复了健康,总不会把咱当仇人吧。”仆人假意答应,却不照办,还跟伙伴一起笑话主人傻。第二天,高玉成亲自去看乞丐,乞丐腿一拐一拐地站起来,感谢他:“先生你的大恩大德,就像把死人救活,叫白骨长肉,我真是感激不尽。只是我的疮刚痊愈,还没完全康复,想吃点好的解解馋。”高玉成这才知道他原来的命令仆人并未执行,便把仆人喊来痛打一顿,命令他马上给乞丐送酒肉来,还要把酒烫热。仆人心中暗恨乞丐,到了夜里,仆人放把火把偏房烧了,故意喊:“失火了!快救火呀!”高玉成赶紧起来一看,偏房已变成一片瓦砾,惋惜地说:“唉,这下乞丐完了。”赶快督促大家把火救灭。这时,大家才见乞丐躺在火堆里,正呼呼大睡,鼾声如雷。大家把他推醒,乞丐睁眼一看,故作惊讶说:“咦!屋子哪去了?”人们这才知道他不是平常人。高玉成也更加敬重他,让他到客房里去住,给他换上新衣服,天天与自己在一起。问起他的姓名,自称“陈九”。住了几天,模样也显得有光泽了,而且谈吐不凡,棋下得也好,高玉成常输给他,就天天跟他学棋艺,还真学到了一些下棋的奥秘。这样过了半年,乞丐也不说走,高玉成也是一刻也离不了他。即使来了贵客,也叫上乞丐陪着饮酒。席间掷骰子行酒令,陈九就替高玉成猜点数,每猜必准,高玉成很惊奇。高玉成知道他不是凡人,常求他显显本事,陈九推辞说自己没什么本事。

有一天,陈九说:“我想走了,过去受你的大恩,今天我设小宴请你,你可别带旁人去呀。”高玉成说:“咱在一起处得好好的,怎么忽然走?你一个钱也没有,我哪能去叨扰你呢。”陈九很坚决地说:“一杯酒能花几个钱!”高玉成说:“上哪里去呢?”陈九回答说:“去你后花园。”这时正是严冬季节,高玉成怕花园亭子里冷,陈九说:“不碍事。”高玉成就跟他到了园子里。一进园子,猛觉气候立刻暖和得像阳春三月,进了亭子,更暖和了,有成群的珍奇鸟类一起展开歌喉鸣叫。仿佛暮春时节。亭子中的案子、茶几都镶嵌着玛瑙玉器。还有一架水晶屏风晶莹光亮,可以照人,可以看见里面有花树摇曳,有的正开花,有的花在落;还有一种小鸟。白的像雪,飞来飞去地叫,声音很好听,用手去摸时。却啥也没有。高玉成愣了半天才坐下,又见一只鸜鹆在架上学人说话:“上茶!”一会儿就见一只丹凤鸟叼一个红玉盘飞来,盘中有两只玻璃杯,杯中盛着香茶,来到高玉成面前,伸着长脖子,恭敬地站着。等高玉成饮了,放回茶杯,丹凤鸟又叼了红玉盘子,展翅飞去了。鸜鹆又叫:“上酒!”马上就从太阳里边飞来一只青鸾、一只黄鹤,一只叼酒壶,一只叼酒杯,放在桌上。又有许多鸟儿送来菜肴,纷纷扬扬,鼓翅声不绝于耳。不大功夫,山珍海味摆满了桌案。酒菜都是罕见的上等品。

席上,陈九见高玉成酒量很大,说:“您是海量,得用大杯。”鸜鹆又叫:“大杯伺侯!”忽然,太阳边上光点闪闪,有一只大蝴蝶扇动翅膀用脚抓着刻了鹦鹉花样的酒杯向园中飞来,酒杯装了有一斗的酒;待落到案桌上,高玉成才看出这蝴蝶比大雁还大,蝴蝶的两翅膀形态美丽,上有五彩花纹,高玉成赞不绝口。陈九呼道: “蝶子劝酒!”蝴蝶飞动一下,变成了一个美人,绣衣飞舞。前来敬酒。陈九又说:“还得伴酒呀。”美人于是翩翩起舞,舞到高潮处,两脚离地有一尺多高,不时向后仰头,折腰,头都快碰到脚了;又来了个倒空翻,连点土星也没沾着,边舞边唱道:连翩笑语踏芳丛,低亚花枝拂面红。曲折不知金钿落,更随蝴蝶过篱东。

唱罢,余音袅袅不绝。高玉成高兴得拉她到身边一同饮酒。陈九同意她坐下,并给她酒喝。

高玉成酒后控制不住自己,动了心,猛地把美人抱在怀里。美人却突然变成一只夜叉,眼球突出眼外,牙齿伸出嘴唇,一脸黑疙瘩肉,成了个丑八怪。高玉成吓得赶快放了手,趴在桌子上打哆嗦。陈九用筷子敲敲她的牙,喝斥说:“还不快走!”一敲,又变成了蝴蝶,忽闪忽闪飞走了。高玉成定了定神,告辞出来,仰面见天上月光如水,对陈九说:“您招待我的好酒菜从空中来,您家一定是在天上了。可不可以领我去游玩一番?”陈九说:“可以。”就拉了他的手一跳,离了地面,高玉成立刻觉得身子到了空中,离天不远了。渐渐地看见了一座很高的门,门口像井口那样圆,进到里面亮得跟白昼一样,路面都用苍色石头砌成,又平滑又干净,没有一星儿尘土。有棵几丈高的大树,上面开放着莲花那么大的红花。满满一树。树下有位美貌女郎正在石头上捶一件绛红色韵衣服,漂亮极了。高玉成看得呆呆地站在那里像根木头。女郎发现了,生气地说:“哪里的狂小子,来干吗!”用捶衣捧投中了他的脊背。陈九忙拉他到僻静地方,狠狠责备他。高玉成挨了一棒。酒也醒了,觉得很惭愧,就随陈九出来了,门外有白云接住他们的脚。陈九说:“从现在起,咱们就分别了。我嘱咐你一句,记住:你活不了多大年纪,明天赶快躲到西山去,或许可以免死。”高玉成想挽留他。他转身就走了。高发觉云朵渐渐降低,竞落在自家后园中,可园中景物与陈九请他赴宴时已大不一样了。回到屋里跟妻子一说,两人都很惊异。看看上衣挨棍子的地方,像晚霞一样红,还有股特别的香气。

次日早上,高玉成按陈九的嘱咐,带上干粮上了西山,正逢大雾满天,路都看不清了。踩着荒坡急走,忽然掉进个雾气弥漫的大窟窿里,只觉得很深很深,幸亏没有摔伤。清醒过来,定神一看。雾气蒸腾好像刚打开馒头笼,不免叹息说:“仙人叫我躲灾,终于躲不过命运,在这里头什么时候能出去,还不是一死?”坐了一个时辰,看见洞穴深处隐隐有光亮,便站起来走进去,谁知里边又是一番天地,有两个老者正在下棋,见了他,也不答理,只顾下棋。高玉成蹲在一旁看,下完了棋,老者把棋子收到盒里,才问他怎么到了这个地方。高回答说:“迷了路,掉进来的。”老者说:“这里不是人间,不便久留。我送你回去。”于是领他回到窟窿中。高玉成就觉着脚下有云气托他往上升,一会儿到了平地。一看,山里的树成了深黄色,叶子哗哗地往下落,像是秋末季节,惊得他说:“咦?我冬天来的,怎么忽然变成晚秋了?”跑回家里,妻和孩子都大吃一惊,与他抱头痛哭。高玉成奇怪,问妻子,妻说:“你从上了西山,已经三年了,俺还以为你不在人世了。”高说:“怪了,这是刚才的事呀!”拿出带的干粮一看,全变成粉末了。一家人都很诧异,妻说:“你走后,我梦见两个穿黑衣扎着亮腰带的人,好像来催租税的官差,气势汹汹进屋张望,说:‘高玉成哪里去了?’我不客气地说:‘他出去了。你们即使是官差,也不该闯进人家的内室呀!’两人就走了,边走边嘟哝:‘怪事怪事’什么的。”高玉成才恍然大悟:自已在山里遇见的是仙,妻子梦见的是鬼。

高玉成每逢对着客人,里边穿了挨过棍子的褂子,满座都能闻见那种特别的香气,既不像麝香,也不是兰花香,沾了汗,香气就更浓。


\subsection{1.12.42   人 妖}
\label{\detokenize{p00_u5176_u5b83/_u767d_u8bdd_u804a_u658b_u5fd7_u5f02:id505}}
书生马万宝,是东昌人,为人狂放不羁;妻子田氏,也是放诞风流,夫妻二人感情敦厚。

一天,有个女子来到村中,寄居在马生邻居一个寡老太太家里,自己说是受不了公婆虐待,暂时跑出来躲避。女子缝纫手艺非常精巧,常为老太太做些针线活,老太太很高兴,便长留住了她。过了几天,女子又自称能在深夜给人按摩,专治女人腹部疾病。老太太常到马生家串门,一次向田氏宣扬女子的医术,田氏也没在意。又一天,马生从墙缝中窥见女子,年龄约十八九岁,模样很标致,心里不觉喜欢上了她。暗地里和妻子商量,让妻子装病把女子诱来。田氏便假装生起病来。老太太先过来问候,说:“蒙娘子招呼,她马上就过来;但她怕见男人,到时请不要让你丈夫进来。”田氏说:“家里房子不多,他还得出出进进,可怎么办呢?”装着沉思了一会,说:“晚上西村阿舅家叫他去喝酒,就让他别回来了。这也是容易的事。”老太太答应着走了。田氏便和马生商量好,用以人换人之计来算计这个女子。

天黑后,老太太领着女子来了,问:“郎君晚上回来吗?”田氏答道:“不回来了!”女子高兴地说:“这样才好。”说了几句话,老太太走了,田氏便点起蜡烛,展开被子,让女子先上床,自己也脱了衣服,灭了灯。忽然说:“差点忘了,厨房的门没关上,可别叫狗偷吃了东西。”下床开门出去,换成马生。马生蹑手蹑脚地进来,上床与女子一个枕头躺下。女子颤声说:“我要为娘子治病了!”又说些亲昵的话,马生不语。女子就用手抚摸马生的肚子,渐渐地到了肚脐下,停住手不动,忽然探摸下处,一声惊叫,女子惊讶恐怖的样子,不亚于抓住了毒蛇或蝎子,翻身下床,就想逃走。马生一把捉住,把手伸进女子的两腿间,一抓一把,原来也是男子!马生大骇,急忙喊叫点灯。田氏以为女子不同意,两人闹翻了,点上灯过来想给二人调停调停,进门一看,只见一个男子跪在地上哀求饶命,又羞又怕,忙跑了出去。马生细细究问,他自称是谷城人王二喜,因为哥哥王大喜是桑冲的弟子,所以学到了男扮女装的方法。马生又问:“玷污了多少人?”王二喜答道:“刚出道不久,才十六个人。”马生觉得王二喜的罪恶,应该诛杀,想告到郡府;但又爱怜他生的美貌,不忍心他死,便将他反绑起来阉割了。王二喜鲜血涌流,昏厥过去,一顿饭的功夫才苏醒过来。马生又将他扶到床上,盖上被子,嘱咐说:“我用药给你治伤,伤好后,必须跟我一辈子;否则,我就告到官府,让你去死!”王二喜唯唯连声。

第二天,老太太又来看望,马生骗她说:“她是我的表侄女王二姐。因为是石女,被丈夫家赶出了家门。昨夜跟我说明缘由,我才知道。夜晚她忽然身子不适,我才要去给她买药治病,还要到她丈夫家要求留下她来与我妻子作伴。”老太太听说,进屋探望王二喜,见面色如土,便询问病情。王回答说:“阴处忽然肿胀,可能是生疮。”老太太信以为真,走了。马生便为王二喜疗伤,一天天好起来。夜里二人经常鬼混,早上起来,王二喜就替田氏提水做饭,洒扫庭院,缝补衣服,俨然是个奴婢。住了不长时间桑冲便事败被杀,同党六十七人一并被凌迟处死。只有王二喜漏网,官府传令各地严行缉拿。村里人都怀疑王二喜,便召集村中老太太们,让她们隔着衣裳探摸王二喜的下处,证实是“女子”,大家才打消了疑虑。王二喜很感激马生,后来果真跟了他一辈子,死后,就葬在马氏墓的一侧,坟墓至今隐约还在。


\subsection{1.12.43   附录}
\label{\detokenize{p00_u5176_u5b83/_u767d_u8bdd_u804a_u658b_u5fd7_u5f02:id506}}

\subsection{1.12.44   蛰 蛇}
\label{\detokenize{p00_u5176_u5b83/_u767d_u8bdd_u804a_u658b_u5fd7_u5f02:id507}}
予邑郭生,设帐于东山之和庄,蒙童五六人,皆初入馆者也。书室之南为厕所,乃一牛栏;靠山石壁,壁上多杂草蓁莽。童子入厕,多历时刻而后返。郭责之。则曰:“予在厕中腾云。”郭疑之。童子入厕,从旁睨脱之,见共起空中二三尺,倏起倏堕;移时不动。郭进而细审,见壁缝中一蛇,昂首大于盆,吸气而上。遂遍告庄人共视之。以炬火焚壁,蛇死壁裂。蛇不甚长,而粗则如巨桶。盖蛰于内而不能出,已历多年者也。


\subsection{1.12.45   晋 人}
\label{\detokenize{p00_u5176_u5b83/_u767d_u8bdd_u804a_u658b_u5fd7_u5f02:id508}}
晋人某有勇力,不屑格拒之术,而搏技家当之尽靡。过中州,有少林弟子受其辱,忿告其师,群谋设席相邀,将以困之。既至,先陈茗果。胡桃连壳,坚不可食。某取就案边,伸食指敲之,应手而碎。寺众大骇,优礼而散。


\subsection{1.12.46   龙}
\label{\detokenize{p00_u5176_u5b83/_u767d_u8bdd_u804a_u658b_u5fd7_u5f02:id509}}
博邑有乡民王茂才,早赴田。田畔拾一小儿,四五岁,貌丰美而言笑巧妙。归家子之,灵通非常。至四五年后,有一僧至其家。儿见之,惊避无迹。僧告乡民曰:“此儿乃华山池中五百小龙之一,窃逃于此。”遂出一钵,注水其中,宛一小白蛇游衍于内,袖钵而去。


\subsection{1.12.47   爱 才}
\label{\detokenize{p00_u5176_u5b83/_u767d_u8bdd_u804a_u658b_u5fd7_u5f02:id510}}
仕宦中有妹养宫中而字贵人者,有将官某代作启,中警句云:“令弟从长,奕世近龙光,貂珥曾参于画室;舍妹夫人,十年陪凤辇,霓裳遂灿于朝霞。寒砧之杵可掬,不..夜月之霜:御沟之水可托,无劳云英之咏。”当事者奇其才,遂以文阶换武阶,后至通政使。


\chapter{1   袁了凡-了凡四训}
\label{\detokenize{p00_u5176_u5b83/_u8881_u4e86_u51e1-_u4e86_u51e1_u56db_u8bad:id1}}\label{\detokenize{p00_u5176_u5b83/_u8881_u4e86_u51e1-_u4e86_u51e1_u56db_u8bad::doc}}
\begin{sphinxShadowBox}
\sphinxstyletopictitle{目录}
\begin{itemize}
\item {} 
\phantomsection\label{\detokenize{p00_u5176_u5b83/_u8881_u4e86_u51e1-_u4e86_u51e1_u56db_u8bad:id9}}{\hyperref[\detokenize{p00_u5176_u5b83/_u8881_u4e86_u51e1-_u4e86_u51e1_u56db_u8bad:id1}]{\sphinxcrossref{1   袁了凡-了凡四训}}}
\begin{itemize}
\item {} 
\phantomsection\label{\detokenize{p00_u5176_u5b83/_u8881_u4e86_u51e1-_u4e86_u51e1_u56db_u8bad:id10}}{\hyperref[\detokenize{p00_u5176_u5b83/_u8881_u4e86_u51e1-_u4e86_u51e1_u56db_u8bad:id3}]{\sphinxcrossref{1.1   第一篇 立命之学}}}

\item {} 
\phantomsection\label{\detokenize{p00_u5176_u5b83/_u8881_u4e86_u51e1-_u4e86_u51e1_u56db_u8bad:id11}}{\hyperref[\detokenize{p00_u5176_u5b83/_u8881_u4e86_u51e1-_u4e86_u51e1_u56db_u8bad:id4}]{\sphinxcrossref{1.2   第二篇 改过之法}}}

\item {} 
\phantomsection\label{\detokenize{p00_u5176_u5b83/_u8881_u4e86_u51e1-_u4e86_u51e1_u56db_u8bad:id12}}{\hyperref[\detokenize{p00_u5176_u5b83/_u8881_u4e86_u51e1-_u4e86_u51e1_u56db_u8bad:id5}]{\sphinxcrossref{1.3   第三篇 积善之方}}}

\item {} 
\phantomsection\label{\detokenize{p00_u5176_u5b83/_u8881_u4e86_u51e1-_u4e86_u51e1_u56db_u8bad:id13}}{\hyperref[\detokenize{p00_u5176_u5b83/_u8881_u4e86_u51e1-_u4e86_u51e1_u56db_u8bad:id6}]{\sphinxcrossref{1.4   第四篇 谦德之效}}}

\item {} 
\phantomsection\label{\detokenize{p00_u5176_u5b83/_u8881_u4e86_u51e1-_u4e86_u51e1_u56db_u8bad:id14}}{\hyperref[\detokenize{p00_u5176_u5b83/_u8881_u4e86_u51e1-_u4e86_u51e1_u56db_u8bad:id7}]{\sphinxcrossref{1.5   【袁了凡居士传】}}}

\item {} 
\phantomsection\label{\detokenize{p00_u5176_u5b83/_u8881_u4e86_u51e1-_u4e86_u51e1_u56db_u8bad:id15}}{\hyperref[\detokenize{p00_u5176_u5b83/_u8881_u4e86_u51e1-_u4e86_u51e1_u56db_u8bad:id8}]{\sphinxcrossref{1.6   【袁了凡居士传】【注】}}}

\end{itemize}

\end{itemize}
\end{sphinxShadowBox}


\section{1.1   第一篇 立命之学}
\label{\detokenize{p00_u5176_u5b83/_u8881_u4e86_u51e1-_u4e86_u51e1_u56db_u8bad:id3}}
余童年丧父,母命弃举业学医,谓可以养生,可以济人,且习一艺以成名,尔父夙心也。后余在慈云寺,遇一老者,修髯伟貌,飘飘若仙,余敬礼之。

语余曰:“子仕路中人也,明年即进学矣,何不读书?”余告以故,并叩老者姓氏里居。

曰:“吾姓孔,云南人也。得邵子皇极正传,数该传汝。”余即引之归,告母。

母曰:“善待之。”试其数,谶悉皆验。余遂启读书之念,谋之表兄沈称, 言:“郁海谷先生,在沈友夫家开馆,我送汝寄学甚便。”余遂礼郁为师。

孔为余起数:县考童生,当十四名;府考七十一名,提学考第九名。明年赴考,三处名数皆合。复为卜终身休咎,言:某年考第几名,某年当补廪,某年当贡,贡后某年,当选四川一大尹,在任三年半,即宜告归。五十三岁八月十四日丑时,当终于正寝,惜无子。余备录而谨记之。

自此以后,凡遇考校,其名数先后,皆不出孔公所悬定者。独算余食廪米九十一石五斗当出贡。及食米七十一石,屠宗师即批准补贡,余窃疑之。后果为署印杨公所驳,直至丁卯年(西元1567年),殷秋溟宗师见余场中备卷,叹曰:“五策,即五篇奏议也,岂可使博洽淹贯之儒,老于窗下乎!”遂依县申文准贡,连前食米计之,实九十一石五斗也。余因此益信进退有命,迟速有时,澹然无求矣。

贡入燕都,留京一年,终日静坐,不阅文字。己巳(西元1569年)归,游南雍,未入监,先访云谷会禅师于栖霞山中,对坐一室,凡三昼夜不瞑目。

云谷问曰:“凡人所以不得作圣者,只为妄念相缠耳。汝坐三日,不见起一妄念,何也?”

余曰:“吾为孔先生算定,荣辱生死,皆有定数,即要妄想,亦无可妄想。”

云谷笑曰:“我待汝是豪杰,原来只是凡夫。”问其故?

曰:“人未能无心,终为阴阳所缚,安得无数?但惟凡人有数。极善之人,数固拘他不定;极恶之人,数亦拘他不定。汝二十年来,被他算定,不曾转动一毫,岂非是凡夫?”

余问曰:“然则数可逃乎?”

曰:“命由我作,福自己求。诗书所称,的为明训。我教典中说:‘求富贵得富贵,求男女得男女,求长寿得长寿。’夫妄语乃释迦大戒,诸佛菩萨,岂诳语欺人?”

余进曰:“孟子言:‘求则得之’,是求在我者也。道德仁义可以力求;功名富贵,如何求得?”

云谷曰:“孟子之言不错,汝自错解耳。汝不见六祖说:‘一切福田,不离方寸;从心而觅,感无不通。’求在我,不独得道德仁义,亦得功名富贵;内外双得,是求有益于得也。若不反躬内省,而徒向外驰求,则求之有道,而得之有命矣,内外双失,故无益。”

因问:“孔公算汝终身若何?”余以实告。

云谷曰:“汝自揣应得科第否?应生子否?”余追省良久,曰:“不应也。科第中人,有福相,余福薄,又不能积功累行,以基厚福;兼不耐烦剧,不能容人;时或以才智盖人,直心直行,轻言妄谈。凡此皆薄福之相也,岂宜科第哉。地之秽者多生物,水之清者常无鱼;余好洁,宜无子者一;和气能育万物,余善怒,宜无子者二;爱为生生之本,忍为不育之根;余矜惜名节,常不能舍己救人,宜无子者三; 多言耗气,宜无子者四;喜饮铄精,宜无子者五; 好彻夜长坐,而不知葆元毓神,宜无子者六。其余过恶尚多,不能悉数。”

云谷曰:“岂惟科第哉。世间享千金之者,定是千金人物;享百金之产者,定是百金人物;应饿死者,定是饿死人物;天不过因材而笃,几曾加纤毫意思。即如生子,有百世之德者,定有百世子孙保之;有十世之德者,定有十世子孙保之;有三世二世之德者,定有三世二世子孙保之;其斩焉无后者,德至薄也。 汝今既知非。将向来不发科第,及不生子之相,尽情改刷;务要积德,务要包荒,务要和爱,务要惜精神。从前种种,譬如昨日死;从后种种,譬如今日生;此义理再生之身。夫血肉之身,尚然有数;义理之身,岂不能格天。太甲曰:‘天作孽,犹可违;自作孽,不可活。’诗云:‘永言配命,自求多福。’孔先生算汝不登科第,不生子者,此天作之孽,犹可得而违;汝今扩充德性,力行善事,多积阴德,此自己所作之福也,安得而不受享乎?易为君子谋,趋吉避凶;若言天命有常,吉何可趋,凶何可避?开章第一义,便说:‘积善之家,必有余庆。’汝信得及否?”

余信其言,拜而受教。因将往日之罪,佛前尽情发露,为疏一通,先求登科;誓行善事三千条,以报天地祖宗之德。

云谷出功过格示余,令所行之事,逐日登记;善则记数,恶则退除,且教持准提咒,以期必验。

语余曰:“符录家有云:‘不会书符,被鬼神笑。’此有秘传,只是不动念也。执笔书符,先把万缘放下,一尘不起。从此念头不动处,下一点,谓之混沌开基。由此而一笔挥成,更无思虑,此符便灵。凡祈天立命,都要从无思无虑处感格。孟子论立命之学,而曰:‘夭寿不贰。’夫夭寿,至贰者也。当其不动念时,孰为夭,孰为寿?细分之,丰歉不贰,然后可立贫富之命;穷通不贰,然后可立贵贱之命;夭寿不贰,然后可立生死之命。人生世间,惟死生为重,曰夭寿,则一切顺逆皆该之矣。至修身以俟之,乃积德祈天之事。曰修,则身有过恶,皆当治而去之;曰俟,则一毫觊觎,一毫将迎,皆当斩绝之矣。到此地位,直造先天之境,即此便是实学。汝未能无心,但能持准提咒,无记无数,不令间断,持得纯熟,于持中不持,于不持中持。到得念头不动,则灵验矣。”

余初号学海,是日改号了凡;盖悟立命之说,而不欲落凡夫窠臼也。从此而后,终日兢兢,便觉与前不同。前日只是悠悠放任,到此自有战兢惕厉景象,在暗室屋漏中,常恐得罪天地鬼神;遇人憎我毁我,自能恬然容受。

到明年(西元1570年)礼部考科举,孔先生算该第三,忽考第一;其言不验,而秋闱中式矣。然行义未纯,检身多误;或见善而行之不勇,或救人而心常自疑;或身勉为善,而口有过言;或醒时操持,而醉后放逸;以过折功,日常虚度。自己巳岁(西元1569年)发愿,直至己卯岁(西元1579年),历十余年,而三千善行始完。

时方从李渐庵入关,未及回向。庚辰(西元1580年)南还。始请性空、慧空诸上人,就东塔禅堂回向。遂起求子愿,亦许行三千善事。辛巳(西元1581年),生男天启。

余行一事,随以笔记;汝母不能书,每行一事,辄用鹅毛管,印一朱圈于历日之上。或施食贫人,或放生命,一日有多至十余者。至癸未(西元1583年)八月,三千之数已满。复请性空辈,就家庭回向。九月十三日,复起求中进士愿,许行善事一万条,丙戌(西元1586年)登第,授宝坻知县。

余置空格一册,名曰治心篇。晨起坐堂,家人携付门役,置案上,所行善恶,纤悉必记。夜则设桌于庭,效赵阅道焚香告帝。

汝母见所行不多,辄颦蹙曰:“我前在家,相助为善,故三千之数得完;今许一万,衙中无事可行,何时得圆满乎?”

夜间偶梦见一神人,余言善事难完之故。神曰:“只减粮一节,万行俱完矣。”盖宝坻之田,每亩二分三厘七毫。余为区处,减至一分四厘六毫,委有此事,心颇惊疑。适幻余禅师自五台来,余以梦告之,且问此事宜信否?

师曰:“善心真切,即一行可当万善,况合县减粮,万民受福乎?”吾即捐俸银,请其就五台山斋僧一万而回向之。

孔公算予五十三岁有厄,余未尝祈寿,是岁竟无恙,今六十九矣。书曰:“天难谌,命靡常。”又云:“惟命不于常”,皆非诳语。

吾于是而知,凡称祸福自己求之者,乃圣贤之言。若谓祸福惟天所命,则世俗之论矣。

汝之命,未知若何?即命当荣显,常作落寞想;即时当顺利,常作拂逆想;即眼前足食,常作贫窭想;即人相爱敬,常作恐惧想;即家世望重,常作卑下想;即学问颇优,常作浅陋想。

远思扬德,近思盖父母之愆;上思报国之恩,下思造家之福;外思济人之急,内思闲己之邪。

务要日日知非,日日改过;一日不知非,即一日安于自是; 一日无过可改,即一日无步可进;天下聪明俊秀不少,所以德不加修,业不加广者,只为因循二字,耽阁一生。

云谷禅师所授立命之说,乃至精至邃,至真至正之理,其熟读而勉行之,毋自旷也。


\section{1.2   第二篇 改过之法}
\label{\detokenize{p00_u5176_u5b83/_u8881_u4e86_u51e1-_u4e86_u51e1_u56db_u8bad:id4}}
春秋诸大夫,见人言动,亿而谈其祸福,靡不验者,左国诸记可观也。大都吉凶之兆,萌乎心而动乎四体,其过於厚者常获福,过於薄者常近祸,俗眼多翳,谓有未定而不可测者。至诚合天,福之将至,观而必先知之矣。祸之将至,观其不善而必先知之矣。今欲获福而远祸,未论行善,先须改过。

但改过者,第一,要发耻心。思古之圣贤,与我同为丈夫,彼何以百世可师?我何以一身瓦裂?耽染尘情,私行不义,谓人不知,傲然无愧,将日沦於禽兽而不自知矣;世之可羞可耻者,莫大乎此。孟子曰:耻之於人大矣。以其得之则圣贤,失之则禽兽耳。此改过之要机也。

第二,要发畏心。天地在上,鬼神难欺,吾虽过在隐微,而天地鬼神,实鉴临之,重则降之百殃,轻则损其现福,吾何可以不惧?不惟此也。闲居之地,指视昭然;吾虽掩之甚密,文之甚巧,而肺肝早露,终难自欺;被人觑破,不值一文矣,乌得不懔懔?不惟是也。一息尚存,弥天之恶,犹可悔改;古人有一生作恶,临死悔悟,发一善念,遂得善终者。谓一念猛厉,足以涤百年之恶也。譬如千年幽谷,一灯才照,则千年之暗俱除;故过不论久近,惟以改为贵。但尘世无常,肉身易殒,一息不属,欲改无由矣。明则千百年担负恶名,虽孝子慈孙,不能洗涤;幽则千百劫沈沦狱报,虽圣贤佛菩萨,不能援引。乌得不畏?

第三,须发勇心。人不改过,多是因循退缩;吾须奋然振作,不用迟疑,不烦等待。小者如芒刺在肉,速与抉剔;大者如毒蛇啮指,速与斩除,无丝毫凝滞,此风雷之所以为益也。

具是三心,则有过斯改,如春冰遇日,何患不消乎?然人之过,有从事上改者,有从理上改者,有从心上改者;工夫不同,效验亦异。

如前日杀生,今戒不杀;前日怒詈,今戒不怒;此就其事而改之者也。强制於外,其难百倍,且病根终在,东灭西生,非究竟廓然之道也。

善改过者,未禁其事,先明其理;如过在杀生,即思曰:上帝好生,物皆恋命,杀彼养己,岂能自安?且彼之杀也,既受屠割,复入鼎镬,种种痛苦,彻入骨髓;己之养也,珍膏罗列,食过即空,疏食菜羹,尽可充腹,何必戕彼之生,损己之福哉?又思血气之属,皆含灵知,既有灵知,皆我一体;纵不能躬修至德,使之尊我亲我,岂可日戕物命,使之仇我憾我於无穷也?一思及此,将有对食痛心,不能下咽者矣。

如前日好怒,必思曰:人有不及,情所宜矜;悖理相干,於我何与?本无可怒者。又思天下无自是之豪杰,亦无尤人之学问;有不得,皆己之德未修,感未至也。吾悉以自反,则谤毁之来,皆磨炼玉成之地;我将欢然受赐,何怒之有?又闻而不怒,虽谗焰薰天,如举火焚空,终将自息;闻谤而怒,虽巧心力辩,如春蚕作茧,自取缠绵;怒不惟无益,且有害也。其馀种种过恶,皆当据理思之。此理既明,过将自止。

何谓从心而改?过有千端,惟心所造;吾心不动,过安从生?学者於好色,好名,好货,好怒,种种诸过,不必逐类寻求;但当一心为善,正念现前,邪念自然污染不上。如太阳当空,魍魉潜消,此精一之真传也。过由心造,亦由心改,如斩毒树,直断其根,奚必枝枝而伐,叶叶而摘哉?

大抵最上治心,当下清净;才动即觉,觉之即无;苟未能然,须明理以遣之;又未能然,须随事以禁之;以上事而兼行下功,未为失策。执下而昧上,则拙矣。

顾发愿改过,明须良朋提醒,幽须鬼神证明;一心忏悔,昼夜不懈,经一七,二七,以至一月,二月,三月,必有效验。

或觉心神恬旷;或觉智慧顿开;或处冗沓而触念皆通;或遇怨仇而回镇作喜;或梦吐黑物;或梦往圣先贤,提携接引;或梦飞步太虚;或梦幢幡宝盖,种种胜事,皆过消灭之象也。然不得执此自高,画而不进。

昔蘧伯玉当二十岁时,已觉前日之非而尽改之矣。至二十一岁,乃知前之所改,未尽也;及二十二岁,回视二十一岁,犹在梦中,岁复一岁,递递改之,行年五十,而犹知四十九年之非,古人改过之学如此。

吾辈身为凡流,过恶猬集,而回思往事,常若不见其有过者,心粗而眼翳也。然人之过恶深重者,亦有效验:或心神昏塞,转头即忘;或无事而常烦恼;或见君子而赧然相沮;或闻正论而不乐;或施惠而人反怨;或夜梦颠倒,甚则妄言失志;皆作孽之相也,苟一类此,即须奋发,舍旧图新,幸勿自误。


\section{1.3   第三篇 积善之方}
\label{\detokenize{p00_u5176_u5b83/_u8881_u4e86_u51e1-_u4e86_u51e1_u56db_u8bad:id5}}
易曰:「积善之家,必有馀庆。」昔颜氏将以女妻叔梁纥,而历叙其祖宗积德之长,逆知其子孙必有兴者。孔子称舜之大孝,曰:「宗庙飨之,子孙保之」,皆至论也。试以往事徵之。

杨少师荣,建宁人。世以济渡为生,久雨溪涨,横流冲毁民居,溺死者顺流而下,他舟皆捞取货物,独少师曾祖及祖,惟救人,而货物一无所取,乡人嗤其愚。逮少师父生,家渐裕,有神人化为道者,语之曰:「汝祖父有阴功,子孙当贵显,宜葬某地。」遂依其所指而窆之,即今白兔坟也。后生少师,弱冠登第,位至三公,加曾祖,祖,父,如其官。子孙贵盛,至今尚多贤者。

鄞人杨自惩,初为县吏,存心仁厚,守法公平。时县宰严肃,偶挞一囚,血流满前,而怒犹未息,杨跪而宽解之。宰曰:「怎奈此人越法悖理,不由人不怒。」自惩叩首曰:「上失其道,民散久矣,如得其情,哀矜勿喜;喜且不可,而况怒乎?」宰为之霁颜。

家甚贫,馈遗一无所取,遇囚人乏粮,常多方以济之。一日,有新囚数人待哺,家又缺米;给囚则家人无食;自顾则囚人堪悯;与其妇商之。

妇曰:「囚从何来?」

曰:「自杭而来。沿路忍饥,菜色可掬。」因撤己之米,煮粥以食囚。后生二子,长曰守陈,次曰守址,为南北吏部侍郎;长孙为刑部侍郎;次孙为四川廉宪,又俱为名臣;今楚亭,德政,亦其裔也。

昔正统间,邓茂七倡乱於福建,士民从贼者甚众;朝廷起鄞县张都宪楷南征,以计擒贼,后委布政司谢都事,搜杀东路贼党;谢求贼中党附册籍,凡不附贼者,密授以白布小旗,约兵至日,插旗门首,戒军兵无妄杀,全活万人;后谢之子迁,中状元,为宰辅;孙丕,复中探花。

莆田林氏,先世有老母好善,常作粉团施人,求取即与之,无倦色;一仙化为道人,每旦索食六七团。母日日与之,终三年如一日,乃知其诚也。因谓之曰:「吾食汝三年粉团,何以报汝?府后有一地,葬之,子孙官爵,有一升麻子之数。」其子依所点葬之,初世即有九人登第,累代簪缨甚盛,福建有无林不开榜之谣。

冯琢庵太史之父,为邑庠生。隆冬早起赴学,路遇一人,倒卧雪中,扪之,半僵矣。遂解己绵裘衣之,且扶归救苏。梦神告之曰:「汝救人一命,出至诚心,吾遣韩琦为汝子。」及生琢庵,遂名琦。

台州应尚书,壮年习业於山中。夜鬼啸集,往往惊人,公不惧也;一夕闻鬼云:「某妇以夫久客不归,翁姑逼其嫁人。明夜当缢死於此,吾得代矣。」公潜卖田,得银四两。即伪作其夫之书,寄银还家;其父母见书,以手迹不类,疑之。既而曰:「书可假,银不可假,想儿无恙。」妇遂不嫁。其子后归,夫妇相保如初。

公又闻鬼语曰:「我当得代,奈此秀才坏吾事。」

旁一鬼曰:「尔何不祸之?」

曰:「上帝以此人心好,命作阴德尚书矣,吾何得而祸之?」应公因此益自努励,善日加修,德日加厚;遇岁饥,辄捐谷以赈之;遇亲戚有急,辄委曲维持;遇有横逆,辄反躬自责,怡然顺受;子孙登科第者,今累累也。

常熟徐凤竹〔木式〕,其父素富,偶遇年荒,先捐租以为同邑之倡,又分谷以赈贫乏,夜闻鬼唱於门曰:「千不诓,万不诓;徐家秀才,做到了举人郎。」相续而呼,连夜不断。是岁,凤竹果举於乡,其父因而益积德,孳孳不怠,修桥修路,斋僧接众,凡有利益,无不尽心。后又闻鬼唱於门曰:「千不诓,万不诓;徐家举人,直做到都堂。」凤竹官终两浙巡抚。

喜兴屠康僖公,初为刑部主事,宿狱中,细询诸囚情状,得无辜者若干人,公不自以为功,密疏其事,以白堂官。后朝审,堂官摘其语,以讯诸囚,无不服者,释冤抑十馀人。一时辇下咸颂尚书之明。

公复禀曰:「辇毂之下,尚多冤民,四海之广,兆民之众,岂无枉者?宜五年差一减刑官,核实而平反之。」尚书为奏,允其议。时公亦差减刑之列,梦一神告之曰:「汝命无子,今减刑之议,深合天心,上帝赐汝三子,皆衣紫腰金。」是夕夫人有娠,后生应埙,应坤,应【俊】,皆显官。

嘉兴包凭,字信之,其父为池阳太守,生七子,凭最少,赘平湖袁氏,与吾父往来甚厚,博学高才,累举不第,留心二氏之学。一日东游泖湖,偶至一村寺中,见观音像,淋漓露立,即解橐中十金,授主僧,令修屋宇,僧告以功大银少,不能竣事;复取松布四疋,检箧中衣七件与之,内〔纟宁〕褶,系新置,其仆请已之。

凭曰:「但得圣像无恙,吾虽裸裎何伤?」

僧垂泪曰:「舍银及衣布,犹非难事。只此一点心,如何易得。」后功完,拉老父同游,宿寺中。公梦伽蓝来曰:「汝子当享世禄矣。」后子汴,孙柽芳,皆登第,作显官。

嘉善支立之父,为刑房吏,有囚无辜陷重辟,意哀之,欲求其生。囚语其妻曰:「支公嘉意,愧无以报,明日延之下乡,汝以身事之,彼或肯用意,则我可生也。」其妻泣而听命。及至,妻自出劝酒,具告以夫意。支不听,卒为尽力平反之。囚出狱,夫妻登门叩谢曰:「公如此厚德,晚世所稀,今无子,吾有弱女,送为箕帚妾,此则礼之可通者。」支为备礼而纳之,生立,弱冠中魁,官至翰林孔目,立生高,高生禄,皆贡为学博。禄生大纶,登第。

凡此十条,所行不同,同归於善而已。若复精而言之,则善有真,有假;有端,有曲;有阴,有阳;有是,有非;有偏,有正;有半,有满;有大,有小;有难,有易;皆当深辨。为善而不穷理,则自谓行持,岂知造孽,枉费苦心,无益也。

何谓真假?昔有儒生数辈,谒中峰和尚,

问曰:「佛氏论善恶报应,如影随形。今某人善,而子孙不兴;某人恶,而家门隆盛;佛说无稽矣。」

中峰云:「凡情未涤,正眼未开,认善为恶,指恶为善,往往有之。不憾己之是非颠倒,而反怨天之报应有差乎?」

众曰:「善恶何致相反?」中峰令试言。

一人谓「詈人殴人是恶;敬人礼人是善。」

中峰云:「未必然也。」

一人谓「贪财妄取是恶,廉洁有守是善。」

中峰云:「未必然也。」众人历言其状,中峰皆谓不然。因请问。

中峰告之曰:「有益於人,是善;有益於己,是恶。有益於人,则殴人,詈人皆善也;有益於己,则敬人,礼人皆恶也。是故人之行善,利人者公,公则为真;利己者私,私则为假。又根心者真,袭迹者假;又无为而为者真,有为而为者假;皆当自考。」

何谓端曲?今人见谨愿之士,类称为善而取之;圣人则宁取狂狷。至於谨愿之士,虽一乡皆好,而必以为德之贼;是世人之善恶,分明与圣人相反。推此一端,种种取舍,无有不谬;天地鬼神之福善祸淫,皆与圣人同是非,而不与世俗同取舍。凡欲积善,决不可徇耳目,惟从心源隐微处,默默洗涤,纯是济世之心,则为端;苟有一毫媚世之心,即为曲;纯是爱人之心,则为端;有一毫愤世之心,即为曲;纯是敬人之心,则为端;有一毫玩世之心,即为曲;皆当细辨。 何谓阴阳?凡为善而人知之,则为阳善;为善而人不知,则为阴德。阴德,天报之;阳善,享世名。名,亦福也。名者,造物所忌;世之享盛名而实不副者,多有奇祸;人之无过咎而横被恶名者,子孙往往骤发,阴阳之际微矣哉。

何谓是非?鲁国之法,鲁人有赎人臣妾於诸侯,皆受金於府,子贡赎人而不受金。孔子闻而恶之曰:「赐失之矣。夫圣人举事,可以移风易俗,而教道可施於百姓,非独适己之行也。今鲁国富者寡而贫者众,受金则为不廉,何以相赎乎?自今以后,不复赎人於诸侯矣。」

子路拯人於溺,其人谢之以牛,子路受之。孔子喜曰:「自今鲁国多拯人於溺矣。」自俗眼观之,子贡不受金为优,子路之受牛为劣;孔子则取由而黜赐焉。乃知人之为善,不论现行而论流弊;不论一时而论久远;不论一身而论天下。现行虽善,其流足以害人;则似善而实非也;现行虽不善,而其流足以济人,则非善而实是也。然此就一节论之耳。他如非义之义,非礼之礼,非信之信,非慈之慈,皆当抉择。

何谓偏正?昔吕文懿公,初辞相位,归故里,海内仰之,如泰山北斗。有一乡人,醉而詈之,吕公不动,谓其仆曰:「醉者勿与较也。」闭门谢之。逾年,其人犯死刑入狱。吕公始悔之曰:「使当时稍与计较,送公家责治,可以小惩而大戒;吾当时只欲存心於厚,不谓养成其恶,以至於此。」此以善心而行恶事者也。

又有以恶心而行善事者。如某家大富,值岁荒,穷民白昼抢粟於市;告之县,县不理,穷民愈肆,遂私执而困辱之,众始定;不然,几乱矣。故善者为正,恶者为偏,人皆知之;其以善心行恶事者,正中偏也;以恶心而行善事者,偏中正也;不可不知也。

何谓半满?易曰:「善不积,不足以成名;恶不积,不足以灭身。」书曰:「商罪贯盈,如贮物於器。」勤而积之,则满;懈而不积,则不满。此一说也。

昔有某氏女入寺,欲施而无财,止有钱二文,捐而与之,主席者亲为忏悔;及后入宫富贵,携数千金入寺舍之,主僧惟令其徒回向而已。

因问曰:「吾前施钱二文,师亲为忏悔,今施数千金,而师不回向,何也?」

曰:「前者物虽薄,而施心甚真,非老僧亲忏,不足报德;今物虽厚,而施心不若前日之切,令人代忏足矣。」此千金为半,而二文为满也。

锺离授丹於吕祖,点铁为金,可以济世。

吕问曰:「终变否?」

曰:「五百年后,当复本质。」

吕曰:「如此则害五百年后人矣,吾不愿为也。」

曰:「修仙要积三千功行,汝此一言,三千功行已满矣。」此又一说也。

又为善而心不著善,则随所成就,皆得圆满。心著於善,虽终身勤励,止於半善而已。譬如以财济人,内不见己,外不见人,中不见所施之物,是谓三轮体空,是谓一心清净,则斗粟可以种无涯之福,一文可以消千劫之罪,倘此心未忘,虽黄金万镒,福不满也。此又一说也。 何谓大小?昔卫仲达为馆职,被摄至冥司,主者命吏呈善恶二录,比至,则恶录盈庭,其善录一轴,仅如筋而已。索秤称之,则盈庭者反轻,而如筋者反重。

仲达曰:「某年未四十,安得过恶如是多乎?」

曰:「一念不正即是,不待犯也。」因问轴中所书何事?

曰:「朝廷尝兴大工,修三山石桥,君上疏谏之,此疏稿也。」

仲达曰:「某虽言,朝廷不从,於事无补,而能有如是之力。」

曰:「朝廷虽不从,君之一念,已在万民;向使听从,善力更大矣。」故志在天下国家,则善虽少而大;苟在一身,虽多亦小。

何谓难易?先儒谓克己须从难克处克将去。夫子论为仁,亦曰先难。必如江西舒翁,舍二年仅得之束修,代偿官银,而全人夫妇;与邯郸张翁,舍十年所积之钱,代完赎银,而活人妻子,皆所谓难舍处能舍也。如镇江靳翁,虽年老无子,不忍以幼女为妾,而还之邻,此难忍处能忍也;故天降之福亦厚。凡有财有势者,其立德皆易,易而不为,是为自暴。贫贱作福皆难,难而能为,斯可贵耳。

随缘济众,其类至繁,约言其纲,大约有十:第一,与人为善;第二,爱敬存心;第三,成人之美;第四,劝人为善;第五,救人危急;第六,兴建大利;第七,舍财作福;第八,护持正法;第九,敬重尊长;第十,爱惜物命。

何谓与人为善?昔舜在雷泽,见渔者皆取深潭厚泽,而老弱则渔於急流浅滩之中,恻然哀之,往而渔焉;见争者皆匿其过而不谈,见有让者,则揄扬而取法之。期年,皆以深潭厚泽相让矣。夫以舜之明哲,岂不能出一言教众人哉?乃不以言教而以身转之,此良工苦心也。

吾辈处未世,勿以己之长而盖人;勿以己之善而形人;勿以己之多能而困人。收敛才智,若无若虚;见人过失,且涵容而掩覆之。一则令其可改,一则令其有所顾忌而不敢纵,见人有微长可取,小善可录,翻然舍己而从之;且为艳称而广述之。凡日用间,发一言,行一事,全不为自己起念,全是为物立则;此大人天下为公之度也。

何谓爱敬存心?君子与小人,就形迹观,常易相混,惟一点存心处,则善恶悬绝,判然如黑白之相反。故曰:君子所以异於人者,以其存心也。君子所存之心,只是爱人敬人之心。盖人有亲疏贵贱,有智愚贤不肖;万品不齐,皆吾同胞,皆吾一体,孰非当敬爱者?爱敬众人,即是爱敬圣贤;能通众人之志,即是通圣贤之志。何者?圣贤志,本欲斯世斯人,各得其所。吾合爱合敬,而安一世之人,即是为圣贤而安之也。

何谓成人之美?玉之在石,抵掷则瓦砾,追琢则圭璋;故凡见人行一善事,或其人志可取而资可进,皆须诱掖而成就之。或为之奖借,或为之维持;或为白其诬而分其谤;务使成立而后已。

大抵人各恶其非类,乡人之善者少,不善者多。善人在俗,亦难自立。且豪杰铮铮,不甚修形迹,多易指摘;故善事常易败,而善人常得谤;惟仁人长者,匡直而辅翼之,其功德最宏。

何谓劝人为善?生为人类,孰无良心?世路役役,最易没溺。凡与人相处,当方便提撕,开其迷惑。譬犹长夜大梦,而令之一觉;譬犹久陷烦恼,而拔之清凉,为惠最溥。韩愈云:「一时劝人以口,百世劝人以书。」较之与人为善,虽有形迹,然对证发药,时有奇效,不可废也;失言失人,当反吾智。

何谓救人危急?患难颠沛,人所时有。偶一遇之,当如恫【环】在身,速为解救。或以一言伸其屈抑;或以多方济其颠连。崔子曰:「惠不在大,赴人之急可也。」盖仁人之言哉。

何谓兴建大利?小而一乡之内,大而一邑之中,凡有利益,最宜兴建;或开渠导水,或筑堤防患;或修桥梁,以便行旅;或施茶饭,以济饥渴;随缘劝导,协力兴修,勿避嫌疑,勿辞劳怨。

何谓舍财作福?释门万行,以布施为先。所谓布施者,只是舍之一字耳。达者内舍六根,外舍六尘,一切所有,无不舍者。苟非能然,先从财上布施。世人以衣食为命,故财为最重。吾从而舍之,内以破吾之悭,外以济人之急;始而勉强,终则泰然,最可以荡涤私情,〔衤去〕除执吝。

何谓护持正法?法者,万世生灵之眼目也。不有正法,何以参赞天地?何以裁成万物?何以脱尘离缚?何以经世出世?故凡见圣贤庙貌,经书典籍,皆当敬重而修饬之。至於举扬正法,上报佛恩,尤当勉励。

何谓敬重尊长?家之父兄,国之君长,与凡年高,德高,位高,识高者,皆当加意奉事。在家而奉侍父母,使深爱婉容,柔声下气,习以成性,便是和气格天之本。出而事君,行一事,毋谓君不知而自恣也。刑一人,毋谓君不知而作威也。事君如天,古人格论,此等处最关阴德。试看忠孝之家,子孙未有不绵远而昌盛者,切须慎之。

何谓爱惜物命?凡人之所以为人者,惟此恻隐之心而已;求仁者求此,积德者积此。周礼,「孟春之月,牺牲毋用牝。」孟子谓君子远庖厨,所以全吾恻隐之心也。故前辈有四不食之戒,谓闻杀不食,见杀不食,自养者不食,专为我杀者不食。学者未能断肉,且当从此戒之。

渐渐增进,慈心愈长,不特杀生当戒,蠢动含灵,皆为物命。求丝煮茧,锄地杀虫,念衣食之由来,皆杀彼以自活。故暴殄之孽,当与杀生等。至於手所误伤,足所误践者,不知其几,皆当委曲防之。古诗云:「爱鼠常留饭,怜蛾不点灯。」何其仁也!

善行无穷,不能殚述;由此十事而推广之,则万德可备矣。


\section{1.4   第四篇 谦德之效}
\label{\detokenize{p00_u5176_u5b83/_u8881_u4e86_u51e1-_u4e86_u51e1_u56db_u8bad:id6}}
易曰:「天道亏盈而益谦;地道变盈而流谦;鬼神害盈而福谦;人道恶盈而好谦。」是故谦之一卦,六爻皆吉。书曰:「满招损,谦受益。」予屡同诸公应试,每见寒士将达,必有一段谦光可掬。

辛未(西元1571年)计偕,我嘉善同袍凡十人,惟丁敬宇宾,年最少,极其谦虚。

予告费锦坡曰:「此兄今年必第。」

费曰:「何以见之?」

予曰:「惟谦受福。兄看十人中,有恂恂款款,不敢先人,如敬宇者乎?有恭敬顺承,小心谦畏,如敬宇者乎?有受侮不答,闻谤不辩,如敬宇者乎?人能如此,即天地鬼神,犹将佑之,岂有不发者?」及开榜,丁果中式。

丁丑(西元1577年)在京,与冯开之同处,见其虚己敛容,大变其幼年之习。李霁岩直谅益友,时面攻其非,但见其平怀顺受,未尝有一言相报。予告之曰:「福有福始,祸有祸先,此心果谦,天必相之,兄今年决第矣。」已而果然。

赵裕峰,光远,山东冠县人,童年举於乡,久不第。其父为嘉善三尹,随之任。慕钱明吾,而执文见之,明吾悉抹其文,赵不惟不怒,且心服而速改焉。明年,遂登第。

壬辰岁(西元1592年),予入觐,晤夏建所,见其人气虚意下,谦光逼人,归而告友人曰:「凡天将发斯人也,未发其福,先发其慧;此慧一发,则浮者自实,肆者自敛;建所温良若此,天启之矣。」及开榜,果中式。

江阴张畏岩,积学工文,有声艺林。甲午(西元1594年),南京乡试,寓一寺中,揭晓无名,大骂试官,以为眯目。时有一道者,在傍微笑,张遽移怒道者。

道者曰:「相公文必不佳。」

张怒曰:「汝不见我文,乌知不佳?」

道者曰:「闻作文,贵心气和平,今听公骂詈,不平甚矣,文安得工?」张不觉屈服,因就而请教焉。

道者曰:「中全要命;命不该中,文虽工,无益也。须自己做个转变。」

张曰:「既是命,如何转变?」

道者曰:「造命者天,立命者我;力行善事,广积阴德,何福不可求哉?」

张曰:「我贫士,何能为?」

道者曰:「善事阴功,皆由心造,常存此心,功德无量,且如谦虚一节,并不费钱,你如何不自反而骂试官乎?」

张由此折节自持,善日加修,德日加厚。丁酉(西元1597年),梦至一高房,得试录一册,中多缺行。问旁人,

曰:「此今科试录。」

问:「何多缺名?」

曰:「科第阴间三年一考较,须积德无咎者,方有名。如前所缺,皆系旧该中式,因新有薄行而去之者也。」

后指一行云:「汝三年来,持身颇慎,或当补此,幸自爱。」是科果中一百五名。

由此观之,举头三尺,决有神明;趋吉避凶,断然由我。须使我存心制行,毫不得罪於天地鬼神,而虚心屈己,使天地鬼神,时时怜我,方有受福之基。彼气盈者,必非远器,纵发亦无受用。稍有识见之士,必不忍自狭其量,而自拒其福也,况谦则受教有地,而取善无穷,尤修业者所必不可少者也。

古语云:「有志於功名者,必得功名;有志於富贵者,必得富贵。」人之有志,如树之有根,立定此志,须念念谦虚,尘尘方便,自然感动天地,而造福由我。今之求登科第者,初未尝有真志,不过一时意兴耳;兴到则求,兴阑则止。孟子曰:「王之好乐甚,齐其庶几乎?」予於科名亦然。


\section{1.5   【袁了凡居士传】}
\label{\detokenize{p00_u5176_u5b83/_u8881_u4e86_u51e1-_u4e86_u51e1_u56db_u8bad:id7}}
(原文系文言文,为清朝彭绍升撰)

袁了凡先生,本名袁黄,字坤仪;江苏省吴江县人。年轻时入赘到浙江省嘉善县姓殳的人家;因此,在嘉善县得了公费做县里的公读生。他於明穆宗隆庆四年(西元一五七○年),在乡里中了举人;明神宗万历十四年(西元一五八六年)考上进士,奉命到河北省宝坻县做县长。过了七年升拔为兵部「职方司」的主管人,任中刚好碰到日寇侵犯朝鲜,朝鲜向中国求救兵。当时的「经略」(驻朝鲜军事长官)宋应昌奏准请了凡为「军前赞画」(参谋长)的职务,并兼督导支援朝鲜的军队。提督李如松掌握兵权,假装赐给高官俸禄与日寇谈和,日寇信以为真,没有设防;李如松发动突击,攻破形势险要的平壤,因而打败了日寇。

了凡先生因为这件事当面指责李如松,不应用诡诈的手段对付日寇,这样有损大明朝的国威;而且李如松手下的士兵随便杀害百姓,并以头来记功。了凡向李如松据理力争,李如松发怒;不但不接受劝诫,反而独自带著军队东走,使得了凡所率领的军队孤立无援。日寇因而乘机攻击了凡的军队,幸赖了凡机智应对,将日寇击退。而李如松的军队,最后终於被日寇击败了;他想要脱却自己的罪状,反而以十项罪名弹劾袁了凡;了凡很快地被提出审判,终於在「拾遗」(谏官)的仕内,被迫停职返乡。在家里,了凡非常恳切,认真地行善直到去世,过逝时享年七十四岁。

明熹宗天启年间,了凡的冤案终於真相大白,朝廷追叙了凡征讨日寇的功绩,赠封他为「尚宝司少卿」的官衔。了凡先生从当学生时,就非常喜欢研究学问,书不论古今,事不分轻重,他都认真研究,并且非常通达。例如:星象,法律,水利,理数,兵备,政治,堪舆等。

了凡先生在宝坻县当县长时,非常注重人民的福利,常常想做些有利地方的事情;宝坻县当时常有水灾泛滥,了凡先生於是积极兴办水利,将三汊河疏通,筑堤防以抵挡水患侵袭;并且教导百姓沿著海岸种植柳树,每当海水泛滥,挟带沙土冲上岸时,遇到柳树就积挡下来,久而久之变成一道堤防。於是了凡先生又督导百姓在堤防上建造沟渠,并鼓励百姓耕种;因此,荒废的土地渐渐地开垦,了凡先生又免除百姓种种杂役以便民,使得百姓安居乐业。

了凡先生家里并不富有,可是却非常喜欢布施,家居生活俭朴,每天诵经持咒,参禅打坐,修习止观。不管公私事务再忙,早晚定课从不间断。在这当中,了凡先生写下四篇短文,当时命名为「戒子文」,用来训诫他儿子,就是后来广行於世的「了凡四训」这本书。

了凡先生的夫人非常贤慧,经常帮助他行善布施,并且依照功过格记下所做的功德,因为她没有读过书,不会写字;因此用鹅毛管沾红墨水,每天在历书上做记号。有时了凡先生较忙,当天所做功德较少,她就皱眉头,希望先生能多做些善事。有一次,她为儿子裁制冬天的大袍子,想买棉絮做内里。

了凡先生问:「家里有丝绵又轻又暖和,为什麽还买棉絮呢?」

了凡夫人答:「丝绵较贵,棉絮便宜,我想将家里的丝绵拿去换棉絮,这样可以多裁几件棉袄,赠送给贫寒的人家过冬!」

了凡先生听了非常高兴说:「你这样虔诚的布施,不怕我们孩子没有福报了!」他们的儿子袁俨,后来中了进士,最后以广东省高要县的县长退休。


\section{1.6   【袁了凡居士传】【注】}
\label{\detokenize{p00_u5176_u5b83/_u8881_u4e86_u51e1-_u4e86_u51e1_u56db_u8bad:id8}}\begin{enumerate}
\sphinxsetlistlabels{\arabic}{enumi}{enumii}{}{.}%
\item {} 
代用字:

\end{enumerate}
\begin{itemize}
\item {} 
【俊】:如「俊」字形,「人」旁换成「土」旁

\item {} 
【环】:取「环」字右侧,填入「病」中「丙」字的位置

\end{itemize}
\begin{enumerate}
\sphinxsetlistlabels{\arabic}{enumi}{enumii}{}{.}%
\setcounter{enumi}{1}
\item {} 
本文输入和初校所据如下:

\end{enumerate}

了凡四训白话解释【精简本】

著作:明朝,袁了凡

演述:民初,黄智海

整理:民国,王丽民


\chapter{1   Hi,p01散文}
\label{\detokenize{p01_u6563_u6587/Hello_uff0cp01_u6563_u6587:hi-p01}}\label{\detokenize{p01_u6563_u6587/Hello_uff0cp01_u6563_u6587::doc}}
\begin{sphinxShadowBox}
\sphinxstyletopictitle{目录}
\begin{itemize}
\item {} 
\phantomsection\label{\detokenize{p01_u6563_u6587/Hello_uff0cp01_u6563_u6587:id2}}{\hyperref[\detokenize{p01_u6563_u6587/Hello_uff0cp01_u6563_u6587:hi-p01}]{\sphinxcrossref{1   Hi,p01散文}}}
\begin{itemize}
\item {} 
\phantomsection\label{\detokenize{p01_u6563_u6587/Hello_uff0cp01_u6563_u6587:id3}}{\hyperref[\detokenize{p01_u6563_u6587/Hello_uff0cp01_u6563_u6587:post}]{\sphinxcrossref{1.1   post}}}

\end{itemize}

\end{itemize}
\end{sphinxShadowBox}


\section{1.1   post}
\label{\detokenize{p01_u6563_u6587/Hello_uff0cp01_u6563_u6587:post}}

\chapter{1   宋濂-送东阳马生序}
\label{\detokenize{p01_u6563_u6587/_u5b8b_u6fc2-_u9001_u4e1c_u9633_u9a6c_u751f_u5e8f:id1}}\label{\detokenize{p01_u6563_u6587/_u5b8b_u6fc2-_u9001_u4e1c_u9633_u9a6c_u751f_u5e8f::doc}}
\begin{sphinxShadowBox}
\sphinxstyletopictitle{目录}
\begin{itemize}
\item {} 
\phantomsection\label{\detokenize{p01_u6563_u6587/_u5b8b_u6fc2-_u9001_u4e1c_u9633_u9a6c_u751f_u5e8f:id14}}{\hyperref[\detokenize{p01_u6563_u6587/_u5b8b_u6fc2-_u9001_u4e1c_u9633_u9a6c_u751f_u5e8f:id1}]{\sphinxcrossref{1   宋濂-送东阳马生序}}}
\begin{itemize}
\item {} 
\phantomsection\label{\detokenize{p01_u6563_u6587/_u5b8b_u6fc2-_u9001_u4e1c_u9633_u9a6c_u751f_u5e8f:id15}}{\hyperref[\detokenize{p01_u6563_u6587/_u5b8b_u6fc2-_u9001_u4e1c_u9633_u9a6c_u751f_u5e8f:id3}]{\sphinxcrossref{1.1   作品原文}}}

\item {} 
\phantomsection\label{\detokenize{p01_u6563_u6587/_u5b8b_u6fc2-_u9001_u4e1c_u9633_u9a6c_u751f_u5e8f:id16}}{\hyperref[\detokenize{p01_u6563_u6587/_u5b8b_u6fc2-_u9001_u4e1c_u9633_u9a6c_u751f_u5e8f:id4}]{\sphinxcrossref{1.2   词句注释}}}

\item {} 
\phantomsection\label{\detokenize{p01_u6563_u6587/_u5b8b_u6fc2-_u9001_u4e1c_u9633_u9a6c_u751f_u5e8f:id17}}{\hyperref[\detokenize{p01_u6563_u6587/_u5b8b_u6fc2-_u9001_u4e1c_u9633_u9a6c_u751f_u5e8f:id5}]{\sphinxcrossref{1.3   白话译文}}}

\item {} 
\phantomsection\label{\detokenize{p01_u6563_u6587/_u5b8b_u6fc2-_u9001_u4e1c_u9633_u9a6c_u751f_u5e8f:id18}}{\hyperref[\detokenize{p01_u6563_u6587/_u5b8b_u6fc2-_u9001_u4e1c_u9633_u9a6c_u751f_u5e8f:id6}]{\sphinxcrossref{1.4   创作背景}}}

\item {} 
\phantomsection\label{\detokenize{p01_u6563_u6587/_u5b8b_u6fc2-_u9001_u4e1c_u9633_u9a6c_u751f_u5e8f:id19}}{\hyperref[\detokenize{p01_u6563_u6587/_u5b8b_u6fc2-_u9001_u4e1c_u9633_u9a6c_u751f_u5e8f:id7}]{\sphinxcrossref{1.5   作品鉴赏}}}
\begin{itemize}
\item {} 
\phantomsection\label{\detokenize{p01_u6563_u6587/_u5b8b_u6fc2-_u9001_u4e1c_u9633_u9a6c_u751f_u5e8f:id20}}{\hyperref[\detokenize{p01_u6563_u6587/_u5b8b_u6fc2-_u9001_u4e1c_u9633_u9a6c_u751f_u5e8f:id8}]{\sphinxcrossref{1.5.1   第一段}}}

\item {} 
\phantomsection\label{\detokenize{p01_u6563_u6587/_u5b8b_u6fc2-_u9001_u4e1c_u9633_u9a6c_u751f_u5e8f:id21}}{\hyperref[\detokenize{p01_u6563_u6587/_u5b8b_u6fc2-_u9001_u4e1c_u9633_u9a6c_u751f_u5e8f:id9}]{\sphinxcrossref{1.5.2   第二段}}}

\item {} 
\phantomsection\label{\detokenize{p01_u6563_u6587/_u5b8b_u6fc2-_u9001_u4e1c_u9633_u9a6c_u751f_u5e8f:id22}}{\hyperref[\detokenize{p01_u6563_u6587/_u5b8b_u6fc2-_u9001_u4e1c_u9633_u9a6c_u751f_u5e8f:id10}]{\sphinxcrossref{1.5.3   第三段}}}

\item {} 
\phantomsection\label{\detokenize{p01_u6563_u6587/_u5b8b_u6fc2-_u9001_u4e1c_u9633_u9a6c_u751f_u5e8f:id23}}{\hyperref[\detokenize{p01_u6563_u6587/_u5b8b_u6fc2-_u9001_u4e1c_u9633_u9a6c_u751f_u5e8f:id11}]{\sphinxcrossref{1.5.4   总结}}}

\end{itemize}

\item {} 
\phantomsection\label{\detokenize{p01_u6563_u6587/_u5b8b_u6fc2-_u9001_u4e1c_u9633_u9a6c_u751f_u5e8f:id24}}{\hyperref[\detokenize{p01_u6563_u6587/_u5b8b_u6fc2-_u9001_u4e1c_u9633_u9a6c_u751f_u5e8f:id12}]{\sphinxcrossref{1.6   名家点评}}}

\item {} 
\phantomsection\label{\detokenize{p01_u6563_u6587/_u5b8b_u6fc2-_u9001_u4e1c_u9633_u9a6c_u751f_u5e8f:id25}}{\hyperref[\detokenize{p01_u6563_u6587/_u5b8b_u6fc2-_u9001_u4e1c_u9633_u9a6c_u751f_u5e8f:id13}]{\sphinxcrossref{1.7   作者简介}}}

\end{itemize}

\end{itemize}
\end{sphinxShadowBox}

《送东阳马生序》是明代文学家宋濂创作的一篇赠序。在这篇赠序里,作者叙述个人早年虚心求教和勤苦学习的经历,生动而具体地描述了自己借书求师之难,饥寒奔走之苦,并与太学生优越的条件加以对比,有力地说明学业能否有所成就,主要在于主观努力,不在天资的高下和条件的优劣,以勉励青年人珍惜良好的读书环境,专心治学。全文结构严谨,详略有致,用对比说理,在叙事中穿插细节描绘,读来生动感人。


\section{1.1   作品原文}
\label{\detokenize{p01_u6563_u6587/_u5b8b_u6fc2-_u9001_u4e1c_u9633_u9a6c_u751f_u5e8f:id3}}
送东阳马生序1

余幼时即嗜学2。家贫,无从致书以观3,每假借于藏书之家4,手自笔录,计日以还。天大寒,砚冰坚,手指不可屈伸,弗之怠5。录毕,走送之6,不敢稍逾约7。以是人多以书假余,余因得遍观群书。既加冠8,益慕圣贤之道9,又患无硕师、名人与游10,尝趋百里外11,从乡之先达执经叩问12。先达德隆望尊,门人弟子填其室,未尝稍降辞色13。余立侍左右,援疑质理14,俯身倾耳以请;或遇其叱咄15,色愈恭,礼愈至,不敢出一言以复;俟其欣悦16,则又请焉。故余虽愚,卒获有所闻17。

当余之从师也,负箧曳屣18,行深山巨谷中,穷冬烈风19,大雪深数尺,足肤皲裂而不知20。至舍,四支僵劲不能动21,媵人持汤沃灌22,以衾拥覆23,久而乃和。寓逆旅24,主人日再食25,无鲜肥滋味之享。同舍生皆被绮绣26,戴朱缨宝饰之帽27,腰白玉之环28,左佩刀,右备容臭29,烨然若神人30;余则缊袍敝衣处其间31,略无慕艳意。以中有足乐者,不知口体之奉不若人也。盖余之勤且艰若此。

今虽耄老32,未有所成,犹幸预君子之列33,而承天子之宠光,缀公卿之后34,日侍坐备顾问,四海亦谬称其氏名35,况才之过于余者乎?

今诸生学于太学36,县官日有廪稍之供37,父母岁有裘葛之遗38,无冻馁之患矣;坐大厦之下而诵《诗》《书》,无奔走之劳矣;有司业、博士为之师39,未有问而不告,求而不得者也;凡所宜有之书,皆集于此,不必若余之手录,假诸人而后见也。其业有不精,德有不成者,非天质之卑40,则心不若余之专耳,岂他人之过哉!

东阳马生君则,在太学已二年,流辈甚称其贤41。余朝京师42,生以乡人子谒余43,撰长书以为贽44,辞甚畅达,与之论辩45,言和而色夷46。自谓少时用心于学甚劳,是可谓善学者矣!其将归见其亲也47,余故道为学之难以告之。谓余勉乡人以学者,余之志也48;诋我夸际遇之盛而骄乡人者49,岂知余者哉!{[}2{]}


\section{1.2   词句注释}
\label{\detokenize{p01_u6563_u6587/_u5b8b_u6fc2-_u9001_u4e1c_u9633_u9a6c_u751f_u5e8f:id4}}
1.东阳:今浙江东阳市{[}3{]},当时与潜溪同属金华府。马生:姓马的太学生,即文中的马君则。序:文体名,有书序、赠序二种,本篇为赠序。

2.余:我。嗜(shì)学:爱好读书。

3.致:得到。

4.假借:借。

5.弗之怠:即“弗怠之”,不懈怠,不放松读书。弗,不。之,指代抄书。

6.走:跑,这里意为“赶快”。

7.逾约:超过约定的期限。

8.既:已经,到了。加冠:古代男子到二十岁时,举行加冠(束发戴帽)仪式,表示已成年。

9.圣贤之道:指孔孟儒家的道统。宋濂是一个主张仁义道德的理学家,所以十分推崇它。

10.硕(shuò)师:学问渊博的老师。游:交游。

11.尝:曾。趋:奔赴。

12.乡之先达:当地在道德学问上有名望的前辈。这里指浦江的柳贯、义乌的黄溍等古文家。执经叩问:携带经书去请教。

13.稍降辞色:把言辞放委婉些,把脸色放温和些。辞色,言辞和脸色。

14.援疑质理:提出疑难,询问道理。

15.叱(chì)(咄duō):训斥,呵责。

16.俟(sì):等待。忻(xīn):同“欣”。

17.卒:终于。

18.箧(qiè):箱子。曳屣(yèxǐ):拖着鞋子。

19.穷冬:隆冬。

20.皲(jūn)裂:皮肤因寒冷干燥而开裂。

21.僵劲:僵硬。

22.媵人:陪嫁的女子。这里指女仆。持汤沃灌:指拿热水喝或拿热水浸洗。汤:热水。沃灌:浇水洗。

23.衾(qīn):被子。

24.逆旅:旅店。

25.日再食:每日两餐。

26.被(pī)绮绣:穿着华丽的绸缎衣服。被,同“披”。绮,有花纹的丝织品。

27.朱缨宝饰:红穗子上穿有珠子等装饰品。

28.腰白玉之环:腰间悬着白玉圈。

29.容臭:香袋子。臭(xiù):气味,这里指香气。

30.烨(yè页)然:光采照人的样子。

31.缊(yùn)袍:粗麻絮制作的袍子。敝衣:破衣。

32.耄(mào)老:年老。八九十岁的人称耄。宋濂此时已六十九岁。

33.幸预:有幸参与。君子指有道德学问的读书人,另译指有官位的人{[}4{]}。

34.缀:这里意为“跟随”。

35.谬称:不恰当地赞许。这是作者的谦词。

36.诸生:指太学生。太学:明代中央政府设立的教育士人的学校,称作太学或国子监。

37.县官:这里指朝廷。廪(lǐn)稍:当时政府免费供给的俸粮称“廪”或“稍”。

38.裘(qiú):皮衣。葛:夏布衣服。遗(wèi):赠,这里指接济。

39.司业、博士:分别为太学的次长官和教授。

40.非天质之卑:如果不是由于天资太低下。

41.流辈:同辈。

42.朝:旧时臣下朝见君主。宋濂写此文时,正值他从家乡到京城应天(南京)见朱元璋。

43.以乡人子:以同乡之子的身份。谒(yè):拜见。

44.撰(zhuàn):同“撰”,写。长书:长信。贽(zhì):古时初次拜见时所赠的礼物。

45.辩:同辨。{[}5{]}

46.夷:平易。

47.归见:回家探望。

48.“谓余”二句:认为我是在勉励同乡人努力学习,这是说到了我的本意。

49.诋(dǐ):毁谤。际遇之盛:遭遇的得意,指得到皇帝的赏识重用。骄乡人:对同乡骄傲。


\section{1.3   白话译文}
\label{\detokenize{p01_u6563_u6587/_u5b8b_u6fc2-_u9001_u4e1c_u9633_u9a6c_u751f_u5e8f:id5}}
我年幼时就爱学习。因为家中贫穷,无法得到书来看,常向藏书的人家求借,亲手抄录,约定日期送还。天气酷寒时,砚池中的水冻成了坚冰,手指不能屈伸,我仍不放松抄书。抄写完后,赶快送还人家,不敢稍稍超过约定的期限。因此人们大多肯将书借给我,我因而能够看各种各样的书。已经成年之后,更加仰慕圣贤的学说,又苦于不能与学识渊博的老师和名人交往,曾快步走到百里之外,手拿着经书向同乡前辈求教。前辈德高望重,门人学生挤满了他的房间,他的言辞和态度从未稍有委婉。我站着陪侍在他左右,提出疑难,询问道理,低身侧耳向他请教;有时遭到他的训斥,表情更为恭敬,礼貌更为周到,不敢答复一句话;等到他高兴时,就又向他请教。所以我虽然愚钝,最终还是得到不少教益。

当我寻师时,背着书箱,把鞋后帮踩在脚后跟下,行走在深山大谷之中,严冬寒风凛冽,大雪深达几尺,脚和皮肤受冻裂开都不知道。到学舍后,四肢僵硬不能动弹,仆人给我灌下热水,用被子围盖身上,过了很久才暖和过来。住在旅馆,我每天吃两顿饭,没有新鲜肥嫩的美味享受。同学舍的求学者都穿着锦绣衣服,戴着有红色帽带、饰有珍宝的帽子,腰间挂着白玉环,左边佩戴着刀,右边备有香囊,光彩鲜明,如同神人;我却穿着旧棉袍、破衣服处于他们之间,毫无羡慕的意思。因为心中有足以使自己高兴的事,并不觉得吃穿的享受不如人家。我的勤劳和艰辛大概就是这样。

如今我虽已年老,没有什么成就,但所幸还得以置身于君子的行列中,承受着天子的恩宠荣耀,追随在公卿之后,每天陪侍着皇上,听候询问,天底下也不适当地称颂自己的姓名,更何况才能超过我的人呢?

如今的学生们在太学中学习,朝廷每天供给膳食,父母每年都赠给冬天的皮衣和夏天的葛衣,没有冻饿的忧虑了;坐在大厦之下诵读经书,没有奔走的劳苦了;有司业和博士当他们的老师,没有询问而不告诉,求教而无所收获的了;凡是所应该具备的书籍,都集中在这里,不必再像我这样用手抄录,从别人处借来然后才能看到了。他们中如果学业有所不精通,品德有所未养成的,如果不是天赋、资质低下,就是用心不如我这样专一,难道可以说是别人的过错吗!

东阳马生君则,在太学中已学习二年了,同辈人很称赞他的德行。我到京师朝见皇帝时,马生以同乡晚辈的身份拜见我,写了一封长信作为礼物,文辞很顺畅通达,同他论辩,言语温和而态度谦恭。他自己说少年时对于学习很用心、刻苦,这可以称作善于学习者吧!他将要回家拜见父母双亲,我特地将自己治学的艰难告诉他。如果说我勉励同乡努力学习,则是我的志意;如果诋毁我夸耀自己遭遇之好而在同乡前骄傲,难道是了解我吗?


\section{1.4   创作背景}
\label{\detokenize{p01_u6563_u6587/_u5b8b_u6fc2-_u9001_u4e1c_u9633_u9a6c_u751f_u5e8f:id6}}
明洪武十一年(1378),宋濂告老还乡的第二年,应诏从家乡浦江(浙江省浦江县)到应天(今江苏南京)去朝见,同乡晚辈马君则前来拜访,宋濂写下了此篇赠序,介绍自己的学习经历和学习态度,以勉励他人勤奋。


\section{1.5   作品鉴赏}
\label{\detokenize{p01_u6563_u6587/_u5b8b_u6fc2-_u9001_u4e1c_u9633_u9a6c_u751f_u5e8f:id7}}
此篇赠序是宋濂写给他的同乡晚生马君则的。作者赠他这篇文章,是以勉励他勤奋学习,但意思却不直接说出,而是从自己的亲身经历和体会中引申而出,婉转含蓄,平易亲切,字里行间充满了一个硕德长者对晚生后辈的殷切期望,读来令人感动。

全文分三大段。


\subsection{1.5.1   第一段}
\label{\detokenize{p01_u6563_u6587/_u5b8b_u6fc2-_u9001_u4e1c_u9633_u9a6c_u751f_u5e8f:id8}}
写自己青少年时代求学的情形,着意突出其“勤且艰”的好学精神。内中又分四个层次。第一层从借书之难写自己学习条件的艰苦。因家贫无书,只好借书、抄书,尽管天大寒,砚结冰,手指冻僵,也不敢稍有懈怠。第二层从求师之难,写虚心好学的必要。百里求师,恭谨小心。虽遇叱咄,终有所获。第三层从生活条件之难,写自己安于清贫,不慕富贵,因学有所得,故只觉其乐而不觉其苦,强调只要精神充实,生活条件的艰苦是微不足道的。第四层是这一段的总结。由于自己不怕各种艰难,勤苦学习,所以终于学有所成。虽然作者谦虚地说自己“未有所成”,但一代大儒的事实,是不待自言而人都明白的。最后“况才之过于余者乎”的反诘句承前启后,内容十分丰富。首先作者用反诘的语气强调了天分稍高的人若能像自己这样勤奋,必能取得越自己的卓绝成就。同时言外之意是说自己并不是天才,所以能取得现在的成绩,都是勤奋苦学的结果。推而言之,人若不是天资过分低下,学无所成,就只怪自己刻苦努力不够了。从下文知道,马生是一个勤奋好学的青年,他只要坚持下去,其前途也是不可限量的。所以这一句话虽寥寥数字,但含义深厚,作用大,既照应了上文,又关联了下文,扣紧了赠序的主题,把自己对马生的劝诫、勉励和期望,诚恳而又不失含蓄地从容道出,表现出“雍容浑穆”的大家风度。


\subsection{1.5.2   第二段}
\label{\detokenize{p01_u6563_u6587/_u5b8b_u6fc2-_u9001_u4e1c_u9633_u9a6c_u751f_u5e8f:id9}}
紧承第一段,写当代太学生学习条件的优越,与作者青年时代求学的艰难形成鲜明的对照,从反面强调了勤苦学习的必要性。“日有廪稍之供”云云是与上文生活条件之苦对比,“有司业、博士为之师”云云是与上文求师之难对比,“凡所宜有之书,皆集于此”云云,与上文借书之难对比。通过对比,人们很清楚地看出当今太学生在读书、求师、生活等几个方面,都比作者当年的求学条件优越得多,但却业有未精,德有未成。最后用一个选择句式又加一个反诘句式,强调指出:关键就在于这些太学生既不勤奋又不刻苦。这又是对上段第四层的照应。


\subsection{1.5.3   第三段}
\label{\detokenize{p01_u6563_u6587/_u5b8b_u6fc2-_u9001_u4e1c_u9633_u9a6c_u751f_u5e8f:id10}}
以上两段从正反两个方面强调了勤苦学习的重要性,虽未明言是对马生的劝励,而劝励之意自明。然而文章毕竟是为马生而作的,所以至第三段便明确地写到马生,点明写序的目的,这就是“道为学之难”,“勉乡人以学者”。因为劝励的内容在上两段中已经写足,所以这里便只讲些推奖褒美的话,但是殷切款诚之意,马生是不难心领神会的。


\subsection{1.5.4   总结}
\label{\detokenize{p01_u6563_u6587/_u5b8b_u6fc2-_u9001_u4e1c_u9633_u9a6c_u751f_u5e8f:id11}}
宋濂为人宽厚诚谨,谦恭下人。此文也是一如其人,写得情辞婉转,平易亲切。其实按他的声望、地位,他完全可以摆出长者的架子,正面说理大发议论,把这个青年教训一通的。然而他却不这样做。他绝口不说你们青年应当怎样怎样,而只是说“我”曾经怎样怎样,自己放在与对方平等的地位上,用自己亲身的经历和切身的体会去和人谈心。不仅从道理上,而且从形象上、情感上去启发影响读者,使人感到在文章深处有一种崇高的人格感召力量,在阅读过程中,读者会在不知不觉中缩短了与作者思想上的距离,赞同他的意见,并乐于照着他的意见去做。写文章要能达到这一步,决非只是一个文章技巧问题,这是需要有深厚的思想修养作基础的。

其次,作者在说理上,也不是凭空论道,而是善于让思想、道理从事实的叙述中自然地流露出来。而在事实的叙述中,又善于将概括的述说与典型的细节描绘有机地结合起来,这就使文章具体实在,仅在行文上简练生动,而且还具有很强的说服力和感染力。例如在说到读书之难时,作者在概括地叙述了自己因家贫无书,不得不借书、抄书,计日以还的情形后说:“天大寒,砚冰坚,手指不可屈伸,弗之怠。”通过这样一个典型的细节描写,就使人对作者当初读书的勤奋及学习条件的艰苦,有了一个生动形象的具体感受。理在事中,而事颇感人。这也是此文使人乐于赞同并接受作者意见的又一个内在的原因。

而且,文章浑然天成,内在结构却十分严密而紧凑。本来文章所赠送的对象是一篇之主体。然而文章却偏把主体抛在一边,先从自己谈起,从容道来,由己及人,至最后才谈及赠送的对象。看似漫不经心,实则匠心独运。在文章的深层结构中,主宾之间有一种紧密的内在联系,时时针对着主,处处照应到主,而却避免了一般赠序文章直露生硬的缺点,使文章委婉含蓄,意味深长。在写作中又成功地运用了对比映衬的手法,使左右有对比,前后有照应,文章于宽闲中显示严整,“鱼鱼雅雅,自中节度”。这一点给人的印象也是十分深刻的。


\section{1.6   名家点评}
\label{\detokenize{p01_u6563_u6587/_u5b8b_u6fc2-_u9001_u4e1c_u9633_u9a6c_u751f_u5e8f:id12}}
辽宁省作协主席、辽宁大学中文系教授王充闾《中国好文章·你不能错过的文言文》:“本文通过叙述自己年轻时求学的迫切、境遇的艰难,勉励马生等太学生刻苦向学,期于有成。感情真挚,语重心长,说理透彻,颇具感人力量。”


\section{1.7   作者简介}
\label{\detokenize{p01_u6563_u6587/_u5b8b_u6fc2-_u9001_u4e1c_u9633_u9a6c_u751f_u5e8f:id13}}
宋濂

宋濂(1310—1381),字景濂,号潜溪,别号玄真子、玄真道士、玄真遁叟。谥号文宪。浦江(今浙江浦江)人,汉族。明初文学家,曾被明太祖朱元璋誉为“开国文臣”。因其长孙宋慎牵连胡惟庸党案,全家流放茂州。其散文质朴简洁,或雍容典雅,各有特色。他推崇台阁文学,文风淳厚飘逸,为其后“台阁体”作家的文学创作提供范本。其作品大部分被合刻为《宋学士全集》七十五卷。


\chapter{1   岳飞-满江红·怒发冲冠}
\label{\detokenize{p01_u6563_u6587/_u5cb3_u98de-_u6ee1_u6c5f_u7ea2_xb7_u6012_u53d1_u51b2_u51a0:id1}}\label{\detokenize{p01_u6563_u6587/_u5cb3_u98de-_u6ee1_u6c5f_u7ea2_xb7_u6012_u53d1_u51b2_u51a0::doc}}
\begin{sphinxShadowBox}
\sphinxstyletopictitle{目录}
\begin{itemize}
\item {} 
\phantomsection\label{\detokenize{p01_u6563_u6587/_u5cb3_u98de-_u6ee1_u6c5f_u7ea2_xb7_u6012_u53d1_u51b2_u51a0:id10}}{\hyperref[\detokenize{p01_u6563_u6587/_u5cb3_u98de-_u6ee1_u6c5f_u7ea2_xb7_u6012_u53d1_u51b2_u51a0:id1}]{\sphinxcrossref{1   岳飞-满江红·怒发冲冠}}}
\begin{itemize}
\item {} 
\phantomsection\label{\detokenize{p01_u6563_u6587/_u5cb3_u98de-_u6ee1_u6c5f_u7ea2_xb7_u6012_u53d1_u51b2_u51a0:id11}}{\hyperref[\detokenize{p01_u6563_u6587/_u5cb3_u98de-_u6ee1_u6c5f_u7ea2_xb7_u6012_u53d1_u51b2_u51a0:id3}]{\sphinxcrossref{1.1   作品原文}}}

\item {} 
\phantomsection\label{\detokenize{p01_u6563_u6587/_u5cb3_u98de-_u6ee1_u6c5f_u7ea2_xb7_u6012_u53d1_u51b2_u51a0:id12}}{\hyperref[\detokenize{p01_u6563_u6587/_u5cb3_u98de-_u6ee1_u6c5f_u7ea2_xb7_u6012_u53d1_u51b2_u51a0:id4}]{\sphinxcrossref{1.2   词句注释}}}

\item {} 
\phantomsection\label{\detokenize{p01_u6563_u6587/_u5cb3_u98de-_u6ee1_u6c5f_u7ea2_xb7_u6012_u53d1_u51b2_u51a0:id13}}{\hyperref[\detokenize{p01_u6563_u6587/_u5cb3_u98de-_u6ee1_u6c5f_u7ea2_xb7_u6012_u53d1_u51b2_u51a0:id5}]{\sphinxcrossref{1.3   白话译文}}}

\item {} 
\phantomsection\label{\detokenize{p01_u6563_u6587/_u5cb3_u98de-_u6ee1_u6c5f_u7ea2_xb7_u6012_u53d1_u51b2_u51a0:id14}}{\hyperref[\detokenize{p01_u6563_u6587/_u5cb3_u98de-_u6ee1_u6c5f_u7ea2_xb7_u6012_u53d1_u51b2_u51a0:id6}]{\sphinxcrossref{1.4   创作背景}}}

\item {} 
\phantomsection\label{\detokenize{p01_u6563_u6587/_u5cb3_u98de-_u6ee1_u6c5f_u7ea2_xb7_u6012_u53d1_u51b2_u51a0:id15}}{\hyperref[\detokenize{p01_u6563_u6587/_u5cb3_u98de-_u6ee1_u6c5f_u7ea2_xb7_u6012_u53d1_u51b2_u51a0:id7}]{\sphinxcrossref{1.5   作品鉴赏}}}

\item {} 
\phantomsection\label{\detokenize{p01_u6563_u6587/_u5cb3_u98de-_u6ee1_u6c5f_u7ea2_xb7_u6012_u53d1_u51b2_u51a0:id16}}{\hyperref[\detokenize{p01_u6563_u6587/_u5cb3_u98de-_u6ee1_u6c5f_u7ea2_xb7_u6012_u53d1_u51b2_u51a0:id8}]{\sphinxcrossref{1.6   名家点评}}}

\item {} 
\phantomsection\label{\detokenize{p01_u6563_u6587/_u5cb3_u98de-_u6ee1_u6c5f_u7ea2_xb7_u6012_u53d1_u51b2_u51a0:id17}}{\hyperref[\detokenize{p01_u6563_u6587/_u5cb3_u98de-_u6ee1_u6c5f_u7ea2_xb7_u6012_u53d1_u51b2_u51a0:id9}]{\sphinxcrossref{1.7   作者简介}}}

\end{itemize}

\end{itemize}
\end{sphinxShadowBox}

《满江红·怒发冲冠》,一般认为是宋代抗金将领岳飞的词作。此词上片抒写作者对中原重陷敌手的悲愤,对局势前功尽弃的痛惜,表达了自己继续努力争取壮年立功的心愿;下片抒写作者对民族敌人的深仇大恨,对祖国统一的殷切愿望,对国家朝廷的赤胆忠诚。全词情调激昂,慷慨壮烈,显示出一种浩然正气和英雄气质,表现了作者报国立功的信心和乐观主义精神。


\section{1.1   作品原文}
\label{\detokenize{p01_u6563_u6587/_u5cb3_u98de-_u6ee1_u6c5f_u7ea2_xb7_u6012_u53d1_u51b2_u51a0:id3}}
满江红⑴

怒发冲冠⑵,凭阑处⑶、潇潇雨歇⑷。抬望眼,仰天长啸⑸,壮怀激烈⑹。三十功名尘与土⑺,八千里路云和月⑻。莫等闲⑼、白了少年头,空悲切⑽。

靖康耻⑾,犹未雪。臣子恨,何时灭。驾长车,踏破贺兰山缺⑿。壮志饥餐胡虏肉⒀,笑谈渴饮匈奴血⒁。待从头、收拾旧山河,朝天阙⒂。


\section{1.2   词句注释}
\label{\detokenize{p01_u6563_u6587/_u5cb3_u98de-_u6ee1_u6c5f_u7ea2_xb7_u6012_u53d1_u51b2_u51a0:id4}}
⑴满江红:词牌名,又名“上江虹”“念良游”“伤春曲”等。双调九十三字。

⑵怒发(fà)冲冠:气得头发竖起,以至于将帽子顶起。形容愤怒至极。

⑶凭阑:身倚栏杆。阑,同“栏”。

⑷潇潇:形容雨势急骤。

⑸长啸:大声呼叫。啸,蹙口发出的叫声。

⑹壮怀:奋发图强的志向。

⑺“三十”句:谓自己已经三十岁了,得到的功名,如同尘土一样微不足道。三十,是约数。功名,或指岳飞攻克襄阳六郡以后建节晋升之事。

⑻“八千”句:形容南征北战、路途遥远、披星戴月。八千,是约数,极言沙场征战行程之远。

⑼等闲:轻易,随便。

⑽空悲切:即白白的痛苦。

⑾靖康耻:宋钦宗靖康二年(1127),金兵攻陷汴京,虏走徽、钦二帝。靖康,宋钦宗赵桓的年号。

⑿贺兰山:贺兰山脉,位于宁夏回族自治区与内蒙古自治区交界处,当时被金兵占领。一说是位于邯郸市磁县境内的贺兰山。

⒀胡虏:对女真贵族入侵者的蔑称。

⒁匈奴:古代北方民族之一,这里指金入侵者。

⒂朝天阙:朝见皇帝。天阙,本指宫殿前的楼观,此指皇帝居住的地方。明代王熙书《满江红》词碑作“朝金阙”。


\section{1.3   白话译文}
\label{\detokenize{p01_u6563_u6587/_u5cb3_u98de-_u6ee1_u6c5f_u7ea2_xb7_u6012_u53d1_u51b2_u51a0:id5}}
我怒发冲冠登高倚栏杆,一场潇潇细雨刚刚停歇。抬头放眼四望辽阔一片,仰天长声啸叹。壮怀激烈,三十年勋业如今成尘土,征战千里只有浮云明月。莫虚度年华白了少年头,只有独自悔恨悲悲切切。

靖康年的奇耻尚未洗雪,臣子愤恨何时才能泯灭。我只想驾御着一辆辆战车踏破贺兰山敌人营垒。壮志同仇饿吃敌军的肉,笑谈蔑敌渴饮敌军的血。我要从头彻底地收复旧日河山,再回京阙向皇帝报捷。{[}5{]}


\section{1.4   创作背景}
\label{\detokenize{p01_u6563_u6587/_u5cb3_u98de-_u6ee1_u6c5f_u7ea2_xb7_u6012_u53d1_u51b2_u51a0:id6}}
关于此词的创作背景,有多种说法。有学者认为此词约创作于宋高宗绍兴二年(1132)前后,也有人认为作于绍兴四年(1134)岳飞克复襄阳六郡晋升清远军节度使之后。


\section{1.5   作品鉴赏}
\label{\detokenize{p01_u6563_u6587/_u5cb3_u98de-_u6ee1_u6c5f_u7ea2_xb7_u6012_u53d1_u51b2_u51a0:id7}}
此词上片写作者悲愤中原重陷敌手,痛惜前功尽弃的局面,也表达自己继续努力,争取壮年立功的心愿。

开头五句,起势突兀,破空而来。胸中的怒火在熊熊燃烧,不可阻遏。这时,一阵急雨刚刚停止,作者站在楼台高处,正凭栏远望。他看到那已经收复却又失掉的国土,想到了重陷水火之中的百姓,不由得“怒发冲冠”,“仰天长啸”,“壮怀激烈”。“怒发冲冠”是艺术夸张,是说由于异常愤怒,以致头发竖起,把帽子也顶起来了。作者表现出如此强烈的愤怒的感情并不是偶然的,这是他的理想与现实发生尖锐激烈的矛盾的结果。他面对投降派的不抵抗政策,气愤填膺。岳飞之怒,是金兵侵扰中原,烧杀虏掠的罪行所激起的雷霆之怒;岳飞之啸,是无路请缨,报国无门的忠愤之啸;岳飞之怀,是杀敌为国的宏大理想和豪壮襟怀。这几句一气贯注,生动地描绘了一位忠臣义士和忧国忧民的英雄形象。

接着四句激励自己,不要轻易虚度这壮年光阴,争取早日完成抗金大业。“三十功名尘于土”,是对过去的反省,表现作者渴望建立功名、努力抗战的思想。三十岁左右正当壮年,古人认为这时应当有所作为,可是,岳飞悔恨自己功名还与尘土一样,没有什么成就。宋朝以“三十之节”为殊荣。然而,岳飞梦寐以求的并不是建节封侯,身受殊荣,而是渡过黄河,收复国土,完成抗金救国的神圣事业。正如他自己所说“誓将直节报君仇”,“不问登坛万户侯”,对功名感觉不过像尘土一样,微不足道。“八千里路云和月”,是说不分阴晴,转战南北,在为收复中原而战斗。这是对未来的瞻望。“云和月”是特意写出,说出师北伐是十分艰苦的,任重道远,尚须披星戴月,日夜兼程,才能“北逾沙漠,喋血虏廷”(《五岳祠盟记》),赢得最后抗金的胜利。上一句写视功名为尘土,下一句写杀敌任重道远,个人为轻,国家为重,生动地表现了作者强烈的爱国热忱。“莫等闲”二句与“少壮不努力,老大徒伤悲”的意思相同,反映了作者积极进取的精神。这对当时抗击金兵,收复中原的斗争,显然起到了鼓舞斗志的作用;与主张议和,偏安江南,苟延残喘的投降派,形成了鲜明的对照。这既是岳飞的自勉之辞,也是对抗金将士的鼓励和鞭策。

词的下片运转笔端,抒写词人对于民族敌人的深仇大恨,统一祖国的殷切愿望,忠于朝廷即忠于祖国的赤诚之心。

“靖康耻”四句突出全词中心,由于没有雪“靖康”之耻,岳飞发出了心中的恨何时才能消除的感慨。这也是他要“驾长车踏破贺兰山缺”的原因,又把“驾长车踏破贺兰山缺”具体化了。从“驾长车”到“笑谈渴饮匈奴血”都以夸张的手法表达了对凶残敌人的愤恨之情,同时表现了英勇的信心和大无畏的乐观精神。

“壮志”二句把收复山河的宏愿,把艰苦的征战,以一种乐观主义精神表现出来。“待从头”二句,既表达要胜利的信心,也说了对朝廷和皇帝的忠诚。岳飞在这里不直接说凯旋、胜利等,而用了“收拾旧山河”,显得有诗意又形象。一腔忠愤,碧血丹心,肺腑倾出,以此收拾全篇,神完气足,无复毫发遗憾。

这首词代表了岳飞“精忠报国”的英雄之志,词里句中无不透出雄壮之气,显示了作者忧国报国的壮志胸怀。它作为爱国将领的抒怀之作,情调激昂,慷慨壮烈,充分表现了中华民族不甘屈辱,奋发图强,雪耻若渴的神威,从而成为反侵略战争的名篇。


\section{1.6   名家点评}
\label{\detokenize{p01_u6563_u6587/_u5cb3_u98de-_u6ee1_u6c5f_u7ea2_xb7_u6012_u53d1_u51b2_u51a0:id8}}
明代沈际飞:“胆量、意见、文章悉无今古。”(《草堂诗余正集》)

明末清初潘游龙:“胆量意见,俱超今古。”(《古今诗余醉》)

明末清初刘体仁:“词有与古诗同义者,‘潇潇雨歇’,《易水》之歌也。”(《七颂堂词绎》)

清代沈雄:“《满江红》忠愤可见,其不欲等闲白了少年头,可以明其心事。”(《古今词话·词话》上卷)

清代丁绍仪:“至寓议论于协律宫,犹觉激昂慷慨,读之色舞。”(《听秋声馆词话》卷九)

清代陈廷焯:“何等气概!何等志向!千载下读之,凛凛有生气焉。‘莫等闲’二语,当为千古箴铭。”(《白雨斋词话》)


\section{1.7   作者简介}
\label{\detokenize{p01_u6563_u6587/_u5cb3_u98de-_u6ee1_u6c5f_u7ea2_xb7_u6012_u53d1_u51b2_u51a0:id9}}
岳飞(1103—1142),南宋抗金将领。字鹏举,相州汤阴(今属河南)人。官至枢密副使,封武昌郡开国公。以不附和议,被秦桧所陷,被害于大理寺狱。孝宗时追谥武穆,宁宗时追封鄂王,理宗时改谥忠武。《宋史》有传。《直斋书录解题》著录《岳武穆集》十卷,不传。明徐阶编《岳武穆遗文》一卷。《全宋词》录其词三首。


\chapter{1   峻青-海滨仲夏夜}
\label{\detokenize{p01_u6563_u6587/_u5cfb_u9752-_u6d77_u6ee8_u4ef2_u590f_u591c:id1}}\label{\detokenize{p01_u6563_u6587/_u5cfb_u9752-_u6d77_u6ee8_u4ef2_u590f_u591c::doc}}
\begin{sphinxShadowBox}
\sphinxstyletopictitle{目录}
\begin{itemize}
\item {} 
\phantomsection\label{\detokenize{p01_u6563_u6587/_u5cfb_u9752-_u6d77_u6ee8_u4ef2_u590f_u591c:id5}}{\hyperref[\detokenize{p01_u6563_u6587/_u5cfb_u9752-_u6d77_u6ee8_u4ef2_u590f_u591c:id1}]{\sphinxcrossref{1   峻青-海滨仲夏夜}}}
\begin{itemize}
\item {} 
\phantomsection\label{\detokenize{p01_u6563_u6587/_u5cfb_u9752-_u6d77_u6ee8_u4ef2_u590f_u591c:id6}}{\hyperref[\detokenize{p01_u6563_u6587/_u5cfb_u9752-_u6d77_u6ee8_u4ef2_u590f_u591c:id3}]{\sphinxcrossref{1.1   作品原文}}}

\item {} 
\phantomsection\label{\detokenize{p01_u6563_u6587/_u5cfb_u9752-_u6d77_u6ee8_u4ef2_u590f_u591c:id7}}{\hyperref[\detokenize{p01_u6563_u6587/_u5cfb_u9752-_u6d77_u6ee8_u4ef2_u590f_u591c:id4}]{\sphinxcrossref{1.2   注释}}}

\end{itemize}

\end{itemize}
\end{sphinxShadowBox}


\section{1.1   作品原文}
\label{\detokenize{p01_u6563_u6587/_u5cfb_u9752-_u6d77_u6ee8_u4ef2_u590f_u591c:id3}}
夕阳落山不久,西方的天空,还燃烧着一片橘红色的晚霞。大海,也被这霞光染成了红色,而且比天空的景色更要壮观。因为它是活动的,每当一排排波浪涌起的时候,那映照在浪峰上的霞光,又红又亮,就像一片片霍霍①燃烧的火焰,闪烁着,消失了。而后面的一排,又闪烁着,滚动着,涌了过来。

天空的霞光渐渐地淡下去了,深红的颜色变成了绯红②,绯红又变为浅红。最后,当这一切红光都消失了的时候,那突然显得高而远了的天空,呈现出一片肃穆,最早出现的启明星③,在这深蓝色的天幕上闪烁起来了。它是那么大,那么亮,整个广漠④的天幕上只有它在那里放射着令人注目的光辉,活像一盏悬挂在高空的明灯。

夜色加浓,苍空中的“明灯”越来越多了。而城市各处的真的灯火也次第⑤亮了起来,尤其是围绕在海港周围山坡上的那一片灯光,从半空倒映在乌蓝的海面上,随着波浪,晃动着,闪烁着,像一串流动着的珍珠,和那一片片密布在苍穹⑥里的星斗互相辉映,煞⑦是好看。

在这幽美的夜色中,我踏着软绵绵的沙滩,沿着海边,慢慢地向前走去。海水,轻轻地抚摸着细软的沙滩,发出温柔的刷刷声。晚来的海风,清新而又凉爽。我的心里,有着说不出的兴奋和愉快。

夜风轻飘飘地吹拂着,空气中飘荡着一种大海和田禾相混合的香味,柔软的沙滩上还残留着白天太阳炙晒的余温。那些在各个工作岗位上劳动了一天的人们,三三两两地来到了这软绵绵的沙滩上,他们浴着凉爽的海风,望着那缀满了星星的夜空,尽情地说笑,尽情地休憩。愉快的笑声,不时地从这儿那儿飞扬开来,像平静的海面上不断地从这儿那儿涌起的波浪。

我漫步沙滩,徘徊在我的乡亲朋友们中间。

我看到,在那边,在一只底儿朝上反扣在沙滩上的木船旁边,是一群刚从田里收割麦子归来的人们,他们在谈论着今年的收成。今春,雨水足,麦苗长得旺,收成比去年好。眼下,又下了一场透雨,秋后的丰收局面,也大体可以确定下来了。人们为这大好年景所鼓舞着,谈话中也充满了愉快欢乐的笑声。

月亮上来了。

是一轮灿烂的满月。它像一面光辉四射的银盘似的,从那平静的大海里涌了出来。大海里,闪烁着一片鱼鳞似的银波。沙滩上,也突然明亮了起来,一片片坐着、卧着、走着的人影,看得清清楚楚了。啊!海滩上,居然有这么多的人在乘凉。说话声、欢笑声、唱歌声、嬉闹声,响遍了整个的海滩。

月亮升得很高了。它是那么皎洁⑧,那么明亮。

夜已经深了。

沙滩上的人,有的躺在那软绵绵的沙滩上睡着了,有的还在谈笑。凉爽的风轻轻地吹拂着,皎洁的月光照耀着。让这些英雄的人们,在这自由的天幕下,干净的沙滩上,海阔天空地尽情谈笑吧,酣畅地休憩吧。{[}1{]}


\section{1.2   注释}
\label{\detokenize{p01_u6563_u6587/_u5cfb_u9752-_u6d77_u6ee8_u4ef2_u590f_u591c:id4}}
①霍霍:这里是闪动的样子。

②绯红:鲜红。绯,红色。

③启明星:早晨出现于天空东方的金星。

④广漠:广大空旷。

⑤次第:一个挨一个。

⑥苍穹:天空。

⑦煞(shà):这里是“很”的意思。

⑧皎洁:(月亮)明亮洁白。


\chapter{1   曹操-观沧海}
\label{\detokenize{p01_u6563_u6587/_u66f9_u64cd-_u89c2_u6ca7_u6d77:id1}}\label{\detokenize{p01_u6563_u6587/_u66f9_u64cd-_u89c2_u6ca7_u6d77::doc}}
\begin{sphinxShadowBox}
\sphinxstyletopictitle{目录}
\begin{itemize}
\item {} 
\phantomsection\label{\detokenize{p01_u6563_u6587/_u66f9_u64cd-_u89c2_u6ca7_u6d77:id10}}{\hyperref[\detokenize{p01_u6563_u6587/_u66f9_u64cd-_u89c2_u6ca7_u6d77:id1}]{\sphinxcrossref{1   曹操-观沧海}}}
\begin{itemize}
\item {} 
\phantomsection\label{\detokenize{p01_u6563_u6587/_u66f9_u64cd-_u89c2_u6ca7_u6d77:id11}}{\hyperref[\detokenize{p01_u6563_u6587/_u66f9_u64cd-_u89c2_u6ca7_u6d77:id3}]{\sphinxcrossref{1.1   作品原文}}}

\item {} 
\phantomsection\label{\detokenize{p01_u6563_u6587/_u66f9_u64cd-_u89c2_u6ca7_u6d77:id12}}{\hyperref[\detokenize{p01_u6563_u6587/_u66f9_u64cd-_u89c2_u6ca7_u6d77:id4}]{\sphinxcrossref{1.2   词语注释}}}

\item {} 
\phantomsection\label{\detokenize{p01_u6563_u6587/_u66f9_u64cd-_u89c2_u6ca7_u6d77:id13}}{\hyperref[\detokenize{p01_u6563_u6587/_u66f9_u64cd-_u89c2_u6ca7_u6d77:id5}]{\sphinxcrossref{1.3   白话译文}}}

\item {} 
\phantomsection\label{\detokenize{p01_u6563_u6587/_u66f9_u64cd-_u89c2_u6ca7_u6d77:id14}}{\hyperref[\detokenize{p01_u6563_u6587/_u66f9_u64cd-_u89c2_u6ca7_u6d77:id6}]{\sphinxcrossref{1.4   创作背景}}}

\item {} 
\phantomsection\label{\detokenize{p01_u6563_u6587/_u66f9_u64cd-_u89c2_u6ca7_u6d77:id15}}{\hyperref[\detokenize{p01_u6563_u6587/_u66f9_u64cd-_u89c2_u6ca7_u6d77:id7}]{\sphinxcrossref{1.5   作品鉴赏}}}

\item {} 
\phantomsection\label{\detokenize{p01_u6563_u6587/_u66f9_u64cd-_u89c2_u6ca7_u6d77:id16}}{\hyperref[\detokenize{p01_u6563_u6587/_u66f9_u64cd-_u89c2_u6ca7_u6d77:id8}]{\sphinxcrossref{1.6   名家点评}}}

\item {} 
\phantomsection\label{\detokenize{p01_u6563_u6587/_u66f9_u64cd-_u89c2_u6ca7_u6d77:id17}}{\hyperref[\detokenize{p01_u6563_u6587/_u66f9_u64cd-_u89c2_u6ca7_u6d77:id9}]{\sphinxcrossref{1.7   作者简介}}}

\end{itemize}

\end{itemize}
\end{sphinxShadowBox}

《观沧海》是东汉末年诗人曹操创作的一首四言诗。这首诗是曹操在碣石山登山望海时,用饱蘸浪漫主义激情的大笔,所勾勒出的大海吞吐日月、包蕴万千的壮丽景象;描绘了祖国河山的雄伟壮丽,既刻画了高山大海的壮阔,更表达了诗人以景托志,胸怀天下的进取精神。全诗语言质朴,想象丰富,气势磅礴,苍凉悲壮。


\section{1.1   作品原文}
\label{\detokenize{p01_u6563_u6587/_u66f9_u64cd-_u89c2_u6ca7_u6d77:id3}}
观沧海

东临⑴碣⑵石,以观沧⑶海⑷。

水何⑸澹澹⑹,山岛竦峙⑺。

树木丛生,百草丰茂。

秋风萧瑟⑻,洪波⑼涌起。

日月⑽之行,若⑾出其中。

星汉⑿灿烂,若出其里。

幸⒀甚⒁至⒂哉,歌以咏志⒃。


\section{1.2   词语注释}
\label{\detokenize{p01_u6563_u6587/_u66f9_u64cd-_u89c2_u6ca7_u6d77:id4}}
⑴临:登上,有游览的意思。

⑵碣(jié)石:山名。碣石山,河北昌黎碣石山。公元207年秋天,曹操征乌桓得胜回师时经过此地。

⑶沧:通“苍”,青绿色。

⑷海:渤海。

⑸何:多么。

⑹澹澹(dàn):水波摇动的样子。

⑺竦峙(sǒngzhì):高高地挺立。竦,高起。峙,挺立。

⑻萧瑟:树木被秋风吹的声音。

⑼洪波:汹涌澎湃的波浪。

⑽日月:太阳和月亮。

⑾若:如同,好像是。

⑿星汉:银河,天河。

⒀幸:庆幸。

⒁甚:非常。

⒂至:极点。

⒃幸甚至哉,歌以咏志:乐府歌结束用语,不影响全诗内容与感情。意为太值得庆幸了!就用诗歌来表达心志吧。


\section{1.3   白话译文}
\label{\detokenize{p01_u6563_u6587/_u66f9_u64cd-_u89c2_u6ca7_u6d77:id5}}
东行登上碣石山,来观赏那苍茫的海。

海水多么宽阔浩荡,山岛高高地挺立在海边。

树木和百草丛生,十分繁茂。

秋风吹动树木发出悲凉的声音,海中涌着巨大的海浪。

太阳和月亮的运行,好像是从这浩瀚的海洋中发出的。

银河星光灿烂,好像是从这浩瀚的海洋中产生出来的。

我很高兴,就用这首诗歌来表达自己内心的志向。


\section{1.4   创作背景}
\label{\detokenize{p01_u6563_u6587/_u66f9_u64cd-_u89c2_u6ca7_u6d77:id6}}
乌桓是当时东北方的大患,建安十一年(206年),乌桓攻破幽州,俘虏了汉民十余万户。同年,袁绍的儿子袁尚和袁熙又勾结辽西乌桓首领蹋顿,屡次骚扰边境,以致曹操不得不在建安十二年(207年)毅然决定北上征伐乌桓。后来在田畴的指引下,小用计策。大约在这年八月的一次大战中,曹操终于取得了决定性的胜利。这次胜利巩固了曹操的后方,奠定了次年挥戈南下,以期实现统一中国的宏愿。而《观沧海》正是北征乌桓得胜回师经过碣石山时写的。{[}3{]}


\section{1.5   作品鉴赏}
\label{\detokenize{p01_u6563_u6587/_u66f9_u64cd-_u89c2_u6ca7_u6d77:id7}}
这首诗是曹操北征乌桓胜利班师,途中登临碣石山时所作,诗人借大海的雄伟壮丽景象,表达了自己渴望建功立业,统一中原的雄心伟志和宽广的胸襟。从诗的体裁看,这是一首古体诗;从表达方式看,这是一首四言写景诗。

“东临碣石,以观沧海”这两句话点明“观沧海”的位置:诗人登上碣石山顶,居高临海,视野寥廓,大海的壮阔景象尽收眼底。以下十句描写,概由此拓展而来。“观”字起到统领全篇的作用,体现了这首诗意境开阔,气势雄浑的特点。

“水何澹澹,山岛竦峙。树木丛生,百草丰茂。秋风萧瑟,洪波涌起”是实写眼前的景观,神奇而又壮观。“水何澹澹,山岛竦峙”是望海初得的大致印象,有点像绘画的轮廓。在这水波“澹澹”的海上,最先映入眼帘的是那突兀耸立的山岛,它们点缀在平阔的海面上,使大海显得神奇壮观。这两句写出了大海远景的一般轮廓,下面再层层深入描写。“树木丛生,百草丰茂。秋风萧瑟,洪波涌起。”前二句具体写竦峙的山岛:虽然已到秋风萧瑟,草木摇落的季节,但岛上树木繁茂,百草丰美,给人诗意盎然之感。后二句则是对“水何澹澹”一句的进一层描写:定神细看,在秋风萧瑟中的海面竟是洪波巨澜,汹涌起伏。作者面对萧瑟秋风,老骥伏枥,志在千里”的“壮志”胸怀。虽是秋天的典型环境,却无半点萧瑟凄凉的悲秋意绪。作者面对萧瑟秋风,极写大海的辽阔壮美:在秋风萧瑟中,大海汹涌澎湃,浩淼接天;山岛高耸挺拔,草木繁茂,没有丝毫凋衰感伤的情调。这种新的境界,新的格调,正反映了他“老骥伏枥,志在千里”的“烈士”胸襟。

“日月之行,若出其中;星汉灿烂,若出其里”则是虚写,作者运用想象,写出了自己的壮志情怀。前面的描写,将大海的气势和威力凸显在读者面前;在丰富的联想中表现出作者博大的胸怀、开阔的胸襟、宏大的抱负,暗含一种要像大海容纳万物一样把天下纳入自己掌中的胸襟。“幸甚至哉,歌以咏志。”这是合乐时的套语,与诗的内容无关,也说明这是乐府唱过的。

这首诗全篇写景,其中并无直抒胸臆的感慨之词,但是诵读全诗,仍能令人感到它所深深寄托的诗人的情怀。通过诗人对波涛汹涌、吞吐日月的大海的生动描绘,读者仿佛看到了曹操奋发进取,立志统一国家的伟大抱负和壮阔胸襟,触摸到了作为一个诗人、政治家、军事家的曹操,在一种典型环境中思想感情的流动。写景部分准确生动地描绘出海洋的形象,单纯而又饱满,丰富而不琐细,好像一幅粗线条的炭笔画一样。尤其可贵的是,这首诗不仅仅反映了海洋的形象,同时也赋予它以性格。句句写景,又是句句抒情。既表现了大海,也表现了诗人自己。诗人不满足于对海洋做形似的摹拟,而是通过形象,力求表现海洋那种孕大含深、动荡不安的性格。海,本来是没有生命的,然而在诗人笔下却具有了性格。这样才更真实、更深刻地反映了大海的面貌。

这首诗不但写景,而且借景抒情,把眼前的海上景色和自己的雄心壮志很巧妙地融合在一起。这首诗的高潮放在诗的末尾,它的感情非常奔放,思想却很含蓄。不但做到了情景交融,而且做到了情理结合、寓情于景。因为它含蓄,所以更有启发性,更能激发我们的想像,更耐人寻味。过去人们称赞曹操的诗深沉饱满、雄健有力,“如幽燕老将,气韵沉雄”,从这里可以得到印证。全诗的基调为苍凉慷慨的,这也是建安风骨的代表作。全诗语言质朴,想象丰富,气势磅礴,苍凉悲壮。


\section{1.6   名家点评}
\label{\detokenize{p01_u6563_u6587/_u66f9_u64cd-_u89c2_u6ca7_u6d77:id8}}
唐·吴兢《乐府古题要解》:“东临碣石,见沧海之广,日月出入其中。”{[}6{]}

清·张玉榖《古诗赏析》:“此志在容纳,而以海自比也;‘日月’四句,转就日月星汉,凭空想象其包含度量,写沧海,正自写也。”{[}1{]}

清·沈德潜《古诗源》:“有吞吐宇宙气象。”{[}1{]}


\section{1.7   作者简介}
\label{\detokenize{p01_u6563_u6587/_u66f9_u64cd-_u89c2_u6ca7_u6d77:id9}}
曹操(155~220年),字孟德,谯(今安徽亳县)县人,建安时代杰出的政治家、军事家和文学家。建安元年(196年)迎献帝都许(今河南许昌东),挟天子以令诸侯,先后削平吕布等割据势力。官渡之战大破军阀袁绍后,逐渐统一了中国北部。建安十三年(208年),进位为丞相,率军南下,被孙权和刘备的联军击败于赤壁。后封魏王。子曹丕称帝,追尊为武帝。事迹见《三国志》卷一本纪。有集三十卷,已散佚。明人辑有《魏武帝集》,今又有《曹操集》。


\chapter{1   朱自清-春}
\label{\detokenize{p01_u6563_u6587/_u6731_u81ea_u6e05-_u6625:id1}}\label{\detokenize{p01_u6563_u6587/_u6731_u81ea_u6e05-_u6625::doc}}
\begin{sphinxShadowBox}
\sphinxstyletopictitle{目录}
\begin{itemize}
\item {} 
\phantomsection\label{\detokenize{p01_u6563_u6587/_u6731_u81ea_u6e05-_u6625:id4}}{\hyperref[\detokenize{p01_u6563_u6587/_u6731_u81ea_u6e05-_u6625:id1}]{\sphinxcrossref{1   朱自清-春}}}
\begin{itemize}
\item {} 
\phantomsection\label{\detokenize{p01_u6563_u6587/_u6731_u81ea_u6e05-_u6625:id5}}{\hyperref[\detokenize{p01_u6563_u6587/_u6731_u81ea_u6e05-_u6625:id3}]{\sphinxcrossref{1.1   作品原文}}}

\end{itemize}

\end{itemize}
\end{sphinxShadowBox}


\section{1.1   作品原文}
\label{\detokenize{p01_u6563_u6587/_u6731_u81ea_u6e05-_u6625:id3}}
盼望着,盼望着,东风来了,春天的脚步近了。

一切都像刚睡醒的样子,欣欣然张开了眼。山朗润起来了,水涨起来了,太阳的脸红起来了。

小草偷偷地从土里钻出来,嫩嫩的,绿绿的。园子里,田野里,瞧去,一大片一大片满是的。坐着,躺着,打两个滚,踢几脚球,赛几趟跑,捉几回迷藏。风轻悄悄的,草软绵绵的。

桃树、杏树、梨树,你不让我,我不让你,都开满了花赶趟儿。红的像火,粉的像霞,白的像雪。花里带着甜味儿;闭了眼,树上仿佛已经满是桃儿、杏儿、梨儿。花下成千成百的蜜蜂嗡嗡地闹着,大小的蝴蝶飞来飞去。野花遍地是:杂样儿,有名字的,没名字的,散在草丛里,像眼睛,像星星,还眨呀眨的。

“吹面不寒杨柳风”,不错的,像母亲的手抚摸着你。风里带来些新翻的泥土的气息,混着青草味儿,还有各种花的香,都在微微润湿的空气里酝酿。鸟儿将窠巢安在繁花嫩叶当中,高兴起来了,呼朋引伴地卖弄清脆的喉咙,唱出宛转的曲子,与轻风流水应和着。牛背上牧童的短笛,这时候也成天嘹亮地响着。

雨是最寻常的,一下就是三两天。可别恼。看,像牛毛,像花针,像细丝,密密地斜织着,人家屋顶上全笼着一层薄烟。树叶儿却绿得发亮,小草儿也青得逼你的眼。傍晚时候,上灯了,一点点黄晕的光,烘托出一片安静而和平的夜。在乡下,小路上,石桥边,有撑起伞慢慢走着的人,地里还有工作的农民,披着蓑戴着笠。他们的房屋,稀稀疏疏的在雨里静默着。

天上风筝渐渐多了,地上孩子也多了。城里乡下,家家户户,老老小小,也赶趟儿似的,一个个都出来了。舒活舒活筋骨,抖擞抖擞精神,各做各的一份事去。“一年之计在于春”,刚起头儿,有的是工夫,有的是希望。

春天像刚落地的娃娃,从头到脚都是新的,它生长着。

春天像小姑娘,花枝招展的,笑着,走着。

春天像健壮的青年,有铁一般的胳膊和腰脚,领着我们上前去。


\chapter{1   朱自清-梅雨潭的绿}
\label{\detokenize{p01_u6563_u6587/_u6731_u81ea_u6e05-_u6885_u96e8_u6f6d_u7684_u7eff:id1}}\label{\detokenize{p01_u6563_u6587/_u6731_u81ea_u6e05-_u6885_u96e8_u6f6d_u7684_u7eff::doc}}
\begin{sphinxShadowBox}
\sphinxstyletopictitle{目录}
\begin{itemize}
\item {} 
\phantomsection\label{\detokenize{p01_u6563_u6587/_u6731_u81ea_u6e05-_u6885_u96e8_u6f6d_u7684_u7eff:id9}}{\hyperref[\detokenize{p01_u6563_u6587/_u6731_u81ea_u6e05-_u6885_u96e8_u6f6d_u7684_u7eff:id1}]{\sphinxcrossref{1   朱自清-梅雨潭的绿}}}
\begin{itemize}
\item {} 
\phantomsection\label{\detokenize{p01_u6563_u6587/_u6731_u81ea_u6e05-_u6885_u96e8_u6f6d_u7684_u7eff:id10}}{\hyperref[\detokenize{p01_u6563_u6587/_u6731_u81ea_u6e05-_u6885_u96e8_u6f6d_u7684_u7eff:id3}]{\sphinxcrossref{1.1   作品原文}}}

\item {} 
\phantomsection\label{\detokenize{p01_u6563_u6587/_u6731_u81ea_u6e05-_u6885_u96e8_u6f6d_u7684_u7eff:id11}}{\hyperref[\detokenize{p01_u6563_u6587/_u6731_u81ea_u6e05-_u6885_u96e8_u6f6d_u7684_u7eff:id4}]{\sphinxcrossref{1.2   梅雨潭位置}}}

\item {} 
\phantomsection\label{\detokenize{p01_u6563_u6587/_u6731_u81ea_u6e05-_u6885_u96e8_u6f6d_u7684_u7eff:id12}}{\hyperref[\detokenize{p01_u6563_u6587/_u6731_u81ea_u6e05-_u6885_u96e8_u6f6d_u7684_u7eff:id5}]{\sphinxcrossref{1.3   梅雨潭主要景点}}}
\begin{itemize}
\item {} 
\phantomsection\label{\detokenize{p01_u6563_u6587/_u6731_u81ea_u6e05-_u6885_u96e8_u6f6d_u7684_u7eff:id13}}{\hyperref[\detokenize{p01_u6563_u6587/_u6731_u81ea_u6e05-_u6885_u96e8_u6f6d_u7684_u7eff:id6}]{\sphinxcrossref{1.3.1   通元洞}}}

\item {} 
\phantomsection\label{\detokenize{p01_u6563_u6587/_u6731_u81ea_u6e05-_u6885_u96e8_u6f6d_u7684_u7eff:id14}}{\hyperref[\detokenize{p01_u6563_u6587/_u6731_u81ea_u6e05-_u6885_u96e8_u6f6d_u7684_u7eff:id7}]{\sphinxcrossref{1.3.2   飞瀑}}}

\item {} 
\phantomsection\label{\detokenize{p01_u6563_u6587/_u6731_u81ea_u6e05-_u6885_u96e8_u6f6d_u7684_u7eff:id15}}{\hyperref[\detokenize{p01_u6563_u6587/_u6731_u81ea_u6e05-_u6885_u96e8_u6f6d_u7684_u7eff:id8}]{\sphinxcrossref{1.3.3   潭水}}}

\end{itemize}

\end{itemize}

\end{itemize}
\end{sphinxShadowBox}


\section{1.1   作品原文}
\label{\detokenize{p01_u6563_u6587/_u6731_u81ea_u6e05-_u6885_u96e8_u6f6d_u7684_u7eff:id3}}
我第二次到仙岩的时候,我惊诧于梅雨潭的绿了。

梅雨潭是一个瀑布潭。仙岩有三个瀑布,梅雨瀑最低。走到山边,便听见哗哗哗哗的声音;抬起头,镶在两条湿湿的黑边儿里的,一带白而发亮的水便呈现于眼前了。我们先到梅雨亭。梅雨亭正对着那条瀑布;坐在亭边,不必仰头,便可见它的全体了。亭下深深的便是梅雨潭。这个亭踞在突出的一角的岩石上,上下都空空儿的;仿佛一只苍鹰展着翼翅浮在天宇中一般。三面都是山,像半个环儿拥着;人如在井底了。这是一个秋季的薄阴的天气。微微的云在我们顶上流着;岩面与草丛都从润湿中透出几分油油的绿意。而瀑布也似乎分外的响了。那瀑布从上面冲下,仿佛已被扯成大小的几绺;不复是一幅整齐而平滑的布。岩上有许多棱角;瀑流经过时,作急剧的撞击,便飞花碎玉般乱溅着了。那溅着的水花,晶莹而多芒;远望去,像一朵朵小小的白梅,微雨似的纷纷落着。据说,这说是梅雨潭之所以得名了。但我觉得像杨花,格外确切些。轻风起来时,点点随风飘散,那更是杨花了。这时偶然有几点送入我们温暖的怀里,便倏的钻了进去,再也寻它不着。

梅雨潭闪闪的绿色招引着我们;我们开始追捉她那离合的神光了。揪着草,攀着乱石,小心探身下去,又鞠躬过了一个石穹门,便到了汪汪一碧的潭边了。瀑布在襟袖之间;但我的心中已没有瀑布了。我的心随潭水的绿而摇荡。那醉人的绿呀,仿佛一张极大极大的荷叶铺着,满是奇异的绿呀。我想张开两臂抱住她;但这是怎样一个妄想呀。—站在水边,望到那面,居然觉着有些远呢!这平铺着,厚积着的绿,着实可爱。她松松的皱缬着,像少妇拖着的裙幅;她轻轻的摆弄着,像跳动的初恋的处女的心;她滑滑的明亮着,像涂了“明油”一般,有鸡蛋清那样软,那样嫩,令人想着所曾触过的最嫩的皮肤;她又不杂些儿法滓,宛然一块温润的碧玉,只清清的一色—但你却看不透她!我曾见过北京什刹海指地的绿杨,脱不了鹅黄的底子,似乎太淡了。我又曾见过杭州虎跑寺旁高峻而深密的“绿壁”,重叠着无穷国的碧草与绿叶的,那又似乎太浓了。其余呢,西湖的波太明了,秦淮河的又太暗了。可爱的,我将什么来比拟你呢?我怎么比拟得出呢?大约潭是很深的、故能蕴蓄着这样奇异的绿;仿佛蔚蓝的天融了一块在里面似的,这才这般的鲜润呀。—那醉人的绿呀!我若能裁你以为带,我将赠给那轻盈的舞女;她必能临风飘举了。我若能挹你以为眼,我将赠给那善歌的盲妹;她必明眸善睐了。我舍不得你;我怎舍得你呢?我用手拍着你,抚摩着你,如同一个十二三岁的小姑娘。我又掬你入口,便是吻着她了。我送你一个名字,我从此叫你“女儿绿”,好么?

我第二次到仙岩的时候,我不禁惊诧于梅雨潭的绿了。


\section{1.2   梅雨潭位置}
\label{\detokenize{p01_u6563_u6587/_u6731_u81ea_u6e05-_u6885_u96e8_u6f6d_u7684_u7eff:id4}}
梅雨潭,是位于浙江温州市瓯海区仙岩街道的一处名胜,是国家AAA级景区。它东临东海,南北各距瑞安和温州三十多里。

温州一带的山,都属于连绵不断的雁荡山脉。然而仙岩所属的大罗山却远离群山,巍然坐落在温瑞平原上。其山平地拔起,峻崖陡壁,水源充沛,虽方圆不过数十里,却多瀑布潭,而尤集中在西麓瑞安境内的仙岩附近。瀑布潭比较著名的有三个:梅雨潭、雷响潭和龙须潭。其中以梅雨潭最有特色。


\section{1.3   梅雨潭主要景点}
\label{\detokenize{p01_u6563_u6587/_u6731_u81ea_u6e05-_u6885_u96e8_u6f6d_u7684_u7eff:id5}}
远远望去,梅雨潭的瀑布狂奔直下;梅雨亭坐落在瀑布前一块突出的
巨石之上,非常显眼,乍一看去,正如《绿》中写的,“仿佛一只苍鹰展着翼翅浮在天宇中一般”。此亭正对瀑布,原为明代少师张孚敬所建,初名泽润亭,因为安坐其中可观赏瀑布的全貌,作为建筑物又恰到好处地与梅雨潭的自然景色融为一体,故后人改称为“梅雨亭”。


\subsection{1.3.1   通元洞}
\label{\detokenize{p01_u6563_u6587/_u6731_u81ea_u6e05-_u6885_u96e8_u6f6d_u7684_u7eff:id6}}
亭下有洞通潭边,叫做“通元洞”,有个石穹门,旁边刻有“四时梅雨”四个丰满有力的大字。


\subsection{1.3.2   飞瀑}
\label{\detokenize{p01_u6563_u6587/_u6731_u81ea_u6e05-_u6885_u96e8_u6f6d_u7684_u7eff:id7}}
梅雨潭的两侧,双崖对耸,绝不可攀,崖壁上附满绿苔及草木,呈自然的暗绿色,飞瀑自崖合掌处喷吐而出,轰轰作响。

悬崖上岩石颇多棱角,瀑布跌撞而下,似散珠一般注入潭中,轻风吹来,水珠飘飘洒洒,犹如朵朵白梅。


\subsection{1.3.3   潭水}
\label{\detokenize{p01_u6563_u6587/_u6731_u81ea_u6e05-_u6885_u96e8_u6f6d_u7684_u7eff:id8}}
潭水很深,经石穹门下到潭边,水珠、雾气、绿水、悬崖,构成一幅奇妙壮观的图画。清代潘耒在《游仙岩记》中云:“常若梅天细雨,故名梅雨潭。”这个奇观使得在温州执教不到一年的朱自清,竟先后两次来此“追捉她那离合的神光”,与梅雨潭结下了不解之缘。

现在有人在梅雨潭的石穹门旁刻了一个斗大的“绿”字,以此纪念这位著名散文家朱自清的不朽名作《绿》。

那溅着的水花,晶莹而多芒,远望去,像一朵朵小小的白梅,微雨似的纷纷落着.这就是梅雨潭的由来

“踞”字表现出梅雨亭的雄伟 而“浮”字又突出了亭的轻盈
像这样用得生动传神的动词还有“镶” 。“镶”表现出了梅雨亭的—————优美


\chapter{1   朱自清-背影}
\label{\detokenize{p01_u6563_u6587/_u6731_u81ea_u6e05-_u80cc_u5f71:id1}}\label{\detokenize{p01_u6563_u6587/_u6731_u81ea_u6e05-_u80cc_u5f71::doc}}
\begin{sphinxShadowBox}
\sphinxstyletopictitle{目录}
\begin{itemize}
\item {} 
\phantomsection\label{\detokenize{p01_u6563_u6587/_u6731_u81ea_u6e05-_u80cc_u5f71:id14}}{\hyperref[\detokenize{p01_u6563_u6587/_u6731_u81ea_u6e05-_u80cc_u5f71:id1}]{\sphinxcrossref{1   朱自清-背影}}}
\begin{itemize}
\item {} 
\phantomsection\label{\detokenize{p01_u6563_u6587/_u6731_u81ea_u6e05-_u80cc_u5f71:id15}}{\hyperref[\detokenize{p01_u6563_u6587/_u6731_u81ea_u6e05-_u80cc_u5f71:id3}]{\sphinxcrossref{1.1   作品原文}}}

\item {} 
\phantomsection\label{\detokenize{p01_u6563_u6587/_u6731_u81ea_u6e05-_u80cc_u5f71:id16}}{\hyperref[\detokenize{p01_u6563_u6587/_u6731_u81ea_u6e05-_u80cc_u5f71:id4}]{\sphinxcrossref{1.2   词语注释编辑}}}

\item {} 
\phantomsection\label{\detokenize{p01_u6563_u6587/_u6731_u81ea_u6e05-_u80cc_u5f71:id17}}{\hyperref[\detokenize{p01_u6563_u6587/_u6731_u81ea_u6e05-_u80cc_u5f71:id5}]{\sphinxcrossref{1.3   创作背景}}}

\item {} 
\phantomsection\label{\detokenize{p01_u6563_u6587/_u6731_u81ea_u6e05-_u80cc_u5f71:id18}}{\hyperref[\detokenize{p01_u6563_u6587/_u6731_u81ea_u6e05-_u80cc_u5f71:id6}]{\sphinxcrossref{1.4   内容赏析}}}
\begin{itemize}
\item {} 
\phantomsection\label{\detokenize{p01_u6563_u6587/_u6731_u81ea_u6e05-_u80cc_u5f71:id19}}{\hyperref[\detokenize{p01_u6563_u6587/_u6731_u81ea_u6e05-_u80cc_u5f71:id7}]{\sphinxcrossref{1.4.1   第一部分(第一至第三段)}}}

\item {} 
\phantomsection\label{\detokenize{p01_u6563_u6587/_u6731_u81ea_u6e05-_u80cc_u5f71:id20}}{\hyperref[\detokenize{p01_u6563_u6587/_u6731_u81ea_u6e05-_u80cc_u5f71:id8}]{\sphinxcrossref{1.4.2   第二部分(第四至第六段)}}}

\item {} 
\phantomsection\label{\detokenize{p01_u6563_u6587/_u6731_u81ea_u6e05-_u80cc_u5f71:id21}}{\hyperref[\detokenize{p01_u6563_u6587/_u6731_u81ea_u6e05-_u80cc_u5f71:id9}]{\sphinxcrossref{1.4.3   第三部分(最后一段)}}}

\end{itemize}

\item {} 
\phantomsection\label{\detokenize{p01_u6563_u6587/_u6731_u81ea_u6e05-_u80cc_u5f71:id22}}{\hyperref[\detokenize{p01_u6563_u6587/_u6731_u81ea_u6e05-_u80cc_u5f71:id10}]{\sphinxcrossref{1.5   语言特色}}}

\item {} 
\phantomsection\label{\detokenize{p01_u6563_u6587/_u6731_u81ea_u6e05-_u80cc_u5f71:id23}}{\hyperref[\detokenize{p01_u6563_u6587/_u6731_u81ea_u6e05-_u80cc_u5f71:id11}]{\sphinxcrossref{1.6   写作特色}}}

\item {} 
\phantomsection\label{\detokenize{p01_u6563_u6587/_u6731_u81ea_u6e05-_u80cc_u5f71:id24}}{\hyperref[\detokenize{p01_u6563_u6587/_u6731_u81ea_u6e05-_u80cc_u5f71:id12}]{\sphinxcrossref{1.7   行文立意}}}

\item {} 
\phantomsection\label{\detokenize{p01_u6563_u6587/_u6731_u81ea_u6e05-_u80cc_u5f71:id25}}{\hyperref[\detokenize{p01_u6563_u6587/_u6731_u81ea_u6e05-_u80cc_u5f71:id13}]{\sphinxcrossref{1.8   名家点评}}}

\end{itemize}

\end{itemize}
\end{sphinxShadowBox}


\section{1.1   作品原文}
\label{\detokenize{p01_u6563_u6587/_u6731_u81ea_u6e05-_u80cc_u5f71:id3}}
我与父亲不相见已二年余了,我最不能忘记的是他的背影。

那年冬天,祖母死了,父亲的差使1也交卸了,正是祸不单行的日子。我从北京到徐州,打算跟着父亲奔丧2回家。到徐州见着父亲,看见满院狼藉3的东西,又想起祖母,不禁簌簌地流下眼泪。父亲说:“事已如此,不必难过,好在天无绝人之路!”

回家变卖典质4,父亲还了亏空;又借钱办了丧事。这些日子,家中光景5很是惨澹,一半为了丧事,一半为了父亲赋闲6。丧事完毕,父亲要到南京谋事,我也要回北京念书,我们便同行。

到南京时,有朋友约去游逛,勾留7了一日;第二日上午便须渡江到浦口,下午上车北去。父亲因为事忙,本已说定不送我,叫旅馆里一个熟识的茶房8陪我同去。他再三嘱咐茶房,甚是仔细。但他终于不放心,怕茶房不妥帖9;颇踌躇10了一会。其实我那年已二十岁,北京已来往过两三次,是没有什么要紧的了。他踌躇了一会,终于决定还是自己送我去。我再三劝他不必去;他只说:“不要紧,他们去不好!”

我们过了江,进了车站。我买票,他忙着照看行李。行李太多,得向脚夫11行些小费才可过去。他便又忙着和他们讲价钱。我那时真是聪明过分,总觉他说话不大漂亮,非自己插嘴不可,但他终于讲定了价钱;就送我上车。他给我拣定了靠车门的一张椅子;我将他给我做的紫毛大衣铺好座位。他嘱我路上小心,夜里要警醒些,不要受凉。又嘱托茶房好好照应我。我心里暗笑他的迂;他们只认得钱,托他们只是白托!而且我这样大年纪的人,难道还不能料理自己么?我现在想想,我那时真是太聪明了。

我说道:“爸爸,你走吧。”他望车外看了看,说:“我买几个橘子去。你就在此地,不要走动。”我看那边月台的栅栏外有几个卖东西的等着顾客。走到那边月台,须穿过铁道,须跳下去又爬上去。父亲是一个胖子,走过去自然要费事些。我本来要去的,他不肯,只好让他去。我看见他戴着黑布小帽,穿着黑布大马褂12,深青布棉袍,蹒跚13地走到铁道边,慢慢探身下去,尚不大难。可是他穿过铁道,要爬上那边月台,就不容易了。他用两手攀着上面,两脚再向上缩;他肥胖的身子向左微倾,显出努力的样子。这时我看见他的背影,我的泪很快地流下来了。我赶紧拭干了泪。怕他看见,也怕别人看见。我再向外看时,他已抱了朱红的橘子往回走了。过铁道时,他先将橘子散放在地上,自己慢慢爬下,再抱起橘子走。到这边时,我赶紧去搀他。他和我走到车上,将橘子一股脑儿放在我的皮大衣上。于是扑扑衣上的泥土,心里很轻松似的。过一会儿说:“我走了,到那边来信!”我望着他走出去。他走了几步,回过头看见我,说:“进去吧,里边没人。”等他的背影混入来来往往的人里,再找不着了,我便进来坐下,我的眼泪又来了。

近几年来,父亲和我都是东奔西走,家中光景是一日不如一日。他少年出外谋生,独力支持,做了许多大事。哪知老境却如此颓唐!他触目伤怀,自然情不能自已。情郁于中,自然要发之于外;家庭琐屑便往往触他之怒。他待我渐渐不同往日。但最近两年不见,他终于忘却我的不好,只是惦记着我,惦记着他的儿子。我北来后,他写了一信给我,信中说道:“我身体平安,惟膀子疼痛厉害,举箸14提笔,诸多不便,大约大去之期15不远矣。”我读到此处,在晶莹的泪光中,又看见那肥胖的、青布棉袍黑布马褂的背影。唉!我不知何时再能与他相见!{[}2{]}


\section{1.2   词语注释编辑}
\label{\detokenize{p01_u6563_u6587/_u6731_u81ea_u6e05-_u80cc_u5f71:id4}}
1.差(chāi)使:旧时官场中称临时委任的职务,后来泛指职务或官职。

2.奔丧:在外闻亲人去世而归。

3.狼藉(jí):散乱不整齐的样子。亦作“狼籍”。

4.典质:典当,抵押。

5.光景:境况。

6.赋闲:没有职业在家闲居。

7.勾留:逗留。

8.茶房:旧时称在旅馆、茶馆、轮船、火车、剧场等地方从事供应茶水等杂务工作的人。

9.妥帖:恰当,十分合适。

10.踌躇(chóuchú):犹豫。

11.脚夫:旧称搬运工人。

12.马褂:旧时男子穿在长袍外面的对襟短褂。

13.蹒跚(pánshān):走路缓慢、摇摆的样子。也作“盘跚”。

14.箸(zhù):筷子。

15.大去之期:辞世的日子。


\section{1.3   创作背景}
\label{\detokenize{p01_u6563_u6587/_u6731_u81ea_u6e05-_u80cc_u5f71:id5}}
1917年,作者的祖母去世,父亲任徐州烟酒公卖局局长的差事也交卸了。办完丧事,父子同到南京,父亲送作者上火车北去,那年作者20岁。在那特定的场合下,做为父亲对儿子的关怀、体贴、爱护,使儿子极为感动,这印象经久不忘,并且几年之后,想起那背影,父亲的影子出现在“晶莹的泪光中”,使人不能忘怀。1925年,作者有感于世事,便写了此文。{[}4{]}


\section{1.4   内容赏析}
\label{\detokenize{p01_u6563_u6587/_u6731_u81ea_u6e05-_u80cc_u5f71:id6}}
全文可分为三部分。


\subsection{1.4.1   第一部分(第一至第三段)}
\label{\detokenize{p01_u6563_u6587/_u6731_u81ea_u6e05-_u80cc_u5f71:id7}}
交代人物,叙述跟父亲奔丧回家的有关情节,为描写父亲的背影作好铺垫。文章开头一句,落笔点题。“二年余”表明“我”清楚地记得和父亲分离的日子。副词“已”体现出“二年余”在作者的心目中已相当漫长,想望之情,不言而喻。两年多的分离,“我”对父亲的思念是多方面的。其中“最不能忘记的是他的背影”,点出题目。接着,转入对“那年冬天”往事的追述。“祖母死了,父亲的差使也交卸了”,短短两句呈现出人事错迁、谋生艰难之感。“我”从北京到了父亲的住地以后,“看见满院狼藉的东西”,其潦倒之状,又使“我不禁簌簌地流下眼泪”。因为“祸不单行”,所以回家之后,靠“变卖典质”,才还了“亏空”,又“借钱办了丧事”。这里所用的“祸不单行”、“亏空”,“借钱”、“丧事”等词语,一方面是当时情况的真实写照,同时也使后面“家中光景很是惨澹”的形容更有着落。这些叙述和描写,生动地反映了当时世态的灰暗。毛泽东主席在《中国社会各阶级的分析》一文中,曾对当时小资产阶级左翼的情况做过分析,说:“这种人因为他们过去过着好日子,后来逐年下降,负债渐多,渐次过着凄凉的日子,瞻念前途,不寒而栗”。这篇散文所叙述的情节,所抒发的感情,具有一定的典型意义的,也是此文为之感动共鸣的重要原因。


\subsection{1.4.2   第二部分(第四至第六段)}
\label{\detokenize{p01_u6563_u6587/_u6731_u81ea_u6e05-_u80cc_u5f71:id8}}
写父亲为“我”送行的情景,重点描写父亲的背影,表现父子间的真挚感情。丧事完毕,因为父亲要到南京谋事,“我”也要回北京念书,所以父子便一路同行到了南京。到南京之后,因为父亲要谋事,须接交各种关系,忙是可以想见的。所以说定要一个熟识的茶房为“我”送行。“他再三嘱咐茶房,甚是仔细。”这既表现了父亲对“我”的关怀,同时也说明了他对茶房的不放心。父亲当时异地谋生,正须多方奔走,又难以抽身,因此,他“颇踌躇了一会”。“踌躇”,反映了在父亲心中谋事与送子的矛盾。而“终于决定还是自己送我去”,则又表现了父亲毅然将生计暂时搁置,执意为“我”送行的真切感情。“终予”二字,把父亲对“我”无限关切、过分忧虑的心理,表现得淋漓尽致。接下去写的便是车站送行的场面。进了车站以后,父亲“忙着照看行李”,“忙着向脚夫讲价钱”,“送我上车”,“给我拣定靠车门的一张椅子”,“嘱我路上小心”。父亲操劳忙碌的形象展现在面前。可“我”那时由于太年轻,对父亲尚不能完全理解,以至于还在“心里暗笑他的迂”。作者行文至此,一种近乎忏悔的感情不觉流注笔端——“唉,我现在想想,那时真是太聪明了!”自我责备之中,包含着深切的内疚与怀念。在车上坐定之后,父亲又要为“我”去买橘子。但买橘子,“须穿过铁道,须跳下去又爬上去”。父亲又胖,吃力之状可以想见。因此,父亲当时去买橘子的情景,给“我”留下了极为深刻的记忆。当父亲“蹒跚地走到铁道边”时,“我”心中的酸楚是自不待言的。“蹒跚”一词,说明父亲年事已高,步履不稳,过铁路需人扶持。而今,为了“我”却在铁道间蹒跚前往。因而当看见父亲“用两手攀着……努力的样子”的背影时,“我的眼泪”便“很快地流下来了”。这“背影”集中地体现了父亲待“我”的全部感情,这“背影”使“我”念之心酸,感愧交并!望着父亲那吃力的背影,“我”禁不住热泪涌流,但为了“怕他看见”,“我”又“赶紧拭干了泪”,互相体谅的父子真情,表现得维妙维肖。父亲终于买来了橘子。当他走到这边时,“我赶紧去搀他”。这赶紧去搀的动作,表现了“我”又疼,又愧,又欣然若释的复杂心理。疼的是父亲为“我”受累,愧的是父亲为“我”买橘,欣然若释的是父亲终于安全归来。父亲回来之后,“我”虽然没讲一句话,但一腔深情都流露在这“赶紧去搀扶”的动作之中。回到车上,父亲“将橘子一股脑儿放在我的皮大衣上”。“一股脑儿”一词,表现了父亲当时高兴的心情。但父亲高兴的仅仅是为“我”买到了橘子,他的心头是并不轻松的。他谋生无着,而“我”又即将离他远去,兴从何来,所以文章说“心里很轻松似的”,“似的”二字说明父亲并不真正轻松,之所以做出仿佛轻松的样子,是为了宽慰那正心中眷眷的儿子,橘子已经买来,行李也早就安放停当,嘱咐的话也已经说过,看来没什么事了。但父亲并没有马上离去,而是“过一会”才说出告别的话。这“一会”之间,有拳拳的依恋,有惜别的惆怅。父亲终于说,“我走了;到那边来信!”临别的嘱咐,又一次表现了父亲对“我”的牵挂与系念。一直到他走了几步之后,还回过头来说“进去吧,里边没人”,仍关心着“我”的安全。但“我”并没有马上进去,而是“等他的背影……我便进来坐下”。这里的“等”、“再’、“便”三个字,用得极有层次,它们真实地表现了“我”站在车门口,追寻注视着父亲的背影,直到再也看不见时,才进去坐下的那种怅然若失的心情。“我”坐下之后,也许又看到了刚才父亲买来的橘子,一股热辣辣的感情又从心底兜起,“我的眼泪又来了”。


\subsection{1.4.3   第三部分(最后一段)}
\label{\detokenize{p01_u6563_u6587/_u6731_u81ea_u6e05-_u80cc_u5f71:id9}}
写对父亲的想念。作者在描写了父亲的背影之后,予深沉的怀念之中,又想起了父亲的一生。“他少年出外谋生,独力支持,做了许多大事。”父亲是坚强而能干的。虽然如此,家庭生活仍然每况愈下,“光景是一日不如一日”。父亲“触目伤怀”,脾气也变得易于暴怒了。因而,“他待我渐渐不同往日”,但这并非父亲本来的感情,父亲仍旧是父亲。两年不见,又使他在“举箸提笔,诸多不便”的情况下,写了信来,仍旧“惦记着我,惦记着我的儿子”。并在信中写道,“大约大去之期不远矣”,哀矜之中流露出孤寂、颓唐的况昧。它使“我”震悚,使“我”苦痛,使“我”想起父亲待“我”的种种好处,使“我”透过晶莹的泪光,又看见了父亲那凄楚的背影。父亲现在究竟怎样了,“唉!我不知何时再能与他相见。”盼望之中蕴蓄着热切的思念。


\section{1.5   语言特色}
\label{\detokenize{p01_u6563_u6587/_u6731_u81ea_u6e05-_u80cc_u5f71:id10}}
这篇散文的语言非常忠实朴素,又非常典雅文质。这种高度民族化的语言,和文章所表现的民族的精神气质,和文章的完美结构,恰成和谐的统一。没有《背影》语言的简洁明丽、古朴质实,就没有《背影》的一切风采。《背影》的语言还有文白夹杂的特点。例如不说“失业”,而说“赋闲”,最后一节因父亲来信是文言,引用原句,更见真实,也表达了家庭、父亲的困境和苍凉的心情与复杂的感受,同时,文白夹杂的语句,也笼上了一层时代赋予小资产阶级知识分子的特殊语言色彩。


\section{1.6   写作特色}
\label{\detokenize{p01_u6563_u6587/_u6731_u81ea_u6e05-_u80cc_u5f71:id11}}
这篇散文写作上的主要特点是白描。全文集中描写的,是父亲在特定场合下使作者极为感动的那一个背影。作者写了当时父亲的体态、穿着打扮,更主要地写了买橘子时穿过铁路的情形。并不借助于什么修饰、陪衬之类,只把当时的情景再现于眼前。这种白描的文字,读起来清淡质朴,却情真昧浓,蕴藏着一段深情。所谓于平淡中见神奇。其次,作品还运用了侧面烘托的手法。如写儿子“看见他的背影”,“泪很快地流下来了”。又写父亲买桔子回来时,儿子“赶紧去搀他”。这些侧面烘托手法的运用,更加反衬出父亲爱子的动人力量。


\section{1.7   行文立意}
\label{\detokenize{p01_u6563_u6587/_u6731_u81ea_u6e05-_u80cc_u5f71:id12}}
这篇散文的特点是抓住人物形象的特征“背影”命题立意,在叙事中抒发父子深情。“背影”在文章中出现了四次,每次的情况有所不同,而思想感情却是一脉相承。第一次开篇点题“背影”,有一种浓厚的感情气氛笼罩全文。第二次车站送别,作者对父亲的“背影”做了具体的描绘。第三次是父亲和儿子告别后,儿子眼望着父亲的“背影”在人群中消逝,离情别绪,催人泪下。第四次在文章的结尾,儿子读着父亲的来信,在泪光中再次浮现了父亲的“背影”,思念之情不能自已,与文章开头呼应,把父子之间的真挚感情表现得淋漓尽致。


\section{1.8   名家点评}
\label{\detokenize{p01_u6563_u6587/_u6731_u81ea_u6e05-_u80cc_u5f71:id13}}
李广田《最完整的人格》:《背影》论行数不满五十行,论字数不过千五百言,它之所以能够历久传诵而有感人至深的力量者,当然并不是凭藉了甚么宏伟的结构和华瞻的文字,而只是凭了它的老实,凭了其中所表达的真情。这种表面上看起来简单朴素,而实际上却能发生极大的感动力的文章,最可以作为朱先生的代表作品,因为这样的作品,也正好代表了作者之为人。

叶圣陶《文章例话》:“这篇文章通体干净,没有多余的话,没有多余的字眼,即使一个“的”字,一个“了”字,也是必须用才用”。

吴晗《他们走到了它的反面——朱自清颂》:“《背影》虽然只有一千五百字,却历久传诵,有感人至深的力量,这篇短文被选为中学国文教材,在中学生心目中,‘朱自清’三个字已经和《背影》成为不可分割的一体了”。


\chapter{1   朱自清-荷塘月色}
\label{\detokenize{p01_u6563_u6587/_u6731_u81ea_u6e05-_u8377_u5858_u6708_u8272:id1}}\label{\detokenize{p01_u6563_u6587/_u6731_u81ea_u6e05-_u8377_u5858_u6708_u8272::doc}}
\begin{sphinxShadowBox}
\sphinxstyletopictitle{目录}
\begin{itemize}
\item {} 
\phantomsection\label{\detokenize{p01_u6563_u6587/_u6731_u81ea_u6e05-_u8377_u5858_u6708_u8272:id5}}{\hyperref[\detokenize{p01_u6563_u6587/_u6731_u81ea_u6e05-_u8377_u5858_u6708_u8272:id1}]{\sphinxcrossref{1   朱自清-荷塘月色}}}
\begin{itemize}
\item {} 
\phantomsection\label{\detokenize{p01_u6563_u6587/_u6731_u81ea_u6e05-_u8377_u5858_u6708_u8272:id6}}{\hyperref[\detokenize{p01_u6563_u6587/_u6731_u81ea_u6e05-_u8377_u5858_u6708_u8272:id3}]{\sphinxcrossref{1.1   作品原文}}}

\item {} 
\phantomsection\label{\detokenize{p01_u6563_u6587/_u6731_u81ea_u6e05-_u8377_u5858_u6708_u8272:id7}}{\hyperref[\detokenize{p01_u6563_u6587/_u6731_u81ea_u6e05-_u8377_u5858_u6708_u8272:id4}]{\sphinxcrossref{1.2   词语注释}}}

\end{itemize}

\end{itemize}
\end{sphinxShadowBox}


\section{1.1   作品原文}
\label{\detokenize{p01_u6563_u6587/_u6731_u81ea_u6e05-_u8377_u5858_u6708_u8272:id3}}
这几天心里颇不宁静。今晚在院子里坐着乘凉,忽然想起日日走过的荷塘,在这满月的光里,总该另有一番样子吧。月亮渐渐地升高了,墙外马路上孩子们的欢笑,已经听不见了;妻在屋里拍着闰儿⑴,迷迷糊糊地哼着眠歌。我悄悄地披了大衫,带上门出去。

沿着荷塘,是一条曲折的小煤屑路。这是一条幽僻的路;白天也少人走,夜晚更加寂寞。荷塘四面,长着许多树,蓊蓊郁郁⑵的。路的一旁,是些杨柳,和一些不知道名字的树。没有月光的晚上,这路上阴森森的,有些怕人。今晚却很好,虽然月光也还是淡淡的。

路上只我一个人,背着手踱⑶着。这一片天地好像是我的;我也像超出了平常的自己,到了另一个世界里。我爱热闹,也爱冷静;爱群居,也爱独处。像今晚上,一个人在这苍茫的月下,什么都可以想,什么都可以不想,便觉是个自由的人。白天里一定要做的事,一定要说的话,现在都可不理。这是独处的妙处,我且受用这无边的荷香月色好了。

曲曲折折的荷塘上面,弥望⑷的是田田⑸的叶子。叶子出水很高,像亭亭的舞女的裙。层层的叶子中间,零星地点缀着些白花,有袅娜⑹地开着的,有羞涩地打着朵儿的;正如一粒粒的明珠,又如碧天里的星星,又如刚出浴的美人。微风过处,送来缕缕清香,仿佛远处高楼上渺茫的歌声似的。这时候叶子与花也有一丝的颤动,像闪电般,霎时传过荷塘的那边去了。叶子本是肩并肩密密地挨着,这便宛然有了一道凝碧的波痕。叶子底下是脉脉⑺的流水,遮住了,不能见一些颜色;而叶子却更见风致⑻了。

月光如流水一般,静静地泻在这一片叶子和花上。薄薄的青雾浮起在荷塘里。叶子和花仿佛在牛乳中洗过一样;又像笼着轻纱的梦。虽然是满月,天上却有一层淡淡的云,所以不能朗照;但我以为这恰是到了好处——酣眠固不可少,小睡也别有风味的。月光是隔了树照过来的,高处丛生的灌木,落下参差的斑驳的黑影,峭楞楞如鬼一般;弯弯的杨柳的稀疏的倩影,却又像是画在荷叶上。塘中的月色并不均匀;但光与影有着和谐的旋律,如梵婀玲⑼上奏着的名曲。

荷塘的四面,远远近近,高高低低都是树,而杨柳最多。这些树将一片荷塘重重围住;只在小路一旁,漏着几段空隙,像是特为月光留下的。树色一例是阴阴的,乍看像一团烟雾;但杨柳的丰姿⑽,便在烟雾里也辨得出。树梢上隐隐约约的是一带远山,只有些大意罢了。树缝里也漏着一两点路灯光,没精打采的,是渴睡⑾人的眼。这时候最热闹的,要数树上的蝉声与水里的蛙声;但热闹是它们的,我什么也没有。

忽然想起采莲的事情来了。采莲是江南的旧俗,似乎很早就有,而六朝时为盛;从诗歌里可以约略知道。采莲的是少年的女子,她们是荡着小船,唱着艳歌去的。采莲人不用说很多,还有看采莲的人。那是一个热闹的季节,也是一个风流的季节。梁元帝《采莲赋》里说得好:

于是妖童媛女⑿,荡舟心许;鷁首⒀徐回,兼传羽杯⒁;棹⒂将移而藻挂,船欲动而萍开。尔其纤腰束素⒃,迁延顾步⒄;夏始春余,叶嫩花初,恐沾裳而浅笑,畏倾船而敛裾⒅。

可见当时嬉游的光景了。这真是有趣的事,可惜我们现在早已无福消受了。

于是又记起,《西洲曲》里的句子:

采莲南塘秋,莲花过人头;低头弄莲子,莲子清如水。

今晚若有采莲人,这儿的莲花也算得“过人头”了;只不见一些流水的影子,是不行的。这令我到底惦着江南了。——这样想着,猛一抬头,不觉已是自己的门前;轻轻地推门进去,什么声息也没有,妻已睡熟好久了。

一九二七年七月,北京清华园。


\section{1.2   词语注释}
\label{\detokenize{p01_u6563_u6587/_u6731_u81ea_u6e05-_u8377_u5858_u6708_u8272:id4}}
1、闰儿:指朱闰生,朱自清第二子。

2、蓊蓊(wěng)郁郁:树木茂盛的样子。

3、踱(duó):慢慢地走

4、弥望:满眼。弥,满。

5、田田:形容荷叶相连的样子。古乐府《江南曲》中有“莲叶何田田”之句。

6、袅娜(niǎonuó):柔美的样子。

7、脉脉(mò):这里形容水没有声音,好像饱含深情的样子。

8、风致:美的姿态。

9、梵婀玲:violin,小提琴的音译。

10、丰姿:风度,仪态,一般指美好的姿态。也写作“风姿”

11、渴睡:也写作“瞌睡”。

12、妖童媛女:俊俏的少年和美丽的少女。妖,艳丽。媛,女子。

13、鷁首(yìshǒu):船头。古代画鷁鸟于船头。

14、羽杯:古代饮酒用的耳杯。又称羽觞、耳杯。

15、棹(zhào):船桨。

16、纤腰束素:腰如束素,齿如含贝(宋玉《登徒子好色赋》),形容女子腰肢细柔

17、迁延顾步:形容走走退退不住回视自己动作的样子,有顾影自怜之意。

18、敛裾(jū):这里是提着衣襟的意思。裾,衣襟。


\chapter{1   李白-将进酒}
\label{\detokenize{p01_u6563_u6587/_u674e_u767d-_u5c06_u8fdb_u9152:id1}}\label{\detokenize{p01_u6563_u6587/_u674e_u767d-_u5c06_u8fdb_u9152::doc}}
\begin{sphinxShadowBox}
\sphinxstyletopictitle{目录}
\begin{itemize}
\item {} 
\phantomsection\label{\detokenize{p01_u6563_u6587/_u674e_u767d-_u5c06_u8fdb_u9152:id10}}{\hyperref[\detokenize{p01_u6563_u6587/_u674e_u767d-_u5c06_u8fdb_u9152:id1}]{\sphinxcrossref{1   李白-将进酒}}}
\begin{itemize}
\item {} 
\phantomsection\label{\detokenize{p01_u6563_u6587/_u674e_u767d-_u5c06_u8fdb_u9152:id11}}{\hyperref[\detokenize{p01_u6563_u6587/_u674e_u767d-_u5c06_u8fdb_u9152:id3}]{\sphinxcrossref{1.1   作品原文}}}

\item {} 
\phantomsection\label{\detokenize{p01_u6563_u6587/_u674e_u767d-_u5c06_u8fdb_u9152:id12}}{\hyperref[\detokenize{p01_u6563_u6587/_u674e_u767d-_u5c06_u8fdb_u9152:id4}]{\sphinxcrossref{1.2   词句注释}}}

\item {} 
\phantomsection\label{\detokenize{p01_u6563_u6587/_u674e_u767d-_u5c06_u8fdb_u9152:id13}}{\hyperref[\detokenize{p01_u6563_u6587/_u674e_u767d-_u5c06_u8fdb_u9152:id5}]{\sphinxcrossref{1.3   白话译文}}}

\item {} 
\phantomsection\label{\detokenize{p01_u6563_u6587/_u674e_u767d-_u5c06_u8fdb_u9152:id14}}{\hyperref[\detokenize{p01_u6563_u6587/_u674e_u767d-_u5c06_u8fdb_u9152:id6}]{\sphinxcrossref{1.4   创作背景}}}

\item {} 
\phantomsection\label{\detokenize{p01_u6563_u6587/_u674e_u767d-_u5c06_u8fdb_u9152:id15}}{\hyperref[\detokenize{p01_u6563_u6587/_u674e_u767d-_u5c06_u8fdb_u9152:id7}]{\sphinxcrossref{1.5   作品鉴赏}}}

\item {} 
\phantomsection\label{\detokenize{p01_u6563_u6587/_u674e_u767d-_u5c06_u8fdb_u9152:id16}}{\hyperref[\detokenize{p01_u6563_u6587/_u674e_u767d-_u5c06_u8fdb_u9152:id8}]{\sphinxcrossref{1.6   名家点评}}}

\item {} 
\phantomsection\label{\detokenize{p01_u6563_u6587/_u674e_u767d-_u5c06_u8fdb_u9152:id17}}{\hyperref[\detokenize{p01_u6563_u6587/_u674e_u767d-_u5c06_u8fdb_u9152:id9}]{\sphinxcrossref{1.7   作者简介}}}

\end{itemize}

\end{itemize}
\end{sphinxShadowBox}

《将进酒》是唐代大诗人李白沿用乐府古题创作的一首诗。此诗为李白长安放还以后所作,思想内容非常深沉,艺术表现非常成熟,在同题作品中影响最大。诗人豪饮高歌,借酒消愁,抒发了忧愤深广的人生感慨。诗中交织着失望与自信、悲愤与抗争的情怀,体现出强烈的豪纵狂放的个性。全诗情感饱满,无论喜怒哀乐,其奔涌迸发均如江河流泻,不可遏止,且起伏跌宕,变化剧烈;在手法上多用夸张,且往往以巨额数量词进行修饰,既表现出诗人豪迈洒脱的情怀,又使诗作本身显得笔墨酣畅,抒情有力;在结构上大开大阖,充分体现了李白七言歌行的特色。


\section{1.1   作品原文}
\label{\detokenize{p01_u6563_u6587/_u674e_u767d-_u5c06_u8fdb_u9152:id3}}
将进酒⑴

君不见,黄河之水天上来⑵,奔流到海不复回。

君不见,高堂明镜悲白发,朝如青丝暮成雪⑶。

人生得意须尽欢⑷,莫使金樽空对月。

天生我材必有用,千金散尽还复来。

烹羊宰牛且为乐,会须一饮三百杯⑸。

岑夫子,丹丘生⑹,将进酒,杯莫停⑺。

与君歌一曲⑻,请君为我倾耳听⑼。

钟鼓馔玉不足贵⑽,但愿长醉不复醒⑾。

古来圣贤皆寂寞,惟有饮者留其名。

陈王昔时宴平乐,斗酒十千恣欢谑⑿。

主人何为言少钱⒀,径须沽取对君酌⒁。

五花马⒂,千金裘,呼儿将出换美酒,与尔同销万古愁⒃。


\section{1.2   词句注释}
\label{\detokenize{p01_u6563_u6587/_u674e_u767d-_u5c06_u8fdb_u9152:id4}}
⑴将(qiāng)进酒:请饮酒。乐府古题,原是汉乐府短箫铙歌的曲调。《乐府诗集》卷十六引《古今乐录》曰:“汉鼓吹铙歌十八曲,九曰《将进酒》。”《敦煌诗集残卷》三个手抄本此诗均题作“惜罇空”。《文苑英华》卷三三六题作“惜空罇酒”。将,请。

⑵君不见:乐府诗常用作提醒人语。天上来:黄河发源于青海,因那里地势极高,故称。

⑶高堂:房屋的正室厅堂。一说指父母,不合诗意。一作“床头”。青丝:喻柔软的黑发。一作“青云”。成雪:一作“如雪”。

⑷得意:适意高兴的时候。

⑸会须:正应当。

⑹岑夫子:岑勋。丹丘生:元丹丘。二人均为李白的好友。

⑺杯莫停:一作“君莫停”。

⑻与君:给你们,为你们。君,指岑、元二人。

⑼倾耳听:一作“侧耳听”。

⑽钟鼓:富贵人家宴会中奏乐使用的乐器。馔(zhuàn)玉:形容食物如玉一样精美。

⑾不复醒:也有版本为“不用醒”或“不愿醒”。

⑿陈王:指陈思王曹植。平乐(lè):观名。在洛阳西门外,为汉代富豪显贵的娱乐场所。恣:纵情任意。谑(xuè):戏。

⒀言少钱:一作“言钱少”。

⒁径须:干脆,只管。沽:通“酤”,买。

⒂五花马:指名贵的马。一说毛色作五花纹,一说颈上长毛修剪成五瓣。

⒃尔:你。


\section{1.3   白话译文}
\label{\detokenize{p01_u6563_u6587/_u674e_u767d-_u5c06_u8fdb_u9152:id5}}
你可见黄河水从天上流下来,波涛滚滚直奔向东海不回还。

你可见高堂明镜中苍苍白发,早上满头青丝晚上就如白雪。

人生得意时要尽情享受欢乐,不要让金杯空对皎洁的明月。

天造就了我成材必定会有用,即使散尽黄金也还会再得到,

煮羊宰牛姑且尽情享受欢乐,一气喝他三百杯也不要嫌多。

岑夫子啊、丹丘生啊,快喝酒啊,不要停啊。

我为在坐各位朋友高歌一曲,请你们一定要侧耳细细倾听。

钟乐美食这样的富贵不稀罕,我愿永远沉醉酒中不愿清醒。

圣者仁人自古就寂然悄无声,只有那善饮的人才留下美名。

当年陈王曹植平乐观摆酒宴,一斗美酒值万钱他们开怀饮。

主人你为什么说钱已经不多,你尽管端酒来让我陪朋友喝。

管它名贵五花马还是狐皮裘,快叫侍儿拿去统统来换美酒,

与你同饮来消融这万古常愁。李白


\section{1.4   创作背景}
\label{\detokenize{p01_u6563_u6587/_u674e_u767d-_u5c06_u8fdb_u9152:id6}}
关于这首诗的写作时间,说法不一。郁贤皓《李白集》认为此诗约作于开元二十四年(736)前后。黄锡珪《李太白编年诗集目录》系于天宝十一载(752)。一般认为这是李白天宝年间离京后,漫游梁、宋,与友人岑勋、元丹丘相会时所作。

唐玄宗天宝初年,李白由道士吴筠推荐,由唐玄宗招进京,命李白为供奉翰林。不久,因权贵的谗毁,于天宝三载(744年),李白被排挤出京,唐玄宗赐金放还。此后,李白在江淮一带盘桓,思想极度烦闷,又重新踏上了云游祖国山河的漫漫旅途。李白作此诗时距李白被唐玄宗“赐金放还”已有八年之久。这一时期,李白多次与友人岑勋(岑夫子)应邀到嵩山另一好友元丹丘的颍阳山居为客,三人登高饮宴,借酒放歌。诗人在政治上被排挤,受打击,理想不能实现,常常借饮酒来发泄胸中的郁积。人生快事莫若置酒会友,作者又正值“抱用世之才而不遇合”之际,于是满腔不合时宜借酒兴诗情,以抒发满腔不平之气。


\section{1.5   作品鉴赏}
\label{\detokenize{p01_u6563_u6587/_u674e_u767d-_u5c06_u8fdb_u9152:id7}}
这首诗非常形象地表现了李白桀骜不驯的性格:一方面对自己充满自信,孤高自傲;一方面在政治前途出现波折后,又流露出纵情享乐之情。在这首诗里,李白演绎庄子的乐生哲学,表示对富贵、圣贤的藐视。而在豪饮行乐中,实则深含怀才不遇之情。诗人借题发挥,借酒浇愁,抒发自己的愤激情绪。全诗气势豪迈,感情奔放,语言流畅,具有很强的感染力。

时光流逝,如江河入海一去无回;人生苦短,看朝暮间青丝白雪;生命的渺小似乎是个无法挽救的悲剧,能够解忧的惟有金樽美酒。这便是李白式的悲哀:悲而能壮,哀而不伤,极愤慨而又极豪放。表是在感叹人生易老,里则在感叹怀才不遇。诗篇开头是两组排比长句,如挟天风海雨向读者迎面扑来,气势豪迈。“君不见黄河之水天上来,奔流到海不复回”,李白此时在颍阳山,距离黄河不远,登高纵目,所以借黄河来起兴。黄河源远流长,落差极大,如从天而降,一泻千里,东走大海。景象之壮阔,并不是肉眼可见,所以此情此景是李白幻想的,“自道所得”,言语中带有夸张。上句写大河之来,势不可挡;下句写大河之去,势不可回。一涨一消,形成舒卷往复的咏叹味,是短促的单句(如“黄河落天走东海”)所没有的。

紧接着,“君不见高堂明镜悲白发,朝如青丝暮成雪”,恰似一波未平、一波又起。前二句为空间范畴的夸张,这二句则是时间范畴的夸张。悲叹人生短促;而不直接说出自己感伤生命短暂而人一下就会变老,却说“高堂明镜悲白发”,显现出一种对镜自照手抚两鬓、却无可奈何的情态。将人生由青春至衰老的全过程说成“朝”“暮”之事,把本来短暂的说得更短暂,与前两句把本来壮浪的说得更壮浪,是“反向”的夸张。于是,开篇的这组排比长句既有比意——以河水一去不返喻人生易逝,又有反衬作用——以黄河的伟大永恒形出生命的渺小脆弱。这个开端可谓悲感已极,却不堕纤弱,可说是巨人式的感伤,具有惊心动魄的艺术力量,同时也是由长句排比开篇的气势感造成的。这种开篇的手法作者常用,他如“弃我去者,昨日之日不可留;乱我心者,今日之日多烦忧”(《宣城谢朓楼饯别校书叔云》),沈德潜说:“此种格调,太白从心化出”,可见其颇具创造性。此诗两作“君不见”的呼告(一般乐府诗只于篇首或篇末偶一用之),又使诗句感情色彩大大增强。诗有所谓大开大阖者,此可谓大开。

“夫天地者,万物之逆旅也;光阴者,百代之过客也”(《春夜宴从弟桃李园序》),悲感虽然不免,但悲观却非李白性分之所近。在他看来,只要“人生得意”便无所遗憾,当纵情欢乐。五六两句便是一个逆转,由“悲”而翻作“欢“”乐”。从此直到“杯莫停”,诗情渐趋狂放。“人生达命岂暇愁,且饮美酒登高楼”(《梁园吟》),行乐不可无酒,这就入题。但句中没有直写杯中之物,而用“金樽”、“对月”的形象语言来突出隐喻,更将饮酒诗意化了;未直写应该痛饮狂欢,而以“莫使”、“空”的双重否定句式代替直陈,语气更为强调。“人生得意须尽欢”,这似乎是宣扬及时行乐的思想,然而只不过是现象而已。诗人此时郁郁不得志。“凤凰初下紫泥诏,谒帝称觞登御筵”(《玉壶吟》),奉诏进京、皇帝赐宴的时候似乎得意过,然而那不过是一场幻影。再到“弹剑作歌奏苦声,曳裾王门不称情”(《行路难三首》其二),古时冯谖在孟尝君门下作客,觉得孟尝君对自己不够礼遇,开始时经常弹剑而歌,表示要回去。李白希望“平交王侯”的,而在长安,权贵们并不把他当一回事,李白借冯谖的典故比喻自己的处境。这时又似乎并没有得意,有的是失望与愤慨。但并不就此消沉。诗人于是用乐观好强的口吻肯定人生,肯定自我:“天生我材必有用”,这是一个令人击节赞叹的句子。“有用”而“必”,非常自信,简直像是人的价值宣言,而这个人——“我”——是须大写的。于此,从貌似消极的现象中露出了深藏其内的一种怀才不遇而又渴望入世的积极的本质内容来。正是“长风破浪会有时”,实现自我理想的这一天总会来到的,应为这样的未来痛饮高歌,破费又算得了什么。“千金散尽还复来!”这又是一个高度自信的惊人之句,能驱使金钱而不为金钱所使,真足令一切凡夫俗子们咋舌。诗如其人,想诗人“曩者(过去)游维扬,不逾一年(不到一年),散金三十余万”(《上安州裴长史书》),是何等豪举。故此句深蕴在骨子里的豪情,绝非装腔作势者可得其万一。与此气派相当,作者描绘了一场盛筵,那决不是“菜要一碟乎,两碟乎?酒要一壶乎,两壶乎?”而是整头整头地“烹羊宰牛”,不喝上“三百杯”决不甘休。筵宴中展示的痛快气氛,诗句豪壮。

至此,狂放之情趋于高潮,诗的旋律加快。诗人那眼花耳热的醉态跃然纸上,恍然使人如闻其高声劝酒:“岑夫子,丹丘生,将进酒,杯莫停!”几个短句忽然加入,不但使诗歌节奏富于变化,而且写来逼肖席上声口。既是生逢知己,又是酒逢对手,不但“忘形到尔汝”,诗人甚而忘却是在写诗,笔下之诗似乎还原为生活,他还要“与君歌一曲,请君为我倾耳听”。以下八句就是诗中之歌了。这着想奇之又奇,纯系神来之笔。

“钟鼓馔玉”意即富贵生活(富贵人家吃饭时鸣钟列鼎,食物精美如玉),可诗人以为“不足贵”,并放言“但愿长醉不复醒”。诗情至此,便分明由狂放转而为愤激。这里不仅是酒后吐狂言,而且是酒后吐真言了。以“我”天生有用之才,本当位至卿相,飞黄腾达,然而“大道如青天,我独不得出”(《行路难》)。说富贵“不足贵”,乃出于愤慨。以下“古来圣贤皆寂寞”二句亦属愤语。李白曾喟叹“自言管葛竟谁许”,称自己有管仲之才,诸葛亮之智却没人相信,所以说古人“寂寞”,同时表现出自己“寂寞”。因此才情愿醉生梦死长醉不醒了。这里,诗人已是用古人酒杯,浇自己块垒了。说到“唯有饮者留其名”,便举出“陈王”曹植作代表,并化用其《名都篇》“归来宴平乐,美酒斗十千”之句。古来酒徒历历,而偏举“陈王”,这与李白一向自命不凡分不开,他心目中树为榜样的是谢安之类高级人物,而这类人物中,“陈王”与酒联系较多。这样写便有气派,与前文极度自信的口吻一贯。再者,“陈王”曹植于丕、睿两朝备受猜忌,有志难展,亦激起诗人的同情。一提“古来圣贤”,二提“陈王”曹植,满纸不平之气。此诗开始似只涉人生感慨,而不染政治色彩,其实全篇饱含一种深广的忧愤和对自我的信念。诗情所以悲而不伤,悲而能壮,即根源于此。

刚露一点深衷,又回到说酒了,酒兴更高。以下诗情再入狂放,而且愈来愈狂。“主人何为言少钱”,既照应“千金散尽”句,又故作跌宕,引出最后一番豪言壮语:即便千金散尽,也当不惜将出名贵宝物——“五花马”(毛色作五花纹的良马)、“千金裘”来换取美酒,图个一醉方休。这结尾之妙,不仅在于“呼儿”、“与尔”,口气甚大;而且具有一种作者一时可能觉察不到的将宾作主的任诞情态。须知诗人不过是被友招饮的客人,此刻他却高踞一席,气使颐指,提议典裘当马,几令人不知谁是“主人”。浪漫色彩极浓。快人快语,非不拘形迹的豪迈知交断不能出此。诗情至此狂放至极,令人嗟叹咏歌,直欲“手之舞之,足之蹈之”。情犹未已,诗已告终,突然又迸出一句“与尔同销万古愁”,与开篇之“悲”关合,而“万古愁”的含义更其深沉。这“白云从空,随风变灭”的结尾,显见诗人奔涌跌宕的感情激流。通观全篇,真是大起大落,非如椽巨笔不办。

《将进酒》篇幅不算长,却五音繁会,气象不凡。它笔酣墨饱,情极悲愤而作狂放,语极豪纵而又沉着。诗篇具有震动古今的气势与力量,这诚然与夸张手法不无关系,比如诗中屡用巨额数目字(“千金”、“三百杯”、“斗酒十千”、“千金裘”、“万古愁”等等)表现豪迈诗情,同时,又不给人空洞浮夸感,其根源就在于它那充实深厚的内在感情,那潜在酒话底下如波涛汹涌的郁怒情绪。此外,全篇大起大落,诗情忽翕忽张,由悲转乐、转狂放、转愤激、再转狂放、最后结穴于“万古愁”,回应篇首,如大河奔流,有气势,亦有曲折,纵横捭阖,力能扛鼎。其歌中有歌的包孕写法,又有鬼斧神工、“绝去笔墨畦径”之妙,既不是刻意刻画和雕凿能学到的,也不是草率就可达到的境界。通篇以七言为主,而以三、五十言句“破”之,极参差错综之致;诗句以散行为主,又以短小的对仗语点染(如“岑夫子,丹丘生”“五花马,千金裘”),节奏疾徐尽变,奔放而不流易。{[}6{]}{[}7{]}{[}8{]}


\section{1.6   名家点评}
\label{\detokenize{p01_u6563_u6587/_u674e_u767d-_u5c06_u8fdb_u9152:id8}}
《李太白诗集》:严羽评:一结豪情,使人不能句字赏摘。盖他人作诗用笔想,太白但用胸口一喷即是,此其所长。

《唐诗广选》:转折动荡自然(“岑夫子”二句下)。杨升庵曰:太白狂歌。实中玄理,非故为狂语者。

《唐诗解》卷上:此怀才不遇,托于酒以自放也。

《唐诗选脉会通评林》:周珽曰:首以“黄河”起兴,见人之年貌倏改,有如河流莫返。一篇主意全在“人生得意须尽欢,莫使金樽空对月”两句。

《此木轩论诗汇编》:“惟有饮者留其名”,乱道故妙,一学便俗。

《古唐诗合解》:太白此歌豪放极矣。

《而庵说唐诗》:太白此歌,最为豪放,才气干古无双。

《唐诗选胜直解》:此诗妙在自解又以劝人。“主人”是谁?“对君”是谁?骂尽窃高位、守钱虏辈,妙,妙!

《唐诗合选详解》:王翼云曰:此篇用长短句为章法,篇首两个“君不见”领起,亦一局也。

《唐宋诗举要》:吴曰:驱迈淋漓之气(“人生得意”一句下)。吴曰:豪健(末句下)。

《李太白诗醇》:一起奇想,亦自天外来。


\section{1.7   作者简介}
\label{\detokenize{p01_u6563_u6587/_u674e_u767d-_u5c06_u8fdb_u9152:id9}}
李白(701~762),字太白,号青莲居士。是屈原之后最具个性特色、最伟大的浪漫主义诗人。有“诗仙”之美誉,与杜甫并称“李杜”。其诗以抒情为主,表现出蔑视权贵的傲岸精神,对人民疾苦表示同情,又善于描绘自然景色,表达对祖国山河的热爱。诗风雄奇豪放,想象丰富,语言流转自然,音律和谐多变,善于从民间文艺和神话传说中吸取营养和素材,构成其特有的瑰玮绚烂的色彩,达到盛唐诗歌艺术的巅峰。存世诗文千余篇,有《李太白集》30卷。


\chapter{1   李白-梦游天姥吟留别}
\label{\detokenize{p01_u6563_u6587/_u674e_u767d-_u68a6_u6e38_u5929_u59e5_u541f_u7559_u522b:id1}}\label{\detokenize{p01_u6563_u6587/_u674e_u767d-_u68a6_u6e38_u5929_u59e5_u541f_u7559_u522b::doc}}
\begin{sphinxShadowBox}
\sphinxstyletopictitle{目录}
\begin{itemize}
\item {} 
\phantomsection\label{\detokenize{p01_u6563_u6587/_u674e_u767d-_u68a6_u6e38_u5929_u59e5_u541f_u7559_u522b:id10}}{\hyperref[\detokenize{p01_u6563_u6587/_u674e_u767d-_u68a6_u6e38_u5929_u59e5_u541f_u7559_u522b:id1}]{\sphinxcrossref{1   李白-梦游天姥吟留别}}}
\begin{itemize}
\item {} 
\phantomsection\label{\detokenize{p01_u6563_u6587/_u674e_u767d-_u68a6_u6e38_u5929_u59e5_u541f_u7559_u522b:id11}}{\hyperref[\detokenize{p01_u6563_u6587/_u674e_u767d-_u68a6_u6e38_u5929_u59e5_u541f_u7559_u522b:id3}]{\sphinxcrossref{1.1   作品原文}}}

\item {} 
\phantomsection\label{\detokenize{p01_u6563_u6587/_u674e_u767d-_u68a6_u6e38_u5929_u59e5_u541f_u7559_u522b:id12}}{\hyperref[\detokenize{p01_u6563_u6587/_u674e_u767d-_u68a6_u6e38_u5929_u59e5_u541f_u7559_u522b:id4}]{\sphinxcrossref{1.2   词句注释}}}

\item {} 
\phantomsection\label{\detokenize{p01_u6563_u6587/_u674e_u767d-_u68a6_u6e38_u5929_u59e5_u541f_u7559_u522b:id13}}{\hyperref[\detokenize{p01_u6563_u6587/_u674e_u767d-_u68a6_u6e38_u5929_u59e5_u541f_u7559_u522b:id5}]{\sphinxcrossref{1.3   白话译文}}}

\item {} 
\phantomsection\label{\detokenize{p01_u6563_u6587/_u674e_u767d-_u68a6_u6e38_u5929_u59e5_u541f_u7559_u522b:id14}}{\hyperref[\detokenize{p01_u6563_u6587/_u674e_u767d-_u68a6_u6e38_u5929_u59e5_u541f_u7559_u522b:id6}]{\sphinxcrossref{1.4   创作背景}}}

\item {} 
\phantomsection\label{\detokenize{p01_u6563_u6587/_u674e_u767d-_u68a6_u6e38_u5929_u59e5_u541f_u7559_u522b:id15}}{\hyperref[\detokenize{p01_u6563_u6587/_u674e_u767d-_u68a6_u6e38_u5929_u59e5_u541f_u7559_u522b:id7}]{\sphinxcrossref{1.5   作品鉴赏}}}

\item {} 
\phantomsection\label{\detokenize{p01_u6563_u6587/_u674e_u767d-_u68a6_u6e38_u5929_u59e5_u541f_u7559_u522b:id16}}{\hyperref[\detokenize{p01_u6563_u6587/_u674e_u767d-_u68a6_u6e38_u5929_u59e5_u541f_u7559_u522b:id8}]{\sphinxcrossref{1.6   名家点评}}}

\item {} 
\phantomsection\label{\detokenize{p01_u6563_u6587/_u674e_u767d-_u68a6_u6e38_u5929_u59e5_u541f_u7559_u522b:id17}}{\hyperref[\detokenize{p01_u6563_u6587/_u674e_u767d-_u68a6_u6e38_u5929_u59e5_u541f_u7559_u522b:id9}]{\sphinxcrossref{1.7   作者简介}}}

\end{itemize}

\end{itemize}
\end{sphinxShadowBox}

《梦游天姥吟留别》是唐代大诗人李白的诗作。这是一首记梦诗,也是一首游仙诗。此诗以记梦为由,抒写了对光明、自由的渴求,对黑暗现实的不满,表现了蔑视权贵、不卑不屈的叛逆精神。诗人运用丰富奇特的想象和大胆夸张的手法,组成一幅亦虚亦实、亦幻亦真的梦游图。全诗构思精密,意境雄伟,内容丰富曲折,形象辉煌流丽,感慨深沉激烈,富有浪漫主义色彩。其在形式上杂言相间,兼用骚体,不受律束,笔随兴至,体制解放,堪称绝世名作。


\section{1.1   作品原文}
\label{\detokenize{p01_u6563_u6587/_u674e_u767d-_u68a6_u6e38_u5929_u59e5_u541f_u7559_u522b:id3}}
梦游天姥吟留别1

海客谈瀛洲,烟涛微茫信难求2。

越人语天姥3,云霞明灭或可睹4。

天姥连天向天横5,势拔五岳掩赤城6。

天台四万八千丈7,对此欲倒东南倾8。

我欲因之梦吴越9,一夜飞度镜湖月10。

湖月照我影,送我至剡溪11。

谢公宿处今尚在12,渌水荡漾清猿啼13。

脚著谢公屐14,身登青云梯15。

半壁见海日16,空中闻天鸡17。

千岩万转路不定,迷花倚石忽已暝18。

熊咆龙吟殷岩泉19,栗深林兮惊层巅20。

云青青兮欲雨21,水澹澹兮生烟22。

列缺霹雳23,丘峦崩摧。

洞天石扉24,訇然中开25。

青冥浩荡不见底26,日月照耀金银台27。

霓为衣兮风为马28,云之君兮纷纷而来下29。

虎鼓瑟兮鸾回车30,仙之人兮列如麻。

忽魂悸以魄动31,恍惊起而长嗟32。

惟觉时之枕席33,失向来之烟霞34。

世间行乐亦如此,古来万事东流水35。

别君去兮何时还?

且放白鹿青崖间36,须行即骑访名山37。

安能摧眉折腰事权贵38,使我不得开心颜!


\section{1.2   词句注释}
\label{\detokenize{p01_u6563_u6587/_u674e_u767d-_u68a6_u6e38_u5929_u59e5_u541f_u7559_u522b:id4}}
1.天姥山:在浙江新昌东面。传说登山的人能听到仙人天姥唱歌的声音,山因此得名。

2.瀛洲:古代传说中的东海三座仙山之一(另两座叫蓬莱和方丈)。烟涛:波涛渺茫,远看像烟雾笼罩的样子。微茫:景象模糊不清。信:确实,实在。

3.越人:指浙江一带的人。

4.明灭:忽明忽暗。

5.向天横:直插天空。横,直插。

6.”势拔“句:山势高过五岳,遮掩了赤城。拔,超出。五岳,指东岳泰山、西岳华山、中岳嵩山、北岳恒山、南岳衡山。赤城,山名,在浙江天台西北。

7.天台(tāi):山名,在浙江天台北部。

8.”对此“句:对着天姥这座山,天台山就好像要倒向它的东南一样。意思是天台山和天姥山相比,显得低多了。

9.因:依据。之:指代前边越人的话。

10.度:一作“渡”{[}2{]}。镜湖:又名鉴湖,在浙江绍兴南面。

11.剡(shàn)溪:水名,在浙江嵊州南面。

12.谢公:指南朝诗人谢灵运。谢灵运喜欢游山。游天姥山时,他曾在剡溪这个地方住宿。

13.渌(lù):清。清:这里是凄清的意思。

14.谢公屐(jī):谢灵运穿的那种木屐。《南史·谢灵运传》记载:谢灵运游山,必到幽深高峻的地方;他备有一种特制的木屐,屐底装有活动的齿,上山时去掉前齿,下山时去掉后齿。木屐,以木板作底,上面有带子,形状像拖鞋。

15.青云梯:指直上云霄的山路。

16.半壁见海日:上到半山腰就看到从海上升起的太阳。

17.天鸡:古代传说,东南有桃都山,山上有棵大树叫桃都,树枝绵延三千里,树上栖有天鸡,每当太阳初升,照到这棵树上,天鸡就叫起来,天下的鸡也都跟着它叫。

18.”迷花“句:迷恋着花,依靠着石,不觉天色已经很晚了。暝(míng),日落,天黑。

19.”熊咆“句:熊在怒吼,龙在长鸣,岩中的泉水在震响。殷(yǐn),这里用作动词,震响。

20.”栗深林“句:使深林战栗,使层巅震惊。栗、惊,使动用法。

21.青青:黑沉沉的。

22.澹澹:波浪起伏的样子。

23.列缺:指闪电。

24.洞天:仙人居住的洞府。扉:门扇。一作“扇”。

25.訇(hōng)然:形容声音很大。

26.青冥:指天空。浩荡:广阔远大的样子。

27.金银台:金银铸成的宫阙,指神仙居住的地方。

28.风:一作“凤”。

29.云之君:云里的神仙。

30.鸾回车:鸾鸟驾着车。鸾,传说中的如凤凰一类的神鸟。回,旋转,运转。

31.魂悸:心跳。

32.恍:恍然,猛然。

33.觉时:醒时。

34.失向来之烟霞:刚才梦中所见的烟雾云霞消失了。向来,原来。烟霞,指前面所写的仙境。

35.东流水:像东流的水一样一去不复返。

36.白鹿:传说神仙或隐士多骑白鹿。青崖:青山。

37.须:等待。

38.摧眉折腰:低头弯腰。摧眉,即低眉。


\section{1.3   白话译文}
\label{\detokenize{p01_u6563_u6587/_u674e_u767d-_u68a6_u6e38_u5929_u59e5_u541f_u7559_u522b:id5}}
海外来客们谈起瀛洲,烟波渺茫实在难以寻求。

越中来人说起天姥山,在云雾忽明忽暗间有人可以看见。

天姥山仿佛连接着天遮断了天空,山势高峻超过五岳,遮掩过赤城山。

天台山虽高一万八千丈,面对着它好像要向东南倾斜拜倒一样。

我根据越人说的话梦游到吴越,一天夜晚飞渡过明月映照下的镜湖。

镜湖上的月光照着我的影子,一直伴随我到了剡溪。

谢灵运住的地方如今还在,清澈的湖水荡漾,猿猴清啼。

我脚上穿着谢公当年特制的木鞋,攀登直上云霄的山路。

上到半山腰就看见了从海上升起的太阳,在半空中传来天鸡报晓的叫声。

无数山岩重叠,道路盘旋弯曲,方向不定,迷恋着花,依倚着石头,不觉天色已晚。

熊在怒吼,龙在长鸣,岩中的泉水在震响,使森林战栗,使山峰惊颤。

云层黑沉沉的,像是要下雨,水波动荡生起了烟雾。

电光闪闪,雷声轰鸣,山峰好像要被崩塌似的。

仙府的石门,“訇”的一声从中间打开。

洞中蔚蓝的天空广阔无际,看不到尽头,日月照耀着金银做的宫阙。

用彩虹做衣裳,将风作为马来乘,云中的神仙们纷纷下来。

老虎弹奏着琴瑟,鸾鸟驾着车,仙人们成群结队密密如麻。

忽然魂魄惊动,我猛然惊醒,不禁长声叹息。

醒来时只有身边的枕席,刚才梦中所见的烟雾云霞全都消失了。

人世间的欢乐也是像梦中的幻境这样,自古以来万事都像东流的水一样一去不复返。

告别诸位朋友远去东鲁啊,什么时候才能回来?

暂且把白鹿放牧在青崖间,等到要远行时就骑上它访名山。

岂能卑躬屈膝去侍奉权贵,使我不能有舒心畅意的笑颜!


\section{1.4   创作背景}
\label{\detokenize{p01_u6563_u6587/_u674e_u767d-_u68a6_u6e38_u5929_u59e5_u541f_u7559_u522b:id6}}
此诗作于李白出翰林之后,其作年一说天宝四载(745年),一说天宝五载(746年)。唐玄宗天宝三载(744年),李白在长安受到权贵的排挤,被放出京,返回东鲁(在今山东)家园。之后再度踏上漫游的旅途。这首描绘梦中游历天姥山的诗,大约作于李白即将离开东鲁南游吴越之时。

李白早年就有济世的抱负,但不屑于经由科举登上仕途。因此他漫游全国各地,结交名流,以此广造声誉。唐玄宗天宝元年(742年),李白的朋友道士吴筠向玄宗推荐李白,玄宗于是召他到长安来。李白对这次长安之行抱有很大的希望,在给妻子的留别诗《别内赴征》中写道:“归时倘佩黄金印,莫见苏秦不下机。”李白初到长安,也曾有过短暂的得意,但他一身傲骨,不肯与权贵同流合污,又因得罪了权贵,及翰林院同事进谗言,连玄宗也对他不满。他在长安仅住了一年多,就被唐玄宗赐金放还,他那由布衣而卿相的梦幻从此完全破灭。这是李白政治上的一次大失败。离开长安后,他曾与杜甫、高适游梁、宋、齐、鲁,又在东鲁家中居住过一个时期。这时东鲁的家已颇具规模,尽可在家中怡情养性,以度时光。可是李白没有这么做。他有一个不安定的灵魂,他有更高更远的追求,于是离别东鲁家园,又一次踏上漫游的旅途。这首诗就是他告别东鲁朋友时所作,所以又题作“梦游天姥山别东鲁诸公”。{[}4{]}{[}5{]}{[}6{]}{[}7{]}{[}8{]}


\section{1.5   作品鉴赏}
\label{\detokenize{p01_u6563_u6587/_u674e_u767d-_u68a6_u6e38_u5929_u59e5_u541f_u7559_u522b:id7}}
这是一首记梦诗,也是一首游仙诗。意境雄伟,变化惝恍莫测,缤纷多采的艺术形象,新奇的表现手法,向来为人传诵,被视为李白的代表作之一。

这首诗的思想内容相当复杂。李白从离开长安后,因政治上遭受挫折,精神上的苦闷愤怨郁结于怀。在现实社会中找不到出路,只有向虚幻的神仙世界和远离尘俗的山林去寻求解脱。这种遁世思想看似消沉,却不能一笔抹杀,它在一定程度上表现了李白在精神上摆脱了尘俗的桎梏。而这才导致他在诗的最后发出“安能摧眉折腰事权贵,使我不得开心颜”那样激越的呼声。这种坚决不妥协的精神和强烈的反抗情绪正是这首诗的基调。

李白一生徜徉山水之间,热爱山水,达到梦寐以求的境地。此诗所描写的梦游,也许并非完全虚托,但无论是否虚托,梦游就更适于超脱现实,更便于发挥他的想象和夸张的才能了。

“海客谈瀛洲,烟涛微茫信难求;越人语天姥,云霓明灭或可睹。”诗一开始先说古代传说中的海外仙境──瀛洲,虚无缥缈,不可寻求;而现实中的天姥山在浮云彩霓中时隐时现,真是胜似仙境。以虚衬实,突出了天姥胜景,暗蕴着诗人对天姥山的向往,写得富有神奇色彩,引人入胜。

天姥山临近剡溪,传说登山的人听到过仙人天姥的歌唱,因此得名。天姥山与天台山相对,峰峦峭峙,仰望如在天表,冥茫如堕仙境,容易引起游者想入非非的幻觉。浙东山水是李白青年时代就向往的地方,初出川时曾说“此行不为鲈鱼鲙,自爱名山入剡中”。入翰林前曾不止一次往游,他对这里的山水不但非常热爱,也是非常熟悉的。

天姥山号称奇绝,是越东灵秀之地。但比之其他崇山峻岭如我国的五大名山──五岳,在人们心目中的地位仍有小巫见大巫之别。可是李白却在诗中夸说它“势拔五岳掩赤城”,比五岳还更挺拔。有名的天台山则倾斜着如拜倒在天姥的足下一样。这个天姥山,被写得耸立天外,直插云霄,巍巍然非同凡比。这座梦中的天姥山,应该说是李白平生所经历的奇山峻岭的幻影,它是现实中的天姥山在李白笔下夸大了的影子。

接着展现出的是一幅一幅瑰丽变幻的奇景:天姥山隐于云霓明灭之中,引起了诗人探求的想望。诗人进入了梦幻之中,仿佛在月夜清光的照射下,他飞渡过明镜一样的镜湖。明月把他的影子映照在镜湖之上,又送他降落在谢灵运当年曾经歇宿过的地方。他穿上谢灵运当年特制的木屐,登上谢公当年曾经攀登过的石径──青云梯。只见:“半壁见海日,空中闻天鸡。千岩万转路不定,迷花倚石忽已暝。熊咆龙吟殷岩泉,栗深林兮惊层巅。云青青兮欲雨,水澹澹兮生烟。”继飞渡而写山中所见,石径盘旋,深山中光线幽暗,看到海日升空,天鸡高唱,这本是一片曙色;却又于山花迷人、倚石暂憩之中,忽觉暮色降临,旦暮之变何其倏忽。暮色中熊咆龙吟,震响于山谷之间,深林为之战栗,层巅为之惊动。不止有生命的熊与龙以吟、咆表示情感,就连层巅、深林也能战栗、惊动,烟、水、青云都满含阴郁,与诗人的情感,协成一体,形成统一的氛围。前面是浪漫主义地描写天姥山,既高且奇;这里又是浪漫主义地抒情,既深且远。这奇异的境界,已经使人够惊骇的了,但诗人并未到此止步,而诗境却由奇异而转入荒唐,全诗也更进入高潮。在令人惊悚不已的幽深暮色之中,霎时间“丘峦崩摧”,一个神仙世界“訇然中开”,“青冥浩荡不见底,日月照耀金银台。霓为衣兮风为马,云之君兮纷纷而来下。”洞天福地,于此出现。“云之君”披彩虹为衣,驱长风为马,虎为之鼓瑟,鸾为之驾车,皆受命于诗人之笔,奔赴仙山的盛会来了。这是多么盛大而热烈的场面。“仙之人兮列如麻”!群仙好像列队迎接诗人的到来。金台、银台与日月交相辉映,景色壮丽,异彩缤纷,何等的惊心眩目,光耀夺人!仙山的盛会正是人世间生活的反映。这里除了有他长期漫游经历过的万壑千山的印象、古代传说、屈原诗歌的启发与影响,也有长安三年宫廷生活的迹印,这一切通过浪漫主义的非凡想象凝聚在一起,才有这般辉煌灿烂、气象万千的描绘。

这首诗写梦游奇境,不同于一般游仙诗,它感慨深沉,抗议激烈,并非真正依托于虚幻之中,而是在神仙世界虚无飘渺的描述中,依然着眼于现实。神游天上仙境,而心觉“世间行乐亦如此”。

仙境倏忽消失,梦境旋亦破灭,诗人终于在惊悸中返回现实。梦境破灭后,人,不是随心所欲地轻飘飘地在梦幻中翱翔了,而是沉甸甸地躺在枕席之上。“古来万事东流水”,其中包含着诗人对人生的几多失意和深沉的感慨。此时此刻诗人感到最能抚慰心灵的是“且放白鹿青崖间,须行即骑访名山”。徜徉山水的乐趣,才是最快意的,也就是在《春夜宴从弟桃花园序》中所说:“古人秉烛夜游,良有以也。”本来诗意到此似乎已尽,可是最后却愤愤然加添了两句“安能摧眉折腰事权贵,使我不得开心颜!”一吐长安三年的郁闷之气。天外飞来之笔,点亮了全诗的主题:对于名山仙境的向往,是出之于对权贵的抗争,它唱出封建社会中多少怀才不遇的人的心声。在等级森严的封建社会中,多少人屈身权贵,多少人埋没无闻!唐朝比之其他朝代是比较开明的,较为重视人才,但也只是比较而言。人才在当时仍然摆脱不了“臣妾气态间”的屈辱地位。“折腰”一词出之于东晋的陶渊明,他由于不愿忍辱而赋《归去来兮辞》。李白虽然受帝王优宠,也不过是个词臣,在宫廷中所受到的屈辱,大约可以从这两句诗中得到一些消息。封建君主把自己称“天子”,君临天下,把自己升高到至高无上的地位,却抹煞了一切人的尊严。李白在这里所表示的决绝态度,是向封建统治者所投过去的一瞥蔑视。在封建社会,敢于这样想、敢于这样说的人并不多。李白说了,也做了,这是他异乎常人的伟大之处。

这首诗的内容丰富、曲折、奇谲、多变,它的形象辉煌流丽,缤纷多彩,构成了全诗的浪漫主义华赡情调。它的主观意图本来在于宣扬“古来万事东流水”这样颇有消极意味的思想,可是它的格调却是昂扬振奋的,潇洒出尘的,有一种不卑不屈的气概流贯其间,并无消沉之感。{[}7{]}{[}8{]}


\section{1.6   名家点评}
\label{\detokenize{p01_u6563_u6587/_u674e_u767d-_u68a6_u6e38_u5929_u59e5_u541f_u7559_u522b:id8}}
明代高棅《唐诗品汇》:范云:瀛洲难求而不必求,天姥可睹而实未睹,故欲因梦而睹之耳(“海客”四句下)。甚显(“半壁”二句下)。甚晦(“千岩万转”二句下)。又甚显(“洞天”四句下)。又甚晦(“霓为衣兮”四句下)。范云:“梦吴越”以下,梦之源也;次诸节,梦之波澜。其间显而晦,晦而显,至“失向来之烟霞”极而与人接矣,非太白之胸次、笔力,亦不能发此。“枕席”“烟霞”二句最有力。结语平衍,亦文势之当如此也。

明代桂天祥《批点唐诗正声》:《梦游天姥吟》胸次皆烟霞云石,无分毫尘浊,别是一副言语,故特为难到。

明代郭濬《增订评注唐诗正声》:郭云:恍恍惚惚,奇奇幻幻,非满肚皮烟霞,决挥洒不出。

明代周敬、周珽《唐诗选脉会通评林》:周珽曰:出于千丝铁网之思,运以百色流苏之局,忽而飞步凌顶,忽而烟云自舒。想其拈笔时,神魂毛发尽脱于毫楮而不自知,其神耶!吴山民曰:“天台四万八千丈”,形容语,“白发三千丈”同意,有形容天姥高意。“千岩万转”句,语有概括。下三句,梦中危景。又八句,梦中奇景。又四句,梦中所遇。“唯觉时之枕席”二语,篇中神句,结上启下。“世间行乐”二句,因梦生意。结超。

清代朱之荆《增订唐诗摘钞》:“忽魂”四句,束上生下,笔意最紧。万斛之舟,收于一柁(末二句下)。

清代沈德潜《唐诗别裁》:“飞渡镜湖月”以下,皆言梦中所历。一路离奇灭没,恍恍惚惚,是梦境,是仙境(“列缺霹雳”十二句下)。托言梦游,穷形尽相以极“洞天”之奇幻;至酲后,顿失烟霞矣。知世间行乐,亦同一梦,安能于梦中屈身权贵乎?吾当别去,遍游名山,以终天年也。诗境虽奇,脉理极细。

清高宗敕编《唐宋诗醇》:七古歌行,本出楚骚、乐府。至于太白,然后穷极笔力,优入圣域。昔人谓其“以气为主,以自然为宗,以俊逸高畅为贵,咏之使人飘飘欲仙”,而尤推其《天姥吟》《远别离》等篇,以为虽子美不能道。盖其才横绝一世,故兴会标举,非学可及,正不必执此谓子美不能及也。此篇夭矫离奇,不可方物,然因语而梦,因梦而悟,因悟而别,节次柑生,丝毫不乱;若中间梦境迷离,不过词意伟怪耳。胡应麟以为“无首无尾,窈冥昏默”,是真不可以说梦也特谓非其才力,学之立见踬踣,则诚然耳。

清代翁方纲《赵秋谷所传声调谱》:方纲按:《扶风豪士歌》《梦游天姥吟》二篇,虽句法、音节极其变化,然实皆自然入拍,非任意参错也。秋谷于《豪士》篇但评其神变,于《天姥》篇则第云“观此知转韵元无定格”,正恐难以示后学耳。

清代宋宗元《网师园唐诗笺》:纵横变化,离奇光怪,以奇笔写梦境,吐句皆仙,着纸谷飞(“列缺霹雳”十句下)。砉然收勒,通体宗主攸在,线索都灵(“世间行乐”二句下)。

清代方东树《昭昧詹言》:陪起,令人迷。“我欲”以下正叙梦,愈唱愈高,愈出愈奇“失向”句,收住。“世间”二句,入作意,因梦游推开,见世事皆成虚幻也;不如此,则作诗之旨无归宿。留别意,只末后一点。韩《记梦》之本。

清代延君寿《老生常谈》:《梦游天姥吟留别》诗,奇离惝恍,似无门径可寻。细玩之,起首入梦不突,后幅出梦不竭,极恣肆幻化之中,又极经营惨淡之苦,若只貌其右句字面,则失之远矣。一起淡淡引入,至“我欲因之梦吴越”句,乘势即入,使笔如风,所谓缓则按辔徐行,急则短兵相接也。“湖月照我影”八句,他人捉笔可云已尽能事矣,岂料后边尚有许多奇奇怪怪。“千岩万转”二句,用仄韵一束以下至“仙之人兮”句,转韵不转气,全以笔力驱驾,遂成鞭山倒海之能,读云似未曾转韵者,有真气行乎其间也。此妙可心悟,不可言喻。出梦时,用“忽动悸以魄动”四句,似亦可以收煞得住,试想若不再足“世间行乐”二句,非但叫题不酲,抑亦尚欠圆满。“且放白鹿”二句,一纵一收,用笔灵妙不测。后来慢东坡解此法,他人多昧昧耳。

日本近藤元粹《李太白诗醇》:严云:“半壁”一句,不独境界超绝,语音亦复高朗。严云:有意味在“青青”“澹澹”字作叠(“云青青兮”二句下)。严云:太白写仙人境界皆渺茫寂历,独此一段极真,极雄,反不似梦中语(“霓为衣兮”四句下)。又云:“世间”云云,甚达,甚警策,然自是唐人语,无宋气。{[}9{]}


\section{1.7   作者简介}
\label{\detokenize{p01_u6563_u6587/_u674e_u767d-_u68a6_u6e38_u5929_u59e5_u541f_u7559_u522b:id9}}
李白(701—762),字太白,号青莲居士。是屈原之后最具个性特色、最伟大的浪漫主义诗人。有“诗仙”之美誉,与杜甫并称“李杜”。其诗以抒情为主,表现出蔑视权贵的傲岸精神,对人民疾苦表示同情,又善于描绘自然景色,表达对祖国山河的热爱。诗风雄奇豪放,想象丰富,语言流转自然,音律和谐多变,善于从民间文艺和神话传说中吸取营养和素材,构成其特有的瑰玮绚烂的色彩,达到盛唐诗歌艺术的巅峰。存世诗文千余篇,有《李太白集》30卷。


\chapter{1   李白-蜀道难}
\label{\detokenize{p01_u6563_u6587/_u674e_u767d-_u8700_u9053_u96be:id1}}\label{\detokenize{p01_u6563_u6587/_u674e_u767d-_u8700_u9053_u96be::doc}}
\begin{sphinxShadowBox}
\sphinxstyletopictitle{目录}
\begin{itemize}
\item {} 
\phantomsection\label{\detokenize{p01_u6563_u6587/_u674e_u767d-_u8700_u9053_u96be:id10}}{\hyperref[\detokenize{p01_u6563_u6587/_u674e_u767d-_u8700_u9053_u96be:id1}]{\sphinxcrossref{1   李白-蜀道难}}}
\begin{itemize}
\item {} 
\phantomsection\label{\detokenize{p01_u6563_u6587/_u674e_u767d-_u8700_u9053_u96be:id11}}{\hyperref[\detokenize{p01_u6563_u6587/_u674e_u767d-_u8700_u9053_u96be:id3}]{\sphinxcrossref{1.1   作品原文}}}

\item {} 
\phantomsection\label{\detokenize{p01_u6563_u6587/_u674e_u767d-_u8700_u9053_u96be:id12}}{\hyperref[\detokenize{p01_u6563_u6587/_u674e_u767d-_u8700_u9053_u96be:id4}]{\sphinxcrossref{1.2   词句注释}}}

\item {} 
\phantomsection\label{\detokenize{p01_u6563_u6587/_u674e_u767d-_u8700_u9053_u96be:id13}}{\hyperref[\detokenize{p01_u6563_u6587/_u674e_u767d-_u8700_u9053_u96be:id5}]{\sphinxcrossref{1.3   白话译文}}}

\item {} 
\phantomsection\label{\detokenize{p01_u6563_u6587/_u674e_u767d-_u8700_u9053_u96be:id14}}{\hyperref[\detokenize{p01_u6563_u6587/_u674e_u767d-_u8700_u9053_u96be:id6}]{\sphinxcrossref{1.4   创作背景}}}

\item {} 
\phantomsection\label{\detokenize{p01_u6563_u6587/_u674e_u767d-_u8700_u9053_u96be:id15}}{\hyperref[\detokenize{p01_u6563_u6587/_u674e_u767d-_u8700_u9053_u96be:id7}]{\sphinxcrossref{1.5   整体赏析}}}

\item {} 
\phantomsection\label{\detokenize{p01_u6563_u6587/_u674e_u767d-_u8700_u9053_u96be:id16}}{\hyperref[\detokenize{p01_u6563_u6587/_u674e_u767d-_u8700_u9053_u96be:id8}]{\sphinxcrossref{1.6   历代评论}}}

\item {} 
\phantomsection\label{\detokenize{p01_u6563_u6587/_u674e_u767d-_u8700_u9053_u96be:id17}}{\hyperref[\detokenize{p01_u6563_u6587/_u674e_u767d-_u8700_u9053_u96be:id9}]{\sphinxcrossref{1.7   作者简介}}}

\end{itemize}

\end{itemize}
\end{sphinxShadowBox}

《蜀道难》是中国唐代大诗人李白的代表作品。此诗袭用乐府旧题,以浪漫主义的手法,展开丰富的想象,艺术地再现了蜀道峥嵘、突兀、强悍、崎岖等奇丽惊险和不可凌越的磅礴气势,借以歌咏蜀地山川的壮秀,显示出祖国山河的雄伟壮丽,充分显示了诗人的浪漫气质和热爱自然的感情。全诗二百九十四字,采用律体与散文间杂,文句参差,笔意纵横,豪放洒脱,感情强烈,一唱三叹。诗中诸多的画面此隐彼现,无论是山之高,水之急,河山之改观,林木之荒寂,连峰绝壁之险,皆有逼人之势,气象宏伟,境界阔大,集中体现了李白诗歌的艺术特色和创作个性,深受学者好评,被誉为“奇之又奇”之作。


\section{1.1   作品原文}
\label{\detokenize{p01_u6563_u6587/_u674e_u767d-_u8700_u9053_u96be:id3}}
蜀道难1

噫吁嚱2,危乎高哉!蜀道之难,难于上青天!

蚕丛及鱼凫3,开国何茫然4!尔来四万八千岁5,不与秦塞通人烟6。西当太白有鸟道7,可以横绝峨眉巅8。地崩山摧壮士死9,然后天梯石栈相钩连10。

上有六龙回日之高标11,下有冲波逆折之回川12。黄鹤之飞尚不得过13,猿猱欲度愁攀援14。青泥何盘盘15,百步九折萦岩峦16。扪参历井仰胁息17,以手抚膺坐长叹18。

问君西游何时还19?畏途巉岩不可攀20。但见悲鸟号古木21,雄飞雌从绕林间22。又闻子规啼夜月,愁空山23。蜀道之难,难于上青天,使人听此凋朱颜24。

连峰去天不盈尺25,枯松倒挂倚绝壁。飞湍瀑流争喧豗26,砯崖转石万壑雷27。其险也如此,嗟尔远道之人胡为乎来哉28!

剑阁峥嵘而崔嵬29,一夫当关,万夫莫开30。所守或匪亲31,化为狼与豺。

朝避猛虎32,夕避长蛇;磨牙吮血33,杀人如麻。锦城虽云乐34,不如早还家。蜀道之难,难于上青天,侧身西望长咨嗟35!


\section{1.2   词句注释}
\label{\detokenize{p01_u6563_u6587/_u674e_u767d-_u8700_u9053_u96be:id4}}
1.蜀道难:南朝乐府旧题,属《相和歌·瑟调曲》。

2.噫(yī)吁(xū)嚱(xī):惊叹声,蜀方言,表示惊讶的声音。宋庠《宋景文公笔记》卷上:“蜀人见物惊异,辄曰‘噫吁嚱’。”

3.蚕丛、鱼凫(fú):传说中古蜀国两位国王的名字;难以考证。

4.何茫然:何:多么。茫然:完全不知道的样子。指古史传说悠远难详,不知道。据西汉扬雄《蜀本王纪》记载:“蜀王之先,名蚕丛、柏灌、鱼凫,蒲泽、开明。……从开明上至蚕丛,积三万四千岁。”

5.尔来:从那时以来。四万八千岁:极言时间之漫长,夸张而大约言之。

6.秦塞(sài):秦的关塞,指秦地。秦地四周有山川险阻,故称”四塞之地”。通人烟:人员往来。

7.西当:在西边的。当:在。太白:太白山,又名太乙山,在长安西(今陕西眉县、太白县一带)。鸟道:指连绵高山间的低缺处,只有鸟能飞过,人迹所不能至。

8.横绝:横越。峨眉巅(diān):峨眉顶峰。苏教版语文课本为“峨眉颠”。

9.“地崩”句:《华阳国志·蜀志》:相传秦惠王想征服蜀国,知道蜀王好色,答应送给他五个美女。蜀王派五位壮士去接人。回到梓潼(今四川剑阁之南)的时候,看见一条大蛇进入穴中,一位壮士抓住了它的尾巴,其余四人也来相助,用力往外拽。不多时,山崩地裂,壮士和美女都被压死。山分为五岭,入蜀之路遂通。这便是有名的“五丁开山”的故事。摧,倒塌。

10.天梯:非常陡峭的山路。石栈(zhàn):栈道。

11.六龙回日:《淮南子》注云:“日乘车,驾以六龙。羲和御之。日至此面而薄于虞渊,羲和至此而回六螭。”,意思就是传说中的羲和驾驶着六龙之车(即太阳)到此处便迫近虞渊(传说中的日落处)。高标:指蜀山中可作一方之标识的最高峰。

12.冲波:水流冲击腾起的波浪,这里指激流。逆折:水流回旋。回川:有漩涡的河流。

13.黄鹤:即黄鹄(hú),善飞的大鸟。尚:尚且。得:能。

14.猿猱(náo):蜀山中最善攀援的猴类。

15.青泥:青泥岭,在今甘肃徽县南,陕西略阳县北。《元和郡县志》卷二十二:“青泥岭,在县西北五十三里,接溪山东,即今通路也。悬崖万仞,山多云雨,行者屡逢泥淖,故号青泥岭。”盘盘:曲折回旋的样子。

16.百步九折:百步之内拐九道弯。萦(yíng):盘绕。岩峦:山峰。

17.扪(mén)参(shēn)历井:参、井是二星宿名。古人把天上的星宿分别指配于地上的州国,叫做“分野”,以便通过观察天象来占卜地上所配州国的吉凶。参星为蜀之分野,井星为秦之分野。扪,用手摸。历,经过。胁息:屏气不敢呼吸。

18.膺(yīng):胸。坐:徒,空。

19.君:入蜀的友人。

20.畏途:可怕的路途。巉(chán)岩:险恶陡峭的山壁。

21.但见:只听见。号(háo)古木:在古树木中大声啼鸣。

22.从:跟随。

23.“又闻”二句:一本断为“又闻子规啼,夜月愁空山”。子规,即杜鹃鸟,蜀地最多,鸣声悲哀,若云“不如归去”。《蜀记》曰:“昔有人姓杜名宇,王蜀,号曰望帝。宇死,俗说杜宇化为子规。子规,鸟名也。蜀人闻子规鸣,皆曰望帝也。”

24.凋朱颜:红颜带忧色,如花凋谢。凋,使动用法,使……凋谢,这里指脸色由红润变成铁青。

25.去:距离。盈:满。

26.飞湍(tuān):飞奔而下的急流。喧豗(huī):喧闹声,这里指急流和瀑布发出的巨大响声。

27.砯(pīng)崖:水撞石之声。砯,水冲击石壁发出的响声,这里作动词用,冲击的意思。转:使滚动。壑(hè):山谷。

28.嗟(jiē):感叹声。尔:你。胡为:为什么。来:指入蜀。

29.剑阁:又名剑门关,在四川剑阁县北,是大、小剑山之间的一条栈道,长约三十余里。峥嵘、崔(cuīwéi)嵬:都是形容山势高大雄峻的样子。

30.“一夫”两句:《文选》卷四左思《蜀都赋》:“一人守隘,万夫莫向”。《文选》卷五十六张载《剑阁铭》:“一人荷戟,万夫趦趄。形胜之地,匪亲勿居。”一夫,一人。当关,守关。莫开,不能打开。

31.所守:指把守关口的人。或匪(fěi)亲:倘若不是可信赖的人。匪,同“非”。

32.朝(zhāo):早上。

33.吮(shǔn)血(xuè):吸血。

34.锦城:成都古代以产棉闻名,朝廷曾经设官于此,专收棉织品,故称锦城或锦官城。《元和郡县志》卷三十一剑南道成都府成都县:“锦城在县南十里,故锦官城也。”今四川成都市。

35.咨(zī)嗟:叹息。{[}2{]}{[}3-4{]}


\section{1.3   白话译文}
\label{\detokenize{p01_u6563_u6587/_u674e_u767d-_u8700_u9053_u96be:id5}}
唉呀呀,多么危险多么高峻伟岸!蜀道真太难攀简直难于上青天。

传说中蚕丛和鱼凫建立了蜀国,开国的年代实在久远无法详谈。自从那时至今约有四万八千年,秦蜀被秦岭所阻从不沟通往返。西边太白山有飞鸟能过的小道。从那小路走可横渡峨嵋山顶端。山崩地裂蜀国五壮士被压死了,两地才有天梯栈道开始相通连。

上有挡住太阳神六龙车的山巅,下有激浪排空纡回曲折的大川。善于高飞的黄鹤尚且无法飞过,即使猢狲要想翻过也愁于攀援。青泥岭多么曲折绕着山峦盘旋,百步之内萦绕岩峦转九个弯弯。可以摸到参井星叫人仰首屏息,用手抚胸惊恐不已坐下来长叹。

好朋友呵请问你西游何时回还?可怕的岩山栈道实在难以登攀!只见那悲鸟在古树上哀鸣啼叫,雄雌相随飞翔在原始森林之间。月夜听到的是杜鹃悲惨的啼声,令人愁思绵绵呵这荒荡的空山!蜀道真难走呵简直难于上青天,叫人听到这些怎么不脸色突变?

山峰座座相连离天还不到一尺;枯松老枝倒挂倚贴在绝壁之间。漩涡飞转瀑布飞泻争相喧闹着;水石相击转动象万壑鸣雷一般。那去处恶劣艰险到了这种地步;唉呀呀你这个远方而来的客人,为了什么要来到这个险要地方?

剑阁那地方崇峻巍峨高入云端,只要一人把守千军万马难攻占。驻守的官员若不是皇家的近亲;难免要变为豺狼踞此为非造反。

清晨你要提心吊胆地躲避猛虎,傍晚你要警觉防范长蛇的灾难。豺狼虎豹磨牙吮血真叫人不安,毒蛇猛兽杀人如麻即令你胆寒。锦官城虽然说是个快乐的所在,如此险恶还不如早早地把家还。蜀道太难走呵简直难于上青天,侧身西望令人不免感慨与长叹!{[}2{]}


\section{1.4   创作背景}
\label{\detokenize{p01_u6563_u6587/_u674e_u767d-_u8700_u9053_u96be:id6}}
对《蜀道难》的创作背景,从唐代开始人们就多有猜测,主要有四种说法:甲、此诗系为房琯、杜甫二人担忧,希望他们早日离开四川,免遭剑南节度使严武的毒手;乙、此诗是为躲避安史之乱逃亡至蜀的唐玄宗李隆基而作,劝喻他归返长安,以免受四川地方军阀挟制;丙、此诗旨在讽刺当时蜀地长官章仇兼琼想凭险割据,不听朝廷节制;丁,此诗纯粹歌咏山水风光,并无寓意。

这首诗最早见录于唐人殷璠所编的《河岳英灵集》,该书编成于唐玄宗天宝十二载(753年),由此可知李白这首诗的写作年代最迟也应该在《河岳英灵集》编成之前。而那时,安史之乱尚未发生,唐玄宗安居长安,房(琯)、杜甫也都还未入川,所以,甲、乙两说显然错误。至于讽刺章仇兼琼的说法,从一些史书的有关记载来看,也缺乏依据。章仇兼琼镇蜀时一直理想去长安做官。相对而言,还是最后一种说法比较客观,接近于作品实际。

这可能是一首赠友诗。有学者认为这首诗可能是天宝元年至三年(742至744年)李白在长安时为送友人王炎入蜀而写的,目的是规劝王炎不要羁留蜀地,早日回归长安,避免遭到嫉妒小人不测之手;也有学者认为此诗是开元年间李白初入长安无成而归时,送友人寄意之作。


\section{1.5   整体赏析}
\label{\detokenize{p01_u6563_u6587/_u674e_u767d-_u8700_u9053_u96be:id7}}
《蜀道难》是李白袭用乐府古题,展开丰富的想象,着力描绘了秦蜀道路上奇丽惊险的山川,并从中透露了对社会的某些忧虑与关切。

诗人大体按照由古及今,自秦入蜀的线索,抓住各处山水特点来描写,以展示蜀道之难。

从“噫吁嚱”到“然后天梯石栈相钩连”为一个段落。一开篇就极言蜀道之难,以感情强烈的咏叹点出主题,为全诗奠定了雄放的基调。以下随着感情的起伏和自然场景的变化,“蜀道之难,难于上青天”的咏叹反复出现,像一首乐曲的主旋律一样激荡着读者的心弦。

说蜀道的难行比上天还难,这是因为自古以来秦、蜀之间被高山峻岭阻挡,由秦入蜀,太白峰首当其冲,只有高飞的鸟儿能从低缺处飞过。太白峰在秦都咸阳西南,是关中一带的最高峰。民谚云:“武公太白,去天三百。”诗人以夸张的笔墨写出了历史上不可逾越的险阻,并融汇了五丁开山的神话,点染了神奇色彩,犹如一部乐章的前奏,具有引人入胜的妙用。下面即着力刻画蜀道的高危难行了。

从“上有六龙回日之高标”至“使人听此凋朱颜”为又一段落。这一段极写山势的高危,山高写得愈充分,愈可见路之难行。你看那突兀而立的高山,高标接天,挡住了太阳神的运行;山下则是冲波激浪、曲折回旋的河川。诗人不但把夸张和神话融为一体,直写山高,而且衬以“回川”之险。唯其水险,更见山势的高危。诗人意犹未足,又借黄鹤与猿猱来反衬。山高得连千里翱翔的黄鹤也不得飞度,轻疾敏捷的猿猴也愁于攀援,不言而喻,人行走就难上加难了。以上用虚写手法层层映衬,下面再具体描写青泥岭的难行。

青泥岭,“悬崖万仞,山多云雨”(《元和郡县志》),为唐代入蜀要道。诗人着重就其峰路的萦回和山势的峻危来表现人行其上的艰难情状和畏惧心理,捕捉了在岭上曲折盘桓、手扪星辰、呼吸紧张、抚胸长叹等细节动作加以摹写,寥寥数语,便把行人艰难的步履、惶悚的神情,绘声绘色地刻画出来,困危之状如在目前。

至此蜀道的难行似乎写到了极处。但诗人笔锋一转,借“问君”引出旅愁,以忧切低昂的旋律,把读者带进一个古木荒凉、鸟声悲凄的境界。杜鹃鸟空谷传响,充满哀愁,使人闻声失色,更觉蜀道之难。诗人借景抒情,用“悲鸟号古木”、“子规啼夜月”等感情色彩浓厚的自然景观,渲染了旅愁和蜀道上空寂苍凉的环境气氛,有力地烘托了蜀道之难。

然而,逶迤千里的蜀道,还有更为奇险的风光。自“连峰去天不盈尺”至全篇结束,主要从山川之险来揭示蜀道之难,着力渲染惊险的气氛。如果说“连峰去天不盈尺”是夸饰山峰之高,“枯松倒挂倚绝壁”则是衬托绝壁之险。

诗人先托出山势的高险,然后由静而动,写出水石激荡、山谷轰鸣的惊险场景。好像一串电影镜头:开始是山峦起伏、连峰接天的远景画面;接着平缓地推成枯松倒挂绝壁的特写;而后,跟踪而来的是一组快镜头,飞湍、瀑流、悬崖、转石,配合着万壑雷鸣的音响,飞快地从眼前闪过,惊险万状,目不暇接,从而造成一种势若排山倒海的强烈艺术效果,使蜀道之难的描写,简直达到了登峰造极的地步。如果说上面山势的高危已使人望而生畏,那此处山川的险要更令人惊心动魄了。

风光变幻,险象丛生。在十分惊险的气氛中,最后写到蜀中要塞剑阁,在大剑山和小剑山之间有一条三十里长的栈道,群峰如剑,连山耸立,削壁中断如门,形成天然要塞。因其地势险要,易守难攻,历史上在此割据称王者不乏其人。诗人从剑阁的险要引出对政治形势的描写。他化用西晋张载《剑阁铭》中“形胜之地,匪亲勿居”的语句,劝人引为鉴戒,警惕战乱的发生,并联系当时的社会背景,揭露了蜀中豺狼的“磨牙吮血,杀人如麻”,从而表达了对国事的忧虑与关切。唐天宝初年,太平景象的背后正潜伏着危机,后来发生的安史之乱,证明诗人的忧虑是有现实意义的。

李白以变化莫测的笔法,淋漓尽致地刻画了蜀道之难,艺术地展现了古老蜀道逶迤、峥嵘、高峻、崎岖的面貌,描绘出一幅色彩绚丽的山水画卷。诗中那些动人的景象宛如历历在目。

李白之所以描绘得如此动人,还在于融贯其间的浪漫主义激情。诗人寄情山水,放浪形骸。他对自然景物不是冷漠的观赏,而是热情地赞叹,借以抒发自己的理想感受。那飞流惊湍、奇峰险壑,赋予了诗人的情感气质,因而才呈现出飞动的灵魂和瑰伟的姿态。诗人善于把想象、夸张和神话传说融为一体进行写景抒情。言山之高峻,则曰“上有六龙回日之高标”;状道之险阻,则曰“地崩山摧壮士死,然后天梯石栈相钩连”。诗人“驰走风云,鞭挞海岳”(陆时雍《诗镜总论》评李白七古语),从蚕丛开国说到五丁开山,由六龙回日写到子规夜啼,天马行空般地驰骋想象,创造出博大浩渺的艺术境界,充满了浪漫主义色彩。透过奇丽峭拔的山川景物,仿佛可以看到诗人那“落笔摇五岳、笑傲凌沧洲”的高大形象。

唐以前的《蜀道难》作品,简短单薄。李白对东府古题有所创新和发展,用了大量散文化诗句,字数从三言、四言、五言、七言,直到十一言,参差错落,长短不齐,形成极为奔放的语言风格。诗的用韵,也突破了梁陈时代旧作一韵到底的程式。后面描写蜀中险要环境,一连三换韵脚,极尽变化之能事。所以殷璠编《河岳英灵集》称此诗“奇之又奇,自骚人以还,鲜有此体调”。

关于此篇,前人有种种寓意之说,断定是专为某人某事而作的。明人胡震亨、顾炎武认为,李白“自为蜀咏”,“别无寓意”。今人有谓此诗表面写蜀道艰险,实则写仕途坎坷,反映了诗人在长期漫游中屡逢踬碍的生活经历和怀才不遇的愤懑,迄无定论。


\section{1.6   历代评论}
\label{\detokenize{p01_u6563_u6587/_u674e_u767d-_u8700_u9053_u96be:id8}}
《本事诗》:李太白初自蜀至京师,舍于逆旅,贺监知孕闻其名,首访之。既奇其姿,复请所为文。出《蜀道难》以示之。读未竟,称叹者数四,号为“谪仙”,解金龟换酒,与倾尽醉,期不间日。由足称誉光赫。

《木天禁语》:七言长古篇法……旧题乃篇末一、二句缴上起句,又谓之“顾首”、如《蜀道难》、《古别离》、《洗兵马行》是也。

《唐诗品汇》:刘须溪云:妙在起伏,其才思放肆,语次崛奇,自不在言。

《四溟诗话》:九言体,无名氏拟之曰:“昨夜西风摇落千林梢,渡头小舟卷入寒塘坳。”声调散缓而无气魄。惟太白上篇突出两句,殊不可及,若“上有六龙回日之高标,下有冲波逆折之回川”是也。

《批选唐诗正声》:辞旨深远,雄浑飘逸,杜子美所不可到。欧阳子以《庐山高》方之,殊为哂。

《唐诗援》:太白创体,空前绝后。诸说纷纷不一,然细观此诗,定为明皇幸蜀而作。萧说是。

《批选唐诗》:太白长歌,森秀飞扬,疾于风雨,本其才性独诣,非由人力。人所不及在此,诗教大坏亦在此。后生学步,奋猛亢厉之音作,而温柔敦厚之意尽,露才扬己,长慠负气、辞人所以多轻薄,由来远已。嗟乎,西日东流,又岂人力哉!但可谓之唐体而已矣。

《唐音癸签》:《蜀道难》自是古曲,梁陈作者,止言其险,时不及其他。白则兼采张载《剑阁铭》“一人荷戟,万夬趑趄,形胜之地,匪亲弗居”等语用之,为恃险割据与羁留佐逆者著戒。惟其海说事理,故苞括大,而有合乐府讽世立教本旨。若第取一时一人事实之,反失之细而不足味矣。

《唐诗镜》:《蜀道难》近赋体,魁梧奇谲,知是伟大。

《唐诗选脉会通评林》:周珽曰:……“一夫当关”四句,设意外之忧;“朝避猛虎”四句,指阶见之恐,见变生肘腋,地终不可居。总言蜀道之难也。劈空落想,窍凿幽发,应使笔墨生而混沌死。

《诗源辨体》:屈原《离骚》本千古辞赋之宗,而后人摹仿盗袭,不胜厌饫……至《远别离》、《蜀道难》、《天姥吟》,则变幻恍惚,尽脱蹊径,实与屈子互相照映。

《唐风定》:变幻神奇,仙而不鬼,长吉魔语视之何如?亘古代无能仿象,才涉意即入长吉魔中矣。通篇奇险,不涉旁意,不参平调,其胜《天姥》、《鸣皋》以此。

《王文简古诗平仄论》:(七言古)又有长短句者,唐惟李太白多有之,然不必学。如《蜀道难》……效之而无其才,洵难免沧溟“英雄欺人”之诮。

《增订唐诗摘钞》:倏起倏落,忽虚忽实。真如烟水杳渺,绝世奇文也。

《载酒园诗话又编》:《蜀道难》一篇,真与河岳并垂不朽。即起句“噫吁戏,危沪高哉”七字,如累棋架卵,谁敢并于一处?至其造句之妙:“连峰去天不盈尺,枯松倒扯倚绝壁。飞湍瀑流争喧豗,砅虚转石万壑雷。”每读之。剑阁、阴平、如在目前。又如“一夫当关,万夫莫开。所守或匪亲,化为狼与豺”、不惟刘璋、李势恨事如见,即孟知祥一辈亦逆揭其肺肝。此真诗之有关系者,岂特文词之雄!

《唐音审体》:篇中三言蜀道之难,所谓一唱三叹也。突然以嗟叹起,嗟叹结,创格也。

《放胆诗》:太白《蜀道难》、《远别离》等篇出鬼入冲,惝恍莫测。

《此木轩论诗汇编》:《蜀道难》、旧题也,太白为之,加奇肆耳。此千古绝调也,后人妄意学步,何其不知量也!“噫吁嚱,危乎高哉”,七字五句。“连峰去天不盈尺”无理之极,俗本作“连峰入烟几千尺”,有理之极。无理之妙,妙不可言。有理之不妙,其不妙亦不可胜言。举此一隅,即是学诗家万金良药也。

《而庵说唐诗》:“尔来四万八千岁”,此云总非实据也。人言文人无实语,而不知文章家妙在跌宕;每说到已甚,太白用此,正跌宕法也。“蜀道之难,难于上青天”再一提,此句妙有关锁,上来笔气纵横,逸宕不如此,则散无统束矣。“锦城虽云乐”:上面说到蜀如此可惊、可畏,而忽下一“乐”字,妙极。“不如早还家”:此虽是乐,不可久居,“不如早还家”之句尤乐也。文势至此甚紧,必须一放,方得宽转,所谓“一张一弛,文武之道”也。“蜀道之难,难于上青天”,复提此句为结束,妙。篇中凡三见,与《庄子·逍遥游》叙鲲鹏同。吾尝谓作长篇古诗,须读《庄子》、《史记》。子美歌行纯学《史记》,太白歌行纯学《庄子》。故两先生为歌行之双绝,不诬也。

《唐诗别裁》:笔阵纵撗,如虬飞蠖动,起雷霆于指顾之间。任华,卢仝辈仿之,适得其怪耳,太白所以为仙才也。

《剑溪说诗》:太白诗“蜀道之难,难于上青天”句,凡三叠。管子曰:“使海于有蔽,渠弥于有渚,纲山于有牢。”谷梁氏曰:“梁山崩,壅遏河三日不流。”一篇之中,三番叙述,愈见其妙。所谓“闭户造车,出门合辙”者也。

《网师园唐诗笺》:造语奇险(“地崩山摧”二句下)。玩此,为明皇幸蜀作无疑(“问君西游”句下)。兜来何等力量。(“其险”句下)!高文险语,动魄惊心(“磨牙”二句下)。主意在此(“不如”句下)。

《李太白诗醇》:严云:提“蜀道难”,篇中三致意;用“噫吁戏”三字起,非无谓。后人学袭,便成恶道。{[}8{]}


\section{1.7   作者简介}
\label{\detokenize{p01_u6563_u6587/_u674e_u767d-_u8700_u9053_u96be:id9}}
李白(701—762),字太白,号青莲居士。是屈原之后最具个性特色、最伟大的浪漫主义诗人。有“诗仙”之美誉,与杜甫并称“李杜”。其诗以抒情为主,表现出蔑视权贵的傲岸精神,对人民疾苦表示同情,又善于描绘自然景色,表达对祖国山河的热爱。诗风雄奇豪放,想象丰富,语言流转自然,音律和谐多变,善于从民间文艺和神话传说中吸取营养和素材,构成其特有的瑰玮绚烂的色彩,达到盛唐诗歌艺术的巅峰。存世诗文千余篇,有《李太白集》三十卷。


\chapter{1   毛泽东-七律·长征}
\label{\detokenize{p01_u6563_u6587/_u6bdb_u6cfd_u4e1c-_u4e03_u5f8b_xb7_u957f_u5f81:id1}}\label{\detokenize{p01_u6563_u6587/_u6bdb_u6cfd_u4e1c-_u4e03_u5f8b_xb7_u957f_u5f81::doc}}
\begin{sphinxShadowBox}
\sphinxstyletopictitle{目录}
\begin{itemize}
\item {} 
\phantomsection\label{\detokenize{p01_u6563_u6587/_u6bdb_u6cfd_u4e1c-_u4e03_u5f8b_xb7_u957f_u5f81:id7}}{\hyperref[\detokenize{p01_u6563_u6587/_u6bdb_u6cfd_u4e1c-_u4e03_u5f8b_xb7_u957f_u5f81:id1}]{\sphinxcrossref{1   毛泽东-七律·长征}}}
\begin{itemize}
\item {} 
\phantomsection\label{\detokenize{p01_u6563_u6587/_u6bdb_u6cfd_u4e1c-_u4e03_u5f8b_xb7_u957f_u5f81:id8}}{\hyperref[\detokenize{p01_u6563_u6587/_u6bdb_u6cfd_u4e1c-_u4e03_u5f8b_xb7_u957f_u5f81:id3}]{\sphinxcrossref{1.1   作品原文}}}

\item {} 
\phantomsection\label{\detokenize{p01_u6563_u6587/_u6bdb_u6cfd_u4e1c-_u4e03_u5f8b_xb7_u957f_u5f81:id9}}{\hyperref[\detokenize{p01_u6563_u6587/_u6bdb_u6cfd_u4e1c-_u4e03_u5f8b_xb7_u957f_u5f81:id4}]{\sphinxcrossref{1.2   词句注释}}}

\item {} 
\phantomsection\label{\detokenize{p01_u6563_u6587/_u6bdb_u6cfd_u4e1c-_u4e03_u5f8b_xb7_u957f_u5f81:id10}}{\hyperref[\detokenize{p01_u6563_u6587/_u6bdb_u6cfd_u4e1c-_u4e03_u5f8b_xb7_u957f_u5f81:id5}]{\sphinxcrossref{1.3   白话译文}}}

\item {} 
\phantomsection\label{\detokenize{p01_u6563_u6587/_u6bdb_u6cfd_u4e1c-_u4e03_u5f8b_xb7_u957f_u5f81:id11}}{\hyperref[\detokenize{p01_u6563_u6587/_u6bdb_u6cfd_u4e1c-_u4e03_u5f8b_xb7_u957f_u5f81:id6}]{\sphinxcrossref{1.4   创作背景}}}

\end{itemize}

\end{itemize}
\end{sphinxShadowBox}

《七律·长征》是一首七言律诗,选自《毛泽东诗词集》,这首诗写于1935年10月,当时毛泽东率领中央红军越过岷山,长征即将结束。回顾长征一年来所战胜的无数艰难险阻,他满怀喜悦的战斗豪情。


\section{1.1   作品原文}
\label{\detokenize{p01_u6563_u6587/_u6bdb_u6cfd_u4e1c-_u4e03_u5f8b_xb7_u957f_u5f81:id3}}
七律·长征

七律⑴·长征⑵

红军不怕远征难⑶,万水千山只等闲⑷。

五岭⑸逶迤⑹腾细浪⑺,乌蒙⑻磅礴走泥丸⑼。

金沙⑽水拍云崖暖⑾,大渡桥⑿横铁索⒀寒⒁。

更喜岷山⒂千里雪,三军⒃过后尽开颜⒄。{[}1{]}


\section{1.2   词句注释}
\label{\detokenize{p01_u6563_u6587/_u6bdb_u6cfd_u4e1c-_u4e03_u5f8b_xb7_u957f_u5f81:id4}}
⑴七律:七律是律诗的一种,每篇一般为八句,每句七个字,分四联:首联、颔联、颈联和尾联;偶句末一字押平声韵,首句末字可押可不押,必须一韵到底;句内和句间要讲平仄,中间四句按常规要用对仗。

⑵长征:1934年10月间,中央红军主力从中央革命根据地出发作战略大转移,经过福建、江西、广东、湖南、广西、贵州、四川、云南、西藏、甘肃、陕西等十一省,击溃了敌人多次的围追和堵截,战胜了军事上、政治上和自然界的无数艰险,行军二万五千里,终于在1935年10月到达陕北革命根据地。

⑶难:艰难险阻。

⑷等闲:不怕困难,不可阻止。

⑸五岭:大庾岭,骑田岭,都庞岭,萌渚岭,越城岭,横亘在江西、湖南、两广之间。

⑹逶迤:形容道路、山脉、河流等弯弯曲曲,连绵不断的样子。

⑺细浪:作者自释:“把山比作‘细浪’、‘泥丸’,是‘等闲’之意。”

⑻乌蒙:山名。乌蒙山,在贵州西部与云南东北部的交界处,北临金沙江,山势陡峭。1935年4月,红军长征经过此地。

⑼泥丸:小泥球,整句意思说险峻的乌蒙山在红军战士的脚下,就像是一个小泥球一样。

⑽金沙:金沙江,指长江上游自青海省玉树县至四川省宜宾市的一段,云南等地也有支流。1935年5月,红军曾强渡云南省禄劝县皎平渡渡口。

⑾云崖暖:是指浪花拍打悬崖峭壁,溅起阵阵雾水,在红军的眼中像是冒出的蒸汽一样。(云崖:高耸入云的山崖。暖:被一些学者指为红军巧渡金沙江后的欢快心情,也有学者说意思为直译后的温暖。)

⑿大渡桥:指四川省西部泸定县大渡河上的泸定桥。

⒀铁索:大渡河上泸定桥,它是用十三根铁索组成的桥。

⒁寒:影射敌人的冷酷与形势的严峻。

⒂岷(mín)山:中国西部大山。位于甘肃省西南、四川省北部。西北-东南走向。西北接西倾山,南与邛崃山相连。包括甘肃南部的迭山,甘肃、四川边境的摩天岭。

⒃三军:作者自注:“红军一方面军,二方面军,四方面军。”

⒄尽开颜:红军的长征到达目的地了,他们取得了胜利,所以个个都笑逐颜开。


\section{1.3   白话译文}
\label{\detokenize{p01_u6563_u6587/_u6bdb_u6cfd_u4e1c-_u4e03_u5f8b_xb7_u957f_u5f81:id5}}
红军不怕万里长征路上的一切艰难困苦,把千山万水都看得极为平常。绵延不断的五岭,在红军看来只不过是微波细浪在起伏,而气势雄伟的乌蒙山,在红军眼里也不过是一颗泥丸。

金沙江浊浪滔天,拍击着高耸入云的峭壁悬崖,热气腾腾。大渡河险桥横架,晃动着凌空高悬的根根铁索,寒意阵阵。

更加令人喜悦的是踏上千里积雪的岷山,红军翻越过去以后个个笑逐颜开。


\section{1.4   创作背景}
\label{\detokenize{p01_u6563_u6587/_u6bdb_u6cfd_u4e1c-_u4e03_u5f8b_xb7_u957f_u5f81:id6}}
1934年10月,中国工农红军为粉碎国民政府的围剿,保存自己的实力,也为了北上抗日,挽救民族危亡,从江西瑞金出发,开始了举世闻名的长征。

这首七律是作于红军战士越过岷山后,长征即将胜利结束前不久的途中。作为红军的领导人,毛泽东在经受了无数次考验后,如今,曙光在前,胜利在望,他心潮澎湃,满怀豪情地写下了这首壮丽的诗篇。《七律·长征》写于1935年9月下旬,10月定稿。


\chapter{1   毛泽东-沁园春·雪}
\label{\detokenize{p01_u6563_u6587/_u6bdb_u6cfd_u4e1c-_u6c81_u56ed_u6625_xb7_u96ea:id1}}\label{\detokenize{p01_u6563_u6587/_u6bdb_u6cfd_u4e1c-_u6c81_u56ed_u6625_xb7_u96ea::doc}}
\begin{sphinxShadowBox}
\sphinxstyletopictitle{目录}
\begin{itemize}
\item {} 
\phantomsection\label{\detokenize{p01_u6563_u6587/_u6bdb_u6cfd_u4e1c-_u6c81_u56ed_u6625_xb7_u96ea:id8}}{\hyperref[\detokenize{p01_u6563_u6587/_u6bdb_u6cfd_u4e1c-_u6c81_u56ed_u6625_xb7_u96ea:id1}]{\sphinxcrossref{1   毛泽东-沁园春·雪}}}
\begin{itemize}
\item {} 
\phantomsection\label{\detokenize{p01_u6563_u6587/_u6bdb_u6cfd_u4e1c-_u6c81_u56ed_u6625_xb7_u96ea:id9}}{\hyperref[\detokenize{p01_u6563_u6587/_u6bdb_u6cfd_u4e1c-_u6c81_u56ed_u6625_xb7_u96ea:id3}]{\sphinxcrossref{1.1   作品原文}}}

\item {} 
\phantomsection\label{\detokenize{p01_u6563_u6587/_u6bdb_u6cfd_u4e1c-_u6c81_u56ed_u6625_xb7_u96ea:id10}}{\hyperref[\detokenize{p01_u6563_u6587/_u6bdb_u6cfd_u4e1c-_u6c81_u56ed_u6625_xb7_u96ea:id4}]{\sphinxcrossref{1.2   词句注释}}}

\item {} 
\phantomsection\label{\detokenize{p01_u6563_u6587/_u6bdb_u6cfd_u4e1c-_u6c81_u56ed_u6625_xb7_u96ea:id11}}{\hyperref[\detokenize{p01_u6563_u6587/_u6bdb_u6cfd_u4e1c-_u6c81_u56ed_u6625_xb7_u96ea:id5}]{\sphinxcrossref{1.3   白话译文}}}

\item {} 
\phantomsection\label{\detokenize{p01_u6563_u6587/_u6bdb_u6cfd_u4e1c-_u6c81_u56ed_u6625_xb7_u96ea:id12}}{\hyperref[\detokenize{p01_u6563_u6587/_u6bdb_u6cfd_u4e1c-_u6c81_u56ed_u6625_xb7_u96ea:id6}]{\sphinxcrossref{1.4   创作背景}}}

\item {} 
\phantomsection\label{\detokenize{p01_u6563_u6587/_u6bdb_u6cfd_u4e1c-_u6c81_u56ed_u6625_xb7_u96ea:id13}}{\hyperref[\detokenize{p01_u6563_u6587/_u6bdb_u6cfd_u4e1c-_u6c81_u56ed_u6625_xb7_u96ea:id7}]{\sphinxcrossref{1.5   名家点评}}}

\end{itemize}

\end{itemize}
\end{sphinxShadowBox}

《沁园春·雪》是无产阶级革命家毛泽东创作的一首词。该词上片描写北国壮丽的雪景,纵横千万里,展示了大气磅礴、旷达豪迈的意境,抒发了词人对祖国壮丽河山的热爱。下片议论抒情,重点评论历史人物,歌颂当代英雄,抒发无产阶级要做世界的真正主人的豪情壮志。全词熔写景、议论和抒情于一炉,意境壮美,气势恢宏,感情奔放,胸襟豪迈,颇能代表毛泽东诗词的豪放风格。


\section{1.1   作品原文}
\label{\detokenize{p01_u6563_u6587/_u6bdb_u6cfd_u4e1c-_u6c81_u56ed_u6625_xb7_u96ea:id3}}
沁园春·雪1

北国风光,千里冰封,万里雪飘。望长城内外,惟余莽莽2;大河上下,顿失滔滔3。山舞银蛇,原驰蜡象4,欲与天公试比高。须晴日5,看红装素裹,分外妖娆6。

江山如此多娇,引无数英雄竞折腰7。惜秦皇汉武,略输文采8;唐宗宋祖9,稍逊风骚。一代天骄,成吉思汗,只识弯弓射大雕。俱往矣,数风流人物,还看今朝。


\section{1.2   词句注释}
\label{\detokenize{p01_u6563_u6587/_u6bdb_u6cfd_u4e1c-_u6c81_u56ed_u6625_xb7_u96ea:id4}}
1.沁园春:词牌名,又名“东仙”“寿星明”“洞庭春色”等。双调,一百十四字。前段十三句,四平韵;后段十二句,五平韵。

2.惟余:只剩下。余:有版本作“馀”。莽莽:即茫茫,白茫茫一片。形容空旷无际。

3.顿失:立刻失去。顿:顿时,立刻。滔滔:滚滚的波涛。

4.原驰蜡象:作者原注“原指高原,即秦晋高原”。驰:有版本作“驱”。蜡象:白色的象。

5.须:待、等到。

6.“看红装”二句:红日和白雪互相映照,看去好像装饰艳丽的美女裹着白色的外衣,格外娇媚。红装:身着艳丽服饰的美女。一作银装。妖娆(ráo):娇艳妩媚。

7.竞折腰:争着为江山奔走效劳。折腰:倾倒,躬着腰侍候。

8.“秦皇汉武”二句:是说秦皇汉武,功业甚盛,相比之下,文治方面的成就略有逊色。秦皇:秦始皇赢政,秦朝的创业皇帝。汉武:汉武帝刘彻,西汉第七位皇帝。略输:稍差。文采:本指辞藻、才华。这里引申为文治。

9.唐宗:唐太宗李世民,唐朝建立统一大业的皇帝。宋祖:宋太祖赵匡胤,宋朝的创业皇帝。

10.稍逊风骚:意近“略输文采”。逊:差。风骚:本指《诗经》里的《国风》和《楚辞》里的《离骚》,后来泛指文章辞藻。

11.天骄:汉时匈奴自称为“天之骄子”,以后泛称强盛的边地民族。

12.成吉思汗:元太祖铁木真统一蒙古后的尊称,意思是“强者之汗”。

13.“只识”句:是说只以武功见长。识:知道,懂得。雕:一种鹰类的大型猛禽,善飞难射,古代因用“射雕手”比喻高强的射手。{[}2{]}


\section{1.3   白话译文}
\label{\detokenize{p01_u6563_u6587/_u6bdb_u6cfd_u4e1c-_u6c81_u56ed_u6625_xb7_u96ea:id5}}
北方的风光,千里冰封冻,万里雪花飘。望长城内外,只剩下无边无际白茫茫一片;宽广的黄河上下,顿时失去了滔滔水势。山岭好像银白色的蟒蛇在飞舞,高原上的丘陵好像许多白象在奔跑,它们都想与老天爷比比高。要等到晴天的时候,看红艳艳的阳光和白皑皑的冰雪交相辉映,分外美好。

江山如此媚娇,引得无数英雄竞相倾倒。只可惜秦始皇、汉武帝,略差文学才华;唐太宗、宋太祖,稍逊文治功劳。称雄一世的人物成吉思汗,只知道拉弓射大雕。这些人物全都过去了,称得上能建功立业的英雄人物,还要看今天的人们。{[}3{]}


\section{1.4   创作背景}
\label{\detokenize{p01_u6563_u6587/_u6bdb_u6cfd_u4e1c-_u6c81_u56ed_u6625_xb7_u96ea:id6}}
1936年,红军组织东征部队,准备东渡黄河对日军作战。红军从子长县出发,挺进到清涧县高杰村的袁家沟一带时,部队在这里休整了16天。2月5日至20日,毛泽东在这里居住期间,曾下过一场大雪,长城内外白雪皑皑,隆起的秦晋高原,冰封雪盖。天气严寒,连平日奔腾咆哮的黄河都结了一层厚厚的冰,失去了往日的波涛。毛泽东当时住在农民白治民家中,深夜。见此情景,颇有感触,填写了这首词。《沁园春·雪》最早发表于1945年11月14日重庆《新民报晚刊》,后正式发表于《诗刊》1957年1月号。{[}4-5{]}


\section{1.5   名家点评}
\label{\detokenize{p01_u6563_u6587/_u6bdb_u6cfd_u4e1c-_u6c81_u56ed_u6625_xb7_u96ea:id7}}
近代诗人柳亚子《沁园春·雪》跋:毛润之沁园春一阕,余推为千古绝唱,虽东坡、幼安,犹瞠乎其后,更无论南唐小令、南宋慢词矣。


\chapter{1   老舍-济南的冬天}
\label{\detokenize{p01_u6563_u6587/_u8001_u820d-_u6d4e_u5357_u7684_u51ac_u5929:id1}}\label{\detokenize{p01_u6563_u6587/_u8001_u820d-_u6d4e_u5357_u7684_u51ac_u5929::doc}}
\begin{sphinxShadowBox}
\sphinxstyletopictitle{目录}
\begin{itemize}
\item {} 
\phantomsection\label{\detokenize{p01_u6563_u6587/_u8001_u820d-_u6d4e_u5357_u7684_u51ac_u5929:id11}}{\hyperref[\detokenize{p01_u6563_u6587/_u8001_u820d-_u6d4e_u5357_u7684_u51ac_u5929:id1}]{\sphinxcrossref{1   老舍-济南的冬天}}}
\begin{itemize}
\item {} 
\phantomsection\label{\detokenize{p01_u6563_u6587/_u8001_u820d-_u6d4e_u5357_u7684_u51ac_u5929:id12}}{\hyperref[\detokenize{p01_u6563_u6587/_u8001_u820d-_u6d4e_u5357_u7684_u51ac_u5929:id3}]{\sphinxcrossref{1.1   作品原文}}}

\item {} 
\phantomsection\label{\detokenize{p01_u6563_u6587/_u8001_u820d-_u6d4e_u5357_u7684_u51ac_u5929:id13}}{\hyperref[\detokenize{p01_u6563_u6587/_u8001_u820d-_u6d4e_u5357_u7684_u51ac_u5929:id4}]{\sphinxcrossref{1.2   写景手法}}}
\begin{itemize}
\item {} 
\phantomsection\label{\detokenize{p01_u6563_u6587/_u8001_u820d-_u6d4e_u5357_u7684_u51ac_u5929:id14}}{\hyperref[\detokenize{p01_u6563_u6587/_u8001_u820d-_u6d4e_u5357_u7684_u51ac_u5929:id5}]{\sphinxcrossref{1.2.1   1.基调统一,色彩和谐}}}

\item {} 
\phantomsection\label{\detokenize{p01_u6563_u6587/_u8001_u820d-_u6d4e_u5357_u7684_u51ac_u5929:id15}}{\hyperref[\detokenize{p01_u6563_u6587/_u8001_u820d-_u6d4e_u5357_u7684_u51ac_u5929:id6}]{\sphinxcrossref{1.2.2   2.景物层次,安排得当}}}

\item {} 
\phantomsection\label{\detokenize{p01_u6563_u6587/_u8001_u820d-_u6d4e_u5357_u7684_u51ac_u5929:id16}}{\hyperref[\detokenize{p01_u6563_u6587/_u8001_u820d-_u6d4e_u5357_u7684_u51ac_u5929:id7}]{\sphinxcrossref{1.2.3   3.远近大细,各得其宜}}}

\item {} 
\phantomsection\label{\detokenize{p01_u6563_u6587/_u8001_u820d-_u6d4e_u5357_u7684_u51ac_u5929:id17}}{\hyperref[\detokenize{p01_u6563_u6587/_u8001_u820d-_u6d4e_u5357_u7684_u51ac_u5929:id8}]{\sphinxcrossref{1.2.4   4.虚实手法,同时并用}}}

\item {} 
\phantomsection\label{\detokenize{p01_u6563_u6587/_u8001_u820d-_u6d4e_u5357_u7684_u51ac_u5929:id18}}{\hyperref[\detokenize{p01_u6563_u6587/_u8001_u820d-_u6d4e_u5357_u7684_u51ac_u5929:id9}]{\sphinxcrossref{1.2.5   5.适当点题,意义深远}}}

\item {} 
\phantomsection\label{\detokenize{p01_u6563_u6587/_u8001_u820d-_u6d4e_u5357_u7684_u51ac_u5929:id19}}{\hyperref[\detokenize{p01_u6563_u6587/_u8001_u820d-_u6d4e_u5357_u7684_u51ac_u5929:id10}]{\sphinxcrossref{1.2.6   6.山水画法,以大观小}}}

\end{itemize}

\end{itemize}

\end{itemize}
\end{sphinxShadowBox}


\section{1.1   作品原文}
\label{\detokenize{p01_u6563_u6587/_u8001_u820d-_u6d4e_u5357_u7684_u51ac_u5929:id3}}
对于一个在北平住惯的人,像我,冬天要是不刮风,便觉得是奇迹;济南的冬天是没有风声的。对于一个刚由伦敦回来的人,像我,冬天要能看得见日光,便觉得是怪事;济南的冬天是响晴的。自然,在热带的地方,日光是永远那么毒,响亮的天气,反有点叫人害怕。可是,在北中国的冬天,而能有温晴的天气,济南真得算个宝地。

设若单单是有阳光,那也算不了出奇。请闭上眼睛想:一个老城,有山有水,全在天底下晒着阳光,暖和安适地睡着,只等春风来把它们唤醒,这是不是个理想的境界?

小山整把济南围了个圈儿,只有北边缺着点口儿。这一圈小山在冬天特别可爱,好像是把济南放在一个小摇篮里,它们安静不动地低声地说:“你们放心吧,这儿准保暖和。”真的,济南的人们在冬天是面上含笑的。他们一看那些小山,心中便觉得有了着落,有了依靠。他们由天上看到山上,便不知不觉地想起:“明天也许就是春天了吧?这样的温暖,今天夜里山草也许就绿起来了吧?”就是这点幻想不能一时实现,他们也并不着急,因为有这样慈善的冬天,干啥还希望别的呢!

最妙的是下点小雪呀。看吧,山上的矮松越发的青黑,树尖上顶着一髻儿白花,好像日本看护妇。山尖全白了,给蓝天镶上一道银边。山坡上,有的地方雪厚点,有的地方草色还露着,这样,一道儿白,一道儿暗黄,给山们穿上一件带水纹的花衣;看着看着,这件花衣好像被风儿吹动,叫你希望看见一点更美的山的肌肤。等到快日落的时候,微黄的阳光斜射在山腰上,那点薄雪好像忽然害了羞,微微露出点粉色。就是下小雪吧,济南是受不住大雪的,那些小山太秀气!

古老的济南,城里那么狭窄,城外又那么宽敞,山坡上卧着些小村庄,小村庄的房顶上卧着点雪,对,这是张小水墨画,也许是唐代的名手画的吧。

那水呢,不但不结冰,倒反在绿萍上冒着点热气,水藻真绿,把终年贮蓄的绿色全拿出来了。天儿越晴,水藻越绿,就凭这些绿的精神,水也不忍得冻上,况且那些长枝的垂柳还要在水里照个影儿呢!看吧,由澄清的河水慢慢往上看吧,空中,半空中,天上,自上而下全是那么清亮,那么蓝汪汪的,整个的是块空灵的蓝水晶。这块水晶里,包着红屋顶,黄草山,像地毯上的小团花的灰色树影。这就是冬天的济南。{[}1{]}


\section{1.2   写景手法}
\label{\detokenize{p01_u6563_u6587/_u8001_u820d-_u6d4e_u5357_u7684_u51ac_u5929:id4}}

\subsection{1.2.1   1.基调统一,色彩和谐}
\label{\detokenize{p01_u6563_u6587/_u8001_u820d-_u6d4e_u5357_u7684_u51ac_u5929:id5}}
济南虽然地处北中国,但是冬天无大风而多日照,它在冬天最显著的气候特点是“温晴”(温暖晴朗)。文章紧紧抓住这一点,使笔下的种种景物跟这“温晴”天气紧密联系在一起,构成一幅温暖晴朗的济南冬天图景。文章写山,写水,写城,写人,都无不涂上一层温暖晴朗的色彩,就是写雪景,也仍然跟温暖有联系──因为暖和,所以“最妙的是下点小雪”;而同晴朗分不开──因为晴朗,所以有“等到快日落的时候,微黄的阳光斜射在山腰上,那点薄雪好像忽然害了羞,微微露出点粉色”的景致。

在文中,第二段主要写的是济南全景,第三、四段主要写的是济南的山色,第五段主要写的是济南的水上景色,那么,全文就是由这几幅互相联系而又相对独立的画图组成的长轴。而这幅长轴,也就靠这“温晴”的基调统一起来,给人以和谐一致的美感。


\subsection{1.2.2   2.景物层次,安排得当}
\label{\detokenize{p01_u6563_u6587/_u8001_u820d-_u6d4e_u5357_u7684_u51ac_u5929:id6}}
古老的济南,景色秀丽,素有“家家泉水,户户插柳”、“一城山色半城湖”的美誉。文章依照写景的先后层次,更好地把这些美好的景色展现于出来。文章首先鸟瞰全城,得其全貌(第二段),然后给人以那一城山色,雪后斜阳(第三、四段),最后才写那垂柳岸边,那“水不但不结冰,倒反在绿萍上冒着点热气”,而水藻越晴越绿的水上景色(第五段)。由大到小地写来,从山到水地写去,层次分明,脉络清晰。自然这是就各大层次来说的,各大层次的内部,又同中有异,如第二段的由写景而兼及写人,第三段的由写雪而兼及写晴,第五段的由写水面而兼及写天空。写来笔法活脱,不失参差错落之致。


\subsection{1.2.3   3.远近大细,各得其宜}
\label{\detokenize{p01_u6563_u6587/_u8001_u820d-_u6d4e_u5357_u7684_u51ac_u5929:id7}}
偌大的一个济南,在作者笔下,竟然可以放在一个由四面群山环抱而成的小小摇篮里,而水天一碧的宏伟景色,只不过是一块“空灵的蓝水晶”。这是景物的远者大者。再看,“树尖上顶着一髻儿白花,好像日本看护妇”,“水藻真绿,把终年贮蓄的绿色全拿出来了”。这是景物的近者细者。远景大景,使人视野开阔,顿感心旷神怡;近景小景,叫人近看谛听,更觉景象真切。而且远景大景,还可以冲破“不识庐山真面目,只缘身在此山中”的局限,而近景小景,又能够避免“只见树木不见森林”的弊病。古诗云:“远观山有色,近听水无声。”这是说的非远观不能看到高山居然有色,非近听无以觉出流水竟然无声。这说明,写景手法,远近大细,不可偏废。运用得宜,就可以兼收其效。

该文写景时,不但远近并用,大细兼行,而且往往是由近而远、由细而大,或由远而近、由大而细,写来衔接紧密,推进自然。比如第五段的写景,就是由近而远,由细而大的:先写水冒着点热气,再写水藻,再写垂柳,再写水面的上空以至于半空中、天空上。而第四段的写景,则是由远而近、由大而细的:先写城外,再写城外的山坡,再写山坡上的小村庄,再写小村庄的房顶上的雪。这种写法,既符合叙述的逻辑顺序,又适应读者的视觉需要。


\subsection{1.2.4   4.虚实手法,同时并用}
\label{\detokenize{p01_u6563_u6587/_u8001_u820d-_u6d4e_u5357_u7684_u51ac_u5929:id8}}
实写景物的形象,对景物描写来说,无疑是十分必要的,诸如文章中的“树尖上顶着一髻儿白花,好像日本看护妇”之类。但是,要不止于摹状,还要传神,就得更多地仰仗虚写的手法。因此,在作者笔下,冬天阳光照耀下的济南,就出现了“暖和安适地睡着,只等春风来把它们唤醒”的神情;一圈围城的小山,也就说出“你们放心吧,这儿准保暖和”的细语;薄雪会有“微微露出点粉色”的羞容;水藻会有“把终年贮蓄的绿色全拿出来了”的“精神”;而那水呢,对那水藻也就可以有一副“不忍得冻上”的和善心肠了。至于小雪覆盖不匀的山坡,要“给山们穿上一件带水纹的花衣”,“那些长枝的垂柳还要在水里照个影儿”,自然也是文章中虚写传神的佳句。


\subsection{1.2.5   5.适当点题,意义深远}
\label{\detokenize{p01_u6563_u6587/_u8001_u820d-_u6d4e_u5357_u7684_u51ac_u5929:id9}}
画之所以有题跋,原因之一是题跋可以使画本身蕴含的意义更为显豁。应该说,题跋是一幅画的一个有机的组成部分,虽然它并不是所画的景物的本身。同样,对所写的景物,作者出面直接点题,也是容许的,这些点明题旨的话,不是可有可无的。该文点题得法,寥寥数语,便收到画龙点睛的效果。比如说,文章在描写了小山雪景之后,突然掉转笔锋,作者以评论者的身份,说起点题话来:“就是下小雪吧,济南是受不住大雪的,那些小山太秀气!”这话,既可以说是在所描绘的画面之外,又可以说是在所描绘的画面之中,因为它是画面所本有而又有点不甚明了的。一经点出,济南下点小雪(不能是大雪)的妙处,也就跃然纸上了。

题不可不点,也不可滥点,本文点题恰到好处。最后一句“这就是冬天的济南”,令人读起来有意犹未尽、话犹未了之感,引发读者更深远的思考,这也许正是作者使文章戛然而止的原因吧。


\subsection{1.2.6   6.山水画法,以大观小}
\label{\detokenize{p01_u6563_u6587/_u8001_u820d-_u6d4e_u5357_u7684_u51ac_u5929:id10}}
描绘济南的大地,老舍先生所用的是“以大观小”的中国山水画的构图取景方法。作者展开想像的翅膀飞上济南的云天俯瞰大地,然后对济南大地作了简笔的写意描绘。画城,不画它的东西南北,“一个老城,有山有水,全在天底下晒着阳光,暖和安适地睡着,只等春风来把它们唤醒”(注:此句中的山是济南城中的山)。一些琐碎的细部都被略去了,画的只是冬天济南城秀美的睡态,留下充分的余地让读者去联想、想像,进行艺术的再创造。画山,不画它的上下左右,“小山整把济南围了个圈儿,只有北边缺着点口儿”。一起笔就抓住了景物的主要特征,紧接着就引导读者展开艺术的联想和想像:“这一圈小山在冬天特别可爱,好像是把济南放在一个小摇篮里,它们安静不动地低声地说:‘你们放心吧,这儿准保暖和。’”借这种联想、想像,使画面活灵飞动起来。画人,不画人的男女老少,不但如国画一样略去耳鼻眉目,连形体也完全略去,而只画了济南冬天人物情态的最主要的特征:“济南的人们在冬天是面上含笑的。”和城与山,浑然构成一幅完美的图画。


\chapter{1   艾青-大堰河——我的保姆}
\label{\detokenize{p01_u6563_u6587/_u827e_u9752-_u5927_u5830_u6cb3_u2014_u2014_u6211_u7684_u4fdd_u59c6:id1}}\label{\detokenize{p01_u6563_u6587/_u827e_u9752-_u5927_u5830_u6cb3_u2014_u2014_u6211_u7684_u4fdd_u59c6::doc}}
\begin{sphinxShadowBox}
\sphinxstyletopictitle{目录}
\begin{itemize}
\item {} 
\phantomsection\label{\detokenize{p01_u6563_u6587/_u827e_u9752-_u5927_u5830_u6cb3_u2014_u2014_u6211_u7684_u4fdd_u59c6:id7}}{\hyperref[\detokenize{p01_u6563_u6587/_u827e_u9752-_u5927_u5830_u6cb3_u2014_u2014_u6211_u7684_u4fdd_u59c6:id1}]{\sphinxcrossref{1   艾青-大堰河——我的保姆}}}
\begin{itemize}
\item {} 
\phantomsection\label{\detokenize{p01_u6563_u6587/_u827e_u9752-_u5927_u5830_u6cb3_u2014_u2014_u6211_u7684_u4fdd_u59c6:id8}}{\hyperref[\detokenize{p01_u6563_u6587/_u827e_u9752-_u5927_u5830_u6cb3_u2014_u2014_u6211_u7684_u4fdd_u59c6:id3}]{\sphinxcrossref{1.1   作品原文}}}

\item {} 
\phantomsection\label{\detokenize{p01_u6563_u6587/_u827e_u9752-_u5927_u5830_u6cb3_u2014_u2014_u6211_u7684_u4fdd_u59c6:id9}}{\hyperref[\detokenize{p01_u6563_u6587/_u827e_u9752-_u5927_u5830_u6cb3_u2014_u2014_u6211_u7684_u4fdd_u59c6:id4}]{\sphinxcrossref{1.2   创作背景}}}

\item {} 
\phantomsection\label{\detokenize{p01_u6563_u6587/_u827e_u9752-_u5927_u5830_u6cb3_u2014_u2014_u6211_u7684_u4fdd_u59c6:id10}}{\hyperref[\detokenize{p01_u6563_u6587/_u827e_u9752-_u5927_u5830_u6cb3_u2014_u2014_u6211_u7684_u4fdd_u59c6:id5}]{\sphinxcrossref{1.3   作品鉴赏}}}

\item {} 
\phantomsection\label{\detokenize{p01_u6563_u6587/_u827e_u9752-_u5927_u5830_u6cb3_u2014_u2014_u6211_u7684_u4fdd_u59c6:id11}}{\hyperref[\detokenize{p01_u6563_u6587/_u827e_u9752-_u5927_u5830_u6cb3_u2014_u2014_u6211_u7684_u4fdd_u59c6:id6}]{\sphinxcrossref{1.4   名家点评}}}

\end{itemize}

\end{itemize}
\end{sphinxShadowBox}


\section{1.1   作品原文}
\label{\detokenize{p01_u6563_u6587/_u827e_u9752-_u5927_u5830_u6cb3_u2014_u2014_u6211_u7684_u4fdd_u59c6:id3}}
大堰河——我的保姆

大堰河,是我的保姆。

她的名字就是生她的村庄的名字,

她是童养媳,

大堰河,是我的保姆。//

我是地主的儿子;

也是吃了大堰河的奶而长大了的

大堰河的儿子。

大堰河以养育我而养育她的家,

而我,是吃了你的奶而被养育了的,

大堰河啊,我的保姆。//

大堰河,今天我看到雪使我想起了你:

你的被雪压着的草盖的坟墓,

你的关闭了的故居檐头的枯死的瓦菲,

你的被典押了的一丈平方的园地,

你的门前的长了青苔的石椅,

大堰河,今天我看到雪使我想起了你。//

你用你厚大的手掌把我抱在怀里,抚摸我;

在你搭好了灶火之后,

在你拍去了围裙上的炭灰之后,

在你尝到饭已煮熟了之后,

在你把乌黑的酱碗放到乌黑的桌子上之后,

在你补好了儿子们的为山腰的荆棘扯破的衣服之后,

在你把小儿被柴刀砍伤了的手包好之后,

在你把夫儿们的衬衣上的虱子一颗颗地掐死之后,

在你拿起了今天的第一颗鸡蛋之后,

你用你厚大的手掌把我抱在怀里,抚摸我。//

我是地主的儿子,

在我吃光了你大堰河的奶之后,

我被生我的父母领回到自己的家里。

啊,大堰河,你为什么要哭?//

我做了生我的父母家里的新客了!

我摸着红漆雕花的家具,

我摸着父母的睡床上金色的花纹,

我呆呆地看着檐头的我不认得的“天伦叙乐”的匾,

我摸着新换上的衣服的丝的和贝壳的纽扣,

我看着母亲怀里的不熟识的妹妹,

我坐着油漆过的安了火钵的炕凳,

我吃着碾了三番的白米的饭,

但,我是这般忸怩不安!因为我

我做了生我的父母家里的新客了。//

大堰河,为了生活,

在她流尽了她的乳汁之后,

她就开始用抱过我的两臂劳动了;

她含着笑,洗着我们的衣服,

她含着笑,提着菜篮到村边的结冰的池塘去,

她含着笑,切着冰屑悉索的萝卜,

她含着笑,用手掏着猪吃的麦糟,

她含着笑,扇着炖肉的炉子的火,

她含着笑,背了团箕到广场上去,

晒好那些大豆和小麦,

大堰河,为了生活,

在她流尽了她的乳液之后,

她就用抱过我的两臂,劳动了。//

大堰河,深爱着她的乳儿;

在年节里,为了他,忙着切那冬米的糖,

为了他,常悄悄地走到村边的她的家里去,

为了他,走到她的身边叫一声“妈”,

大堰河,把他画的大红大绿的关云长

贴在灶边的墙上,

大堰河,会对她的邻居夸口赞美她的乳儿;

大堰河曾做了一个不能对人说的梦:

在梦里,她吃着她的乳儿的婚酒,

坐在辉煌的结彩的堂上,

而她的娇美的媳妇亲切的叫她“婆婆”

……//

大堰河,深爱着她的乳儿!

大堰河,在她的梦没有做醒的时候已死了。

她死时,乳儿不在她的旁侧,

她死时,平时打骂她的丈夫也为她流泪,

五个儿子,个个哭得很悲,

她死时,轻轻地呼着她的乳儿的名字,

大堰河,已死了,

她死时,乳儿不在她的旁侧。//

大堰河,含泪的去了!

同着四十几年的人世生活的凌侮,

同着数不尽的奴隶的凄苦,

同着四块钱的棺材和几束稻草,

同着几尺长方的埋棺材的土地,

同着一手把的纸钱的灰,

大堰河,她含泪的去了。//

这是大堰河所不知道的:

她的醉酒的丈夫已死去,

大儿做了土匪,

第二个死在炮火的烟里,

第三,第四,第五

在师傅和地主的叱骂声里过着日子。

而我,我是在写着给予这不公道的世界的咒语。

当我经了长长的漂泊回到故土时,

在山腰里,田野上,

兄弟们碰见时,是比六七年前更要亲密!

这,这是为你,静静地睡着的大堰河

所不知道的啊!//

大堰河,今天,你的乳儿是在狱里,

写着一首呈给你的赞美诗,

呈给你黄土下紫色的灵魂,

呈给你拥抱过我的直伸着的手,

呈给你吻过我的唇,

呈给你泥黑的温柔的脸颜,

呈给你养育了我的乳房,

呈给你的儿子们,我的兄弟们,

呈给大地上一切的,

我的大堰河般的保姆和她们的儿子,

呈给爱我如爱她自己的儿子般的大堰河。//

大堰河,

我是吃了你的奶而长大了的

你的儿子,

我敬你

爱你!

一九三三年一月十四日,雪朝


\section{1.2   创作背景}
\label{\detokenize{p01_u6563_u6587/_u827e_u9752-_u5927_u5830_u6cb3_u2014_u2014_u6211_u7684_u4fdd_u59c6:id4}}
1932年,诗人因加入左翼美术家联盟被捕,以“宣传与三民主义不相容主义”罪被判入狱6年。在狱中他写下了这首《大堰河——我的保姆》。{[}2{]}


\section{1.3   作品鉴赏}
\label{\detokenize{p01_u6563_u6587/_u827e_u9752-_u5927_u5830_u6cb3_u2014_u2014_u6211_u7684_u4fdd_u59c6:id5}}
《大堰河,我的保姆》是艾青的成名之作。这是一个地主阶级叛逆的儿子献给他的真正母亲——中国大地善良而不幸的普通农妇的颂歌。

这首诗感情真挚深切。诗中反复陈述:“大堰河,是我的保姆”,诗人是地主的儿子,长在“大堰河”的怀中,吮吸着她的乳汁,这不仅养育了诗人和身体,也养育了诗人的感情。诗人深深领受了她的爱,及至到了上学的年龄离开养母回到亲生父母身边的时候,他感到父母的陌生,更感到养母的对他的重要。养母正直、善良、朴素的品格影响了诗人的一生。这首诗从头到尾,始终围绕“我”与“她”的关系来写,他对大堰河深厚的感情,都表现在娓娓动情的陈述之中,他在监狱里,看见了雪就想到大堰河“被雪压着的草盖的坟墓”,想起她的故居园地,想起她对他的关怀和爱……于是他用他的深情的诗,表现了大堰河的具体劳作情景,也写了她心灵深处的感情波纹,就连她美丽的梦境,也同对乳儿的“幸福命运”的祝愿融合在一起。有了这样的真情,这样的心灵,才使这位劳动妇女形象更加崇高、完美,所以诗人要把热烈的颂扬,“呈给大地上一切的/我的大堰河般的保姆和他们的儿子/呈给爱我如爱她自己的儿子般的大堰河”。这样就使“大堰河”以某种象征意义,升华为永远与山河、村庄同在的人民的化身,或者说是中国农民的化身。

艾青在《大堰河,我的保姆》开始表现他诗作的艺术特色,他首先是从“感觉”出发,像印象派画家那么重视感觉和感受,而且注意主观情感对感觉的渗入与融合。并在二者的融合中产生出多层次的联想,创造出既是清晰的,又具有广阔象征意义的视觉形象。诗总是具体的、有着鲜明形象的,如这首诗写大堰河的劳作,写大堰河的笑,写大堰河的爱和死。都呈现可视可感的立体的意象符号附加形容。最后叠句排比旬的运用,如“呈给你黄土下紫色的灵魂/呈给你拥抱过我的直伸着的手/呈给你吻过我的唇。/呈给你泥黑的温柔的脸颜/呈给你蒜育我的乳房……”具体的描写,保证语言的形象性,这也是艾青诗的艺术魅力的奥秘所在,他后来的诗作,更自觉地将它发扬光大了。

这是一首献给保姆大堰河的诗篇。诗人叙述了这位普通中国妇女平凡而坎坷、不幸的一生,表达了对这位伟大母亲由衷的感恩之情。大堰河,也是千千万万中国母亲的代表,正是这片如同慈母一样宽阔的土地和这个伟大的祖国,尽管她受尽欺辱,满身疮痍,历尽沧桑,然而却永远不失母性和母爱伟大的光辉诗歌饱含深情,反复咏唱,如泣如诉。


\section{1.4   名家点评}
\label{\detokenize{p01_u6563_u6587/_u827e_u9752-_u5927_u5830_u6cb3_u2014_u2014_u6211_u7684_u4fdd_u59c6:id6}}
现代文学家茅盾:“用沉郁的笔调细写了乳娘兼女佣(《大堰河》)的生活痛苦”。(《中国现代文学管窥》)

中国作家协会会员张同吾:它像一颗光华熠熠的新星,出现在30年代的中国诗坛上;它以深沉隽永的情思,在广大读者的心田里镌刻着久远而常新的记忆。(《张同吾文集》){[}6{]}

现代文艺理论家、诗人胡风:“至于《大堰河——我的保姆》,在这里有了一个用乳汁用母爱喂养别人的孩子,用劳力用忠诚服侍别人的农妇的形象,乳儿的作者用着朴素的真实的言语对这形象呈诉了切切的爱心。在这里他提出了对于‘这不公道的世界’的诅咒,告白了他和被侮辱的兄弟们比以前‘更要亲密’。虽然全篇流着私情地温暖,但他和我们之间已没有了难越的界限了。”(《通三统:一种文学史实验》)


\chapter{1   范仲淹-岳阳楼记}
\label{\detokenize{p01_u6563_u6587/_u8303_u4ef2_u6df9-_u5cb3_u9633_u697c_u8bb0:id1}}\label{\detokenize{p01_u6563_u6587/_u8303_u4ef2_u6df9-_u5cb3_u9633_u697c_u8bb0::doc}}
\begin{sphinxShadowBox}
\sphinxstyletopictitle{目录}
\begin{itemize}
\item {} 
\phantomsection\label{\detokenize{p01_u6563_u6587/_u8303_u4ef2_u6df9-_u5cb3_u9633_u697c_u8bb0:id10}}{\hyperref[\detokenize{p01_u6563_u6587/_u8303_u4ef2_u6df9-_u5cb3_u9633_u697c_u8bb0:id1}]{\sphinxcrossref{1   范仲淹-岳阳楼记}}}
\begin{itemize}
\item {} 
\phantomsection\label{\detokenize{p01_u6563_u6587/_u8303_u4ef2_u6df9-_u5cb3_u9633_u697c_u8bb0:id11}}{\hyperref[\detokenize{p01_u6563_u6587/_u8303_u4ef2_u6df9-_u5cb3_u9633_u697c_u8bb0:id3}]{\sphinxcrossref{1.1   作品原文}}}

\item {} 
\phantomsection\label{\detokenize{p01_u6563_u6587/_u8303_u4ef2_u6df9-_u5cb3_u9633_u697c_u8bb0:id12}}{\hyperref[\detokenize{p01_u6563_u6587/_u8303_u4ef2_u6df9-_u5cb3_u9633_u697c_u8bb0:id4}]{\sphinxcrossref{1.2   词句注释}}}

\item {} 
\phantomsection\label{\detokenize{p01_u6563_u6587/_u8303_u4ef2_u6df9-_u5cb3_u9633_u697c_u8bb0:id13}}{\hyperref[\detokenize{p01_u6563_u6587/_u8303_u4ef2_u6df9-_u5cb3_u9633_u697c_u8bb0:id5}]{\sphinxcrossref{1.3   白话译文}}}

\item {} 
\phantomsection\label{\detokenize{p01_u6563_u6587/_u8303_u4ef2_u6df9-_u5cb3_u9633_u697c_u8bb0:id14}}{\hyperref[\detokenize{p01_u6563_u6587/_u8303_u4ef2_u6df9-_u5cb3_u9633_u697c_u8bb0:id6}]{\sphinxcrossref{1.4   创作背景}}}

\item {} 
\phantomsection\label{\detokenize{p01_u6563_u6587/_u8303_u4ef2_u6df9-_u5cb3_u9633_u697c_u8bb0:id15}}{\hyperref[\detokenize{p01_u6563_u6587/_u8303_u4ef2_u6df9-_u5cb3_u9633_u697c_u8bb0:id7}]{\sphinxcrossref{1.5   文学赏析}}}

\item {} 
\phantomsection\label{\detokenize{p01_u6563_u6587/_u8303_u4ef2_u6df9-_u5cb3_u9633_u697c_u8bb0:id16}}{\hyperref[\detokenize{p01_u6563_u6587/_u8303_u4ef2_u6df9-_u5cb3_u9633_u697c_u8bb0:id8}]{\sphinxcrossref{1.6   名家点评}}}

\item {} 
\phantomsection\label{\detokenize{p01_u6563_u6587/_u8303_u4ef2_u6df9-_u5cb3_u9633_u697c_u8bb0:id17}}{\hyperref[\detokenize{p01_u6563_u6587/_u8303_u4ef2_u6df9-_u5cb3_u9633_u697c_u8bb0:id9}]{\sphinxcrossref{1.7   作者简介}}}

\end{itemize}

\end{itemize}
\end{sphinxShadowBox}

《岳阳楼记》是北宋文学家范仲淹于庆历六年九月十五日(1046年10月17日)应好友巴陵郡太守滕子京之请为重修岳阳楼而创作的一篇散文。这篇文章通过写岳阳楼的景色,以及阴雨和晴朗时带给人的不同感受,揭示了“不以物喜,不以己悲”的古仁人之心,也表达了自己“先天下之忧而忧,后天下之乐而乐”的爱国爱民情怀。文章超越了单纯写山水楼观的狭境,将自然界的晦明变化、风雨阴晴和“迁客骚人”的“览物之情”结合起来写,从而将全文的重心放到了纵议政治理想方面,扩大了文章的境界。全文记叙、写景、抒情、议论融为一体,动静相生,明暗相衬,文词简约,音节和谐,用排偶章法作景物对比,成为杂记中的创新。


\section{1.1   作品原文}
\label{\detokenize{p01_u6563_u6587/_u8303_u4ef2_u6df9-_u5cb3_u9633_u697c_u8bb0:id3}}
岳阳楼记1

庆历四年春2,滕子京谪守巴陵郡3。越明年4,政通人和5,百废具兴6。乃重修岳阳楼7,增其旧制8,刻唐贤今人诗赋于其上9。属予作文以记之10。

予观夫巴陵胜状11,在洞庭一湖。衔远山12,吞长江13,浩浩汤汤14,横无际涯15;朝晖夕阴,气象万千16。此则岳阳楼之大观也17,前人之述备矣18。然则北通巫峡19,南极潇湘20,迁客骚人21,多会于此22,览物之情,得无异乎23?

若夫淫雨霏霏24,连月不开25,阴风怒号26,浊浪排空27;日星隐曜28,山岳潜形29;商旅不行30,樯倾楫摧31;薄暮冥冥32,虎啸猿啼。登斯楼也,则有去国怀乡33,忧谗畏讥34,满目萧然35,感极而悲者矣36。

至若春和景明37,波澜不惊38,上下天光39,一碧万顷;沙鸥翔集,锦鳞游泳40;岸芷汀兰41,郁郁青青42。而或长烟一空43,皓月千里44,浮光跃金45,静影沉璧46,渔歌互答47,此乐何极48!登斯楼也,则有心旷神怡49,宠辱偕忘50,把酒临风51,其喜洋洋者矣52。

嗟夫53!予尝求古仁人之心54,或异二者之为55。何哉?不以物喜,不以己悲56;居庙堂之高则忧其民57;处江湖之远则忧其君58。是进亦忧,退亦忧。然则何时而乐耶?其必曰:“先天下之忧而忧,后天下之乐而乐”乎59。噫!微斯人,吾谁与归60?

时六年九月十五日。{[}1{]}


\section{1.2   词句注释}
\label{\detokenize{p01_u6563_u6587/_u8303_u4ef2_u6df9-_u5cb3_u9633_u697c_u8bb0:id4}}
1.记:一种文体。可以写景、叙事,多为议论。但目的是为了抒发作者的情怀和政治抱负(阐述作者的某些观念)。

2.庆历四年:公元1044年。庆历,宋仁宗赵祯的年号。文章末句中的“时六年”,指庆历六年(1046),点明作文的时间。

3.滕子京谪(zhé)守巴陵郡:滕子京降职任岳州太守。滕子京,名宗谅,子京是他的字,范仲淹的朋友。谪守,把被革职的官吏或犯了罪的人充发到边远的地方。在这里作为动词被贬官,降职解释。谪,封建王朝官吏降职或远调。守,做郡的长官。汉朝“守某郡”,就是做某郡的太守;宋朝废郡称州,应说“知某州”。巴陵郡,即岳州,治所在今湖南岳阳,这里沿用古称。“守巴陵郡”就是“守岳州”。

4.越明年:有三说,其一指庆历五年,为针对庆历四年而言;其二指庆历六年,此“越”为经过、经历;其三指庆历七年,针对作记时间庆历六年而言。

5.政通人和:政事顺利,百姓和乐。政,政事。通,通顺。和,和乐。这是赞美滕子京的话。

6.百废具兴:各种荒废的事业都兴办起来了。百,不是确指,形容其多。废,这里指荒废的事业。具,通“俱”,全,皆。兴,复兴。

7.乃:于是。

8.制:规模。

9.唐贤今人:唐代和当代名人。贤,形容词作名词用。

10.属(zhǔ):通“嘱”,嘱托、嘱咐。予:我。作文:写文章。以:连词,用来。记:记述。

11.夫:那。胜状:胜景,好景色。

12.衔:包含。

13.吞:吞吐。

14.浩浩汤汤(shāng):水波浩荡的样子。汤汤,水流大而急。

15.横无际涯:宽阔无边。横,广远。际涯,边。际专指陆地边界,涯专指水的边界)。

16.朝晖夕阴,气象万千:或早或晚(一天里)阴晴多变化。朝,在早晨,名词做状语。晖,日光。气象,景象。万千,千变万化。

17.此则岳阳楼之大观也:这就是岳阳楼的雄伟景象。此,这。则,就。大观,雄伟景象。

18.前人之述备矣:前人的记述很详尽了。前人之述,指上面说的“唐贤今人诗赋”。备,详尽,完备。矣,语气词“了”。之,助词,的。

19.然则:虽然如此,那么。

20.南极潇湘:南面直到潇水、湘水。潇水是湘水的支流。湘水流入洞庭湖。南,向南。极,尽,最远到达。

21.迁客:谪迁的人,指降职远调的人。骚人:诗人。战国时屈原作《离骚》,因此后人也称诗人为骚人。

22.多:大多。会:聚集。

23.览物之情,得无异乎:看到自然景物而引发的情感,怎能不有所不同呢?览,观看,欣赏。得无……乎,大概……吧。

24.若夫:用在一段话的开头以引起下文。下文的“至若”,同此。“若夫”近似“像那”。“至若”近似“至于”。淫雨,连绵不断的雨。霏霏,雨或雪(繁密)的样子。

25.开:(天气)放晴。

26.阴,阴冷。

27.排空,冲向天空。

28.日星隐曜(yào):太阳和星星隐藏起光辉。曜(不为耀,古文中以此当作日光),光辉,日光。

29.山岳潜形:山岳隐没了形体。岳,高大的山。潜,隐没。形,形迹。

30.行:走,此指前行。

31.樯(qiáng)倾楫(jí)摧:桅杆倒下,船桨折断。樯,桅杆。楫,船桨。倾,倒下。摧,折断。

32.薄暮冥冥:傍晚天色昏暗。薄,迫近。冥冥,昏暗的样子。

33.则,就。有:产生……的(情感)。

34.去国怀乡,忧谗畏讥:离开国都,怀念家乡,担心(人家)说坏话,惧怕(人家)批评指责。去,离开。国,国都,指京城。忧,担忧。谗,谗言。畏,害怕,惧怕。讥,嘲讽。

35.萧然:凄凉冷落的样子。

36.感极,感慨到了极点。而,连词,表顺接。

37.至若春和景明:至于到了春天气候暖和,阳光普照。至若,至于。春和,春风和煦。景,日光。明,明媚。

38.波澜不惊:湖面平静,没有惊涛骇浪。惊,这里有“起”“动”的意思。

39.上下天光,一碧万顷:天色湖面光色交映,一片碧绿,广阔无边。一,一片。万顷,极言其广。

40.沙鸥翔集,锦鳞游泳:沙鸥时而飞翔,时而停歇,美丽的鱼在水中游来游去。沙鸥,沙洲上的鸥鸟。翔集,时而飞翔,时而停歇。集,栖止,鸟停息在树上。锦鳞,指美丽的鱼。鳞,代指鱼。游泳,或浮或沉。游,贴着水面游。泳,潜入水里游。

41.岸芷(zhǐ)汀(tīng)兰:岸上的小草,小洲上的兰花。芷,香草的一种。汀,小洲,水边平地。

42.郁郁:形容草木茂盛。

43.而或长烟一空:有时大片烟雾完全消散。或,有时。长,大片。一,全。空,消散。

44.皓月千里:皎洁的月光照耀千里。

45.浮光跃金:湖水波动时,浮在水面上的月光闪耀起金光。这是描写月光照耀下的水波。有些版本作“浮光耀金”。

46.静影沉璧:湖水平静时,明月映入水中,好似沉下一块玉璧。这里是写无风时水中的月影。璧,圆形正中有孔的玉。沉璧,像沉入水中的璧玉。

47.互答:一唱一和。

48.何极:哪有穷尽。何,怎么。极,穷尽。

49.心旷神怡:心情开朗,精神愉快。旷,开阔。怡,愉快。

50.宠辱偕忘:荣耀和屈辱一并都忘了。宠,荣耀。辱,屈辱。偕,一起,一作“皆”。

51.把酒临风:端酒面对着风,就是在清风吹拂中端起酒来喝。把,持,执。临,面对。

52.洋洋:高兴的样子。

53.嗟(jiē)夫:唉。嗟夫为两个词,皆为语气词。

54.尝:曾经。求:探求。古仁人:古时品德高尚的人。心:思想(感情心思)。

55.或异二者之为:或许不同于(以上)两种心情。或,近于“或许”“也许”的意思,表委婉口气。为,这里指心理活动,即两种心情。二者,这里指前两段的“悲”与“喜”。

56.不以物喜,不以己悲:不因为外物好坏和自己得失而或喜或悲(此句为互文)。以,因为。

57.居庙堂之高则忧其民:在朝中做官就担忧百姓。居庙堂之高:处在高高的庙堂上,意为在朝中做官。庙,宗庙。堂,殿堂。庙堂:指朝廷。下文的“进”,即指“居庙堂之高”。

58.处江湖之远则忧其君:处在僻远的地方做官就为君主担忧。处江湖之远:处在偏远的江湖间,意思是不在朝廷上做官。之:定语后置的标志。是,这样。下文的“退”,即指“处江湖之远”。

59.先天下之忧而忧,后天下之乐而乐:在天下人担忧之前先担忧,在天下人享乐之后才享乐。先,在……之前。后,在……之后。其,指“古仁人”。

60.微斯人,吾谁与归:(如果)没有这种人,那我同谁一道呢?微,(如果)没有。斯人,这种人(指前文的“古仁人”)。谁与归,就是“与谁归”。归,归依。{[}1{]}{[}2-3{]}


\section{1.3   白话译文}
\label{\detokenize{p01_u6563_u6587/_u8303_u4ef2_u6df9-_u5cb3_u9633_u697c_u8bb0:id5}}
庆历四年春天,滕子京降职到岳州做太守。到了第二年,政务顺利,百姓和乐,各种荒废了的事业都兴办起来了。于是重新修建岳阳楼,扩展它原有的规模,把唐代名人家和今人的诗赋刻在上面。嘱咐我写一篇文章来记述这件事。

我看那巴陵郡的美景,全在洞庭湖上。洞庭湖包含远方的山脉,吞吐着长江的流水,浩浩荡荡,宽阔无边,清晨湖面上撒满阳光、傍晚又是一片阴暗,景物的变化无穷无尽。这就是岳阳楼雄伟壮丽的景象。前人对这些景象的记述已经很详尽了,虽然这样,那么这里北面通向巫峡,南面直到潇水、湘江,被降职远调的人员和吟诗作赋的诗人,大多在这里聚会,观赏这里的自然景物而触发的感情,大概会有所不同吧?

像那连绵细雨纷纷而下,整月不放晴的时候,阴冷的风怒吼着,浑浊的波浪冲向天空;日月星辰隐藏起光辉,山岳也隐没了形迹;商人和旅客无法通行,桅杆倒下,船桨折断;傍晚时分天色昏暗,只听到老虎的吼叫和猿猴的悲啼。这时登上这座楼,就会产生被贬官离开京城,怀念家乡,担心人家说坏话,惧怕人家讥讽的心情,再抬眼望去尽是萧条冷落的景象,一定会感慨万千而十分悲伤了。

至于春风和煦、阳光明媚时,湖面波平浪静,天色与湖光相接,一片碧绿,广阔无际;沙洲上的白鸥,时而飞翔时而停歇,美丽的鱼儿或浮或沉;岸上的小草,小洲上的兰花,香气浓郁,颜色青翠。有时湖面上的大片烟雾完全消散,皎洁的月光一泻千里,有时湖面上微波荡漾,浮动的月光闪着金色;有时湖面波澜不起,静静的月影像沉在水中的玉璧。渔夫的歌声响起了,一唱一和,这种乐趣真是无穷无尽!这时登上这座楼,就会感到胸怀开阔,精神愉快,光荣和屈辱一并忘了,在清风吹拂中端起酒杯痛饮,那心情真是快乐高兴极了。

唉!我曾经探求古时品德高尚的人的思想感情,他们或许不同于以上两种心情,这是什么缘故呢?是因为古时品德高尚的人不因外物好坏和自己得失而或喜或悲。在朝廷做官就为百姓忧虑;不在朝廷做官而处在僻远的江湖中间就为国君忧虑。这样他们进入朝廷做官也忧虑,退处江湖也忧虑。虽然这样,那么他们什么时候才快乐呢?那一定要说“在天下人忧愁之前先忧愁,在天下人快乐以后才快乐”吧?唉!如果没有这种人,我同谁一路呢?

写于庆历六年九月十五日。


\section{1.4   创作背景}
\label{\detokenize{p01_u6563_u6587/_u8303_u4ef2_u6df9-_u5cb3_u9633_u697c_u8bb0:id6}}
这篇文章写于庆历六年(1046)。范仲淹生活在北宋王朝内忧外患的年代,对内阶级矛盾日益突出,对外契丹和西夏虎视眈眈。为了巩固政权,改善这一处境,以范仲淹为首的政治集团开始进行改革,后人称之为“庆历新政”。但改革触犯了封建大地主阶级保守派的利益,遭到了他们的强烈反对。而皇帝改革的决心也不坚定,在以太后为首的保守官僚集团的压迫下,改革以失败告终。“庆历新政”失败后,范仲淹又因得罪了宰相吕夷简,范仲淹贬放河南邓州,这篇文章便是写于邓州,而非写于岳阳楼。

按照宋代人的习惯,写“记”以及散文一类的文章,本人并不一定要身在其地,主要是通过这种文章记录事情、写景、记人来抒发作者的感情或见解,借景抒情,托物言志。古时,邀人作记通常要附带一份所记之物的样本,也就是画卷或相关文献之类的资料,以供作记之人参考。滕子京虽然被贬岳州,但他在任期间,做了三件政绩工程,希望能够取得朝廷的谅解。重修岳阳楼便是其中之一,完成于庆历五年(1045)。滕子京为了提高其政绩工程的知名度,赠给范仲淹《洞庭晚秋图》,并向他求作两记,一则就是《岳阳楼记》,另一则是《偃虹堤记》。《岳阳楼记》所述内容有实物可查,然而《偃虹堤记》则无迹可寻。但是在《偃虹堤记》中,范仲淹也同样将偃虹堤描写得具体翔实,相较岳阳楼毫不逊色。因而,便引发了少数学者关于范仲淹写《岳阳楼记》时是否去过岳阳楼的争议。{[}4-6{]}


\section{1.5   文学赏析}
\label{\detokenize{p01_u6563_u6587/_u8303_u4ef2_u6df9-_u5cb3_u9633_u697c_u8bb0:id7}}
《岳阳楼记》全文有三百六十八字,共六段。

文章开头即切入正题,叙述事情的本末缘起。以“庆历四年春”点明时间起笔,格调庄重雅正;说滕子京为“谪守”,已暗喻对仕途沉浮的悲慨,为后文抒情设伏。下面仅用“政通人和,百废具兴”八个字,写出滕子京的政绩,引出重修岳阳楼和作记一事,为全篇文字的导引。

第二段,格调振起,情辞激昂。先总说“巴陵胜状,在洞庭一湖”,设定下文写景范围。以下“衔远山,吞长江”寥寥数语,写尽洞庭湖之大观胜概。一“衔”一“吞”,有气势。“浩浩汤汤,横无际涯”,极言水波壮阔;“朝晖夕阴,气象万千”,概说阴晴变化,简练而又生动。前四句从空间角度,后两句从时间角度,写尽了洞庭湖的壮观景象。“前人之述备矣”一句承前启后,并回应前文“唐贤今人诗赋”一语。这句话既是谦虚,也暗含转机,经“然则”一转,引出新的意境,由单纯写景,到以情景交融的笔法来写“迁客骚人”的“览物之情”,从而构出全文的主体。

三、四两段是两个排比段,并行而下,一悲一喜,一暗一明,像两股不同的情感之流,传达出景与情互相感应的两种截然相反的人生情境。

第三段写览物而悲者。以“若夫”起笔,意味深长。这是一个引发议论的词,又表明了虚拟的情调,而这种虚拟又是对无数实境的浓缩、提炼和升华,颇有典型意义。“若夫”以下描写了一种悲凉的情境,由天气的恶劣写到人心的凄楚。这里用四字短句,层层渲染,渐次铺叙。淫雨、阴风、浊浪构成了主景,不但使日星无光,山岳藏形,也使商旅不前;或又值暮色沉沉、“虎啸猿啼”之际,令过往的“迁客骚人”有“去国怀乡”之慨、“忧谗畏讥”之惧、“感极而悲”之情。

第四段写览物而喜者。以“至若”领起,打开了一个阳光灿烂的画面。“至若”尽管也是列举性的语气,但从音节上已变得高亢嘹亮,格调上已变得明快有力。下面的描写,虽然仍为四字短句,色调却为之一变,绘出春风和畅、景色明丽、水天一碧的良辰美景。更有鸥鸟在自由翱翔,鱼儿在欢快游荡,连无知的水草兰花也充满活力。作者以极为简练的笔墨,描摹出一幅湖光春色图,读之如在眼前。值得注意的是,这一段的句式、节奏与上一段大体相仿,却也另有变奏。“而或”一句就进一步扩展了意境,增强了叠加咏叹的意味,把“喜洋洋”的气氛推向高潮,而“登斯楼也”的心境也变成了“宠辱偕忘”的超脱和“把酒临风”的挥洒自如。

第五段是全篇的重心,以“嗟夫”开启,兼有抒情和议论的意味。作者在列举了悲喜两种情境后,笔调突然激扬,道出了超乎这两者之上的一种更高的理想境界,那就是“不以物喜,不以己悲”。感物而动,因物悲喜虽然是人之常情,但并不是做人的最高境界。古代的仁人,就有坚定的意志,不为外界条件的变化动摇。无论是“居庙堂之高”还是“处江湖之远”,忧国忧民之心不改,“进亦忧,退亦忧”。这似乎有悖于常理,有些不可思议。作者也就此拟出一问一答,假托古圣立言,发出了“先天下之忧而忧,后天下之乐而乐”的誓言,曲终奏雅,点明了全篇的主旨。“噫!微斯人,吾谁与归”一句结语,“如怨如慕,如泣如诉”,悲凉慷慨,一往情深,令人感喟。文章最后标明写作时间,与篇首照应。

这篇文章表现作者虽身居江湖,心忧国事,虽遭迫害,仍不放弃理想的顽强意志,同时,也是对被贬战友的鼓励和安慰。《岳阳楼记》的著名,是因为它的思想境界崇高。和它同时的另一位文学家欧阳修在为他写的碑文中说,他从小就有志于天下,常自诵曰:“士当先天下之忧而忧,后天下之乐而乐也。”可见《岳阳楼记》末尾所说的“先天下之忧而忧,后天下之乐而乐”,是范仲淹一生行为的准则。孟子说:“达则兼善天下,穷则独善其身”。这已成为封建时代许多士大夫的信条。范仲淹写这篇文章的时候正贬官在外,“处江湖之远”,本来可以采取独善其身的态度,落得清闲快乐,但他提出正直的士大夫应立身行一的准则,认为个人的荣辱升迁应置之度外,“不以物喜,不以己悲”要“先天下之忧而忧,后天下之乐而乐”,勉励自己和朋友,这是难能可贵的。这两句话所体现的精神,那种吃苦在前,享乐在后的品质,无疑仍有教育意义。

就艺术而论,《岳阳楼记》也是一篇优秀的文章。

第一,岳阳楼之大观,前人已经说尽了,再重复那些老话没有意思。范仲淹就是采取了换一个新的角度,找一个新的题目,另说他的一套。文章的题目是“岳阳楼记”,却巧妙地避开楼不写,而去写洞庭湖,写登楼的迁客骚人看到洞庭湖的不同景色时产生的不同感情,以衬托最后一段所谓“古仁人之心”。范仲淹的别出心裁,让人佩服。

第二,记事、写景、抒情和议论交融在一篇文章中,记事简明,写景铺张,抒情真切,议论精辟。议论的部分字数不多,但有统帅全文的作用,所以有人说这是一篇独特的议论文。《岳阳楼记》的议论技巧,确实有值得借鉴的地方。

第三,这篇文章的语言很有特色。它虽然是一篇散文,却穿插了许多四言的对偶句,如“日星隐曜,山岳潜形。”“沙鸥翔集,锦鳞游泳。”“长烟一空,皓月千里;浮光跃金,静影沉璧。”这些骈句为文章增添了色彩。作者锤炼字句的功夫也很深,如“衔远山,吞长江”这两句的“衔”字、“吞”字,恰切地表现了洞庭湖浩瀚的气势。“不以物喜,不以己悲”,简洁的八个字,像格言那样富有启示性。“先天下之忧而忧,后天下之乐而乐”,把丰富的意义熔铸到短短的两句话中,字字有力。

全文记叙、写景、抒情、议论融为一体,动静相生,明暗相衬,文词简约,音节和谐,用排偶章法作景物对比,成为杂记中的创新。{[}7-9{]}


\section{1.6   名家点评}
\label{\detokenize{p01_u6563_u6587/_u8303_u4ef2_u6df9-_u5cb3_u9633_u697c_u8bb0:id8}}
北宋陈师道《后山诗话》:范文正公为《岳阳楼记》,用对语说时景,世以为奇。尹师鲁读之曰:传奇体尔。传奇,唐裴铏所著小说也。

明代孙绪《无用闲谈》:范文正公《岳阳楼记》,或谓其用赋体,殆未深考耳。此是学吕温《三堂记》,体制如出一轴。然《岳阳楼记》闳远超越,青出于蓝矣。夫以文正千载人物,而乃肯学吕温,亦见君子不以人废言之盛心也。

清代金圣叹《天下才子必读书》:中间悲喜二大段,只是借来翻出后丈优乐耳,不然便是赋体类。一肚皮圣贤心地,圣贤学问,发而为才子文章。

清代顾兖《文章规范百家坪注》:楼迁斋评:首尾布置与中间状物之妙,不可及矣。然最妙处在临末断遗一转语。乃知此老胸襟度量,直与岳阳洞庭同其广。

清代蔡世远《古文雅正》:前半设局造句,犹是文人手笔。末段直达胸臆,非文正公不足以当之。或问史臣吕本中及朱文公,皆以文正公为宋朝人物第一,何也?曰:魏文会大矣,而本领徽不及;派公诚矣,而规局徽不及。尧舜君民之念,无日不存于中心,事如白日青天;公诚绝伦超群也。

清代林云铭《古文折义》:题是记岳阳楼,任他高手,少不得要说此楼前此如何倾坏,如何狭小,然后叙增修之劳。再写楼外佳景。以为滕公此举大有益于登临已耳。文正却把这些话头点过,便尽情阁起,单就迁客骚人登楼异情处,转入古仁人用心,遂将平日胸中致君泽民、先忧后乐大本领一齐揭出。盖滕公以司谏谪守巴陵,居庙堂之高者忽处江湖之远。其忧谗畏讥之念,宠辱之怀,抚景感触,不能自遣,情所必至。若知念及君民之当忧,自有不暇于为物喜,为己悲者。篇首提出“谪守”二字,本是此意。妙在借他方之迁客骚人,闲闲点缀,不即不离。谓之为子京说法可也,谓之自述其怀抱可也,即谓之遍告天下后世君予俱应如此存心,亦无不可也。嘻,此其所以为文,公正之文欤。

清代吴楚材、吴调候《古文观止》:岳阳楼大观,已被前人写尽,先生更不赘述,止将登楼者览物之情,写出悲喜二意。只是翻出后文忧乐一段正论。以圣贤忧国忧民心地,发而为文幸,非先生其孰能之?

清代过珙《古文评注》:首尾布置与中间状物之妙不可及矣。尤妙在入后忧乐一段,见得惟贤者而后有真忧,亦惟贤者而后有真乐。乐不以忧而废,忧不以乐而忘。此虽文正自负之词,而期望子京,隐然言外。必如是始得斯文本旨。

清代余城《重订古文释义新编》:通体俱在谪守上着笔,确是子京重修击阳楼记,一字不肯苟下。圣贤经济,才子文章,于此可兼得矣。

清代浦起龙《古文眉诠》:先忧后乐两言,先生生平所持诵也。缘情设景,借题引合,想见万物一体胸襟。

清代唐德宜《古文翼》:撇过岳阳之景,专写览物之情,引起忧乐二意,又从忧乐写出绝大本领。从来名公作记,未有若此篇之正大堂皇者,可想见文公一生节概。

清代李扶九原编、黄仁黼重订《古文笔法百篇》:入手即将题点过,而“谪守”二字,已伏一篇之意。盖谪者多悲而少喜,故将景物随写一笔,即便昂开,提出主意,隐对子京。切定洞庭畅发两段,得宽题走窄境法。末段提出仁人之用心,以规勉之,何其正大。不知此即文正公自己写照也。公为秀才时,尝言“士君子当先天下之忧而忧,后天下之乐而乐。”不觉因上悲喜,即便吐露,而忧乐正与悲喜对也。亦岂己所不能而貌为大言乎?楼记发此大议,可谓小中见大之文。看其一结,虚托闪开,作想慕不已之情,冷冷而住,不自任而矜张,不打照子京而寡迹,尤为巧妙绝伦。至中间两对,已早开有明八股之风矣。黼按君子之所以异于人者,以其存心也。心可即境而存,心不可随境而变。其所存于中者大,斯其所遇于外者小矣。文正此记,前半为岳阳写景绘情,经营惨淡,已到十分。而其中或悲或喜,处处隐对子京,即处处从请守著想。故末以忧乐二字,易悲喜二字,归到仁人身上。见得境虽变,心不与之俱变;心所存,道即与之俱存。出忧其民,处忧其君,仁人之心,自有其所以异者存也。通幅不矜才,不使气,使自己胸襟显得磊磊落落,正大而光明。非其存于中者大,而能若是乎?

清代尤焴《可斋杂稿》:文正《岳阳楼记》,精切高古,而欧公犹不以文章许之。然要皆磊磊落落,确实典重,凿凿乎如五谷之疗饥,与世之图章绘句、不根事实者,不可同年而语也。


\section{1.7   作者简介}
\label{\detokenize{p01_u6563_u6587/_u8303_u4ef2_u6df9-_u5cb3_u9633_u697c_u8bb0:id9}}
范仲淹(989-1052),字希文,北宋思想家、政治家、文学家。大中祥符八年(1015),进士及第。庆历三年(1043),参与庆历新政,提出了十项改革主张。庆历五年(1045),新政受挫,范仲淹被贬出京。皇祐四年(1052),溘然长逝,享年六十四岁,谥号文正,世称范文正公。范仲淹文学成就突出,其“先天下之忧而忧,后天下之乐而乐”思想,对后世影响深远。有《范文正公文集》。


\chapter{1   茅盾-白杨礼赞}
\label{\detokenize{p01_u6563_u6587/_u8305_u76fe-_u767d_u6768_u793c_u8d5e:id1}}\label{\detokenize{p01_u6563_u6587/_u8305_u76fe-_u767d_u6768_u793c_u8d5e::doc}}
\begin{sphinxShadowBox}
\sphinxstyletopictitle{目录}
\begin{itemize}
\item {} 
\phantomsection\label{\detokenize{p01_u6563_u6587/_u8305_u76fe-_u767d_u6768_u793c_u8d5e:id5}}{\hyperref[\detokenize{p01_u6563_u6587/_u8305_u76fe-_u767d_u6768_u793c_u8d5e:id1}]{\sphinxcrossref{1   茅盾-白杨礼赞}}}
\begin{itemize}
\item {} 
\phantomsection\label{\detokenize{p01_u6563_u6587/_u8305_u76fe-_u767d_u6768_u793c_u8d5e:id6}}{\hyperref[\detokenize{p01_u6563_u6587/_u8305_u76fe-_u767d_u6768_u793c_u8d5e:id3}]{\sphinxcrossref{1.1   作品原文}}}

\item {} 
\phantomsection\label{\detokenize{p01_u6563_u6587/_u8305_u76fe-_u767d_u6768_u793c_u8d5e:id7}}{\hyperref[\detokenize{p01_u6563_u6587/_u8305_u76fe-_u767d_u6768_u793c_u8d5e:id4}]{\sphinxcrossref{1.2   词语注释}}}

\end{itemize}

\end{itemize}
\end{sphinxShadowBox}

《白杨礼赞》是现代作家茅盾于1941年所写的一篇散文。作者以西北黄土高原上“参天耸立,不折不挠,对抗着西北风”的白杨树,来象征坚韧、勤劳的北方农民,歌颂他们在民族解放斗争中的朴实、坚强和力求上进的精神,同时对于那些“贱视民众,顽固的倒退的人们”也投出了辛辣的嘲讽。文章立意高远,形象鲜明,结构严谨,语言简练。


\section{1.1   作品原文}
\label{\detokenize{p01_u6563_u6587/_u8305_u76fe-_u767d_u6768_u793c_u8d5e:id3}}
白杨树实在不是平凡的,我赞美白杨树!

汽车在望不到边际的高原上奔驰,扑入你的视野2的,是黄绿错综的一条大毡子。黄的是土,未开垦的处女土,几十万年前由伟大的自然力堆积成功的黄土高原的外壳;绿的呢,是人类劳力战胜自然的成果,是麦田。和风吹送,翻起了一轮一轮的绿波——这时你会真心佩服昔人所造的两个字“麦浪”,若不是妙手偶得,便确是经过锤炼的语言的精华。黄与绿主宰着,无边无垠,坦荡如砥3,这时如果不是宛若4并肩的远山的连峰提醒了你(这些山峰凭你的肉眼来判断,就知道是在你脚底下的),你会忘记了汽车是在高原上行驶。这时你涌起来的感想也许是“雄壮”,也许是“伟大”,诸如此类的形容词;然而同时你的眼睛也许觉得有点倦怠,你对当前的“雄壮”或“伟大”闭了眼,而另一种的味儿在你心头潜滋暗长5了——“单调”。可不是?单调,有一点儿吧?

然而刹那间,要是你猛抬眼看见了前面远远有一排——不,或者甚至只是三五株,一株,傲然地耸立,像哨兵似的树木的话,那你的恹恹6欲睡的情绪又将如何?我那时是惊奇地叫了一声的。

那就是白杨树,西北极普通的一种树,然而实在不是平凡的一种树。

那是力争上游的一种树,笔直的干,笔直的枝。它的干呢,通常是丈把高,像是加以人工似的,一丈以内绝无旁枝。它所有的丫枝呢,一律向上,而且紧紧靠拢,也像是加以人工似的,成为一束,绝无横斜逸出7。它的宽大的叶子也是片片向上,几乎没有斜生的,更不用说倒垂了;它的皮,光滑而有银色的晕圈8,微微泛出淡青色。这是虽在北方的风雪的压迫下却保持着倔强挺立的一种树。哪怕只有碗来粗细罢,它却努力向上发展,高到丈许,二丈,参天耸立,不折不挠,对抗着西北风。

这就是白杨树,西北极普通的一种树,然而决不是平凡的树!

它没有婆娑9的姿态,没有屈曲盘旋的虬枝10,也许你要说它不美丽,──如果美是专指“婆娑”或“横斜逸出”之类而言,那么白杨树算不得树中的好女子;但是它却是伟岸11,正直,朴质,严肃,也不缺乏温和,更不用提它的坚强不屈与挺拔,它是树中的伟丈夫!当你在积雪初融的高原上走过,看见平坦的大地上傲然挺立这么一株或一排白杨树,难道你觉得树只是树,难道你就不想到它的朴质,严肃,坚强不屈,至少也象征了北方的农民;难道你竟一点也不联想到,在敌后的广大土地上,到处有坚强不屈,就像这白杨树一样傲然挺立的守卫他们家乡的哨兵!难道你又不更远一点想到这样枝枝叶叶靠紧团结,力求上进的白杨树,宛然象征了今天在华北平原纵横决荡12用血写出新中国历史的那种精神和意志。

白杨不是平凡的树。它在西北极普遍,不被人重视,就跟北方农民相似;它有极强的生命力,磨折不了,压迫不倒,也跟北方的农民相似。我赞美白杨树,就因为它不但象征了北方的农民,尤其象征了今天我们民族解放斗争中所不可缺的朴质,坚强,以及力求上进的精神。

让那些看不起民众,贱视民众,顽固的倒退的人们去赞美那贵族化的楠木13(那也是直干秀颀14的),去鄙视这极常见,极易生长的白杨罢,但是我要高声赞美白杨树!

(原载《文艺阵地》月刊第6卷第3期,1941年3月10日出版)


\section{1.2   词语注释}
\label{\detokenize{p01_u6563_u6587/_u8305_u76fe-_u767d_u6768_u793c_u8d5e:id4}}
1.礼赞:崇敬和赞美。

2.视野:视力所及的范围。

3.坦荡如砥(dǐ):平坦得像磨刀石一样。

4.宛若:很像,简直就是。

5.潜滋暗长:暗暗地不知不觉地生长。滋,生长。

6.恹恹(yānyān):困倦的样子。

7.横斜逸出:意思是,(树枝)从树干的旁边斜伸出来。

8.晕(yùn)圈:模模糊糊的圈。

9.婆娑(suō):树木的枝叶随风飘荡,像舞蹈一样的姿态。

10.虬(qiú)枝:像龙一样盘旋的枝条。虬,传说中的一种龙。

11.伟岸:魁梧,高大。

12.纵横决荡:纵横驰骋,冲杀突击。

13.楠(nán)木:常绿乔木,木质坚实,是贵重的木材。

14.秀颀(qí):美而高。颀,高大的意思。


\chapter{1   荀子-劝学}
\label{\detokenize{p01_u6563_u6587/_u8340_u5b50-_u529d_u5b66:id1}}\label{\detokenize{p01_u6563_u6587/_u8340_u5b50-_u529d_u5b66::doc}}
\begin{sphinxShadowBox}
\sphinxstyletopictitle{目录}
\begin{itemize}
\item {} 
\phantomsection\label{\detokenize{p01_u6563_u6587/_u8340_u5b50-_u529d_u5b66:id5}}{\hyperref[\detokenize{p01_u6563_u6587/_u8340_u5b50-_u529d_u5b66:id1}]{\sphinxcrossref{1   荀子-劝学}}}
\begin{itemize}
\item {} 
\phantomsection\label{\detokenize{p01_u6563_u6587/_u8340_u5b50-_u529d_u5b66:id6}}{\hyperref[\detokenize{p01_u6563_u6587/_u8340_u5b50-_u529d_u5b66:id3}]{\sphinxcrossref{1.1   作品原文}}}

\item {} 
\phantomsection\label{\detokenize{p01_u6563_u6587/_u8340_u5b50-_u529d_u5b66:id7}}{\hyperref[\detokenize{p01_u6563_u6587/_u8340_u5b50-_u529d_u5b66:id4}]{\sphinxcrossref{1.2   词句注释}}}

\end{itemize}

\end{itemize}
\end{sphinxShadowBox}

《劝学》是战国时期思想家、文学家荀子创作的一篇论说文,是《荀子》一书的首篇。文章较系统地论述了学习的理论和方法,分别从学习的重要性、学习的态度以及学习的内容和方法等方面,全面而深刻地论说了有关学习的问题。全文可分四段,第一段阐明学习的重要性,第二段讲正确的学习态度,第三段讲学习的内容,第四段讲学习应当善始善终。全文说理深入,结构严谨,代表了先秦论说文成熟阶段的水平。


\section{1.1   作品原文}
\label{\detokenize{p01_u6563_u6587/_u8340_u5b50-_u529d_u5b66:id3}}
君子曰1:学不可以已2。

青,取之于蓝3,而青于蓝;冰,水为之,而寒于水。木直中绳4,輮以为轮5,其曲中规6。虽有槁暴7,不复挺者8,輮使之然也。故木受绳则直9,金就砺则利10,君子博学而日参省乎己11,则知明而行无过矣12。

故不登高山,不知天之高也;不临深溪,不知地之厚也;不闻先王之遗言13,不知学问之大也。干、越、夷、貉之子14,生而同声,长而异俗,教使之然也。诗曰:“嗟尔君子,无恒安息。靖共尔位,好是正直。神之听之,介尔景福15。”神莫大于化道,福莫长于无祸。

吾尝终日而思矣,不如须臾之所学也16;吾尝跂而望矣17,不如登高之博见也18。登高而招,臂非加长也,而见者远;顺风而呼,声非加疾也19,而闻者彰20。假舆马者21,非利足也22,而致千里;假舟楫者,非能水也23,而绝江河24。君子生非异也25,善假于物也。

南方有鸟焉,名曰蒙鸠26,以羽为巢,而编之以发,系之苇苕27,风至苕折,卵破子死。巢非不完也,所系者然也。西方有木焉,名曰射干28,茎长四寸,生于高山之上,而临百仞之渊,木茎非能长也,所立者然也。蓬生麻中,不扶而直;白沙在涅,与之俱黑29。兰槐之根是为芷30,其渐之滫31,君子不近,庶人不服32。其质非不美也,所渐者然也33。故君子居必择乡,游必就士,所以防邪辟而近中正也34。

物类之起,必有所始。荣辱之来,必象其德。肉腐出虫,鱼枯生蠹35。怠慢忘身,祸灾乃作。强自取柱36,柔自取束37。邪秽在身,怨之所构38。施薪若一,火就燥也,平地若一,水就湿也。草木畴生39,禽兽群焉,物各从其类也。是故质的张40,而弓矢至焉;林木茂,而斧斤至焉41;树成荫,而众鸟息焉。醯酸42,而蜹聚焉43。故言有招祸也,行有招辱也,君子慎其所立乎!

积土成山,风雨兴焉;积水成渊,蛟龙生焉;积善成德,而神明自得,圣心备焉。故不积跬步44,无以至千里;不积小流,无以成江海。骐骥一跃45,不能十步;驽马十驾46,功在不舍47。锲而舍之48,朽木不折;锲而不舍,金石可镂49。蚓无爪牙之利,筋骨之强,上食埃土,下饮黄泉,用心一也。蟹六跪而二螯50,非蛇鳝93之穴无可寄托者,用心躁也。

是故无冥冥之志者51,无昭昭之明52;无惛惛之事者,无赫赫之功。行衢道者不至,事两君者不容。目不能两视而明,耳不能两听而聪。螣蛇无足而飞53,鼫鼠五技而穷54。《诗》曰:“尸鸠在桑,其子七兮。淑人君子,其仪一兮。其仪一兮,心如结兮55!”故君子结于一也56。

昔者瓠巴鼓瑟57,而沈鱼出听58;伯牙鼓琴59,而六马仰秣60。故声无小而不闻,行无隐而不形。玉在山而草木润,渊生珠而崖不枯61。为善不积邪62?安有不闻者乎?

学恶乎始?恶乎终?曰:其数则始乎诵经63,终乎读礼;其义则始乎为士,终乎为圣人,真积力久则入,学至乎没而后止也。故学数有终,若其义则不可须臾舍也。为之,人也;舍之,禽兽也。故书者,政事之纪也;诗者,中声之所止也;礼者,法之大分64,类之纲纪也。故学至乎礼而止矣。夫是之谓道德之极。礼之敬文也,乐之中和也,诗书之博也,春秋之微也,在天地之间者毕矣。

君子之学也,入乎耳,著乎心,布乎四体,形乎动静。端而言,蝡而动65,一可以为法则。小人之学也,入乎耳,出乎口;口耳之间,则四寸耳,曷足以美七尺之躯哉!古之学者为己,今之学者为人。君子之学也,以美其身;小人之学也,以为禽犊。故不问而告谓之傲66,问一而告二谓之囋67。傲、非也,囋、非也;君子如向矣68。

学莫便乎近其人。礼乐法而不说,诗书故而不切,春秋约而不速。方其人之习君子之说69,则尊以遍矣,周于世矣。故曰:学莫便乎近其人。

学之经莫速乎好其人,隆礼次之。上不能好其人,下不能隆礼,安特将学杂识志,顺诗书而已耳70。则末世穷年,不免为陋儒而已。将原先王,本仁义71,则礼正其经纬蹊径也72。若挈裘领73,诎五指而顿之74,顺者不可胜数也。不道礼宪75,以诗书为之,譬之犹以指测河也,以戈舂黍也76,以锥飡壶也77,不可以得之矣。故隆礼,虽未明,法士也;不隆礼,虽察辩,散儒也。

问楛者78,勿告也;告楛者,勿问也;说楛者,勿听也。有争气者79,勿与辩也。故必由其道至,然后接之;非其道则避之。故礼恭,而后可与言道之方;辞顺,而后可与言道之理;色从而后可与言道之致80。故未可与言而言,谓之傲;可与言而不言,谓之隐81;不观气色而言,谓之瞽82。故君子不傲、不隐、不瞽,谨顺其身83。诗曰:“匪交匪舒,天子所予84。”此之谓也。

百发失一,不足谓善射;千里跬步不至,不足谓善御;伦类不通85,仁义不一,不足谓善学。学也者,固学一之也。一出焉,一入焉,涂巷之人也;其善者少,不善者多,桀纣盗跖也86;全之尽之,然后学者也。

君子知夫不全不粹之不足以为美也,故诵数以贯之87,思索以通之,为其人以处之,除其害者以持养之。使目非是无欲见也88,使耳非是无欲闻也,使口非是无欲言也,使心非是无欲虑也。及至其致好之也,目好之五色,耳好之五声89,口好之五味90,心利之有天下。是故权利不能倾也,群众不能移也,天下不能荡也。生乎由是,死乎由是,夫是之谓德操。德操然后能定,能定然后能应91。能定能应,夫是之谓成人92。天见其明,地见其光,君子贵其全也。{[}1{]}


\section{1.2   词句注释}
\label{\detokenize{p01_u6563_u6587/_u8340_u5b50-_u529d_u5b66:id4}}
1.君子:指有学问有修养的人。

2.学不可以已(yǐ):学习不能停止。

3.青取之于蓝:靛青,从蓝草中取得。青,靛青,一种染料。蓝,蓼蓝,一年生草本植物,叶子含蓝汁,可以做蓝色染料。

4.中(zhòng)绳:(木材)合乎拉直的墨线。绳,墨线。

5.輮(róu):通“煣”,古代用火烤使木条弯曲的一种工艺。

6.规:圆规,画圆的工具。

7.虽有(yòu)槁暴(pù):即使又晒干了。有,通“又”。槁,枯。暴,同“曝”,晒干。

8.挺:直。

9.受绳:用墨线量过。

10.金:指金属制的刀剑等。就砺:拿到磨刀石上去磨。砺,磨刀石。就,动词,接近,靠近。

11.日参(cān)省(xǐng)乎己:每天对照反省自己。参,一译检验,检查;二译同“叁”,多次。省,省察。乎,介词,于。博学:广泛地学习。日:每天。

12.知(zhì):通“智”,智慧。明:明达。行无过:行为没有过错。

13.遗言:犹古训。

14.干(hán):同“邗”,古国名,在今江苏扬州东北,春秋时被吴国所灭而成为吴邑,此指代吴国。夷:中国古代居住在东部的民族。貉(mò):通“貊”,中国古代居住在东北部的民族。

15.“嗟尔君子”六句:引诗见《诗经·小雅·小明》。靖,安。共,通“供”。介,给予。景,大。

16.须臾(yú):片刻,一会儿。

17.跂(qǐ):踮起脚后跟。

18.博见:看见的范围广,见得广。

19.疾:声音宏大。

20.彰:明显,清楚。这里指听得更清楚。

21.假:凭借,利用。舆:车厢,这里指车。

22.利足:脚走得快。

23.水:游泳。

24.绝:横渡。

25.生(xìng)非异:本性(同一般人)没有差别。生,通“性”,天赋,资质。

26.蒙鸠:即鹪鹩,俗称黄脰鸟,又称巧妇鸟,全身灰色*,有斑,常取茅苇一毛一毳为巢。 4) (5) (6)滫(xiu朽音):淘米水,此引为脏水、臭水。

27.苕(tiáo):芦苇的花穗。

28.射(yè)干:又名乌扇,一种草本植物,根入药,茎细长,多生于山崖之间,形似树木,所以荀子称它为“木”,其实是一种草。一说“木”为“草”字之误。

29.“蓬生麻中”四句:草长在麻地里,不用扶持也能挺立住,白沙混进了黑土里,就会变得和土一样黑。比喻生活在好的环境里,也能成为好人。蓬,蓬草。麻,麻丛。涅,黑色染料。《集解》无“白沙在涅与之俱黑”八字,据《尚书·洪范》“时人斯其惟皇之极”《正义》引文补。

30.兰槐:香草名,又叫白芷,开白花,味香。古人称其苗为“兰”,称其根为“芷”。

31.渐(jiān):浸。滫(xiǔ):泔水,已酸臭的淘米水。此引为脏水、臭水。

32.服:穿戴。

33.所渐者然也:被熏陶、影响的情况就是这样的。然,这样。

34.邪辟:品行不端的人。中正:正直之士。

35.蠹(dù):蛀蚀器物的虫子。

36.强自取柱:谓物性过硬则反易折断。柱,通“祝”(王引之说),折断。《大戴礼记·劝学》作“折”。

37.柔自取束:柔弱的东西自己导致约束。

38.构:结,造成。

39.畴:通“俦”,类。

40.质:箭靶。的(dì):箭靶的中心。

41.斤:斧子。

42.醯(xī):本意指醋。

43.蜹(ruì):飞虫名,属蚊类。

44.跬(kuǐ):行走时两脚之间的距离,等于现在所说的一步、古人所说的半步。步:古人说一步,指左右脚都向前迈一次的距离,等于现在的两步。

45.骐(qí)骥(jì):骏马,千里马。

46.驽马十驾:劣马拉车连走十天也能到达。驽马,劣马。驾,古代马拉车时,早晨套一上车,晚上卸去。套车叫驾,所以这里用“驾”指代马车一天的行程。十驾就是套十次车,指十天的行程。此指千里的路程。

47.舍:舍弃。指不放弃行路。

48.锲(qiè):用刀雕刻。

49.镂(lòu):原指在金属上雕刻,泛指雕刻。

50.蟹六跪而二螯(áo):螃蟹有六只爪子,两个钳子。六跪,六条腿。蟹实际上是八条腿。跪,蟹脚。一说,海蟹后面的两条腿只能划水,不能用来走路或自卫,所以不能算在“跪”里面。螯,螃蟹等节肢动物身前的大爪,形如钳。

51.冥冥:昏暗不明的样子,形容专心致志、埋头苦干。下文“惛惛”与此同义。

52.昭昭:明白的样子。

53.螣(téng)蛇:古代传说中的一种能飞的神蛇。

54.鼫(shí)鼠:原作“梧鼠”,据《大戴礼记·劝学》改。鼫鼠能飞但不能飞上屋面,能爬树但不能爬到树梢,能游泳但不能渡过山谷,能挖洞但不能藏身,能奔跑但不能追过人,所以说它“五技而穷”。穷:窘困。

55.“尸鸠在桑”六句:引诗见《诗经·曹风·鸤鸠》。仪,通“义”。

56.结:结聚不散开,比喻专心一致,坚定不移。

57.瓠(hù)巴:楚国人,善于弹瑟。

58.沈:同“沉”。《集解》作“流”,据《大戴礼记·劝学》改。

59.伯牙:古代善于弹琴的人。

60.六马:古代天子之车驾用六匹马拉;此指拉车之马。仰秣:《淮南子·说山训》高诱注:“仰秣,仰头吹吐,谓马笑也。”一说“秣”通“末”,头。

61.崖:岸边。

62.邪:同“耶”,疑问语气词。

63.数:术,即方法、途径,引申为“科目”。

64.大分:大的原则、界限。

65.蝡(rú):同“蠕”,微动。

66.傲:浮躁。

67.囋:形容言语繁碎。

68.向:通“响”,回音。

69.方:通“仿”,仿效。

70.顺:通“训”,解释词义。

71.原、本:均作动词,指追溯本源。

72.经纬:直线与横线,这里指道路。另辟蹊径:小路,这里泛指道路。

73.挈:提,拎。裘:皮衣。

74.诎:通“屈”,弯曲。顿:抖动,整理。

75.道:由,遵。礼宪:礼法。

76.舂:把谷类的皮捣掉。黍:黍子,谷类。

77.飡:即“餐”,吃。壶:古代盛食物的器皿,这里指饭。

78.楛:原指器物粗糙恶劣,这里是恶劣的意思,即指不合礼义。

79.争气:指意气用事。

80.致:极致,最高的境界。

81.隐:有意隐瞒。

82.瞽:盲目从事。

83.谨顺其身:指君子谨慎修养自己,做到不傲、不隐、不瞽,待人接物恰到好处。

84.“匪交匪舒”二句:语本《诗经·小雅·采菽》。匪,非,不。交,通“侥”,侥幸急躁。舒,缓,慢。予,通“与”,赞成。

85.伦:与“类”同义,指类别。

86.桀纣:夏朝和商朝的亡国之君。盗跖:古代一个名叫跖的大盗。

87.数:数说,与“诵”意义相近。

88.是:指全而粹合乎礼仪之美。

89.五声:宫、商、角、徵、羽,这里指美妙的音乐。

90.五味:甜、酸、苦、辣、咸,这里指美味。

91.应:指对外界事物的应变能力。

92.成人:全人,完美的人。{[}2{]}

93.蛇鳝:异文“蛇蟮”。{[}3{]}


\chapter{1   郁达夫-古都的秋}
\label{\detokenize{p01_u6563_u6587/_u90c1_u8fbe_u592b-_u53e4_u90fd_u7684_u79cb:id1}}\label{\detokenize{p01_u6563_u6587/_u90c1_u8fbe_u592b-_u53e4_u90fd_u7684_u79cb::doc}}
\begin{sphinxShadowBox}
\sphinxstyletopictitle{目录}
\begin{itemize}
\item {} 
\phantomsection\label{\detokenize{p01_u6563_u6587/_u90c1_u8fbe_u592b-_u53e4_u90fd_u7684_u79cb:id5}}{\hyperref[\detokenize{p01_u6563_u6587/_u90c1_u8fbe_u592b-_u53e4_u90fd_u7684_u79cb:id1}]{\sphinxcrossref{1   郁达夫-古都的秋}}}
\begin{itemize}
\item {} 
\phantomsection\label{\detokenize{p01_u6563_u6587/_u90c1_u8fbe_u592b-_u53e4_u90fd_u7684_u79cb:id6}}{\hyperref[\detokenize{p01_u6563_u6587/_u90c1_u8fbe_u592b-_u53e4_u90fd_u7684_u79cb:id3}]{\sphinxcrossref{1.1   作品原文}}}

\item {} 
\phantomsection\label{\detokenize{p01_u6563_u6587/_u90c1_u8fbe_u592b-_u53e4_u90fd_u7684_u79cb:id7}}{\hyperref[\detokenize{p01_u6563_u6587/_u90c1_u8fbe_u592b-_u53e4_u90fd_u7684_u79cb:id4}]{\sphinxcrossref{1.2   词语注释}}}

\end{itemize}

\end{itemize}
\end{sphinxShadowBox}


\section{1.1   作品原文}
\label{\detokenize{p01_u6563_u6587/_u90c1_u8fbe_u592b-_u53e4_u90fd_u7684_u79cb:id3}}
秋天,无论在什么地方的秋天,总是好的;可是啊,北国的秋,却特别地来得清,来得静,来得悲凉。我的不远千里,要从杭州赶上青岛,更要从青岛赶上北平来的理由,也不过想饱尝一尝这“秋”,这故都的秋味。

江南,秋当然也是有的,但草木凋得慢,空气来得润,天的颜色显得淡,并且又时常多雨而少风;一个人夹在苏州上海杭州,或厦门香港广州的市民中间,混混沌沌地过去,只能感到一点点清凉,秋的味,秋的色,秋的意境与姿态,总看不饱,尝不透,赏玩不到十足。秋并不是名花,也并不是美酒,那一种半开、半醉的状态,在领略秋的过程上,是不合适的。

不逢北国之秋,已将近十余年了。在南方每年到了秋天,总要想起陶然亭(1)的芦花,钓鱼台(2)的柳影,西山(3)的虫唱,玉泉(4)的夜月,潭柘寺(5)的钟声。在北平即使不出门去吧,就是在皇城人海之中,租人家一椽(6)破屋来住着,早晨起来,泡一碗浓茶,向院子一坐,你也能看得到很高很高的碧绿的天色,听得到青天下驯鸽的飞声。从槐树叶底,朝东细数着一丝一丝漏下来的日光,或在破壁腰中,静对着像喇叭似的牵牛花(朝荣)的蓝朵,自然而然地也能够感觉到十分的秋意。说到了牵牛花,我以为以蓝色或白色者为佳,紫黑色次之,淡红色最下。最好,还要在牵牛花底,叫长着几根疏疏落落的尖细且长的秋草,使作陪衬。

北国的槐树,也是一种能使人联想起秋来的点缀。像花而又不是花的那一种落蕊,早晨起来,会铺得满地。脚踏上去,声音也没有,气味也没有,只能感出一点点极微细极柔软的触觉。扫街的在树影下一阵扫后,灰土上留下来的一条条扫帚的丝纹,看起来既觉得细腻,又觉得清闲,潜意识下并且还觉得有点儿落寞(7),古人所说的梧桐一叶而天下知秋(8)的遥想,大约也就在这些深沉的地方。

秋蝉的衰弱的残声,更是北国的特产,因为北平处处全长着树,屋子又低,所以无论在什么地方,都听得见它们的啼唱。在南方是非要上郊外或山上去才听得到的。这秋蝉的嘶叫,在北方可和蟋蟀耗子一样,简直像是家家户户都养在家里的家虫。

还有秋雨哩,北方的秋雨,也似乎比南方的下得奇,下得有味,下得更像样。

在灰沉沉的天底下,忽而来一阵凉风,便息列索落地下起雨来了。一层雨过,云渐渐地卷向了西去,天又晴了,太阳又露出脸来了,着(9)着很厚的青布单衣或夹袄的都市闲人,咬着烟管,在雨后的斜桥影里,上桥头树底下去一立,遇见熟人,便会用了缓慢悠闲的声调,微叹着互答着地说:

“唉,天可真凉了——”(这了字念得很高,拖得很长。)

“可不是吗?一层秋雨一层凉了!”

北方人念阵字,总老像是层字,平平仄仄起来(10),这念错的歧韵,倒来得正好。

北方的果树,到秋天,也是一种奇景。第一是枣子树,屋角,墙头,茅房边上,灶房门口,它都会一株株地长大起来。像橄榄又像鸽蛋似的这枣子颗儿,在小椭圆形的细叶中间,显出淡绿微黄的颜色的时候,正是秋的全盛时期,等枣树叶落,枣子红完,西北风就要起来了,北方便是沙尘灰土的世界,只有这枣子、柿子、葡萄,成熟到八九分的七八月之交,是北国的清秋的佳日,是一年之中最好也没有的GoldenDays(11)。

有些批评家说,中国的文人学士,尤其是诗人,都带着很浓厚的颓废的色彩,所以中国的诗文里,赞颂秋的文字的特别的多。但外国的诗人,又何尝不然?我虽则外国诗文念的不多,也不想开出帐来,做一篇秋的诗歌散文钞(12),但你若去一翻英德法意等诗人的集子,或各国的诗文的Anthology来(13),总能够看到许多并于秋的歌颂和悲啼。各著名的大诗人的长篇田园诗或四季诗里,也总以关于秋的部分,写得最出色而最有味。足见有感觉的动物,有情趣的人类,对于秋,总是一样地特别能引起深沉,幽远、严厉、萧索的感触来的。不单是诗人,就是被关闭在牢狱里的囚犯,到了秋天,我想也一定能感到一种不能自已的深情,秋之于人,何尝有国别,更何尝有人种阶级的区别呢?不过在中国,文字里有一个“秋士”(14)的成语,读本里又有着很普遍的欧阳子的《秋声》(15)与苏东坡的《赤壁赋》等,就觉得中国的文人,与秋和关系特别深了,可是这秋的深味,尤其是中国的秋的深味,非要在北方,才感受得到底。

南国之秋,当然也是有它的特异的地方的,比如廿四桥的明月,钱塘江的秋潮,普陀山的凉雾,荔枝湾(16)的残荷等等,可是色彩不浓,回味不永。比起北国的秋来,正像是黄酒之与白干,稀饭之与馍馍,鲈鱼之与大蟹,黄犬之与骆驼。

秋天,这北国的秋天,若留得住的话,我愿把寿命的三分之二折去,换得一个三分之一的零头。

一九三四年八月在北平


\section{1.2   词语注释}
\label{\detokenize{p01_u6563_u6587/_u90c1_u8fbe_u592b-_u53e4_u90fd_u7684_u79cb:id4}}
⑴陶然亭:位于北京城南,亭名出自白居易诗句“更待菊黄家酿熟,共君一醉一陶然”。

⑵钓鱼台:在北京阜成门外三里河,玉渊潭公园北面。

⑶西山:北京西郊群山的总称,是京郊名胜。

⑷玉泉:指玉泉山,是西山东麓支脉。

⑸潭柘寺:在北京西山,相传“寺址本在青龙潭上,有古柘千章,寺以此得名。”

⑹一椽:一间屋。椽,放在房檩上架着木板或瓦的木条。

⑺落寞:冷落,寂寞。

⑻梧桐一叶而天下知秋:《淮南子,说山》:“以小明大,见叶落而知岁之将暮。”《太平御览》卷二十四引用“一叶落而知天下秋”。

⑼着:穿(衣)。

⑽平平仄仄起来:意即推敲起字的韵律来。

⑾GoldenDays:英语中指”黄金般的日子”。

⑿钞:同“抄”。

⒀Anthology:英语中指”选集”。

⒁秋士:古时指到了暮年仍不得志的知识分子。

⒂欧阳子的《秋声》:指欧阳修的《秋声赋》。

⒃荔枝湾:位于广州城西。


\chapter{1   韩愈-师说}
\label{\detokenize{p01_u6563_u6587/_u97e9_u6108-_u5e08_u8bf4:id1}}\label{\detokenize{p01_u6563_u6587/_u97e9_u6108-_u5e08_u8bf4::doc}}
\begin{sphinxShadowBox}
\sphinxstyletopictitle{目录}
\begin{itemize}
\item {} 
\phantomsection\label{\detokenize{p01_u6563_u6587/_u97e9_u6108-_u5e08_u8bf4:id14}}{\hyperref[\detokenize{p01_u6563_u6587/_u97e9_u6108-_u5e08_u8bf4:id1}]{\sphinxcrossref{1   韩愈-师说}}}
\begin{itemize}
\item {} 
\phantomsection\label{\detokenize{p01_u6563_u6587/_u97e9_u6108-_u5e08_u8bf4:id15}}{\hyperref[\detokenize{p01_u6563_u6587/_u97e9_u6108-_u5e08_u8bf4:id3}]{\sphinxcrossref{1.1   作品原文}}}

\item {} 
\phantomsection\label{\detokenize{p01_u6563_u6587/_u97e9_u6108-_u5e08_u8bf4:id16}}{\hyperref[\detokenize{p01_u6563_u6587/_u97e9_u6108-_u5e08_u8bf4:id4}]{\sphinxcrossref{1.2   词句注释}}}

\item {} 
\phantomsection\label{\detokenize{p01_u6563_u6587/_u97e9_u6108-_u5e08_u8bf4:id17}}{\hyperref[\detokenize{p01_u6563_u6587/_u97e9_u6108-_u5e08_u8bf4:id5}]{\sphinxcrossref{1.3   白话译文}}}

\item {} 
\phantomsection\label{\detokenize{p01_u6563_u6587/_u97e9_u6108-_u5e08_u8bf4:id18}}{\hyperref[\detokenize{p01_u6563_u6587/_u97e9_u6108-_u5e08_u8bf4:id6}]{\sphinxcrossref{1.4   创作背景}}}

\item {} 
\phantomsection\label{\detokenize{p01_u6563_u6587/_u97e9_u6108-_u5e08_u8bf4:id19}}{\hyperref[\detokenize{p01_u6563_u6587/_u97e9_u6108-_u5e08_u8bf4:id7}]{\sphinxcrossref{1.5   文学赏析}}}

\item {} 
\phantomsection\label{\detokenize{p01_u6563_u6587/_u97e9_u6108-_u5e08_u8bf4:id20}}{\hyperref[\detokenize{p01_u6563_u6587/_u97e9_u6108-_u5e08_u8bf4:id8}]{\sphinxcrossref{1.6   名家点评}}}
\begin{itemize}
\item {} 
\phantomsection\label{\detokenize{p01_u6563_u6587/_u97e9_u6108-_u5e08_u8bf4:id21}}{\hyperref[\detokenize{p01_u6563_u6587/_u97e9_u6108-_u5e08_u8bf4:id9}]{\sphinxcrossref{1.6.1   唐代}}}

\item {} 
\phantomsection\label{\detokenize{p01_u6563_u6587/_u97e9_u6108-_u5e08_u8bf4:id22}}{\hyperref[\detokenize{p01_u6563_u6587/_u97e9_u6108-_u5e08_u8bf4:id10}]{\sphinxcrossref{1.6.2   宋代}}}

\item {} 
\phantomsection\label{\detokenize{p01_u6563_u6587/_u97e9_u6108-_u5e08_u8bf4:id23}}{\hyperref[\detokenize{p01_u6563_u6587/_u97e9_u6108-_u5e08_u8bf4:id11}]{\sphinxcrossref{1.6.3   元代}}}

\item {} 
\phantomsection\label{\detokenize{p01_u6563_u6587/_u97e9_u6108-_u5e08_u8bf4:id24}}{\hyperref[\detokenize{p01_u6563_u6587/_u97e9_u6108-_u5e08_u8bf4:id12}]{\sphinxcrossref{1.6.4   明代}}}

\item {} 
\phantomsection\label{\detokenize{p01_u6563_u6587/_u97e9_u6108-_u5e08_u8bf4:id25}}{\hyperref[\detokenize{p01_u6563_u6587/_u97e9_u6108-_u5e08_u8bf4:id13}]{\sphinxcrossref{1.6.5   清代}}}

\end{itemize}

\end{itemize}

\end{itemize}
\end{sphinxShadowBox}

《师说》是唐代文学家韩愈创作的一篇议论文。文章阐说从师求学的道理,讽刺耻于相师的世态,教育了青年,起到转变风气的作用。文中列举正反面的事例层层对比,反复论证,论述了从师表学习的必要性和原则,批判了当时社会上“耻学于师”的陋习,表现出非凡的勇气和斗争精神,也表现出作者不顾世俗独抒己见的精神。全文幅虽不长,但涵义深广,论点鲜明,结构严谨,说理透彻,富有较强的说服力和感染力。


\section{1.1   作品原文}
\label{\detokenize{p01_u6563_u6587/_u97e9_u6108-_u5e08_u8bf4:id3}}
古之学者1必有师。师者,所以传道受业解惑也2。人非生而知之3者,孰能无惑?惑而不从师,其为惑也4,终不解矣。生乎吾前5,其闻6道也固先乎吾,吾从而师之7;生乎吾后,其闻道也亦先乎吾,吾从而师之。吾师道也8,夫庸知其年之先后生于吾乎9?是故10无11贵无贱,无长无少,道之所存,师之所存也。

嗟乎!师道之不传也久矣!欲人之无惑也难矣!古之圣人,其出人也远矣,犹且从师而问焉;今之众人,其下圣人也亦远矣,而耻学于师。是故圣益圣,愚益愚。圣人之所以为圣,愚人之所以为愚,其皆出于此乎?爱其子,择师而教之;于其身也,则耻师焉,惑矣。彼童子之师,授之书而习其句读者,非吾所谓传其道解其惑者也。句读之不知,惑之不解,或师焉,或不焉,小学而大遗,吾未见其明也。巫医乐师百工之人,不耻相师。士大夫之族,曰师曰弟子云者,则群聚而笑之。问之,则曰:“彼与彼年相若也,道相似也,位卑则足羞,官盛则近谀。”呜呼!师道之不复可知矣。巫医乐师百工之人,君子不齿,今其智乃反不能及,其可怪也欤!

圣人无常师。孔子师郯子、苌弘、师襄、老聃。郯子之徒,其贤不及孔子。孔子曰:“三人行,则必有我师”。是故弟子不必不如师,师不必贤于弟子。闻道有先后,术业有专攻,如是而已。

李氏子蟠,年十七,好古文,六艺经传皆通习之,不拘于时,学于余。余嘉其能行古道,作《师说》以贻之。{[}2{]}


\section{1.2   词句注释}
\label{\detokenize{p01_u6563_u6587/_u97e9_u6108-_u5e08_u8bf4:id4}}
1.学者:求学的人。

2.师者,所以传道受业解惑也:老师,是用来传授道理、交给学业、解释疑难问题的人。所以:用来……的。道:指儒家之道。受:通“授”,传授。业:泛指古代经、史、诸子之学及古文写作。惑:疑难问题。

3.人非生而知之者:人不是生下来就懂得道理。之:指知识和道理。《论语·季氏》:“生而知之者,上也;学而知之者,次也;困而学之,又其次之;困而不学,民斯为下矣。”知:懂得。

4.其为惑也:他所存在的疑惑。

5.生乎吾前:即生乎吾前者。乎:相当于“于”,与下文“先乎吾”的“乎”相同。

6.闻:听见,引申为知道,懂得。

7.从而师之:跟从(他),拜他为老师。从师:跟从老师学习。师:意动用法,以……为师。

8.吾师道也:我(是向他)学习道理。

9.夫庸知其年之先后生于吾乎:哪里去考虑他的年龄比我大还是小呢?庸:发语词,难道。知:了解、知道。

10.是故:因此,所以。

11.无:无论、不分。

12.道之所存,师之所存也:意思说哪里有道存在,哪里就有我的老师存在。

13.师道:从师的传统。即“古之学者必有师”。

14.出人:超出于众人之上。

15.犹且:尚且。

16.众人:普通人,一般人。

17..下:不如,名词作动词。

18.耻学于师:以向老师学习为耻。耻:以……为耻。

19.是故圣益圣,愚益愚:因此圣人更加圣明,愚人更加愚昧。益:更加、越发。

20.于其身:对于他自己。身:自身、自己。

21.惑矣:糊涂啊!

22.彼童子之师:那些教小孩子的启蒙老师。

23.授之书而习其句读(dòu):教给他书,帮助他学习其中的文句。之:指童子。习:使……学习。其:指书。句读:也叫句逗,古人指文辞休止和停顿处。文辞意尽处为句,语意未尽而须停顿处为读(逗)。古代书籍上没有标点,老师教学童读书时要进行句读(逗)的教学。

24.句读之不知:不知断句风逗。

25.或师焉,或不(fǒu)焉:有的从师,有的不从师。不:通“否”。

26.小学而大遗:学了小的(指“句读之不知”)却丢了大的(指“惑之不解”)。遗:丢弃,放弃。

27.巫医:古时巫、医不分,指以看病和降神祈祷为职业的人。

28.百工:各种手艺。

29.相师:拜别人为师。

30.族:类。

31.曰师曰弟子云者:说起老师、弟子的时候。

32.年相若:年岁相近。

33.位卑则足羞,官盛则近谀:以地位低的人为师就感到羞耻,以高官为师就近乎谄媚。足:可,够得上。盛:高大。谀:谄媚。

34.复:恢复。

35.君子:即上文的“士大夫之族”。

36.不齿:不屑与之同列,即看不起。或作“鄙之”。

37.乃:竟,竟然。

38.其可怪也欤:难道值得奇怪吗。其:难道,表反问。欤:语气词,表感叹。

39.圣人无常师:圣人没有固定的老师。常:固定的。

40.郯(tán)子:春秋时郯国(今山东省郯城县境)的国君,相传孔子曾向他请教官职。

41.苌(cháng)弘:东周敬王时候的大夫,相传孔子曾向他请教古乐。

42.师襄:春秋时鲁国的乐官,名襄,相传孔子曾向他学琴。

43.老聃(dān):即老子,姓李名耳,春秋时楚国人,思想家,道家学派创始人。相传孔子曾向他学习周礼。聃是老子的字。

44.之徒:这类。

45.三人行,则必有我师:三人同行,其中必定有我的老师。《论语·述而》原话:“子曰:‘三人行,必有我师焉。择其善者而从之,其不善者而改之。’”

46.不必:不一定。

47.术业有专攻:在业务上各有自己的专门研究。攻:学习、研究。

48.李氏子蟠(pán):李家的孩子名蟠。李蟠:韩愈的弟子,唐德宗贞元十九年(803)进士。

49.六艺经传(zhuàn)皆通习之:六艺的经文和传文都普遍的学习了。六艺:指六经,即《诗》《书》《礼》《乐》《易》《春秋》六部儒家经典。《乐》已失传,此为古说。经:两汉及其以前的散文。传,古称解释经文的著作为传。通:普遍。

50.不拘于时:指不受当时以求师为耻的不良风气的束缚。时:时俗,指当时士大夫中耻于从师的不良风气。于:被。

51.余嘉其能行古道:我赞许他能遵行古人从师学习的风尚。嘉:赞许,嘉奖。

52.贻(yí):赠送,赠予。{[}3-4{]}


\section{1.3   白话译文}
\label{\detokenize{p01_u6563_u6587/_u97e9_u6108-_u5e08_u8bf4:id5}}
古代求学的人一定有老师。老师,是可以依靠来传授道理、教授学业、解答疑难问题的。人不是生下来就懂得道理的,谁能没有疑惑?有了疑惑,如果不跟从老师学习,那些成为疑难问题的,就最终不能理解了。生在我前面,他懂得道理本来就早于我,我应该跟从他把他当作老师;生在我后面,如果他懂得的道理也早于我,我也应该跟从他把他当作老师。我是向他学习道理啊,哪管他的生年比我早还是比我晚呢?因此,无论地位高低贵贱,无论年纪大小,道理存在的地方,就是老师存在的地方。

唉,古代从师学习的风尚不流传已经很久了,想要人没有疑惑难啊!古代的圣人,他们超出一般人很远,尚且跟从老师而请教;现在的一般人,他们的才智低于圣人很远,却以向老师学习为耻。因此圣人就更加圣明,愚人就更加愚昧。圣人之所以能成为圣人,愚人之所以能成为愚人,大概都出于这吧?人们爱他们的孩子,就选择老师来教他,但是对于他自己呢,却以跟从老师学习为可耻,真是糊涂啊!那些孩子们的老师,是教他们读书,帮助他们学习断句的,不是我所说的能传授那些道理,解答那些疑难问题的。一方面不通晓句读,另一方面不能解决疑惑,有的句读向老师学习,有的疑惑却不向老师学习;小的方面倒要学习,大的方面反而放弃不学,我没看出那种人是明智的。巫医乐师和各种工匠这些人,不以互相学习为耻。士大夫这类人,听到称“老师”称“弟子”的,就成群聚在一起讥笑人家。问他们为什么讥笑,就说:“他和他年龄差不多,道德学问也差不多,以地位低的人为师,就觉得羞耻,以官职高的人为师,就近乎谄媚了。”唉!古代那种跟从老师学习的风尚不能恢复,从这些话里就可以明白了。巫医乐师和各种工匠这些人,君子们不屑一提,现在他们的见识竟反而赶不上这些人,真是令人奇怪啊!

圣人没有固定的老师。孔子曾以郯子、苌弘、师襄、老聃为师。郯子这些人,他们的贤能都比不上孔子。孔子说:“几个人一起走,其中一定有可以当我的老师的人。”因此学生不一定不如老师,老师不一定比学生贤能,听到的道理有早有晚,学问技艺各有专长,如此罢了。

李家的孩子蟠,年龄十七,喜欢古文,六经的经文和传文都普遍地学习了,不受时俗的拘束,向我学习。我赞许他能够遵行古人从师的途径,写这篇《师说》来赠送他。{[}5{]}


\section{1.4   创作背景}
\label{\detokenize{p01_u6563_u6587/_u97e9_u6108-_u5e08_u8bf4:id6}}
《师说》大约是作者于贞元十七年至十八年(801—802),在京任国子监四门博士时所作。贞元十七年(801),辞退徐州官职,闲居洛阳传道授徒的作者,经过两次赴京调选,方于当年十月授予国子监四门博士之职。此时的作者决心借助国子监这个平台来振兴儒教、改革文坛,以实现其报国之志。但来到国子监上任后,却发现科场黑暗,朝政腐败,吏制弊端重重,致使不少学子对科举入仕失去信心,因而放松学业;当时的上层社会,看不起教书之人。在士大夫阶层中存在着既不愿求师,又“羞于为师”的观念,直接影响到国子监的教学和管理。作者对此痛心疾首,借用回答李蟠的提问撰写这篇文章,以澄清人们在“求师”和“为师”上的模糊认识。{[}6{]}


\section{1.5   文学赏析}
\label{\detokenize{p01_u6563_u6587/_u97e9_u6108-_u5e08_u8bf4:id7}}
文中虽也正面论及师的作用、从师的重要性和以什么人为师等问题,但重点是批判当时流行于士大夫阶层中的耻于从师的不良风气。就文章的写作意图和主要精神看,这是一篇针对性很强的批驳性论文。

文章开头一段,先从正面论述师道:从师的必要性和从师的标准(以谁为师)。劈头提出“古之学者必有师”这个论断,紧接着概括指出师的作用:“传道受业解惑”,作为立论的出发点与依据。从“解惑”(道与业两方面的疑难)出发,推论人非生而知之者,不能无惑,惑则必从师的道理;从“传道”出发,推论从师即是学道,因此无论贵贱长幼都可为师,“道之所存,师之所存也”。这一段,层层顶接,逻辑严密,概括精练,一气呵成,在全文中是一个纲领。这一段的“立”,是为了下文的“破”。一开头郑重揭出“古之学者必有师”,就隐然含有对“今之学者”不从师的批判意味。势如风雨骤至,先声夺人。接着,就分三层从不同的侧面批判当时士大夫中流行的耻于从师的不良风气。先以“古之圣人”与“今之众人”作对比,指出圣与愚的分界就在于是否从师而学;再以士大夫对待自己的孩子跟对待自己在从师而学问题上的相反态度作对比,指出这是“小学而大遗”的糊涂作法;最后以巫医、乐师、百工不耻相师与士大夫耻于相师作对比,指出士大夫之智不及他们所不齿的巫医、乐师、百工。作者分别用“愚”、“惑”、“可怪”来揭示士大夫耻于从师的风气的不正常。由于对比的鲜明突出,作者的这种贬抑之辞便显得恰如其分,具有说服力。

在批判的基础上,文章又转而从正面论述“圣人无常师”,以孔子的言论和实践,说明师弟关系是相对的,凡是在道与业方面胜过自己或有一技之长的人都可以为师。这是对“道之所存,师之所存”这一观点的进一步论证,也是对士大夫之族耻于师事“位卑”者、“年近”者的现象进一步批判。

文章的最后一段,交待作这篇文章的缘由。李蟠“能行古道”,就是指他能继承久已不传的“师道”,乐于从师而学。因此这个结尾不妨说是借表彰“行古道”来进一步批判抛弃师道的今之众人。“古道”与首段“古之学者必有师”正遥相呼应。

在作者的论说文中,《师说》是属于文从字顺、平易畅达一类的,与《原道》一类豪放磅礴、雄奇桀傲的文章显然有别。但在平易畅达中仍贯注着一种气势。这种气势的形成,有多方面的因素。

首先是理论本身的说服力和严密的逻辑所形成的夺人气势。作者对自己的理论主张高度自信,对事理又有透彻的分析,因而在论述中不但步骤严密,一气旋折,而且常常在行文关键处用极概括而准确的语言将思想的精粹鲜明地表达出来,形成一段乃至一篇中的警策,给读者留下强烈深刻的印象。如首段在一路顶接,论述从师学道的基础上,结尾处就势作一总束:“是故无贵无贱,无长无少,道之所存,师之所存也。”大有如截奔马之势。“圣人无常师”一段,于举孔子言行为例之后,随即指出:“是故弟子不必不如师,师不必贤于弟子。闻道有先后,术业有专攻,如是而已。”从“无常师”的现象一下子引出这样透辟深刻的见解,有一种高瞻远瞩的气势。

其次是硬转直接,不作任何过渡,形成一种陡直峭绝的文势。开篇直书“古之学者必有师”,突兀而起,已见出奇;中间批判不良风气三小段,各以“嗟乎”、“爱其子”、“巫医、乐师、百工之人”发端,段与段问,没有任何承转过渡,兀然峭立,直起直落,了不相涉。这种转接发端,最为韩愈所长,读来自觉具有一种雄直峭兀之势。

此外,散体中参入对偶与排比句式,使奇偶骈散结合,也有助于加强文章的气势。{[}7{]}


\section{1.6   名家点评}
\label{\detokenize{p01_u6563_u6587/_u97e9_u6108-_u5e08_u8bf4:id8}}

\subsection{1.6.1   唐代}
\label{\detokenize{p01_u6563_u6587/_u97e9_u6108-_u5e08_u8bf4:id9}}
柳宗元《答韦中立论师道书》:孟子称人之患在好为人师。由魏晋氏以下,人益不事师。今之世不闻有师。有辄哗笑之以为狂人。独韩愈奋不顾流俗,犯笑侮,收召后学,作《师说》,抗颜而为师,世果群怪聚骂,指目牵引,而增与为言词,愈以是得狂名。又《答严厚舆论师道书》:言道讲古穷文辞以为师,则固吾属事。仆才能勇敢不如韩退之,故又不为人师。人之所见有异同,吾子无以韩责我。


\subsection{1.6.2   宋代}
\label{\detokenize{p01_u6563_u6587/_u97e9_u6108-_u5e08_u8bf4:id10}}
朱熹《朱子考异》:余观退之《师说》云:“弟子不必不如师,师不必贤于弟子。”其言非好为人师者也。学者不归子厚,归退之,故子厚有此说耳。

黄震《黄氏日抄》:前起后收,中排三节,皆以轻重相形。初以圣与愚相形,圣且从师,况愚乎?次以子与身相似,子且择师,况身乎?次以巫医、乐师、百工与士大夫相形,巫、乐、百工且从师,况士大夫乎?公之提诲后学,亦可谓深切著明矣。而文法则自然而成者也。


\subsection{1.6.3   元代}
\label{\detokenize{p01_u6563_u6587/_u97e9_u6108-_u5e08_u8bf4:id11}}
程端礼《昌黎文式》:此篇有诗人讽喻法,读之自知师道不可废。


\subsection{1.6.4   明代}
\label{\detokenize{p01_u6563_u6587/_u97e9_u6108-_u5e08_u8bf4:id12}}
茅坤《唐宋八大家文钞》:昌黎当时抗师道,以号召后辈,故为此倡赤帜云。


\subsection{1.6.5   清代}
\label{\detokenize{p01_u6563_u6587/_u97e9_u6108-_u5e08_u8bf4:id13}}
蔡世远《古文雅正》:师道立则善人多。汉世经学详明者,以师弟子相承故也。宋代理学昌明者,以师弟子相信故也。唐时知道者,独有一韩子,而当时又少肯师者,即如张文昌、李习之、皇甫持正,韩子得意弟子也,然诸人集中亦鲜推尊为师者,况其它乎?以此知唐时气习最重,故韩子痛切言之。唐学不及汉宋者,亦以此也。

储欣《唐宋十大家全集录·昌黎先生全集录》:题易迂,就浅处指点,乃无一点迂气。曾、王理学文,似未解此。又云:以眼前事指点化诲,使人易知,颇与《讳辩》一例。

孙琮《山晓阁选唐大家韩昌黎全集》:大意是欲李氏子能自得师,故一起提出师之为道,以下便说师无长幼贵贱,惟人自择。借写时人不肯从师,历引童子、巫医、孔子喻之,总是欲其能自得师。劝勉李氏子蟠,非是訾议世人。

爱新觉罗·玄烨《古文渊鉴》引洪迈:此文如常山蛇势,救首救尾,段段有力,学者宜熟读。

林云铭《韩文起》:其行文错综变化,反复引证,似无段落可寻。一气读之,只觉意味无穷。

吴楚材、吴调侯《古文观止》:通篇只是“吾师道也”一语,言触处皆师,无论长幼贵贱,惟人自择。因借时人不肯从师,历引童子、巫医、孔子喻之,总是欲李氏子能自得师,不必谓公慨然以师道自任,而作此以倡后学也。

张伯行《唐宋八大家文钞》:师者,师其道也,年之先后,位之尊卑,自不必论。彼不知求师者,曾百工之不若,乌有长进哉!《说命》篇曰:“德无常师。”朱子释之,以为天下之德,无一定之师,惟善是从。则凡有善者,皆可师,亦此意也。

方苞《方望溪先生全集·集外文·古文约选》:自“人非生而知之者”至“吾未见其明也”,言解惑。自“巫医乐师百工之人”至“如是而已”,言授业。而皆以传道贯之,盖舍授业无所谓传道也。

浦起龙《古文眉诠》:柳子谓韩子犯笑侮,收召后学,抗颜而为师,作《师说》,故知“师道不传”及“耻”“笑”等字,是著眼处。世不知古必有师,徒以为年不先我,以为不必贤于我,风俗人心,浇可知已。韩子见道于文,起衰八代,思得吾与,借李氏子发所欲言,不敢以告年长而自贤者,而私以告十七岁人,思深哉。

何焯《义门读书记》引李锺伦:“无贵无贱”,见不当挟贵;“无少无长”,见不当挟长;“圣人出人也远矣,犹且从师”,见不当挟贤。后即此三柱而申之。童子之师是年不相若者,引起世俗以年相若相师为耻;巫医、乐师、百工是无名位之人,引起世俗以官位不同相师为耻,而语势错综,不露痕也。


\chapter{1   Hi,p02读书}
\label{\detokenize{p02_u8bfb_u4e66/Hello_uff0cp02_u8bfb_u4e66:hi-p02}}\label{\detokenize{p02_u8bfb_u4e66/Hello_uff0cp02_u8bfb_u4e66::doc}}
\begin{sphinxShadowBox}
\sphinxstyletopictitle{目录}
\begin{itemize}
\item {} 
\phantomsection\label{\detokenize{p02_u8bfb_u4e66/Hello_uff0cp02_u8bfb_u4e66:id2}}{\hyperref[\detokenize{p02_u8bfb_u4e66/Hello_uff0cp02_u8bfb_u4e66:hi-p02}]{\sphinxcrossref{1   Hi,p02读书}}}
\begin{itemize}
\item {} 
\phantomsection\label{\detokenize{p02_u8bfb_u4e66/Hello_uff0cp02_u8bfb_u4e66:id3}}{\hyperref[\detokenize{p02_u8bfb_u4e66/Hello_uff0cp02_u8bfb_u4e66:post}]{\sphinxcrossref{1.1   post}}}

\end{itemize}

\end{itemize}
\end{sphinxShadowBox}


\section{1.1   post}
\label{\detokenize{p02_u8bfb_u4e66/Hello_uff0cp02_u8bfb_u4e66:post}}

\chapter{1   水浒-宋江之绰号}
\label{\detokenize{p02_u8bfb_u4e66/_u6c34_u6d52-_u5b8b_u6c5f_u4e4b_u7ef0_u53f7:id1}}\label{\detokenize{p02_u8bfb_u4e66/_u6c34_u6d52-_u5b8b_u6c5f_u4e4b_u7ef0_u53f7::doc}}
\begin{sphinxShadowBox}
\sphinxstyletopictitle{目录}
\begin{itemize}
\item {} 
\phantomsection\label{\detokenize{p02_u8bfb_u4e66/_u6c34_u6d52-_u5b8b_u6c5f_u4e4b_u7ef0_u53f7:id3}}{\hyperref[\detokenize{p02_u8bfb_u4e66/_u6c34_u6d52-_u5b8b_u6c5f_u4e4b_u7ef0_u53f7:id1}]{\sphinxcrossref{1   水浒-宋江之绰号}}}

\end{itemize}
\end{sphinxShadowBox}

宋江是《水浒传》里边名号最多的一个,共有四个。

第一个是黑宋江。
因为他长得面黑,身材比较矮,这是就他的形体来讲的,其貌不扬。

第二个是孝义黑三郎。
讲的是他对待父母,讲究孝道,他的孝道贯穿到了他的思想当中,成为他思想的一个部分,并且是他的思想的一个很重要的支撑点。

第三个是及时雨。
讲的是他仗义疏财,扶危济困,这在后面他陆续和弟兄们交往中能够看得出来。

第四个是呼保义。
这个词,一直到现在,大家都无法把它解释清楚。有一种解释说,保义是南宋时候武官的一个称呼,叫保义郎。“保义”本是宋代最低的武官名,逐渐成了人人可用的自谦之词。“呼保义”这个词是动宾结构,宋江以“自呼保义”来表示谦虚,意思是说,自己是最低等的人。另外一种解释,说“保”,就是保持的保;“义”就是忠义的义,“保义”即保持忠义,呼的意思,就是大家都那样叫他。大体上说,呼保义这个词实际上讲的是宋江对待国家的态度,对待朝廷的态度,对待皇帝的态度。水浒传里有云:“呼群保义”。


\chapter{1   Hi,p03旅游}
\label{\detokenize{p03_u65c5_u6e38/Hello_uff0cp03_u65c5_u6e38:hi-p03}}\label{\detokenize{p03_u65c5_u6e38/Hello_uff0cp03_u65c5_u6e38::doc}}
\begin{sphinxShadowBox}
\sphinxstyletopictitle{目录}
\begin{itemize}
\item {} 
\phantomsection\label{\detokenize{p03_u65c5_u6e38/Hello_uff0cp03_u65c5_u6e38:id2}}{\hyperref[\detokenize{p03_u65c5_u6e38/Hello_uff0cp03_u65c5_u6e38:hi-p03}]{\sphinxcrossref{1   Hi,p03旅游}}}
\begin{itemize}
\item {} 
\phantomsection\label{\detokenize{p03_u65c5_u6e38/Hello_uff0cp03_u65c5_u6e38:id3}}{\hyperref[\detokenize{p03_u65c5_u6e38/Hello_uff0cp03_u65c5_u6e38:post}]{\sphinxcrossref{1.1   post}}}

\end{itemize}

\end{itemize}
\end{sphinxShadowBox}


\section{1.1   post}
\label{\detokenize{p03_u65c5_u6e38/Hello_uff0cp03_u65c5_u6e38:post}}

\chapter{1   五泄瀑布}
\label{\detokenize{p03_u65c5_u6e38/_u4e94_u6cc4_u7011_u5e03:id1}}\label{\detokenize{p03_u65c5_u6e38/_u4e94_u6cc4_u7011_u5e03::doc}}
\begin{sphinxShadowBox}
\sphinxstyletopictitle{目录}
\begin{itemize}
\item {} 
\phantomsection\label{\detokenize{p03_u65c5_u6e38/_u4e94_u6cc4_u7011_u5e03:id3}}{\hyperref[\detokenize{p03_u65c5_u6e38/_u4e94_u6cc4_u7011_u5e03:id1}]{\sphinxcrossref{1   五泄瀑布}}}

\end{itemize}
\end{sphinxShadowBox}

五泄构成了天然的山水画卷,素有“小雁荡”之称。

当地人称瀑布为洩,一水折为五级,叫“五洩”,正称“五泄”。

月笼轻纱第一泄,
双龙争壑第二泄,
珠帘风动第三泄,
烈马奔腾第四泄,
蛟龙出海第五泄。

五泄从青口进入,古人云:“五泄名山青口锁,到此看山山便可”。沿公路前行,路旁曲溪青流,远处便是叠石岩。壁立数十丈,层层叠叠如彩屏。

再前便是“五泄湖”的水库,弯弯曲曲,长2公里许,犹似一条绿色的绸带飘浮在群山之中,颇有富春山水的风采。

在游船中还可以观赏许多奇特的山石景观,夹岩洞为其中一景。当年夹岩洞下有夹岩寺,香火较旺,水库建成后,寺庙成为水底龙宫。夹岩洞恰好位于湖面之上,洞高16米,深20米,内曾供奉千手观音,外观幽暗莫测,颇具神秘色彩。

沿湖还可以观赏元宝峰、鹫鹰峰、仙桃峰、老僧峰等。

在天一碧码头登岩后,沿五泄溪北上,过遇龙桥,就是五泄禅寺。

继续沿溪往前,不多远,过竹林,便是奔腾而下的第五泄。沿山势而上,依次四泄,三泄,二泄,一泄,逐次趋缓。风景各具。

脚劲不错,还可以继续向上,登上山顶观峡谷。

沿着深谷清溪,可以转回五泄禅寺。


\chapter{1   Hi,p04财经}
\label{\detokenize{p04_u8d22_u7ecf/Hello_uff0cp04_u8d22_u7ecf:hi-p04}}\label{\detokenize{p04_u8d22_u7ecf/Hello_uff0cp04_u8d22_u7ecf::doc}}
\begin{sphinxShadowBox}
\sphinxstyletopictitle{目录}
\begin{itemize}
\item {} 
\phantomsection\label{\detokenize{p04_u8d22_u7ecf/Hello_uff0cp04_u8d22_u7ecf:id2}}{\hyperref[\detokenize{p04_u8d22_u7ecf/Hello_uff0cp04_u8d22_u7ecf:hi-p04}]{\sphinxcrossref{1   Hi,p04财经}}}
\begin{itemize}
\item {} 
\phantomsection\label{\detokenize{p04_u8d22_u7ecf/Hello_uff0cp04_u8d22_u7ecf:id3}}{\hyperref[\detokenize{p04_u8d22_u7ecf/Hello_uff0cp04_u8d22_u7ecf:post}]{\sphinxcrossref{1.1   post}}}

\end{itemize}

\end{itemize}
\end{sphinxShadowBox}


\section{1.1   post}
\label{\detokenize{p04_u8d22_u7ecf/Hello_uff0cp04_u8d22_u7ecf:post}}

\chapter{1   Hi,p05技术}
\label{\detokenize{p05_u6280_u672f/Hello_uff0cp05_u6280_u672f:hi-p05}}\label{\detokenize{p05_u6280_u672f/Hello_uff0cp05_u6280_u672f::doc}}
\begin{sphinxShadowBox}
\sphinxstyletopictitle{目录}
\begin{itemize}
\item {} 
\phantomsection\label{\detokenize{p05_u6280_u672f/Hello_uff0cp05_u6280_u672f:id2}}{\hyperref[\detokenize{p05_u6280_u672f/Hello_uff0cp05_u6280_u672f:hi-p05}]{\sphinxcrossref{1   Hi,p05技术}}}
\begin{itemize}
\item {} 
\phantomsection\label{\detokenize{p05_u6280_u672f/Hello_uff0cp05_u6280_u672f:id3}}{\hyperref[\detokenize{p05_u6280_u672f/Hello_uff0cp05_u6280_u672f:post}]{\sphinxcrossref{1.1   post}}}

\end{itemize}

\end{itemize}
\end{sphinxShadowBox}


\section{1.1   post}
\label{\detokenize{p05_u6280_u672f/Hello_uff0cp05_u6280_u672f:post}}

\chapter{1   Hi,p06历史}
\label{\detokenize{p06_u5386_u53f2/Hello_uff0cp06_u5386_u53f2:hi-p06}}\label{\detokenize{p06_u5386_u53f2/Hello_uff0cp06_u5386_u53f2::doc}}
\begin{sphinxShadowBox}
\sphinxstyletopictitle{目录}
\begin{itemize}
\item {} 
\phantomsection\label{\detokenize{p06_u5386_u53f2/Hello_uff0cp06_u5386_u53f2:id2}}{\hyperref[\detokenize{p06_u5386_u53f2/Hello_uff0cp06_u5386_u53f2:hi-p06}]{\sphinxcrossref{1   Hi,p06历史}}}
\begin{itemize}
\item {} 
\phantomsection\label{\detokenize{p06_u5386_u53f2/Hello_uff0cp06_u5386_u53f2:id3}}{\hyperref[\detokenize{p06_u5386_u53f2/Hello_uff0cp06_u5386_u53f2:post}]{\sphinxcrossref{1.1   post}}}

\end{itemize}

\end{itemize}
\end{sphinxShadowBox}


\section{1.1   post}
\label{\detokenize{p06_u5386_u53f2/Hello_uff0cp06_u5386_u53f2:post}}

\chapter{1   瓦岗寨之李密}
\label{\detokenize{p06_u5386_u53f2/_u74e6_u5c97_u5be8_u4e4b_u674e_u5bc6:id1}}\label{\detokenize{p06_u5386_u53f2/_u74e6_u5c97_u5be8_u4e4b_u674e_u5bc6::doc}}
\begin{sphinxShadowBox}
\sphinxstyletopictitle{目录}
\begin{itemize}
\item {} 
\phantomsection\label{\detokenize{p06_u5386_u53f2/_u74e6_u5c97_u5be8_u4e4b_u674e_u5bc6:id3}}{\hyperref[\detokenize{p06_u5386_u53f2/_u74e6_u5c97_u5be8_u4e4b_u674e_u5bc6:id1}]{\sphinxcrossref{1   瓦岗寨之李密}}}

\end{itemize}
\end{sphinxShadowBox}


\chapter{1   Hi,p07创投}
\label{\detokenize{p07_u521b_u6295/Hello_uff0cp07_u521b_u6295:hi-p07}}\label{\detokenize{p07_u521b_u6295/Hello_uff0cp07_u521b_u6295::doc}}
\begin{sphinxShadowBox}
\sphinxstyletopictitle{目录}
\begin{itemize}
\item {} 
\phantomsection\label{\detokenize{p07_u521b_u6295/Hello_uff0cp07_u521b_u6295:id2}}{\hyperref[\detokenize{p07_u521b_u6295/Hello_uff0cp07_u521b_u6295:hi-p07}]{\sphinxcrossref{1   Hi,p07创投}}}
\begin{itemize}
\item {} 
\phantomsection\label{\detokenize{p07_u521b_u6295/Hello_uff0cp07_u521b_u6295:id3}}{\hyperref[\detokenize{p07_u521b_u6295/Hello_uff0cp07_u521b_u6295:post}]{\sphinxcrossref{1.1   post}}}

\end{itemize}

\end{itemize}
\end{sphinxShadowBox}


\section{1.1   post}
\label{\detokenize{p07_u521b_u6295/Hello_uff0cp07_u521b_u6295:post}}

\chapter{1   风投的前生}
\label{\detokenize{p07_u521b_u6295/_u98ce_u6295_u7684_u524d_u751f:id1}}\label{\detokenize{p07_u521b_u6295/_u98ce_u6295_u7684_u524d_u751f::doc}}
\begin{sphinxShadowBox}
\sphinxstyletopictitle{目录}
\begin{itemize}
\item {} 
\phantomsection\label{\detokenize{p07_u521b_u6295/_u98ce_u6295_u7684_u524d_u751f:id3}}{\hyperref[\detokenize{p07_u521b_u6295/_u98ce_u6295_u7684_u524d_u751f:id1}]{\sphinxcrossref{1   风投的前生}}}

\end{itemize}
\end{sphinxShadowBox}


\chapter{1   Hi,p08写作}
\label{\detokenize{p08_u5199_u4f5c/Hello_uff0cp08_u5199_u4f5c:hi-p08}}\label{\detokenize{p08_u5199_u4f5c/Hello_uff0cp08_u5199_u4f5c::doc}}
\begin{sphinxShadowBox}
\sphinxstyletopictitle{目录}
\begin{itemize}
\item {} 
\phantomsection\label{\detokenize{p08_u5199_u4f5c/Hello_uff0cp08_u5199_u4f5c:id2}}{\hyperref[\detokenize{p08_u5199_u4f5c/Hello_uff0cp08_u5199_u4f5c:hi-p08}]{\sphinxcrossref{1   Hi,p08写作}}}
\begin{itemize}
\item {} 
\phantomsection\label{\detokenize{p08_u5199_u4f5c/Hello_uff0cp08_u5199_u4f5c:id3}}{\hyperref[\detokenize{p08_u5199_u4f5c/Hello_uff0cp08_u5199_u4f5c:post}]{\sphinxcrossref{1.1   post}}}

\end{itemize}

\end{itemize}
\end{sphinxShadowBox}


\section{1.1   post}
\label{\detokenize{p08_u5199_u4f5c/Hello_uff0cp08_u5199_u4f5c:post}}

\chapter{1   Hi,p09work}
\label{\detokenize{p09work/Hello_uff0cp09work:hi-p09work}}\label{\detokenize{p09work/Hello_uff0cp09work::doc}}
\begin{sphinxShadowBox}
\sphinxstyletopictitle{目录}
\begin{itemize}
\item {} 
\phantomsection\label{\detokenize{p09work/Hello_uff0cp09work:id2}}{\hyperref[\detokenize{p09work/Hello_uff0cp09work:hi-p09work}]{\sphinxcrossref{1   Hi,p09work}}}
\begin{itemize}
\item {} 
\phantomsection\label{\detokenize{p09work/Hello_uff0cp09work:id3}}{\hyperref[\detokenize{p09work/Hello_uff0cp09work:post}]{\sphinxcrossref{1.1   post}}}

\end{itemize}

\end{itemize}
\end{sphinxShadowBox}


\section{1.1   post}
\label{\detokenize{p09work/Hello_uff0cp09work:post}}

\chapter{Indices and tables}
\label{\detokenize{index:indices-and-tables}}\begin{itemize}
\item {} 
\DUrole{xref,std,std-ref}{search}

\end{itemize}



\renewcommand{\indexname}{索引}
\printindex
\end{document}